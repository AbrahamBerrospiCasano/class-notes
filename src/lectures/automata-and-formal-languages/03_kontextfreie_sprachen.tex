\section{%
    Kontextfreie Sprachen%
}

\begin{Bem}
    Reguläre Sprachen sind nützlich, aber doch sehr begrenzt
    (sie haben sozusagen ein "`endliches"' Gedächtnis).
    Geht man zur größeren Klasse der Typ-2- oder kontextfreien Sprachen über,
    so stehen mehr Möglichkeiten of"|fen.
    Allein schon die Sprache $L = \{a^n b^n \;|\; n \ge 1\}$ ist eine
    Typ-2-Sprache, die nicht regulär ist (siehe oben).
    Weitere Beispiele umfassen Grammatiken für arithmetische Ausdrücke
    oder Programmiersprachen (die Syntax von praktisch jeder Programmiersprache
    lässt sich mit einer kontextfreien Grammatik beschreiben).\\
    Eine typische Regel wäre $\langle\text{Anweisung}\rangle ::=
    \text{Zuweisung} \;|\;
    \text{Anweisung}\mathbf{;}\; \text{Anweisung} \;|\;$\\
    $\mathbf{if}\; \text{bedingung}\; \mathbf{then}\; \text{Anweisung}\;
    \mathbf{fi} \;|\;
    \mathbf{while}\; \text{bedingung}\; \mathbf{do}\; \text{Anweisung}\;
    \mathbf{od}$
    (dabei sind Terminale fett, inklusive dem Semikolon).
\end{Bem}

\subsection{%
    Normalformen%
}

\begin{Bem}
    Man will jeder kontextfreien Grammatik eine andere
    kontextfreie Grammatik zuweisen, die möglichst
    "`einfach"' aufgebaut ist und die gleiche Sprache erzeugt.\\
    Im Folgenden wird dafür angenommen, dass es keine $\varepsilon$-Regeln
    gibt (sonst ersetzt man $L$ durch $L \setminus \{\varepsilon\}$).
\end{Bem}

\begin{Lemma}{Beseitigung von Ringableitungen}
    Sei $G$ eine Typ-2-Grammatik.\\
    Dann gibt es eine Typ-2-Grammatik $G'$ mit $L(G) = L(G')$, die keine
    \begriff{Ringableitungen} enthält
    (d.\,h. es gibt keine Variablen $B_1, \dotsc, B_k$ mit
    $B_1 \rightarrow \dotsb \rightarrow B_k$ und $B_k = B_1$).
\end{Lemma}

\begin{Beweis}
    Ist $B_1 \rightarrow \dotsb \rightarrow B_k$ mit $B_k = B_1$ eine
    Ringableitung, so ersetzt man in den Regeln
    alle $B_i$, $i = 2, \dotsc, k$ durch $B_1$.
    Dabei können natürlich Duplikate auftreten
    (z.\,B. bei $A \rightarrow AB_1$ und $A \rightarrow AB_2$),
    diese lässt man weg (Produktionsmenge ist ja eine Menge).
    Die Regel $B_1 \rightarrow B_1$ kann man ebenfalls weglassen.
    Die so entstandene Grammatik $G'$ erzeugt dieselbe Sprache wie die
    ursprüngliche Grammatik $G$.
\end{Beweis}

\begin{Lemma}{Beseitigung von Regeln der Form $A \rightarrow B$}
    Sei $G$ eine Typ-2-Grammatik.\\
    Dann gibt es eine Typ-2-Grammatik $G'$ mit $L(G) = L(G')$, die keine
    Regeln der Form $A \rightarrow B$ mit $A$ und $B$ Variablen enthält.
\end{Lemma}

\begin{Beweis}
    Zunächst lässt sich die Variablenmenge $V$ ordnen, sodass
    $V = \{A_1, \dotsc, A_n\}$ mit $A_i \not= A_j$ für $i \not= j$ und
    $A_i \rightarrow A_j$ tritt nur für $i < j$ auf
    (d.\,h. $\forall_{i=1,\dots,n} \forall_{j=1,\dotsc,i}\;
    A_i \rightarrow A_j \notin P$).
    
    Wieso geht das?
    Betrachte den gerichteten Graphen mit Knotenmenge
    $V = \{C_1, \dotsc, C_n\}$ und Regeln
    als Kanten (d.\,h. es gibt eine Kante von
    $C_i$ nach $C_j$ genau dann, falls $C_i \rightarrow C_j \in P$).
    Dieser Graph ist kreisfrei, da oBdA keine Ringableitungen vorhanden sind
    (siehe Lemma von eben).
    Nun gibt es ein $i$, sodass für alle $j$ gilt, dass
    $C_i \rightarrow C_j \notin P$
    (andernfalls gäbe es einen Kreis).
    Man setzt nun $A_n := C_i$, entfernt $A_n$ inkl. den eingehenden Kanten
    aus dem Graphen und wiederholt diese Prozedur.
    Es kommen also keine Regeln $A_i \rightarrow A_j \in P$ mit
    $i \ge j$ vor
    (die z.\,B. im ersten Schritt entfernten Kanten
    $A_i \rightarrow A_n$, $i = 1, \dotsc, n - 1$ sind ja erlaubt).
    
    Nun kann man alle Regeln der Form $A_i \rightarrow A_j$ eliminieren:
    Jede Regel $A_n \rightarrow w$ mit $w \in (V \cup \Sigma)^+$ hat als
    rechte Seite $w \notin V$ keine Variable.
    Falls es Regeln $A_i \rightarrow A_n$ mit $i < n$ gibt, ersetzt man diese
    Regeln durch $A_i \rightarrow w$ für jede Regel der Form
    $A_n \rightarrow w$.
    Anschließend gibt es keine Regeln $A_i \rightarrow A_n$ mit $i < n$ mehr.
    Man wiederholt dies mit $A_{n-1}$ usw. bis $A_2$.
    
    Die Sprache wird dabei nicht verändert und die so entstandene
    Grammatik $G'$ enthält keine Regeln der Form $A \rightarrow B$ mit
    $A$ und $B$ Variablen.
\end{Beweis}

\linie
\pagebreak

\begin{Def}{\name{Chomsky}-Normalform}
    Eine Typ-2-Grammatik heißt in \begriff{\name{Chomsky}-Normalform}, falls
    alle Regeln von der Form $A \rightarrow BC$ oder $A \rightarrow a$ sind
    ($A, B, C$ Variablen und $a$ Terminal).
\end{Def}

\begin{Bem}
    Die Ableitungsbäume bekommen damit eine sehr regelmäßige Form, denn sie
    sind alle binär.
    Jede Ableitung eines Worts der Länge $n$ hat in einer CNF-Grammatik
    die Länge von $2n - 1$ Ableitungsschritten.
\end{Bem}

\begin{Satz}{\name{Chomsky}-Normalform}
    Zu jeder kontextfreien Grammatik $G$ mit $\varepsilon \notin L(G)$ gibt
    es eine kontextfreie Grammatik $G'$ in Chomsky-Normalform mit
    $L(G) = L(G')$.
\end{Satz}

\begin{Beweis}
    Zunächst führt man für jedes Terminalzeichen eine sog.
    \begriff{Pseudovariable} ein.
    Sei also $\Sigma = \{\sigma_1, \dotsc, \sigma_k\}$.
    Führe für alle $i = 1, \dotsc, k$ eine Variable $V_{\sigma_i}$ ein.
    Ersetze in allen Regeln Terminale grundsätzlich durch die entsprechenden
    Pseudoterminale und füge die Regel $V_{\sigma_i} \rightarrow \sigma_i$
    hinzu.
    Dies verändert die Sprache nicht und alle Regeln haben jetzt die Form
    von entweder $A \rightarrow a$ oder $A \rightarrow w$ mit
    $w \in V^+$ und $|w| \ge 2$
    (oBdA nach obigem Lemma).
    
    Die Regeln der Form $A \rightarrow a$ sind okay (kompatibel zur CNF).
    Die Regeln $A \rightarrow B_1 \dotsb B_k$ mit $k \ge 2$ sind
    für $k = 2$ ebenfalls okay, für $k > 2$ müssen diese umgeformt werden:
    Diese Regel wird ersetzt durch die Regeln
    $A \rightarrow B_1 C_2$, $C_2 \rightarrow B_2 C_3$ usw. bis
    $C_{k-1} \rightarrow B_{k-1} B_k$.
    Dabei sind $C_2, \dotsc, C_{k-1}$ neue Variablen.
    Durch diese Anpassung wird die Sprache ebenfalls nicht geändert,
    die entstandene Grammatik $G'$ ist in Chomsky-Normalform.
\end{Beweis}

\begin{Bem}
    Die Vorgehensweise, um eine gegebene kontextfreie Grammatik in ein
    CNF umzuwandeln, wird aus den konstruktiven Beweisen ersichtlich:
    \begin{enumerate}
        \item
        Elimination von Ringableitungen durch Ersetzen aller in der
        Ringableitung vorkommenden Variablen durch eine einzige Variable
        
        \item
        Sortierung der Variablen zu einer Menge $\{A_1, \dotsc, A_n\}$,
        sodass keine Regeln $A_i \rightarrow A_j$ mit $j \le i$ auftreten
        
        \item
        Elimination der Regeln der Form $A \rightarrow B$
        durch Ersetzen von $A_i \rightarrow A_j$ für $i = n - 1, \dotsc, 1$
        
        \item
        Pseudoterminale einführen und Ersetzen aller Terminale in den Regeln
        
        \item
        Elimination der 
        Regeln mit mehr als zwei Variablen auf der rechten Seite
    \end{enumerate}
\end{Bem}

\linie

\begin{Def}{\name{Greibach}-Normalform}
    Eine Typ-2-Grammatik heißt in \begriff{\name{Greibach}-Normalform}, falls
    alle Regeln von der Form $A \rightarrow a B_1 \dotsc B_k$ für $k \ge 0$
    sind ($A, B_i$ Variablen und $a$ Terminal).
\end{Def}

\begin{Bem}
    Mit der Zusatzbedingung $k \le 1$ erhält man genau die regulären
    Grammatiken.\\
    (Man kann sogar jede kontextfreie Grammatik in Greibach-Normalform bringen,
    wobei $k \le 2$.)
\end{Bem}

\begin{Satz}{\name{Greibach}-Normalform}
    Zu jeder kontextfreien Grammatik $G$ mit $\varepsilon \notin L(G)$ gibt
    es eine kontextfreie Grammatik $G'$ in Greibach-Normalform mit
    $L(G) = L(G')$.
\end{Satz}

\begin{Beweis}
    Ist $A$ eine Variable, so werden im Folgenden alle Regeln der Form
    $A \rightarrow w$ mit einer beliebigen Satzform $w$ als
    \begriff{$A$-Regeln} bezeichnet.
    
    Mithilfe folgender Vorüberlegung kann man alle
    \begriff{linksrekursiven Regeln} entfernen:\\
    Für jede Variable $A$ teilt man die $A$-Regeln in die linksrekursiven
    Regeln (d.\,h. $A \rightarrow A\alpha$ mit beliebigen $\alpha$) und
    die übrigen Regeln (d.\,h. $A \rightarrow \beta$, wobei $\beta$ nicht mit
    $A$ beginnt) auf.\\
    Entsprechend dieser Aufteilung seien die $A$-Regeln
    $A \rightarrow A\alpha_1 \;|\; \dotsb \;|\; A\alpha_k \;|\;
    \beta_1 \;|\; \dotsb \;|\; \beta_\ell$\\
    ($k \ge 0$, $l \ge 1$).
    Diese Regeln kann man ersetzen durch die Regeln\\
    $A \rightarrow \beta_1 \;|\; \dotsb \;|\; \beta_\ell$,\qquad
    $A \rightarrow \beta_1 B \;|\; \dotsb \;|\; \beta_\ell B$,\qquad
    $B \rightarrow \alpha_1 \;|\; \dotsb \;|\; \alpha_k$,\qquad
    $B \rightarrow \alpha_1 B \;|\; \dotsb \;|\; \alpha_k B$.\\
    Dabei ist $B$ eine neue Variable.
    Die erzeugte Sprache wird dadurch nicht verändert.\\
    Dann sind also keine linksrekursiven $A$-Regeln mehr vorhanden\\
    ($\beta$ starten nicht mit $A$ und $\alpha$ starten nicht mit $B$, da $B$
    eine neue Variable ist).
    
    \pagebreak
    
    Ohne Einschränkung kann man nach obigem Satz von einer kontextfreien
    Grammatik in\\
    Chomsky-Normalform ausgehen, wobei die Variablen durchnummeriert sind\\
    ($V = \{A_1, \dotsc, A_m\}$).
    
    Der erste Algorithmus formt die Grammatik so um, dass Regeln der Form
    $A_i \rightarrow A_j \beta$ nur mit $i < j$ vorkommen.
    Dabei müssen ggf. entsprechend der Überlegung wie eben neue Variablen
    eingeführt werden, um Linksrekursion zu vermeiden.
    \begin{align*}&
        \FOR i := 1 \;\TO m \;\DO\\&
        \qquad\FOR j := 1 \;\TO i - 1 \;\DO\\&
        \qquad\qquad\FORALL A_i \rightarrow A_j \alpha \in P \;\DO\\&
        \qquad\qquad\qquad\IF A_j \rightarrow
        \beta_1 \;|\; \dotsb \;|\; \beta_n
        \;\THEN \text{streiche }A_i \rightarrow A_j \alpha,\;
        \text{neu in }P\text{: } A_i \rightarrow \beta_1 \alpha \;|\; \dotsb
        \;|\; \beta_n \alpha\\&
        \qquad\qquad\END\\&
        \qquad\END\\&
        \qquad\FORALL A_i \rightarrow A_i \alpha \in P:
        \text{beseitige wie in Vorüberlegung}\\&
        \END
    \end{align*}
    Es werden also die Ableitungsmöglichkeiten
    $A_i \Rightarrow A_j \alpha \Rightarrow \beta_k \alpha$ ersetzt durch
    $A_i \Rightarrow \beta_k \alpha$.\\
    Nun gibt es zwei Sorten von Regeln:
    $A_i \rightarrow A_j \beta \;|\; \alpha_k$ mit $i < j$ und
    $B \rightarrow \beta_k \;|\; \beta_k B$, wobei die
    $\alpha_k$ nicht mit einer Variable und die
    $\beta_k$ nicht mit $B$ anfangen.
    
    Der zweite Algorithmus erreicht, dass die rechten Seiten aller $A_i$-Regeln
    mit Terminalen beginnen.
    Bei den $A_m$-Regeln ist dies schon der Fall
    (Form $A_m \rightarrow A_j \beta \;|\; \alpha_k$ mit $j > m$,
    dies ist aber nicht möglich, daher
    $A_m \rightarrow \alpha_k$, wobei die $\alpha_k$ nicht mit Variablen
    beginnen).\\
    Die $A_{m-1}$-Regeln beginnen entweder mit $A_m$ oder mit Terminalzeichen.
    Durch Einsetzen der rechten Seiten aller $A_m$-Regeln erhält man auch in
    allen $A_{m-1}$-Regeln rechte Seiten, die mit Terminalen beginnen.
    So verfährt man induktiv mit $A_{m-2}$ usw. bis zu $A_1$.
    
    Nun haben sind alle $A_i$-Regeln GNF-konform
    (die Grammatik war zu Beginn in CNF, d.\,h. es kommen nur Variablen
    nach dem beginnenden Terminal).
    Es gibt nun noch die $B$-Regeln, die bei der Entfernung der linksrekursiven
    Regeln eingeführt wurden.
    Die rechten Seiten der $B$-Regeln beginnen entweder mit $A_i$ oder mit
    Terminalen.
    Da die $A_i$-Regeln alle schon mit Terminalen beginnen, kann man einfach
    einsetzen (analog zu eben).
    Die so entstandene Grammatik erzeugt die gleiche Sprache und ist in
    Greibach-Normalform.
\end{Beweis}

\linie

\begin{Bsp}
    Gegeben sei die Grammatik in CNF
    $A_1 \rightarrow A_1 A_2 \;|\; A_2 A_3 \;|\; a$,\quad
    $A_2 \rightarrow A_3 A_1$,\quad
    $A_3 \rightarrow b$.\\
    Beim ersten Algorithmus werden zunächst Regeln der Form
    $A_i \rightarrow A_j \beta$ mit $j \le i$ entfernt.
    Es ist die linksrekursive Regel $A_1 \rightarrow A_1 A_2$ vorhanden,
    die wie in der Vorüberlegung ersetzt wird, sodass die neuen Regeln
    $A_1 \rightarrow A_2 A_3 \;|\; a \;|\; A_2 A_3 B \;|\; a B$,\quad
    $B \rightarrow A_2 \;|\; A_2 B$,\\
    $A_2 \rightarrow A_3 A_1$,\quad
    $A_3 \rightarrow b$ sind.
    Für $A_2$ und $A_3$ ist nichts zu tun.\\
    Beim zweiten Algorithmus ersetzt man alle $A_i$-Regeln, sodass auf den
    rechten Seiten nur noch Terminale vorkommen, d.\,h.
    $A_1 \rightarrow b A_1 A_3 \;|\; a \;|\; b A_1 A_3 B \;|\; a B$,\quad
    $B \rightarrow A_2 \;|\; A_2 B$,\\
    $A_2 \rightarrow b A_1$,\quad
    $A_3 \rightarrow b$.
    Nun macht man dasselbe für die $B$-Regeln:\\
    $A_1 \rightarrow b A_1 A_3 \;|\; a \;|\; b A_1 A_3 B \;|\; a B$,\qquad
    $A_2 \rightarrow b A_1$,\qquad
    $A_3 \rightarrow b$,\qquad
    $B \rightarrow b A_1 \;|\; b A_1 B$.
\end{Bsp}

\pagebreak

\subsection{%
    Das Pumping-Lemma%
}

\begin{Bem}
    Man kann ein Analogon zum Pumping-Lemma für reguläre Sprachen auch für
    kontextfreie Sprachen aufstellen.
    Um die beiden Lemmata zu unterscheiden, nennt man das Pumping-Lemma für
    reguläre Sprachen auch \begriff{$uvw$-Theorem}, während man das
    Pumping-Lemma für kontextfreie Sprachen \begriff{$uvwxy$-Theorem} nennt.
\end{Bem}

\begin{Satz}{Pumping-Lemma}
    Sei $L \subset \Sigma^\ast$ eine kontextfreie Sprache.\\
    Dann gilt
    $\exists_{n \in \natural} \forall_{z \in L,\; |z| \ge n}
    \exists_{u, v, w, x, y \in \Sigma^\ast,\; uvwxy = z}\;
    (1. \land 2. \land 3.)$ mit
    \begin{enumerate}
        \item
        $|vx| \ge 1$
        
        \item
        $|vwx| \le n$
        
        \item
        $\forall_{i \in \natural_0}\; u v^i w x^i y \in L$
    \end{enumerate}
\end{Satz}

\begin{Beweis}
    Da die Sprache $L$ kontextfrei ist, kann man von einer Grammatik
    $G = (V, \Sigma, P, S)$ in Chomsky-Normalform ausgehen, wobei $L = L(G)$.
    
    Wähle $n = 2^{|V|}$ und sei $z \in L$ mit $|z| \ge n$ beliebig.
    Ein Ableitungsbaum von $z$ hat folgende Form:
    Unter der Wurzel $S$ befindet sich ein Binärbaum
    (Regeln vom Typ $A \rightarrow BC$).
    Für jedes Kind dieses Binärbaums befindet sich unterhalb vom Binärbaum
    noch ein weiterer Kindknoten des Ableitungsbaums
    (Regeln vom Typ $A \rightarrow a$).
    $z$ kann nun an diesen Kindern abgelesen werden, d.\,h.
    der Baum hat $|z| \ge 2^{|V|}$ viele Kinder.
    Nach dem Lemma (siehe unten) hat der Ableitungsbaum also mindestens einen
    Pfad der Länge $\ge |V|$.
    
    Nun fixiert man einen längsten Pfad im Ableitungsbaum
    (dieser hat eine Länge $\ge |V|$).
    In diesem Pfad kommen daher $> |V|$ Variablen vor,
    es muss sich also mindestens eine Variable wiederholen.
    Wähle von unten kommend die erste wiederholte Variable,
    diese sei nun $A$.\\
    Man teilt nun den Baum und $z$ folgendermaßen ein:
    Der Teil unter dem (von unten) ersten Vorkommen von $A$,
    der Teil unter dem (von unten) zweiten Vorkommen von $A$
    ohne den ersten Teil
    sowie der Teil unter $S$ (also der komplette Baum) ohne
    die ersten beiden Teile.
    
    $w$ sei das Wort gebildet aus den Kindern des ersten Teils,
    $v$ und $x$ die Wörter gebildet aus den Kindern des zweiten Teils
    (durch den ersten Teil zerfallen die Kinder in zwei Abschnitte) sowie
    $u$ und $y$ die Wörter gebildet aus den Kindern des dritten Teils
    (durch die ersten beiden Teile zerfallen die Kinder in zwei Abschnitte).\\
    Dabei gilt $|vx| \ge 1$ und $|vwx| \le n$.
    Das Erste gilt, da die beiden $A$'s nicht identisch sind und
    das zweite $A$ mindestens ein Kind hat, in dem sich das erste $A$ nicht
    befindet (Regel $A \rightarrow BC$ wird angewendet).
    Dieses Kind erzeugt ein nicht-leeres Wort, also ist $v$ oder $x$
    nicht-leer.
    Das Zweite gilt, weil der größte Abstand des ersten Vorkommens von $A$ zu
    den Blättern $< |V|$ ist.
    Es kann also nach dem Lemma (siehe unten) höchstens
    $n = 2^{|V|}$ Blätter unter diesem $A$ geben.
    
    Außerdem gilt anhand des Ableitungsbaums, dass
    $S \Rightarrow^\ast uAy$,
    $A \Rightarrow^\ast vAx$ und
    $A \Rightarrow^\ast w$.\\
    Daher kann man pumpen, d.\,h.
    $S \Rightarrow^\ast uAy \Rightarrow^\ast uvAxy \Rightarrow^\ast uv^2Ax^2y
    \Rightarrow^\ast \dotsb \Rightarrow^\ast uv^iAx^iy \Rightarrow^\ast
    uv^iwx^iy$ für $i \in \natural_0$.
    Anschaulich "`hängt"' man den Baum unter dem zweiten $A$
    (ohne den Baum unter dem ersten $A$) so oft wie gewünscht untereinander,
    bis man mit dem Baum unter dem ersten $A$ terminiert.
\end{Beweis}

\begin{Bem}
    Für $u = v = \varepsilon$ erhält man das Pumping-Lemma für
    reguläre Sprachen.
\end{Bem}

\linie
\pagebreak

\begin{Bem}
    Für den Beweis wird das folgende Lemma genutzt.
\end{Bem}

\begin{Def}{Binärbaum}
    Ein \begriff{Binärbaum} ist ein Baum, in dem jeder Knoten,
    der kein Blatt ist, genau zwei Nachfolger hat.
\end{Def}

\begin{Lemma}{Pfade im Binärbaum}
    Ein Binärbaum mit mindestens $2^k$ Blättern hat mindestens einen Pfad der
    Länge $\ge k$.
\end{Lemma}

\begin{Beweis}
    Der Beweis erfolgt mittels vollständiger Induktion
    über $k \in \natural_0$.
    
    Für $k = 0$ ist die Behauptung trivial, denn dann hat der Baum mindestens
    einen Knoten und somit auch einen Pfad der Länge $0$.
    
    Für $k \to k + 1$ betrachtet man den linken und den rechten Teilbaum der
    Wurzel des Baums mit mindestens $2^{k+1}$ Blättern.
    Dann hat einer der beiden Teilbäume mindestens $\frac{2^{k+1}}{2} = 2^k$
    Blätter.
    Nach Induktionsvoraussetzung gibt es in diesem Teilbaum einen Pfad der
    Länge $\ge k$.
    Dieser kann bis zur Wurzel vom "`großen"' Baum verlängert werden und
    liefert einen Pfad der Länge $\ge k + 1$.
\end{Beweis}

\linie

\begin{Bsp}
    Die Sprache $L = \{a^n b^n c^n \;|\; n \ge 1\}$ ist nicht kontextfrei.
    Andernfalls gäbe es nach dem Pumping-Lemma ein $n \in \natural$ mit
    obigen Eigenschaften.
    Wähle $z = a^n b^n c^n \in L$,
    es gilt $|z| = 3n \ge n$.
    Dann gibt es $u, v, w, x, y \in \{a, b, c\}^\ast$ mit $z = uvwxy$ und
    $|vx| \ge 1$, $|vwx| \le n$ und
    $\forall_{i \in \natural_0}\; uv^iwx^iy \in L$.
    Aufgrund $|vwx| \le n$ kann $vwx$ nicht $a$'s, $b$'s und $c$'s enthalten
    (also nicht alle drei Buchstaben auf einmal, sondern höchstens zwei).
    Daher ist $uwy \notin L$, denn
    $|uwy|_a \not= |uwy|_b$ oder
    $|uwy|_b \not= |uwy|_c$, ein Widerspruch zur dritten Eigenschaft.
\end{Bsp}

\begin{Bsp}
    Die Sprache $L = \{a^i b^j c^k \;|\; i > j > k,\; k < i - 7\}$ ist nicht
    kontextfrei.
    Andernfalls gäbe es nach dem Pumping-Lemma ein $n \in \natural$ mit
    obigen Eigenschaften.\\
    Wähle $z = a^{n+9} b^{n+8} c^{n+1} \in L$,
    es gilt $|z| = 3n + 18 \ge n$.
    Dann gibt es $u, v, w, x, y \in \{a, b, c\}^\ast$ mit $z = uvwxy$ und
    $|vx| \ge 1$, $|vwx| \le n$ und
    $\forall_{i \in \natural_0}\; uv^iwx^iy \in L$.
    Es gibt nun drei Fälle:
    \begin{enumerate}
        \item
        $vwx$ enthält $a$'s.
        Dann enthält $vwx$ keine $c$'s (wegen der zweiten Eigenschaft) und
        es gilt $uwy \notin L$, falls $v$ oder $x$ $a$'s enthalten
        (dann ist $|uwy|_c = |uvwxy|_c = n + 1$, aber
        $n + 1 \not< |uwy|_a - 7 < |uvwxy|_a - 7 = n + 2$).\\
        Falls $v$ und $x$ keine $a$'s enthalten, so ist
        $vwx = a^r b^s$ mit $v = \varepsilon$, $w = a^r b^{s_1}$ und
        $x = b^{s - s_1}$.
        Dann muss $u = a^t$ sowie $y = b^{s_2} c^{n+1}$ gelten, wobei
        $t + r = n + 9$ und $s + s_2 = n + 8$.\\
        In diesem Fall ist $uv^2 wx^2 y =
        a^{t+r} b^{s_1 + 2(s - s_1) + s_2} c^{n+1} \notin L$, da\\
        $|uv^2 wx^2 y|_b = s_1 + 2(s - s_1) + s_2 = 2s - s_1 + s_2 =
        n + 8 + s - s_1 \not< n + 9 = |uv^2 wx^2 y|_a$.
        
        \item
        $vwx$ enthält $c$'s.
        Dann enthält $vwx$ keine $a$'s und es gilt
        $uv^9 wx^9 y \notin L$, denn es $v$ oder $x$ enthält ein $b$ oder
        ein $c$.
        Weil aber keine $a$'s enthalten sind, gilt
        $|uv^9 wx^9 y|_a = |uvwxy|_a = n + 9$, aber
        $|uv^9 wx^9 y|_b \ge |uvwxy|_b + 8 = n + 16$ oder
        $|uv^9 wx^9 y|_c \ge |uvwxy|_c + 8 = n + 9$.
        
        \item
        $vwx$ enthält weder $a$'s noch $c$'s.
        Dann besteht $vwx$ nur aus $b$'s und es gilt
        $uv^2 wx^2 y \notin L$, denn $|uv^2 wx^2 y|_a = |uvwxy|_a = n + 9$,
        aber $|uv^2 wx^2 y|_b > |uvwxy|_b = n + 8$.
    \end{enumerate}
\end{Bsp}

\linie
\pagebreak

\begin{Bsp}
    Dass $L = \{a^p \;|\; p \text{ prim}\}$ nicht kontextfrei ist, kann man
    wie beim Beweis für die Nicht-Regularität zeigen (siehe oben).
    Das ist kein Zufall, sondern bei allen Sprachen über einelementigen
    Alphabeten der Fall:
    Sei $L$ kontextfrei und $z \in L$ mit $|z| \ge n$.
    Dann gibt eine Zerlegung $z = uvwxy$.
    Aufgrund des einelementigen Alphabets ist $z = wvxyu = u'v'w'$ mit
    $u' = w$, $v' = vx$ und $w' = yu$.
    Es gilt $|vx| = |v'| \ge 1$,
    $|vwx| = |u'v'| \le n$ sowie\\
    $uv^i wx^i y = u' (v')^i w' \in L$, d.\,h. man erhält die gleiche Aussage
    wie beim $uvw$-Theorem.
    Der folgende Satz führt das genauer aus.
\end{Bsp}

\begin{Satz}{kontextfreie Sprachen über einelementige Alphabete sind regulär}\\
    Sei $L \subset \Sigma^\ast$ eine kontextfreie Sprache mit $|\Sigma| = 1$.
    Dann ist $L$ regulär.
\end{Satz}

\begin{Beweis}
    Sei $L \subset a^\ast$ eine kontextfreie Sprache.
    Nach dem Pumping-Lemma für kontextfreie Sprachen gibt es für alle
    $z \in L$ mit $|z| \ge n$ eine Zerlegung $z = uvwxy =
    a^{k_1} a^{\ell_1} a^{k_2} a^{\ell_2} a^{k_3} = a^k a^\ell$ mit
    $k = k_1 + k_2 + k_3$ und $\ell = \ell_1 + \ell_2$.
    Dabei gilt $\ell > 0$ und $a^{k + i\ell} \in L$ für alle
    $i \in \natural_0$.
    
    Man erhält also für jedes $z \in L$ mit $|z| \ge n$ eine Zahl
    $\ell \in \{1, \dotsc, n\}$, die \begriff{Periode} von $z$ genannt wird
    (die Periode von $z$ ist evtl. nicht eindeutig).
    
    Sei $q := n!$, dann gilt $\ell \;|\; q$ für alle Perioden $\ell$.
    Für eine beliebige Zahl $q' > q$ wird die Sprache
    $L' := \{x \in L \;|\; |x| < q\} \cup \{a^{r + iq} \;|\;
    q \le r \le q',\; a^r \in L,\; i \in \natural_0\}$ definiert.\\
    $L'$ ist für alle $q' > q$ regulär, denn
    die erste Menge ist endlich und somit regulär.
    Die zweite Menge ist regulär, da $\{a^{r + iq} \;|\; i \in \natural_0\}$
    für festes $r, q$ regulär ist (z.\,B. mit einem entsprechenden DEA)
    und die zweite Menge eine endliche Vereinigung solcher Mengen ist.\\
    Außerdem gilt $L' \subset L$, denn die erste Menge ist eine Teilmenge von
    $L$ und bei der zweiten Menge ist $a^{r + iq} \in L$ für $a^r \in L$,
    $r \ge q$ und $i \ge 0$, da $a^r$ die Periode $\ell$ hat und diese $q$
    teilt.\\
    Falls $q'$ so gefunden kann, dass $L \subset L'$ gilt, so gilt
    $L = L'$ und der Beweis ist abgeschlossen, da $L'$ regulär ist.
    
    Wähle zunächst $q' = q + 1$.
    Falls $L' = L$, so ist man fertig.\\
    Falls $L' \subsetneqq L$, so wähle ein kürzestes Wort
    $a^s \in L \setminus L'$.
    Wird $q' > s$ gewählt, so ist $a^{s + iq} \in L'$ für $i \in \natural_0$
    (ist in der zweiten Menge).
    Somit sind alle $a^m$ mit $m \ge s$ und $m \equiv s \mod q$ in $L'$.\\
    Dieser Vorgang wird nun iteriert.
    Es gibt allerdings höchstens $q - 1$ Restklassen, sodass die Iteration in
    endlich vielen Schritten endet.
\end{Beweis}

\pagebreak

\subsection{%
    Abschlusseigenschaften%
}

\begin{Satz}{Abschlusseigenschaften der kontextfreien Sprachen}
    Die Klasse der kontextfreien Sprachen ist abgeschlossen unter
    Vereinigung, Produkt und Stern.
\end{Satz}

\begin{Beweis}
    Abschluss unter Vereinigung:
    Seien $L_1 = L(G_1)$ und $L_2 = L(G_2)$ kontextfreie Sprachen mit
    $G_1 = (V_1, \Sigma, P_1, S_1)$ und $G_2 = (V_2, \Sigma, P_2, S_2)$, wobei
    $V_1 \cap V_2 = \emptyset$ und $S \notin V_1 \cup V_2$.
    Dann gilt $L(G) = L_1 \cup L_2$ mit der kontextfreien Grammatik\\
    $G = (V_1 \cup V_2 \cup \{S\}, \Sigma,
    P_1 \cup P_2 \cup \{S \rightarrow S_1 | S_2\}, S)$.\\
    Für "`$\supset$"' nimmt man z.\,B. ein Wort $w \in L_1$.
    Dann ist $S_1 \Rightarrow_{G_1}^\ast w$ und es gilt
    $S \Rightarrow_G S_1 \Rightarrow_G^\ast w$, da $P_1$ eine Teilmenge der
    Regelmenge von $G$ ist.\\
    Für "`$\subset$"' sei $w \in L(G)$, also $S \Rightarrow_G w$.
    Da $S$ in keiner anderen Regel außer $S \rightarrow S_1 | S_2$ vorkommt,
    muss z.\,B. $S \Rightarrow_G S_1 \Rightarrow_G w$ gelten.
    Dann gilt allerdings auch $S_1 \Rightarrow_{G_1}^\ast w$, denn
    $S_1$ kann nur nach Variablen in $V_1$ abgeleitet werden,
    diese Variablen können auch nur nach Variablen in $V_1$ abgeleitet werden
    usw.
    Daraus folgt dann $w \in L_1$.
    
    Abschluss unter Produkt:
    Seien $G_1$ und $G_2$ wie eben.
    Dann gilt $L(G) = L_1 L_2$ mit der kontextfreien Grammatik
    $G = (V_1 \cup V_2 \cup \{S\}, \Sigma,
    P_1 \cup P_2 \cup \{S \rightarrow S_1 S_2\}, S)$.
    Der Beweis lässt sich analog durchführen.
    
    Abschluss unter Stern:
    Sei $L_1 = L(G_1)$ eine kontextfreie Sprache mit
    $G_1 = (V_1, \Sigma, P_1, S_1)$, wobei $S \notin V_1$ und
    $S_1$ oBdA auf keiner rechten Seite vorkommt.
    Dann gilt $L(G) = (L_1)^\ast$ mit der kontextfreien Grammatik
    $G = (V_1 \cup \{S\}, \Sigma, P, S)$ mit\\
    $P = (P_1 \setminus \{S_1 \rightarrow \varepsilon\}) \cup
    \{S \rightarrow \varepsilon | S_1\} \cup \{S_1 \rightarrow S_1 S_1\}$.\\
    Für "`$\supset$"' sei $w \in (L_1)^\ast$, d.\,h.
    $w = w_1 \dotsb w_n$ mit $n \in \natural_0$ und $w_i \in L_1$.
    Ist $n = 0$, so gilt $S \Rightarrow_G \varepsilon = w$.\\
    Ist $n \ge 1$, so gilt $S \Rightarrow_G S_1 \Rightarrow_G S_1 S_1
    \Rightarrow_G S_1 S_1 S_1 \Rightarrow_G \dotsb \Rightarrow_G (S_1)^n
    \Rightarrow_G^\ast w_1 \dotsb w_n = w$.\\
    Für "`$\subset$"' ist $w \in L(G)$, d.\,h. $S \Rightarrow_G^\ast w$.
    Der Fall $w = \varepsilon$ ist trivial.
    Da $S_1$ auf keiner rechten Seite in $P_1$ vorkommt, kann man oBdA
    $S \Rightarrow_G S_1 \Rightarrow_G S_1 S_1
    \Rightarrow_G S_1 S_1 S_1 \Rightarrow_G \dotsb \Rightarrow_G (S_1)^n
    \Rightarrow_G^\ast w$ schreiben, indem man die Anwendungen der Regel
    $S_1 \rightarrow S_1 S_1$ zuerst vornimmt
    (die Anzahl dieser Anwendungen sei $n$).
    Es gilt $S_1 \Rightarrow_{G_1} w_i$ für $i = 1, \dotsc, n$ und
    $w = w_1 \dotsb w_n$.
    Somit gilt $w \in (L_1)^\ast$, da $w_i \in L_1$ für $i = 1, \dotsc, n$.
\end{Beweis}

\linie

\begin{Satz}{negative Abschlusseigenschaften der kontextfreien Sprachen}
    Die Klasse der kontextfreien Sprachen ist nicht abgeschlossen unter
    Schnitt und Komplement.
\end{Satz}

\begin{Beweis}
    Nicht-Abgeschlossenheit unter Schnitt:
    Die Sprachen $L_1 = \{a^i b^j c^j \;|\; i, j \in \natural\}$ und
    $L_2 = \{a^i b^i c^j \;|\; i, j \in \natural\}$ sind kontextfrei,
    wie man leicht prüfen kann
    (z.\,B. für $L_1$ die Grammatik $S \rightarrow AB$,
    $A \rightarrow Aa \;|\; a$, $B \rightarrow bBc \;|\; bc$).
    Der Schnitt ist die Sprache $L = \{a^i b^i c^i \;|\; i \in \natural\}$.
    Weiter oben wurde gezeigt, dass $L$ nicht kontextfrei ist.
    
    Nicht-Abgeschlossenheit unter Komplement:
    Angenommen, die Klasse der kontextfreien Sprachen wäre abgeschlossen
    unter Komplement.
    Seien $L_1$ und $L_2$ kontextfreie Sprachen.
    Dann ist $L_1 \cap L_2 = \Sigma^\ast \setminus
    ((\Sigma^\ast \setminus L_1) \cup (\Sigma^\ast \setminus L_2))$.
    Damit wäre dann auch Abgeschlossenheit unter Schnitt vorhanden,
    ein Widerspruch.
\end{Beweis}

\pagebreak

\subsection{%
    Der CYK-Algorithmus%
}

\begin{Bem}
    Das Wortproblem ist für Typ-1-Sprachen entscheidbar, d.\,h.
    es gibt einen Algorithmus, der zu jedem gegebenen Wort $w \in \Sigma^\ast$
    und einer Typ-1-Grammatik $G$ in endlicher Zeit entscheidet, ob
    $w \in L(G)$.
    Der zugehörige Algorithmus hat allerdings exponentielle Zeitkomplexität.\\
    Für den Spezialfall der Typ-2-Sprachen existiert ein optimierter
    Algorithmus zur Lösung des Wortproblems, der höchstens kubische
    Zeitkomplexität besitzt, allerdings die Grammatik in Chomsky-Normalform
    voraussetzt.
    Der Algorithmus nennt sich \begriff{CYK-Algorithmus}
    (\name{Cocke}-\name{Younger}-\name{Kasami}).
\end{Bem}

\linie

\begin{Bem}
    Sei also eine CNF-Grammatik $G$ und ein Wort $w \in \Sigma^\ast$ gegeben.
    Zur Entscheidung der Frage, ob $w \in L(G)$ gilt, werden die Fälle
    $|w| = 1$ und $|w| > 1$ unterschieden.\\
    Für $|w| = 1$ ist $w \in \Sigma$, also schaut man, ob eine Regel
    $S \rightarrow w$ in der Regelmenge $P$ existiert.\\
    Für $|w| > 1$ (unter der Annahme, dass $w \in L(G)$)
    gilt $S \Rightarrow_G AB$ mit $A \Rightarrow_G^\ast x$,
    $B \Rightarrow_G^\ast y$ und $w = xy$, wobei $|x|, |y| \ge 1$.
    Diesen Vorgang kann man für $x$ und $y$ wiederholen usw.
    Bei Wörtern $x$ der Länge $|x| = 1$ gilt $A \Rightarrow^\ast x$ genau dann,
    wenn $A \rightarrow x \in P$.
    
    Allgemein definiert man für ein Wort $w = w_1 \dotsb w_n \in \Sigma^\ast$
    Mengen $T_{i,j} \subset V$ mit $A \in T_{i,j}$ genau dann, wenn
    $A \Rightarrow_G^\ast w_i \dotsb w_{i+j-1}$ mit
    $j \in \{1, \dotsc, n\}$
    und $i \in \{1, \dotsc, n + 1 - j\}$.\\
    Diese Mengen werden induktiv berechnet (mit steigendem $j$).
    Am Ende entscheidet, ob $S \in T_{1,n}$.
    Dies ist der Fall genau dann, wenn $w \in L(G)$.\\
    Wie berechnet man die Mengen $T_{i,j}$?
    Für $j = 1$ ist $T_{i,1} = \{A \in V \;|\; A \rightarrow w_i \in P\}$.
    Für $j = 2$ ist $T_{i,2} = \{A \in V \;|\;
    \exists_{B, C \in V}\; A \rightarrow BC \in P,\; B \rightarrow w_i \in P,\;
    C \rightarrow w_{i+1} \in P\}$.
    Das kann auch umgeschrieben werden zu
    $T_{i,2} = \{A \in V \;|\; \exists_{B \in T_{i,1},\; C \in T_{i+1,1}}\;
    A \rightarrow BC \in P\}$ usw.\\
    Allgemein gilt für $j \ge 2$, dass $T_{i,j} = \{A \in V \;|\;
    \exists_{k \in \{1, \dotsc, j - 1\}}
    \exists_{B \in T_{i,k},\; C \in T_{i+k,j-k}}\; A \rightarrow BC \in P\}$.
    
    Für $j \ge 2$ kann man dies auch als Vereinigung\\
    $T_{i,j} = \bigcup_{k=1}^{j-1} \{A \in V \;|\;
    \exists_{B \in T_{i,k},\; C \in T_{i+k,j-k}}\; A \rightarrow BC \in P\}$
    schreiben.
    Man berechnet die $T_{i,j}$ zuerst für $j = 1$, dann für $j = 2$ usw.
    Um $T_{i,j}$ zu bestimmen, kann man ausnutzen, dass $T_{r,s}$ für
    $s < j$ und beliebige $r$ schon bekannt ist.
    Da andere Mengen in der Vereinigung $T_{i,j}$ nicht vorkommen, kann man
    so alle $T_{i,j}$ algorithmisch bestimmen.
    Genauer organisiert man die Mengen in einer Tabelle:
    \[\begin{array}{r|c|c|ccc|c}
        & w_1 & w_2 & w_3 & \cdots & w_{n-1} & w_n \\\hline
        \text{Länge } 1 & T_{1,1} & T_{2,1} & T_{3,1} & \cdots &
        T_{n-1,1} & T_{n,1} \\\hline
        \text{Länge } 2 & T_{1,2} & T_{2,2} & T_{3,2} & \cdots &
        T_{n-1,2} & - \\\hline
        \vdots \\\hline
        \text{Länge } n - 1 & T_{1,n-1} & T_{2,n-1} & - & \cdots & - &
        - \\\hline
        \text{Länge } n & T_{1,n} & - & - & \cdots & - & -
    \end{array}\]
    Dann ist $w \in L(G)$ äquivalent zu $S \in T_{1,n}$.
\end{Bem}

\linie
\pagebreak

\begin{Satz}{CYK-Algorithmus}
    Sei $G$ eine kontextfreie Grammatik in Chomsky-Normalform.
    Dann ermittelt folgender Algorithmus für alle
    $w = w_1 \dotsb w_n \in \Sigma^\ast$, dass $w \in L(G)$
    (Wortproblem):
    \begin{align*}&
        \FOR i := 1 \;\TO n \;\DO\\&
        \qquad T[i, 1] := \{A \in V \;|\; A \rightarrow w_i \in P\};\\&
        \END\\&
        \FOR j := 2 \;\TO n \;\DO\\&
        \qquad\FOR i := 1 \;\TO n + 1 - j \;\DO\\&
        \qquad\qquad T[i, j] := \emptyset;\\&
        \qquad\qquad\FOR k := 1 \;\TO j - 1 \;\DO\\&
        \qquad\qquad\qquad T[i, j] := T[i, j] \cup \{A \in V \;|\;
            \exists_{B \in T[i, k],\; C \in T[i + k, j - k]}\;
            A \rightarrow BC \in P\};\\&
        \qquad\qquad\END\\&
        \qquad\END\\&
        \END\\&
        \IF S \in T[1, n] \;\THEN
        \text{output}(1) \;\ELSE
        \text{output}(0);
    \end{align*}
    Der Algorithmus hat die bestmögliche Zeitkomplexität $\O(n^3)$.
\end{Satz}

\linie

\begin{Bsp}
    Sei die CNF-Grammatik $G = (\{S, A, B, X, Y\}, \{a, b\}, P, S)$
    gegeben mit\\
    $P = \{S \rightarrow AX \;|\; YB,
    A \rightarrow XA \;|\; AB \;|\; a,
    B \rightarrow XY \;|\; BB,
    X \rightarrow YA \;|\; a,
    Y \rightarrow XX \;|\; b\}$.\\
    Gilt $w = aabbaba \in L(G)$?
    Der Algorithmus erzeugt folgende Tabelle:
    \[\begin{array}{r|c|c|c|c|c|c|c}
        \text{Länge} & a & a & b & b & a & b & a \\\hline
        1 & AX & AX & Y & Y & AX & Y & AX \\\hline
        2 & SAY & B & \emptyset & X & B & X & - \\\hline
        3 & A & \emptyset & \emptyset & SB & SY & - & - \\\hline
        4 & \emptyset & \emptyset & S & Y & - & - & - \\\hline
        5 & S & B & \emptyset & - & - & - & - \\\hline
        6 & A & \emptyset & - & - & - & - & - \\\hline
        7 & S & - & - & - & - & - & -
    \end{array}\]
    (Dabei bedeutet $V_1 \dotsb V_n$ die Menge
    $\{V_1, \dotsc, V_n\} \subset V$.)
    Also gilt $w \in L(G)$.
\end{Bsp}

\pagebreak

\subsection{%
    Kellerautomaten%
}

\begin{Bem}
    Das Modell des nicht-deterministischen endlichen Automaten (NEA) soll so
    erweitert werden, dass auch kontextfreie Sprachen erkannt werden.
    Dazu muss ein Speicher eingeführt werden, z.\,B. bei der Sprache
    $\{a^n b^n \;|\; n \in \natural\}$ muss $n$ gespeichert werden,
    bei $\{a_1 \dotsb a_n \dollar a_n \dotsb a_1 \;|\; n \in \natural\}$
    muss $a_1 \dotsb a_n$ gespeichert werden usw.
\end{Bem}

\begin{Bem}
    Einen Kellerautomat stellt man sich als Maschine vor, die aus
    Eingabeband, Lesekopf, Zustandskontrolle, Keller und
    Schreib-/Lesekopf für den Keller besteht.
    In jedem Schritt kann die Zustandskontrolle höchstens ein Zeichen lesen,
    der Lesekopf bewegt sich dabei unwiderruflich nach vorne.
    Gleichzeitig ist das letzte dem Keller hinzugefügte Element sichtbar,
    bei Bedarf kann mehr Information auf dem Keller gespeichert oder
    auch bestehende Information des Kellers gelöscht werden.
    Der Keller ist dabei ein Pushdown-Stack, d.\,h. die Informationen sind
    nach ihrem Ablegen absteigend sortiert.
\end{Bem}

\linie

\begin{Def}{nicht-deterministischer Kellerautomat (PDA)}\\
    Ein \begriff{nicht-deterministischer Kellerautomat} oder
    \begriff{PDA (pushdown automaton)}
    ist ein $6$-Tupel $M = (Z, \Sigma, \Gamma, \delta, z_0, \#)$, wobei
    \begin{itemize}
        \item
        $Z$ eine endliche, nicht-leere Menge
        (die Menge der \begriff{Zustände}),
        
        \item
        $\Sigma$ eine endliche, nicht-leere Menge
        mit $Z \cap \Sigma = \emptyset$
        (das \begriff{Eingabealphabet}),
        
        \item
        $\Gamma$ eine endliche, nicht-leere Menge
        mit $Z \cap \Gamma = \emptyset$
        (das \begriff{Kelleralphabet}),
        
        \item
        $\delta\colon Z \times (\Sigma \cup \{\varepsilon\}) \times \Gamma
        \rightarrow \P_E(Z \times \Gamma^\ast)$
        (die \begriff{Überführungsfunktion}),
        
        \item
        $z_0 \in Z$ (der \begriff{Startzustand}) und
        
        \item
        $\# \in \Gamma$ (das \begriff{unterste Kellersymbol}) ist.
    \end{itemize}
    Dabei ist $\P_E(A) := \{B \subset A \;|\; B \text{ endlich}\}$.
\end{Def}

\begin{Bem}
    Es gibt zwei Arten von Übergängen:\\
    Bei \begriff{normalen Übergängen}
    $(q, B_1 \dotsb B_k) \in \delta(p, a, A)$
    wird der Buchstabe $a$ vom Eingabeband gelesen und der Lesekopf um
    eins weitergerückt.
    Gleichzeitig wird der oberste Kellerbuchstabe $A$ vom Keller gelöscht und
    durch $B_1 \dotsb B_k$ mit $k \in \natural_0$ ersetzt.\\
    Bei \begriff{$\varepsilon$-Übergängen}
    $(q, B_1 \dotsb B_k) \in \delta(p, \varepsilon, A)$
    läuft alles analog, außer dass kein Eingabebuchstabe gelesen wird
    (d.\,h. die Position des Lesekopfs bleibt unverändert).
\end{Bem}

\linie

\begin{Def}{Konfiguration}\\
    Eine \begriff{Konfiguration} $k$ des PDA
    $M = (Z, \Sigma, \Gamma, \delta, z_0, \#)$
    ist ein Element $k \in Z \times \Sigma^\ast \times \Gamma^\ast$.
\end{Def}

\begin{Def}{Übergangsrelation}
    Auf der Menge $Z \times \Sigma^\ast \times \Gamma^\ast$ wird eine
    Relation $\vdash$ definiert durch
    $(z, a_1 a_2 \dotsb a_n, A_1 A_2 \dotsb A_m) \vdash
    (z', a_2 \dotsb a_n, B_1 \dotsb B_k A_2 \dotsb A_m)$ für
    $(z', B_1 \dotsb B_k) \in \delta(z, a_1, A_1)$ bzw.\!
    $(z, a_1 \dotsb a_n, A_1 A_2 \dotsb A_m) \vdash
    (z', a_1 \dotsb a_n, B_1 \dotsb B_k A_2 \dotsb A_m)$ für
    $(z', B_1 \dotsb B_k) \in \delta(z, \varepsilon, A_1)$
    mit $n \in \natural$ bzw. $n \in \natural_0$,
    $m \in \natural$ und $k \in \natural_0$ (\begriff{Übergangsrelation}).\\
    $\vdash^\ast$ ist der ref"|lexive und transitive Abschluss von $\vdash$.
\end{Def}

\begin{Def}{akzeptierte Sprache}
    Die von einem PDA $M = (Z, \Sigma, \Gamma, \delta, z_0, \#)$
    \begriff{akzeptierte Sprache} ist
    $N(M) := \{x \in \Sigma^\ast \;|\;
    \exists_{z \in Z}\; (z_0, x, \#) \vdash^\ast
    (z, \varepsilon, \varepsilon)\}$.
\end{Def}

\begin{Bem}
    Diese Version der Definition heißt akzeptierte Sprache durch leeren Keller,
    es gibt auch Akzeptierung durch Endzustand.
    Wenn der Keller leer ist, endet die Berechnung in jedem Fall, denn für
    leeres Eingabeband ist die Berechnung erfolgreich, andernfalls ist sie
    nicht erfolgreich (für jeden Übergang wird ein Kellerzeichen benötigt).
\end{Bem}

\linie
\pagebreak

\begin{Bsp}
    Gesucht wird ein PDA für \begriff{markierte Palindrome}, d.\,h. für\\
    $L = \{a_1 \dotsb a_n \dollar a_n \dotsb a_1 \;|\; n \in \natural\}$.
    Man definiert $M = (\{z_0, z_1\}, \{a, b, \dollar\}, \{\#, A, B\},
    \delta, z_0, \#)$ und die Kurzschreibweise
    $zaA \rightarrow z'x$ für $(z', x) \in \delta(z, a, A)$.
    Dann setzt man\\
    $z_0 a \# \rightarrow z_0 A \#$,
    $z_0 a A \rightarrow z_0 A A$,
    $z_0 a B \rightarrow z_0 A B$ und analog\\
    $z_0 b \# \rightarrow z_0 B \#$,
    $z_0 b A \rightarrow z_0 B A$,
    $z_0 b B \rightarrow z_0 B B$.
    Außerdem ist\\
    $z_0 \dollar \# \rightarrow z_1 \#$,
    $z_0 \dollar A \rightarrow z_1 A$,
    $z_0 \dollar B \rightarrow z_1 B$ sowie
    $z_1 a A \rightarrow z_1 \varepsilon$,
    $z_1 b B \rightarrow z_1 \varepsilon$,
    $z_1 \varepsilon \# \rightarrow z_1 \varepsilon$.\\
    Um zu zeigen, dass $N(M) = L$ gilt, zeigt man zunächst
    $w \dollar w^R \in N(M)$ für alle $w \in \{a, b\}^\ast$, d.\,h.
    $(z_0, w \dollar w^R, \#) \vdash^\ast (z_i, \varepsilon, \varepsilon)$
    für ein $i \in \{0, 1\}$
    (dabei ist $w^R := w_n \dotsb w_1$ für $w = w_1 \dotsb w_n$).\\
    Ein möglicher Zwischenschritt ist dabei die stärkere Behauptung\\
    $\forall_{w \in \{a, b\}^+}\; (z_0, w \dollar w^R, \#) \vdash^\ast
    (z_0, \dollar w^R, \widehat{w}^R \#) \vdash
    (z_1, w^R, \widehat{w}^R \#) \vdash^\ast
    (z_1, \varepsilon, \varepsilon)$ mit
    $\widehat{w} = \widehat{w}_1 \dotsb \widehat{w}_n$ und
    $\widehat{a} = A$, $\widehat{b} = B$ für $w = w_1 \dotsb w_n$.
    Die erste Relation $\vdash^\ast$ lässt sich durch Induktion zeigen
    ($(z_0, v \dollar w^R, y) \vdash^\ast
    (z_0, \dollar w^R, \widehat{v}^R y)$ für $v \in \{a, b\}^\ast$ beliebig).
    Analog zeigt man die andere Richtung\\
    $\forall_{x \in \{a, b\}^\ast}\; (\exists_{i \in \{0, 1\}}\;
    (z_0, x, \#) \vdash^\ast (z_i, \varepsilon, \varepsilon)) \;\Rightarrow\;
    (\exists_{w \in \{a, b\}^\ast}\; x = w \dollar w^R)$
    (zunächst stellt man fest, dass nur $i = 1$ möglich ist, da
    von $z_0$ aus $\#$ nicht entfernt wird, danach verfährt man ähnlich wie
    eben).
\end{Bsp}

\linie

\begin{Satz}{PDA charakterisieren die kontextfreien Sprachen}\\
    Eine Sprache $L$ ist genau dann kontextfrei, wenn sie von einem PDA
    erkannt wird.
\end{Satz}

\begin{Beweis}
    Zunächst sei eine kontextfreie Grammatik $G = (V, \Sigma, P, S)$ gegeben.
    Gesucht ist also ein PDA $M$ mit $N(M) = L(G)$.
    Dazu wählt man $M = (\{z\}, \Sigma, V \cup \Sigma, \delta, z, S)$ und
    $\delta$ gegeben durch
    $(z, \alpha) \in \delta(z, \varepsilon, X)$ für alle Regeln
    $X \rightarrow \alpha \in P$ und
    $(z, \varepsilon) \in \delta(z, a, a)$ für alle Terminale $a \in \Sigma$.
    Anschaulich wird also $X$ auf dem Keller durch $\alpha$ ersetzt bzw.
    passende Terminale, die ganz oben auf dem Keller
    (also ganz am Anfang der Ableitung) liegen, einfach weggelesen.
    Der Kellerinhalt symbolisiert die momentane Ableitung von oben nach unten.
    Man kann zeigen, dass $N(M) = L(G)$ gilt.
    
    Nun sei ein PDA $M = (Z, \Sigma, \Gamma, \delta, z_0, \#)$ gegeben.
    Gesucht ist eine Typ-2-Grammatik $G$ mit $N(M) = L(G)$.
    OBdA vergrößere $M$ den Keller bei jedem Übergang um maximal ein Symbol.
    Dies kann man immer erreichen, indem man andernfalls mehr Zustände
    einführt, die per $\varepsilon$-Übergänge wie in einer Kette miteinander
    verbunden sind.
    Die Grammatik $G$ sei definiert durch $G = (V, \Sigma, P, S)$ mit
    $V = \{S\} \cup (Z \times \Gamma \times Z)$.
    Anschaulich bedeutet die Variable $(z_1, A, z_2)$, dass man im Zustand
    $z_1$ mit dem obersten Kellersymbol $A$ startet und das Ziel hat,
    im Zustand $z_2$ zu sein, wenn $A$ aus dem Keller erstmals entfernt wird.
    In der Grammatik werden $\varepsilon$-Regeln erlaubt
    (diese kann man oBdA in eine Grammatik ohne $\varepsilon$-Regeln umformen).
    Dann befinden sich für $a \in \Sigma \cup \{\varepsilon\}$ und
    $z \in Z$ die Regeln
    $S \rightarrow (z_0, \#, z)$,
    $(z, A, z') \rightarrow a$
    (falls $(z', \varepsilon) \in \delta(z, a, A)$),
    $(z, A, z') \rightarrow a (z_1, B, z')$
    (falls $(z_1, B) \in \delta(z, a, A)$) und
    $(z, A, z') \rightarrow a (z_1, B, z_2) (z_2, C, z')$
    (falls $(z_1, BC) \in \delta(z, a, A)$) in $P$.
    Man kann zeigen, dass $N(M) = L(G)$ gilt.
\end{Beweis}

\begin{Bem}
    Im Beweis sieht man:
    Jede kontextfreie Sprache kann von einem PDA erkannt werden, der nur
    einen einzigen Zustand besitzt.
    Zu jeder kontextfreien Sprache
    (z.\,B. in Greibach-Normalform gegeben)
    gibt es einen PDA, der in Echtzeit arbeitet, d.\,h. in jedem Schritt wird
    ein Zeichen eingelesen.
\end{Bem}

\pagebreak

\subsection{%
    Deterministisch kontextfreie Sprachen%
}

\begin{Bem}
    Für viele Anwendungen sind die kontextfreien Sprachen zu allgemein,
    während die regulären Sprachen zu speziell sind.
    Man führt daher eine echte Teilmenge bzw. Obermenge der kontextfreien bzw.
    regulären Sprachen ein.
\end{Bem}

\begin{Def}{deterministischer Kellerautomat (DPDA)}\\
    Ein \begriff{deterministischer Kellerautomat} oder
    \begriff{DPDA (deterministic pushdown automaton)} ist ein
    PDA $M = (Z, \Sigma, \Gamma, \delta, z_0, \#)$ mit einer
    Endzustandsmenge $E \subset Z$, sodass\\
    $\forall_{z \in Z} \forall_{a \in \Sigma} \forall_{A \in \Gamma}\;
    |\delta(z, a, A)| + |\delta(z, \varepsilon, A)| \le 1$.
    DPDAs akzeptieren durch Endzustand und nicht durch leeren Keller, d.\,h.
    $N(M) := \{x \in \Sigma^\ast \;|\;
    \exists_{z \in E} \exists_{W \in \Gamma^\ast}\;
    (z_0, x, \#) \vdash^\ast (z, \varepsilon, W)\}$.
\end{Def}

\begin{Bsp}
    Anschaulich gibt es in jeder Konfiguration
    $k \in Z \times \Sigma^\ast \times \Gamma^\ast$ höchstens eine
    Folgekonfiguration $k' \in Z \times \Sigma^\ast \times \Gamma^\ast$ mit
    $k \vdash k'$.
    Für PDAs ist Akzeptierung durch Endzustand und leeren Keller äquivalent.
    Für DPDAs gilt dies nicht mehr, denn ist nach dem Lesen eines Wortes
    $w \in \Sigma^\ast$ durch den DPDA der Keller leer, so gibt es kein
    Wort $ww'$ mit $w' \in \Sigma^+$, das ebenfalls akzeptiert werden würde,
    denn nach dem Lesen von $w$ befindet sich der Automat immer in demselben
    Zustand.
\end{Bsp}

\linie

\begin{Def}{$\CFL$}\\
    Die Menge $\CFL := \{L \subset \Sigma^\ast \;|\;
    \exists_{\text{PDA } M}\; N(M) = L\}$
    ist die Menge aller kontextfreien Sprachen.
\end{Def}

\begin{Def}{$\DCFL$}
    Die Menge $\DCFL := \{L \subset \Sigma^\ast \;|\;
    \exists_{\text{DPDA } M}\; N(M) = L\}$
    ist die Menge aller\\
    \begriff{deterministisch kontextfreien Sprachen}.
\end{Def}

\begin{Bsp}
    Beispiele für deterministisch kontextfreie Sprachen sind
    die markierten Palindrome $\{w \dollar w^R \;|\; w \in \Sigma^\ast\}$,
    $\{a^n b^n \;|\; n \in \natural\}$,
    $L_1 = \{a^n b^n c^m \;|\; m, n \in \natural\}$,
    $L_2 = \{a^m b^n c^n \;|\; m, n \in \natural\}$ und
    $\{a^n b^m c^n \;|\; m, n \in \natural\}$.
\end{Bsp}

\linie

\begin{Satz}{Abschlusseigenschaften von $\DCFL$}\\
    $\DCFL$ ist abgeschlossen unter Komplement, aber nicht unter
    Durchschnitt und Vereinigung.
\end{Satz}

\begin{Beweis}
    Für das Komplement komplementiert man die Endzustandsmenge des DPDAs.
    Dies genügt allerdings noch nicht:
    Befindet sich der DPDA nach dem Lesen von $w$ in einem Zustand in
    $Z \setminus E$, heißt das noch nicht, dass $w \notin N(M)$ gilt,
    denn man könnte noch durch $\varepsilon$-Übergänge in einen Endzustand
    wechseln.
    Das entstehende Problem ist nicht-trivial, kann aber bewiesen werden.
    
    Die Sprachen $L_1$ und $L_2$ aus obigem Beispiel sind in $\DCFL$,
    aber der Schnitt $L_1 \cap L_2$ ist nicht einmal in $\CFL$, also auch
    nicht in $\DCFL$.
    
    Die Nicht-Abgeschlossenheit unter Vereinigung ergibt sich aus den Regeln
    von de Morgan.
\end{Beweis}

\begin{Bem}
    $\DCFL$ ist aufgrund der unterschiedlichen Abschlusseigenschaften echt in
    $\CFL$ enthalten.
    Außerdem ist $\DCFL$ eine echte Obermenge von $\REG$, denn auch hier
    unterscheiden sich die Abschlusseigenschaften
    (alternativ sucht man entsprechende Sprachen).
\end{Bem}

\linie

\begin{Satz}{Abschlusseigenschaften von $\DCFL$ und $\CFL$ mit $\REG$}\\
    Aus $L \in \DCFL$ und $L' \in \REG$ folgt $L \cap L' \in \DCFL$.\\
    Aus $L \in \CFL$ und $L' \in \REG$ folgt $L \cap L' \in \CFL$.
\end{Satz}

\begin{Beweis}
    Für beide Aussagen wählt man Akzeptierung durch Endzustand und kombiniert
    den jeweiligen (D)PDA für $L$ mit dem DEA für $L'$ (Kreuzprodukt).
    Endzustände des kombinierten (D)PDA sind die Paare, bei denen beiden
    Komponenten (im (D)PDA und im DEA) Endzustände sind.
\end{Beweis}

\pagebreak

\subsection{%
    Entscheidbarkeit bei kontextfreien Sprachen%
}

\begin{Satz}{Entscheidbarkeit bei kontextfreien Sprachen}
    Das Wortproblem, das Leerheitsproblem und das Endlichkeitsproblem sind
    für kontextfreie Sprachen entscheidbar.
\end{Satz}

\begin{Beweis}
    Das Wortproblem ist sogar \begriff{ef"|fizient entscheidbar}
    (durch CYK-Algorithmus bei Grammatik in CNF).
    
    Das Leerheitsproblem ist z.\,B. mit dem Pumping-Lemma für kontextfreie
    Sprachen $L$ lösbar.
    Man wählt die Zahl $n$ aus dem Pumping-Lemma
    (Anzahl der Variablen der Grammatik) und prüft alle Wörter der Länge $< n$
    auf Mitgliedschaft in $L$.
    Dies sind endlich viele, d.\,h. man kann dies mit einem Algorithmus
    entscheiden.
    Falls es ein solches gibt, so ist $L \not= \emptyset$.
    Umgekehrt folgt aus $L \not= \emptyset$, dass es ein Wort der Länge $< n$
    in $L$ gibt
    (ein Wort in $L$ der Länge $\ge n$ kann man negativ pumpen und erhält ein
    kürzeres Wort in $L$).\\
    Alternativ kann man einen Markierungsalgorithmus entwickeln, der die
    Menge der \begriff{produktiven Variablen}
    (die Variablen, die in ein Terminalwort abgeleitet werden können)
    findet (hier ist die Frage, ob $S$ produktiv ist).
    
    Die Endlichkeit geht auch mit dem Pumping-Lemma und analog zu den regulären
    Sprachen (gibt es kein $w \in L$ mit $n \le |w| < 2n$?).
\end{Beweis}

\linie

\begin{Satz}{Entscheidbarkeit bei deterministisch kontextfreien Sprachen}
    Das Problem "`Gleichheit mit regulären Sprachen"' ist für deterministisch
    kontextfreie Sprachen entscheidbar, d.\,h.
    für $L_1 \in \DCFL$ und $L_2 \in \REG$ ist die Frage, ob
    $L_1 = L_2$ gilt, entscheidbar.
\end{Satz}

\begin{Beweis}
    Es gilt $L_1 = L_2$ genau dann, wenn
    $L_1 \subset L_2$ und $L_2 \subset L_1$.
    Dies kann man umschreiben zu $L_1 \setminus L_2 = \emptyset$ und
    $L_2 \setminus L_1 = \emptyset$.\\
    Mit Komplementen dargestellt ist dies äquivalent zu
    $L_1 \cap (\Sigma^\ast \setminus L_2) = \emptyset$ und
    $L_2 \cap (\Sigma^\ast \setminus L_1) = \emptyset$.
    Da $L_1 \cap (\Sigma^\ast \setminus L_2)$ und
    $L_2 \cap (\Sigma^\ast \setminus L_1)$ aufgrund der Abschlusseigenschaften
    in $\DCFL$ sind, ist das Gleichheitsproblem mit dem Leerheitsproblem
    entscheidbar.
\end{Beweis}

\pagebreak
