\section{%
    Reguläre Sprachen%
}

\subsection{%
    Deterministische Automaten%
}

\begin{Bem}
    Während Grammatiken mit vordefinierten Regeln Wörter einer Sprache
    \emph{erzeugen} können, tun Automaten in gewisser Weise das Gegenteil.
    Automaten \emph{erkennen} Wörter, d.\,h. man gibt ein Wort ein und der
    Automat gibt zurück, ob das Wort erkannt wurde
    (zu einer bestimmten Sprache gehört) oder nicht.\\
    Das Ziel der folgenden Abschnitte wird es sein zu zeigen, dass
    DEAs dasselbe "`können"' wie Typ-3-Grammatiken.
    Dafür wird das Hilfsmittel des NEAs eingeführt, der ebenfalls genau so viel
    kann wie DEAs.
\end{Bem}

\linie

\begin{Def}{deterministischer endlicher Automat (DEA)}
    Ein \begriff{deterministischer endlicher Automat (DEA)}
    oder \begriff{DFA} ist ein $5$-Tupel $M = (Z, \Sigma, \delta, z_0, E)$,
    wobei
    \begin{itemize}
        \item
        $Z$ eine endliche, nicht-leere Menge
        (die Menge der \begriff{Zustände}),

        \item
        $\Sigma$ eine endliche, nicht-leere Menge
        mit $Z \cap \Sigma = \emptyset$
        (das \begriff{Alphabet}),

        \item
        $\delta\colon Z \times \Sigma \rightarrow Z$
        (die \begriff{Überführungsfunktion}),

        \item
        $z_0 \in Z$ (der \begriff{Startzustand}) und

        \item
        $E \subset Z$ (die \begriff{akzeptierenden Endzustände}) ist.
    \end{itemize}
\end{Def}

\begin{Bem}
    Bildhaft kann man sich einen Automat vorstellen als eine Maschine,
    die von einem endlichen Eingabeband Zeichen für Zeichen einliest.
    Die Maschine hat einen Lesekopf, der sich zu Beginn am Anfang des
    Eingabebands befindet, und speichert sich den aktuellen Zustand $q$,
    der anfangs $z_0$ ist.
    Die Maschine liest ein Zeichen $y$ und ermittelt durch $\delta(q, y)$
    den Zustand, der als nächstes angenommen wird.
    Diesen speichert sie sich als neuen Zustand und setzt den Lesekopf um ein
    Zeichen weiter.
    Ist das Ende das Bands erreicht, so zeigt die Maschine an,
    ob der erreichte Zustand $q$ ein Endzustand ist
    (d.\,h. $q \in E$).
\end{Bem}

\begin{Bem}
    Automaten kann man durch gerichtete, beschriftete Graphen
    (den \begriff{Zustandsgra\-phen}) darstellen.
    Die Zustände entsprechen den Knoten.
    Der dem Startzustand entsprechende Knoten wird durch einen eingehenden
    Pfeil ("`aus dem Nichts"') besonders markiert.
    Endzustände werden durch doppelte Kreise gekennzeichnet.
    Die Kanten veranschaulichen $\delta$:\\
    Für alle $z_1 \in Z$ und $a \in \Sigma$ geht eine mit $a$ beschriftete
    Kante von $z_1$ nach $z_2 = \delta(z_1, a)$.
\end{Bem}

\linie

\begin{Def}{akzeptierte Sprache}
    Sei $M = (Z, \Sigma, \delta, z_0, E)$ ein DEA.\\
    Man definiert induktiv die Abbildung
    $\widehat{\delta}\colon Z \times \Sigma^\ast \rightarrow Z$ durch
    $\widehat{\delta}(z, \varepsilon) := z$ für alle $z \in Z$\\
    und $\widehat{\delta}(z, ax) := \widehat{\delta}(\delta(z, a), x)$ für alle
    $z \in Z$, $a \in \Sigma$ und $x \in \Sigma^\ast$.\\
    Die von $M$ \begriff{akzeptierte Sprache} ist
    $T(M) := \{x \in \Sigma^\ast \;|\; \widehat{\delta}(z_0, x) \in E\}$.
\end{Def}

\begin{Bem}
    Die Abbildung $\widehat{\delta}$ gibt an, in welchen Zustand man gelangt,
    wenn man von einem bestimmten Zustand startet und ein Wort aus
    $\Sigma^\ast$ eingelesen wird.
    In der Tat gilt\\
    $\widehat{\delta}(z, a_1 \dotsb a_n) = z'
    \;\Leftrightarrow\; \exists_{q_1, \dotsc, q_n \in Z}\;
    \delta(z, a_1) = q_1,\; \delta(q_i, a_{i+1}) = q_{i+1},
    i = 1, \dotsc, n - 1, q_n = z'$.\\
    Außerdem folgt aus $x, y \in \Sigma^\ast$, dass
    $\widehat{\delta}(z, xy) = \widehat{\delta}(\widehat{\delta}(z, x), y)$
    für alle $z \in Z$.\\
    Diese Aussagen folgen direkt aus
    $\widehat{\delta}(z, a_1 \dotsb a_n) =
    \delta(\delta(\dotsb\delta(\delta(z,
    a_1), a_2)\dotsb, a_{n-1}), a_n)$,
    was man induktiv beweisen kann.
\end{Bem}

\linie
\pagebreak

\begin{Def}{$\DEA$}
    Die Menge $\DEA := \{L \subset \Sigma^\ast \;|\;
    \exists_{\text{det. endl. Automat } M}\; T(M) = L\}$
    ist die Menge aller Sprachen, die von DEAs akzeptiert werden.
\end{Def}

\begin{Def}{$\REG$}
    Die Menge $\REG := \{L \subset \Sigma^\ast \;|\;
    \exists_{\text{Typ-3-Grammatik } G}\; L(G) = L\}$
    ist die Menge aller Typ-3-Sprachen.
\end{Def}

\begin{Satz}{obere Schranke für DEA}
    Es gilt $\DEA \subset \REG$\\
    (d.\,h. für jeden DEA $M$ gibt es eine Typ-3-Grammatik
    $G$ mit $L(G) = T(M)$).
\end{Satz}

\begin{Beweis}
    Sei $M = (Z, \Sigma, \delta, z_0, E)$ ein DEA.
    Definiere $G = (V, \Sigma, P, S)$ mit $V = Z$, $S = z_0$ und $P$ wie folgt:
    Für alle $p, q \in Z$ und $a \in \Sigma$ mit $q = \delta(p, a)$ wird
    die Regel $p \rightarrow aq$ in $P$ aufgenommen.
    Ist zusätzlich $q \in E$, so wird auch noch $p \rightarrow a$ in $P$
    aufgenommen.
    $G$ ist regulär.

    Zu zeigen ist $T(M) = L(G)$.
    Sei $x = a_1 \dotsb a_n$. Dann gilt
    $x \in T(M)
    \iff \widehat{\delta}(z_0, x) \in E$\\
    $\iff \exists_{q_1, \dotsc, q_n \in Z}\; \delta(z_0, a_1) = q_1,\;
    \delta(q_i, a_{i+1}) = q_{i+1}\; (i = 1, \dotsc, n - 1),\;
    q_n \in E$\\
    $\iff \exists_{q_1, \dotsc, q_n \in V}\; z_0 \rightarrow a_1 q_1 \in P,\;
    q_i \rightarrow a_{i+1} q_{i+1} \in P\; (i = 1, \dotsc, n - 1),\;
    q_{n-1} \rightarrow a_n \in P$\\
    $\iff \exists_{q_1, \dotsc, q_n \in V}\;
    z_0 \Rightarrow_G a_1 q_1 \Rightarrow_G a_1 a_2 q_2 \Rightarrow_G
    \dotsb \Rightarrow_G a_1 a_2 \dotsb a_{n-1} q_{n-1} \Rightarrow_G
    a_1 a_2 \dotsb a_{n-1} a_n = x$\\
    $\iff x \in L(G)$.
\end{Beweis}

\subsection{%
    Nichtdeterministische Automaten%
}

\begin{Def}{nichtdeterministischer endlicher Automat (NEA)}
    Ein \begriff{nichtdeterministischer\\
    endlicher Automat (NEA)}
    oder \begriff{NFA} ist ein $5$-Tupel $M = (Z, \Sigma, \delta, Z_0, E)$,
    wobei
    \begin{itemize}
        \item
        $Z$ eine endliche, nicht-leere Menge
        (die Menge der \begriff{Zustände}),

        \item
        $\Sigma$ eine endliche, nicht-leere Menge
        mit $Z \cap \Sigma = \emptyset$
        (das \begriff{Alphabet}),

        \item
        $\delta\colon Z \times \Sigma \rightarrow \P(Z)$
        (die \begriff{Überführungsfunktion}),

        \item
        $Z_0 \subset Z$ (die \begriff{Startzustände}) und

        \item
        $E \subset Z$ (die \begriff{akzeptierenden Endzustände}).
    \end{itemize}
\end{Def}

\begin{Bem}
    Auch NEAs können durch Zustandsgraphen dargestellt werden.
    Es gibt nun jedoch evtl. mehrere Startzustände und von jedem Knoten
    können mehrere mit demselben Buchstaben beschriftete Kanten ausgehen
    (oder auch keine).
\end{Bem}

\begin{Def}{akzeptierte Sprache}
    Sei $M = (Z, \Sigma, \delta, z_0, E)$ ein NEA.\\
    Man definiert induktiv die Abbildung
    $\widehat{\delta}\colon \P(Z) \times \Sigma^\ast \rightarrow \P(Z)$ durch
    $\widehat{\delta}(Q, \varepsilon) := Q$ für alle $Q \subset Z$ und
    $\widehat{\delta}(Q, ax) := \bigcup_{q \in Q}
    \widehat{\delta}(\delta(q, a), x)$ für alle
    $Q \subset Z$, $a \in \Sigma$ und $x \in \Sigma^\ast$.\\
    Die von $M$ \begriff{akzeptierte Sprache} ist
    $T(M) := \{x \in \Sigma^\ast \;|\;
    \widehat{\delta}(Z_0, x) \cap E \not= \emptyset\}$.
\end{Def}

\linie

\pagebreak

\begin{Def}{$\NEA$}
    Die Menge $\NEA := \{L \subset \Sigma^\ast \;|\;
    \exists_{\text{nichtdet. endl. Automat } M}\; T(M) = L\}$
    ist die Menge aller Sprachen, die von NEAs akzeptiert werden.
\end{Def}

\begin{Bem}
    Es ist klar, dass $\DEA \subset \NEA$ gilt.
\end{Bem}

\begin{Satz}{Satz von \name{Rabin} und \name{Scott}}
    Es gilt $\NEA \subset \DEA$\\
    (d.\,h. für jeden NEA $M$ gibt es einen DEA $M'$ mit $T(M') = T(M)$).
\end{Satz}

\begin{Beweis}
    Sei $M = (Z, \Sigma, \delta, Z_0, E)$ ein NEA.
    Definiere den DEA $M' = (\P(Z), \Sigma, \delta', Z_0, E')$ wie folgt:
    $\delta'(Q, a) := \widehat{\delta}(Q, a) =
    \bigcup_{q \in Q} \delta(q, a)$ und
    $E' := \{Q \in \P(Z) \;|\; Q \cap E \not= \emptyset\}$.

    Zunächst beweist man folgendes Lemma:
    Für alle $Q \subset Z$ und $w \in \Sigma^\ast$ gilt
    $\widehat{\delta}(Q, w) = \widehat{\delta'}(Q, w)$.\\
    Der Beweis erfolgt über Induktion über $n = |w| \in \natural_0$.\\
    Der Induktionsanfang ist klar
    ($Q = \widehat{\delta}(Q, \varepsilon) =
    \widehat{\delta'}(Q, \varepsilon) = Q$).\\
    Beim Induktionsschritt $n \rightarrow n + 1$ ist die
    Induktionsvoraussetzung, dass
    $\widehat{\delta}(Q, x) = \widehat{\delta'}(Q, x)$
    für alle $Q \subset Z$ und $x \in \Sigma^\ast$ mit $|x| \le n$.
    Für beliebige $P \subset Z$ und $a \in \Sigma$ gilt somit\\
    $\widehat{\delta}(P, ax) =
    \widehat{\delta}(\delta(P, a), x) =
    \widehat{\delta'}(\delta(P, a), x) =
    \widehat{\delta'}(\widehat{\delta}(P, a), x) =
    \widehat{\delta'}(\delta'(P, a), x) =
    \widehat{\delta'}(P, ax)$.

    Zu zeigen ist $T(M) = T(M')$.
    Mit der Hilfsbehauptung ergibt sich\\
    $w \in T(M)
    \iff \widehat{\delta}(Z_0, w) \cap E \not= \emptyset
    \iff \widehat{\delta'}(Z_0, w) \cap E \not= \emptyset
    \iff \widehat{\delta'}(Z_0, w) \in E'$\\
    $\iff w \in T(M')$.
\end{Beweis}

\begin{Bem}
    Somit ist $\DEA = \NEA$, d.\,h. DEAs und NEAs "`können"' dasselbe.
\end{Bem}

\linie

\begin{Def}{Potenzmengenkonstruktion}
    Die Konstruktion eines DEA $M'$ aus einem NEA $M$ mit $L(M') = L(M)$
    wie im Beweis vom Satz von Rabin und Scott bezeichnet man als\\
    \begriff{Potenzmengenkonstruktion}.
\end{Def}

\begin{Bem}
    Für $M = (Z, \Sigma, \delta, Z_0, E)$ hat
    $M'$ dann $|\P(Z)| = 2^{|Z|}$ viele Zustände.\\
    Im Allgemeinen geht es nicht viel besser, d.\,h. selbst minimale DEAs
    haben $\O(2^{|Z|})$ viele Zustände (\begriff{Blow-Up}).
    Betrachte dafür die Sprache $L_k = \{x0y \in \{0, 1\}^\ast \;|\;
    |y| = k - 1\}$ für $k \in \natural$ fest.
    Ein NEA lässt sich mit $k + 1$ Zuständen konstruieren.\\
    Nach der Potenzmengenkonstruktion gibt es
    einen DEA mit $2^{k+1}$ Zuständen, allerdings kann es keinen DEA geben,
    der weniger als $2^k$ Zustände besitzt,
    da dieser sich die letzten $k$ Buchstaben "`merken"' muss
    (um zu entscheiden, ob der momentan $k$-letzte Buchstabe eine $0$ ist).\\
    Um dies zu beweisen, zeigt man
    $\widehat{\delta}(z_0, w_1) \not= \widehat{\delta}(z_0, w_2)$ für
    $w_1 \not= w_2$ mit $|w_1| = |w_2| = k$
    (somit muss es mindestens so viele Zustände geben wie
    Wörter der Länge $k$).
    Wegen $w_1 \not= w_2$ gilt $w_1 = x0y_1$ und $w_2 = x1y_2$ für
    bestimmte $x, y_1, y_2 \in \Sigma^\ast$.
    Wäre $\widehat{\delta}(z_0, w_1) = \widehat{\delta}(z_0, w_2)$, dann
    wäre $\widehat{\delta}(z_0, w_1 x) = \widehat{\delta}(z_0, w_2 x)$.
    Der $k$-letzte Buchstabe von $w_1 x$ ist $0$
    (da $|0y_1 x| = |x0y_1| = k$), der von $w_2 x$ ist $1$,
    d.\,h. $\widehat{\delta}(z_0, w_1 x) \in E$ und
    $\widehat{\delta}(z_0, w_2 x) \notin E$, ein Widerspruch.
\end{Bem}

\begin{Satz}{obere Schranke für $\REG$}
    Es gilt $\REG \subset \NEA$\\
    (d.\,h. für jede Typ-3-Grammatik $G$ existiert ein NEA $M$ mit
    $T(M) = L(G)$).
\end{Satz}

\begin{Beweis}
    Sei $G = (V, \Sigma, P, S)$ eine Typ-3-Grammatik.
    Definiere den NEA\\
    $M = (V \cup \{X\}, \Sigma, \delta, \{S\}, E)$ mit $X \notin V$ durch\\
    $E := \{X\}$ für $\varepsilon \notin L(G)$ und
    $E := \{S, X\}$ für $\varepsilon \in L(G)$ sowie\\
    $\delta(A, a) := \{B \in V \;|\; A \rightarrow aB \in P\}$ für
    $A \rightarrow a \notin P$ und\\
    $\delta(A, a) := \{B \in V \;|\; A \rightarrow aB \in P\} \cup \{X\}$ für
    $A \rightarrow a \in P$.

    Man kann sich leicht überlegen, dass $T(M) = L(G)$.
\end{Beweis}

\begin{Bem}
    Damit gilt $\REG = \DEA = \NEA$.
\end{Bem}

\pagebreak

\subsection{%
    Reguläre Ausdrücke%
}

\begin{Def}{reguläre Ausdrücke}
    Sei $\Sigma$ ein Alphabet.\\
    Die Menge $\RegExp$ aller \begriff{regulären Ausdrücke} über $\Sigma$ ist
    wie folgt definiert:
    \begin{itemize}
        \item
        $\emptyset \in \RegExp$

        \item
        $\varepsilon \in \RegExp$

        \item
        $a \in \RegExp$ für alle $a \in \Sigma$
    \end{itemize}
    Diese regulären Ausdrücke heißen \begriff{atomar}.
    Für $\alpha, \beta \in \RegExp$ sei:
    \begin{itemize}
        \item
        $\alpha\beta \in \RegExp$

        \item
        $(\alpha|\beta) \in \RegExp$

        \item
        $(\alpha)^\ast \in \RegExp$
    \end{itemize}
    Klammern dürfern ggf. weggelassen werden.\\
    (Rein formal definiert man
    $\RegExp_0 := \{\emptyset, \varepsilon\} \cup \{a \;|\; a \in \Sigma\}$
    und $\RegExp_{i+1} :=$\\
    $\RegExp_i \cup
    \{\alpha\beta \;|\; \alpha, \beta \in \RegExp_i\} \cup
    \{(\alpha|\beta) \;|\; \alpha, \beta \in \RegExp_i\} \cup
    \{(\alpha)^\ast \;|\; \alpha \in \RegExp_i\}$
    für $i \in \natural_0$ und schließlich
    $\RegExp := \bigcup_{i=0}^\infty \RegExp_i$.)

    $\emptyset$, $\varepsilon$ und $a$ sind zunächst einmal nur Zeichen
    ohne Bedeutung (syntaktische Definition).
\end{Def}

\linie

\begin{Def}{Semantik regulärer Ausdrücke}\\
    Jedem regulären Ausdruck $\alpha \in \RegExp$ über $\Sigma$ ordnet man eine
    Sprache $L(\alpha) \subset \Sigma^\ast$ zu:
    \begin{itemize}
        \item
        $L(\emptyset) := \emptyset$

        \item
        $L(\varepsilon) := \{\varepsilon\}$

        \item
        $L(a) := \{a\}$ für alle $a \in \Sigma$
    \end{itemize}
    Außerdem sei für $\alpha, \beta \in \RegExp$:
    \begin{itemize}
        \item
        $L(\alpha \beta) := L(\alpha) L(\beta) =
        \{xy \;|\; x \in L(\alpha),\; y \in L(\beta)\}$

        \item
        $L((\alpha|\beta)) := L(\alpha) \cup L(\beta)$

        \item
        $L((\alpha)^\ast) := L(\alpha)^\ast =
        \{a_1 \dotsc a_n \;|\; n \in \natural_0,
        a_1, \dotsc, a_n \in L(\alpha)\}$
    \end{itemize}
\end{Def}

\begin{Bem}
    Es gilt $\varepsilon \in L(\alpha)^\ast$ für alle $\alpha \in \RegExp$,
    d.\,h. insbesondere $\varepsilon \in L(\emptyset)^\ast$.\\
    Beispiele für korrekte reguläre Ausdrücke über $\{0, 1\}$ sind
    $0111010$, $11|0^\ast$ und $(11|0)^\ast$ (man beachte die Klammerung).
\end{Bem}

\linie
\pagebreak

\begin{Satz}{Satz von \upshape\,\!\name{Kleene}}\\
    Die Menge der durch reguläre Ausdrücke beschreibbaren Sprachen ist gleich
    $\REG$.
\end{Satz}

\begin{Beweis}
    Sei $\gamma \in \RegExp$.
    Man zeigt induktiv, dass es einen NEA $M$ gibt mit $T(M) = L(\gamma)$.\\
    NEAs für $L(\emptyset) = \emptyset$,
    $L(\varepsilon) = \{\varepsilon\}$ und $L(a) = \{a\}$ sind klar
    (kein Endzustand, Anfangs- gleich Endzustand bzw.
    einfacher Automat mit zwei Zuständen).\\
    Seien also $M_1$ ein NEA für $L(\alpha)$ und $M_2$ ein NEA für $L(\beta)$.
    Konstruiere einen NEA für $L(\alpha)L(\beta)$ durch
    Zusammenschalten der zwei NEAs:
    Für $\varepsilon \notin L(\alpha)$ wird jeder Übergang
    $p \xrightarrow{a} e$ mit $e$ Endzustand in $M_1$ ergänzt durch
    $p \xrightarrow{a} q$ für alle Startzustände $q$ von $M_2$.
    Startzustände des neuen Automaten sind die von $M_1$,
    Endzustände sind die von $M_2$.
    Für $\varepsilon \in L(M_1)$ fügt man einen zusätzlichen (isolierten)
    Zustand ein, der gleichzeitig Start- und Endzustand ist.\\
    Für $L(\alpha) \cup L(\beta)$ "`vereinigt"' man die beiden Automaten
    (Zustände, Startzustände, Endzustände usw.,
    Annahme: Automaten sind disjunkt).\\
    Für $(L(\alpha))^\ast$ verfährt man ähnlich wie für
    $L(\alpha)L(\beta)$, nur dass man hier den Automaten $L_1$ mit sich selbst
    zusammenschaltet.

    Für die andere Richtung geht man von einem DEA
    $M = (Z, \Sigma, \delta, z_1, E)$ mit $Z = \{z_1, \dotsc, z_n\}$ aus
    und konstruiert einen regulären Ausdruck $\gamma \in \RegExp$ mit
    $L(\gamma) = T(M)$.\\
    Definiere $R_{i,j}^k := \{x \in \Sigma^\ast \;|\;
    \widehat{\delta}(z_i, x) = z_j \text{ über Zwischenzustände mit Index}
    \le k\}$.\\
    Man zeigt nun durch Induktion über $k \in \natural_0$,
    dass es für alle $R_{i,j}^k$
    reguläre Ausdrücke gibt, die diese Sprachen beschreiben.
    Klar ist, dass für alle $R_{i,j}^0$ solche regulären Ausdrücke existieren,
    da
    $R_{i,j}^0 = \{a \in \Sigma \;|\; \delta(z_i, a) = z_j\}$ endlich
    und somit durch reguläre Ausdrücke beschreibbar ist.\\
    Wenn für alle $R_{i,j}^k$ die Behauptung gezeigt ist,
    dann gilt sie auch für $R_{i,j}^{k+1}$, denn:\\
    Für $w \in R_{i,j}^{k+1}$ ist $\widehat{\delta}(z_i, w) = z_j$ über
    Zwischenzustände mit Index $\le k + 1$.
    Für den Fall, dass die Zwischenzustände sogar alle Index $\le k$ besitzen,
    lässt sich die Induktionsvoraussetzung direkt anwenden.
    Andernfalls lässt sich $w$ zerlegen zu $w = w_1 x_1 \dotsb x_r w_2$
    mit $w_1 \in R_{i,k+1}^k$, $w_2 \in R_{k+1,j}^k$ und
    $x_i \in R_{k+1,k+1}^k$ für $i = 1, \dotsc, r$.
    Also gilt $R_{i,j}^{k+1} = R_{i,j}^k \cup
    R_{i,k+1}^k (R_{k+1,k+1}^k)^\ast R_{k+1,j}^k$
    und die Induktionsvoraussetzung lässt sich anwenden.\\
    Da $T(M) = \bigcup_{z_j \in E} R_{1,j}^n$ gilt, ist somit auch
    $T(M)$ durch einen regulären Ausdruck $\gamma \in \RegExp$ beschreibbar
    (mittels $(\dotsb|\dotsb)$).
\end{Beweis}

\pagebreak

\subsection{%
    Das Pumping-Lemma%
}

\begin{Satz}{Pumping-Lemma}
    Sei $L \subset \Sigma^\ast$ eine reguläre Sprache.\\
    Dann gilt
    $\exists_{n \in \natural} \forall_{x \in L,\; |x| \ge n}
    \exists_{u, v, w \in \Sigma^\ast,\; uvw = x}\; (1. \land 2. \land 3.)$ mit
    \begin{enumerate}
        \item
        $|v| \ge 1$

        \item
        $|uv| \le n$

        \item
        $\forall_{i \in \natural_0}\; u v^i w \in L$
    \end{enumerate}
\end{Satz}

\begin{Beweis}
    Sei $L$ eine reguläre Sprache.
    Dann gibt es wegen $\REG = \DEA$ einen DEA\\
    $M = (Z, \Sigma, \delta, z_0, E)$ mit $L(M) = L$.
    Setze $n := |Z|$.\\
    Sei $x \in L$ mit $|x| \ge n$, z.\,B. $x = x_1 \dotsb x_m$ mit $m \ge n$.
    Setze $q_j := \widehat{\delta}(z_0, x_1 \dotsb x_j)$ für
    $j = 0, \dotsc, m$.
    Unter den $n + 1$ Zuständen $q_0, \dotsc, q_n$ müssen zwei gleiche sein,
    da $|Z| = n$.
    Wähle $j, k \in \natural_0$, sodass $0 \le j < k \le n$ und $q_j = q_k$.
    Setze $u := x_1 \dotsb x_j$, $v := x_{j+1} \dotsb x_k$ und
    $w := x_{k+1} \dotsb x_m$.

    Es gilt $x = uvw$ und
    \begin{enumerate}
        \item
        $|v| \ge 1$, da $j < k$ und somit
        $x_{j+1} \dotsb x_k \not= \varepsilon$,

        \item
        $|uv| \le n$, da $k \le n$, sowie

        \item
        $\forall_{i \in \natural_0}\; u v^i w \in L$, da aus
        $\widehat{\delta}(z_0, u) = q_j = q_k = \widehat{\delta}(z_0, uv) =
        \widehat{\delta}(\widehat{\delta}(z_0, u), v)$
        mit $p := \widehat{\delta}(z_0, u)$ folgt, dass
        $\widehat{\delta}(p, v) = p$,
        also $\widehat{\delta}(p, v^i) = p$ für alle $i \in \natural_0$.
        Wegen $\widehat{\delta}(p, w) = \widehat{\delta}(z_0, uvw) \in E$
        gilt somit $\widehat{\delta}(z_0, u v^i w) =
        \widehat{\delta}(\widehat{\delta}(p, v^i), w) =
        \widehat{\delta}(p, w) \in E$ für alle $i \in \natural_0$.
    \end{enumerate}
\end{Beweis}

\begin{Bem}
    Das Pumping-Lemma ist keine Charakterisierung von regulären Sprachen,
    d.\,h. es gibt nicht-reguläre Sprachen, die trotzdem die Eigenschaft des
    Pumping-Lemmas erfüllen.\\
    Das Pumping-Lemma kann benutzt werden, um über einen Widerspruch die
    Nicht-Regulärität von Sprachen zu beweisen.
    (Auch dies geht nicht für alle nicht-regulären Sprachen.)
\end{Bem}

\linie

\begin{Bsp}
    $L = \{a^m b^m \;|\; m \ge 1\}$ ist nicht regulär, denn andernfalls
    gäbe es nach dem Pumping-Lemma ein $n \in \natural$, sodass für alle
    Wörter $x \in L$ mit $|x| \ge n$ es Wörter $u, v, w \in \Sigma^\ast$
    mit $uvw = x$ und 1., 2. und 3. geben würde.
    Wählt man $x = a^n b^n \in L$ (es gilt $|a^n b^n| = 2n \ge n$),
    dann gilt $a^n b^n = uvw$ mit $|v| \ge 1$ und $|uv| \le n$.
    $v$ kann also nur aus $a$'s bestehen (mindestens jedoch aus einem $a$).
    Es gilt allerdings $uv^2 w = a^{n + |v|} b^n \notin L$, da
    $n + |v| > n$, somit gilt 3. nicht.
\end{Bsp}

\begin{Bsp}
    $L = \{0^{m^2} \;|\; m \ge 1\}$ ist nicht regulär, denn andernfalls
    gilt Ähnliches wie eben.
    Wählt man $x = 0^{n^2} \in L$ (es gilt $|0^{n^2}| = n^2 \ge n$),
    dann gilt $0^{n^2} = uvw$ mit $u = 0^a$, $v = 0^b$ und $w = 0^c$,
    sodass $b \ge 1$ und $a + b \le n$,
    insbesondere gilt $1 \le b \le n$.
    Es gilt allerdings $uv^2 w = 0^{n^2 + b} \notin L$, da aufgrund
    $n^2 < n^2 + b < n^2 + n + 1 < (n + 1)^2$ die Zahl $n^2 + b$ keine
    Quadratzahl ist.
\end{Bsp}

\begin{Bsp}
    $L = \{0^p \;|\; p \text{ prim}\}$ ist nicht regulär, denn andernfalls
    gilt Ähnliches wie eben.
    Wählt man $x = 0^p \in L$, wobei $p$ eine Primzahl mit $p > n + 2$ ist
    (es gilt $|0^p| = p \ge n$),
    dann gilt $0^p = uvw$ mit $u = 0^a$, $v = 0^b$ und $w = 0^c$,
    sodass $b \ge 1$ und $a + b \le n$,
    insbesondere gilt $1 \le b \le n$.
    Für $i = a + c$ gilt allerdings $uv^i w = 0^{a + b(a + c) + c} \notin L$,
    da $a + b(a + c) + c = (b + 1)(a + c)$ keine Primzahl ist.
\end{Bsp}

\pagebreak

\subsection{%
    Äquivalenzrelation und Minimalautomat%
}

\begin{Def}{Äquivalenzrelation $R_L$}
    Für eine gegebene Sprache $L \subset \Sigma^\ast$ definiert man eine
    \begriff{Relation $R_L$} auf $\Sigma^\ast$ durch
    $x R_L y$ für $x, y \in \Sigma^\ast$, falls
    $\forall_{z \in \Sigma^\ast}\; (xz \in L \iff yz \in L)$.\\
    Diese Relation ist eine Äquivalenzrelation.
\end{Def}

\begin{Bem}
    Die Äquivalenzklassen von $R_L$ teilen nicht die "`Grenze"' zwischen $L$
    und $\Sigma^\ast \setminus L$, d.\,h.
    $\lnot (\exists_{x, y \in \Sigma^\ast,\; [x] = [y]}\; x \in L,\;
    y \notin L)$,
    denn für $x R_L y$ folgt mit $z = \varepsilon$,
    dass $x \in L \iff y \in L$.
\end{Bem}

\begin{Lemma}{Verfeinerung von $R_L$}
    Für jede reguläre Sprache $L = L(M)$ mit dem DEA\\
    $M = (Z, \Sigma, \delta, z_0, E)$ gilt
    $\forall_{x, y \in \Sigma^\ast}\;
    (\widehat{\delta}(z_0, x) = \widehat{\delta}(z_0, y)
    \;\Rightarrow\; x R_L y)$.
\end{Lemma}

\begin{Beweis}
    Seien $x, y \in \Sigma^\ast$ mit
    $\widehat{\delta}(z_0, x) = \widehat{\delta}(z_0, y)$ und
    $z \in \Sigma^\ast$ beliebig.\\
    Dann gilt $xz \in L
    \iff \widehat{\delta}(z_0, xz) \in E
    \iff \widehat{\delta}(\widehat{\delta}(z_0, x), z) \in E
    \iff \widehat{\delta}(\widehat{\delta}(z_0, y), z) \in E$\\
    $\iff \widehat{\delta}(z_0, yz) \in E
    \iff yz \in L$.
\end{Beweis}

\begin{Def}{Äquivalenzrelation $R_M$}
    Für einen DEA $M = (Z, \Sigma, \delta, z_0, E)$ definiert man eine
    \begriff{Relation $R_M$} auf $\Sigma^\ast$ durch
    $x R_M y$ für $x, y \in \Sigma^\ast$, falls
    $\widehat{\delta}(z_0, x) = \widehat{\delta}(z_0, y)$.\\
    Diese Relation ist eine Äquivalenzrelation und es gilt $R_M \subset R_L$,
    d.\,h. $R_M$ ist eine Verfeinerung von $R_L$.
    (die Äquivalenzklassen von $R_L$ werden durch $R_M$ "`verfeinert"').
\end{Def}

\linie

\begin{Def}{Index}
    Seien $M$ eine Menge und $R \subset M \times M$ eine Äquivalenzrelation.
    Dann heißt die Anzahl $|\{[m]_R \;|\; m \in M\}|$ der
    Äquivalenzklassen \begriff{Index} der Äquivalenzrelation $R$.
\end{Def}

\begin{Satz}{Satz von \name{Myhill} und \name{Nerode}}\\
    Eine Sprache $L$ ist regulär genau dann, wenn
    die zugehörige Relation $R_L$ endlichen Index hat.
\end{Satz}

\begin{Beweis}
    "`$\Rightarrow$"':
    Sei $L = L(M)$ mit dem DEA $M = (Z, \Sigma, \delta, z_0, E)$.
    Dann gilt nach obigem Lemma $R_M \subset R_L$, also ist
    der Index von $R_L$ kleiner oder gleich dem Index von $R_M$.
    Dieser ist allerdings maximal $|Z|$ (aufgrund der Definition von $R_M$)
    und damit endlich.

    "`$\Leftarrow$"':
    Sei $L \subset \Sigma^\ast$ eine Sprache, sodass $R_L$ endlichen Index $k$
    hat.
    Man wählt $k$ Repräsentanten $x_1, \dotsc, x_k \in \Sigma^\ast$ der
    Äquivalenzklassen
    (d.\,h. es gilt $\Sigma^\ast = [x_1] \cup \dotsb \cup [x_k]$)
    und setzt oBdA $\varepsilon \in [x_1]$.
    Nun konstruiert man einen DEA $M = (Z, \Sigma, \delta, z_0, E)$ mit
    $T(M) = L$ wie folgt:\\
    $Z := \{[x_1], \dotsc, [x_k]\}$, $z_0 := [x_1] = [\varepsilon]$,
    $E := \{[x_i] \;|\; x_i \in L\}$ und
    $\delta([x_i], a) := [x_i a]$.\\
    $E$ ist wohldefiniert, da $[x] = [y]$ impliziert,
    dass $x \in L \iff y \in L$ (siehe oben).\\
    $\delta$ ist wohldefiniert, denn aus $[x] = [y]$ folgt
    $[xa] = [ya]$ für alle $a \in \Sigma$
    (für $z \in \Sigma^\ast$ beliebig ist
    $xaz \in L \iff yaz \in L$ aufgrund
    $xz' \in L \iff yz' \in L$ für alle $z' \in \Sigma^\ast$,
    also auch für $z' = az$).\\
    Es gilt $x \in T(M)
    \iff \widehat{\delta}(z_0, x) =
    \widehat{\delta}([\varepsilon], x) =
    [x] \in E \iff x \in L$, also ist $L$ regulär.
\end{Beweis}

\linie

\begin{Bsp}
    Sei $L = \{0^{m^2} \;|\; m \ge 1\}$.
    $R_L$ muss unendlich viele Äquivalenzklassen besitzen, denn $L$ ist nicht
    regulär (siehe oben).
    Dies kann man auch direkt nachweisen:
    Für $m < n$ gilt $[0^{m^2}] \not= [0^{n^2}]$, denn
    wählt man $z = 0^{2m + 1}$, so gilt
    $0^{m^2} z = 0^{(m+1)^2} \in L$, aber $0^{n^2} z \notin L$.
\end{Bsp}

\begin{Bsp}
    Betrachtet man $L = \{x \in \{a, b\}^\ast \;|\; x \text{ enthält } abb\}$,
    so sind die paarweise disjunkten Äquivalenzklassen
    $\Sigma^\ast = [abb] \cup [\varepsilon] \cup [a] \cup [ab]$, denn:
    $[abb] = L$ und
    $\lnot (\varepsilon R_L a)$ (mit $z = ab$),
    $\lnot (\varepsilon R_L ab)$ und
    $\lnot (a R_L ab)$ (jeweils mit $z = b$).
    Wegen $\varepsilon, a, ab \notin L = [abb]$ sind die Äquivalenzklassen
    disjunkt.
    Es gibt keine weiteren, da für jedes Wort
    $x \in \Sigma^\ast \setminus L$ gilt,
    dass $x \in [ab]$, falls $x$ mit $ab$ endet,
    dass $x \in [a]$, falls $x$ mit $a$ endet, und
    dass $x \in [\varepsilon]$, falls $x$ mit $b$ endet, aber nicht mit $ab$
    (in diesem Fall kann $x$ nur aus $b$'s bestehen oder leer sein).
    Somit ist $L$ regulär.
\end{Bsp}

\begin{Bsp}
    Für $L = \{x \in \{a, b, c\}^\ast \;|\; |x|_a - |x|_b \equiv 3 \mod 5\}$
    sind die disjunkten Äquivalenzklassen
    $\Sigma^\ast = [aaa] \cup [\varepsilon] \cup [a] \cup [aa] \cup [aaaa]$,
    d.\,h. auch diese Sprache ist regulär
    (auch siehe oben).
\end{Bsp}

\linie
\pagebreak

\begin{Def}{minimaler Automat}
    Sei $L \subset \Sigma^\ast$ eine reguläre Sprache.\\
    Ein DEA bzw. NEA $M$ heißt \begriff{minimal}, falls
    $T(M) = L$ und es keinen DEA bzw. NEA gibt, der dieselbe Sprache erkennt
    und weniger Zustände besitzt.
\end{Def}

\begin{Satz}{Minimalität des Äquivalenzklassen-DEA}
    Der im Beweis des Satzes von Myhill und Nerode konstruierte
    Äquivalenzklassenautomat ist ein
    minimaler DEA für jede reguläre Sprache.\\
    Der Minimalautomat ist bis auf Isomorphie (Umbenennen der Zustände)
    eindeutig bestimmt.
\end{Satz}

\begin{Beweis}
    Sei $M_0$ der Äquivalenzklassen-DEA und $M$ ein weiterer DEA mit
    $T(M) = L$.\\
    Dann gilt $R_M \subset R_L = R_{M_0}$
    ($R_{M_0} \subset R_L$ klar,
    $R_L \subset R_{M_0}$ gilt, da aus $x R_L y$ folgt, dass\\
    $\widehat{\delta}(z_0, x) =
    [x] = [y] = \widehat{\delta}(z_0, y)$).
    Also ist $R_M$ eine Verfeinerung von $R_{M_0}$, die Zahl der Zustände
    von $M$ kann also nicht kleiner als die von $M_0$ sein
    (Anzahl der Zustände von $M_0$
    $=$ Anzahl der vom Startzustand erreichbaren Zustände von $M_0$
    $=$ Index von $R_{M_0}$ $\le$ Index von $R_M$
    $=$ Anzahl der vom Startzustand erreichbaren Zustände von $M$
    $\le$ Anzahl der Zustände von $M$).\\
    Falls $M$ die minimale Zustandszahl besitzt, gilt $R_M = R_L$.
\end{Beweis}

\begin{Bem}\\
    Der minimale NEA für eine gegebene reguläre Sprache ist \emph{nicht}
    eindeutig bestimmt.
\end{Bem}

\linie

\begin{Def}{Algorithmus zur Bestimmung des Minimalautomaten}
    Der \begriff{Algorithmus zur Bestim\-mung des minimalen DEA} bekommt
    als Eingabe einen DEA, in dem alle Zustände erreichbar sind, und
    gibt Teilmengen von der Zustandsmenge $Z$ aus, die verschmolzen werden
    können.\\
    Dazu legt sich der Algorithmus eine Matrix $Z \times Z$ an und
    verfährt folgendermaßen:
    \begin{enumerate}
        \item
        Markiere alle Paare $(z, z')$ mit $z \in E \land z' \notin E$ oder
        $z \notin E \land z' \in E$.

        \item
        Markiere jedes Zustandspaar $(p, q)$ mit $\delta(p, a) = z$,
        $\delta(q, a) = z'$ und $(z, z')$ bereits markiert für ein
        $a \in \Sigma$.

        \item
        Wiederhole 2., bis sich nichts mehr ändert.

        \item
        Die nun unmarkierten Paare von Zuständen können jeweils zu einem
        Zustand verschmolzen werden.
    \end{enumerate}
\end{Def}

\begin{Bem}
    Man kann sich den Algorithmus herleiten, indem man sich überlegt, dass
    ein Automat dann nicht minimal ist, wenn es zwei verschiedene Zustände
    $z, z'$ gibt mit $\widehat{\delta}(z, x) \in E \iff
    \widehat{\delta}(z', x) \in E$ für alle $x \in \Sigma^\ast$
    (es reicht dabei, nur Wörter mit $|x| \le |Z|$ zu betrachten).
\end{Bem}

\pagebreak

\subsection{%
    \emph{Einschub}: Erkennung durch Monoide%
}

\begin{Def}{Monoid}
    Das Paar $(M, \ast)$ heißt \begriff{Monoid}, falls
    $M$ eine Menge und
    $\ast\colon M \times M \rightarrow M$ eine Abbildung ist mit
    $\forall_{a, b, c \in M}\;
    a \ast (b \ast c) = (a \ast b) \ast c$ und
    $\exists_{e \in M} \forall_{a \in M}\; e \ast a = a = a \ast e$.
\end{Def}

\begin{Def}{Monoidhomomorphismus}
    Seien $(M_1, \ast_1)$ und $(M_2, \ast_2)$ Monoide.\\
    Eine Abbildung $\varphi\colon M_1 \rightarrow M_2$ heißt
    \begriff{Monoidhomomorphismus}, falls
    $\varphi(m \ast_1 n) = \varphi(m) \ast_2 \varphi(n)$ für alle
    $m, n \in M_1$ und $\varphi(e_1) = e_2$.
\end{Def}

\begin{Def}{Erkennung durch Monoide}
    Seien $L \subset \Sigma^\ast$ eine Sprache und $M$ ein Monoid.\\
    $M$ \begriff{erkennt} $L$, falls es eine Teilmenge $A \subset M$ und
    einen Homomorphismus $\varphi\colon \Sigma^\ast \rightarrow M$
    gibt mit $L = \varphi^{-1}(A)$
    (d.\,h. $w \in L \iff \varphi(w) \in A$).
\end{Def}

\begin{Bem}
    Alternativ kann man definieren, dass
    ein Homomorphismus $\varphi\colon \Sigma^\ast \rightarrow M$ existieren
    soll mit $L = \varphi^{-1}(\varphi(L))$
    (hier ist $A = \varphi(L)$).
\end{Bem}

\begin{Def}{erkennbar}\\
    Eine Sprache heißt \begriff{erkennbar}, falls sie von einem endlichen
    Monoid erkannt wird.
\end{Def}

\linie

\begin{Def}{syntaktische Kongruenz}
    Sei $L \subset \Sigma^\ast$ eine Sprache.
    Zwei Wörter $w_1, w_2 \in \Sigma^\ast$ heißen \begriff{äquivalent},
    falls $\forall_{x, y \in \Sigma^\ast}\;
    x w_1 y \in L \iff x w_2 y \in L$.
    Man schreibt dafür auch $w_1 \equiv_L w_2$ oder $w_1 \equiv w_2$.
    $\equiv_L$ ist eine Äquivalenzrelation und sogar eine \begriff{Kongruenz},
    d.\,h.\\
    $w_1 \equiv_L w_2 \iff
    \forall_{x, y \in \Sigma^\ast}\; x w_1 y \equiv_L x w_2 y$.
    Man nennt $\equiv_L$ daher auch \begriff{syntaktische Kongruenz}.
\end{Def}

\begin{Bem}
    $\equiv_L$ ist eine Verfeinerung von $R_L$, d.\,h.
    $w_1 \equiv_L w_2 \;\Rightarrow\; w_1 R_L w_2$.
\end{Bem}

\begin{Def}{syntaktisches Monoid}
    Das Quotientenmonoid $\Synt(L) := \Sigma^\ast/\!\!\equiv_L \;=
    \{[w]_{\equiv_L} \;|\; w \in \Sigma^\ast\}$ heißt
    \begriff{syntaktisches Monoid} von $L$.
\end{Def}

\begin{Bem}
    Um zu zeigen, dass dies auch tatsächlich wieder ein Monoid ist,
    muss man zunächst die Wohldefiniertheit der Monoidoperation zeigen.
    Für $[a] = [a']$ und $[b] = [b']$ ist
    $a \equiv_L a'$ und $b \equiv_L b'$, d.\,h.
    für $x, y \in \Sigma^\ast$ beliebig gilt
    $x ab y \in L \iff x a'b y \in L \iff x a'b' y \in L$,
    also $[ab] = [a'b']$.
    Die Assoziativität gilt wegen der Assoziativität in $\Sigma^\ast$,
    außerdem ist $[\varepsilon]$ neutral.
    Damit ist $\Synt(L)$ ein Monoid.
\end{Bem}

\begin{Bem}
    $\Synt(L)$ erkennt $L$, denn
    wähle als Homomorphismus die Quotientenabbildung
    $\varphi\colon \Sigma^\ast \rightarrow \Synt(L)$,
    $\varphi(a) = [a]$ und als Menge $A = \{[a] \;|\; a \in L\}$.\\
    Dann gilt $L = \varphi^{-1}(A)$, denn
    $a \in L \iff \varphi(a) = [a] \in A$
    ("`$\Leftarrow$"':
    $[a] = [b]$ für ein $b \in L$,
    also $a \equiv_L b$, daraus folgt wegen $b \in L$, dass auch $a \in L$
    gilt).
\end{Bem}

\linie

\begin{Satz}{Zusammenhang des syntaktischen Monoids mit regulären Sprachen}\\
    Sei $L \subset \Sigma^\ast$ eine Sprache.
    Dann sind folgende Aussagen äquivalent:
    \begin{enumerate}
        \item
        $L$ ist regulär.

        \item
        $L$ ist erkennbar.

        \item
        $\Synt(L)$ ist endlich.
    \end{enumerate}
\end{Satz}

\begin{Beweis}
    3. $\Rightarrow$ 2. klar, da $\Synt(L)$ die Sprache $L$ erkennt.\\
    3. $\Rightarrow$ 1. gilt, weil
    $\equiv_L$ einen endlichen Index hat, wenn $\Synt(L)$ endlich ist.
    Da aber $\equiv_L$ eine Verfeinerung von $R_L$ ist, ist der Index von
    $R_L$ höchstens so groß wie der von $\equiv_L$, d.\,h.
    $R_L$ hat endlichen Index und somit ist $L$ regulär.
\end{Beweis}

\pagebreak

\subsection{%
    Abschlusseigenschaften%
}

\begin{Satz}{Abschluss von $\REG$}\\
    Die Klasse $\REG$ der regulären Sprachen ist abgeschlossen unter
    Vereinigung, Schnitt und Komplement
    (\begriff{\name{boole}sche Operationen}) sowie unter
    Produkt (Konkatenation) und Stern.
\end{Satz}

\begin{Beweis}
    Abschluss unter Vereinigung:
    Sind $L_1$ und $L_2$ regulär,
    dann gibt es reguläre Ausdrücke $\alpha_1$ und $\alpha_2$ mit
    $L(\alpha_1) = L_1$ und $L(\alpha_2) = L_2$.
    Es gilt $L(\alpha_1 | \alpha_2) = L_1 \cup L_2$, d.\,h.
    $L_1 \cup L_2$ ist regulär.

    Abschluss unter Komplement:
    Ist $L$ regulär, so gibt es ein endliches Monoid $M$, einen
    Homomorphismus $\varphi\colon \Sigma^\ast \rightarrow M$
    und eine Teilmenge $A \subset M$ mit $L = \varphi^{-1}(A)$.
    Dann gilt aber auch
    $\Sigma^\ast \setminus L = \varphi^{-1}(M \setminus A)$,
    d.\,h. $\Sigma^\ast \setminus L$ wird von demselben endlichen Monoid
    erkannt und ist somit regulär.
    (Alternativ kann man in einem DEA $M$ mit $T(M) = L$ Endzustände und
    Nicht-Endzustände vertauschen, um einen DEA $M'$ mit
    $T(M') = \Sigma^\ast \setminus L$ zu erhalten.)

    Somit folgt der Abschluss unter booleschen Operationen,
    denn alle booleschen Operationen (auch der Durchschnitt) sind mit
    Vereinigung und Komplement darstellbar.
    (Alternativ kann man zu zwei Automaten
    $M_1 = (Z_1, \Sigma, \delta_1, z_{01}, E_1)$ und
    $M_2 = (Z_2, \Sigma, \delta_2, z_{02}, E_2)$ mit
    $T(M_1) = L_1$ und $T(M_2) = L_2$ den \begriff{Kreuzproduktautomaten}
    $M := (Z_1 \times Z_2, \Sigma, \delta, (z_{01}, z_{02}), E_1 \times E_2)$
    mit $\delta((z, z'), a) := (\delta_1(z, a), \delta_2(z', a))$
    betrachten, für den $T(M) = L_1 \cap L_2$ gilt.)

    Abschluss unter Produkt:
    Sind $L_1$ und $L_2$ regulär,
    dann gibt es reguläre Ausdrücke $\alpha_1$ und $\alpha_2$ mit
    $L(\alpha_1) = L_1$ und $L(\alpha_2) = L_2$.
    Es gilt $L(\alpha_1 \alpha_2) = L_1 L_2$, d.\,h.
    $L_1 L_2$ ist regulär.

    Abschluss unter Stern:
    Ist $L$ regulär, dann gibt es einen regulären Ausdruck $\alpha$ mit
    $L(\alpha) = L$.
    Es gilt $L(\alpha^\ast) = L^\ast$, d.\,h. $L^\ast$ ist regulär.
\end{Beweis}

\pagebreak

\subsection{%
    Entscheidbarkeit%
}

\begin{Bem}
    In diesem Abschnitt wird untersucht, welche Probleme in Bezug auf
    reguläre Sprachen entscheidbar sind.
\end{Bem}

\begin{Bem}
    Das \begriff{Wortproblem} besteht darin, bei gegebener Sprache $L$
    und einem Wort $x$ zu entscheiden, ob $x \in L$ gilt.
    Das Wortproblem ist für reguläre Sprachen entscheidbar
    (sogar schon für Typ-1-Sprachen).\\
    Ist ein DEA $M$ mit $T(M) = L$ gegeben, dann ist die Entscheidung in
    Linearzeit möglich:
    Zeichen für Zeichen kann man die Zustandsübergänge im Automaten verfolgen,
    die durch die Eingabe eines Wortes $x \in \Sigma^\ast$ hervorgerufen
    werden.
    Falls ein Endzustand erreicht wird, ist $x \in L$.
    Man spricht von \begriff{Echtzeit}, da man vorhersehen kann, wie lange
    die Lösung des Wortproblems mit einem DEA dauern wird.\\
    Dies geht nicht so ef"|fizient, wenn $L$ durch einen NEA gegeben ist
    (mehrere Möglichkeiten).
\end{Bem}

\linie

\begin{Bem}
    Das \begriff{Leerheitsproblem} besteht darin, bei gegebener Sprache $L$
    zu entscheiden, ob $L = \emptyset$ gilt.
    Das Leerheitsproblem ist für reguläre Sprachen entscheidbar.\\
    In einem DEA kann man z.\,B. prüfen, ob es einen Weg vom Startzustand zu
    einem Endzustand gibt.
    Dies gilt genau dann, wenn $L \not= \emptyset$.\\
    Alternativ kann man (bei algorithmisch nicht akzeptablem Zeitaufwand)
    das Pumping-Lemma anwenden.
    Es gilt $L \not= \emptyset \iff \exists_{w \in L}\; |w| < n$, wobei
    $n$ das $n$ aus dem Pumping-Lemma ist.
    Man prüft also alle Wörter der Länge $< n$ auf Mitgliedschaft
    in $L$ (Wortproblem).
\end{Bem}

\linie

\begin{Bem}
    Das \begriff{Endlichkeitsproblem} besteht darin, bei gegebener Sprache $L$
    zu entscheiden, ob $|L| < \infty$ gilt.
    Das Endlichkeitsproblem ist für reguläre Sprachen entscheidbar.\\
    In einem DEA kann man z.\,B. prüfen, ob es einen Zyklus gibt, der
    vom Startzustand erreichbar ist und von dem aus ein Endzustand erreichbar
    ist.
    Dies gilt genau dann, wenn $|L| = \infty$.\\
    Alternativ kann man (bei algorithmisch nicht akzeptablem Zeitaufwand)
    das Pumping-Lemma anwenden.
    Es gilt $|L| = \infty \iff \exists_{w \in L}\; n \le |w| < 2n$, wobei
    $n$ das $n$ aus dem Pumping-Lemma ist.
    Man prüft also alle Wörter der Länge $\ge n$ und $< 2n$ auf Mitgliedschaft
    in $L$ (Wortproblem).
\end{Bem}

\begin{Beweis}
    "`$\Leftarrow$"':
    Sei $x \in L$ mit $n \le |x| < 2n$.
    Dann gilt aufgrund des Pumping-Lemmas $x = uvw$, d.\,h.
    $uv^i w \in L$ für alle $i \in \natural_0$.
    Somit ist $L$ unendlich.

    "`$\Rightarrow$"':
    Sei $|L| = \infty$ und entgegen der Behauptung habe das kürzeste Wort
    $x \in L$ mit $|x| \ge n$ eine Länge $\ge 2n$.
    Aufgrund des Pumping-Lemmas gilt $x = uvw$ mit $uv^0 w = uw \in L$.
    Wegen $|v| \le |uv| \le n$ gilt $|uw| \ge n$.
    Damit ist aber $x$ nicht minimal gewesen, ein Widerspruch.
\end{Beweis}

\linie

\begin{Bem}
    Das \begriff{Äquivalenzproblem} besteht darin, bei gegebenen Sprachen $L_1$
    und $L_2$ zu entscheiden, ob $L_1 = L_2$ gilt.
    Das Äquivalenzproblem ist für reguläre Sprachen entscheidbar.\\
    Die Klasse $\REG$ der regulären Sprachen ist ef"|fektiv abgeschlossen
    unter booleschen Operationen.
    Man kann also $L_1 \vartriangle L_2$ bilden und auf Leerheit prüfen
    (Lösung des Leerheitsproblems).\\
    Alternativ kann man die Minimalautomaten bilden und vergleichen.
\end{Bem}

\linie

\begin{Bem}
    Das \begriff{Schnittproblem} besteht darin, bei gegebenen Sprachen $L_1$
    und $L_2$ zu entscheiden, ob $L_1 \cap L_2 = \emptyset$ gilt.
    Das Schnittproblem ist für reguläre Sprachen entscheidbar.\\
    Die Klasse $\REG$ der regulären Sprachen ist ef"|fektiv abgeschlossen
    unter booleschen Operationen
    (d.\,h. die Ergebnisse dieser Operationen können algorithmisch in
    endlicher Zeit bestimmt werden).
    Man kann also $L_1 \cap L_2$ bilden und auf Leerheit prüfen
    (Lösung des Leerheitsproblems).
\end{Bem}

\pagebreak
