\chapter{%
    Funktionenfolgen, Funktionenreihen, parameterabhängige Integrale%
}

\section{%
    Das Prinzip der Gleichmäßigkeit%
}

Eine Eigenschaft $A(p)$ gilt gleichmäßig bzgl. $p \in P$
(\emph{Parameter} aus \emph{Parametermenge}), falls
\begin{enumerate}
    \item[(1)] $A(p)$ ist für alle $p \in P$ wahr
    \item[(2)] die Konstanten in $A(p)$ sind von $p$ unabhängig wählbar.
\end{enumerate}

\linie

\textbf{gleichmäßige Konvergenz einer Folge}:
Sei $a: \natural \times P \rightarrow M$ eine parameterabhängige Folge
$\{a_k(p)\}_{k \in \natural}$ ($P$ Parametermenge),
wobei $(M,d)$ ein metrischer Raum ist.

\textbf{punktweise Konvergenz}: $a_n \xrightarrow{(\cdot)} a(p)
\quad\Leftrightarrow\quad \forall_{p \in P} \forall_{\varepsilon > 0}
\exists_{N(\varepsilon, p)} \forall_{n \ge N(\varepsilon, p)}\;
d(a_n(p), a(p)) < \varepsilon$

\textbf{gleichmäßige Konvergenz}: $a_n \rightrightarrows a(p)
\quad\Leftrightarrow\quad \forall_{\varepsilon > 0} \exists_{N(\varepsilon)}
\forall_{p \in P} \forall_{n \ge N(\varepsilon)}\;
d(a_n(p), a(p)) < \varepsilon$

Zum Beispiel gilt bei der Folge $a_n(p) = \frac{n/p}{1 + (n / p)^2}$
($n, p \in \natural$) für $n = p$, dass $a_n(p) = \frac{1}{2}$,
jedoch gilt für jedes feste $p$, dass $\lim_{n \to \infty} a_n(p) = 0$,
aber nicht gleichmäßig.

Dagegen ist die Folge $a_n(p) = \frac{1}{1 + n + p} < \frac{1}{1 + n}$
für $n, p \in \natural$ gleichmäßig konvergent bzgl. $p \in P = \natural$,
da für $n \ge \frac{1}{\varepsilon}$ gilt, dass
$|a_n(p) - 0| < \varepsilon$ für alle $p \in \natural$.

\linie

\textbf{gleichmäßige Konvergenz einer Reihe}:
Sei $a: \natural \times P \rightarrow \field^n$ eine Folge,
wobei $\field \in \{\real, \complex\}$. \\
$\sum_{k=1}^\infty a_n(p)$ konvergiert gleichmäßig bzgl. $p \in P$,
falls $S_m(p) = \sum_{n=1}^m a_n(p) \rightrightarrows S(p)$ gleichmäßig.

\textbf{gleichmäßige Konvergenz uneigentlicher Integrale}: \\
Sei $f: \left[0, +\infty\right[ \times P \rightarrow \field^d$ mit
$\forall_{p \in P} \forall_{R > 0}\; f(x, p) \in \R[0,R]_x$.
Dann konvergiert $\int_0^{+\infty} f(x, p)\dx$ gleichmäßig bzgl. $p \in P$,
falls $\forall_{\varepsilon > 0} \exists_{R(\varepsilon)}
\forall_{R', R'' \ge R(\varepsilon)} \forall_{p \in P}\;
\big|\int_{R'}^{R''} f(x, p)\dx\big| < \varepsilon$.

\linie

\textbf{gleichmäßig stetige Funktionen}: \\
Seien $(M_1, d_1)$, $(M_2, d_2)$ metrische Räume sowie
$f: X \subset M_1 \rightarrow M_2$.

\textbf{punktweise Stetigkeit}:
$\forall_{x_0 \in X} \forall_{\varepsilon > 0}
\exists_{\delta(x_0, \varepsilon) = \delta > 0}
\forall_{x \in U_\delta(x_0) \cap X}\; f(x) \in U_\varepsilon(f(x_0))$

\textbf{gleichmäßige Stetigkeit}:
$\forall_{\varepsilon > 0}
\exists_{\delta(\varepsilon) = \delta > 0} \forall_{x_0 \in X}
\forall_{x \in U_\delta(x_0) \cap X}\; f(x) \in U_\varepsilon(f(x_0))$

Gleichmäßige Stetigkeit/Konvergenz impliziert punktweise
Stetigkeit/Konvergenz. \\
\emph{Die Umkehrung gilt nicht!}

\linie

\textbf{Lemma}: Sei $a: \natural \times P \rightarrow M$.
Dann ist $a_n(p) \rightrightarrows a(p) \;\Leftrightarrow\;
\lim_{n \to \infty} \left(\sup_{p \in P} d(a_n(p), a(p))\right) = 0$.

\section{%
    Satz zum Vertauschen von Grenzwerten%
}

Man betrachtet Doppelfolgen mit $P = \natural$, d.\,h. $a_n(p) = a_{n,p}$.
Angenommen, es gibt Grenzwerte $\lim_{n \to \infty} a_{n,p} = u(p)$
und $\lim_{p \to \infty} a_{n,p} = v(n)$.
Im Allgemeinen gilt dann \textbf{nicht} \\
$\lim_{p \to \infty} (\lim_{n \to \infty} a_{n,p}) =
\lim_{p \to \infty} u(p) = \lim_{n \to \infty} v(n) =
\lim_{n \to \infty} (\lim_{p \to \infty} a_{n,p})$.

\textbf{Satz}:
Sei $a: \natural \times \natural \rightarrow M$ eine Doppelfolge
mit $(M,d)$ vollständig.
Außerdem existiere für alle $p \in \natural$ der Grenzwert
$\lim_{n \to \infty} a_{n,p} = u(p)$ sowie
für alle $n \in \natural$ existiere der Grenzwert
$\lim_{p \to \infty} a_{n,p} = v(n)$.
Einer dieser Grenzwerte sei gleichmäßig angenommen. \\
Dann existieren die Grenzwerte
$\lim_{p \to \infty} u(p) = \lim_{n \to \infty} v(n)$
und sind gleich.

Unter diesen Voraussetzungen gilt somit
$\lim_{p \to \infty} (\lim_{n \to \infty} a_{n,p}) =
\lim_{n \to \infty} (\lim_{p \to \infty} a_{n,p})$.

\section{%
    Zur Stetigkeit der Grenzfunktion und
    zum Vertauschen von Grenzwerten vom Typ
    \texorpdfstring
    {$\lim_{n \to \infty}$ und $\lim_{x \to \xi}$}%
    {n → ∞ und ξ → ∞}%
}

Seien $M_1, M_2$ metrische Räume, $M_2$ vollständig, $X \subset M_1$,
$\xi \in \acc(X)$ sowie $f: \natural \times X \rightarrow M_2$
eine Folge von Funktionen $f_n(x)$ mit $n \in \natural$, $x \in X$.

\textbf{Vertauschen von $\lim_{n \to \infty}$ und $\lim_{x \to \xi}$}: \\
Für alle $n \in \natural$ existiere der Grenzwert
$\lim_{x \to \xi} f_n(x) = a_n$
sowie der Grenzwert \\
$\lim_{n \to \infty} f_n(x) = \varphi(x)$ existiere
gleichmäßig bzgl. $x \in X$. \\
Dann existieren die Grenzwerte
$\lim_{n \to \infty} a_n = \lim_{x \to \xi} \varphi(x)$ und sind gleich.

Unter diesen Voraussetzungen gilt somit
$\lim_{n \to \infty} \left(\lim_{x \to \xi} f_n(x)\right) =
\lim_{x \to \xi} \left(\lim_{n \to \infty} f_n(x)\right)$.

\linie

\textbf{Anwendung ($\varphi$ stetig in $\xi$)}: \\
Gilt zudem $\xi \in X$ und $f_n$ ist stetig in $\xi$ für alle
$n \in \natural$, dann ist auch $\varphi$ stetig in $\xi$.

\textbf{Anmerkung}: Ist $f_n \in \C([a,b], \field^d)$ für alle
$n \in \natural$ und $\lim_{n \to \infty} f_n(x) = \varphi(x)$ existiert
gleichmäßig bzgl. $x \in [a,b]$, so ist auch
$\varphi \in \C([a,b], \field^d)$ und
$f_n \xrightarrow{\norm{\cdot}_\C} \varphi$.

\emph{Die Voraussetzung "`gleichmäßig"' ist wesentlich!}

\linie

\textbf{Banachraum}: Ein Banachraum ist ein vollständiger, normierter
Vektorraum (d.\,h. vollständig bzgl. der von der Norm induzierten Metrik).

Seien nun $Y$ ein Banachraum, $M$ ein metrischer Raum, $X \subset M$,
$\xi \in \acc(X)$ und $a: \natural \times X \rightarrow Y$ eine
Funktionenfolge $a_n(x)$.

\textbf{Vertauschen von $\sum_{n=1}^\infty$ und $\lim_{x \to \xi}$}: \\
Sei $\lim_{x \to \xi} a_n(x) = b_n$ konvergent für alle $n \in \natural$ und
$\sum_{n=1}^\infty a_n(x) = S(x)$ konvergiere gleichmäßig bzgl. $x \in X$. \\
Dann existieren die Grenzwerte
$\sum_{n=1}^\infty b_n = \lim_{x \to \xi} S(x)$ und sind gleich.

Unter diesen Voraussetzungen gilt somit
$\sum_{n=1}^\infty \left(\lim_{x \to \xi} a_n(x)\right) =
\lim_{x \to \xi} \left(\sum_{n=1}^\infty a_n(x)\right)$.

\textbf{Folgerung}: Ist zusätzlich $\xi \in X$ und sind alle $a_n(x)$
stetig in $\xi$, dann ist auch \\
$S(x) = \sum_{n=1}^\infty a_n(x)$ in $\xi$ stetig.

\linie

Wie zeigt man, dass $\sum_{n=1}^\infty a_n(x)$ gleichmäßig bzgl. $x \in X$
konvergiert?

\textbf{Majorantenkriterium von \textsc{Weierstraß}}:
Für alle $n \in \natural$ und $x \in X$ sei $\norm{a_n(x)} \le C_n$
(d.\,h. $C_n$ ist von $x$ unabhängig).
Zudem sei $\sum_{n=1}^\infty C_n$ konvergent. \\
Dann konvergiert $\sum_{n=1}^\infty a_n(x)$ gleichmäßig bzgl. $x \in X$.

\textbf{Anwendung (\textsc{Fourier}-Reihen)}:
Seien $a_n \in \complex$ eine komplexe Folge und $\sum_{n=1}^\infty |a_k|$
sei konvergent.
Dann konvergieren folgende Funktionenreihen ebenfalls (absolut) für alle
$x \in \real$ und sind stetig:
$S(x) = \sum_{n=1}^\infty a_n \sin(nx)$,
$C(x) = a_0 \sum_{n=1}^\infty a_n \cos(nx)$,
$E(x) = \sum_{n \in \integer} a_n e^{inx}$.

\pagebreak

\section{%
    Zur Stetigkeit der Grenzfunktion zweier Variablen%
}

\textbf{Vertauschen von $\lim_{x \to x_0}$ und $\lim_{y \to y_0}$}:
Seien $M_1, M_2, M_3$ metrische Räume mit $M_3$ vollständig,
$X \subset M_1$, $Y \subset M_2$, $x_0 \in \acc(X)$ und $y_0 \in \acc(Y)$. \\
Weiterhin sei $f: X \times Y \rightarrow M_3$ eine Funktion, wobei
für alle $x \in X$ der Grenzwert \\
$\lim_{y \to y_0} f(x, y) = \varphi(x)$
und für alle $y \in Y$ der Grenzwert $\lim_{x \to x_0} f(x, y) = \psi(y)$
existiert. \\
Einer dieser beiden Grenzwerte werde gleichmäßig angenommen. \\
Dann gibt es die Grenzwerte
$\lim_{x \to x_0} \varphi(x) = \lim_{y \to y_0} \psi(y)$ und sind gleich.

Unter diesen Voraussetzungen gilt somit
$\lim_{x \to x_0} \left(\lim_{y \to y_0} f(x, y)\right) =
\lim_{y \to y_0} \left(\lim_{x \to x_0} f(x, y)\right)$.

\textbf{Folgerung}:
Seien zusätzlich $x_0 \in X$, $f(x, y)$ stetig im Punkt
$x = x_0$ (für jedes beliebige $y$) und der erste Grenzwert werde gleichmäßig
erreicht. \\
Dann ist auch $\varphi(x)$ stetig im Punkt $x = x_0$.

\section{%
    Zum Vertauschen von Grenzwert und \textsc{Riemann}-Integral%
}

\textbf{Vertauschen von $\lim_{n \to \infty}$ und $\int_a^b$}:
Seien $f_n \in \C([a,b], \field^d)$ für $n \in \natural$, wobei \\
$\lim_{n \to \infty} f_n(x) = f(x)$ gleichmäßig bzgl. $x \in [a,b]$. \\
Dann gibt es die Grenzwerte
$\lim_{n \to \infty} \left(\int_a^b f_n(x)\dx\right) = \int_a^b f(x)\dx$
und sind gleich.

\textbf{Vertauschen von $\sum_{n=1}^\infty$ und $\int_a^b$}:
Seien $a_n \in \C([a,b], \field^d)$ für $n \in \natural$, wobei \\
$\sum_{n=1}^\infty a_n(x) = S(x)$ gleichmäßig bzgl. $x \in [a,b]$. \\
Dann gibt es die Grenzwerte
$\sum_{n=1}^\infty \left(\int_a^b a_n(x)\dx\right) = \int_a^b S(x)\dx$
und sind gleich.

\textbf{Vertauschen von $\lim_{x \to x_0}$ und $\int_a^b$}:
Seien $M$ ein metrischer Raum mit $X \subset M$ und \\
$x_0 \in \acc(X)$.
Außerdem sei $f: X \times [a,b] \rightarrow \field^d$ eine Funktion mit
$\forall_{x \in X}\; f(x, \cdot) \in \C([a,b])$ und
$\lim_{x \to x_0} f(x, y) = \varphi_{x_0}(y)$ werde gleichmäßig bzgl.
$y \in [a,b]$ angenommen. \\
Dann gibt es die Grenzwerte
$\lim_{x \to x_0} \left(\int_a^b f(x, y)\dy\right) =
\int_a^b \varphi_{x_0}(y)\dy$
und sind gleich.

\linie

\textbf{kartesisches Produkt zweier metrischer Räume}:
Seien $(M_1, d_1)$ und $(M_2, d_2)$ zwei metrische Räume und
$M = M_1 \times M_2$.
Definiere für $m' = (x', y') \in M$ und $m'' = (x'', y'') \in M$
die Funktion $d(m', m'') = d_1(x', x'') + d_2(y', y'')$.
Damit wird $(M, d)$ zum metrischen Raum,
$d$ ist die von $M = M_1 \times M_2$ induzierte Metrik.
Ist eine Folge von $m_k = (x_k, y_k) \in M$ und $m = (x, y)$ gegeben,
so ist $m_k \xrightarrow{d} m \;\Leftrightarrow\;
(x_k \xrightarrow{d_1} x) \land (y_k \xrightarrow{d_2} y)$.

\textbf{Lemma für kompakte Mengen}:
Seien $X \subset M_1$ kompakt und $Y \subset M_2$ kompakt. \\
Dann ist $X \times Y \subset M$ ebenfalls kompakt.

\linie

\textbf{Satz (Stetigkeit von $J(x)$)}:
Seien $M$ ein metrischer Raum, $X \subset M$ kompakt und \\
$f: X \times [a,b] \rightarrow \field^d$ mit
$f \in \C(X \times [a,b], \field^d)$. \\
Dann ist $J(x) = \int_a^b f(x, y)\dy$ stetig in $x$.

\pagebreak

\section{%
    Zum Vertauschen von Grenzwert und Ableitung%
}


\textbf{Vorsicht}:
Seien $f_n \in \C([a,b])$ gegeben mit $f_n$ dif"|fb. in $\left]a,b\right[$
sowie $f_n(x) \rightrightarrows f(x)$ gleichmäßig bzgl. $x \in [a,b]$.
Dann gilt i.\,A. \emph{nicht}, dass $f$ in $\left]a,b\right[$ dif"|fb. ist
und $f'(x) = \lim_{n \to \infty} f_n'(x)$!

\textbf{Vertauschen von $\lim_{n \to \infty}$ und $\frac{d}{dx}$}:
Seien $f_n \in \C^1([a,b], \field^d)$ für $n \in \natural$
und für alle $x \in [a,b]$ existiere der Grenzwert
$\lim_{n \to \infty} f_n(x) = f(x)$ sowie der Grenzwert
$\lim_{n \to \infty} f_n'(x) = \varphi(x)$ werde gleichmäßig bzgl.
$x \in [a,b]$ angenommen.
Dann ist auch $f \in \C^1([a,b])$ und $f'(x) = \varphi(x)$.

Unter diesen Voraussetzungen gilt somit
$\lim_{n \to \infty} \left(\frac{d}{dx} f_n(x)\right) =
\frac{d}{dx} \left(\lim_{n \to \infty} f_n(x)\right)$.

\textbf{Vertauschen von $\sum_{n=1}^\infty$ und $\frac{d}{dx}$}:
Seien $a_k \in \C^1([a,b], \field^d)$ für $k \in \natural$ und für alle \\
$x \in [a,b]$ existiere die Reihe
$\sum_{k=1}^\infty a_k(x) = S(x)$ sowie der Grenzwert
$\sum_{k=1}^\infty a_k'(x) = T(x)$ werde gleichmäßig bzgl.
$x \in [a,b]$ angenommen. \\
Dann ist auch $S \in \C^1([a,b])$ und $S'(x) = T(x)$.

\linie

\textbf{partielle Ableitung}:
Seien $f: \left]a,b\right[ \times Y \rightarrow \field^d$ eine Funktion
mit $x_0 \in \left]a,b\right[$ und $y_0 \in Y$. \\
Dann ist
\fracsize{$\left.\frac{\partial f}{\partial x}\right|_{(x_0, y_0)} :=
\lim_{h \to 0,\; h \in \field} \frac{f(x_0 + h, y_0) - f(x_0, y_0)}{h}$} \\
die \emph{partielle Ableitung} von $f$ nach $x$ im Punkt $(x_0, y_0)$.

\textbf{Vertauschen von $\lim_{y \to y_0}$ und $\frac{d}{dx}$}:
Seien $M$ ein metrischer Raum, $Y \subset M$, $y_0 \in \acc(Y)$ und
$f: \left]a,b\right[ \times Y \rightarrow \field^d$ eine Funktion, wobei
$\forall_{y \in Y}\; f(\cdot, y) \in \C^1([a,b], \field^d)$,
für alle $x \in [a,b]$ existiere der Grenzwert
$\lim_{y \to y_0} f(x,y) = \varphi(x)$ sowie der Grenzwert
$\lim_{y \to y_0}$ \fracsize{$\frac{\partial f(x,y)}{\partial x}$} $= \psi(x)$
werde gleichmäßig bzgl. $x \in [a,b]$ angenommen.
Dann ist auch $\varphi \in \C^1([a,b])$ und $\varphi'(x) = \psi(x)$.

Unter diesen Voraussetzungen gilt somit
$\frac{d}{dx} \left(\lim_{y \to y_0} f(x,y)\right) =
\lim_{y \to y_0} \left(\frac{\partial}{\partial x} f(x,y)\right)$.

%\textbf{Folgerung}: Ist $f(x,y)$ stetig in $y$ ??

\section{%
    Dif"|ferenzieren und Integrieren von parameterabhängigen Integralen%
}

\textbf{Satz}:
Seien $\Omega = [a,b] \times [c,d]$
und $f, \frac{\partial f}{\partial y} \in \C(\Omega, \field^d)$. \\
Dann ist $J(y) = \int_a^b f(x,y)\dx \in \C^1([c,d])$
und $J'(y) = \int_a^b \frac{\partial f(x,y)}{\partial y} \dx$.

\textbf{Satz}: Seien $\Omega = [a,b] \times [c,d]$ und
$f \in \C(\Omega, \field^d)$. \\
Dann ist $\int_a^b \left(\int_c^d f(x,y)\dy\right)\dx
= \int_c^d \left(\int_a^b f(x,y)\dx\right)\dy$.

\section{%
    Stetigkeit und Dif"|f.barkeit von Integralen mit parameterabh.
    Grenzen%
}

Seien $\Omega = [a,b] \times [c,d]$, $f \in \C(\Omega, \field^d)$
und $\alpha, \beta: [c,d] \rightarrow [a,b]$. \\
Man betrachtet nun das Integral $J(y) = \int_{\alpha(y)}^{\beta(y)} f(x,y)\dx$.

\textbf{Satz 1}: Seien $f \in \C(\Omega, \field^d)$ und
$\alpha, \beta \in \C([c,d], [a,b])$. \\
Dann ist $J(y) = \int_{\alpha(y)}^{\beta(y)} f(x,y)\dx \in \C([c,d])$.

\textbf{Satz 2}: Seien $\Omega_\delta = [a,b] \times [c - \delta, c + \delta]$
für $\delta > 0$, $f \in \C(\Omega_\delta)$,
$\frac{\partial f}{\partial y} \in \C(\Omega_\delta)$
und $\alpha, \beta$ in $\left]c,d\right[$ dif"|fb.
Dann ist $J(y)$ ist dif"|fb. für $y \in \left]c,d\right[$ und \\
$J'(y_0) = \int_{\alpha(y_0)}^{\beta(y_0)}
\left.\frac{\partial f(x, y)}{\partial y}\right|_{y = y_0}\dx +
\beta'(y_0) \cdot f(\beta(y_0), y_0) - \alpha'(y_0) \cdot f(\alpha(y_0), y_0)$.

\section{%
    Zum Vertauschen von Grenzwert und uneigentlichem Integral%
}

\textbf{Vertauschen von $\lim_{n \to \infty}$ und $\int_0^{+\infty}$}:
Seien $f_n \in \C(\left[0,+\infty\right[, \real)$ für $n \in \natural$,
wobei \\
$\lim_{R \to \infty} \left(\int_0^R f_n(x)\dx\right) =
\int_0^{+\infty} f_n(x)\dx$ gleichmäßig bzgl. $n \in \natural$ und \\
$f(x) = \lim_{n \to \infty} f_n(x)$ gleichmäßig bzgl. $x \in [0,R]$
für jedes fixierte $R > 0$ angenommen werden. \\
Dann gibt es die Grenzwerte
$\int_0^{+\infty} f(x)\dx =
\lim_{n \to \infty} \left(\int_0^{+\infty} f_n(x)\dx\right)$
und sind gleich.

\textbf{Vertauschen von $\sum_{n=1}^\infty$ und $\int_0^{+\infty}$}:
Seien $f_n \in \C(\left[0,+\infty\right[, \real)$ für $n \in \natural$,
wobei \\
$\lim_{R \to \infty} \left(\int_0^R f_n(x)\dx\right) =
\int_0^{+\infty} f_n(x)\dx$ gleichmäßig bzgl. $n \in \natural$ und \\
$f(x) = \sum_{n=1}^\infty f_n(x)$ gleichmäßig bzgl. $x \in [0,R]$
für jedes fixierte $R > 0$ angenommen werden. \\
Dann gibt es die Grenzwerte
$\int_0^{+\infty} f(x)\dx =
\sum_{n=1}^\infty \left(\int_0^{+\infty} f_n(x)\dx\right)$
und sind gleich.

\textbf{Vertauschen von $\lim_{y \to y_0}$ und $\int_0^{+\infty}$}:
Seien $M$ ein metrischer Raum, $Y \subset M$, $y_0 \in \acc(Y)$ und
$f \in \C(\left[0, +\infty\right[ \times Y, \real)$, wobei \\
$\lim_{R \to \infty} \left(\int_0^R f(x,y)\dx\right) =
\int_0^{+\infty} f(x,y)\dx$ gleichmäßig bzgl. $y \in Y$ und \\
$\lim_{y \to y_0} f(x,y) = \varphi_{y_0}(x)$ gleichmäßig bzgl. $x \in [0,R]$
angenommen werden. \\
Dann existieren die Grenzwerte
$\int_0^{+\infty} \varphi_{y_0}(x)\dx =
\lim_{y \to y_0} \left(\int_0^{+\infty} f(x,y)\dx\right)$ und sind gleich.

\linie

\textbf{Vertauschen von $\frac{d}{dx}$ und $\int_0^{+\infty}$}:
Seien $\Omega = \left[0, +\infty\right[ \times [c,d]$ und
$f, \frac{\partial f}{\partial y} \in \C(\Omega, \real)$, \\
wobei für alle $y \in [c,d]$ der Grenzwert
$\lim_{R \to \infty} \left(\int_0^R f(x,y)\dx\right) =
\int_0^{+\infty} f(x,y)\dx$ existiert und \\
$\lim_{R \to \infty}
\left(\int_0^R \frac{\partial f(x, y)}{\partial y}\dx\right) =
\int_0^{+\infty} \frac{\partial f(x,y)}{\partial y}\dx$ gleichmäßig
bzgl. $y \in [c,d]$ angenommen wird. \\
Dann ist $\int_0^{+\infty} f(x,y)\dx$ dif"|ferenzierbar und
$\frac{d}{dy} \left(\int_0^{+\infty} f(x,y)\dx\right) =
\int_0^{+\infty} \frac{\partial f(x,y)}{\partial y}\dx$.

\textbf{Vertauschen von $\int_c^d$ und $\int_0^{+\infty}$}:
Seien $\Omega = \left[0, +\infty\right[ \times [c,d]$ und
$f \in \C(\Omega, \real)$, wobei \\
$\lim_{R \to \infty} \int_0^R f(x,y)\dx = \int_0^{+\infty} f(x,y)\dx$
gleichmäßig bzgl. $y \in [c,d]$ angenommen wird. \\
Dann ist $\int_c^d \left(\int_0^{+\infty} f(x,y)\dx\right)\dy =
\int_0^{+\infty} \left(\int_c^d f(x,y)\dy\right)\dx$.

\section{%
    Potenzreihen%
}

\textbf{Potenzreihe}:
Für $k \in \natural_0$ seien $a_k \in \complex$ sowie $z_0 \in \complex$
gegeben. \\
Man definiert $S_n(z) = a_0 + \sum_{k=1}^n a_k(z - z_0)^k$ für
$z \in \complex$. \\
Dann heißt $S(z) = \lim_{n \to \infty} S_n(z) =
\sum_{k=0}^\infty a_k(z - z_0)^k$ \emph{Potenzreihe}.

Man will untersuchen, für welche $z$ eine gegebene Potenzreihe konvergiert
und für welche nicht.
Im Weiteren betrachten wir durch $\widetilde{z} = z - z_0$, also
$\sum_{k=0}^\infty a_k (z - z_0)^k = \sum_{k=0}^\infty a_k \widetilde{z}^k$,
ohne Einschränkung nur noch Potenzreihen mit $z_0 = 0$.

\textbf{Konvergenzkreis/-radius}:
Sei eine Potenzreihe $\sum_{k=0}^\infty a_k z^k$ gegeben.
$U_R = \{z \in \complex \;|\; |z| < R\}$ heißt \emph{Konvergenzkreis}
der Potenzreihe mit dem \emph{Konvergenzradius} $R$, falls \\
$\sum_{k=0}^\infty a_k z^k$ für alle $z \in U_R$ konvergent ist
(d.\,h. $|z| < R$) und \\
$\sum_{k=0}^\infty a_k z^k$ für alle $z \notin \overline{U_R}$ divergent ist
(d.\,h. $|z| > R$). \\
Für $|z| = R$ macht der Konvergenzkreis keine Aussage.
Möglich sind für $R$ auch $R = 0$ (konvergent nur für $z = 0$)
und $R = +\infty$ (konvergent für alle $z \in \complex$).

\linie
\pagebreak

\textbf{Satz von \textsc{Cauchy}-\textsc{Hadamard}}:
Jede Potenzreihe $\sum_{k=0}^\infty a_k z^k$ ($a_k, z \in \complex$)
besitzt einen Konvergenzkreis mit dem Konvergenzradius
\fracsize{$R = \frac{1}{\limsup_{k \to \infty} \sqrt[k]{|a_k|}}$}. \\
Für $a_n \not= 0$, $n \ge N$ gilt
$R = \lim_{n \to \infty} \Big|$\fracsize{$\frac{a_n}{a_{n+1}}$}$\Big|$, falls
der Grenzwert existiert. \\
(Alle Fälle $R \in \left]0, +\infty\right[ \cup \{0\} \cup \{+\infty\}$
sind zugelassen.)

\textbf{Satz}: Sei $R \le +\infty$ und $R > 0$.
Wähle $R_1 < R$ mit $R_1 > 0$. \\
Dann konvergiert $S(z) = \sum_{k=0}^\infty a_k z^k$ gleichmäßig
bzgl $|z| < R_1$.

\linie

\textbf{Satz (Dif"|ferenzieren von Potenzreihen)}:
Seien $S(z) = \sum_{k=0}^\infty c_k (z - a)^k$ eine Potenzreihe mit
Konvergenzradius $R > 0$ und
$S_1(z) = \sum_{k=1}^\infty k c_k (z - a)^{k-1}$. \\
Dann ist $S(z)$ im Konvergenzkreis komplex dif"|ferenzierbar und die
Ableitung erfolgt gliedweise, d.\,h. es gilt $S'(z) = S_1(z)$ für alle
$z \in \complex$ mit $|z - a| < R$. \\
Dabei ist die Ableitung eine Potenzreihe mit demselben Konvergenzradius.

\textbf{Satz (Integrieren von Potenzreihen)}:
Seien $S(z) = \sum_{k=0}^\infty c_k (z - a)^k$ eine Potenzreihe mit
Konvergenzradius $R > 0$ und
$S_{-1}(z) = C + \sum_{k=0}^\infty \frac{c_k}{k + 1} (z - a)^{k+1}$. \\
Dann ist $S_{-1}(z)$ eine Potenzreihe mit demselben Konvergenzradius und
gliedweise dif"|fb. mit $S_{-1}'(z) = S(z)$.
Also ist $S_{-1}(z)$ eine Stammfunktion von $S(z)$.

Potenzreihen sind also in ihrem Konvergenzkreis beliebig oft komplex
dif"|ferenzierbar und aufleitbar.
Die Ableitung kann durch gliedweise Dif"|ferenzieren bestimmt werden,
analog wird die Stammfunktion durch gliedweise Integrieren bestimmt.

\linie

\textbf{Potenzreihen als \textsc{Taylor}-Reihen darstellen}:
Sei $S(z) = \sum_{k=0}^\infty c_k (z - a)^k$ eine Potenzreihe mit
Konvergenzradius $R > 0$.
Dann ist $c_0 = S(a)$, $c_1 = S'(a)$, $c_2 = \frac{S''(a)}{2!}$, \ldots,
$c_k = \frac{S^{(k)}(a)}{k!}$, d.\,h. die $c_k$ sind die Taylorkoef"|fizienten.
Also ist jede Potenzreihe in ihrem Konvergenzkreis durch ihre Taylorreihe
darstellbar:
\fracsize{$S(z) = \sum_{k=0}^\infty \frac{S^{(k)}(a)}{k!} (z - a)^k$}.

\textbf{\textsc{Taylor}-Reihe einer Funktion}:
Sei $f: \left]-R,R\right[ \rightarrow \complex$ eine Funktion mit $R > 0$,
wobei $f$ in $x_0 = 0$ beliebig oft reell dif"|ferenzierbar ist. \\
Dann lässt sich bekannterweise $f$ durch $f(x) = T_n(x) + r_n(0, x)$
mit \fracsize{$T_n(x) = \sum_{k=0}^n \frac{f^{(k)}(0)}{k!} x^k$} als
\textbf{\textsc{Taylor}-Polynom} und $r_n(0, x) = r_n(x) = o(x^n)$ für
$x \to 0$ darstellen.

Im Allgemeinen muss $T_n(x)$ für $n \to \infty$ nicht unbedingt
konvergieren. \\
Besitzt jedoch \fracsize{$t(x) = \lim_{n \to \infty} T_n(x) =
\sum_{k=0}^\infty \frac{f^{(k)}(0)}{k!} x^k$} einen Konvergenzradius $R > 0$,
so bezeichnet man $t(x)$ als \textbf{\textsc{Taylor}-Reihe} von $f(x)$.
Allerdings ist i.\,A. $t(x) \not= f(x)$ für alle $x$, d.\,h.
der Rest $r_n(x)$ muss für $n \to \infty$ nicht gegen $0$ konvergieren!

\textbf{als \textsc{Taylor}reihe darstellbar}:
$f$ ist in einer Umgebung von $x_0 = 0$ als Taylor-Reihe darstellbar,
falls $f(x) = t(x)$ ist für $|x| < \varepsilon$.

\textbf{Satz (Kriterium für Darstellbarkeit)}:
Seien $f: \left]-R,R\right[ \rightarrow \complex$ beliebig oft dif"|fb., \\
$\forall_{k \in \natural} \exists_{C(k)} \forall_{|x| < R}\;
|f^{(k)}(x)| \le C(k)$ sowie
\fracsize{$\lim_{k \to \infty} \frac{C(k) R^k}{k!} = 0$}. \\
Dann ist $f(x) = t(x)$ für $|x| < R$.

\section{%
    Der Satz von \textsc{Stone} und \textsc{Weierstraß}%
}

\textbf{Satz von \textsc{Stone} und \textsc{Weierstraß}}:
Die Menge der Polynome ist in $C([a,b])$ dicht.

\textbf{äquivalente Formulierung}:
Gegeben sei eine stetige Funktion $f: [a,b] \rightarrow \complex$.
Dann gibt es eine Folge von Polynomen $P_n(x)$, sodass
$\lim_{n \to \infty} P_n(x) = f(x)$ gleichmäßig bzgl. $x \in [a,b]$.

\section{%
    Die \textsc{Euler}schen Integrale%
}

\textbf{Betafunktion}:
$B(a, b) = \int_0^1 x^{a-1} (1 - x)^{b-1} \dx$, $a, b > 0$

$B(a, b)$ ist konvergent für alle $a, b > 0$: \\
$a \ge 1$, $b \ge 1$: Riemann-Integral, da Integrand beschränkt,
sonst konvergentes uneigentliches Integral, denn
$\int_0^1 x^{a-1} (1 - x)^{b-1} \dx =
\int_0^{\frac{1}{2}} x^{a-1} (1 - x)^{b-1} \dx +
\int_{\frac{1}{2}}^1 x^{a-1} (1 - x)^{b-1} \dx$, \\
für das Integral $\int_0^{\frac{1}{2}} x^{a-1} (1 - x)^{b-1} \dx$
gilt $(1 - x)^{b-1} \le \left(\frac{1}{2}\right)^{b-1}$ für
$0 \le x \le \frac{1}{2}$ sowie \\
$\int_0^{\frac{1}{2}} x^{a-1} \dx = \frac{2^{-a}}{a} < \infty$, \\
für das Integral $\int_{\frac{1}{2}}^1 x^{a-1} (1 - x)^{b-1} \dx$
gilt $x^{a-1} \le \left(\frac{1}{2}\right)^{a-1}$ für
$\frac{1}{2} \le x \le 1$, sowie \\
$\int_{\frac{1}{2}}^1 (1 - x)^{b-1} \dx =
\frac{2^{-b}}{b} < \infty$, $0 < a < 1$, $0 < b < 1$

\textbf{Eigenschaften der Betafunktion}:
\begin{enumerate}
    \item
    $B(a, b) = B(b, a)$

    \item
    $B(a, b) =$
    \fracsize{$\frac{b - 1}{a + b - 1}$} $\cdot\; B(a, b - 1)$
    für $a > 0$, $b > 1$ und \\
    $B(a, b) =$
    \fracsize{$\frac{a - 1}{a + b - 1}$} $\cdot\; B(a - 1, b)$
    für $a > 1$, $b > 0$

    \item
    $B(a, 1 - a) =$
    \fracsize{$\frac{\pi}{\sin a\pi}$}
\end{enumerate}

\linie

\textbf{Gammafunktion}:
$\Gamma(a) = \int_0^\infty x^{a-1} e^{-x} \dx$, $a > 0$

$\Gamma(a)$ ist konvergent für alle $a > 0$: \\
$\int_0^\infty x^{a-1} e^{-x} \dx = \int_0^1 x^{a-1} e^{-x} \dx +
\int_1^\infty x^{a-1} e^{-x} \dx$, \\
das Integral $\int_0^1 x^{a-1} e^{-x} \dx$ ist für
$a \ge 1$ ein Riemann-Integral, da Integrand beschränkt,
sonst gilt $x^{a-1} e^{-x} \le x^{a-1}$ für $a < 1$, $x \in [0,1]$, \\
das Integral $\int_1^\infty x^{a-1} e^{-x} \dx$ konvergiert, da
$x^{a-1} e^{-x} \le x^{-2}$ für genügend große $x$

\textbf{Eigenschaften der Gammafunktion}:
\begin{enumerate}
    \item
    $\Gamma(a)$ stetig

    \item
    $\Gamma(a)$ dif"|ferenzierbar und
    $\Gamma'(a) = \int_0^\infty x^{a-1} \ln(x) e^{-x} \dx$

    \item
    $\Gamma(a + 1) = a \cdot \Gamma(a)$, es gilt daher
    $\Gamma(n + a) = (n + a - 1) (n + a - 2) \dotsm (a + 1) a \cdot \Gamma(a)$
    und insbesondere $\Gamma(n + 1) = n!$
    (da $\Gamma(1) = 1$)

    \item
    $\lim_{a \to 0} \Gamma(a) = \lim_{a \to \infty} \Gamma(a) = +\infty$
\end{enumerate}

\linie

\textbf{Zusammenhang zwischen Beta- und Gammafunktion}:
$B(a, b) =$ \fracsize{$\frac{\Gamma(a) \Gamma(b)}{\Gamma(a + b)}$}

\section{%
    \emph{Zusatz}: Ein analytischer Beweis des Hauptsatzes der Algebra%
}

\textbf{Satz (Hauptsatz der Algebra)}: \\
Jedes Polynom in $\complex$ vom Grad größer/gleich $1$ besitzt mindestens
eine Nullstelle.

\textbf{Lemma 1}:
$\forall_{M > 0} \exists_{R > 0} \forall_{|z| \ge R}\; |p(z)| \ge M$

\textbf{Lemma 2}:
$\forall_{z^\ast \in \complex,\; p(z^\ast) \not= 0} \exists_{h \in \complex}\;
|p(z^\ast + h)| < |p(z^\ast)|$

\pagebreak
