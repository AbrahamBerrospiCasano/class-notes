\chapter{%
    Populationsdynamik%
}

\section{%
    \name{Fibonacci}-Zahlen%
}

\textbf{Kaninchenpopulation}:
Angenommen, Kaninchenpaare bringen jeden Monat ein neues Paar zur Welt.
Die neuen Kaninchenpaare sind jedoch jeweils erst nach einem Monat geschlechtsreif.
Startet man mit einem neugeborenen Kaninchenpaar, so ist die Anzahl der im $n$-ten Monat
vorhandenen Paare $f_n$ gegeben durch $f_0 := 1$, $f_1 := 1$ und $f_n := f_{n-1} + f_{n-2}$
(\begriff{\name{Fibonacci}-Zahlen}).

\textbf{dahinter stehende Modellannahmen}:
\begin{itemize}
    \item
    diskrete Kaninchen werden gezählt

    \item
    zum Start ein neugeborenes Kaninchenpaar

    \item
    Vermehrung genau einmal pro Monat

    \item
    erst nach einem Monat zeugungsfähig

    \item
    Geburten immer paarweise

    \item
    keine negativen Einflüsse: Kaninchen sterben nicht, unendliche Ressourcen (Futter, Platz)
\end{itemize}

\textbf{\name{Fibonacci}-Zahlen}:
Es gilt $f_n = \frac{1}{\sqrt{5}} (\Phi^n - (1 - \Phi)^n)$
mit $\Phi := \frac{1 + \sqrt{5}}{2} \approx 1.618$ dem \begriff{Goldenen Schnitt}.
Somit wächst $f_n$ exponentiell, da $\lim_{n \to \infty} \frac{f_{n+1}}{f_n}
= \lim_{n \to \infty} \frac{\Phi^{n+1}}{\Phi^n} = \Phi$.
Ein anderer Zusammenhang besteht mit der linearen Algebra durch
$A^n = \smallpmatrix{f_{n+1}&f_n\\f_n&f_{n-1}}$ für $A := \smallpmatrix{1&1\\1&0}$.
Dabei sind die Eigenwerte von $A$ gegeben durch $\Phi$ und $1 - \Phi$.

\pagebreak

\section{%
    Modelle mit einer Spezies%
}

Zur besseren Handbarkeit geht man von diskreten zu kontinuierlichen Modellen über.
Für große Populationen ist dies eigentlich keine Einschränkung.
Zunächst wird nur eine Spezies betrachtet.

\textbf{Modell von \name{Malthus}}:
$p(t)$ sei die Population zur Zeit $t \ge 0$.
Pro Zeiteinheit und Individium gibt es die Geburtenrate $\gamma > 0$ und die Sterberate
$\delta > 0$ (jeweils konstant), d.\,h. man erhält die konstante Wachstumsrate
$\lambda := \gamma - \delta$.
Man kommt so auf die ODE $p'(t) = \lambda p(t)$ bzw. diskret
$p(t + \Delta t) = p(t) + \lambda p(t) \Delta t$
mit dem diskreten Zeitschritt $\Delta t$.\\
Die Lösung der ODE ist gegeben duch $p(t) = p_0 e^{\lambda t}$ mit dem Anfangswert $p(0) = p_0$.
Man erhält \begriff{exponentielles Wachstum}, dafür sind jedoch unbegrenzte Ressourcen nötig.\\
Die Herleitung geht auch diskret:
Ist $\lambda$ konstant, so verdoppelt/halbiert sich $p(t)$ alle $\tau$ Zeiteinheiten.
Wenn man mit $p_0$ zur Zeit $t = 0$ startet, so erhält man nach $k\tau$ Zeiteinheiten
für $k \in \natural$, dass $p(k\tau) = 2^k p_0 = p_0 e^{(\ln 2)/\tau \cdot k\tau}
= p_0 e^{\lambda \cdot k\tau}$ mit $\lambda := \ln 2/\tau$.
Lässt man nun auch $k \in \real$ zu, so erhält man das kontinuierliche Modell.

\linie

\textbf{Modell von \name{Verhulst} (beschränkt)}:
Exponentielles Wachstum ist nur bedingt möglich (z.\,B. Weltbevölkerung 1700 -- 1960).
Bei großen Populationen werden beschränkte Ressourcen wichtig.
Daher betrachtet man nun das lineare Modell $p'(t) = \lambda_0 - \lambda_1 p(t)$
mit $\lambda_0, \lambda_1 > 0$.
Für $p(t) = \overline{p} := \frac{\lambda_0}{\lambda_1}$ gilt, dass $p'(t) = 0$.
$\overline{p}$ heißt deshalb \begriff{Gleichgewichtspunkt}.
$\overline{p}$ ist attraktiv, weil $\lambda_0 - \lambda_1 p$ positiv/negativ ist,
wenn $p - \overline{p}$ negativ/positiv ist.
Das Richtungsfeld der ODE ($t$-$p$-Koordinaten) zeigt deshalb zu $\overline{p}$ hin.
Die Lösung ist gegeben durch $p(t) = \overline{p} + (p_0 - \overline{p}) e^{-\lambda_1 t}$.
Diese Art von Wachstum nennt man \begriff{beschränktes Wachstum}.

\linie

\textbf{Modell von \name{Verhulst} (logistisch)}:
Das lineare Modell hat eine Nachteile.
Zum einen entsteht aus dem Nichts ($p(0) = 0$) Population,
zum anderen ist das Wachstum für kleine $p$ linear und nicht exponentiell.
Das Ziel ist es, für kleine $p$ wie Malthus und für große $p$ wie Verhulst (beschränkt) zu wachsen.
Dazu setzt man $\lambda(p) := a - bp$ für $ab > 0$, d.\,h. die Wachstumsrate soll linear zur
Population schrumpfen.\\
Man erhält $p'(t) = \lambda(p(t)) \cdot p(t) = (a - bp(t)) p(t) = -bp(t)^2 + ap(t)$.
Die Lösung ist explizit angebbar und lautet $p(t) = \frac{ap_0}{bp_0 + (a-bp_0) e^{-at}}$.\\
Für $t \to \infty$ gilt $p(t) \to \frac{a}{b}$, falls $p_0 > 0$.
Für $p \approx 0$ ergibt sich $p' \approx ap$, d.\,h. man erhält exponentielles Wachstum.
Für größere $p$ ist das Modell ähnlich wie das Sättigungsmodell.
Diese Art von Wachstum nennt man \begriff{logistisches Wachstum}.

\textbf{Gleichgewichtspunkte}:
$\overline{p}$ heißt \begriff{kritischer Punkt/GG-Punkt}, falls $p'(t) = 0$ für
$p(t) = \overline{p}$.\\
Es gibt drei Arten von GG-Punkten:
\begin{itemize}
    \item
    \begriff{attraktives GG}:
    $p' < 0$ für $p \in (\overline{p}, \overline{p} + \varepsilon)$ und
    $p' > 0$ für $p \in (\overline{p} - \varepsilon, \overline{p})$

    \item
    \begriff{instabiles GG}:
    $p' > 0$ für $p \in (\overline{p}, \overline{p} + \varepsilon)$ und
    $p' < 0$ für $p \in (\overline{p} - \varepsilon, \overline{p})$

    \item
    \begriff{Sattelpunkt}:
    sonst
\end{itemize}

\textbf{kritische Punkte beim logistischen Wachstum}:
Es gibt zwei GG-Punkte $p = 0$ (instabil) und $p = \overline{p} := \frac{a}{b}$ (attraktiv).
Für $0 < p < \frac{\overline{p}}{2}$ steigt das Wachstum streng monoton.
In $p = \frac{\overline{p}}{2}$ befindet sich der Wendepunkt, d.\,h. dort ist $p'$ maximal.
Für $\frac{\overline{p}}{2} < p < \overline{p}$ schrumpft das Wachstum streng monoton
und für $p > \overline{p}$ fällt die Population streng monoton gegen $\overline{p}$.

\linie

\textbf{logistisches Wachstum mit kritischer Grenze}:
Reale Populationen sterben meistens aus, wenn eine bestimmte (positive) kritische Grenze
unterschritten wird, denn dann tref"|fen sich Artgenossen zu selten.
Daher kann man das logistische Modell erweitern zu\\
$p'(t) = \alpha (1 - \frac{p(t)}{\beta}) (1 - \frac{p(t)}{\delta}) p(t)$,
d.\,h. $p'$ ist ein kubisches Polynom in $p$ mit Nullstellen $0$, $\beta$, $\gamma$.

\pagebreak

\section{%
    Lineare Zweispeziesmodelle%
}

Nun betrachtet man zwei Spezies $P$ und $Q$ mit Populationsgrößen $p(t)$ und $q(t)$,
zwischen denen es Wechselwirkungen gibt (Kooperation, Konkurrenz usw.).

\textbf{lineares Modell}:
Analog zum Sättigungsmodell von Verhulst kann man ein System zweier ODEs definieren als
$p'(t) = a_1 + b_1 p(t) + c_1 q(t)$ und $q'(t) = a_2 + c_2 p(t) + b_2 q(t)$.
Dabei sollte $a_1, a_2 > 0$ und $b_1, b_2 < 0$ sein, damit Wachstum und Sättigung sichergestellt
werden.

\linie

\textbf{Wettrüsten}:
Sind $p(t)$ und $q(t)$ die Rüstungsausgaben zweier Großmächte $P$ und $Q$, so definieren
$b_1, b_2 < 0$ die Abrüstraten und $c_1, c_2 > 0$ die Aufrüstraten.
$a_1, a_2$ sind die konstanten Aufrüstbeiträge (Abrüstbeiträge falls negativ).

\textbf{GG-Punkt}:
Man kann das System als
$\smallpmatrix{p'\\q'} = \smallpmatrix{b_1&c_1\\c_2&b_2} \smallpmatrix{p\\q} +
\smallpmatrix{a_1\\a_2}$
darstellen.
Gilt $b_1 b_2 \not= c_1 c_2$ (d.\,h. ist $A := \smallpmatrix{b_1&c_1\\c_2&b_2}$ regulär),
dann gibt es einen eindeutigen GG-Punkt
$\smallpmatrix{\overline{p}\\\overline{q}} := -A^{-1} \smallpmatrix{a_1\\a_2}$,
sodass $p' = q' = 0$.
Der GG-Punkt ist stabil genau dann, wenn alle Eigenwerte einen negativen Realteil besitzen:
Seien $\lambda_1, \lambda_2 \in \complex$ die Eigenwerte und $v_1, v_2 \in \real^2$
zugehörige Eigenvektoren.
Weil die Lösung durch $\smallpmatrix{p(t)\\q(t)} = e^{At} \smallpmatrix{p_0\\q_0}$ gegeben ist
und $\smallpmatrix{p_0\\q_0} = \mu_1 v_1 + \mu_2 v_2$ für bestimmte $\mu_1, \mu_2 \in \real$,
folgt $\smallpmatrix{p(t)\\q(t)} = \sum_{i=1}^2 \mu_i e^{At} v_i
= \sum_{i=1}^2 \mu_i e^{\lambda_i t} v_i$.
Besitzen nun die $\lambda_i$ nur negative Realteile, so konvergiert
$|e^{\lambda_i t}| = e^{\Re(\lambda_i)t}$ für $t \to \infty$ gegen $0$.
Die Umkehrung gilt ebenfalls.

\textbf{Richtungsfeld}:
Weil $t$ irrelevant für das Richtungsfeld ist, zeichnet man die Richtungen $(p', q')$ in
Abhängigkeit von $(p, q)$ in einem $p$-$q$-Koordinatensystem.
Existiert ein GG-Punkt $(\overline{p}, \overline{q})^\tp$,
so können die Pfeile auf den beiden Geraden durch den GG-Punkt in Richtung der Eigenvektoren $v_i$
leicht gezeichnet werden:
Gilt nämlich $(p, q)^\tp - (\overline{p}, \overline{q})^\tp = \mu v_i$ für ein $\mu \in \real$,
so erhält man
$(p', q')^\tp = A(p, q)^\tp + (a_1, a_2)^\tp = A(p, q)^\tp - A(\overline{p}, \overline{q})^\tp
= \mu Av_i = \mu \lambda_i v_i$.
Für $\lambda_i \in \real$ zeigen also die Pfeile zum GG-Punkt hin/weg, wenn $\lambda_i < 0$ bzw.
$\lambda_i > 0$.

\linie

\textbf{attraktives Gleichgewicht}:
$\smallpmatrix{p'(t)\\q'(t)} =
\smallpmatrix{-1/10&1/20\\1/20&-1/10} \smallpmatrix{p(t)\\q(t)} + \smallpmatrix{3/40\\0}$\\
Der GG-Punkt ist $(\overline{p}, \overline{q})^\tp = (1, 1/2)$.
Die Eigenwerte sind $\lambda_1 = -1/20$ und $\lambda_2 = -3/20$ mit Eigenvektoren
$v_1 = (1, 1)^\tp$ und $v_2 = (-1, 1)^\tp$.

\textbf{labiles Gleichgewicht}:
$\smallpmatrix{p'(t)\\q'(t)} =
\smallpmatrix{-1/20&1/10\\1/10&-1/20} \smallpmatrix{p(t)\\q(t)} + \smallpmatrix{3/40\\0}$\\
Der GG-Punkt ist $(\overline{p}, \overline{q})^\tp = (1, 1/2)$.
Die Eigenwerte sind $\lambda_1 = 1/20$ und $\lambda_2 = -3/20$ mit Eigenvektoren
$v_1 = (1, 1)^\tp$ und $v_2 = (-1, 1)^\tp$.

\textbf{friedlicher Nachbar}:
$\smallpmatrix{p'(t)\\q'(t)} =
\smallpmatrix{-3/4&1\\-1&-3/4} \smallpmatrix{p(t)\\q(t)} + \smallpmatrix{0\\5/2}$\\
Beim friedlichen Nachbar ist $c_2 < 0$ (rüstet ab, je mehr der andere aufrüstet).
Der GG-Punkt ist $(\overline{p}, \overline{q})^\tp = (8/5, 6/5)$.
Die Eigenwerte sind $\lambda_{1,2} = -3/4 \pm \iu$.
Hier "`rotiert"' das Richtungsfeld spiralförmig um den GG-Punkt,
die Lösung $(p, q)^\tp$ "`schneckert"' sich zum GG-Punkt.

\linie

\textbf{Realismus}:
Das Modell ist nicht sonderlich realistisch, denn bei zu großen Dif"|ferenzen von $p(t)$ und $q(t)$
würde $P$ oder $Q$ den Gegner angreifen.
Es sollte also so verbessert werden, dass $p(t) \gg q(t)$ zu einem Aufrüsten von $Q$ führt
und $q(t) \gg p(t)$ zu einem Aufrüsten von $P$.

\pagebreak
