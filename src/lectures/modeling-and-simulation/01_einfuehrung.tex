\section{%
    Einführung%
}

\subsection{%
    Modelle und Simulationen%
}

\textbf{wissenschaftlicher Erkenntniserwerb}:
Die zwei klassischen Säulen des Erkenntniserwerbs sind
Theorie (analytische Berechnungen, Gedankenexperimente) und
Experiment (zur Verfikation der Theorie/wo keine Theorie vorhanden).
Allerdings versagen beide Methoden oft, denn die Theorie löst meist nur einfache
Szenarien explizit und Experimente können unmöglich, gefährlich, unerwünscht oder zu teuer sein.
Daher hat sich als dritte Säule die Simulation etabliert.

\linie

\textbf{Modell}:
Ein \begriff{Modell} ist ein vereinfachendes Abbild einer partiellen Realität.\\
Man unterscheidet zwischen \begriff{konkreten} (Modellbau, Windkanal usw.) und
\begriff{abstrakten} (formale, mathematische Beschreibung) Modellen.

\textbf{mathematische Modellierung}:
Die Herleitung/Analyse eines math. Modells ist eine
Formalisierung/Mathematisierung des Problems durch zur besseren Lösbarkeit.
\begin{enumerate}
    \item
    informale Bescheibung des Problems in Prosaform
    
    \item
    semiformale Beschreibung mit dem Instrumentarium der Anwendungswissenschaft
    
    \item
    streng formale, konsistente Beschreibung
\end{enumerate}

\linie

\textbf{Simulation}:
Eine \begriff{Simulation} ist ein virtuelles (i.\,A. rechnergestütztes) Experiment am Modell.
Die Simulation ist das eigentliche Ziel der Modellierung.

\textbf{Modellbildung in versch. Wissenschaften}:
\begin{itemize}
    \item
    \emph{exakte Naturwissenschaften}:
    lange Tradition, z.\,B. Physik, Modellbildung i.\,A. anerkannt
    
    \item
    \emph{staatliche Wirtschaftspolitik}:
    stark umstritten, verschiedene Lager erstellen "`ihre"' Modelle
    
    \item
    \emph{Klimapolitik}:
    stark voneinander abweichende Theorien zu Ozonloch/globaler Erwärmung
    
    \item
    \emph{Spieltheorie}:
    von Neumanns Min-Max-Prinzip kaum realistisch für Zocker
\end{itemize}

\textbf{Ziele von Simulation}:
\begin{itemize}
    \item
    \emph{besseres Verständnis}
    (Erdbeben, Einsturz des WTC an 9/11)
    
    \item
    \emph{Optimierung}
    (Flugeinsatzplan, Wärmeabtransport, Rechensystem-Durchsatz)
    
    \item
    \emph{Vorhersage}
    (Klimaveränderungen, Wetter, Bevölkerungswachstum)
\end{itemize}

\linie

\textbf{Simulationspipeline}:
\begin{itemize}
    \item
    \emph{Modellierung}:
    vereinfachende formale Beschreibung eines geeigneten Ausschnitts
    
    \item
    \emph{Berechnung/Simulation im engeren Sinn}:
    geeignete Aufbereitung des Modells
    
    \item
    \emph{Implementierung}:
    ef"|fiziente Umsetzung der Berechnungsalgorithmen
    
    \item
    \emph{Visualierung/Datenexploration}:
    Interpretation der Ergebnisse eines Simulationslaufs
    
    \item
    \emph{Validierung}:
    Abgleich von Simulationergebnissen z.\,B. mit Experimenten
    
    \item
    \emph{Einbettung}:
    Integration in Simulationskontext
\end{itemize}

\linie

\textbf{Anwendungen}:
Physik,
Chemie,
Biologie,
Materialwissenschaften,
Klima/Wetter,
Automobilindustrie,
Nationalökonomie,
Finanzwirtschaft,
Halbleiterindustrie,
Computergrafik,
Logistik/Ablaufplanung,
Verkehrstheorie,
Strategie,
Wahl-/Meinungsforschung,
Codierungstheorie

\pagebreak

\subsection{%
    Herleitung von Modellen%
}

\textbf{Fragen}:
\begin{itemize}
    \item
    \emph{Was genau soll modelliert werden?}
    (Wirkungsgrad eines Katalysators oder die detaillierten Reaktionsvorgänge,
    Bevölkerungswachstum in Afrika oder nur in Kairo,
    Durchsatz durch ein Rechnernetz oder die mittlere Durchlaufzeit eines Pakets)
    
    \item
    \emph{Welche Größen spielen eine Rolle (qualitativ)
    und wie groß ist ihr Einfluss (quantitativ)?}
    (Raumschiff-Flugbahn: Gravitation von Mond, Pluto,
    Dow Jones morgen: Äußerungen von Bernanke, uns)
    
    \item
    \emph{In welcher Beziehung stehen die Größen zueinander?}
    (qualitativ, quantitativ)
    
    \item
    \emph{Mit welchem Instrumentarium lassen sich die Abhängigkeiten beschreiben?}
\end{itemize}
Frühe Festlegungen bestimmen dabei spätere Simulationsergebnisse!

\linie

\textbf{Instrumentarien zur Beschreibung von Beziehungen}:
\begin{itemize}
    \item
    \emph{algebraische Gleichungen und Ungleichungen}:
    $E = mc^2$, $w^\tp x \le 10$
    
    \item
    \emph{Systeme gewöhnlicher Di"|ferentialgleichungen}:
    $\ddot{y} + y = 0$ (Oszillation eines linearen Pendels),
    $\dot{y} = y$ (exponentielles Wachstum),
    $\dot{x} = -mx + ay + c$, $\dot{y} = bx - ny + d$ mit $a, b, c, d, m, n \ge 0$
    (Wettrüsten zweier Großmächte)
    
    \item
    \emph{Systeme partieller Di"|ferentialgleichungen}:
    $u_xx + u_yy = f$ für $(x, y) \in \Omega$, $u = 0$ für $(x, y) \in \partial \Omega$
    (Verformung einer am Rand eingespanntne Membran unter Last $f$)
    
    \item
    \emph{Automaten/Zustandsübergangsdiagramme}:\\
    Warteschlangen,
    Texterkennung,
    Wachstumsprozesse mit zellulären Automaten
    
    \item
    \emph{Graphen}:
    Rundreisen (TSP),
    Reihenfolgeprobleme,
    Rechensysteme,
    Abläufe
    
    \item
    \emph{Wahrscheinlichkeitsverteilungen}:
    Ankunft in Warteschlange,
    Zustimmung zur Politik in Abh. von Arbeitslosenquote,
    Kontrolltheorie (Störungen),
    randomisierte Heuristiken
    
    \item
    \emph{Fuzzy Logic}:
    Regelungen von Wasch-/Spülmaschinen,
    Fotoapparate
    
    \item
    \emph{neuronale Netze}
    
    \item
    \emph{Sprachkonzepte}:
    UML
    
    \item
    \emph{algebraische Strukturen}:
    Gruppen in der Quantenmechanik,
    endliche Körper (Kryptologie)
\end{itemize}

\linie

\textbf{Simulationsaufgabe}:
Welche Gestalt hat die resultierende Aufgabenstellung?
\begin{itemize}
    \item
    \emph{Finde eine Lösung des LGS}
    (gültige Startlösung für lineare Optimierung).
    
    \item
    \emph{Finde die Lösung des LGS}
    (eindeutig lösbare PDE).
    
    \item
    \emph{Gibt es eine Lösung}
    (Hamilton-Weg im Graphen)?
    
    \item
    \emph{Löse eine unbeschränkte Extremalaufgabe}
    (kürzester Weg Quelle -- Senke).
    
    \item
    \emph{Löse eine beschränkte Extramalaufgabe}
    (Rucksackproblem, lineare Optimierung).
    
    \item
    \emph{Ermittle den Störenfried bzw. den Flaschenhals}
    (Komponente maximaler Auslastung).
\end{itemize}

\pagebreak

\subsection{%
    Analyse%
}

\textbf{Beispiele zu Aussagen zur Handhab- und Lösbarkeit}:
\begin{itemize}
    \item
    \emph{Existenz von Lösungen}:
    Populationsdynamik (Gibt es einen stationären Grenzzustand, wenn ja, wird dieser erreicht?),
    Reihenfolgeproblem (Ist der Präzedenzgraph zyklenfrei?),
    Minimierung (Gibt es Minima oder nur Sattelpunkte?)
    
    \item
    \emph{Eindeutigkeit von Lösungen}:
    Minimierung (Lokales oder globales Minimierung?),
    Molekulardynamik (Stabile Zustände oder Oszillationen zwischen verschiedenen Lösungen?),
    alle Lösungen gleichwertig?
    
    \item
    \emph{stetige Abhängigkeit der Resultate von den Eingabedaten}
    (Eingabe enthält Anfangs- und Randwerte, Startzustände usw.,
    entspricht Kondition/Sensitivität)
\end{itemize}

\linie

\textbf{sachgemäß gestellt}:
Ein Problem heißt \begriff{sachgemäß gestellt}, wenn es stets eine eindeutige Lösung gibt und
diese stetig von den Eingabedaten abhängt.
Die meisten Probleme sind allerdings unsachgemäß gestellt.

\textbf{inverses Problem}:
Bei einem \begriff{inversen Problem} ist ein Ergebnis vorgegeben, gesucht ist die
Anfangseinstellung
(z.\,B. Wirtschaftspolitik, Technik, Rechnernetz).
Strategien zur Lösung umfassen sinnvolles Ausprobieren und Anpassen (Folge von Vorwärtsproblemen)
und die Lösung eines verwandten, regularisierten Problems, das sachgemäß gestellt ist.

\linie

\textbf{Eignung für weitere Verarbeitung}:
Ist das Modell für automatisierte Lösung geeignet?
\begin{itemize}
    \item
    \emph{Verfügbarkeit/Genauigkeit der Eingabedaten}
    
    \item
    \emph{Implementierungsaufwand}:
    Verfügbarkeit notwendiger Software
    
    \item
    \emph{erforderlicher Rechen-/Speicheraufwand absolut}:
    NP-vollständige Probleme, Wetter
    
    \item
    \emph{erforderlicher Rechen-/Speicheraufwand relativ}:
    Ist das Modell kompetitiv?
    
    \item
    \emph{Empfindlichkeit}
\end{itemize}

\subsection{%
    Lösungsmöglichkeiten%
}

\textbf{analytisch}:
Bei einer \begriff{analytischen Lösung} erfolgen
Existenz- und Eindeutigkeitsbeweis sowie Konstruktion direkt.
Dies ist das Bestmögliche, denn es muss nichts vereinfacht/approximiert werden,
allerdings geht das allermeistens nur in einfachen Spezialfällen.

\textbf{heuristisch}:
Bei einer \begriff{heuristischen Lösung} führt man Versuch-und-Irrtum gemäß einer bestimmen
Strategie durch, um die optimale Lösung durch eine gute Lösung anzunähern.
Das ist vor allem bei Problem der diskreten Optimierung nützlich.
Die Frage ist jedoch, ob die heuristisch gefundene Lösung gegen die Optimallösung konvergiert
und wenn ja, wie schnell.

\textbf{direkt-numerisch}:
Bei einer \begriff{direkt-numerischen Lösung} liefert ein numerischer Algorithmus eine exakte
Lösung (mit Rundungsfehler).
Der Vorteil gegenüber Heuristiken ist, dass die Lösung in jedem Fall erreicht wird
(z.\,B. Simplex-Algorithmus).

\textbf{approximativ-numerisch}:
Bei einer \begriff{approximativ-numerischen Lösung} führt man ein iteratives Näherungsverfahren
für genäherte Gleichungen durch.
Hier ist das Erreichen einer beliebig genauen Approximation sichergestellt.
Allerdings ist die Frage, wie schnell das Verfahren konvergiert
(z.\,B. CG-Verfahren für LGS-Lösungen und Newton-Verfahren für Nullstellen).

\pagebreak

\subsection{%
    Bewertung%
}

\textbf{Validierung}:
Stimmt das Modell?
\begin{itemize}
    \item
    \emph{Vergleich mit Experimenten}:
    Windkanal,
    Laborexperimente an verkleinerten Prototypen
    
    \item
    \emph{A-posteriori-Beobachtungen}:
    Realitätstest (Wetter, Börse),
    Zufriedenheitstest (Verkehr)
    
    \item
    \emph{Plausibilitätstest}:
    Test der Ergebnisse auf Konsistenz zu bestehenden Theorien
    
    \item
    \emph{Modellvergleich}:
    Vergleich mit auf anderen Modellen basierenden Simulationen
\end{itemize}

\textbf{Genauigkeit}:
Wie präzise ist das Modell?
\begin{itemize}
    \item
    \emph{bzgl. der Qualität der Eingabedaten}
    
    \item
    \emph{bzgl. der Fragestellung}
    (z.\,B. Bundestagswahl und 5-\%-Hürde)
    
    \item
    \emph{Sicherheit}
    (Aussagen zu Worst-Case oder Average-Case?)
\end{itemize}

\subsection{%
    Klassifikation von Modellen%
}

\textbf{diskret vs. kontinuierlich}:
\begin{itemize}
    \item
    \emph{diskret}:
    diskrete/kombinatorische Beschreibung
    (binäre/ganzzahlige Größen, Graphen)
    
    \item
    \emph{kontinuierlich}:
    kontinuerliche/reellwertige Beschreibung
    (reelle Zahlen, physikalische Größen, algebraische Gleichungen, DGLs)
\end{itemize}
Dasselbe Phänomen kann aber sowohl diskret als auch kontinuierlich modelliert werden
(z.\,B. Verkehrsfluss in der Stadt).

\linie

\textbf{deterministisch vs. stochastisch}:
\begin{itemize}
    \item
    \emph{deterministisch}
    (z.\,B. Crash-Test)
    
    \item
    \emph{stochastisch}
    (z.\,B. Würfeln)
\end{itemize}
Auch hier kann das Phänomen sowohl deterministisch als auch stochastisch modelliert werden.
Beispiele sind die Wettervorhersage und die Internet-Paketankunft an einer Bedieneinheit.

\linie

\textbf{Betrachtungsebene/Hierarchie}:
Selten gibt es "`ein korrektes Modell"'.
Meistens gibt es eine Modellhierarchie (Wechselspiel aus Aufwand und Genauigkeit),
die man durch schrittweise Verfeinerung des Modells durchläuft.
Welche Auf"|lösung gewählt werden soll, hängt vom gewünschten Resultat und dem erforderlichen
Lösungsaufwand ab.
Beispiele beinhalten die Strömung durch einen Zylinder (1D/2D/3D?) und
die Populationsdynamik
(in den USA rein zeitabhängig als $p(t)$ oder Ost-West-Siederstrom als $p(x, t)$?).

\textbf{Multiskaleneigenschaft}:
Die auftretenden Skalen können meist nicht ohne einen inakzeptablen Genauigkeitsverlust getrennt
werden.
Ein Beispiel sind turbulente Strömungen:
Hier müssten (abhängig von der Viskosität des Fluids) auch in einem großen Gebiet kleinste Wirbel
mitberechnet werden, weil die sich zu größeren Verwirbelungen beitragen können.
Abhilfe schaf"|fen Turbulenzmodelle, die feinskalige Einflüsse in grobe Parameter packen und
eine Mittelung bzgl. Raum und Zeit durchführen.

\pagebreak
