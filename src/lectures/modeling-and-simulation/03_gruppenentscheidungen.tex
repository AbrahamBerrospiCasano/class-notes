\section{%
    Gruppenentscheidungen%
}

Im vorherigen Kapitel ging es um Individualentscheidungen.
Jetzt geht es um Entscheidungen des Kollektivs
(Bundestagswahl, ESC, Jahrgangssprecher, Volksabstimmung, US-Präsidenten"-wahl usw.).
Die Ergebnisse werden oft angezweifelt, das Verfahren sei ungerecht
(gerade von den Verlierern).
Im Folgenden wird axiomatisch vorgegangen, d.\,h. man stellt Forderungen von Eigenschaften für
Entscheidungsverfahren auf und prüft, welche Verfahren sie erfüllen.

\subsection{%
    Relationen%
}

\textbf{Relation}:
Eine \begriff{Relation} $R$ auf einer Menge $X$ ist eine Teilmenge $R \subset X^2$.
Für $(x, y) \in R$ schreibt man auch $xRy$.

\textbf{Eigenschaften}:
Eine Relation $R$ auf $X$ heißt
\begin{itemize}
    \item
    \begriff{reflexiv}, falls $\forall_{x \in X}\; xRx$,

    \item
    \begriff{transitiv}, falls $\forall_{x, y, z \in X}\; [(xRy \land yRz) \Rightarrow xRz]$,

    \item
    \begriff{Quasiordnung}, falls $R$ reflexiv und transitiv ist,

    \item
    \begriff{symmetrisch}, falls
    $\forall_{x, y \in X}\; [xRy \Leftrightarrow yRx]$,

    \item
    \begriff{asymmetrisch}, falls
    $\forall_{x, y \in X}\; [xRy \Rightarrow \lnot yRx]$, und

    \item
    \begriff{konnex}, falls
    $\forall_{x, y \in X}\; [\lnot xRy \Rightarrow yRx]$.
\end{itemize}

\subsection{%
    Präferenzrelationen%
}

\textbf{Rangabbildung}:
Sei $A$ die endliche Menge der Kandidaten.
Eine \begriff{Rangabbildung} ist eine Abbildung $r\colon A \to \natural$,
sodass es ein $k \in \natural$ gibt mit $r$ surjektiv auf $\{1, \dotsc, k\}$.
Dabei soll $r(x) < r(y)$ bedeuten, dass der Wähler $x$ gegenüber $y$ bevorzugt.
$k$ kann kleiner sein als $|A|$, d.\,h. $r$ muss nicht injektiv sein
(Gleichstand ist möglich).

\textbf{Präferenzrelation $\varrho$}:\\
Die von $r$ induzierte \begriff{Präferenzrelation} $\varrho$ auf $A$
ist definiert durch $x\varrho y$, falls $r(x) < r(y)$.\\
Die Menge aller Präferenzrelationen ist
$P_A := \{\varrho \subset A^2 \;|\; \text{$\varrho$ durch Rangabbildung induziert}\}$.

Die von $r$ induzierte Präferenzrelation $\varrho$ ist transitiv und asymmetrisch.

\linie

\textbf{invers-komplementäre Relation $\varrho^\ast$}:\\
Die von $r$ induz. \begriff{invers-kompl. Relation} $\varrho^\ast$ auf $A$
ist definiert durch $x\varrho^\ast y$, falls $r(x) \le r(y)$.
$\varrho^\ast$ ist \begriff{invers-komplementär} zu $\varrho$, d.\,h.
$\forall_{x, y \in A}\; x\varrho y \iff \lnot y\varrho^\ast x$.
Die Menge aller so erhaltenen Relationen ist
$P_A^\ast := \{\varrho^\ast \subset A^2 \;|\; \text{$\varrho^\ast$ invers-komplementär zu einem
$\varrho \in P_A$}\}$.

$\varrho^\ast$ ist immer eine konnexe Quasiordnung.
Man kann zeigen, dass $P_A^\ast$ genau die Menge aller konnexen Quasiordnungen ist.
Die Zuordnung zwischen Rangabbildung und konnexer Quasiordnung ist dabei eindeutig,
man kann daher zwischen den Darstellungen als Rangabbildung, als Relation aus $P_A$ und als
Relation aus $P_A^\ast$ wählen.

\pagebreak

\subsection{%
    Kollektive Auswahlfunktionen und demokratische Grundregeln%
}

Bisher wurde nur ein einzelner Wähler betrachtet.
Jetzt geht es darum, die Präferenzen von $n$ Wählern
$I := \{1, \dotsc, n\}$ zu einer Gesamtentscheidung des Kollektivs zu vereinigen.

\textbf{kollektive Auswahlfunktion}:
Seien $n \in \natural$ und $I := \{1, \dotsc, n\}$.\\
Dann heißt eine Funktion $K\colon P_A^n \to P_A$,
$(\varrho_i)_{i \in I} \mapsto K((\varrho_i)_{i \in I})$ \begriff{kollektive Auswahlfunktion}.

\textbf{demokratische Grundregeln}:
\begin{enumerate}
    \item
    \begriff{totale Definition}:
    $K$ muss total definiert sein.

    \item
    \begriff{Bild in $P_A$}:
    $K$ muss in $P_A$ abbilden.

    \item
    \begriff{\name{Pareto}-Bedingung}:
    $\forall_{(\varrho_1, \dotsc, \varrho_n) \in P_A^n}
    \forall_{x, y \in A}\;
    [(\forall_{i \in I}\; x \varrho_i y) \Rightarrow x \varrho y]$
    mit $\varrho := K(\varrho_1, \dotsc, \varrho_n)$

    \item
    \begriff{Unabhängigkeit von irrelevanten Alternativen}:\\
    $\forall_{(\varrho_1, \dotsc, \varrho_n) \in P_A^n}
    \forall_{(\varrho_1', \dotsc, \varrho_n') \in P_A^n}
    \forall_{x, y \in A}\;
    [(\forall_{i \in I}\; x \varrho_i y \Leftrightarrow x \varrho_i' y) \Rightarrow
    (x \varrho y \Leftrightarrow x \varrho' y)]$\\
    mit $\varrho := K(\varrho_1, \dotsc, \varrho_n)$
    und $\varrho' := K(\varrho_1', \dotsc, \varrho_n')$
\end{enumerate}

\textbf{Erklärung}:
Die erste Bedingung stellt die individuelle Entscheidungsfreiheit sicher
(kein $\varrho_i$ ist verboten).
Die zweite Bedingung stellt sicher, dass $K$ eine "`vernünftige"' Relation liefert.
Die dritte (Pareto-)Bedingung bedeutet, dass wenn $x$ von jedem Wähler gegenüber $y$
bevorzugt wird, auch insgesamt $x$ gegenüber $y$ vorgezogen wird.
Die vierte Bedingung bedeutet, dass wenn die Wahlentscheidungen $(\varrho_1, \dotsc, \varrho_n)$
so zu $(\varrho_1', \dotsc, \varrho_n')$ modifiziert werden, dass sich bei jedem Wähler
bzgl. der Präferenz zweier Kandidaten $x$ und $y$ nicht ändert, auch insgesamt sich bzgl. der
Präferenz dieser Kandidaten nichts ändert.

Die hier wiedergegebene Pareto-Bedingung ist die \begriff{schwache \name{Pareto}-Bedingung}.\\
Bei der \begriff{starken \name{Pareto}-Bedingung} soll
$[(\exists_{i \in I}\; x \varrho_i  \land \forall_{y \in I}\; x \varrho_i^\ast y)
\Rightarrow x \varrho y]$ gelten.

\pagebreak

\subsection{%
    Entscheidungsverfahren%
}

\textbf{externer Diktator}:
Beim \begriff{externen Diktator} sei $\varrho_E \in P_A$ fest
und $K_{\varrho_E}^E(\varrho_1, \dotsc, \varrho_n) :\equiv \varrho_E$ konstant.

Der externe Diktator erfüllt alle Bedingungen außer die Pareto-Bedingung.

\linie

\textbf{interner Diktator}:
Beim \begriff{internen Diktator} sei $d \in I$ fest
und $K_d^I(\varrho_1, \dotsc, \varrho_n) :\equiv \varrho_d$
die Projektion auf die $d$-te Komponente.

Der interne Diktator erfüllt alle demokratischen Grundregeln.

\linie

\textbf{\name{Condorcet}-Verfahren}:
Beim \begriff{\name{Condorcet}-Verfahren} (auch \begriff{Mehrheitsentscheid})
ist $x \varrho y$, falls $|\{i \in I \;|\; x \varrho_i y\}| > |\{i \in I \;|\; y \varrho_i x\}|$.

\begin{floatingfigure}[r]{54mm}
    \footnotesize\vspace{-4mm}
    \begin{tabular}{p{12mm}*{3}{>{\centering\arraybackslash}m{8mm}}}
        \toprule

        $i$ & $r_i(x)$ & $r_i(y)$ & $r_i(z)$\\

        \midrule

        $1$ & $1$ & $2$ & $3$\\
        $2$ & $3$ & $1$ & $2$\\
        $3$ & $2$ & $3$ & $1$\\

        \bottomrule
    \end{tabular}
\end{floatingfigure}
Das Condorcet-Verfahren erfüllt die Grundregeln \emph{(1)}, \emph{(3)} und \emph{(4)}.
Es gilt allerdings i.\,A. $\varrho \notin P_A$,
ein Gegenbeispiel sieht ist für $n = 3$ rechts darstellt:
Es gelten $x \varrho y$, $y \varrho z$ und $z \varrho x$.
$\varrho$ ist aber nicht transitiv, da $\lnot x \varrho x$ gilt.

\vspace{2mm}
\linie

\textbf{Einstimmigkeit}:
Beim Verfahren der \begriff{Einstimmigkeit} ist $x \varrho y$, falls
$\forall_{i \in I}\; x \varrho_i y$.

Das Verfahren berücksichtigt den \begriff{Minimalkonsens der Gesamtheit},
d.\,h. wenn ein einziger Wähler $i$ einen Kandidaten $y$ mindestens so schätzt wie $x$,
dann gilt das auch für die kollektive Entscheidung:
$\exists_{i \in I}\; y \varrho_i^\ast x
\iff \exists_{i \in I}\; \lnot x \varrho_i y
\iff \lnot x \varrho y
\iff y \varrho^\ast x$.
In der Praxis gilt bei großem $n$ fast immer $y \varrho^\ast x$ und daher fast nie $x \varrho y$
(Entscheidungsschwäche).

\begin{floatingfigure}[r]{54mm}
    \footnotesize\vspace{-4mm}
    \begin{tabular}{p{12mm}*{3}{>{\centering\arraybackslash}m{8mm}}}
        \toprule

        $i$ & $r_i(x)$ & $r_i(y)$ & $r_i(z)$\\

        \midrule

        $1$ & $1$ & $2$ & $3$\\
        $2$ & $3$ & $1$ & $2$\\

        \bottomrule
    \end{tabular}
\end{floatingfigure}
Das Condorcet-Verfahren erfüllt die Grundregeln \emph{(1)}, \emph{(3)} und \emph{(4)}.
Es gilt allerdings i.\,A. $\varrho \notin P_A$,
ein Gegenbeispiel sieht ist rechts darstellt:
Es gilt $\varrho = \{(y, z)\}$,
also $z \varrho^\ast x$ und $x \varrho^\ast y$, aber
$\lnot z \varrho^\ast y$.
$\varrho^\ast$ ist damit nicht transitiv, was $\varrho^\ast \notin P_A^\ast$ bedeutet.

\linie

\textbf{Rangaddition}:
Bei der \begriff{Rangaddition} ist
$x \varrho y$, falls $\sum_{i \in I} r_i(x) < \sum_{i \in I} r_i(y)$.

Die Summe $\sum_{i \in I} r_i$ ist i.\,A. keine Rangabbildung ("`Lücken"' im Bild vorhanden),
jedoch ist die induzierte Relation in $P_A$.

Die Rangaddition erfüllt alle demokratischen Grundregeln, außer \emph{(4)}.
Ein Gegenbeispiel sieht wie folgt aus:
\begin{floatingfigure}[r]{115mm}
    \footnotesize\vspace{-4mm}
    \begin{tabular}{p{12mm}*{3}{>{\centering\arraybackslash}m{8mm}}}
        \toprule

        $i$ & $r_i(x)$ & $r_i(y)$ & $r_i(z)$\\

        \midrule

        $1$ & $1$ & $2$ & $3$\\
        $2$ & $3$ & $1$ & $2$\\

        \midrule

        $\sum_{i \in I} r_i$ & $4$ & $3$ & $5$\\

        \bottomrule
    \end{tabular}
    \qquad
    \begin{tabular}{p{12mm}*{3}{>{\centering\arraybackslash}m{8mm}}}
        \toprule

        $i$ & $r_i'(x)$ & $r_i'(y)$ & $r_i'(z)$\\

        \midrule

        $1$ & $1$ & $2$ & $3$\\
        $2$ & $2$ & $1$ & $3$\\

        \midrule

        $\sum_{i \in I} r_i'$ & $3$ & $3$ & $6$\\

        \bottomrule
    \end{tabular}
\end{floatingfigure}
Bezüglich $x$ und $y$ hat sich die Präferenz in den $\varrho_i$ verglichen mit den $\varrho_i'$
nicht geändert.\\
Es gilt aber $y \varrho x$ und $\lnot y \varrho' x$.

\subsection{%
    Unmöglichkeitssatz von Arrow%
}

Dass nur der interne Diktator alle Grundregeln erfüllt, ist kein Zufall.

\textbf{Satz (Unmöglichkeitssatz von \name{Arrow})}:
Seien $|A| > 2$ und $K\colon P_A^n \to P_A$ eine kollektive Auswahlfunktion,
die die demokratischen Grundregeln \emph{(1)} -- \emph{(4)} erfüllt.
Dann gibt es einen \begriff{Diktator}, d.\,h.
$\exists_{d \in I}
\forall_{(\varrho_1, \dotsc, \varrho_n) \in P_A^n}
\forall_{x, y \in A}\;
[x \varrho_d y \Rightarrow x \varrho y]$
mit $\varrho := K(\varrho_1, \dotsc, \varrho_n)$.

\pagebreak
