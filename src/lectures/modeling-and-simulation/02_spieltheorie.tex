\chapter{%
    Spieltheorie%
}

\section{%
    Strategische Spiele%
}

Es folgen zwei Beispiele für \begriff{strategische Spiele}.

\textbf{Gefangenendilemma}:
Zwei Bankräuber $A, B$ wurden gefasst.
Man kann ihnen die Tat allerdings nicht nachweisen, man könnte sie nur wegen unerlaubten
Waf"|fenbesitzes zu jeweils 3 Jahren Gefängnis verurteilen.
Nun bietet der Staatsanwalt beiden Bankräubern getrennt an, ein Geständnis abzulegen,
um in den Genuss einer Kronzeugenregelung zu kommen.
Dazu darf der jeweils andere aber nicht gestehen.
Ist das der Fall, so kommt der Kronzeuge nur 1 Jahr ins Gefängnis und der andere 9 Jahre.
Gestehen jedoch beide, dann kommen beide für 7 Jahre ins Gefängnis.
$A$ und $B$ haben keine Möglichkeit, sich abzusprechen.
Wie sollen sie verfahren?

\textbf{Kampf der Geschlechter}:
Zwei Partner $A, B$ befinden sich an unterschiedlichen Orten.
$A$ will ins Stadion und $B$ will einkaufen.
Für beide gilt, dass sie am liebsten mit ihrem Partner am Lieblingsort wären.
Am zweitliebsten ist beiden, mit ihrem Partner am anderen Ort zu sein.
Am wenigsten gerne sind sie allein.
$A$ und $B$ können sich nicht absprechen.
Wo sollen sie hingehen?

\linie

\textbf{Entscheidungsmodell}:
Für solche Situationen ist ein allgemeines Entscheidungsmodell erforderlich.
Die Entscheidungen können erfolgen unter
\begin{itemize}
    \item
    \begriff{Gewissheit}
    (alle Bedingungen/Konsequenzen sind bekannt),

    \item
    \begriff{Risiko}
    (die Bedingungen sind nur mit bestimmter Wahrscheinlichkeit bekannt) und

    \item
    \begriff{Ungewissheit}
    (die Bedingungen sind unbekannt).
\end{itemize}

\section{%
    Modell für Spiele%
}

Im Folgenden beschränkt man sich auf höchstens zwei Spieler $A$ und $B$
mit $X \in \{A, B\}$.

\textbf{Spiele}:
Beide können ihre Handlungen (\begriff{Strategien}) aus Mengen $S_A$ bzw. $S_B$ wählen.
Bei einem \begriff{endlichen Spiel} sind $S_A := \{a_1, \dotsc, a_{n_A}\}$ und
$S_B := \{b_1, \dotsc, b_{n_B}\}$ endlich.
Ein \begriff{Spielzugpaar} ist ein Element aus $S := S_A \times S_B$.
Eine Funktion $u_X\colon S \to \real$ heißt \begriff{Auszahlungsfunktion}.
Bei einem endlichen Spiel erhält man \begriff{Nutzenmatrizen}
$U^X \in \real^{n_A \times n_B}$ mit
$U_{i,j}^X := u_X(a_i, b_j)$.
Beide Matrizen werden manchmal zur \begriff{Bimatrix}
$U^{AB} \in (\real^2)^{n_A \times n_B}$ mit Einträgen $U_{i,j}^{AB} = (U_{i,j}^A, U_{i,j}^B)$
zusammengefasst.

\textbf{Nullsummenspiel}:
Spiele mit $\forall_{s \in S}\; u_A(s) = -u_B(s)$ heißen \begriff{Nullsummenspiele}.
Bei einem endlichen Spiel ist dann $U^A = -U^B$ und es genügt eine Nutzenmatrix $U := U^A$.
Man spricht dann vom \begriff{Gewinnspieler} $A$ und vom \begriff{Verlustspieler} $B$.

\textbf{strategische Normalform}:
Der Strategieraum $S$ zusammen mit den Auszahlungsfunktionen $u_X$ heißt
\begriff{strategische Normalform}.

\textbf{Spiel mit vollständiger Information}:
Bei einem \begriff{Spiel mit vollständiger Information} kennen $A$ und $B$ beide die
vollständige Auszahlungsfunktion.

Im Folgenden geht es nur noch um endliche Spiele mit vollständiger Information.

\pagebreak

\section{%
    Einpersonenspiele%
}

\textbf{Einpersonenspiel}:
Bei einem \begriff{Einpersonenspiel} oder \begriff{Spiel ohne Annahme über Gegner} trifft
$B$ die Wahl unabhängig von $A$ (z.\,B. Wetter).
Es gibt daher nur eine Nutzenmatrix $U = U^A$.

\textbf{Spiel bei Gewissheit}:
Wenn $A$ weiß, dass $B$ $b_j$ spielt, dann spielt $A$ natürlich $a_{\widehat{i}}$ mit\\
$\widehat{i} \in \{1, \dotsc, n_A\}$, sodass $U_{\widehat{i},j} = \max_i U_{i,j}$.

\textbf{Spiel bei Risiko}:
Wenn $A$ keine Information über die Wahl von $B$ hat, dann hängt die Wahl von $A$ von dessen
Mentalität ab.
\begin{itemize}
    \item
    \emph{risikobereiter Spieler}:
    Ermögliche den maximalen Gewinn, d.\,h.\\
    wähle $\widehat{i} \in \{1, \dotsc, n_A\}$
    mit $\max_j U_{\widehat{i},j} = \max_i \max_j U_{i,j}$.

    \item
    \emph{vorsichtiger Spieler}:
    Maximiere den garantierten Gewinn, d.\,h.\\
    wähle $\widehat{i} \in \{1, \dotsc, n_A\}$
    mit $\min_j U_{\widehat{i},j} = \max_i \min_j U_{i,j}$.
\end{itemize}

\textbf{weitere Strategien}:
Man kann für die Entscheidung von $B$ auch eine Wahrscheinlichkeitsverteilung ansetzen,
z.\,B. nach dem \begriff{Prinzip des unzureichenden Grunds} die Gleichverteilung
$\PP(b_j) = \frac{1}{n_B}$.
Das Ziel ist es dann, den Erwartungswert zu maximieren.

\section{%
    Zweipersonenspiele%
}

Nun spielen $A$ und $B$ gleichzeitig, es gibt also einen Interessenskonflikt.
Zur Abkürzung verwendet man die Schreibweisen $-A := B$ und $-B := A$ für den jeweiligen Gegner.

\textbf{Reaktionsabbildung}:
Die \begriff{Reaktionsabbildung} ist definiert durch\\
$r_X\colon S_{-X} \to \P(S_X)$,
$r_X(y) := \{\widehat{x} \in S_X \;|\; u_X(\widehat{x}, y) = \max_{x \in S_X} u_X(x, y)\}$.

$r_X(y)$ ist die Menge der Spielzüge, die optimal sind, wenn man weiß, dass der Gegner $y$ spielt.
Bei unendlichen Strategiemengen muss das Maximum nicht notwendigerweise existieren.

\textbf{Gesamt-Reaktionsabbildung}:
Die \begriff{Gesamt-Reaktionsabbildung} ist definiert durch\\
$r\colon S \to \P(S)$,
$r(a, b) := r_A(b) \times r_B(a)$.

$r(a, b)$ ist die Menge aller Strategiepaare, bei denen $A$ optimal auf $b$ und $B$ optimal auf
$a$ reagiert.

\linie

\textbf{Beispiel}:
Für $U^{AB} := \smallpmatrix{(0, 20) & (30, 20)\\(10, 0) & (10, 10)}$ ist
$r(a_1, b_1) = \{(a_2, b_1), (a_2, b_2)\}$,\\
$r(a_1, b_2) = \{(a_1, b_1), (a_1, b_2)\}$,
$r(a_2, b_1) = \{(a_2, b_2)\}$ und
$r(a_2, b_2) = \{(a_1, b_2)\}$.

\textbf{Beispiel}:
Für $S_A := [0, 1] =: S_B$, $u_A(a, b) := 2ab - a - b$ und $u_B(a, b) := -u_A(a, b)$ ist
$r_A(b) = \{0\}$ für $b < \frac{1}{2}$,
$r_A(b) = [0, 1]$ für $b = \frac{1}{2}$ und
$r_A(b) = \{1\}$ für $b > \frac{1}{2}$,
wobei $r_B(a) = r_A(a)$.

\linie

\textbf{dominante Strategie}:
Eine \begriff{dominante Stratgie} für $X$
ist $x \in S_X$ mit $\forall_{y \in S_Y}\; x \in r_X(y)$.

\textbf{Beispiel}:
Beim Gefangenendilemma ist $U^{AB} := \smallpmatrix{(-7, -7) & (-1, -9)\\(-9, -1) & (-3, -3)}$.\\
Wegen $r_A(b_1) = \{a_1\} = r_A(b_2)$ ist $a_1$ eine dominante Stratgie für $A$
und analog $b_1$ eine dominante Strategie für $B$.
Deswegen müssen aber in der Realität $A$ und $B$ nicht immer $a_1$ bzw. $b_1$ spielen
(z.\,B. Angst, dass sich $-X$ rächen).
Ist das so, dann sollte die Nutzenmatrix geändert werden.

\linie
\pagebreak

\textbf{\name{Nash}-Gleichgewicht}:
Ein \begriff{\name{Nash}-Gleichgewicht (Gleichgewichtspunkt)} ist ein Spielzugpaar
$\widehat{s} := (\widehat{a}, \widehat{b}) \in S$ mit
$\widehat{a} \in r_A(\widehat{b})$ und $\widehat{b} \in r_B(\widehat{a})$
(d.\,h. $\widehat{s} \in r(\widehat{s})$).

Haben $A$ und $B$ ihre Strategie vorher vereinbart, dann ist eine alleinige Abweichung sinnlos.

\textbf{alternative Charakterisierung}:
Sei $\widehat{s} := (\widehat{a}, \widehat{b}) \in S$.
Dann ist $\widehat{s}$ ein Nash-Gleichgewicht
\begin{itemize}
    \item
    genau dann, wenn
    $\forall_{b \in S_B}\; u_B(\widehat{a}, \widehat{b}) \ge u_B(\widehat{a}, b)$ und
    $\forall_{a \in S_A}\; u_A(\widehat{a}, \widehat{b}) \ge u_A(a, \widehat{b})$, und

    \item
    für ein Nullsummenspiel genau dann, wenn
    $\forall_{(a, b) \in S}\;
    u(\widehat{a}, b) \ge u(\widehat{a}, \widehat{b}) \ge u(a, \widehat{b})$.\\
    In diesem Fall heißt $\widehat{s}$ auch \begriff{Sattelpunkt}.
\end{itemize}

\linie

\textbf{Beispiel}:
Bei der Schlacht in der Bismarck-See wollten die Japaner einen Nachschubkonvoi nach Neuguinea
schicken.
Die Amerikaner wollten das verhindern.
Zwei mögliche Routen (Nord- und Südroute) standen zur Auswahl,
die jeweils $3$ Tage Fahrt lang waren.
Bei der Nordroute war die Sicht schlecht, sodass amerikanische Aufklärer einen ganzen Tag zur
Aufklärung benötigten, bevor sie am nächsten Tag mit der Bombardierung beginnen konnten
(wenn die Japaner die Nordroute gewählt haben).
Hatten Japaner und Amerikaner die Südroute gewählt, so war die Sicht für die Amerikaner so gut,
dass sie schon am selben Tag bombadieren konnten.
Wenn $A$ die USA und $B$ Japan ist und man als Auszahlung die Tage mit Bombardierung festlegt,
so ergibt sich die Nutzenmatrix $U = \smallpmatrix{2 & 2\\1 & 3}$.
Weil $(a_1, b_1)$ ein Sattelpunkt ist, ist es für beide optimal, die Nordroute zu wählen
(was auch so eingetreten ist).

\section{%
    Gemischte Strategien%
}

\textbf{Problem}:
Nicht jedes Spiel hat nur einen Gleichgewichtspunkt.
Beim Kampf der Geschlechter kann man z.\,B.
$U^{AB} = \smallpmatrix{(20, 10) & (0, 0)\\(0, 0) & (10, 20)}$ ansetzen,
d.\,h. $(a_1, b_1)$ und $(a_2, b_2)$ sind Gleichgewichtspunkte.
Welcher der Punkte soll ohne Absprache gewählt werden?

Analog ist es beim Nullsummenspiel $U = \smallpmatrix{5 & -5\\-5 & 5}$.
Hier ist ein alleiniges Abweichen immer vorteilhaft, wenn der Gegner bei seiner Strategie bleibt.

\linie

\textbf{gemischte Strategien}:
Bei einem Spiel mit \begriff{gemischten Strategien} geht man von einem endlichen Spiel mit
$S := \{a_1, \dotsc, a_{n_A}\} \times \{b_1, \dotsc, b_{n_B}\}$ und Nutzenmatrizen $U^A, U^B$
aus und setzt\\
$\widetilde{S}_X := \{p_A = (p_{X,1}, \dotsc, p_{X,n_X})^\tp \in [0, 1]^{n_X} \;|\;
\sum_{i=1}^{n_X} p_{X_i} = 1\}$ sowie\\
$\widetilde{u}_X(p_A, p_B) := \EE(u_X) = p_A^\tp \cdot U^X \cdot p_B$
(d.\,h. wählt man $p_A \in \widetilde{S}_A$, dann spielt $A$ mit Wahrscheinlichkeit $p_{A,i}$
die Strategie $a_i$ für $i = 1, \dotsc, n_A$).

\textbf{Beispiel}:
Geht man vom Kampf der Geschlechter aus, so erhält man mit der Identifizierung
$\widetilde{S}_X := [0, 1]$ und $p_X := p_{X,1}$ (geht wegen $p_{X,2} = 1 - p_{X,1}$)
die Nutzenfunktionen\\
$\widetilde{u}_A(p_A, p_B) = 20 p_A p_B + 10 (1 - p_A) (1 - p_B)
= 30 p_A p_B - 10 p_A - 10 p_B + 10$ und\\
$\widetilde{u}_B(p_A, p_B) = 10 p_A p_B + 20 (1 - p_A) (1 - p_B)
= 30 p_A p_B - 20 p_A - 20 p_B + 20$.\\
Man erhält
$\widetilde{r}_A(p_B) = \{0\}$ für $p_B < \frac{1}{3}$,
$\widetilde{r}_A(p_B) = [0, 1]$ für $p_B = \frac{1}{3}$ und
$\widetilde{r}_A(p_B) = \{1\}$ für $p_B > \frac{1}{3}$
(analog $\widetilde{r}_B(p_A)$ mit $p_A = \frac{2}{3}$).
Gleichgewichtspunkte sind damit $(0, 0)$, $(1, 1)$ und $(\frac{2}{3}, \frac{1}{3})$.

\pagebreak
