\chapter{%
    Scheduling%
}

Im Folgenden geht es um \begriff{Scheduling}, d.\,h. die möglichst optimale Zuordnung von
Ressourcen (Personal, Zeit, Aufträge, Maschinen usw.) zu Aufgaben mit
Abhängigkeiten (Reihenfolge, begrenzte Ressourcen usw.).
Beispiele sind Projektplanung, Stundenplanerstellung und die Produktion in einer Fabrik.

\section{%
    Prozess-Scheduling%
}

\textbf{erstes Modell}:
Ein \begriff{Prozess} bestehe aus $n$ \begriff{Aufträgen} $A_1, \dotsc, A_n$.
Jeder Auftrag $A_i$ benötigt eine (deterministische) Bearbeitungszeit $t_i \ge 0$.
Ein \begriff{Zeitplan} ist eine Abb. $\{A_1, \dotsc, A_n\} \to [0, \infty)^n$,\\
$A_i \mapsto s_i$, die jedem Auftrag $A_i$ eine Startzeit $s_i$ zuordnet.
Ein Zeitplan definiert die \begriff{Fertig"-stellungszeiten} $c_i := s_i + t_i$
(die Aufträge werden also am Stück abgearbeitet).
Die \begriff{Kosten} eines Zeitplans sind gegeben durch $\max_{i=1,\dotsc,n} c_i$.

\textbf{Präzendenzrelation}:
Es sei eine \begriff{Präzendenzrelation} "`$\to$"' gegeben,
wobei $A_i \to A_j$ bedeutet, dass $A_j$ von $A_i$ abhängt.
Ein Zeitplan heißt \begriff{zulässig}, falls
$\forall_{i,j=1,\dotsc,n}\; [A_i \to A_j \;\Rightarrow\; c_i \le s_j]$.

"`$\to$"' kann durch die transitive Hülle ersetzt werden, ohne dass sich die zulässigen
Zeitpläne ändern.
Das Ziel ist es nun, einen zulässigen Zeitplan mit minimalen Kosten zu finden.

\linie

\textbf{Scheduling-Problem als Graph}:
Das Scheduling-Problem kann als Graph $G := (V, E)$ modelliert werden mit Knoten
$V := \{A_S, A_1, \dotsc, A_n, A_E\}$ und Kanten
$E := \{(A_i, A_j) \;|\; A_i \to A_j\}$\\
$\cup\; \{(A_S, A_i) \;|\; \text{$A_i$ hat keine eingehende Kante}\} \cup
\{(A_i, A_E) \;|\; \text{$A_i$ hat keine ausgehende Kante}\}$,\\
wobei $S := 0$, $t_S := 0$, $E := n+1$ und $t_E := 0$.

\textbf{Pfad}:
Ein \begriff{Pfad} ist eine Folge von Aufträgen $A_{i_1}, \dotsc, A_{i_k}$ mit
$A_{i_1} \to \dotsb \to A_{i_k}$.
Die Länge des Pfades $A_{i_1} \to \dotsb \to A_{i_k}$ ist
$\sum_{j=1}^k t_{i_j}$.
In jedem zulässigen Zeitplan gilt $c_{i_k} \ge s_{i_1} + \sum_{j=1}^k t_{i_j}$
für jeden Pfad $A_{i_1} \to \dotsb \to A_{i_k}$
(weil $c_{i_k} = s_{i_k} + t_{i_k} \ge c_{i_{k-1}} + t_{i_k} \ge s_{i_1} + \sum_{j=1}^k t_{i_j}$).

\textbf{Zyklen}:
Ein \begriff{Zyklus} ist ein Pfad $A_{i_1} \to \dotsb \to A_{i_k} \to A_{i_1}$.
In jedem zulässigen Zeitplan gilt $s_{i_1} \ge c_{i_k} \ge s_{i_1} + \sum_{j=1}^k t_{i_j}$,
also $t_{i_1} = \dotsb = t_{i_k} = 0$.
OBdA kann man also annehmen, dass der Graph zyklenfrei ist.
Er ist dann ein DAG (gerichteter azyklischer Graph).

\linie
\pagebreak

\textbf{Konstruktion eines opt. Zeitplans für DAGs}:
Ein optimaler Zeitplan kann für DAGS wie folgt konstruiert werden.
Seien $s_i'$ die \begriff{Vorlaufzeit} (frühest möglicher Startpunkt) und
$c_i' := s_i' + t_i$ die \begriff{Fertigstellungszeit} von $A_i$.
\begin{enumerate}
    \item
    Setze $s_S' := c_S' := 0$.

    \item
    Solange es noch unbearbeitete Knoten gibt, wiederhole Folgendes:
    \begin{enumerate}
        \item
        Wähle einen unbearbeiteten Knoten $A_i$, bei dem alle $A_j$ mit $A_j \to A_i$
        bereits bearbeitet wurden
        (der Knoten existiert aufgrund der Zyklenfreiheit).

        \item
        Setze $s_i' := \max\{c_j' \;|\; A_j \to A_i\}$ und $c_i' := s_i' + t_i$.
    \end{enumerate}
\end{enumerate}
DAGs können per modifizierte Tiefensuche topologisch so sortiert werden, dass
$A_i \to A_j$ nur für $i < j$ gelten kann.
In diesem Fall kann man die Knoten in der Reihenfolge $1, \dotsc, n + 1$ bearbeiten.

\textbf{Eigenschaften des Zeitplans}:
Kein Auftrag kann früher gestartet werden (insb. $A_E$).
Wenn man einen Auftrag $A_i$ später startet, kann das $c_E'$ nicht verbessern.

\textbf{alternativ über Restlaufzeit}:
Seien $c_i''$ die späteste Fertigstellungszeit von $A_i$, sodass die optimale
Gesamtfertigstellungszeit $c_E'$ erreicht wird, und $s_i'' := c_i'' - t_i$ die späteste
Startzeit.
Dann ist $c_E' - s_i''$ die \begriff{Restlaufzeit} von Auftrag $A_i$.
Die Berechnung erfolgt analog zur Vorlaufzeit, außer dass man mit
$c_E'' := s_E'' := c_E'$ startet und umgekehrt vorgeht.

\linie

\textbf{kritischer Knoten}:
Ist $s_i' = s_i''$, dann ist $A_i$ ein \begriff{kritischer Knoten}
und es gilt $s_i = s_i' = s_i''$ für jeden optimalen Zeitplan.
Jeder kritische Knoten liegt auf einem \begriff{kritischem Pfad} von $A_S$ nach $A_E$, der nur aus
kritischen Knoten besteht.
Für jede Kante $A_k \to A_\ell$ eines kritischen Pfades gilt $c_k' = s_\ell''$.

\textbf{Schlupf}:
Ist $s_i' < s_i''$, so heißt die Dif"|ferenz $s_i'' - s_i'$ \begriff{Schlupf} von $A_i$.
Für jeden optimalen Zeitplan gilt $s_i \in [s_i', s_i'']$.

\textbf{Kritischer-Pfad-Methode}:
Es gibt immer mindestens einen kritischen Pfad.
Wenn es mehrere gibt, so haben sie dieselbe Länge.
Die Länge eines kritischen Pfads ist eine untere Schranke für $c_E$ für jeden zulässigen Zeitplan.
Das Vorgehen heißt \begriff{Kritischer-Pfad-Methode (CPM)}.
Werkzeuge sind z.\,B. Gantt-Diagramme und Netzpläne.
Die Optimierung von Aufträgen setzt üblicherweise beim kritischem Pfad an.

\linie

\begin{floatingfigure}[r]{67mm}
    \vspace{-8mm}
    \begin{align*}
        \xymatrix@R=3mm@C=6mm{
            &
            *=<2em>[Fo]{1^{[3]}}\ar[r]&
            *=<2em>[Fo]{3^{[2]}}\ar[r]\ar[rd]&
            *=<2em>[Fo]{4^{[3]}}\ar[rd]
            \\
            %
            *=<2em>[Fo]{S^{[0]}}\ar[ru]\ar[r]\ar[rd]&
            *=<2em>[Fo]{2^{[2]}}\ar[ru]&&
            *=<2em>[Fo]{5^{[2]}}\ar[r]&
            *=<2em>[Fo]{E^{[0]}}\\
            %
            &
            *=<2em>[Fo]{6^{[4]}}\ar[r]&
            *=<2em>[Fo]{7^{[4]}}\ar@/_/[rru]
        }
    \end{align*}
    \begin{tabular}{p{5mm}*{9}{p{2mm}}}
        \toprule

        $i$ & $0$ & $1$ & $2$ & $3$ & $4$ & $5$ & $6$ & $7$ & $8$ \\

        \midrule

        $t_i$ & $0$ & $3$ & $3$ & $2$ & $3$ & $2$ & $4$ & $4$ & $0$\\

        \midrule

        $s_i'$ & $0$ & $0$ & $0$ & $3$ & $5$ & $5$ & $0$ & $4$ & $8$\\
        $c_i'$ & $0$ & $3$ & $2$ & $5$ & $8$ & $7$ & $4$ & $8$ & $8$\\

        \midrule

        $s_i''$ & $0$ & $0$ & $1$ & $3$ & $5$ & $6$ & $0$ & $4$ & $8$\\
        $c_i''$ & $0$ & $3$ & $3$ & $5$ & $8$ & $8$ & $4$ & $8$ & $8$\\

        \bottomrule
    \end{tabular}
\end{floatingfigure}
\textbf{Beispiel}:
Rechts sind die Bedingungen gegeben durch
$\{A_1 \to A_3, A_2 \to A_3, A_3 \to A_4, A_3 \to A_5, A_6 \to A_7\}$ und
die Bearbeitungszeiten stehen in eckigen Klammern.
In der Tabelle stehen die Werte von $s_i', c_i', s_i'', c_i''$, wenn man den Algorithmus oben
anwendet.
Wie man leicht sieht, gibt es hier zwei kritische Pfade,
nämlich $A_S \to A_1 \to A_3 \to A_4 \to A_E$, $A_S \to A_6 \to A_7 \to A_E$.

\pagebreak

\section{%
    Job-Shop-Probleme%
}

Das Modell soll nun so erweitert werden, dass Ressourcen beschränkt sind,
d.\,h. es können nicht mehr beliebig viele Aufträge parallel abgearbeitet werden.

\textbf{Job-Shop-Problem}:
Es gibt $n$ Aufträge $A_1, \dotsc, A_n$
und $m$ Maschinen $1, \dotsc, m$.
Jeder Auftrag $A_i$ zerfällt nun in $n_i$ Teilaufträge $A_{i,j}$, $j = 1, \dotsc, n_i$,
wobei ein Teilauftrag $A_{i,j}$ die Zeit $t_{i,j}$ und
die Maschine $m_{i,j} \in \{1, \dotsc, m\}$ zur Bearbeitung benötigt.
Pro Maschine darf immer nur ein Teilauftrag gleichzeitig bearbeitet werden.
Zur Vereinfachung wird einschränkend angenommen, dass für jeden Auftrag $A_i$
jede Maschine nur von höchstens einem Teilauftrag $A_{i,j}$ benötigt wird
(also $m_{i,j} \not= m_{i,j'}$ für $j \not= j'$).

\textbf{Flow-Shop-Modell}:
Bei einem \begriff{Flow-Shop-Modell} werden die Maschinen von den Teilaufträgen in gleicher
Reihenfolge benötigt.

\textbf{Matrixnotation}:
Mit $A_i = \smallpmatrix{m_{i,1} & \dots & m_{i,n_i} \\ t_{i,1} & \dots & t_{i,n_i}}$
wird das Problem vollständig beschrieben.

\textbf{Zeitplan}:
Ein \begriff{Zeitplan} ist eine Abbildung $A_{i,j} \mapsto s_{i,j}$ für
$i = 1, \dotsc, n$, $j = 1, \dotsc, n_i$, wobei $s_{i,j} \ge 0$.
Der Zeitplan heißt \begriff{zulässig}, falls
\begin{itemize}
    \item
    kein Teilauftrag $A_{i,j}$ gestartet wird, bevor der Vorgänger $A_{i,j-1}$ beendet ist, und

    \item
    zu keinem Zeitpunkt mehrere Teilaufträge auf derselben Maschine angesetzt sind.
\end{itemize}
Gesucht ist ein optimaler Zeitplan hinsichtlich der spätesten Fertigstellungszeit.

\linie

\textbf{Präzedenzgraph}:
Um die Abhängigkeiten in einem Graph zu modellieren, erstellt man wieder einen
\begriff{Präzedenzgraphen}, wobei die Teilaufträge $A_{i,j}$ zusammen mit $A_S$ und $A_E$
die Knoten sind.

\textbf{Konjunktivkanten}:
Die Reihenfolge innerhalb von Aufträgen wird durch die \begriff{Konjunktiv"-kanten}
$A_{i,j-1} \to A_{i,j}$ für $i = 1, \dotsc, n$, $j = 2, \dotsc, n_i$
sowie $A_S \to A_{i,1}$ und $A_{i,n_i} \to A_E$ für $i = 1, \dotsc, n$ modelliert.

\textbf{Disjunktivkanten}:
Die Abhängigkeiten mit den Maschinen modelliert man mit \begriff{Disjunktiv"-kanten}:
Betrachte für $k \in \{1, \dotsc, m\}$ die Teilaufträge $M(k) := \{A_{i,j} \;|\; m_{i,j} = k\}$,
die Maschine $k$ benötigen.
Dann dürfen sich die Bearbeitungszeiten für Teilauftragspaare $A_{i,j}, A_{i',j'} \in M(k)$
nicht überlappen.
Ein zulässiger Zeitplan muss deswegen eine der beiden Präzendenzkanten $A_{i,j} \to A_{i',j'}$ oder
$A_{i',j'} \to A_{i,j}$ auswählen.
Weil aber nicht im Voraus bekannt ist, welche Kante am besten gewählt werden soll,
fügt man zunächst beide Kanten als Disjunktivkanten ein und
ein Optimierungsalgorithmus wählt dann eine Kante aus.

\textbf{Disjunktivkanten-Belegung}:
Eine \begriff{Disjunktivkanten-Belegung (DKB)} ist eine Auswahl genau einer Kante aus jedem Paar
von Disjunktivkanten.
Sie heißt \begriff{zulässig}, falls der entstehende Präzedenzgraph zyklenfrei ist
(muss nicht notwendigerweise gelten).

Wenn eine zulässige DKB gegeben ist, dann kann ein optimaler Zeitplan
mit der Kritischer-Pfad-Methode bestimmt werden.
Es gibt immer eine zulässige DKB, die folgendermaßen bestimmt werden kann:
\begin{enumerate}
    \item
    Starte $A_{i,j}$, wenn $A_{i,j-1}$ beendet und $m_{i,j}$ frei ist.

    \item
    Kommen mehrere Teilaufträge für eine Maschine in Frage, wähle eine aus.
\end{enumerate}

Die Ermittlung einer optimalen DKB ist schwierig:
Gibt es $k$ Disjunktivkanten, so gibt es $2^k$ DKBs und es müssen $2^k$ CPE-Läufe durchgeführt
werden.
Für große Probleme ist das unrealistisch.
Das Problem kann, wie viele Probleme aus der diskreten Optimierung, i.\,A. nicht in unabhängige
Teilprobleme zerlegt werden.
Weil auch Branch-and-Bound zu teuer ist, müssen Heuristiken verwendet werden
(z.\,B. \begriff{Shifting Bottleneck}), d.\,h. man gibt die Optimalität auf.

\pagebreak

\section{%
    Stochastisches Scheduling%
}

Dass die Bearbeitungszeit von Aufträgen deterministisch ist, ist unrealistisch.
Vielmehr sind verschieden lange Verzögerungen mit unterschiedlichen Wahrscheinlichkeiten möglich.
Zur Vereinfachung seien die Ressourcen wieder unbeschränkt,
d.\,h. es werden Job-Shop-Probleme betrachtet.

Die Bearbeitungszeiten der Aufträge $A_i$ sind nicht mehr deterministisch, sondern
Zufallsvariablen $T_i$.
Die optimale Gesamtfertigstellungszeit $C_E$ ist dann ebenfalls eine Zufallsvariable.
Mögliche Fragen sind nun z.\,B.:
\begin{itemize}
    \item
    Welche Verteilung hat $C_E$?

    \item
    In welcher Zeit ist der Prozess mit \SI{95}{\percent} Wahrscheinlichkeit abgeschlossen?

    \item
    Wo ist der kritische Pfad?
\end{itemize}

\linie

\textbf{gemeinsame Verteilungfunktion}:
Die \begriff{gemeinsame Verteilungfkt.} der Aufträge $A_1, \dotsc, A_n$ ist
die Verteilungsfunktion
$F_{T_1,\dotsc,T_n}(t_1, \dotsc, t_n) := \PP(T_1 \le T_1, \dotsc, T_n \le t_n)$
des Zufallsvektors $(T_1, \dotsc, T_n)$.
Die Verteilungsfunktion beschreibt die Abhängigkeiten zwischen den $T_i$.
Im Folgenden wird als Modellvereinfachung angenommen, dass die $T_i$ unabhängig sind, d.\,h.\\
$\PP(T_1 \le T_1, \dotsc, T_n \le t_n) = \prod_{i=1}^n \PP(T_i \le t_i)$.

\textbf{optimale Gesamtfertigstellungszeit}:
Die optimale Gesamtfertigstellungszeit $C_E$ ist eine Zufallsvariable und hängt von
$T_1, \dotsc, T_n$ ab.
Ist eine konkrete Realisierung $t_1, \dotsc, t_n$ bekannt, so bestimmt sich $c_E$ mit der CPM.
Allerdings kann man nicht einfach alle Realisierungen ausprobieren:
Selbst wenn jedes $T_i$ diskret verteilt ist und nur drei Werte annimmt, so gibt es
$3^n$ Kombinationen und es müssen $3^n$ viele CPM-Läufe durchgeführt werden.
Eine Abhilfe kann es sein, die $t_i$ durch $\EE(T_i)$ zu ersetzen
und das resultierende $c_E$ als Schätzung für $\EE(c_E)$ zu benutzen.

\linie

\textbf{serielle Bearbeitung}:
Werden die Aufträge seriell bearbeitet ($A_S \to A_1 \to \dotsb \to A_n \to A_E$),
so ist $C_E = \sum_{i=1}^n T_i$.
Wegen der Linearität des Erwartungswerts gilt
$\EE(C_E) = \sum_{i=1}^n \EE(T_i)$, d.\,h. obige Schätzung ist exakt.

\textbf{parallele Bearbeitung}:
Werden die Aufträge parallel bearbeitet ($A_S \to A_i \to A_E$),
so ist $C_E = \max_{i=1,\dotsc,n} T_i$.
Allerdings gilt i.\,A. nur $\EE(C_E) \ge \max_{i=1,\dotsc,n} \EE(T_i)$
(\begriff{\name{jensen}sche Unglei"-chung}),
d.\,h. obige Schätzung ist i.\,A. zu optimistisch.
Dass die Schätzung schon bei mittelgroßen $n$ viel zu optimistisch ist, sieht man
z.\,B. bei auf $[0, 1]$ gleichverteilten $T_i$.
Dann ist $\EE(C_E) = \frac{n}{n+1}$, aber die Schätzung ist stets $\frac{1}{2}$.

Die Schätzung ist nicht einmal sinnvoll nutzbar für die Bestimmung des kritischen Pfades.
Als Beispiel betrachte man die parallelen Aufträge $A_1, A_2$ mit
$\PP(T_1 = 0) = \PP(T_1 = 8) = \frac{1}{2}$,
$\PP(T_2 = 1) = \frac{3}{4}$ und $\PP(T_2 = 9) = \frac{1}{4}$.
Der Pfad über $A_1$ ist kritisch genau dann, wenn $T_1 = 8$ und $T_2 = 1$, was mit
Wahrscheinlichkeit $\frac{1}{2} \cdot \frac{3}{4} = \frac{3}{8}$ passiert.
Somit ist der Pfad über $A_2$ mit Wahrscheinlichkeit $\frac{5}{8}$ kritisch.
Betrachtet man allerdings die Erwartungswerte $\EE(T_1) = 4$ und $\EE(T_2) = 3$,
so sieht man, dass hier der obere Pfad kritisch wäre, wenn man die Erwartungswerte als
Schätzung nutzen würde.

\pagebreak
