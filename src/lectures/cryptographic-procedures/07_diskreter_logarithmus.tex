\section{%
    Diskreter Logarithmus%
}

\textbf{diskreter Logarithmus}:
Gegeben ist eine endliche Gruppe $G$, $g \in G$ und $y \in \erzeugnis{g}$.\\
Gesucht ist ein \begriff{diskreter Logarithmus} $x$ von $y$ zur Basis $g$,
d.\,h. $x \in \natural_0$ mit $y = g^x$.

Im Folgenden sei $n := |G|$.
$n$ ist aber nicht zwangsläufig bekannt
(bei großen Elementordnungen kann man sich auch vorstellen, dass $n \approx \ord(g)$).

\linie

\textbf{naiver Ansatz}:\\
Man berechnet alle Potenzen $g^0, g^1, \dotsc, g^{n-1}$ und überprüft, welches Element gleich
$y$ ist.

\textbf{Zeitbedarf}:
$\O(n)$

\textbf{Speicherbedarf}:
$\O(1)$

Der naive Ansatz ist für große Gruppenordnungen ($n > 2^{160}$) praktisch nicht durchführbar.

\subsection{%
    \name{Shank}s Babystep-Giantstep-Methode%
}

\textbf{Idee}:
Mit $m \in \natural$ mit $m \ge \sqrt{n}$ (z.\,B. $m := \left\lceil\sqrt{n}\right\rceil$)
gilt $\exists_{0 \le s, r < m}\; x = sm + r$.
Man kann nun $s, r$ mehr oder weniger separat berechnen.

\linie

\textbf{\name{Shank}s Babystep-Giantstep-Methode}:
Die \begriff{Babystep-Giantstep-Methode (BSGS-Metho"-de) von \name{Shank}} berechnet einen
diskreten Logarithmus wie folgt.
\begin{enumerate}
    \item
    Berechne $m := \left\lceil\sqrt{n}\right\rceil$ oder, soweit $n$ nicht bekannt ist,
    eine obere Schranke $m \ge \sqrt{n}$.

    \item
    Berechne $(r, yg^{-r}) \in \natural \times G$ für alle $r = 0, \dotsc, m - 1$ und
    speichere die Paare in einer \begriff{Hashtabelle}
    (Nachschlagen in Zeit $\O(1)$).

    \item
    Berechne $h^s \in G$ mit $h := g^m$ für alle $s = 0, \dotsc, m - 1$
    und breche ab, sobald $(r, h^s)$ mit einem $r \in \{0, \dotsc, m - 1\}$
    in der Hashtabelle vorkommt.

    \item
    $x = sm + r$ ist dann der diskrete Logarithmus, weil
    $g^x = g^{sm + r} = h^s g^r = y g^{-r} g^r = y$.
\end{enumerate}
Bei Schritt \emph{(2)} werden die \begriff{Babysteps} und bei Schritt \emph{(3)} die
\begriff{Giantsteps} berechnet.

\linie

\textbf{Zeitbedarf}:
$\O(m)$ Gruppenoperationen (für $m = \left\lceil\sqrt{n}\right\rceil$: $\O(\sqrt{n})$)

\textbf{Speicherbedarf}:
$\O(m)$ Gruppenelemente (für $m = \left\lceil\sqrt{n}\right\rceil$: $\O(\sqrt{n})$)

Die BSGS-Methode ist daher in der Praxis ebenfalls nicht anwendbar (Speicherverbrauch zu groß).

\pagebreak

\subsection{%
    \name{Pollard}s \texorpdfstring{$\varrho$}{ρ}-Methode für den diskreten Logarithmus%
}

\textbf{Idee}:
Partitioniere $G$ in drei Mengen $P_1, P_2, P_3$.
Die Zugehörigkeit eines Elements aus $G$ zu einer drei Klassen sollte sich leicht berechnen
lassen und die Aufteilung sollte sich in etwa zufällig verhalten.
Meistens rechnet man modulo $3$ und sammelt jedes dritte Element in einer Klasse.
Definiere nun die Abbildung $f\colon \ZnZ \times \ZnZ \rightarrow \ZnZ \times \ZnZ$ mit
\begin{itemize}
    \item
    $f(r, s) := (r + 1, s)$, falls $g^r y^s \in P_1$,

    \item
    $f(r, s) := (2r, 2s)$, falls $g^r y^s \in P_2$, und

    \item
    $f(r, s) := (r, s + 1)$, falls $g^r y^s \in P_3$.
\end{itemize}
Sei $r, s \in \ZnZ$.
Ist $(r', s') := f(r, s)$, $h := g^r y^s$ und $h' := g^{r'} y^{s'}$, so gilt
\begin{itemize}
    \item
    $h' = gh$, falls $h \in P_1$,

    \item
    $h' = h^2$, falls $h \in P_2$, und

    \item
    $h' = hy$, falls $h \in P_3$
\end{itemize}
(weil $y \in \erzeugnis{g}$).
Man startet daher mit einem zufälligen Paar $(r_1, s_1) \in \ZnZ \times \ZnZ$.
Die Folge $(h_k)_{k \in \natural}$ mit $h_k := g^{r_k} y^{s_k}$ und
$(r_{k+1}, s_{k+1}) := f(r_k, s_k)$ für $k \in \natural$ ist dann eine Pseudozufallsfolge.
Nach dem Geburtstagsparadoxon gilt $\exists_{i, j \in \natural,\; i < j}\; h_i = h_j$, wobei
$j = \O(\sqrt{n})$ (erwartet).
Es gilt also $g^{r_i+xs_i} = g^{r_i} y^{s_i} = g^{r_j} y^{s_j} = g^{r_j+xs_j}$
und damit $x(s_i - s_j) \equiv_n r_j - r_i$.
Man braucht also nur noch für Lösungen $x$ dieser Kongruenz zu testen, ob $y = g^x$ gilt.
Je kleiner $\ggT(s_i - s_j, n)$ ist, desto weniger Lösungen $x$ müssen überprüft werden.

\linie

Ein Zyklus in der Folge könnte mit dem Verfahren bei Pollards $\varrho$-Methode für die
Faktorisierung gefunden werden.
Es geht aber auch anders (diese Methode kann auch bei der Faktorisierung verwendet werden).

\textbf{\name{Pollard}s $\varrho$-Methode für den DL}:
\begriff{\name{Pollard}s $\varrho$-Methode für den diskreten Logarithmus} berechnet einen
diskreten Logarithmus wie folgt.
\begin{enumerate}
    \item
    Sei $\ell := 1$ und $h_1 := g^{r_1} y^{s_1}$ mit einem zufälligen Paar
    $(r_1, s_1) \in \ZnZ \times \ZnZ$.

    \item
    Berechne sukzessive $h_k$ für $k = \ell + 1, \dotsc, 2\ell$,
    wobei sich $h_k$ aus $h_{k-1}$ wie oben ergibt
    und $(r_k, s_k) := f(r_{k-1}, s_{k-1})$ wie oben
    (damit gilt $h_k = g^{r_k} y^{s_k}$).
    Stimmt eines der Elemente $h_k$ mit $h_\ell$ überein, so breche ab.
    Ansonsten verdopple $\ell$ und wiederhole.

    \item
    Definiere $(r, s) := (r_\ell, s_\ell)$ und $(r', s') := (r_k, s_k)$.
    Dann teste für jede Lösung $x$ der Kongruenz $x(s - s') \equiv_n r' - r$, ob $y = g^x$ gilt.
    Falls ja, ist $x$ der gesuchte diskrete Logarithmus.
\end{enumerate}
Gilt $\exists_{i, j \in \natural,\; i < j}\; h_i = h_j$ mit $j = \O(\sqrt{n})$ (erwartet)
wie oben, so terminiert der Algorithmus spätestens für $\ell \ge j$
(weil dann $h_\ell = h_{\ell+j-i}$).

\linie

\textbf{Zeitbedarf}:
erwartet $\O(\sqrt{n})$ Gruppenoperationen

\textbf{Speicherbedarf}:
$\O(1)$ Gruppenelemente

\linie

\textbf{Vorteile gegenüber BSGS}:
konstanter Speicherverbrauch,
leichter zu implementieren

\textbf{Nachteile gegenüber BSGS}:
probabilistisches Verfahren (Laufzeit kann schlecht sein),
Vielfaches von $\ord(g)$ (z.\,B. $n = |G|$) muss bekannt sein

\pagebreak

\subsection{%
    \name{Pohlig}-\name{Hellman}-Algorithmus%
}

Für den Pohlig-Hellman-Algorithmus seien $G$ zyklisch, d.\,h. $G = \erzeugnis{g}$, und
die Primfaktorzerlegung $\prod_{p \in \PP,\; p \teilt n} p^{e(p)}$ von $n = |G|$ bekannt.

Für jedes $p \in \PP$ mit $p \teilt n$ seien im Folgenden
$n_p := \frac{n}{p^{e(p)}}$,
$g_p := g^{n_p}$,
$y_p := y^{n_p}$ und
$x_p \in \natural_0$ mit $y_p = g_p^{x_p}$.
$g_p$ und $y_p$ sind Elemente der Untergruppe $G_p := \{h^{n_p} \;|\; h \in G\}$ von $G$.
Diese ist zyklisch mit Ordnung $|G_p| = p^{e(p)}$.

\textbf{Lemma}:
Sei $x \in \natural_0$ mit
$\forall_{p \in \PP,\; p \teilt n}\; x \equiv x_p \pmod{p^{e(p)}}$.
Dann ist $y = g^x$.

\begin{Beweis}
    Für jedes $p \in \PP$ mit $p \teilt n$ gilt
    $(g^{-x} y)^{n_p} = g_p^{-x} y_p = g_p^{-x_p} y_p = 1$,
    da $g_p^x = g_p^{x_p}$
    (es gilt $g_p \in G_p$ und $x, x_p$ unterscheiden sich nur um ein Vielfaches der
    Gruppenordnung $p^{e(p)}$ von $G_p$).
    Daraus folgt, dass $\ord_G(g^{-x} y) \teilt n_p$ für jeden Primteiler $p$ von $n$.
    Insbesondere gilt\\
    $\ord_G(g^{-x} y) \teilt \ggT(\{n_p \;|\; p \in \PP,\; p \teilt n\}) = 1$
    (bei jeder Zahl $n_p$ wurde eine Primzahl $p$ herausgeteilt).
    Damit ist $\ord_G(g^{-x} y) = 1$ und $g^x = y$.
\end{Beweis}

\linie

OBdA kann man also von $G$ zyklisch mit $n = p^e$ ($p \in \PP$, $e \in \natural$)
ausgehen.
Sonst bestimmt man für jeden Primteiler $p \teilt n$ die $x_p$ mit $y_p = g_p^{x_p}$
und löst das System $\forall_{p \in \PP,\; p \teilt n}\; x \equiv x_p \pmod{p^{e(p)}}$
von Kongruenzen mit dem chinesischen Restsatz.

\textbf{\name{Pohlig}-\name{Hellman}-Algorithmus}:
Der \begriff{\name{Pohlig}-\name{Hellman}-Algorithmus} bestimmt den diskreten Logarithmus
in einer zyklischen Gruppe (oBdA mit Primzahlpotenzordnung $p^e$) wie folgt.
\begin{enumerate}
    \item
    Schreibe $x = x_0 + x_1 p + \dotsb + x_{e-1} p^{e-1}$ mit
    $x_0, \dotsc, x_{e-1} \in \{0, \dotsc, p-1\}$
    (da $x < |G| = p^e$).

    \item
    Zur Bestimmung der Koef"|fizienten $x_0, \dotsc, x_{e-1}$ wiederhole für
    $i = 0, \dotsc, e - 1$ Folgendes:
    \begin{enumerate}
        \item
        Berechne $z_i := yg^{-(x_0 p^0 + \dotsb + x_{i-1} p^{i-1})}$.
        Dann gilt $g^{x_i p^i + \dotsb + x_{e-1} p^{e-1}} = z_i$.
        Potenzieren mit dem Exponenten $p^{e-i-1}$ auf beiden Seiten führt zu
        $g^{x_i p^{e-1}} = z_i^{p^{e-i-1}}$ wg. $\forall_{e' \ge e}\; g^{p^{e'}} = 1$.

        \item
        Berechne den diskreten Logarithmus $x_i$ von $z_i^{p^{e-i-1}}$ zur Basis $g^{p^{e-1}}$
        in der Untergruppe $\{h^{p^{e-1}} \;|\; h \in G\}$ von $G$
        (zyklisch mit Ordnung $p$),
        also $x_i \in \natural_0$ mit $(g^{p^{e-1}})^{x_i} = z_i^{p^{e-i-1}}$,
        z.\,B. mit der BSGS- oder Pollards $\varrho$-Methode.
    \end{enumerate}
\end{enumerate}

\linie

\textbf{Zeitbedarf}:
$\O(\sum_{p \in \PP,\; p \teilt n} e(p) \cdot (\log n + \sqrt{p}))$ Gruppenoperationen in $G$

Die Laufzeit des Pohlig-Hellman-Algorithmus ist daher gut, wenn $n$ nur kleine Primteiler besitzt.
Der Algorithmus kann nur durchgeführt werden, wenn nicht nur die Gruppenordnung $n$,
sondern sogar ihre Primfaktorzerlegung bekannt ist.

\pagebreak

\subsection{%
    Index-Calculus-Algorithmus%
}

Für den Index-Calculus-Algorithmus ist $g \in G = (\ZpZ)^\ast$ mit $p$ prim
(oft kann man sich $\erzeugnis{g} = (\ZpZ)^\ast$ vorstellen) und $y \in \erzeugnis{g}$.
Gesucht ist $x \in \natural_0$ mit $g^x \equiv_p y$.

\textbf{Index-Calculus-Algorithmus}:
Mit dem \begriff{Index-Calculus-Algorithmus} kann man den diskreten Logarithmus
in $(\ZpZ)^\ast$ für $p$ prim wie folgt bestimmen.
\begin{enumerate}
    \item
    Wähle eine Faktorbasis $B = \{q \in \PP \;|\; q \le B_0\}$ für eine Schranke
    $B_0 \in \natural$.

    \item
    Für alle $q \in B$ bestimme $x_q \in \natural_0$ mit $g^{x_q} \equiv_p q$
    (diskreter Logarithmus von $q$).

    \item
    Bestimme $b \in \natural_0$, sodass $(yg^b \bmod p)$ $B$-glatt ist,
    d.\,h. $yg^b \equiv_p \prod_{q \in B} q^{e_q}$.

    \item
    Wähle $x := (-b + \sum_{q \in B} x_q e_q) \bmod (p-1)$.
\end{enumerate}

\textbf{Lemma (Korrektheit)}:
Der Index-Calculus-Algorithmus arbeitet korrekt, d.\,h. $g^x \equiv_p y$.

\begin{Beweis}
    Es gilt $g^x \equiv_p g^{-b} g^{\sum_{q \in B} x_q e_q}
    = g^{-b} \prod_{q \in B} g^{x_q e_q}
    \equiv_p g^{-b} \prod_{q \in B} q^{e_q}
    \equiv_p g^{-b} \cdot yg^b \equiv_p y$.
\end{Beweis}

\linie

\textbf{zu Schritt \emph{(2)}}:
Wähle zufällige Zahlen $z \in \natural_0$ mit $(g^z \bmod p)$ $B$-glatt,
d.\,h. $g^z \equiv_p \prod_{q \in B} q^{f(q,z)}$ für bestimmte $f(q,z) \in \natural_0$.
Wenn man nun $(f(q,z))_{q \in B}$ als Vektor abspeichert
und $q \equiv_p g^{x_q}$ mit den Variablen $(x_q)_{q \in B}$ setzt,
dann gilt $z \equiv \sum_{q \in B} x_q f(q, z) \pmod{p - 1}$.
Gibt es genügend solche viele Zahlen $z$, dann hat das entstehende LGS modulo den höchsten
Primteilerpotenzen von $(p-1)$ genügend viele Gleichungen und dann gibt es eine Lösung
$(x_q)_{q \in B}$.

\textbf{zu Schritt \emph{(3)}}:
Bestimme $y$ zufällig mit $(yg^b \bmod p)$ $B$-glatt.

\linie

\textbf{Zeitbedarf}:
erwartet $2^{\weakO(\sqrt{\log p})}$

\linie

Der Index-Calculus-Algorithmus ist nicht auf andere Gruppen übertragbar, weil
"`$B$-glatt"' dann keinen Sinn mehr ergibt.

\pagebreak
