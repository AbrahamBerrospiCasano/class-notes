\section{%
    Einfache zahlentheoretische Algorithmen%
}

\textbf{iterierter Logarithmus}:
Sei $n \in \natural$.
Dann ist der \begriff{iterierte Logarithmus} definiert durch
$\log^\ast n := 1 + \log^\ast(\log n)$ für $n > 1$ und $\log^\ast n := 0$ für $n < 1$.
Der iterierte Logarithmus ist die Anzahl, wie oft man den Logarithmus auf eine Zahl anwenden kann,
bis sie negativ wird.
$\log^\ast$ wächst sehr langsam (allerdings schneller als die inverse Ackermann-Funktion),
für $n < 2^{65536}$ gilt $\log_2^\ast n \le 5$.

\textbf{$\weakO$-Notation}:
Seien $f, g\colon \natural \rightarrow \real$ zwei Funktionen.\\
Dann ist $f \in \weakO(g(n))$ (\begriff{Weak-$\O$}, \begriff{$\O$-Tilde}),
falls $\exists_{k \in \natural}\; f \in \O(g(n) \cdot \log^k g(n))$.

\subsection{%
    Multiplikation%
}

\textbf{Problem}:
Gegeben sind zwei Zahlen $r, s \in \natural$ mit je $n$ Bit.\\
Gesucht ist das Produkt $r \cdot s$ der beiden Zahlen.

\textbf{Algorithmen und ihr Zeitbedarf}:
\begin{itemize}
    \item
    \emph{Schulmethode (schriftliches Multiplizieren)}:
    $\O(n^2)$
    
    \item
    \emph{\name{Karatsuba}-Algorithmus}:
    $\O(n^{1.58497})$ mit $1.58497 \approx \log_2 3$
    
    \item
    \emph{\name{Schönhage}-\name{Strassen}-Algorithmus}:
    $O(n \log n \cdot \log \log n) \subset \weakO(n)$
    
    \item
    \emph{\name{Fürer}-Algorithmus}:
    $\O(n \log n \cdot 2^{\log^\ast n}) \subset \weakO(n)$
\end{itemize}

\linie

\textbf{\name{Karatsuba}-Algorithmus}:
Der \begriff{\name{Karatsuba}-Algorithmus} ist ein rekursiver Algorithmus zur Multiplikation
zweier Ganzzahlen.
Seien $r, s$ zwei Zahlen mit je $n = 2k$ Bit.
\begin{enumerate}
    \item
    Berechne zunächst $0 \le A, B, C, D < 2^k$ mit $r = A \cdot 2^k + B$ und
    $s = C \cdot 2^k + D$.
    
    \item
    Berechne rekursiv $AC$, $BD$ und $(A + B)(C + D)$.
    
    \item
    Berechne $r \cdot s = AC \cdot 2^{2k} + (A + B)(C + D) \cdot 2^k -
    (AC + BD) \cdot 2^k + BD$.
\end{enumerate}

\textbf{Laufzeit}:
$\O(n^{\log_2 3})$

\begin{Beweis}
    Würde man $r \cdot s = AC \cdot 2^{2k} + (AD + BC) \cdot 2^k + BD$ schreiben,
    so müsste man vier Multiplikationen durchführen.
    Ist $T(2k)$ die Anzahl der Schritte, die der Algorithmus für die Multiplikation zweier
    Zahlen der Länge $n = 2k$ benötigt, so gilt dann
    $T(2k) = 4T(k) + \O(k)$
    (die Additionen benötigen $\O(k)$ Schritte).
    Mit dem Master-Theorem würde man auf $T \in \Theta(n^2)$ kommen, was keine Verbesserung
    gegenüber der Schulmethode wäre.
    
    Weil allerdings nur drei Multiplikationen pro Rekursionsschritt durchgeführt werden, gilt\\
    $T(2k) = 3T(k) + \O(k)$ und mit dem Master-Theorem kommt man auf
    $T \in \Theta(n^{\log_2 3})$.
\end{Beweis}

\subsection{%
    Modulo-Operation%
}

\textbf{Modulo-Operation}:
Es gilt $a \bmod m = a - m \left\lfloor\frac{a}{m}\right\rfloor$,
d.\,h. wenn man schnell Dividieren und Multiplizieren kann, kann man auch schnell Modulo rechnen.

\textbf{Laufzeit}:
$\weakO(n)$ mit $n$ der Anzahl der Bit von $\max(a, m)$

\pagebreak

\subsection{%
    Division%
}

\textbf{Problem}:
Gegeben ist $m \in \natural$ mit $n$ Bit.
Gesucht ist $\frac{1}{m}$.

\textbf{\name{Newton}-Verfahren zur Berechnung von $\frac{1}{m}$}:
\begin{enumerate}
    \item
    Der Startwert ist $x_0 := 2^{-\lceil\log m \rceil}$,
    wobei $\lceil\log m \rceil$ ca. gleich der Anzahl der Stellen von $m$ und
    daher leicht bestimmbar ist.
    
    \item
    Suche mit dem Newton-Verfahren die Nullstelle von $f(x) = \frac{1}{x} - m$, d.\,h.\\
    $x_{i+1} := x_i - \frac{f(x_i)}{f'(x_i)} = 2x_i - mx_i^2$.
\end{enumerate}

\textbf{Laufzeit}:
$\weakO(n)$

\linie

\textbf{Reduktion von Multiplikation auf Quadrieren}:
Multiplizieren ist im Wesentlichen genauso schwer wie Quadrieren,
da $r \cdot s = \frac{1}{4} ((r + s)^2 - (r - s)^2)$.
Mit dieser Formel kann man mit zweimal Quadrieren einmal Multiplizieren,
d.\,h. Quadrieren geht höchstens doppelt so schnell wie Multiplikation.

\subsection{%
    Erweiterter \name{euklid}ischer Algorithmus%
}

\textbf{Problem}:
Gegeben seien $k, \ell \in \integer$, wobei die kleinere Zahl $n$ Bit lang ist.\\
Gesucht sind $a, b \in \integer$ und $t \in \natural_0$ mit
$ak + b\ell = t = \ggT(k, \ell)$.

\textbf{erweiterter \name{euklid}ischer Algorithmus}:
Der \begriff{erweiterte \name{euklid}ische Algorithmus}\\
$\code{erw\_ggT}(k, \ell)$
berechnet $(a, b, t)$ für $k, \ell \ge 0$.
Ist $k < 0$, so muss am Ende das Vorzeichen von $a$ verändert werden
(analog für $\ell < 0$).
\begin{enumerate}
    \item
    Ist $k = 0$, so gebe $(0, 1, \ell)$ zurück.
    
    \item
    Sonst berechne $(a, b, t) := \code{erw\_ggT}(\ell \bmod k, k)$
    und gebe $(b - a \cdot \lfloor \ell/k \rfloor, a, t)$ zurück.
\end{enumerate}

\textbf{Laufzeit}:
$\weakO(n^2)$

\begin{Beweis}
    Ist $k > \ell$, dann vertauscht der Algorithmus im ersten Rekursionsschritt $k$ und $\ell$.
    Sei also oBdA $k \le \ell$.
    Der Algorithmus ruft sich mit $k' := \ell \bmod k$ und $\ell' := k$ auf.
    Im nächsten Schritt ruft er sich mit $k'' := \ell' \bmod k' = k \bmod (\ell \bmod k)$ und
    $\ell'' := k' = \ell \bmod k$ auf.
    \begin{itemize}
        \item
        Ist $\ell \bmod k \le \frac{k}{2}$, so ist $k'' < \ell \bmod k \le \frac{k}{2}$.
        
        \item
        Ist $\ell \bmod k > \frac{k}{2}$, so ist
        $k'' = k \bmod (\ell \bmod k) = k - (\ell \bmod k) \le \frac{k}{2}$. 
    \end{itemize}
    In jedem Fall gilt $k'' \le \frac{k}{2}$ und spätestens nach zwei rekursiven Aufrufen hat
    sich die kleinere Zahl halbiert.
    Die Rekursionstiefe ist daher $\O(\log k) = \O(n)$ und die Laufzeit $\weakO(n^2)$.
\end{Beweis}

\linie

\textbf{Invertieren modulo $m$}:
Gegeben seien $m, k \in \natural$ mit $m \ge 2$ (mit $n$ der Bitlänge der kleineren Zahl).
Gesucht ist das multiplikative Inverse von $k$ modulo $m$.
\begin{enumerate}
    \item
    Berechne zunächst $(a, b, t) := \code{erw\_ggT}(k, m)$.
    
    \item
    Ist $t \not= 1$, dann ist $k$ nicht invertierbar modulo $m$.
    
    \item
    Ist $t = 1$, dann gilt $ak + bm = 1 \iff ak \equiv_m 1$,
    d.\,h. $a \bmod m \in \ZmZ$ ist das gesuchte multiplikative Inverse von $k$ mod $m$.
\end{enumerate}

\textbf{Laufzeit}:
$\weakO(n^2)$

\pagebreak

\subsection{%
    Exponentiation%
}

\textbf{Problem}:
Gegeben sei ein Monoid $M$ (Gruppe bis auf Existenz von Inversen), ein Element $a \in M$ und
$m \in \natural$.
Gesucht ist $a^m \in M$.

\textbf{schnelle Exponentiation}:\\
Der Algorithmus zur \begriff{schnellen Exponentiation} berechnet $e = a^m$ wie folgt.
\begin{enumerate}
    \item
    Setze zunächst $e := 1$.
    
    \item
    Solange $m > 0$ ist, wiederhole Folgendes:
    \begin{enumerate}
        \item
        Ist $m$ ungerade, so setze $e \leftarrow a \cdot e$.
        
        \item
        Setze $a \leftarrow a^2$ und $m \leftarrow \lfloor m/2\rfloor$.
    \end{enumerate}
\end{enumerate}
Es werden $\O(\log m)$ Monoid-Operationen durchgeführt,
weil $m$ in jedem Durchlauf mindestens halbiert wird.

\textbf{schnelle modulare Exponentiation}:
Ist $M = (\ZkZ, \cdot)$, dann wird bei jeder Operation modulo $k$ gerechnet.
Sei $n$ die größere der Bitlängen von $k$ und $m$.
Dann kostet Multiplizieren und Modulo-Operationen jeweils $\weakO(n)$,
d.\,h. die Laufzeit des obigen Algorithmus ist $\weakO(n^2)$.

\linie

\textbf{Optimierung mittels Additionsketten}:
Am Beispiel $a^{15}$ erkennt man, dass man auch mit weniger Operationen auskommen kann.
Obiger Algorithmus berechnet $a, a^2, a^3, a^4, a^7, a^8, a^{15}$, d.\,h. er benötigt
sechs Multiplikationen.
Wenn man allerdings $a, a^2, a^3, a^6, a^{12}, a^{15}$ berechnet, so benötigt man bloß fünf
Multiplikationen.

\textbf{Additionskette}:
Eine \begriff{Additionskette} $(m_0, \dotsc, m_\ell)$ ist eine Kette mit
$m_0, \dotsc, m_\ell \in \natural$, $m_0 := 1$ und $m_i$ ist eine Summe von zwei beliebigen
Werten mit Index kleiner als $i$
($i = 1, \dotsc, \ell$).

Die zu obigen Multiplikationsfolgen entsprechenden Additionsketten sind
$(1, 2, 3, 4, 7, 8, 15)$ und $(1, 2, 3, 6, 12, 15)$.
Am wenigsten Monoid-Operationen benötigt man, wenn man eine kürzeste Additionskette verwendet.
Das Berechnen einer Additionskette an sich kostet zwar keine Monoid-Operationen, allerdings
müssen bei der Exponentiation mit eine optimalen Additionskette i.\,A. alle bisher aufgetretenen
Potenzen im Speicher gehalten werden.
Bei den zum obigem Algorithmus gehörigen Additionsketten werden hingegen immer nur die letzten
zwei Potenzen benötigt.
Außerdem kann man mit der optimalen Additionskette nicht mehr als die Hälfte an Operationen
gegenüber obigem Algorithmus einsparen.
Am meisten spart man bei Zweierpotenzen minus $1$,
während bei Zweierpotenzen obiger Algorithmus optimal ist.

\linie

\textbf{Invertieren durch Exponentiation}:
Ist $G$ eine endliche Gruppe und $g \in G$,
so kann man $g^{-1}$ durch $g^{-1} = g^{|G| - 1}$ berechnen.
Für $G = ((\ZnZ)^\ast, \cdot)$ ist z.\,B. $a^{-1} \equiv_n a^{\varphi(n) - 1}$.
Damit kann man ebenfalls in Zeit $\weakO(n^2)$ invertieren, allerdings muss $\varphi(n)$ bekannt
sein.

\linie

\textbf{Laufzeit von \name{Miller}-\name{Rabin}}:
Ist $n$ die zu testende Zahl (mit $\log n$ Stellen), so ist die Laufzeit des MR-Tests
$\weakO(\log^2 n)$ pro Durchlauf.

\textbf{Laufzeit von RSA}:
Ist $n$ das RSA-Modul, so ist die Laufzeit einer Ver-/Entschlüsselung von RSA
$\weakO(\log^2 n)$
($\weakO(\log n)$ bei Verschlüsselung mit kleinem $e$).
Durch Speicherung der Primfaktoren $p$ und $q$ kann die Entschlüsselung doppelt so schnell
erfolgen.

\pagebreak
