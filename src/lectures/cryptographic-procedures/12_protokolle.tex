\section{%
    Protokolle%
}

\subsection{%
    Elektronische Verpflichtung%
}

\textbf{elektronische Verpflichtung}:
In gewissen Situationen ist es nötig, dass Alice vorab eine Wahl trifft und dann nach
einem gewissen Zeitpunkt die Wahl von Bob so überprüft wird, ohne dass die Entscheidung hinterher
durch Alice beeinflusst werden könnte, siehe folgendes Beispiel.

Die Anlageberaterin Alice möchte Bob Aktien empfehlen.
Allerdings möchte Bob sich zunächst davon überzeugen, dass die empfohlenen Aktien tatsächlich
auch steigen.
Alice kann Bob nicht einfach
im Voraus die Aktien nennen, sonst investiert Bob in die Aktien, ohne Alice als
Vermittlerin zu bezahlen.
Alice kann Bob aber auch nicht hinterher die Aktien nennen, denn Alice könnte ja genau die Aktien
auswählen, die gestiegen sind (und gar nichts mit ihren ursprünglichen Empfehlungen zu tun haben).

Mit analogen Mitteln würde man die Empfehlungen in einem Briefumschlag an einem sicheren Ort
verwahren.
Die digitale Version dieser Methode ist die \begriff{elektronische Verpflichtung}.

\linie

Im Folgenden wird angenommen, dass Alice sich auf ein einzelnes Bit $t \in \BB$ festlegt.

\textbf{elektr. Verpfl. mit symm. Verschl.verf.}:
Seien $(c_k)_{k \in K}$ und $(d_k)_{k \in K}$ die Ver- bzw. Entschlüsselungsfunktionen
eines symmetrischen Kryptosystems mit Schlüsselmenge $K$.
\begin{enumerate}
    \item
    Alice wählt einen zufälligen Schlüssel $k \in K$.
    
    \item
    Bob erzeugt einen Zufallsstring $x$ und sendet ihn an Alice.
    
    \item
    Alice berechnet $y := c_k(xt)$ und schickt $y$ an Bob.
\end{enumerate}
Später kann Bob die Wahl von Alice wie folgt überprüfen:
\begin{enumerate}
    \item
    Alice schickt $k$ an Bob.
    
    \item
    Bob berechnet $d_k(y)$ und testet, ob $d_k(y) = xt'$ für ein $t' \in \BB$.
\end{enumerate}

\textbf{Angriffsmöglichkeiten}:
Alice könnte nach einem Schlüssel $k'$ suchen, sodass $y = c_{k'}(x\overline{t})$ gilt
(mit $\overline{t} := 1 - t$).
Das würde allerdings $d_{k'}(y) = x\overline{t}$ bedeuten,
d.\,h. eine Known-Plaintext-Attacke, der ein gutes Verfahren widerstehen sollte.
Bob hat eine noch schlechtere Position, denn er kennt nur den Geheimtext $y$ und
ein Präfix $x$ des Klartextes.
Um $k$ und damit $t$ herauszufinden, müsste er zwei Known-Plaintext-Attacken
(eine für jede Möglichkeit für $t$) durchführen.

\linie

\textbf{elektr. Verpfl. mit Hashfunktion}:
Sei $h$ eine öf"|fentliche, kollisionsresistente Hashfunktion.
\begin{enumerate}
    \item
    Alice wählt zwei Zufallsstrings $x_1, x_2$.
    
    \item
    Alice berechnet den Hashwert $h(x_1 x_2 t)$ und schickt den Hashwert und $x_1$ an Bob.
\end{enumerate}
Später kann Bob die Wahl von Alice wie folgt überprüfen:
\begin{enumerate}
    \item
    Alice schickt $x_1 x_2 t$ an Bob.
    
    \item
    Bob testet, ob $x_1 x_2 t$ mit dem vorher für $x_1$ erhaltenen Wert beginnt und ob
    $h(x_1 x_2 t)$ gleich dem vorher erhaltenen Hashwert ist.
\end{enumerate}

\textbf{Angriffsmöglichkeiten}:
Durch Of"|fenlegeung eines Teils der Zufallsinformation kann Bob prüfen,
dass Alice keine speziellen Strings wählt, die es erleichtern würden, eine Kollision zu finden.
Andererseits kann Bob $t$ nicht bestimmen, bevor er $x_1 x_2 t$ erhält
($h$ ist schwierig zu invertieren).

\textbf{Hauptvorteil}:
Bob verschickt keine Nachrichten, d.\,h. Alice kann Radio oder Zeitung benutzen.

\pagebreak

\subsection{%
    Teilen von Geheimnissen%
}

Angenommen, eine Person kennt ein Geheimnis $s$ und möchte dieses Geheimnis so auf
$n$ Personen aufteilen, dass beliebige $t$ dieser Personen das Geheimnis rekonstruieren können
(mit $n, t \in \natural$ und $t \le n$).
Weniger als $t$ Personen sollen aber bei einem Zusammentref"|fen keine Informationen über $s$
gewinnen können.

\textbf{Teilen von Geheimnissen nach \name{Shamir}}:\\
Für das \begriff{Teilen von Geheimnissen nach \name{Shamir}} nimmt man $s \in \natural$ an.
\begin{enumerate}
    \item
    Wähle eine große Primzahl $p$ (mit $p \gg n, s$).
    
    \item
    Wähle zufällige Koef"|fizienten $a_1, \dotsc, a_{t-1} \in \FF_p$.
    
    \item
    Setze $a(X) := s + \sum_{i=1}^{t-1} a_i X^i \in \FF_p[X]$.
    
    \item
    Teile der $j$-ten Person den Schlüssel $(j, a(j))$ mit (für $j = 1, \dotsc, n$).
\end{enumerate}
Wenn nun $t$ Personen zusammenkommen, verläuft die Rekonstruktion von $s$ wie folgt:
\begin{enumerate}
    \item
    Löse das LGS $s + \sum_{i=1}^{t-1} a_i j^i = a(j)$ mit $t$ Gleichungen
    (für jedes $j$ eine) und $t$ Unbekannten ($s, a_1, \dotsc, a_{t-1}$).
    
    \item
    Gebe $s$ aus.
\end{enumerate}

\textbf{Vorteil}:
Weitere Personen können leicht hinzugefügt werden, indem zusätzliche Schlüssel ausgegeben werden
($t$ bleibt allerdings fest, weil sich sonst die Auswertungen $a(j)$ ändern würden).

\textbf{Lemma ($t$ Personen können $s$ rekonstruieren)}:
Das LGS ist eindeutig lösbar.

\begin{Beweis}
    Es gilt $\deg(a(X)) \le t - 1$ und die $t$ Stützstellen sind paarw. verschieden.
    Wegen eindeutig möglicher Polynominterpolation im Körper $\FF_p$
    (Existenz mit Lagrange-Polynomen, Eindeutigkeit, weil $a(X) \not= 0$
    höchstens $t - 1$ Nullstellen hat) folgt die Behauptung.
\end{Beweis}

\linie

Mit der Lagrange-Interpolation kann man nicht nur $a$ rekonstruieren,
sondern auch zeigen, dass eine Gruppe von weniger als $t$ Personen keine
Informationen über $s$ gewinnen kann.

\textbf{\name{Lagrange}-Polynom}:
Seien $x_1, \dotsc, x_t \in \FF_p$ paarweise verschiedene Stützstellen.
Dann heißen die Polynome
$\ell_i(X) := \prod_{j=1,\dotsc,t,\;j\not=i} \frac{X - x_j}{x_i - x_j} \in \FF_p[X]$
für $i = 1, \dotsc, t$ \begriff{\name{Lagrange}-Polynome}.

Es gilt $\deg(\ell_i) = t - 1$ und $\ell_i(x_j) = \delta_{ij}$ für $i, j = 1, \dotsc, n$.

Sind $t$ Stützstellen $x_1, \dotsc, x_t$ mit Werten $a(x_j)$ gegeben, so gilt
für $\widetilde{a}(X) := \sum_{i=1}^t a(x_i) \ell_i(X)$,
dass $a(X) - \widetilde{a}(X)$ die $t$ Nullstellen $x_j$ besitzt und
vom Grad $\le t - 1$ ist, d.\,h. $a(X) = \widetilde{a}(X)$.
Somit können $t$ Personen mit paarweise verschiedenen Stützstellen $a(X)$ rekonstruieren.

\textbf{Lemma (weniger als $t$ Personen)}:\\
Eine Gruppe von weniger als $t$ Personen kann keine Informationen über $s$ gewinnen.

\begin{Beweis}
    Seien nur $t - 1$ Stützstellen $x_1, \dotsc, x_{t-1} \in \FF_p \setminus \{0\}$
    mit Auswertungen $a(x_j)$ bekannt
    ($x_j \not= 0$, sonst wäre $x_j = 0$ und $a(x_j) = s$ direkt bekannt).
    \begin{enumerate}
        \item
        Dann gibt es genau $p$ verschiedene Polynome $\widetilde{a}(X)$ mit Grad $\le t - 1$ und
        $\widetilde{a}(x_j) = a(x_j)$ für $j = 1, \dotsc, t - 1$:
        Wähle $x_t \in \FF_p \setminus \{0\}$ fest mit $x_t \not= x_j$ für $j = 1, \dotsc, t-1$,
        dann ist $\widetilde{a}(X) := \sum_{i=1}^{t-1} a(x_i) \ell_i(X) + b\ell_t(X)$
        verschieden für alle $b \in \FF_p$
        ($\ell_1(X), \dotsc, \ell_t(X)$ l.u.).
        
        \item
        Jedes der $p$ möglichen $\widetilde{a}(X)$ liefert ein anderes $\widetilde{a}(0)$:
        Wäre $\widetilde{a}(0) = \widetilde{a}'(0)$, dann hätte
        $\widetilde{a}(X) - \widetilde{a}'(X)$
        die $t$ Nullstellen $x_1, \dotsc, x_{t-1}, 0$, d.\,h.
        $\widetilde{a}(X) - \widetilde{a}'(X) = 0$.
    \end{enumerate}
    Somit ist für die Gruppe jeder Wert in $\FF_p$ für $s$ gleichwahrscheinlich.
\end{Beweis}

\pagebreak

\subsection{%
    Durchschnittsgehalt%
}

\textbf{Durchschnittsgehalt}:
Gegeben seien mindestens drei Mitarbeiter einer Firma, die ihr Durchschnittsgehalt berechnen
wollen, jedoch ohne ihr eigenes Gehalt preiszugeben
(wobei man annimmt, dass jeder ehrlich ist und keine Personen sich zusammentun).

Im Folgenden seien oBdA drei Personen 1, 2, 3 mit Gehältern $g_1, g_2, g_3$ gegeben.
\begin{enumerate}
    \item
    Person 1 wählt eine Zufallszahl $z$ und schickt $z$ an Person 2.
    
    \item
    Person 2 schickt $z + g_2$ an Person 3.
    
    \item
    Person 3 schickt $z + g_2 + g_3$ an Person 1.
    
    \item
    Person 1 schickt $\frac{1}{3} (g_1 + g_2 + g_3)$ an Person 2 und Person 3.
\end{enumerate}

\subsection{%
    Wer verdient mehr?%
}

\textbf{wer verdient mehr}:
Es seien zwei Mitarbeiter Alice und Bob einer Firma gegeben,
die herausfinden wollen, wer von beiden mehr
verdient, natürlich ohne ihr Gehalt preiszugeben
(wieder angenommen, beide sind ehrlich).

Im Folgenden seien $w_1 < \dotsb < w_n$ die möglichen (diskreten) Gehälter,
$a$ und $b$ das Gehalt von Alice bzw. Bob,
$c_B$ und $d_B$ Bobs öf"|fentliche Ver- bzw. geheime Entschlüsselungsfunktion
und $f$ eine öf"|fentliche Einwegfunktion.
\begin{enumerate}
    \item
    Alice wählt eine Zufallszahl $x$ und sendet $d := c_B(x) - a$ an Bob.
    
    \item
    Bob berechnet $y_i := d_B(d + w_i)$ für $i = 1, \dotsc, n$.
    
    \item
    Bob berechnet $z_i := f(y_i)$.
    OBdA sei $z_i \not= z_j + 1$ und $z_i \not= z_j$
    für $i, j = 1, \dotsc, n$ mit $i \not= j$
    (andernfalls wähle ein neues $x$ oder eine andere Einwegfunktion $f$).
    
    \item
    Sei $k \in \{1, \dotsc, n\}$ mit $w_k = b$.
    Bob sendet $z_1, \dotsc, z_k, z_{k+1} + 1, \dotsc, z_n + 1$ an Alice.
    
    \item
    Es gilt nun $a \le b$ genau dann, wenn $f(x)$ in der Folge vorkommt, die Alice erhält.
\end{enumerate}

\linie

\textbf{Satz (Korrektheit)}:
Der Algorithmus arbeitet korrekt.

\begin{Beweis}
    Sei $j \in \{1, \dotsc, n\}$ mit $w_j = a$.
    Dann gilt $y_j = d_B(d + a) = d_B(c_B(x)) = x$.
    
    "`$\implies$"':
    Sei $a \le b$.
    Dann ist $j \le k$, d.\,h. $y_j = x$ kommt in der Folge $y_1, \dotsc, y_k$ vor.
    Damit kommt auch $z_j = f(y_j) = f(x)$ in der Folge $z_1, \dotsc, z_k$ vor.
    
    "`$\impliedby$"':
    Sei $a > b$.
    Dann ist $j > k$, d.\,h. $z_j = f(x)$ kommt in der Folge $z_{k+1}, \dotsc, z_n$ vor.
    Damit kann $f(x)$ nicht in der Folge $z_1, \dotsc, z_k$ vorkommen
    (sonst $z_i = z_j$ für ein $i \le k$).
    In der Folge $z_{k+1} + 1, \dotsc, z_n + 1$ kommt $f(x)$ ebenfalls nicht vor
    (sonst $z_i + 1 = z_j$ für ein $i > k$).
\end{Beweis}

\linie

\textbf{Nachteil}:
Der Algorithmus benötigt evtl. einen großen Datenaustausch.
Dies kann mit einem Divide-and-Conquer-Ansatz behoben werden.

\pagebreak

\subsection{%
    Kaufen von Geheimnissen%
}

\textbf{Kaufen von Geheimnissen}:
Bob und Carol wollen jeweils eines der Geheimnisse $g_1, \dotsc, g_k$ von Alice kaufen
(z.\,B. ein Passwort),
jedoch soll keiner der anderen, auch nicht Alice, erfahren, welche Geheimnisse von
Bob bzw. Carol gekauft wurden.
Alice, Bob und Carol verbünden sich nicht,
aber jeder Einzelne ist gierig, d.\,h. es soll nicht möglich sein, dass Bob oder Carol mehr
als "`ihr"' Geheimnis kaufen können.

Im Folgenden haben $g_1, \dotsc, g_k$ jeweils $n$ Bit
und $g_b$ und $g_c$ seien die Geheimnisse, die Bob bzw. Carol kaufen will.
\begin{enumerate}
    \item
    Alice erzeugt Schlüssel $s_1, s_2$ für ein asymmetrisches Kryptosystem und schickt
    die öf"|fentlichen Verschlüsselungsfunktionen $c_{s_1}$ und $c_{s_2}$ an Bob bzw. Carol.
    
    \item
    Bob und Carol erzeugen jeweils $k$ Zufallszahlen $z_1, \dotsc, z_k$ bzw.
    $z_1', \dotsc, z_k'$ mit je $n$ Bit.
    Bob schickt $z_1, \dotsc, z_k$ an Carol und
    Carol schickt $z_1', \dotsc, z_k'$ an Bob.
    
    \item
    Bob schickt $y' := z_b' \xor c_{s_1}(z_b')$ an Carol und
    Carol schickt $y := z_c \xor c_{s_2}(z_c)$ an Bob.
    
    \item
    Bob und Carol schicken $x_1, \dotsc, x_k$ mit $x_i := y \xor z_i$ bzw.\\
    $x_1', \dotsc, x_k'$ mit $x_i' := y' \xor z_i'$ an Alice.
    
    \item
    Alice schickt $t_1, \dotsc, t_k$ mit $t_i := g_i \xor d_{s_1}(x_i')$ an Bob bzw.\\
    $t_1', \dotsc, t_k'$ mit $t_i' := g_i \xor d_{s_2}(x_i)$ an Carol
    ($d_{s_1}, d_{s_2}$ geheime Entschlüsselungsfunktionen).
    
    \item
    Bob berechnet $g_b = z_b' \xor t_b$ und
    Carol berechnet $g_c = z_c \xor t_c'$.
\end{enumerate}

\linie

\textbf{Satz (Korrektheit)}:
Es gilt $g_b = z_b' \xor t_b$ und $g_c = z_c \xor t_c'$.

\begin{Beweis}
    Es gilt $z_b' \xor t_b = z_b' \xor g_b \xor d_{s_1}(x_b')
    = z_b' \xor g_b \xor d_{s_1}(y' \xor z_b') = z_b' \xor g_b \xor d_{s_1}(c_{s_1}(z_b')) = g_b$.
    Analog gilt $z_c \xor t_c' = g_c$.
\end{Beweis}

\linie

\textbf{Vorteile}:
\begin{itemize}
    \item
    Bob und Carol erhalten jeweils ihr gewünschtes Geheimnis und keine anderen:
    Würde Bob $g_j$ für ein $j \not= b$ ermitteln können, dann wüsste er
    $d_{s_1}(x_j') = g_j \xor t_j$, d.\,h. er hätte $x_j'$ entschlüsselt.
    Damit ist dies genau so schwierig wie das Brechen der Verschlüsselung
    ($x_j'$ kann durch $z_j'$ zufällig gewählt werden).
    
    \item
    Weil Carol weder $b$ noch den geheimen Schlüssel $s_1$ von Bob kennt,
    ist $y'$ zufällig für Carol (durch Verknüpfung mit $z_b'$).
    Analog ist $y$ zufällig für Bob.
    
    \item
    Durch Verknüpfung mit $z_i$ bzw. $z_i'$ sind die $x_i$ und $x_i'$
    sowie die $t_i$ und $t_i'$ zufällig für Alice,
    d.\,h. Alice erfährt nicht, welche Geheimnisse Bob und Carol kaufen.
\end{itemize}

\textbf{Nachteile}:
\begin{itemize}
    \item 
    Wenn Bob und Carol sind verbünden, dann erfahren sie alle Geheimnisse auf einmal,
    indem zunächst Bob $s_1$ an Carol schickt und
    anschließend Carol $x_i' := c_{s_1}(z_i)$ an Alice schickt.
    
    \item
    Wenn Alice und Bob sich verbünden, dann erfahren sie, welches Geheimnis Carol gekauft hat,
    indem Bob $z_1, \dotsc, z_n$ an Alice schickt
    und Alice dann $z_i \xor t_i'$ für $i = 1, \dotsc, n$ berechnet.
    Gilt $z_j \xor t_j' = g_j$ für genau ein $j$, dann ist $j = c$.
    
    \item
    Alice hat als vertrauenswürdige Stelle zu viel Arbeit
    (weil es so wenig vertrauenswürdige Stellen wie möglich geben soll,
    müssen diese entlastet werden).
\end{itemize}

\pagebreak

\subsection{%
    Mentales Pokern%
}

\textbf{mentales Pokern}:
Drei Spieler (ohne Einschränkung) Alice, Bob und Carol wollen eine Pokervariante spielen,
bei der zunächst fünf Karten an jeden Spieler ausgeteilt werden.

Dazu wählen sich die Spieler jeweils einen Schlüssel $s_1, s_2, s_3$
für ein asymmetrisches Kryptosystem, sodass $d_{s_i}(c_{s_j}(x)) = c_{s_j}(d_{s_i}(x))$
für alle Nachrichten $x$ und $i, j = 1, 2, 3$.
\begin{enumerate}
    \item
    Alice erzeugt für jede Karte $i$ eine Nachricht $x_i$
    (mit Zufallsbits zur Prävention von Known-Plaintext-Angrif"|fen)
    und sendet die $y_i := c_{s_1}(x_i)$ in gemischter Reihenfolge an Bob.
    
    \item
    Bob wählt fünf Karten aus, verschlüsselt sie und schickt $c_{s_2}(y_i)$ für jede
    gewählte Karte an Alice.
    
    \item
    Bob schickt die übrigen 47 Karten ohne erneute Verschlüsselung an Carol.
    
    \item
    Carol wählt fünf Karten aus, verschlüsselt sie und schickt $c_{s_3}(y_i)$ für jede
    gewählte Karte an Alice.
    
    \item
    Carol wählt aus den übrigen 42 Karten fünf für Alice aus
    und schickt sie ohne erneute Verschlüsselung an Alice.
    
    \item
    Alice schickt $d_{s_1}(c_{s_2}(y_i)) = c_{s_2}(x_i)$ an Bob und
    $d_{s_1}(c_{s_3}(y_i)) = c_{s_3}(x_i)$ an Carol (jeweils für jede gewählte Karte).
    
    \item
    Bob und Carol berechnen $d_{s_2}(c_{s_2}(x_i)) = x_i$ bzw. $d_{s_3}(c_{s_3}(x_i)) = x_i$.
    
    \item
    Jeder kennt nun seine Karten und keine anderen und es kann geboten werden.
    Nach der Runde werden alle Schlüssel veröf"|fentlicht und jeder überprüft die Kommunikation.
\end{enumerate}

\pagebreak
