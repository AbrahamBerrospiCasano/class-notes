\section{%
    \name{Miller}-\name{Rabin}-Test%
}

\subsection{%
    Verfahren%
}

Der Miller-Rabin-Test ist ein probabilistischer Primzahltest.
Wenn der Algorithmus ausgibt, dass die Zahl zusammengesetzt ist, dann stimmt das auch,
allerdings stimmt die Ausgabe "`wahrscheinlich prim"' nur zu $\ge$ \SI{50}{\percent}.
Der Miller-Rabin-Test baut auf dem Fermat-Test auf.

\textbf{\name{Fermat}-Test}:
Der \begriff{\name{Fermat}-Test} ist ein probabilistischer Primzahltest und läuft wie folgt ab.\\
Gegeben sei $n \in \natural$.
\begin{enumerate}
    \item
    Wähle $a \in \{1, \dotsc, n - 1\}$ zufällig.
    Falls $\ggT(a, n) > 1$ gilt, dann gib "`$n$ nicht prim"' aus.
    
    \item
    Wenn $a^{n-1} \equiv_n 1$, dann gib "`$n$ wahrscheinlich prim"' aus,
    ansonsten "`$n$ nicht prim"'.
\end{enumerate}
Der Test basiert auf dem kleinen Satz von Fermat:
Für $\ggT(a, n) = 1$ und $n$ prim gilt $a^{n-1} \equiv_n 1$.
Ist die Kongruenz also nicht erfüllt, kann $n$ nicht prim sein.

\textbf{\name{Carmichael}-Zahl}:\\
Eine Zahl $n \in \natural$ heißt \begriff{\name{Carmichael}-Zahl}, falls
$n$ zusammengesetzt ist und $\forall_{a \in (\ZnZ)^\ast}\; a^{n-1} \equiv_n 1$.

Es gibt unendlich vieler solcher Zahlen, die kleinste ist $561 = 3 \cdot 11 \cdot 17$.
Bei bestimmten $n$ gibt also der Fermat-Test immer "`$n$ wahrscheinlich prim"' aus,
obwohl $n$ nicht prim ist
(wenn der Test nicht zufällig einen nicht-trivialen Teiler $a$ von $n$ erwischt).
Daher ist der Fermat-Test als Primzahltest ungeeignet.

\linie

\textbf{\name{Miller}-\name{Rabin}-Test}:
Der \begriff{\name{Miller}-\name{Rabin}-Test (MR-Test)} ist ein probabilistischer Primzahltest
und läuft wie folgt ab.
Gegeben sei $n \in \natural$ ungerade mit $n \ge 3$.
\begin{enumerate}
    \item
    Schreibe $n - 1 = 2^\ell u$ mit $u \in \natural$ ungerade und $\ell \in \natural$.
    
    \item
    Wähle $a \in \{1, \dotsc, n - 1\}$ zufällig.
    Falls $\ggT(a, n) > 1$ gilt, dann gib "`$n$ nicht prim"' aus.\\
    Sonst berechne $b := a^u \bmod n$.
    
    \item
    Wenn $b = 1$ ist, dann gib "`$n$ wahrscheinlich prim"' aus.
    
    \item
    Sonst berechne die Folge $(b, b^2, b^{2^2}, \dotsc, b^{2^{\ell-1}})$ in $\ZnZ$.
    
    \item
    Falls $-1$ in dieser Folge vorkommt, dann gib "`$n$ wahrscheinlich prim"' aus,
    ansonst gib "`$n$ nicht prim"' aus.
\end{enumerate}
Der Miller-Rabin-Test ist gewissermaßen eine Verallgemeinerung des Fermat-Tests:
Wenn der MR-Test "`$n$ wahrscheinlich prim"' ausgibt, dann gibt der Fermat-Test dies auch aus
(wenn der Fermat-Test "`$n$ nicht prim"' ausgibt, dann ist $\ggT(a, n) > 1$ oder
$a^{n-1} \not\equiv_n 1$).
Die Umkehrung gilt allerdings nicht,
d.\,h. der MR-Test erkennt mehr zusammengesetzte Zahlen sicher als solche.

\pagebreak

\subsection{%
    Korrektheit%
}

\textbf{Satz (Korrektheit des MR-Tests)}:\\
Wenn der MR-Test "`$n$ nicht prim"' ausgibt, dann ist $n$ nicht prim.

\begin{Beweis}
    Sei $n \ge 3$ prim.
    Dann gibt es kein $a \in \{1, \dotsc, n - 1\}$ mit $\ggT(a, n) > 1$, d.\,h.
    in Schritt \emph{(2)} wird nichts ausgegeben.
    Sei also $a \in (\ZnZ)^\ast$ beliebig.
    Ist $b := a^u \bmod n = 1$, dann gibt der Test "`$n$ wahrscheinlich prim"' aus.
    
    Sei also $b \not= 1$.
    Es gilt $a^{n-1} \equiv_n 1$ nach dem kleinen Satz von Fermat (weil $n$ prim),
    also $1 \equiv_n a^{2^\ell u} \equiv_n b^{2^\ell}$.
    Daher gilt $b^{2^{\ell-1}} \equiv_n \pm 1$
    ($\ZnZ$ Körper wegen $n$ prim).\\
    Ist $b^{2^{\ell-1}} \equiv_n -1$, dann gibt der Test "`$n$ wahrscheinlich prim"' aus.
    Sonst ist $b^{2^{\ell-1}} \equiv_n 1$ und man kann die Argumentation wiederholen.
    Der Test gibt also "`$n$ wahrscheinlich prim"' aus oder es gilt
    $b^{2^\ell} \equiv_n b^{2^{\ell-1}} \equiv_n \dotsb \equiv_n b \equiv_n 1$,
    ein Widerspruch zur Annahme $b \not= 1$.
    Somit gibt der Algorithmus in jedem Fall "`$n$ wahrscheinlich prim"' aus.
\end{Beweis}

\subsection{%
    Zuverlässigkeit%
}

Im Folgenden wird gezeigt, dass für $n$ zusammengesetzt die Wahrscheinlichkeit,
dass ein $a$ gewählt wird, sodass "`$n$ wahrscheinlich prim"' ausgegeben wird,
höchstens \SI{50}{\percent} ist
(in Wahrheit ist diese Fehlerwahrscheinlichkeit sogar höchstens \SI{25}{\percent}).
Bei $m$ Iterationen des Algorithmus ist deswegen die Fehlerwahrscheinlichkeit
höchstens $\frac{1}{2^m}$.
Andersherum ist die Wahrscheinlichkeit, dass $n$ prim ist, wenn $m$-mal
"`$n$ wahrscheinlich prim"' ausgegeben wurde, mindestens $1 - \frac{1}{2^m}$.

\linie

\textbf{Lemma}:
Seien $p \ge 3$ prim und $d \ge 2$.
Dann gilt $(p^{d-1} + 1)^{p^d-1} \not\equiv \pm 1 \pmod{p^d}$.

\begin{Beweis}
    Im Folgenden wird immer modulo $p^d$ gerechnet.
    
    Nach dem Binomischen Lehrsatz gilt
    $(p^{d-1} + 1)^{p^d-1} = \sum_{k \in \natural_0} \binom{p^d - 1}{k} p^{(d-1)k}$.
    Für $k \ge 2$ ist $(d-1)k \ge d$, weil $k \ge 2 \ge \frac{d}{d-1}$ wegen $d \ge 2$.
    Damit gilt $p^d \teilt p^{(d-1)k}$ für $k \ge 2$, d.\,h. $p^{(d-1)k} \equiv 0$ und
    alle Summanden für $k \ge 2$ verschwinden modulo $p^d$.
    
    Daher gilt
    $(p^{d-1} + 1)^{p^d-1} = 1 + (p^d - 1) p^{d-1} \equiv
    1 + (0 - 1)p^{d-1} = 1 - p^{d-1}$.\\
    Wäre $1 - p^{d-1} \equiv 1$, so würde $p^{d-1} \equiv 0$ gelten,
    also $p^d \teilt p^{d-1}$, ein Widerspruch.\\
    Wäre $1 - p^{d-1} \equiv -1$, so würde $p^{d-1} \equiv 2$ gelten,
    also $p^d \teilt (p^{d-1}-2)$,
    ein Widerspruch.
\end{Beweis}

Insbesondere ist $p^d$ für $p \ge 3$ und $d \ge 2$ keine Carmichael-Zahl, weil dann
$p^d$ zusammengesetzt ist und $\ggT(p^{d-1}+1, p^d) = 1$ gilt.

\linie
\pagebreak

\textbf{Satz}:
Seien $n \ge 3$ ungerade und zusammengesetzt, $u \in \natural$ ungerade und $\ell \in \natural$.\\
Definiere $G := \{a \in (\ZnZ)^\ast \;|\; a^{2^\ell u} \equiv_n 1\}$ und
$H := \{a \in G \;|\; a^{2^{\ell'-1} u} \equiv_n \pm 1\}$ mit\\
$\ell' := \min\{k \in \natural_0 \;|\; \forall_{a \in G}\; a^{2^k u} \equiv_n 1\}$.
Dann gilt $H < G < (\ZnZ)^\ast$ mit $H \not= (\ZnZ)^\ast$.

\begin{Beweis}
    Zunächst ist $\ell' > 0$, da $-1 \in G$ (wegen $\ell \ge 1$)
    und $a^{2^k u} = (-1)^u = -1 \not\equiv_n 1$ für $a = -1 \in G$ und $k = 0$
    (wegen $u$ ungerade und $n \not= 2$).
    Außerdem ist $\ell' \le \ell$, weil $\forall_{a \in G}\; a^{2^k u} \equiv_n 1$ für $k = \ell$.
    (Damit ist $\ell' \in \{1, \dotsc, \ell\}$ tatsächlich endlich.)
    
    Es gilt $G < (\ZnZ)^\ast$, weil $1 \in G$,
    $(ab)^{2^\ell u} = a^{2^\ell u} b^{2^\ell u} \equiv_n 1$ und
    $(a^{-1})^{2^\ell u} = (a^{2^\ell u})^{-1} \equiv_n 1$ für $a, b \in G$.
    Analog ist $H < G$, weil $1 \in H$,
    $(ab)^{2^{\ell'-1} u} = a^{2^{\ell'-1} u} b^{2^{\ell'-1} u} \equiv_n (\pm 1)^2 = 1$ und
    $(a^{-1})^{2^{\ell'-1} u} = (a^{2^{\ell'-1} u})^{-1} \equiv_n (\pm 1)^{-1} = \pm 1$ für
    $a, b \in G$.
    
    Für $G \not= (\ZnZ)^\ast$ ist nichts zu zeigen.
    Sei also $G = (\ZnZ)^\ast$.
    Wegen der Minimalität von $\ell'$ existiert ein $a \in G$ mit
    $a^{2^{\ell'-1} u} \not\equiv_n 1$.
    Schreibe $n = r \cdot s$ mit $r, s \ge 3$ und $\ggT(r, s) = 1$.
    Nach dem chin. Restsatz kann $a^{2^{\ell'-1} u} \equiv 1$
    nicht gleichzeitig modulo $r$ und modulo $s$ gelten.
    Sei also oBdA $a^{2^{\ell'-1} u} \not\equiv_r 1$.
    Wähle ein $c \in (\ZnZ)^\ast$ mit $c \equiv_r a$ und $c \equiv_s 1$
    (existiert nach dem chin. Restsatz).
    Dann gilt $c \in G = (\ZnZ)^\ast$, aber $c \notin H$:
    
    Wäre $c^{2^{\ell'-1} u} \equiv_n 1$, dann
    $a^{2^{\ell'-1} u} \equiv_r c^{2^{\ell'-1} u} \equiv_r 1$ (wegen $c \equiv_r a$),
    Widerspruch zu $a^{2^{\ell'-1} u} \not\equiv_r 1$.\\
    Wäre $c^{2^{\ell'-1} u} \equiv_n -1$, dann
    $1 \equiv_s c^{2^{\ell'-1} u} \equiv_s -1$ (wegen $c \equiv_s 1$),
    was nur für $s = 2$ gehen würde.
    
    Damit gilt $H \lneqq G = (\ZnZ)^\ast$.
\end{Beweis}

\linie

Seien $u \in \natural$ ungerade und
$\ell \in \natural$ ab jetzt wieder so gewählt, dass $n-1 = 2^\ell u$.

\textbf{Lemma}:
Seien $n \ge 3$ ungerade und $a \in (\ZnZ)^\ast$.\\
Wenn $a \notin H$ gilt, dann gibt der MR-Test "`$n$ nicht prim"' aus.

\begin{Beweis}
    Sei $b := a^u \bmod n$.
    Wäre $b = 1$, dann wäre $a^{2^{\ell'-1} u} = b^{2^{\ell'-1}} = 1$,
    d.\,h. $a \in H$, ein Widerspruch zur Voraussetzung.
    Daher gilt $b \not= 1$ und in Schritt \emph{(3)} wird nichts ausgegeben.
    
    Ist $1 \in \{b, b^2, b^{2^2}, \dotsc, b^{2^{\ell-1}}\}$ modulo $n$,
    dann gilt $a \in G$,
    da dann $a^{2^\ell u} = (a^u)^{2^\ell} \equiv_n b^{2^\ell} \equiv_n 1$.
    Damit gilt $a \in G \setminus H$ und
    $b^{2^{\ell'-1}} \not\equiv_n \pm 1$.
    Daraus folgt $b^{2^k} \not\equiv_n -1$ für alle $k \in \{0, \dotsc, \ell' - 1\}$.
    Nach Def. von $\ell' \le \ell$ gilt außerdem $b^{2^{\ell'}} \equiv_n 1$.
    Daraus folgt $b^{2^k} \equiv_n 1$ für alle $k \in \{\ell', \dotsc, \ell - 1\}$.
    In jedem Fall gilt $b^{2^k} \not\equiv_n -1$ für alle $k \in \{0, \dotsc, \ell - 1\}$.
    
    Ist $1 \notin \{b, b^2, b^{2^2}, \dotsc, b^{2^{\ell-1}}\}$ modulo $n$,
    dann ist $b^{2^k} \not\equiv_n -1$ für $k \in \{0, \dotsc, \ell-2\}$.
    Zusätzlich gilt $b^{2^{\ell-1}} \not\equiv_n -1$,
    da sonst $a \in G$
    (wegen $a^{2^\ell u} \equiv_n (b^{2^{\ell-1}})^2 \equiv_n (-1)^2 = 1$)
    und man wie eben argumentieren kann.
    
    Es gilt also immer, dass $-1$ nicht in der Folge $(b, b^2, b^{2^2}, \dotsc, b^{2^{\ell-1}})$
    in $\ZnZ$ vorkommt.
    Damit gibt der Algorithmus "`$n$ nicht prim"' aus.
\end{Beweis}

\linie

\textbf{Satz (Zuverlässigkeit des MR-Tests)}:
Sei $n \ge 3$ ungerade und zusammengesetzt.\\
Dann liefert der MR-Test bei mindestens der Hälfte aller $a \in (\ZnZ)^\ast$
"`$n$ nicht prim"'.

\begin{Beweis}
    Seien $n-1 = 2^\ell u$ und $G$ und $H$ wie im obigen Satz.
    Nach demselben Satz gilt $H \lneqq (\ZnZ)^\ast$,
    d.\,h. $[(\ZnZ)^\ast : H] \ge 2$.
    Für mindestens der Hälfte aller $a \in (\ZnZ)^\ast$ gilt daher $a \notin H$.
    Nach obigem Lemma gibt der MR-Test für diese $a$ "`$n$ nicht prim"' aus.
\end{Beweis}

\linie

\textbf{Faktorisierung}:
Gilt $b^{2^k} \equiv_n 1$ für ein $k \in \{1, \dotsc, \ell - 1\}$,
dann kommt nach dem Beweis des letzten Lemmas
keine $-1$ in der Folge vor und man kann $n$ sogar faktorisieren.\\
Sei dazu $c := (b^{2^{k-1}} \bmod n) \not= 1$ mit $b^{2^k} \equiv_n 1$.
Dann gilt $c^2 \equiv_n 1$ sowie $c \not\equiv_n \pm 1$,
d.\,h. $(c-1)(c+1) \equiv_n 0$ bzw. $n \teilt (c-1)(c+1)$,
aber $n \notteilt (c-1)$ und $n \notteilt (c+1)$.\\
Damit folgt $1 < \ggT(c-1, n), \ggT(c+1, n) < n$.

\pagebreak
