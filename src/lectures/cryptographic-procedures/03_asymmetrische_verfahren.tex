\section{%
    Asymmetrische Verschlüsselungsverfahren%
}

\subsection{%
    RSA-Verfahren%
}

\subsubsection{%
    Verfahren%
}

\textbf{RSA-Verfahren}:
Das \begriff{RSA-Verfahren} wurde 1977 von Rivest, Shamir und Adleman als das erste
Public-Key-Verfahren entwickelt und basiert auf dem Problem der Faktorisierung von großen Zahlen.
RSA ist weit verbreitet, zum einen liegt das daran, dass es sich bei der Faktorisierung
um ein viel untersuchtes Verfahren handelt, zum anderen kann gezeigt werden,
dass Faktorisierung auf die Berechnung des geheimen Schlüssels reduzierbar ist.

\textbf{Schlüsselgenerierung}:
\begin{enumerate}
    \item
    Wähle zwei große Primzahlen $p \not= q$ (zufällig und stochastisch unabhängig).

    \item
    Berechne $n := pq$ (\begriff{RSA-Modul}) und $\varphi(n) = (p - 1)(q - 1)$.

    \item
    Wähle $1 < e < \varphi(n)$ mit $\ggT(e, \varphi(n)) = 1$.

    \item
    Berechne $s \in \integer/\varphi(n)\integer$ mit $es \equiv 1 \pmod{\varphi(n)}$.

    \item
    Veröf"|fentliche $k_e := (n, e)$ und halte $k_s := (n, s)$ geheim.
\end{enumerate}
$p$, $q$ und $\varphi(n)$ werden nicht mehr benötigt und können gelöscht werden.
Allerdings lässt sich beim Entschlüsseln Zeit sparen, wenn man $p$ und $q$ im Speicher behält
(indem man zunächst modulo $p$ und $q$ rechnet und den chinesischen Restsatz anwendet).
$e$ wird aus Ef"|fizienzgründen oft klein gewählt, z.\,B. als die Primzahl $2^{16} + 1 = 65537$.

\textbf{Verschlüsselung}:
Eine Nachricht $x \in \ZnZ$ wird durch $y := x^e \bmod n$ verschlüsselt.

\textbf{Entschlüsselung}:
Eine Nachricht $y \in \ZnZ$ wird durch $z := y^s \bmod n$ entschlüsselt.

Aufgrund der Korrektheit eines kryptografischen Verfahrens gilt $d_{k_s}(c_{k_e}(x)) = x$ immer.
Bei RSA gilt allerdings auch $c_{k_e}(d_{k_s}(y)) = y$, d.\,h.
$k_e$ und $k_s$ sind theoretisch austauschbar
(in der Praxis allerdings nicht zu empfehlen, weil $e$ oft klein ist).\\
Bei RSA gilt außerdem $d_{k_s}(c_{k'_e}(c_{k_e}(x))) = c_{k'_e}(x)$,
d.\,h. eine Verschlüsselung kann wieder "`aufgehoben"' werden, obwohl die Nachricht ein zweites
Mal verschlüsselt wurde.

\subsubsection{%
    Korrektheit%
}

\textbf{Satz (Korrektheit des RSA-Verfahrens)}:
Es gilt $z = x$.

\begin{Beweis}[Beweis der Korrektheit]
    Es gilt $z \equiv_n y^s \equiv_n (x^e)^s$ und nach dem chinesischen Restsatz daher
    $z \equiv_p x^{es} = x^{1+k(p-1)(q-1)}$ für ein $k \in \integer$.
    Für $x \equiv_p 0$ (also $p \teilt x$) gilt $x^{1+k(p-1)(q-1)} \equiv_p 0$.
    Für $x \not\equiv_p 0$ gilt
    $x^{1+k(p-1)(q-1)} = x \cdot (x^{p-1})^{k(q-1)} \equiv_p x \cdot 1 = x$
    wegen $x^{p-1} \equiv_p 1$ (kleiner Satz von Fermat).
    In jedem Fall gilt $x^{1+k(p-1)(q-1)} \equiv_p x$, also $z \equiv_p x$.
    Analog zeigt man $z \equiv_q x$.\\
    Nach dem chinesischen Restsatz gilt $z \equiv_n x$.
    Wegen $x, z \in \ZnZ$ folgt $x = z$.
\end{Beweis}

\pagebreak

\subsubsection{%
    Sicherheit%
}

Der folgende Satz zeigt, dass es genauso schwierig ist, den geheimen Schlüssel $(n, s)$
aus dem öf"|fentlichen Schlüssel $(n, e)$ zu berechnen, wie $n$ zu faktorisieren.
Dabei ist eine Richtung klar:
Ist $n$ in $p \cdot q$ faktorisiert, so kann man wie bei der Schlüsselgenerierung $s$ berechnen.
Die andere Richtung besagt, dass man aus der Kenntnis von $s$ den RSA-Modul $n$ ef"|fizient
faktorisieren kann.
Wenn man nun davon ausgeht, dass Faktorisierung schwierig ist, dann ist auch die Berechnung von $s$
aus $(n, e)$ schwierig (sonst könnte man ja Faktorisierung ef"|fizient durchführen).
Allerdings heißt das nicht, dass das RSA-Verfahren an sich sicher ist:
Es könnte z.\,B. sein, dass $(n, s)$ zwar nicht aus $(n, e)$ ef"|fizient berechnet werden kann,
es aber eine Entschlüsselungsmethode gibt, die $(n, s)$ gar nicht benötigt
(oder nur Teilinformationen).

\textbf{Satz (Sicherheit des geheimen RSA-Schlüssels)}:\\
$p$ und $q$ können ef"|fizient berechnet werden, wenn man $(n, e)$ und $(n, s)$ kennt.

\linie

\textbf{Algorithmus}:
Der Beweis ist konstruktiv und verwendet folgenden Algorithmus:
\begin{enumerate}
    \item
    Schreibe $es - 1 = 2^\ell u$ mit $\ell \in \natural_0$ und $u$ ungerade.

    \item
    Wähle $a \in \{2, \dotsc, n - 1\}$ zufällig und teste, ob $\ggT(a, n) = 1$.
    Falls ja, dann wurde ein Teiler gefunden.
    Falls nicht, so fahre fort.

    \item
    Berechne $\ggT(a^{2^k u} - 1, n)$ für alle $k = 0, \dotsc, \ell - 1$ und brich ab,
    wenn ein nicht-trivialer Teiler gefunden wurde.

    \item
    Falls kein nicht-trivialer Teiler gefunden wurde, dann gehe wieder zu Schritt \emph{(2)}.
\end{enumerate}

\linie

\textbf{Lemma}:
Für $a$ mit $\ggT(a, n) = 1$ gilt $\ord_n(a^u) \in \{2^0, \dotsc, 2^\ell\}$.

\begin{Beweis}
    Wegen $\ggT(a, n) = 1$ ist auch $\ggT(a^u, n) = 1$, d.\,h. $a^u \in (\ZnZ)^\ast$ und
    $\ord_n(a^u)$ ist wohldefiniert.
    Wegen dem Satz von Euler gilt $a^{\varphi(n)} \equiv_n 1$, d.\,h. auch
    $(a^u)^{2^\ell} = a^{es-1} \equiv_n 1$ (wegen $\varphi(n) \teilt (es - 1)$).
    Somit gilt $\ord_n(a^u) \teilt 2^\ell$.
\end{Beweis}

\textbf{Lemma}:
Für $a$ mit $\ggT(a, n) = 1$ und $\ord_p(a^u) \not= \ord_q(a^u)$ gibt es ein
$k \in \{0, \dotsc, \ell - 1\}$, sodass $1 < \ggT(a^{2^k u} - 1, n) < n$.

\begin{Beweis}
    Nach dem Lemma von eben ist $\ord_n(a^u) = 2^m$, d.\,h. $(a^u)^{2^m} \equiv_n 1$.
    Wegen dem chinesischen Restsatz gilt $(a^u)^{2^m} \equiv_p 1$ und
    $(a^u)^{2^m} \equiv_q 1$, d.\,h. $\ord_p(a^u) \teilt 2^m$ und $\ord_q(a^u) \teilt 2^m$.
    Es gilt also $\ord_p(a^u) = 2^k$ und $\ord_q(a^u) = 2^w$ mit $k, w \in \{0, \dotsc, \ell\}$
    und $k \not= w$ nach Voraussetzung, oBdA sei $k < w$.
    Dann gilt $(a^u)^{2^k} \equiv_p 1 \iff p \teilt (a^{2^k u} - 1)$ und
    $(a^u)^{2^k} \not\equiv_q 1 \iff q \notteilt (a^{2^k u} - 1)$,
    weil $2^w$ der kleinste Exponent ist, sodass $a^u$ hoch diesem $\equiv_q 1$ ist
    (aber $2^k < 2^w$).
    Daraus folgt $\ggT(a^{2^k u} - 1, n) = p$, wobei $1 < p < n$.
\end{Beweis}

\linie
\pagebreak

\textbf{Lemma}:
Die Anzahl der Elemente $a \in (\ZnZ)^\ast$, für die $\ord_p(a^u) \not= \ord_q(a^u)$,
ist mindestens $\frac{1}{2} (p-1)(q-1)$.

\begin{Beweis}
    Seien $g_1 \in (\ZpZ)^\ast$ und $g_2 \in (\ZqZ)^\ast$
    Primitivwurzeln modulo $p$ bzw. $q$.
    Dann gibt es nach dem chin. Restsatz ein $g \in (\ZnZ)^\ast$,
    das Primitivwurzel modulo $p$ und modulo $q$ ist.

    Fall 1: $\ord_p(g^u) \not= \ord_q(g^u)$\\
    OBdA sei $\ord_p(g^u) < \ord_q(g^u)$.
    Seien $x \in \{0, \dotsc, p - 2\}$, $y \in \{1, \dotsc, q - 1\}$ mit $y$ ungerade und
    $a \in (\ZnZ)^\ast$ mit $a \equiv_p g^x$ und $a \equiv_q g^y$.

    Dann gilt $\ord_q(a^u) = \ord_q((g^u)^y) = \ord_q(g^u)$.
    Die letzte Gleichung gilt,
    weil $\ord_q(g^u)$ nach dem ersten Lemma und dem chin. Restsatz eine Zweierpotenz ist --
    es gilt immer\\
    $\ord_q((g^u)^y) \teilt \ord_q(g^u)$,
    umgekehrt gilt immer
    $\ord_q(g^u) \teilt y \ord_q((g^u)^y)$
    und wegen $\ord_q(g^u)$ einer Zweierpotenz, aber $y$ ungerade folgt
    $\ord_q(g^u) \teilt \ord_q((g^u)^y)$.

    Für $\ord_p(a^u)$ gilt Ähnliches, allerdings kann $x$ auch gerade sein,
    d.\,h. es gilt nur\\
    $\ord_p((g^u)^x) \teilt \ord_p(g^u)$,
    woraus $\ord_p(a^u) = \ord_p((g^u)^x) \le \ord_p(g^u)$ folgt.

    Insgesamt gilt damit $\ord_p(a^u) \le \ord_p(g^u) < \ord_q(g^u) = \ord_q(a^u)$
    (die mittlere, strikte Ungleichung gilt nach Fallunterscheidung),
    d.\,h. $a$ erfüllt die gewünschte Eigenschaft.
    Für $x$ und $y$ gibt es insgesamt $(p-1) \cdot \frac{q-1}{2}$ Möglichkeiten.
    Weil $g$ eine Primitivwurzel modulo $p$ und modulo $q$ ist, sind die Lösungen
    $a \in (\ZnZ)^\ast$ mit $a \equiv_p g^x$ und $a \equiv_q g^y$
    paarweise verschieden.
    Daher gibt es mindestens $(\frac{1}{2} (p-1)(q-1))$-viele solche $a$.

    Fall 2: $\ord_p(g^u) = \ord_q(g^u)$\\
    Hier wählt man $x$ und $y$ ähnlich wie in Fall 1, nur dass entweder
    $x$ gerade und $y$ ungerade ist oder $x$ ungerade und $y$ gerade ist.
    Mit obigen Argumenten folgt dann\\
    $\ord_p(a^u) < \ord_p(g^u) = \ord_q(g^u) = \ord_q(a^u)$ oder
    $\ord_p(a^u) = \ord_p(g^u) = \ord_q(g^u) > \ord_q(a^u)$.\\
    Somit gibt es $(\frac{p-1}{2} \cdot \frac{q-1}{2} + \frac{p-1}{2} \cdot \frac{q-1}{2}
    = \frac{1}{2} (p-1)(q-1))$-viele mögliche $a$.
\end{Beweis}

\linie

Mit diesen Lemmata kann man nun den Satz beweisen.

\begin{Beweis}[Beweis des Satzes]
    Es gibt $\ge (\frac{1}{2} (p-1)(q-1))$-viele $a \in (\ZnZ)^\ast$ mit
    $\ord_p(a^u) \not= \ord_q(a^u)$.
    Nach Lemma 2 gilt für diese $a$, dass es ein $k \in \{0, \dotsc, \ell - 1\}$ gibt mit
    $1 < \ggT(a^{2^k u} - 1, n) < n$.
    Für ein $a \in (\ZnZ)^\ast$ ist wegen $|(\ZnZ)^\ast| = \varphi(n) = (p-1)(q-1)$
    die Wahrscheinlichkeit, dass ein "`gewünschtes"' $a$ zufällig getrof"|fen wird,
    $\ge \frac{1}{2}$.
    Wegen $\ell \in \O(\log n)$ sind pro $a$ höchstens $\log n$-viele ggT-Berechnungen nötig,
    um ein $a$ zu untersuchen (ggT-Berechnungen können mit dem euklidischen Algorithmus
    ef"|fizient erledigt werden).
    Man kann davon ausgehen, dass der Algorithmus ein gewünschtes $a$ schnell findet,
    da nach $t$ Iterationen die Wahrscheinlichkeit dafür mindestens $1 - \frac{1}{2^t}$
    beträgt.
\end{Beweis}

\subsubsection{%
    Multi-Prime-RSA%
}

\textbf{Multi-Prime-RSA-Verfahren}:
Man kann das RSA-Verfahren auch mit $k$ Primzahlen verwenden (statt den zwei Primzahlen $p, q$).
Auch in diesem Fall arbeitet das Verfahren korrekt
und wird \begriff{Multi-Prime-RSA-Verfahren} genannt.
Das Verfahren arbeitet schneller, wenn der Besitzer der geheimen Schlüssels die Primzahlen im
Speicher behält (weil dann die einzelnen Primzahlen bei gleichem $n$ kleiner sind).
Andererseits ist das Verfahren bei gleichem $n$ unsicherer als das normale RSA-Verfahren,
weil die einzelnen Primzahlen dann deutlich kleiner sind,
d.\,h. die Suche eines Faktors geht wesentlich schneller.
Nach Teilen von $n$ durch diesen Faktor ist die Zahl kleiner und man findet noch schneller
die anderen beiden Faktoren.
Man darf also nicht zu viele Primzahlen wählen, sonst wird das Verfahren unsicher und
die Zeitersparnis bringt nichts.

\pagebreak

\subsection{%
    \name{Rabin}-Verfahren%
}

\subsubsection{%
    Verfahren%
}

\textbf{\name{Rabin}-Verfahren}:
Das \begriff{\name{Rabin}-Verfahren} wurde 1979 von Michael Rabin entwickelt, findet aber
kaum Anwendung, weil die Entschlüsselung im Gegensatz zu RSA nicht eindeutig ist.

\textbf{Schlüsselgenerierung}:
\begin{enumerate}
    \item
    Wähle zwei große Primzahlen $p \not= q$ mit $p \equiv_4 q \equiv_4 3$

    \item
    Berechne $n := pq$.

    \item
    Veröf"|fentliche $k_e := n$ und halte $k_s := (p, q)$ geheim.
\end{enumerate}

\textbf{Verschlüsselung}:
Eine Nachricht $x \in \ZnZ$ wird durch $y := x^2 \bmod n$ verschlüsselt.

\textbf{Entschlüsselung}:
Eine Nachricht $y \in \ZnZ$ wird durch wie folgt entschlüsselt.
Berechne zunächst $x_p := y^{(p+1)/4} \bmod p$ und $x_q := y^{(q+1)/4} \bmod q$.
Dann sind $\pm x_p$ Wurzeln von $y$ modulo $p$ und
$\pm x_q$ Wurzeln von $y$ modulo $q$.
Mit dem chin. Restsatz erhält man nun vier mögliche Kandidaten für $x$.

Weil die Entschlüsselung nicht eindeutig ist, muss man die Menge der möglichen Klartexte
einschränken.
Zum Beispiel können die Kommunikationspartner vereinbaren, dass jede Nachricht mit einem
bestimmten Codewort endet.
Dann ist das Verfahren allerdings nicht mehr so sicher wie die Faktorisierung.
Außerdem muss $p \equiv_4 q \equiv_4 3$ nicht gelten --
dadurch geht aber die Entschlüsselung ef"|fizienter
(sonst müsste man die Quadratwurzeln anderweitig berechnen).

\subsubsection{%
    Korrektheit%
}

\textbf{Quadratzahl/Quadratwurzel}:
$a \in \ZnZ$ heißt \begriff{Quadratzahl} modulo $n$, falls
$\exists_{x \in \ZnZ}\; x^2 \equiv_n a$.
In diesem Fall heißt $x$ \begriff{Quadratwurzel} von $a$ modulo $n$
($x$ ist i.\,A. nicht eindeutig).

\textbf{Lemma (\name{Euler}-Kriterium)}:
Sei $p > 2$ prim.\\
Dann ist die Abbildung $(\ZpZ)^\ast \rightarrow \{1, -1\}$,
$y \mapsto y^{(p-1)/2} \bmod p$ ein surjektiver (multiplikativer) Gruppenhomom. mit
$y$ Quadratzahl modulo $p$ $\iff y^{(p-1)/2} \equiv_p 1$
für $y \in (\ZpZ)^\ast$.

\begin{Beweis}
    Zunächst wird die Äquivalenz gezeigt.

    "`$\implies$"':
    Sei $y \equiv_p x^2$ für ein $x \in \ZpZ$.
    Dann gilt $y^{(p-1)/2} \equiv_p (x^2)^{(p-1)/2} = x^{p-1} \equiv_p 1$
    nach dem kleinen Satz von Fermat.

    "`$\impliedby$"':
    Sei $y \in (\ZpZ)^\ast$ keine Quadratzahl und $g$ eine Primitivwurzel modulo $p$.
    Dann gibt es ein $k \in \natural$ mit $y \equiv_p g^{2k+1}$
    (wäre $y \equiv_p g^{2k}$ für ein $k$, dann wäre $g^k$ eine Quadratwurzel von $y$ modulo $p$).
    Es gilt $(g^{(p-1)/2})^2 - 1 = g^{p-1} - 1 \equiv_p 0$ nach dem kleinen Satz von Fermat,
    d.\,h. $g^{(p-1)/2}$ ist eine Nullstelle von $x^2 - 1 = (x+1)(x-1)$,
    somit gilt $g^{(p-1)/2} \bmod p \in \{1, -1\}$.
    Allerdings gilt $g^{(p-1)/2} \not\equiv_p 1$, weil $\ord((\ZpZ)^\ast) = \ord_p(g) = p - 1$
    der kleinste Exponent ist, sodass $g$ hoch dieser Zahl $\equiv_p 1$ ist.
    Somit gilt $g^{(p-1)/2} \equiv_p -1$.
    Damit erhält man\\
    $y^{(p-1)/2} \equiv_p (g^{2k+1})^{(p-1)/2} = g^{k(p-1)} g^{(p-1)/2}
    \equiv_p (g^{p-1})^k \cdot (-1) \equiv_p -1$
    nach dem kleinen Satz von Fermat.

    Damit wurde bereits die Wohldefiniertheit der Abbildung gezeigt
    (für $y$ Quadratzahl modulo $p$ ist $y^{(p-1)/2} \equiv_p 1$,
    sonst ist $y^{(p-1)/2} \equiv_p -1$).
    Die Homomorphie ist klar nach Definition.
    Die Surjektivität folgt aus $1^{(p-1)/2} \mod p = 1$ und
    $g^{(p-1)/2} \mod p = -1$ für einen Erzeuger\\
    $g \in (\ZpZ)^\ast$ wie oben.
\end{Beweis}

\linie
\pagebreak

\textbf{Korollar}:
Sei $p > 2$ prim.
Dann sind Quadratzahlen modulo $p$ eine Untergruppe von $(\ZpZ)^\ast$ mit der
Gruppenordnung $\frac{p-1}{2}$.

\begin{Beweis}
    Sei $g \in (\ZpZ)^\ast$ eine Primitivwurzel modulo $p$.
    Aus dem Beweis des Lemmas geht hervor, dass $g^{(p-1)/2} \equiv_p -1$.
    Damit gilt $(g^{(p-1)/2})^k \equiv_p 1$ für $k$ gerade und
    $(g^{(p-1)/2})^k \equiv_p -1$ für $k$ ungerade.
    Für $k$ gerade ist also $g^k$ eine Quadratzahl nach dem Lemma und sonst keine.
    Von den Gruppenelementen $g^1, \dotsc, g^{p-1}$ sind also genau die Quadratzahlen
    modulo $p$, die einen geraden Exponenten haben, d.\,h. genau $\frac{p-1}{2}$.

    Dass die Quadratzahlen modulo $p$
    eine Untergruppe von $(\ZpZ)^\ast$ bilden, folgt elementar
    oder nach dem Lemma (als Urbild der trivialen Untergruppe $\{1\} \subset \{1, -1\}$).
\end{Beweis}

\linie

\textbf{Satz (Korrektheit des \name{Rabin}-Verfahrens)}:
Das Rabin-Verfahren arbeitet korrekt.

\begin{Beweis}
    Seien $p \not= q$ prim mit $p \equiv_4 q \equiv_4 3$,
    $n := pq$,
    $x \in \ZnZ$,
    $y := x^2 \bmod n$ und\\
    $x_p := y^{(p+1)/4} \bmod p$.
    Zu zeigen ist, dass $x \equiv_p x_p$ oder $x \equiv_p -x_p$
    (analog gilt dann $x \equiv_q x_q$ oder $x \equiv_q -x_q$).

    Es gilt $x_p^2 \equiv_p y^{(p+1)/2} = y^{(p-1)/2} \cdot y \equiv_p y \equiv_p x^2$
    nach dem Lemma, weil $y$ ein quadr. Rest modulo $n$ ist
    (d.\,h. nach dem chin. Restsatz auch modulo $p$ und modulo $q$).
    Damit gilt $x \equiv_p \pm x_p$.
\end{Beweis}

\subsubsection{%
    Sicherheit%
}

Beim RSA-Verfahren konnte nur gezeigt werden, dass das Faktorisierungsproblem äquivalent zum
Berechnen des geheimen Schlüssels ist.
Beim Rabin-Verfahren kann man sogar zeigen, dass mit einem Entschlüsselungs-Algorithmus
$n$ faktorisiert werden kann.

\textbf{Satz (Sicherheit des Rabin-Verfahrens)}:\\
Existiert ein ef"|fizienter Entschlüsselungs-Algorithmus, der nur mit Kenntnis von $n$
Geheimtexte entschlüsseln kann, so kann $n$ ef"|fizient faktorisiert werden.

\begin{Beweis}
    Sei $R$ ein ef"|fizienter Algorithmus mit $R(y) := \widehat{x}$, wobei
    $\widehat{x}^2 \equiv_n y$ (wenn $y$ eine Quadratzahl modulo $n$ ist).
    Dann kann $n$ wie folgt ef"|fizient faktorisiert werden:
    Wähle zunächst zufällig $x \in \{1, \dotsc, n - 1\}$.
    Berechne anschließend $y := x^2 \bmod n$ und $\widehat{x} := R(y)$.
    Dann gilt $x^2 \equiv_n \widehat{x}^2$, nach dem chin. Restsatz auch
    $x^2 \equiv_p \widehat{x}^2$ und $x^2 \equiv_q \widehat{x}^2$.
    Weil $\ZpZ$ und $\ZqZ$ Körper sind, gilt
    $x \equiv_p \pm\widehat{x}$ und $x \equiv_q \pm\widehat{x}$.
    Es gibt daher vier gleich wahrscheinliche Fälle
    (das liegt daran, dass $X^2 - y \in (\integer/n\integer)[X]$ vier Nullstellen hat,
    obwohl das Polynom nur quadratisch ist).

    Für $x \equiv_p \widehat{x}, x \equiv_q -\widehat{x}$
    gilt $p \teilt (x - \widehat{x})$ und $q \notteilt (x - \widehat{x})$
    (wäre $q \teilt (x - \widehat{x})$, so wäre
    $ -\widehat{x} \equiv_q x \equiv_q \widehat{x}$),
    daher gilt in diesem Fall $\ggT(x - \widehat{x}, n) = p$.
    Analog gilt im Fall $x \equiv_p -\widehat{x}, x \equiv_q \widehat{x}$,
    dass $\ggT(x - \widehat{x}, n) = q$.
    In diesen beiden Fällen hat man also einen Primteiler gefunden.
    Für $x \equiv_p \widehat{x}, x \equiv_q \widehat{x}$ oder
    $x \equiv_p -\widehat{x}, x \equiv_q -\widehat{x}$ lässt sich dagegen keine
    Aussage tref"|fen.

    Insgesamt kann man bei zufälliger Wahl von $x$ die Zahl $n$ mit 50-prozentiger
    Wahrscheinlichkeit faktorisieren.
    Durch Iteration des Verfahrens lässt sich daher $n$ ef"|fizient faktorisieren.
\end{Beweis}

Eine Folgerung aus der Sicherheit des Rabin-Verfahrens ist, dass das RSA-Verfahren auch bei
verhältnismäßig kleinen Exponenten $e$ nicht unsicher wird.
Wenn man die Menge der Klartexte einschränkt (z.\,B. müssen die ersten $k$ Bit gleich den
letzten $k$ Bit sein), dann sinkt die Sicherheit des Rabin-Verfahrens.

\pagebreak

\subsection{%
    \name{Diffie}-\name{Hellman}-Schlüsselaustausch%
}

\textbf{\name{Diffie}-\name{Hellman}-Schlüsselaustausch}:
Mit dem \begriff{\name{Diffie}-\name{Hellman}-Schlüsselaustausch} versucht man, das Grundproblem
symmetrischer Verfahren zu lösen,
nämlich der geheime Austausch eines gemeinsamen Schlüssels.
Es ist selbst kein Verschlüsselungsverfahren, wird aber im ElGamal-Verfahren benutzt.

\linie

\textbf{diskreter Logarithmus}:
Seien $G$ eine Gruppe, $g \in G$ und $A \in \erzeugnis{g}$.\\
Dann heißt $a \in \natural$ mit $A = g^a$ \begriff{diskreter Logarithmus} von $A$ bzgl. der
Basis $g$.

Ist $\ord(g) = \infty$, dann ist der diskrete Logarithmus $a$ eindeutig.\\
Ist $n := \ord(g) < \infty$, dann ist der diskrete Logarithmus $a$ eindeutig in $\ZnZ$.

\textbf{DL-Problem}:
Seien $G$ eine Gruppe und $g \in G$.\\
Das Problem, zu gegebenem $A \in \erzeugnis{g}$ den diskreten Log. $a$ zu bestimmen,
heißt \begriff{DL-Problem}.

Das DL-Problem lässt sich leicht lösen,
wenn $|G|$ nur kleine Primteiler besitzt oder wenn $\ord(g)$ klein ist.

\linie

\textbf{\name{Diffie}-\name{Hellman}-Schlüsselaustausch}:
Der \begriff{\name{Diffie}-\name{Hellman}-Schlüsselaustausch} ist ein Verfahren zum
Schlüsselaustausch und verläuft zwischen Alice und Bob wie folgt.
\begin{enumerate}
    \item
    Wähle eine endliche Gruppe $G$ und $g \in G$ (öf"|fentlich) und berechne $m := \ord(g)$.

    \item
    Alice wählt zufällig ein $a \in \{1, \dotsc, m - 1\}$ und schickt $A := g^a$ an Bob.

    \item
    Bob wählt zufällig ein $b \in \{1, \dotsc, m - 1\}$ und schickt $B := g^b$ an Alice.

    \item
    Alice berechnet $k_1 := B^a$ und Bob berechnet $k_2 := A^b$.
\end{enumerate}
Es gilt $k := k_1 = k_2 = g^{ab}$, d.\,h. $k$ ist nun der gemeinsame Schlüssel und Alice und Bob
können mit diesem Schlüssel über ein symmetrisches Verschlüsselungsverfahren sicher kommunizieren.

\linie

\textbf{Wahl von $G$ und $g$}:
Nach obiger Bemerkung sollte sowohl $|G|$ einen großen Primteiler besitzen als auch
$\ord(g)$ groß sein.
Die Wahl von $G = (\ZnZ, +)$ mit $g$ einem Erzeuger von $G$
(d.\,h. $\ggT(g, n) = 1$) ist schlecht, denn für
$A = a \cdot g \in \ZnZ$ gilt $a = g^{-1} A$ mit $g^{-1}$ dem mult. Inversen
von $g$ in $(\ZnZ)^\ast$, welches sich mit dem euklidischen Algorithmus leicht bestimmen lässt.

Eine bessere Möglichkeit ist $G = ((\ZpZ)^\ast, \cdot)$ mit $p$ prim.
Dann gilt $|G| = p - 1$,
d.\,h. um sicherzustellen, dass $|G|$ einen großen Primteiler besitzt, kann man
$p := kq + 1$ mit $q$ einer großen Primzahl und $k$ klein und gerade wählen
(sodass $p$ prim ist).\\
Bei der Wahl von $g$ wählt man zunächst $g \in (\ZpZ)^\ast$ zufällig und überprüft
$g^q \equiv_p 1$ sowie $g \not\equiv_p 1$
(dann gilt $\ord_p(g) \teilt q$ und $\ord_p(g) \not= 1$, also $\ord_p(g) = q$).
Gilt $g^q \not\equiv_p 1$, so könnte $\ord_p(g)$ ein Vielfaches von $q$ sein.
In diesem Fall definiert man $g' := g^k$ und überprüft $g'$
(damit gilt in jedem Fall bereits $(g')^q = g^{kq} = g^{|G|} \equiv_p 1$).

\linie

\textbf{Sicherheit}:
Wenn man das DL-Problem ef"|fizient lösen kann, dann kann man auch den
Diffie-Hellman-Schlüsselaustausch ef"|fizient knacken
(\begriff{Diffie-Hellman-Problem}: Berechnen von $k$ aus $G, g, A, B$),
indem man aus $A$ den diskreten Logarithmus $a$ und $k := B^a$ berechnet.
Allerdings ist die Gültigkeit der Umkehrung unbekannt,
d.\,h. es kann sein, dass das DL-Problem "`echt"' schwieriger als das Diffie-Hellman-Problem ist.
Die Sicherheit kann auch durch eine Man-in-the-middle-Attacke sabortiert werden, siehe unten.

\pagebreak

\subsection{%
    \name{ElGamal}-Verfahren%
}

\textbf{\name{ElGamal}-Verfahren}:
Das \begriff{\name{ElGamal}-Verfahren} ist im Prinzip das Diffie-Hellman-Verfahren zum
Schlüsselaustausch mit anschließender Multiplikation des Klartexts mit dem gemeinsamen Schlüssel.

\textbf{Schlüsselgenerierung}:
\begin{enumerate}
    \item
    Bestimme eine endliche Gruppe $G$ mit $n := |G|$, $g \in G$, $a \in \{0, \dotsc, n - 1\}$
    und $A := g^a$.

    \item
    Veröf"|fentliche $k_e := (G, g, A)$ und halte $k_s := a$ geheim.
\end{enumerate}

\textbf{Verschlüsselung}:
Eine Nachricht $x \in G$ wird wie folgt verschlüsselt:\\
Wähle $b \in \{1, \dotsc, n - 1\}$ zufällig.
Berechne $B := g^b$, $k := A^b$ und $y := kx$ und sende $(y, B)$.

\textbf{Entschlüsselung}:
Eine Nachricht $(y, B) \in G^2$ wird wie folgt entschlüsselt:\\
Berechne $k := B^a$ und $x := k^{-1} y$.

\linie

\textbf{Sicherheit}:
Das Brechen des geheimen Schlüssels $a$ ist genau das DL-Problem.\\
Das Entschlüsseln einer Nachricht $(y, B)$ ist genauso schwierig wie das
Diffie-Hellman-Problem
(ist der Klartext $x$ berechenbar, so ist $k := yx^{-1}$ der Schlüssel,
ist der Schlüssel $k$ berechenbar, so ist $x := k^{-1} y$ der Klartext).\\
Für jede Nachricht muss ein anderes $b$ gewählt werden.
Ist nämlich ein einziges Klartext-Geheimtext-Paar $(\widetilde{x}, \widetilde{y})$
bekannt (Known-Plaintext-Attacke),
so können alle Nachrichten, die mit demselben
$b$ verschlüsselt wurden, auch entschlüsselt werden
(indem zunächst $k := \widetilde{y} \widetilde{x}^{-1}$ berechnet wird und alle
folgenden Geheimtexte $y \in G$ mit $x := k^{-1} y$ entschlüsselt werden).

Geheimtexte $(y, B)$ sind zwar doppelt so lang wie bei anderen Verfahren,
da allerdings $b$ sowieso zufällig gewählt werden sollte,
ist eine Zufallskomponente (hier $B$),
die bei anderen Verfahren explizit an jeden Klartext angefügt werden müsste,
hier bereits eingebaut.

\linie

\textbf{Man-in-the-middle-Angriff}:
Das Verfahren ist allerdings, wie bereits der Diffie-Hellman-Schlüsselaustausch,
anfällig gegenüber dem sog. \begriff{Man-in-the-middle-Angriff}.
Dazu schaltet sich eine dritte Person (hier Oscar) zwischen die beiden
Kommunikationspartner (Alice und Bob), fängt die Nachrichten ab,
entschlüsselt sie und verschlüsselt sie wieder.
\begin{align*}
    \xymatrix@R=2mm@C=25mm{
        *+++[F]{\txt{\textbf{Alice}\\($a_1, k_1$)}}\ar[r]^{G_1, g_1, A_1, B_1}&
        *+++[F]{\txt{\textbf{Oscar}\\($b_1, k_1$)}}\ar[l]\ar[r]&
        *+++[F]{\txt{\textbf{Bob}\\($b_2, k_2$)}}\ar[l]_{G_2, g_2, A_2, B_2}\\
        %
        x = d_1(y')&
        \txt{$x = d_2(y)$\\$y' = c_1(x)$}\ar@/^5mm/[l]^{y'}&
        y = c_2(x)\ar@/^5mm/[l]^y
    }
\end{align*}
Oscar kann also nicht nur die gesamte Kommunikation mithören,
sondern könnte auch Nachrichten fälschen, indem er eine selbst gewählte Nachricht $x'$
statt $x$ mit Alices Schlüssel verschlüsseln würde.

\pagebreak

\subsection{%
    \name{Merkle}-\name{Hellman}-Kryptosystem%
}

\textbf{\name{Merkle}-\name{Hellman}-Kryptosystem}:
Das \begriff{\name{Merkle}-\name{Hellman}-Kryptosystem} basiert auf\\
dem Subsetsum-Problem, einem
NP-vollständi"-gen Spezialfall des Rucksack-Problems.

\textbf{Subsetsum-Problem}:
Beim \begriff{Subsetsum-Problem} sind Zahlen $s_1, \dotsc, s_n, y \in \natural$ gegeben.\\
Gefragt ist, ob $I \subset \{1, \dotsc, n\}$ existiert mit $y = \sum_{i \in I} s_i$.

Definiert man die Verschlüsselung von $x = x_1 \dotsc x_n \in \BB^n$ durch
$y = \sum_{i=1}^n x_i s_i$, so bekommt man das Problem, dass Alice ein NP-Problem lösen müsste
und die Entschlüsselung eventuell nicht eindeutig wäre.
Daher setzt man voraus, dass $(s_1, \dotsc, s_n)$ eine stark wachsende Folge ist.

\textbf{stark wachsend}:
Die Folge $s_1, \dotsc, s_n \in \real^+$ heißt \begriff{stark wachsend}, falls
$\forall_{i=1,\dotsc,n}\; s_i > \sum_{k=1}^{i-1} s_k$.

Das Subsetsum-Problem für stark wachsende Folgen ist eindeutig in Linearzeit lösbar,
indem man $s_n, \dotsc, s_1$ durchgeht (was wiederum heißt, dass jeder entschlüsseln könnte).

\linie

\textbf{Schlüsselgenerierung}:
\begin{enumerate}
    \item
    Wähle z.\,B. $n = 100$, $s_1, \dotsc, s_n \in \natural$ mit $s_i$ jeweils $n + i - 1$ Bit und
    $p$ prim mit $2n$ Bit.

    \item
    Wähle $u, w \in (\ZmZ)^\ast$ mit $uw \equiv_p 1$ und eine Perm. $\pi \in \Sigma_n$
    und $a_{\pi(i)} := (s_i u \bmod p)$.

    \item
    Veröf"|fentliche $a_1, \dotsc, a_n$ (jeweils etwa $2n$ Bit)
    und halte $u, w, \pi, s_1, \dotsc, s_n$ geheim.
\end{enumerate}

\textbf{Verschlüsselung}:\\
Eine Nachricht $x_1 \dotsb x_n \in \BB^n$ wird verschlüsselt durch
$y := \sum_{i=1}^n x_i a_i < n 2^{2n}$.

\textbf{Entschlüsselung}:
Eine Nachricht $y$ wird wie folgt entschlüsselt.\\
Man berechnet $y \cdot w = \sum_{i=1}^n x_i a_i w
= \sum_{i=1}^n x_{\pi(i)} a_{\pi(i)} w
\equiv_p \sum_{i=1}^n x_{\pi(i)} s_i uw
\equiv_p \sum_{i=1}^n x_{\pi(i)} s_i$.\\
Nach Wahl von $p$ ist $\sum_{i=1}^n x_{\pi(i)} s_i < p$, d.\,h. es gilt
$(yw \bmod p) = \sum_{i=1}^n x_{\pi(i)} s_i$.
Mit dieser Beziehung kann $x = x_1 \dotsb x_n$ berechnet werden.

\linie

\textbf{Sicherheit}:
Shamir hat 1982 das Merkle-Hellman-Kryptosystem gebrochen,
sodass dieses als unsicher angesehen werden muss.

\pagebreak

\subsection{%
    \name{McEliece}-Kryptosystem%
}

\textbf{\name{McEliece}-Kryptosystem}:\\
Für das \begriff{\name{McEliece}-Kryptosystem} benötigt man etwas Codierungstheorie.
Sei $F := \integer/2\integer$.

\textbf{Code}:
Ein \begriff{(linearer) Code} ist ein Unterraum $C \le F^n$.

\textbf{\name{Hamming}-Distanz}:
Die \begriff{\name{Hamming}-Distanz} von $x, y \in F^n$ ist
$|\{i \in \{1, \dotsc, n\} \;|\; x_i \not= y_i\}|$.\\
Die Hamm.-Dist. eines Codes $C$ ist die minimale Hamm.-Dist. zweier versch. Vektoren.

Haben $x, y \in F^n$ die Hamming-Distanz $2d + 1$, dann können bis zu $2d$ Fehler erkannt werden
und man kann $d$ Fehler korrigieren (erkennen, ob von $x$ oder $y$ gestartet wurde).

\textbf{Generatormatrix}:\\
Eine \begriff{Generatormatrix} ist eine Matrix $G \in F^{m \times n}$ mit $m \le n$ und
$\Rang(G) = m$.\\
Der \begriff{durch $G$ definierte Code} ist $C := \{xG \;|\; x \in F^m\}$.
$xG$ heißt \begriff{Codierung} von $x$.

\textbf{Kontrollmatrix}:\\
Zu einer Gen.matrix $G$ existiert eine Kontrollmatrix $H \in F^{n \times m}$ mit
$C = \{y \in F^n \;|\; yH = 0\}$.\\
$H$ entsteht dabei aus einer Basis des Orthogonalraums.
Die maximale Anzahl von linear unabhängigen Vektoren in $H$ liefert die Hamming-Distanz von $C$.

Ist $yH = 0$ (d.\,h. ist $y \in C$), dann ist $x \in F^m$ mit $xG = y$ leicht zu finden.
Gilt aber $y \notin C$, dann ist es i.\,A. schwierig, die Fehler zu korrigieren und somit zu
decodieren.
Das zugehörige Entscheidungsproblem ist außerdem NP-vollständig.

\linie

\textbf{Schlüsselgenerierung}:
\begin{enumerate}
    \item
    Wähle eine Generatormatrix $G \in F^{m \times n}$, für welche man $t$ Fehler ef"|fizient
    korrigieren kann (d.\,h. die Hamming-Distanz des Codes sollte $\ge 2t + 1$ sein).

    \item
    Wähle eine zufällige Permutationsmatrix $M \in F^{m \times m}$ und eine
    zufällige invertierbare Matrix $N \in F^{n \times n}$.

    \item
    Setze $G' := MGN$, veröf"|fentliche $G', t$ und halte $G, M, N$ geheim.
\end{enumerate}

Die Idee ist nun, dass $G'$ immer noch $t$ Fehler korrigieren kann.

\textbf{Verschlüsselung}:
Eine Nachricht $x \in F^m$ wird durch $y := y' \xor z$ verschlüsselt,
wobei $y' := xG'$ und $z \in F^n$ zufällig mit $t$ Einsen.

\textbf{Entschlüsselung}:
Eine Nachricht $y \in F^n$ wird durch $y'' := yN^{-1}$,
$x'$ der Decodierung von $y''$ für $G$ mit Fehlerkorrektur und
$x := x' M^{-1}$ entschlüsselt.

\linie

\textbf{Sicherheit}:
Das Verfahren gilt als sicher, außerdem sind keine Algorithmen für Quantencomputer bekannt.
Allerdings sind die Schlüssellängen um ein Wesentliches größer als z.\,B. bei RSA
(Faktor $1000$).
NTRU verfolgt einen ähnlichen Ansatz wie McEliece.

\pagebreak
\ifthenelse{\equal{\standalonedoc}{true}}{\addtocontents{toc}{\protect\newpage}}{}
