\section{%
    Kryptografische Hashfunktionen%
}

\subsection{%
    Hash-, Kompressions- und Einwegfunktionen, Kollisionen%
}

Im Folgenden ist $\BB := \{0, 1\}$.

\textbf{Hashfunktion}:
Eine Abb. $h\colon \BB^\ast \to \BB^n$ heißt \begriff{Hashfunktion}.

\textbf{Kompressionsfunktion}:
Eine Abb. $g\colon \BB^m \to \BB^n$ mit $m > n$ heißt \begriff{Kompressionsfunktion}.

Hash- und Kompressionsfunktionen sind niemals injektiv.

\linie

\textbf{Einwegfunktion}:
Sei $f$ eine Abbildung, die sich ef"|fizient berechnen lässt
(d.\,h. für alle $x$ ist $f(x)$ ef"|fizient berechenbar).
Dann heißt $f$ \begriff{Einwegfunktion}, falls für beliebiges $s$ sich $x$ mit $f(x) = s$
nicht ef"|fizient berechnen lässt.

"`Ef"|fizient berechenbar"' heißt hier und im Folgenden "`in Polynomialzeit berechenbar"'.

Es ist ein ungelöstes Problem, ob Einwegfunktionen tatsächlich existieren.
Existieren solche, dann würde $\text{P} \not= \text{NP}$ gelten,
die Umkehrung stimmt allerdings nicht.

Es gibt allerdings Funktionen, zu denen bislang kein ef"|fizienter Algorithmus zur Berechnung der
Umkehrfunktion bekannt ist und von denen man deshalb ausgeht, dass sie Einwegfunktionen sind.
Dazu gehört beispielsweise $f\colon \natural \to G$, $f(n) := g^n$ mit einer Gruppe $G$
und $g \in G$ mit großer Ordnung (die Umkehrfunktion ist der diskrete Logarithmus).

\linie

\textbf{Kollision}:
Sei $h$ eine Abbildung.\\
Dann heißt ein Paar $(x, x')$ mit $h(x) = h(x')$ und $x \not= x'$
\begriff{Kollision} von $h$.

\textbf{kollisionsresistent}:\\
Eine Abbildung $h$ heißt \begriff{kollisionsresistent}, falls sich Kollisionen nicht ef"|fizient
berechnen lassen.

\textbf{Lemma}:
Sei $h$ eine ef"|fizient berechenbare Abbildung.\\
Dann gilt: Ist $h$ kollisionsresistent, dann ist $h$ eine Einwegfunktion.

\begin{Beweis}
    Sei $h$ ef"|fizient berechenbar, aber keine Einwegfunktion.
    Dann kann man Kollisionen $(x, x')$ ef"|fizient wie folgt berechnen:
    Wähle $x'$ zufällig und berechne $s = h(x')$.
    Nach Voraussetzung ist ein $x$ mit $h(x) = s$ ef"|fizient berechenbar.
    Dann gilt mit hoher Wahrscheinlichkeit $x \not= x'$,
    d.\,h. $(x, x')$ ist eine Kollision.
    Damit ist $h$ nicht kollisionsresistent.
\end{Beweis}

\subsection{%
    Kompressionsfunktionen aus Verschlüsselungsfunktionen%
}

\textbf{Kompressionsfunktionen aus Verschlüsselungsfunktionen}:\\
Sei $(c_k)_{k \in \BB^n}$ eine Menge von Verschlüsselungsfunktionen $c_k\colon \BB^n \to \BB^n$
mit Klartext-, Geheim"-text- und Schlüsselmenge $P = C = K := \BB^n$.
Dann definiert $(c_k)_{k \in \BB^n}$ auf folgende kanonische Arten Kompressionsfunktionen
$g_i\colon \BB^{2n} \to \BB^n$ (mit $\BB^n \times \BB^n \cong \BB^{2n}$):
\begin{itemize}
    \item
    $g_1(k, x) := c_k(x) \oplus x$

    \item
    $g_2(k, x) := c_k(x) \oplus x \oplus k$

    \item
    $g_3(k, x) := c_k(x \oplus k) \oplus x$

    \item
    $g_4(k, x) := c_k(x \oplus k) \oplus x \oplus k$
\end{itemize}
Die Güte der $g_i$ hängt vom verwendeten Kryptosystem ab,
aber auch vom Verhältnis von $c$ zu $\oplus$, z.\,B. liefert das One-Time-Pad
(also $c_k(x) := x \oplus k$) keine kollisionsresistente Kompressionsfkt.

\pagebreak

\subsection{%
    \name{Merkle}-\name{Damgård}-Konstruktion%
}

\textbf{\name{Merkle}-\name{Damgård}-Verfahren}:
Mit dem \begriff{\name{Merkle}-\name{Damgård}-Verfahren} kann man
aus einer kollisionsresistenten Kompressionsfunktion $g\colon \BB^{2n} \to \BB^n$
eine kollisionsresistente Hashfunktion $h\colon \BB^\ast \to \BB^n$ konstruieren.
Im Folgenden wird der Einfachheit halber eine Kompressionsfunktion $h\colon \BB^{2^n-1} \to \BB^n$
konstruiert (für $n = 128$ kann man die Hashfunktion auf Wörter bis zur Länge $10^{38}$ Bit
anwenden, was für die Praxis ausreicht).\\
Das Verfahren berechnet $h(x)$ für $x \in \BB^{2^n-1}$ wie folgt:
\begin{enumerate}
    \item
    Definiere $x' := x 0^p$, wobei $p \in \natural_0$ kleinstmöglich so gewählt ist, dass
    $n \teilt |x'|$\\
    (d.\,h. $p := (n - (|x| \bmod n)) \bmod n$).

    \item
    Definiere $\overline{x} = x' \ell_x$ mit $\ell_x \in \BB^n$
    der Codierung der Länge $|x|$ von $x$ in $n$ Bits.

    \item
    Teile $\overline{x} =: x_1 \dotsb x_s$ in Blöcke $x_i \in \BB^n$ der Länge $n$ auf
    (insbesondere gilt $x_s = \ell_x$).

    \item
    Berechne $H_i := g(H_{i-1} x_i) \in \BB^n$ für $i = 1, \dotsc, s$ mit $H_0 := 0^n \in \BB^n$.

    \item
    Gebe $h(x) := H_s$ aus.
\end{enumerate}

\linie

\textbf{Satz (Korrektheit)}:
Das Merkle-Damgård-Verfahren arbeitet korrekt, d.\,h. $h\colon \BB^{2^n-1} \to \BB^n$ ist
kollisionsresistent, wenn $g\colon \BB^{2n} \to \BB^n$ kollisionsresistent ist.

\begin{Beweis}
    Sei $h$ nicht kollisionsresistent,
    d.\,h. es lässt sich $x, y \in \BB^{2^n-1}$ mit
    $x \not= y$ und $h(x) = h(y)$ ef"|fizient berechnen.
    Zu $x$ definiere $\overline{x} = x_1 \dotsb x_s$ und berechne $H_0, \dotsc, H_s$,
    definiere analog $\overline{y} = y_1 \dotsb y_t$ zu $y$ und berechne $G_0, \dotsc, G_t$.
    OBdA sei $s \le t$.
    \begin{itemize}
        \item
        Gilt $H_{s-i-1} \not= G_{t-i-1}$ und $H_{s-i} = G_{t-i}$ für ein
        $i \in \{0, \dotsc, s-1\}$,
        so ist das Paar $(H_{s-i-1} x_{s-i}, G_{t-i-1} y_{t-i})$ eine Kollision von $g$, weil\\
        $H_{s-i-1} x_{s-i} \not= G_{t-i-1} y_{t-i}$ sowie
        $g(H_{s-i-1} x_{s-i}) = H_{s-i} = G_{t-i} = g(G_{t-i-1} y_{t-i})$.

        \item
        Wegen $H_s = h(x) = h(y) = G_t$
        ist die andere Möglichkeit $\forall_{i=0,\dotsc,s}\; H_{s-i} = G_{t-i}$.
        Angenommen, es gilt $\forall_{i=0,\dotsc,s-1}\; x_{s-i} = y_{t-i}$.
        Dann gilt insbesondere $\ell_x = x_s = y_t = \ell_y$, d.\,h.
        $|x| = |y|$ bzw. $s = t$.
        Damit würde aber $x = x_1 \dotsb x_s = y_1 \dotsb y_s = y$ gelten,
        ein Widerspruch zu $x \not= y$.
        Es gibt also ein $i \in \{0, \dotsc, s-1\}$ mit $x_{s-i} \not= y_{t-i}$.
        Dann ist $(H_{s-i-1} x_{s-i}, G_{t-i-1} y_{t-i})$ eine Kollision von $g$, weil
        $H_{s-i-1} x_{s-i} \not= G_{t-i-1} y_{t-i}$ sowie
        $g(H_{s-i-1} x_{s-i}) = H_{s-i} = G_{t-i} = g(G_{t-i-1} y_{t-i})$.
    \end{itemize}
    In jedem Fall wurde ef"|fizient eine Koll. von $g$ berechnet, d.\,h. $g$ ist nicht
    kollisionsresistent.
\end{Beweis}

\pagebreak
