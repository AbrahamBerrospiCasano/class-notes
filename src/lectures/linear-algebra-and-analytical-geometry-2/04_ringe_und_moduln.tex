\chapter{%
    Ringe und Moduln%
}

\section{%
    Kommutative Ringe und \texorpdfstring{$K$}{K}-Algebren: \emph{Setting the Stage}%
}

\begin{Bem}
    Mit der Jordanschen Normalform kann man zu einer Matrix eine ähnliche
    Matrix (Jordansche Normalform) bzw. zu einem Endomorphismus eines
    endlich-dimensionalen Vektorraums eine Basis finden, die sich besonders
    "`gutartig"' verhalten.
    Damit dies jedoch für alle Matrizen/Endomorphismen gilt, muss der
    Grundkörper algebraisch abgeschlossen sein, damit das charakteristische
    Polynom immer in Linearfaktoren zerfällt.
    Man sucht nun nach Alternativen, wenn der Körper nicht algebraisch
    abgeschlossen ist.
    Dafür muss man etwas weiter ausholen und die endlich-erzeugten Moduln
    über Hauptidealringen klassifizieren. \\
    Da $\integer$ ein Hauptidealring ist und die $\integer$-Moduln
    genau die abelsche Gruppen sind, bekommt man dabei als Nebenprodukt
    eine Klassifikation aller endlichen abelschen Gruppen.
\end{Bem}

\begin{Bem}
    Beim \begriff{Klassifikationsproblem} ist eine Struktur durch Axiome
    gegeben (z.\,B. Vektorräume, Moduln, Gruppen usw.).
    Außerdem gibt es strukturerhaltende Abbildungen\\
    (\begriff{Morphismen}),
    mit denen man die Objekte vergleichen kann. \\
    Bei der Klassifizierung aller Objekte der Kategorie muss man dann eine
    Liste von Objekten (\begriff{Prototypen}) angeben, sodass \qquad
    1. die Prototypen paarweise nicht isomorph sind und \\
    2. jedes Objekt der Kategorie isomorph zu einem Prototyp ist. \\
    Beim \begriff{Wiedererkennungsproblem} geht es darum, dass eine Kategorie
    durch eine Liste von Prototypen klassifiziert wurde und nun
    ein Objekt der Kategorie gegeben ist.
    Zu welchem Prototyp aus der Liste ist das Objekt dann isomorph?
\end{Bem}

\begin{Bem}
    Im Folgenden seien $K$ ein Körper und $R$ ein kommutativer Ring
    bzw. eine $K$-Algebra mit Einselement $1 = 1_R$.
\end{Bem}

\begin{Def}{Unterring}
    Sei $S \subseteq R$ mit $S \not= \emptyset$ nicht-leere Teilmenge von $R$.
    Dann ist $S$ ein \begriff{Unterring} von $R$, falls
    $r - s \in S$ und $rs \in S$ für alle $r, s \in S$ gilt.
\end{Def}

\begin{Bem}
    Die erste Bedingung sagt aus, dass $(S, +)$ eine abelsche
    Untergruppe von $(R, +)$ ist. \\
    Ist $1_R \in S$, so ist $1_R = 1_S$ das Einselement von $S$.
    Unterringe müssen jedoch nicht notwendigerweise dasselbe Einselement
    wie $R$ haben, sie müssen nicht einmal ein Einselement besitzen.
    Bspw. ist $2 \integer$ ein Unterring von $\integer$, der kein
    Einselement besitzt.
\end{Bem}

\begin{Def}{Ringhomomorphismus}
    Seien $R$ und $S$ Ringe sowie $f: R \rightarrow S$ eine Abbildung.
    $f$ heißt \begriff{Ringhomomorphismus}, falls
    $f(a + b) = f(a) + f(b)$ und $f(ab) = f(a) f(b)$ für alle $a, b \in R$. \\
    Ist $f(1_R) = 1_S$, so \begriff{erhält $f$ das Einselement}.
    $\ker f = \{r \in R \;|\; f(r) = 0_S\}$ heißt \begriff{Kern}
    und $\im f = \{f(r) \;|\; r \in R\}$ heißt \begriff{Bild} von $f$. \\
    Mono-, Epi- und Isomorphismen sind analog zu Vektorräumen definiert.
\end{Def}

\begin{Lemma}{Kern und Bild}
    Sei $f: R \rightarrow S$ Ringhomomorphismus.
    Dann ist $\ker f$ ein Unterring von $R$ und $\im f$ ein Unterring von $S$.
    Ist $r \in \ker f$ sowie $x \in R$, dann ist $rx = xr \in \ker f$.
\end{Lemma}

\begin{Def}{Ideal}
    Ein Unterring $S$ von $R$ heißt \begriff{Ideal} von R, falls
    $rs \in S$ für alle $s \in S, r \in R$.
\end{Def}

\begin{Def}{Faktorring}
    Sei $I \trianglelefteq R$.
    Dann ist die Menge $R/I = \{a + I \;|\; a \in R\}$ der Restklassen modulo
    $I$ eine abelsche Gruppe bzgl. der Addition
    $(a + I) + (b + I) = (a + b) + I$ mit Nullelement $0 + I$.
    Durch $(a + I)(b + I) = ab + I$ für $a, b \in R$ ist eine Multiplikation
    auf $R/I$ definiert, die $R/I$ zum Ring macht
    (Einselement $1 + I$).
    $R/I$ heißt daher \begriff{Faktorring} von $R$ modulo $I$.
\end{Def}

\begin{Lemma}{kanonische Projektion}
    Sei $I \trianglelefteq R$.
    Dann ist die Abbildung $\pi: R \rightarrow R/I$, $\pi(a) = a + I$
    ein Ringepimorphismus, die sog. \begriff{kanonische Projektion}
    von $R$ auf $R/I$.
    Es gilt $\ker \pi = I$, d.\,h. jedes Ideal von $R$ kommt als Kern
    eines Ringhomomorphismus vor.
\end{Lemma}

\begin{Bem}
    $(0)$ und $R$ sind Ideale von $R$.
    Alle anderen Ideale heißen \begriff{nicht-trivial/echt}. \\
    Sei $f: R \rightarrow S$ Ringhomomorphismus, dann ist $f$ surjektiv
    genau dann, wenn $\im f = S$, und injektiv genau dann,
    wenn $\ker f = (0)$ ist. \\
    Sei $A \subseteq R$, $A \not= \emptyset$ eine nicht-leere Teilmenge
    von $R$.
    Dann ist das von $A$ \begriff{erzeugte Ideal}
    $\aufspann{A} = \bigcap_{I \trianglelefteq R,\; A \subseteq I} I$ das
    kleinste Ideal von $R$, das $A$ als Teilmenge enthält. \\
    Es gilt: $\aufspann{A} = \{\sum_{a \in A} r_a a \;|\;
    r_a \in R \text{ fast alle } 0\}$. \\
    Der Durchschnitt von Idealen von $R$ ist ein Ideal von $R$. \\
    Seien $I, J \trianglelefteq R$ Ideale von $R$.
    Dann ist $I + J = \{a + b \;|\; a \in I,\; b \in J\}$ ein Ideal
    von $R$ (die Summe der Ideale $I$ und $J$), wobei
    $I + J = \aufspann{I \cup J}$ gilt. \\
    Die drei Isomorphiesätze gelten wie oben.
\end{Bem}

\begin{Satz}{Ideal ist maximal $\;\Leftrightarrow\;$ Faktorring ist Körper}
    Sei $I \trianglelefteq R$ ein Ideal von $R$.
    Dann ist $I$ maximal genau dann
    (d.\,h. $I \not= R$ und aus
    $I \subsetneqq J \trianglelefteq R$ folgt $J = R$),
    wenn $R/I$ ein Körper ist.
\end{Satz}

\begin{Def}{endlich erzeugt, Hauptideal, noethersch}
    Ein Ideal $I \trianglelefteq R$ heißt \begriff{endlich erzeugt},
    falls $I = \aufspann{S}$ für eine endliche Teilmenge $S \subseteq R$ ist.
    $S$ heißt dann \begriff{endliches Erzeugendensystem von I}.
    Besteht $S$ aus genau aus einem Element $s$, so heißt $I$ Hauptideal.
    In diesem Fall ist $I = sR = \{sr \;|\; r \in R\}$.
    Ein Ring, in dem alle Ideale endlich erzeugt sind, heißt
    \begriff{noethersch}.
\end{Def}

\begin{Satz}{äquivalente Bedinungen für noethersch} \\
    Sei $R$ ein Ring.
    Dann sind folgende Bedingungen äquivalent: \\
    1. $R$ ist noethersch. \\
    2. Jede aufsteigende Kette $I_1 \subseteq I_2 \subseteq \dotsb$ von Idealen
    von $R$ wird stationär, d.\,h. es gibt ein $n \in \natural$ mit $I_k = I_n$
    für alle $k \ge n$. \\
    3. Jede nicht-leere Menge von Idealen von $R$ besitzt maximale
    Elemente bzgl. der Inklusion.
\end{Satz}

\begin{Def}{Produkt zweier Ideale}
    Seien $I, J \trianglelefteq R$ zwei Ideale.
    Das Produkt $I \cdot J$ ist das Ideal, das von der Menge
    $\{i \cdot j \;|\; i \in I,\; j \in J\}$ erzeugt wird.
    Es gilt $I \cdot J \subseteq I \cap J$.
\end{Def}

\begin{Def}{invertierbar, Einheit}
    Ein Element $a \in R$ heißt \begriff{invertierbar} oder \begriff{Einheit},
    falls es ein $b \in R$ mit $ab = 1$ gibt.
    Das Inverse $b = a^{-1} \in R$ ist dann eindeutig bestimmt und selbst
    invertierbar.
    Die Menge $U(R)$ der invertierbaren Elemente von $R$ bildet unter
    Multiplikation eine Gruppe, die \begriff{Einheitengruppe} von $R$.
\end{Def}

\begin{Def}{Polynomring}
    Sei $R$ ein kommutativer Ring mit Eins.
    Dann besteht der \begriff{Polynomring $R[x]$} aus formalen Summen
    $\sum_{i=0}^n \alpha_i x^i$ mit $n \in \natural_0$, $\alpha_i \in R$ und
    $x$ Unbekannte.
    Ist $p(x) = \sum_{i=0}^n \alpha_i x^i$ und $\alpha_k \not= 0$,
    aber $\alpha_m = 0$ für alle $m > k$, so heißt $k = \deg p(x)$ der
    \begriff{Grad} von $p(x)$. \\
    Die Addition und Multiplikation von zwei Polynomen ist wie
    die Multiplikation von Polynomen mit einem Skalar $\lambda \in R$
    wie üblich definiert
    (es gilt nicht mehr notwendigerweise
    $\deg(p(x)q(x)) = \deg p(x) + \deg q(x)$). \\
    Der \begriff{Polynomring $R[x_1, \dotsc, x_n]$} in den Unbestimmten
    $x_1, \dotsc, x_n$ ist induktiv durch \\
    $R[x_1, \dotsc, x_n] = (R[x_1, \dotsc, x_{n-1}])[x_n]$ definiert. \\
    Er besteht aus formalen Summen
    $\sum_{\mi{i} = (i_1, \dotsc, i_n) \in \natural^n}
    \alpha_{\mi{i}} x_1^{i_1} \dotsm x_n^{i_n}$
    mit $\alpha_{\mi{i}} \in R$ gleich $0$ für fast alle
    $\mi{i} \in \natural^n$.
    Terme der Form $x^{\mi{i}} = x_1^{i_1} \dotsm x_n^{i_n}$
    mit $\mi{i} = (i_1, \dotsc, i_n) \in \natural^n$ heißen
    \begriff{Monome}.
\end{Def}

\begin{Satz}{universelle Eigenschaft von $K[x_1, \dotsc, x_n]$}
    Seien $K$ ein Körper und
    $K[x_1, \dotsc, x_n]$ der Polynomring über $K$.
    Dann hat $K[x_1, \dotsc, x_n]$ folgende universelle Eigenschaft: \\
    Es gibt eine Abbildung
    $\iota: \{1, \dotsc, n\} \rightarrow K[x_1, \dotsc, x_n]$,
    nämlich die Abbildung gegeben durch $\iota(i) = x_i$. \\
    Sei $R$ eine kommutative $K$-Algebra mit Einselement und
    $f: \{1, \dotsc, n\} \rightarrow R$ eine Abbildung.
    Dann gibt es genau einen $K$-Algebrahomomorphismus
    $\widehat{f}: K[x_1, \dotsc, x_n] \rightarrow R$ mit
    $\widehat{f}(x_i) = f(i)$ für $i = 1, \dotsc, n$,
    d.\,h. $\widehat{f} \circ \iota = f$.
\end{Satz}

\begin{Bem}
    Sind also $s_1, \dotsc, s_n \in R$ beliebig, so kann man
    die Abbildung $x_i \mapsto s_i$ zu einem $K$-Algebrahomomorphismus
    $\sum_{\mi{i}} \alpha_{\mi{i}} x^{\mi{i}} \mapsto
    \sum_{\mi{i}} \alpha_{\mi{i}} s^{\mi{i}}$ fortsetzen.
\end{Bem}

\section{%
    Hauptidealringe (HIR)%
}

\begin{Bem}
    Im Folgenden sei $R$ ein kommutativer Ring oder $K$-Algebra mit Eins.
\end{Bem}

\begin{Def}{Nullteiler, Integritätsbereich}
    Ein Element $a \in R$ heißt \begriff{Nullteiler}, falls
    es ein $b \in R$, $b \not= 0$ gibt, sodass $ab = 0$ ist. \\
    Besitzt $R$ außer $0$ keinen Nullteiler, so heißt
    $R$ \begriff{Integritätsbereich} oder \begriff{nullteilerfrei}.
\end{Def}

\begin{Def}{Quotientenkörper}
    Sei $R$ ein Integritätsbereich. \\
    Auf der Menge $\{(a, b) \in R \times R \;|\; b \not= 0\}$ definiert
    man eine Äquivalenzrelation durch \\
    $(a, b) \sim (c, d) \;\Leftrightarrow\; ad = bc$.
    Die Äquivalenzklasse von $(a, b)$ wird mit $\frac{a}{b}$
    bezeichnet. \\
    Auf der obigen Menge kann man mit $a, b, c, d \in R$, $b, d \not= 0$
    eine Addition und Multiplikation definieren durch
    $\frac{a}{b} + \frac{c}{d} = \frac{ad + bc}{bd}$ und
    $\frac{a}{b} \cdot \frac{c}{d} = \frac{ac}{bd}$. \\
    Damit wird $K = \{\frac{a}{b} \;|\; a, b \in R,\; b \not= 0\}$ ein
    Körper, der sog. \begriff{Quotientenkörper} $Q(R)$ von $R$. \\
    Die Abbildung $R \rightarrow K$, $r \mapsto \frac{r}{1}$ ist ein
    injektiver Ringhomomorphismus, sodass man $R$ als Unterring des Körpers
    $K = Q(R)$ betrachten kann.
\end{Def}

\begin{Def}{Hauptidealring}
    Ein Integritätsbereich $R$ heißt \begriff{Hauptidealring (HIR)}, falls
    jedes Ideal von $R$ ein Hauptideal ist.
\end{Def}

\begin{Def}{\textsc{Euklid}ische Ringe}
    Ein Integritätsbereich $R$ heißt \begriff{\textsc{Euklid}ischer Ring},
    falls es eine Abbildung (Gradfunktion)
    $\deg: R \rightarrow \natural_0 \cup \{-1\}$ gibt, sodass \\
    1. für alle $r \in R$ mit $r \not= 0$ gilt, dass $\deg 0 < \deg r$, und \\
    2. für alle $f, g \in R$ mit $g \not= 0$ es $q, r \in R$ mit
    $\deg r < \deg g$ gibt, sodass $f = q \cdot g + r$ ist.
\end{Def}

\begin{Bem}
    Beispiele für Euklidische Ringe sind $\integer$ mit $\deg z = |z|$
    sowie $K[x]$.
\end{Bem}

\begin{Satz}{ERs sind HIRs}
    Euklidische Ringe sind Hauptidealringe.
\end{Satz}

\begin{Kor}
    Also sind auch $\integer$ und $K[x]$ Hauptidealringe.
\end{Kor}

\begin{Def}{Teilbarkeit}
    Seien $R$ ein Integritätsbereich und $a, b \in R$. \\
    Dann \begriff{teilt} $a$ $b$, d.\,h. $a|b$, falls es ein $c \in R$ mit
    $b = ca$ gibt.
    Es gilt $a|b \;\Leftrightarrow bR \subseteq aR$.
\end{Def}

\begin{Def}{assoziiert}
    Seien $R$ ein Integritätsbereich und $a, b \in R$. \\
    Dann heißen $a$ und $b$ \begriff{assoziiert}, falls es eine Einheit
    $u \in U(R)$ gibt mit $a = bu$.
\end{Def}

\begin{Lemma}{assoziiert}
    Sei $R$ ein Integritätsbereich.
    Dann ist "`assoziiert sein"' eine Äquivalenzrelation
    und $a, b \in R$ sind assoziiert $\;\Leftrightarrow\; aR = bR
    \;\Leftrightarrow\; a|b \;\land\; b|a$.
\end{Lemma}

\begin{Def}{ggT und kgV}
    Seien $R$ ein Integritätsbereich und $a, b \in R$. \\
    $c \in R$ heißt \begriff{größter gemeinsamer Teiler} von $a$ und $b$,
    falls $c|a$ und $c|b$ sowie
    für $d \in R$ mit $d|a$ und $d|b$ auch $d|c$ gilt.
    Der größte gemeinsame Teiler $\ggT(a, b)$ von $a$ und $b$ ist,
    falls er existiert, bis auf Assoziiertheit eindeutig bestimmt. \\
    $c \in R$ heißt \begriff{kleinstes gemeinsames Vielfaches} von $a$ und $b$,
    falls $a|c$ und $b|c$ sowie
    für $d \in R$ mit $a|d$ und $b|d$ auch $c|d$ gilt.
    Das kleinste gemeinsame Vielfache $\kgV(a, b)$ von $a$ und $b$ ist,
    falls es existiert, bis auf Assoziiertheit eindeutig bestimmt. \\
    Ist $R$ ein HIR, dann existieren $\ggT(a, b)$ und $\kgV(a, b)$ und es
    gilt \\
    $aR + bR = \ggT(a, b)R$, \qquad
    $aR \cap bR = \kgV(a, b)R$ \quad sowie \quad
    $(aR) \cdot (bR) = abR$.
\end{Def}

\begin{Bem}
    Teilbarkeit ist eine Ordnungsrelation auf der Menge der Assoziiertenklassen
    von $R$, nicht auf $R$ selbst.
\end{Bem}

\pagebreak

\begin{Def}{Primideal}
    Seien $R$ ein kommutativer Ring mit Eins und $P \trianglelefteq R$. \\
    Dann heißt $P$ \begriff{Primideal}, falls
    für alle $x, y \in R$ mit $xy \in P$ gilt, dass $x \in P$ oder $y \in P$
    ist.
\end{Def}

\begin{Satz}{Primideale}
    $R$ ist ein Integritätsbereich genau dann, wenn $(0)$ ein Primideal ist. \\
    Sei $P \trianglelefteq R$.
    Dann ist $P$ ein Primideal genau dann, wenn $R/P$ ein Integritätsbereich
    ist.
\end{Satz}

\begin{Kor}
    Sei $M$ ein maximales Ideal von $R$.
    Dann ist $M$ ein Primideal.
\end{Kor}

\begin{Def}{irreduzibel, Primelement}
    Seien $R$ ein kommutativer Ring mit Eins und $a \in R$ mit $a \not= 0$. \\
    $a \not= 0$ heißt \begriff{irreduzibel}, falls $a$ eine Nicht-Einheit und
    nicht als Produkt zweier Nicht-Einheiten darstellbar ist,
    d.\,h. $a \notin U(R)$ sowie für alle $x, y \in R$ mit
    $a = xy$ gilt $x \in U(R)$ oder $y \in U(R)$. \\
    $a \not= 0$ heißt \begriff{Primelement}, falls $aR$ ein Primideal ist,
    d.\,h. aus $a|xy$ folgt $a|x$ oder $a|y$.
\end{Def}

\begin{Satz}{in Integritätsbereichen sind Primelemente irreduzibel} \\
    Seien $R$ ein Integritätsbereich und $p \in R$ Primelement.
    Dann ist $p$ irreduzibel.
\end{Satz}

\begin{Def}{Zerlegung in irreduzible Faktoren}
    Seien $R$ ein kommutativer Ring mit Eins und $a \in R$ mit $a \not= 0$. \\
    Dann besitzt $a \not= 0$ eine \begriff{Zerlegung in irreduzible Faktoren},
    falls $a = \varepsilon \pi_1 \dotsm \pi_r$ mit $\varepsilon \in U(R)$,
    $\pi_i \in R$ irreduzible Elemente und
    $r \in \natural_0$. \\
    $a \not= 0$ besitzt eine
    \begriff{eindeutige Zerlegung in irreduzible Faktoren},
    falls zusätzlich gilt: \\
    Ist $a = \varepsilon' \pi_1' \dotsm \pi_s'$
    mit $\varepsilon' \in U(R)$, $\pi_i' \in R$ irreduzibel und
    $s \in \natural_0$ eine weitere solche Zerlegung, dann ist
    $s = r$ und nach Umnummerierung sind die $\pi_i$ und $\pi_i'$ assoziiert
    ($i = 1, \dotsc, r$), d.\,h. es gibt
    $\varepsilon_1, \dotsc, \varepsilon_r \in U(R)$ mit
    $\pi_i' = \varepsilon_i \pi_i$ für $i = 1, \dotsc, r$.
\end{Def}

\begin{Def}{faktoriell}
    Ein Integritätsbereich heißt \begriff{faktoriell (UFD,
    \emph{unique factorisation domain})}, falls jedes Element von $R$ ungleich
    $0$ eine eindeutige Zerlegung in irreduzible Faktoren besitzt.
\end{Def}

\begin{Satz}{in UFDs stimmen irreduzible und Primelemente überein} \\
    Sei $R$ faktoriell und $p \in R$ irreduzibel. \\
    Dann ist $p$ ein Primelement, d.\,h. für UFDs stimmen irreduzible und
    Primelemente überein.
\end{Satz}

\begin{Satz}{Kriterium für UFD}
    Sei $R$ ein Integritätsbereich.
    Dann ist $R$ UFD genau dann, wenn \\
    1. jede aufsteigende Kette von Hauptidealen stationär wird und \\
    2. jedes irreduzible Element von $R$ Primelement ist.
\end{Satz}

\begin{Satz}{in HIRs sind irreduzible Elemente Primelemente}
    Sei $R$ ein Hauptidealring. \\
    Dann ist jedes irreduzible Element von $R$ ein Primelement.
\end{Satz}

\begin{Satz}{HIRs sind UFDs}
    Hauptidealringe sind UFDs.
\end{Satz}

\begin{Satz}{Primideale von HIRs sind maximal}
    Sei $R$ ein Hauptidealring.
    Dann ist jedes Primideal $P \not= (0)$ von $R$ maximal und daher ist
    $R/P$ ein Körper.
\end{Satz}

\pagebreak

\section{%
    Moduln%
}

\begin{Def}{Modul}
    Sei $A$ ein Ring mit Einselement oder eine $K$-Algebra mit einem Körper
    $K$.
    Ein \begriff{$A$-Linksmodul} ist eine abelsche Gruppe $(M, +)$ zusammen mit
    einer äußeren binären Operation $A \times M \rightarrow M$,
    $(a, m) \mapsto am$, sodass \\
    M1) $1_A m = m$ \qquad\qquad\qquad\;\;
    M2) $a(bm) = (ab)m$ \\
    M3) $(a + b)m = am + bm$ \qquad
    M4) $a(m_1 + m_2) = am_1 + am_2$ \\
    für alle $a, b \in A$ und $m, m_1, m_2 \in M$ gilt. \\
    Analog wird ein \begriff{$A$-Rechtsmodul} definiert
    (Operation $M \times A \rightarrow M$, $(m, a) \mapsto ma$). \\
    Man kann auch Moduln für Ringe ohne Einselement betrachten oder
    Moduln, bei denen $1_A$ nicht notwendigerweise wie die Eins operiert,
    d.\,h. M1) entfällt.
    Will man betonen, dass M1) immer gilt, so spricht man von
    einem \begriff{unitalen Modul}. \\
    Im Folgenden ist ein $A$-Modul immer ein unitaler $A$-Linksmodul.
\end{Def}

\begin{Satz}{Linksmodul als Rechtsmodul und Vektorraum}
    Ist $R$ kommutativer Ring mit Eins und $M$ ein $R$-Linksmodul,
    so wird $M$ zum $R$-Rechtsmodul, indem man $M \times A \rightarrow M$,
    $(m, a) \mapsto am$ definiert.
    Bei nicht-kommutativen Ringen gilt dies i.\,A. nicht, da
    dann M2) verletzt ist. \\
    Ist $A$ eine $K$-Algebra und $M$ ein $A$-Linksmodul,
    so wird $M$ zum $K$-Vektorraum mit \\
    $\lambda m = (\lambda \cdot 1_A) m$ für $\lambda \in K$, $m \in M$.
\end{Satz}

\begin{Satz}{abelsche Gruppe sind $\integer$-Moduln}
    Sei $(M, +)$ eine abelsche Gruppe. \\
    Dann wird $M$ zum $\integer$-Modul mit
    $z \cdot m = m + \dotsb + m$ ($z$-mal) für $z > 0$,
    $z \cdot m = -(m + \dotsb + m)$ ($-z$-mal) für $z < 0$
    und $z \cdot m = 0$ für $z = 0$.
    Umgekehrt ist jeder $\integer$-Modul eine abelsche Gruppe nach Definition.
    Macht man diese zu einem $\integer$-Modul, so erhält man die
    ursprüngliche $\integer$-Modulstruktur zurück.
    Daher sind die $\integer$-Moduln genau die abelschen Gruppen.
\end{Satz}

\begin{Def}{Darstellung}
    Homomorphismen $f: A \rightarrow \End(M,+)$ für Ringe
    und $f: A \rightarrow \End_K(M)$ für $K$-Algebren $A$ heißen
    \begriff{(lineare) Darstellungen} von $A$. \\
    Seien $A$ ein Ring mit Eins und $M$ ein $A$-Modul.
    Für $a \in A$ definiert man $f_a: M \rightarrow M$, $m \mapsto am$.
    Dann ist $f_a \in \End(M,+)$ und $F: A \rightarrow \End(M,+)$,
    $a \mapsto f_a$ ist ein Ringhomomorphismus, der die Eins enthält.
    Ist $A$ eine $K$-Algebra, so ist $f_a \in \End_K(M)$ und $F$ ist
    $K$-Algebrahomomorphismus.
    $F$ heißt \begriff{die zum $A$-Modul $M$ gehörende Darstellung} von $A$. \\
    Darstellungen und Moduln sind völlig äquivalente Konzepte.
\end{Def}

\begin{Def}{trivialer Modul}
    Der Nullmodul $(0)$ ist ein $A$-Modul mit $A$-Operation
    $a \cdot 0 = 0$ für alle $a \in A$.
    Er heißt \begriff{trivialer Modul}.
\end{Def}

\begin{Def}{regulärer Modul}
    $A$ wird zum $A$-Modul ${}_A A$, wobei $a \in A$ auf $A$ durch
    die normale Linksmultiplikation operiert.
    Er heißt \begriff{regulärer Modul}.
\end{Def}

\begin{Kor}
    Jedes Linksideal und daher auch jedes Ideal von $A$ ist $A$-Modul.
\end{Kor}

\begin{Def}{Modulhomomorphismus}
    Seien $A$ ein Ring mit Eins und $M, N$ $A$-Moduln. \\
    Eine Abbildung $f: M \rightarrow N$ heißt
    \begriff{($A$-Modul-)Homomorphismus}, falls $f$ ein
    Homomorphismus der abelschen Gruppen $(M,+)$ und $(N,+)$ ist,
    der zusätzlich die $A$-Operation respektiert, d.\,h.
    $f(am) = af(m)$ für alle $a \in A$, $m \in M$. \\
    $\ker f = \{m \in M \;|\; f(m) = 0_N\}$ heißt \begriff{Kern},
    $\im f = \{f(m) \;|\; m \in M\}$ heißt \begriff{Bild} von $f$. \\
    Injektive/surjektive/bijektive Homomorphismen heißen
    Mono-/Epi-/Isomorphismen. \\
    $M$ und $N$ heißen \begriff{isomorph}
    ($M \cong N$), falls es einen Isomorphismus $f: M \rightarrow N$ gibt.
\end{Def}

\pagebreak

\begin{Bem}
    Seien $A$ ein Ring mit Eins und $M, N$ $A$-Moduln. \\
    \textbf{Untermodul}:
    Eine nicht-leere Teilmenge $U \subseteq M$, $U \not= \emptyset$
    heißt \begriff{Untermodul} von $M$, falls $(U,+)$ abelsche Untergruppe
    von $(M,+)$ ist und $a \cdot u \in U$ für alle $a \in A$, $u \in U$ ist.
    Man schreibt dann $U \ur M$. \\
    Die $A$-Untermoduln von ${}_A A$ sind genau die Linksideale von $A$. \\
    \textbf{Durchschnitt von Untermoduln}:
    Der Durchschnitt von Untermoduln von $M$ ist wieder ein Untermodul von $M$.
    Dabei handelt es sich um den größten Untermodul von $M$, der in allen
    geschnittenen Untermoduln enthalten ist. \\
    \textbf{Aufspann einer Teilmenge}:
    Sei $S \subseteq M$.
    Der von $S$ \begriff{erzeugte Untermodul} $U = \aufspann{S}$ ist definiert
    als $\bigcap_{U \ur M,\; U \supseteq S} U$, der eindeutig bestimmte,
    kleinste Untermodul von $M$, der $S$ als Teilmenge enthält.
    $S$ heißt \begriff{Erzeugendensystem} von $U$.
    $M$ heißt \begriff{endlich erzeugt}, falls es eine endliche Menge
    $S \subseteq M$ gibt mit $\aufspann{S} = M$.
    Es gilt $\aufspann{S} =
    \left.\left\{\sum_{s \in S} a_s s \;\right|\; a_s \in A
    \text{ fast alle } 0_A\right\}$. \\
    \textbf{Summe von Untermoduln}:
    Sei $U_i \ur M$ für $i \in I$. \\
    Die \begriff{Summe} $U = \sum_{i \in I} U_i$ ist definiert
    als $\aufspann{\bigcup_{i \in I} U_i}$.
    Es gilt $U = \left.\left\{\sum_{i \in I} u_i \;\right|\;
    u_i \in U_i \text{ fast alle } 0_A\right\}$. \\
    \textbf{Faktormodul}:
    Sei $U \ur M$.
    Man definiert eine Äquivalenzrelation auf $M$ mit
    $x \equiv y \mod U \;\Leftrightarrow\; x - y \in U$ für $x, y \in M$.
    Die Äquivalenzklasse von $x \in M$ ist dann die \begriff{Nebenklasse} \\
    $x + U = \{x + u \;|\; u \in U\}$.
    Auf der Menge der Äquivalenzklassen $M/U = \{x + U \;|\; x \in M\}$
    wird eine Addition $(x + U) + (y + U) = (x + y) + U$ sowie eine
    $A$-Operation durch $a(x + U) = ax + U$ definiert.
    Diese sind wohldefiniert und machen $M/U$ zu einem $A$-Modul,
    dem \begriff{Faktormodul}.
    Die Abbildung $\pi: M \rightarrow M/U$, $\pi(m) = m + U$ ist ein
    Epimorphismus (\begriff{Projektion} von $M$ auf $M/U$). \\
    \textbf{Modulhomomorphismus}:
    Sei $f: M \rightarrow N$ ein $A$-Homomorphismus.
    Dann ist $\ker f \ur M$ und $\im f \ur N$. \\
    Sei $f: M \rightarrow N$ ein Isomorphismus.
    Dann ist $f^{-1}: N \rightarrow M$ ebenfalls einer.
    Die Relation "`isomorph sein"' ist Äquivalenzrelation auf der Klasse
    der $A$-Moduln. \\
    \textbf{1. Isomorphiesatz}:
    Sei $f: M \rightarrow N$ eine $A$-lineare Abbildung und $U \ur M$ mit
    $U \subseteq \ker f$.
    Dann gibt es genau eine $A$-lineare Abbildung
    $\widetilde{f}: M/U \rightarrow N$ mit $\widetilde{f} \circ \pi = f$.
    Es gilt $\im \widetilde{f} = \im f$ und
    $\ker \widetilde{f} = (\ker f)/U$.
    $\widetilde{f}$ ist gegeben durch $\widetilde{f}(m + U) = f(m)$.
    Ist insbesondere $\ker f = U$, so ist $\widetilde{f}$ ein
    $A$-Modulisomorphismus von $M/(\ker f)$ auf $\im f$ und
    $M/(\ker f) \cong \im f$. \\
    \textbf{2. Isomorphiesatz}:
    Seien $U, V \ur M$.
    Dann ist $(U + V)/V \cong U/(U \cap V)$. \\
    \textbf{3. Isomorphiesatz}:
    Seien $U \ur V \ur M$.
    Dann ist $V/U \ur M/U$ und $(M/U)/(V/U) \cong (M/V)$. \\
    \textbf{Modul über $K$-Algebra als Vektorraum}:
    Ist $A$ eine $K$-Algebra, so ist $M$ ein $K$-Vektorraum mit
    $\lambda m = (\lambda \cdot 1_A) m$ für $\lambda \in K$, $m \in M$.
    Dabei sind Untermoduln von $M$ auch $K$-Unterräume und $A$-lineare
    Abbildungen zwischen $A$-Moduln sind auch $K$-linear. \\
    \textbf{direkte Summe}:
    Seien $M_i \ur M$ für $i \in I$.
    Die Summe $U = \sum_{i \in I} M_i$ heißt
    \begriff{(interne) direkte Summe} der $M_i$, falls
    $M_i \cap \sum_{j \in I,\; j \not= i} M_j = (0)$ für alle $i \in I$ ist.
    Dies gilt genau dann, wenn jedes $u \in U$ eindeutig
    als $u = \sum_{i \in I} x_i$ mit $x_i \in M_i$ fast alle $0$ dargestellt
    werden kann. \\
    Sind $M_i$ für $i \in I$ eine Menge von $A$-Moduln,
    so ist die \begriff{(äußere) direkte Summe} \\
    $\bigoplus_{i \in I} M_i = \{(x_i)_{i \in I} \;|\; x_i \in M_i
    \text{ fast alle } 0\}$
    mit Addition und $A$-Operation definiert durch
    $(x_i)_{i \in I} + (y_i)_{i \in I} = (x_i + y_i)_{i \in I}$ und
    $a (x_i)_{i \in I} = (ax_i)_{i \in I}$.
    Damit ist $\bigoplus_{i \in I} M_i$ ein $A$-Modul.
\end{Bem}

\begin{Def}{freier Modul}
    Ein $A$-Modul $M$ heißt \begriff{frei},
    falls er isomorph zu einer direkten Summe von Kopien des
    regulären $A$-Moduls ${}_A A$ ist.
\end{Def}

\begin{Def}{Basis}
    Sei $M$ ein $A$-Modul.
    Dann heißt eine Teilmenge $S \subseteq N$ \begriff{linear unabhängig},
    falls es nur die triviale Darstellung der $0$ gibt, d.\,h.
    aus $\sum_{s \in S} a_s s = 0$, $a_s \in A$ fast alle $0$ folgt,
    dass $a_s = 0$ für alle $s \in S$. \\
    Eine linear unabhängiges Erzeugendensystem von $N$ heißt
    \begriff{Basis} von $N$. \\
    $S$ ist eine Basis von $N$ genau dann, wenn
    sich jedes Element von $N$ eindeutig als Linearkombination
    $\sum_{s \in S} a_s s$, $a_s \in A$ fast alle $0$ darstellen
    lässt. \\
    In diesem Fall gilt dann $N = \bigoplus_{s \in S} A s$
    mit $A s = \{as \;|\; a \in A\}$.
\end{Def}

\pagebreak

\begin{Satz}{Modul ist frei $\;\Leftrightarrow\;$ Modul hat eine Basis}
    Sei $M$ ein $A$-Modul. \\
    Dann ist $M$ frei genau dann, wenn
    $M$ eine $A$-Basis besitzt.
\end{Satz}

\enlargethispage{10mm}

\begin{Bem}
    Der Basissatz für Vektorräume sagt, dass alle $K$-Vektorräume
    für einen Körper $K$ frei sind.
    I.\,A. haben jedoch $A$-Moduln keine $A$-Basis.
    Ist $A$ eine $K$-Algebra, so ist ein $A$-Modul zugleich ein
    $K$-Vektorraum und muss daher eine $K$-Basis besitzen.
\end{Bem}

\begin{Def}{(kurze) exakte Folge}
    Seien $M_1, M_2, \dotsc, M_i, \dotsc$ $A$-Moduln und
    $\alpha_i: M_i \rightarrow M_{i+1}$ $A$-linear. \\
    $M_1 \xrightarrow{\alpha_1} M_2 \xrightarrow{\alpha_2} \dotsb
    \xrightarrow{\alpha_{i-1}} M_i \xrightarrow{\alpha_i} \dotsb$
    heißt \begriff{exakte Folge}, falls $\ker \alpha_{i+1} = \im \alpha_i$
    für alle $i \in \natural$ ist. \\
    Eine exakte Folge der Form $(0) \rightarrow N \xrightarrow{\alpha} M
    \xrightarrow{\beta} E \rightarrow (0)$
    heißt \begriff{kurze exakte Folge (keF)}.
\end{Def}

\begin{Bem}
    Es gibt genau einen $A$-Modulhomomorphismus $(0) \rightarrow N$
    und $E \rightarrow (0)$ (Nullabbildung). \\
    Die Folge $(0) \rightarrow N \xrightarrow{\alpha} M
    \xrightarrow{\beta} E \rightarrow (0)$
    ist exakt genau dann, wenn $\alpha$ injektiv, $\beta$ surjektiv
    sowie $\ker \beta = \im \alpha$ ist.
    In diesem Fall gilt nach dem 1. Isomorphiesatz $N/\im \alpha \cong E$. \\
    Ist $M$ ein $A$-Modul, $U \ur M$, so gibt es immer eine keF
    $(0) \rightarrow U \xrightarrow{\alpha} M
    \xrightarrow{\beta} M/U \rightarrow (0)$, wobei
    $\alpha$ die natürliche Einbettung von $U$ in $M$ und
    $\beta$ die natürliche Projektion von $M$ auf $M/U$ ist.
\end{Bem}

\begin{Satz}{Erzeugendensystem von epimorphen Bildern}
    Seien $M, N$ $A$-Moduln, $f: M \rightarrow N$ ein $A$-Epimorphismus und
    $S \subseteq M$ ein Erzeugendensystem für $M$.
    Dann wird $N$ von $f(S)$ erzeugt, d.\,h. insbesondere sind
    epimorphe Bilder von endlich erzeugten $A$-Moduln endlich erzeugt.
\end{Satz}

\begin{Satz}{$N, E$ endlich erzeugt $\Rightarrow M$ ebenfalls} \\
    Sei $(0) \rightarrow N \xrightarrow{\alpha} M
    \xrightarrow{\beta} E \rightarrow (0)$ keF von $A$-Moduln.
    Sind $N$ und $E$ endlich erzeugt, so auch $M$.
\end{Satz}

\begin{Satz}{$M$ als direkte Summe} \\
    Seien $(0) \rightarrow N \xrightarrow{\alpha} M
    \xrightarrow{\beta} E \rightarrow (0)$ keF von $A$-Moduln
    und $E$ freier $A$-Modul. \\
    Dann gibt es ein $U \ur M$ mit $U \cong E$ und $M = \im \alpha \oplus U$.
\end{Satz}

\begin{Satz}{Rang freier Moduln über noethersche Ringe ist wohldefiniert} \\
    Seien $R$ ein kommutativer, noetherscher Ring mit Eins und $M$ ein
    freier $R$-Modul. \\
    Sind $\{m_\alpha \;|\; \alpha \in \mathcal{A}\}$ und
    $\{v_\beta \;|\; \beta \in \mathcal{B}\}$ Basen von $M$ mit
    Indexmengen $\mathcal{A}$ und $\mathcal{B}$, so ist
    $|\mathcal{A}| = |\mathcal{B}|$.
\end{Satz}

\begin{Bem}
    Der Beweis des vorherigen Satzes funktioniert auch für Ringe $R$, die
    nicht kommutativ sind und kein Einselement haben, solange $R$ maximale
    Ideale besitzt. \\
    Hat $R$ ein Einselement, so kann man aus dem Zornschen Lemma die Existenz
    von maximalen Idealen folgern, d.\,h. auch hier ist der Rang eines freien
    $R$-Moduls wohldefiniert. \\
    Da Hauptidealringe noethersch sind, gilt der Satz insbesondere für HIRs
    (sogar ohne Zornsches Lemma).
\end{Bem}

\begin{Def}{Rang}
    Seien $R$ ein kommutativer noetherscher Ring mit Eins und $M$ ein
    freier $R$-Modul. \\
    Dann ist der \begriff{Rang} $\rg M$ definiert als Kardinalität einer
    Basis von $M$ (unabhängig von der Wahl der Basis).
\end{Def}

\begin{Lemma}{Annullator}
    Seien $A$ ein beliebiger Ring, $I \trianglelefteq A$ und $M$ ein
    $A$-Modul. \\
    Dann ist $I M$ ein $A$-Untermodul von $M$.
    Die Menge $\ann_A(M) = \{a \in A \;|\; \forall_{m \in M}\; am = 0\}$
    ist ein Ideal von $A$ und heißt \begriff{Annullator} von $M$ in $A$.
    Es gilt $I \subseteq \ann_A(M/IM)$.
    Ist $L \trianglelefteq A$ und $L \subseteq \ann_A(M)$, so ist
    $M$ ein $A/L$-Modul durch $(a + L)m = am$ für $a \in A$, $m \in M$. \\
    $M/IM$ ist $A/I$-Modul mit $A/I$-Operation
    $(a + I)(m + IM) = am + IM$.
\end{Lemma}

\begin{Satz}{freie Moduln über noethersche Ringe gleichen Rangs sind
             isomorph} \\
    Sei $R$ ein kommutativer, noetherscher Ring und seien $M$ und $N$
    freie $R$-Moduln mit $\rg M = \rg N$.
    Dann sind $M$ und $N$ isomorph.
    Für jede Kardinalität $\alpha$ gibt es daher einen bis auf Isomorphie
    eindeutigen freien $R$-Modul $\mathcal{F}_\alpha$ vom Rang $\alpha$,
    nämlich die direkte Summe von $\alpha$ vielen Kopien von ${}_R R$.
\end{Satz}

\pagebreak

\section{%
    \emph{Zusatz}: Projekt 12 (\texorpdfstring{$e$}{ℯ} hoch Matrix und
    lineare Dif"|ferentialgleichungen)%
}

\begin{Satz}{endlich-dimensionale normierte Vektorräume}
    Jeder endlich-dimensionale normierte Vektorraum ist vollständig.
    Zwei Normen auf einem endlich-dimensionalen Vektorraum sind äquivalent.
\end{Satz}

\begin{Def}{Algebranorm}
    Sei $\mathfrak{A}$ eine $K$-Algebra mit $K = \real$ oder $K = \complex$.
    Eine Vektorraum-Norm $\norm{\cdot}$ auf $\mathfrak{A}$ heißt Algebranorm,
    falls $\norm{AB} \le \norm{A} \cdot \norm{B}$ für alle
    $A, B \in \mathfrak{A}$ ist.
\end{Def}

\begin{Def}{$p$-Norm}
    Auf $M_n(K)$ ist mit $1 \le p \le \infty$ eine Norm definiert durch
    $\norm{A}_p = \left(\sum_{i,j=1}^n |\alpha_{ij}|^p\right)^{1/p}$
    für $A = (\alpha_{ij})_{ij} \in M_n(K)$.
    Für $1 \le p \le 2$ ist dies eine Algebranorm.
\end{Def}

\begin{Def}{$e$ hoch Matrix}
    Sei $S_k = \sum_{i=0}^k \frac{A^i}{i!}$ mit $A \in M_n(\complex)$.
    Dann existiert der Grenzwert der Folge $\{S_k\}_{k \in \natural}$ sowohl
    komponentenweise als auch bzgl. jeder Algebranorm auf $M_n(\complex)$. \\
    Der Grenzwert wird mit $e^A = \sum_{i=0}^\infty \frac{A^i}{i!}$
    bezeichnet.
\end{Def}

\begin{Satz}{Aussagen über $e$ hoch Matrix}
    Seien $A, B \in M_n(\complex)$ und $P \in \GL_n(\complex)$. \\
    Dann ist $P^{-1} e^A P = e^{P^{-1} A P}$, \qquad
    $e^A e^B = e^{A + B} = e^B e^A$ für $AB = BA$, \qquad
    $(e^A)^{-1} = e^{-A}$, \\
    $\det e^A = e^{\tr A}$ \quad und \quad
    $e^{\diag\{B_1, \dotsc, B_S\}} = \diag\{e^{B_1}, \dotsc, e^{B_s}\}$. \\
    Sind $\lambda_1, \dotsc, \lambda_n$ die Eigenwerte von $A$, so sind
    $e^{\lambda_1}, \dotsc, e^{\lambda_n}$ die Eigenwerte von $e^A$.
\end{Satz}

\begin{Prozedur}{Berechnung von $e^A$}
    \begin{enumerate}
        \item
        Man bringt $A$ auf Jordanform, d.\,h. man bestimmt eine Matrix
        $P \in \GL_n(\complex)$ mit \\
        $P^{-1} A P = \diag\{J_1, \dotsc, J_s\}$,
        wobei $J_i$ ein Jordanblock ist.

        \item
        Es gilt nun $e^A = e^{P \diag\{J_1, \dotsc, J_s\} P^{-1}} =
        P e^{\diag\{J_1, \dotsc, J_s\}} P^{-1}$.

        \item
        Es ist $e^{\diag\{J_1, \dotsc, J_s\}} =
        \diag\{e^{J_1}, \dotsc, e^{J_s}\}$.

        \item
        Um $e^{J_i}$ zu berechnen, sei $J_i = J_\lambda(k)$ ein Jordanblock
        sowie $N = J_0(k)$. \\
        Dann ist $J_\lambda(k) = \lambda E + N$ sowie
        $\lambda E \cdot N = N \cdot \lambda E$. \\
        Es ist $e^{J_i} = e^{\lambda E + N} = e^{\lambda E} e^N$,
        da $\lambda E$ und $N$ kommutieren. \\
        Es gilt $e^{\lambda E} e^N = e^\lambda e^N$ sowie
        $e^N =$ \matrixsize{$\begin{pmatrix}
        1 & \frac{1}{1!} & \frac{1}{2!} & \frac{1}{3!} & \cdots &
        \frac{1}{(k - 1)!} \\
        0 & 1 & \frac{1}{1!} & \frac{1}{2!} & \cdots & \frac{1}{(k - 2)!} \\
        \vdots & \ddots & \ddots & \ddots & \ddots & \vdots \\
        0 & \cdots & 0 & 1 & \frac{1}{1!} & \frac{1}{2!} \\
        0 & \cdots & 0 & 0 & 1 & \frac{1}{1!} \\
        0 & \cdots & 0 & 0 & 0 & 1
        \end{pmatrix}$}.

        \item
        Also ist $e^A = P \diag\{e^{\lambda_1} e^{N_1}, \dotsc,
        e^{\lambda_s} e^{N_s}\} P^{-1}$.
    \end{enumerate}
\end{Prozedur}

\section{%
    \emph{Zusatz}: Projekt 13 (Beispiele von Ringen)%
}

\begin{Lemma}{Lemma von \name{Gauß}}
    Sei $R$ ein faktorieller Ring und $Q$ der Quotientenkörper von $R$.
    Außerdem sei $p \in R[x]$ ein Polynom, sodass die Koef"|fizienten in $R$
    den größten gemeinsamen Teiler $1$ haben. \\
    Ist $p = g \cdot h$ mit $g, h \in Q[x]$, so gibt es $g', h' \in R[x]$ mit
    $p = g'h'$ und $g'$ bzw. $h'$ unterscheiden sich von $g$ bzw. $h$ nur um
    ein Element aus $Q$.
\end{Lemma}

\begin{Satz}{Satz von \name{Gauß}}
    Sei $R$ ein faktorieller Ring.
    Dann ist $R[x]$ auch ein faktorieller Ring.
\end{Satz}

\pagebreak
