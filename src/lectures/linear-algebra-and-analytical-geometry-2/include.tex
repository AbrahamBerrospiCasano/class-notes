% Aussagen
\newcommand*{\wahr}{\ensuremath{\text{w}}}
\newcommand*{\falsch}{\ensuremath{\text{f}}}

% Vektorräume
\newcommand*{\ur}{\ensuremath{\leqslant}}
\newcommand*{\aufspann}[1]{\ensuremath{\langle #1 \rangle}}
\newcommand*{\im}{\ensuremath{\operatorname{im}}}
\newcommand*{\basis}[1]{\ensuremath{\mathcal{#1}}}
\newcommand*{\stdbasis}[1]{\ensuremath{\mathcal{E}_{\mathnormal{#1}}}}
\newcommand*{\nk}[1]{\ensuremath{\overline{#1}}}

% Lineare Abbildungen und Matrizen
\newcommand*{\matrixm}{\mathcal{M}}
\newcommand*{\hommatrix}[3]{\ensuremath{\matrixm_{#1}(\basis{#2},\basis{#3})}}
\newcommand*{\enmatrix}[2]{\ensuremath{\matrixm_{#1}(\basis{#2})}}
\newcommand*{\id}{\ensuremath{\operatorname{id}}}
\newcommand*{\Hom}{\ensuremath{\operatorname{Hom}}}
\newcommand*{\End}{\ensuremath{\operatorname{End}}}
\newcommand*{\Aut}{\ensuremath{\operatorname{Aut}}}
\newcommand*{\GL}{\ensuremath{\operatorname{GL}}}
\newcommand*{\SL}{\ensuremath{\operatorname{SL}}}
\newcommand*{\rg}{\ensuremath{\operatorname{rg}}}
\newcommand*{\perm}{\ensuremath{\mathfrak{S}}}
\newcommand*{\sign}{\ensuremath{\operatorname{sign}}}
\newcommand*{\diag}{\ensuremath{\operatorname{diag}}}
\newcommand*{\tr}{\ensuremath{\operatorname{tr}}}
\newcommand*{\adj}{\ensuremath{\operatorname{adj}}}

% LGS
\newcommand*{\lgs}[1]{\ensuremath{\mathfrak{#1}}}
\newcommand*{\lgslsg}[1]{\ensuremath{\mathcal{L}_{\lgs{#1}}}}

% Affine Räume
\newcommand*{\affin}[1]{\ensuremath{\mathcal{#1}}}
\newcommand*{\TV}{\ensuremath{\operatorname{TV}}}

% Analytische Geometrie
\newcommand*{\kk}[1]{\ensuremath{\overline{#1}}}
\newcommand*{\eukl}{\ensuremath{\mathcal{E}}}
\newcommand*{\euklebene}{\ensuremath{\mathcal{E}_2}}
\newcommand*{\euklraum}{\ensuremath{\mathcal{E}_3}}
\newcommand*{\strecke}[1]{\ensuremath{\overline{#1}}}
\newcommand*{\vektor}[1]{\ensuremath{\overrightarrow{#1}}}

% Projektive Geometrie
\newcommand*{\pdim}{\ensuremath{\operatorname{p-dim}}}

% Euklidische und unitäre Vektorräume
\newcommand*{\orth}{\ensuremath{\;\bot\;}}

% Körper
\newcommand*{\characteristic}{\ensuremath{\operatorname{char}}}

% Multilineare Algebra
\newcommand*{\grammatrix}{\ensuremath{\mathcal{G}}}
\newcommand*{\rad}{\ensuremath{\operatorname{rad}}}
\newcommand*{\ggT}{\ensuremath{\operatorname{ggT}}}
\newcommand*{\kgV}{\ensuremath{\operatorname{kgV}}}
\newcommand*{\mi}[1]{\ensuremath{{\boldsymbol{\uline{#1}}}}}
\newcommand*{\alt}{\ensuremath{\mathcal{A}}}
\newcommand*{\ann}{\ensuremath{\operatorname{ann}}}

\newcommand*{\ordnung}{\ensuremath{\mathcal{O}}}

\newcommand*{\matrixsize}[1]{{\scriptsize #1}}
\newcommand*{\fracsize}[1]{{\large #1}}

\newenvironment*{Beobachtung}{\emph{Beobachtung}:}{}
\newenvironment*{Notation}{\emph{Notation}:}{}
\newenvironment*{Fakt}[1]{\emph{Fakt} (\textsl{#1}):}{}
\newenvironment*{Prozedur}[1]{\uline{\textbf{\texttt{Prozedur} (#1)}:}}{}

\chapter{%
    Mehr über Faktorräume und Körper%
}

\section{%
    Die Isomorphiesätze%
}

\begin{Bem}
    Im Folgenden sind $V, W, U$ (nicht notwendigerweise endlich-dimensionale)\\
    $K$-Vektorräume.
    Die Isomorphiesätze gelten mit kleinen Änderungen auch für
    Gruppen, Ringe, $K$-Algebren usw.
\end{Bem}

\begin{Def}{faktorisiert}
    Ein Homomorphismus $f: V \rightarrow W$ \begriff{faktorisiert} über $U$,
    falls es Homomorphismen $g: V \rightarrow U$ und $h: U \rightarrow W$
    gibt, sodass $f = h \circ g$ ist.
    Man sagt, dass das entsprechende Diagramm dann kommutiert.
\end{Def}

\begin{Satz}{1. Isomorphiesatz}
    Seien $f: V \rightarrow W$ ein Homomorphismus und $U \ur \ker f$. \\
    Dann faktorisiert $f$ eindeutig über $V/U$, genauer:
    Es gibt genau einen Homomorphismus \\
    $\widetilde{f}: V/U \rightarrow W$,
    sodass $\widetilde{f} \circ \pi = f$ ist, wobei $\pi: V \rightarrow V/U$
    die natürliche Projektion ist
    (es gilt $\widetilde{f}(v + U) = f(v)$).
    Darüber hinaus gilt $\im f = \im \widetilde{f}$
    sowie $\ker \widetilde{f} = (\ker f)/U \ur V/U$.
\end{Satz}

\begin{Kor}
    Sei $f: V \rightarrow W$ ein Homomorphismus. \\
    Dann induziert $f$ einen Monomorphismus
    $\widetilde{f}: V/\ker f \rightarrow W$.
    Insbesondere sind $V/\ker f$ und $\im f$ isomorph (der Isomorphismus
    ist gegeben durch $\widetilde{f}: V/\ker f \rightarrow \im f$).
\end{Kor}

\begin{Kor}
    Sei $f: V \rightarrow W$ ein Homomorphismus.
    Dann ist $\dim_K V = \dim_K \im f + \dim_K \ker f$
    (insbesondere ist $\dim_K V = \dim_K W + \dim_K \ker f$, falls $f$ ein
    Epimorphismus ist).
\end{Kor}

\begin{Satz}{Folgerung aus 1. Isomorphiesatz}
    Seien $f: V \rightarrow W$ ein Homomorphismus und $X \ur W$. \\
    Dann ist $f^{-1}(X) = \{v \in V \;|\; f(x) \in X\}$ ein Unterraum von
    $V$, der $\ker f$ enthält. \\
    Gilt sogar $X \ur \im f$, dann ist $f^{-1}(X) / \ker f \cong X$ und
    $X \mapsto f^{-1}(X)$ ist eine inklusionserhaltende Bijektion zwischen
    der Menge der Unterräume von $\im f$ und der Menge der Unterräume von
    $V$, die $\ker f$ enthalten.
    Diese Inklusion respektiert Summe und Durchschnitt von Unterräumen.
\end{Satz}

\begin{Satz}{2. Isomorphiesatz}
    Seien $U, W \ur V$, dann ist
    $(U + W)/U \cong W/(U \cap W)$.
\end{Satz}

\begin{Satz}{3. Isomorphiesatz}
    Sei $U \ur W \ur V$. \\
    Dann ist $W/U \ur V/U$ sowie
    $(V/U)\big/(W/U) \cong V/W$.
\end{Satz}

\begin{Satz}{Kor}
    Seien $f: V \rightarrow W$ ein Homomorphismus, $U = \ker f \ur V$
    und $U'$ ein Komplement von $U$ in $V$ (d.\,h. $U \oplus U' = V$). \\
    Dann ist $f$ auf $U'$ eingeschränkt ein Isomorphismus von $U'$ auf
    $\im f$. \\
    Ist insbesondere $\basis{A} = (v_1, \ldots, v_k, v_{k+1}, \ldots, v_n)$
    eine Basis von $V$, sodass $(v_1, \ldots, v_k)$ eine Basis von $U'$ und
    $(v_{k+1}, \ldots, v_n)$ eine Basis von $U$ ist, so ist
    $(f(v_1), \ldots, f(v_k))$ eine Basis von $\im f$.
\end{Satz}

\pagebreak

\section{%
    Mehr über Körper%
}

\begin{Lemma}{ggT}
    Seien $p, q \in \natural$ sowie $d \in \natural$ der ggT von $p$ und
    $q$. \\
    Dann gibt es $a, b \in \integer$, sodass $ap + bq = d$ ist.
\end{Lemma}

\begin{Satz}{Restklassenkörper}
    $\integer/(n)$ ist ein Körper genau dann, wenn $n$ eine Primzahl ist.
\end{Satz}

\begin{Def}{Unterkörper}
    Eine Teilmenge $F \subseteq K$ eines Körpers $K$ heißt
    \begriff{Unterkörper} von $K$,
    wenn $F$ mit der Addition und mit der Multiplikation von $K$ eingeschränkt
    auf $F$ wieder einen Körper bildet.
    Es gilt $1_F = 1_K$ sowie $0_F = 0_K$.
\end{Def}

\begin{Lemma}{kleinster Unterkörper}
    Jeder Körper $K$ besitzt einen kleinsten Unterkörper, d.\,h. einen
    Unterkörper, der in jedem Unterkörper enthalten ist
    (dieser kleinste Unterkörper ist der Durchschnitt aller Unterkörper).
\end{Lemma}

\begin{Def}{Primkörper}
    Den kleinsten Unterkörper eines Körpers $K$ nennt man \begriff{Primkörper}
    von $K$.
\end{Def}

\begin{Lemma}{$\rational$, $\integer/(n)$ haben keine echten Unterkörper}
    Die Körper $\rational$ und $\integer/(n)$ haben keine echten Unterkörper
    und sind daher ihre eigenen Primkörper.
\end{Lemma}

\begin{Def}{Charakteristik}
    Die \begriff{Charakteristik} $\char(K)$ eines Körpers $K$ ist definiert
    als \\
    $\char(K) = p \in \natural$, falls $p$ die kleinste natürliche Zahl
    ist mit $\overbrace{1_K + \ldots + 1_K}^{p-\text{mal}} = 0_K$ \\
    und $\char(K) = 0$, falls es keine solche Zahl gibt. \\
    Ist $\char(K) = p > 0$, so ist $p$ eine Primzahl.
\end{Def}

\begin{Satz}{$\rational$, $\integer/(n)$ als einzige Primkörper}
    Sei $K$ ein Körper. \\
    Ist $\char(K) = 0$, dann ist der Primkörper von $K$ isomorph zu
    $\rational$. \\
    Ist $\char(K) = p > 0$, dann ist der Primkörper von $K$ isomorph zu
    $\integer/(p)$.
\end{Satz}

\begin{Lemma}{$|K| = p^n$}
    Ist $K$ ein endlicher Körper, so ist $|K| = p^n$ für eine Primzahl
    $p$, $n \in \natural$.
\end{Lemma}

\pagebreak

\chapter{%
    Etwas multilineare Algebra%
}

\section{%
    Der Dualraum%
}

\begin{Bem}
    Im Folgenden seien $K$ ein Körper und $V, U$ usw. endliche $K$-Vektorräume.
\end{Bem}

\begin{Def}{Dualraum}
    Der $K$-Vektorraum $\Hom_K(V, K)$ wird mit $V^\ast$ bezeichnet. \\
    $V^\ast$ heißt der \begriff{zu $V$ duale Raum}.
    Die Elemente von $V^\ast$ heißen \begriff{Linearformen}.
\end{Def}

\begin{Bem}
    Bspw. sind die Abbildungen $\tr: M_{n \times n}(K) \rightarrow K$ und
    $I_a^b: V \rightarrow \real$,\\
    $I_a^b(f) = \int_a^b f(x)\dx$
    ($V = \{f: [a,b] \rightarrow \real \text{ stetig}\}$) Linearformen.
\end{Bem}

\begin{Def}{durch Basis definierte Linearformen}
    Sei $\basis{B} = \{v_i \;|\; i \in I\}$ eine (nicht notwendigerweise
    endliche) Basis von $V$.
    Dann ist die Linearform $v_i^\ast \in V^\ast$ eindeutig durch
    $v_i^\ast(v_j) = \delta_{ij}$ definiert. \\
    Ist $x \in V$ mit $x = \sum_{j \in I} \lambda_j v_j$ ($\lambda_j \in K$
    fast alle $0$), so ist $v_i^\ast(x) = \lambda_i$.
\end{Def}

\begin{Satz}{Basis von $V^\ast$}
    Sei $\basis{B} = (v_1, \ldots, v_n)$ eine Basis von $V$
    (endlich-dimensional!). \\
    Dann ist $\basis{B}^\ast = (v_1^\ast, \ldots, v_n^\ast)$ eine Basis von
    $V^\ast$ (die zu $\basis{B}$ \begriff{duale Basis}).
    Insbesondere sind $V$ und $V^\ast$ isomorph, ein Isomorphismus ist gegeben
    durch $v_i \mapsto v_i^\ast$ (linear ausgedehnt). \\
    Ist $f \in V^\ast$, so ist $f = \sum_{i=1}^n f(v_i) v_i^\ast$.
\end{Satz}

\begin{Bem}
    Für $\dim_K V = \infty$ ist $\sum f(v_i) v_i^\ast$ nicht definiert,
    dann ist $\dim_K V^\ast > \dim_K V$ und $\basis{B}^\ast$ ist keine
    Basis von $V^\ast$. \\
    Der Isomorphismus $\ast: V \rightarrow V^\ast$,
    $\sum_{i=1}^n \lambda_i v_i \mapsto \sum_{i=1}^n \lambda_i v_i^\ast$
    hängt wesentlich von der gewählten $\basis{B} = (v_1, \ldots, v_n)$ von
    $V$ ab.
    Die Bezeichnung $v^\ast = \sum_{i=1}^n \lambda_i v_i^\ast$ ist daher
    irreführend, wird aber doch behalten, wenn keine Missverständnisse zu
    befürchten sind.
\end{Bem}

\begin{Def}{duales Kompliment}
    Sei $U \ur V$.
    Dann ist $U^\bot = \{f \in V^\ast \;|\; f(U) = (0_K)\}$ ein
    Unterraum von $V^\ast$ und wird \begriff{duales Kompliment} von $U$
    in $V^\ast$ genannt.
    Ist $(v_1, \ldots, v_n)$ eine Basis von $V$, sodass $(v_1, \ldots, v_k)$
    eine Basis von $U$ ist, so ist $(v_{k+1}^\ast, \ldots, v_n^\ast)$ eine
    Basis von $U^\bot$. \\
    Insbesondere ist $\dim_K U^\bot = \dim_K V - \dim_K U$.
\end{Def}

\begin{Satz}{Doppeldualraum}
    Für $v \in V$ ist durch $f_v: V^\ast \rightarrow K$, $f_v(x) = x(v)$
    eine $K$-lineare Abbildung definiert, d.\,h. $f_v$ ist eine Linearform
    auf $V^\ast$ und daher Element des Dualraums $V^{\ast\ast} = (V^\ast)^\ast$
    von $V^\ast$.
    Die Abbildung $\mathcal{E}: V \rightarrow V^{\ast\ast}$, $v \mapsto f_v$
    ist ein Isomorphismus.
\end{Satz}

\begin{Bem}
    Der Isomorphismus $\mathcal{E}: V \rightarrow V^{\ast\ast}$ hängt nicht
    von einer gewählten Basis ab.
    Man spricht von einem \begriff{kanonischen/natürlichen Isomorphismus}.
\end{Bem}

\begin{Satz}{$\mathcal{E}$ unabhängig von Basis}
    Sei $V$ ein $K$-Vektorraum.
    Dann wird durch $\mathcal{E}: V \rightarrow V^{\ast\ast}$, $v \mapsto f_v$
    ein Monomorphismus definiert.
    Ist zusätzlich $V$ endlich dimensional,
    $\basis{B}$ eine Basis von $V$,
    $\basis{B}^\ast$ die zugehörige duale Basis von $V$,
    $\basis{B}^{\ast\ast}$ die zugehörige doppelduale Basis von $V$
    und $b \in \basis{B}$, so ist $b^{\ast\ast} = f_b$.
    Man bezeichnet daher $\mathcal{E}$ auch mit $\ast\ast$.
    $\ast\ast: V \rightarrow V^{\ast\ast}$ ist dann ein Isomorphismus.
\end{Satz}

\begin{Satz}{$\ast$ bei linearen Abbildungen}
    Sei $f: V \rightarrow U$ ein Homomorphismus.
    Dann wird durch $f^\ast: U^\ast \rightarrow V^\ast$,
    $f^\ast(h) = h \circ f \in V^\ast$ für $h \in U^\ast$ eine
    $K$-lineare Abbildung $f^\ast$ definiert. \\
    Sind $V$ und $U$ endlich-dimensional, so gilt \qquad\;\;\,
    1. $\ker f^\ast = (\im f)^\bot$ \\
    2. $\dim_K(\im f) = \dim_K(\im f^\ast)$ \qquad\qquad\qquad\qquad\,
    3. $f^\ast$ ist surjektiv $\;\Leftrightarrow\; f$ ist injektiv \\
    4. $f^\ast$ ist injektiv $\;\Leftrightarrow\; f$ ist surjektiv
    \qquad\qquad\qquad
    5. $f^{\ast\ast}(v^{\ast\ast}) = (f(v))^{\ast\ast}$ \\
    6. Ist $g: U \rightarrow W$ Homomorphismus, so gilt
    $(g \circ f)^\ast = f^\ast \circ g^\ast$ \\
    ($\ast: \Hom_K(V, U) \rightarrow \Hom_K(U^\ast, V^\ast)$ ist
    \begriff{kontravariant})
\end{Satz}

\begin{Satz}{Matrix von $f^\ast$}
    Seien $f: V \rightarrow U$ Homomorphismus,
    $\basis{B} = (v_1, \ldots, v_n)$ bzw.\\
    $\basis{C} = (u_1, \ldots, u_m)$
    Basen von $V$ bzw. $U$ sowie $A = \hommatrix{f}{C}{B}$.
    Dann ist $\hommatrix{f^\ast}{B^\ast}{C^\ast} = A^t$.
\end{Satz}

\begin{Kor}
    Für $A \in M_{m \times n}(K)$ sind Spalten- und Zeilenrang gleich.
\end{Kor}

\section{%
    Bilinearformen%
}

\begin{Def}{Bilinearform}
    Seien $V$, $U$ und $W$ $K$-Vektorräume.
    Eine Abbildung $f: V \times U \rightarrow W$ heißt \begriff{bilinear},
    falls $f(v_1 + v_2, u) = f(v_1, u) + f(v_2, u)$,
    $f(v, u_1 + u_2) = f(v, u_1) + f(v, u_2)$ und \\
    $f(\lambda v, u) = f(v, \lambda u) = \lambda f(v, u)$ gilt
    für alle $v, v_1, v_2 \in V$, $u, u_1, u_2 \in U$ und $\lambda \in K$. \\
    Eine bilineare Abbildung $f: V \times V \rightarrow K$ heißt
    \begriff{Bilinearform} auf $K$. \\
    Ersetzt man die dritte Bedingung durch
    $f(\lambda v, u) = f(v, \kk{\lambda} u) = \lambda f(v, u)$, wobei
    $\kk{\;\;}: K \rightarrow K$ ein Automorphismus von $K$ mit
    $\kk{\kk{\lambda}} = \lambda$ für alle $\lambda \in K$ ist,
    so heißt die Abbildung \begriff{semilinear}.
\end{Def}

\begin{Satz}{Festlegung einer Bilinearform}
    Seien $\sp{\cdot, \cdot}: V \times V \rightarrow K$ eine Bilinearform \\
    und $\basis{B} = \{v_i \;|\; i \in I\}$ eine Basis von $V$. \\
    Dann ist $\sp{\cdot, \cdot}$ durch die Angabe der Skalare
    $\lambda_{ij} = \sp{v_i, v_j} \in K$ vollständig bestimmt. \\
    Gibt man umgekehrt Skalare $\lambda_{ij} \in K$ vor und definiert
    $\sp{v, w} = \sp{\sum_{i \in I} \alpha_i v_i, \sum_{j \in I} \beta_j v_j}
    = \sum_{i, j \in I} \alpha_i \lambda_{ij} \beta_j \in K$
    für $v = \sum_{i \in I} \alpha_i v_i$ und
    $w = \sum_{j \in I} \beta_j v_j$ ($\alpha_i, \beta_j \in K$ fast alle $0$),
    dann ist $\sp{\cdot, \cdot}: V \times V \rightarrow K$ eine Bilinearform
    auf $V$.
\end{Satz}

\begin{Def}{Grammatrix}
    Die Matrix $\grammatrix = \grammatrix(\basis{B}) = (\lambda_{ij})_{ij}$
    (mit $i, j \in I$) heißt \begriff{Grammatrix} der Bilinearform
    $\sp{\cdot, \cdot}$ bzgl. der Basis $\basis{B}$.
    Ist $V$ endlich-dimensional und $\basis{B} = (v_1, \ldots, v_n)$,
    so ist $\grammatrix(\basis{B})$ eine $n \times n$-Matrix.
\end{Def}

\begin{Bem}
    Ist $\grammatrix(\basis{B}) = (\lambda_{ij})_{ij}$ die Grammatrix
    von $\sp{\cdot, \cdot}$ bzgl. $\basis{B}$ und
    $v = \sum_{i \in I} \alpha_i v_i$ und $w = \sum_{j \in I} \beta_j v_j$
    ($\alpha_i, \beta_j \in K$ fast alle $0$), so ist
    $\sp{v, w} = (\alpha_i)_i^t \cdot (\lambda_{ij})_{ij} \cdot (\beta_j)_j$
    als Matrizenprodukt, wobei $(\alpha_i)_i, (\beta_j)_j$ Spaltenvektoren
    sind.
\end{Bem}

\begin{Bem}
    Die Menge der Bilinearformen auf $V$ wird ein Vektorraum, wenn
    man $f+g: V \times V \rightarrow K$, $(f+g)(v, w) = f(v, w) + g(v, w)$
    und $\lambda f: V \times V \rightarrow K$,
    $(\lambda f)(v, w) = \lambda f(v, w)$ für \\
    $f, g: V \times V \rightarrow K$
    Bilinearformen und $\lambda \in K$ definiert. \\
    Dann wird $\grammatrix_f(\basis{B})$ (die Abbildung, die jeder Bilinearform
    auf $V$ die Grammatrix bzgl. einer festen Basis $\basis{B}$ zuordnet)
    zum Vektorraum-Isomorphismus zwischen der Menge der Bilinearformen
    auf $V$ und $M_{n \times n}(K)$.
    Es gilt $\grammatrix_f(\basis{C}) = (\hommatrix{\id}{B}{C})^t
    \grammatrix_f(\basis{B}) \hommatrix{\id}{B}{C}$.
\end{Bem}

\begin{Def}{links-/rechtsorthogonal}
    Seien $V$ ein $K$-Vektorraum, $\sp{\cdot, \cdot}: V \times V \rightarrow K$
    eine Bilinearform auf $V$ und $x, y \in V$.
    Dann heißt $x$ \begriff{linksorthogonal} zu $y$ und $y$
    \begriff{rechtsorthogonal} zu $x$, falls $\sp{x, y} = 0$.
    Man schreibt auch $x \orth y$.
\end{Def}

\begin{Def}{Links-/Rechtsradikal}
    $\rad_l(\sp{\cdot, \cdot}) = \{x \in V \;|\;
    \forall_{y \in V}\; \sp{x, y} = 0\}$
    heißt \begriff{Linksradikal} und
    $\rad_r(\sp{\cdot, \cdot}) = \{x \in V \;|\;
    \forall_{y \in V}\; \sp{y, x} = 0\}$
    heißt \begriff{Rechtsradikal} der Bilinearform $\sp{\cdot, \cdot}$.
\end{Def}

\begin{Satz}{Links-/Rechtsradikal als Unterraum} \\
    Links- und Rechtsradikal einer Bilinearform auf $V$ sind Unterräume von
    $V$.
\end{Satz}

\begin{Satz}{assoziierter Links-/Rechtshomomorphismus}
    Sei $f = \sp{\cdot, \cdot}: V \times V \rightarrow K$ bilinear.
    Dann wird durch
    $E_l: V \rightarrow V^\ast$, $E_l(v) = \lambda_v$ mit
    $\lambda_v: V \rightarrow K$, $\lambda_v(x) = \sp{v, x}$
    ein Homomorphismus definiert, dieser heißt der zu $f$ assoziierte
    (kanonische) \begriff{Linkshomomorphismus} von $V$ nach $V^\ast$.
    Zur Verdeutlichung, dass $E_l$ bzgl. $f$ gebildet wurde, schreibt man
    auch $E_l^f$.
    Analog wird $E_r: V \rightarrow V^\ast$, $E_r(v) = \rho_v$ mit
    $\rho_v: V \rightarrow K$, $\rho_v(x) = \sp{x, v}$
    der \begriff{Rechtshomomorphismus} definiert. \\
    Es gilt $\rad_l(\sp{\cdot, \cdot}) = \ker E_l$ sowie
    $\rad_r(\sp{\cdot, \cdot}) = \ker E_r$. \\
    Ist $V$ endlich-dimensional und $\basis{B}$ Basis von $V$, so gilt
    $\hommatrix{E_r}{B^\ast}{B} = \grammatrix_f(\basis{B}) =
    (\hommatrix{E_l}{B^\ast}{B})^t$.
\end{Satz}

\begin{Kor}
    Sei $V$ endlich-dimensional.
    Dann ist $\dim_K \rad_l(f) = \dim_K \rad_r(f) =
    n - \rg(\grammatrix_f(\basis{B}))$.
    Außerdem ist $\rad_l(f) = (0) \;\Leftrightarrow\; \rad_r(f) = (0)$.
    In diesem Fall heißt $f$ \begriff{nicht ausgeartet}, sonst
    \begriff{ausgeartet}.
    Für $f$ nicht ausgeartet definieren $E_l$, $E_r$ kanonische Isomorphismen
    von $V$ auf $V^\ast$.
    (Im Falle von $V$ unendlich-dimensional sind $E_l$, $E_r$ injektiv.)
\end{Kor}

\begin{Satz}{Bijektion}
    $f \mapsto E_l^f$ und $f \mapsto E_r^f$ definieren Bijektionen zwischen
    der Menge der Bilinearformen $f$ auf $V$ und $\Hom_K(V, V^\ast)$.
    Für $\dim_K V < \infty$ ist dies ein Isomorphismus.
\end{Satz}

\begin{Def}{spezielle Bilinearformen}
    Sei $\sp{\cdot, \cdot}$ eine Bilinearform auf $V$. \\
    $\sp{\cdot, \cdot}$ heißt \begriff{symmetrisch}, falls
    $\sp{v_1, v_2} = \sp{v_2, v_1}$ für alle $v_1, v_2 \in V$. \\
    $\sp{\cdot, \cdot}$ heißt \begriff{alternierend}, falls
    $\sp{v_1, v_2} = -\sp{v_2, v_1}$ für alle $v_1, v_2 \in V$.
\end{Def}

\begin{Lemma}{Eigenschaften spezieller Bilinearformen}
    Ist $\sp{\cdot, \cdot}$ symmetrisch oder alternierend, so ist
    $x \orth y \;\Leftrightarrow\; y \orth x$ und die Relation $\bot$ ist
    symmetrisch. \\
    Ist $\sp{\cdot, \cdot}$ symmetrisch oder alternierend, so braucht man
    daher nicht mehr zwischen Links- und Rechtsradikal zu unterscheiden. \\
    Für $\sp{\cdot, \cdot}$ symmetrisch ist $E_l = E_r$,
    für $\sp{\cdot, \cdot}$ alternierend ist $E_l = -E_r$. \\
    Ist $\char K = 2$ (also $1 = -1$ in $K$), so ist alternierend und
    symmetrisch dasselbe. \\
    $\sp{\cdot, \cdot}$ ist symmetrisch genau dann, wenn
    $\grammatrix_{\sp{\cdot, \cdot}}(\basis{B})$ bzgl. einer Basis $\basis{B}$
    symmetrisch ist. \\
    $\sp{\cdot, \cdot}$ ist alternierend genau dann, wenn
    $\grammatrix_{\sp{\cdot, \cdot}}(\basis{B})$ bzgl. einer Basis $\basis{B}$
    schiefsymmetrisch ist (d.\,h. $A^t = -A$).
\end{Lemma}

\section{%
    Symmetrische Gruppen%
}

\begin{Satz}{Existenz der Ordnung einer endlichen Gruppe}
    Seien $G$ eine endliche Gruppe und $g \in G$. \\
    Dann gibt es ein $k \in \natural$, sodass $g^k = g \dotsm g = 1_G$ ist.
\end{Satz}

\begin{Def}{Ordnung}
    Die kleinste Zahl $k \in \natural$, für die $g^k = 1_G$ gilt,
    heißt \begriff{Ordnung} $|g|$ von $g \in G$.
\end{Def}

\begin{Def}{Bahn, Zykel}
    Sei $\pi \in \perm_n$ und $i \in \{1, \dotsc, n\}$.
    Wegen $\pi^{|\pi|}(i) = \id(i) = i$ gibt es eine kleinste Zahl
    $k \in \natural$, sodass $\pi^k(i) = i$ ist.
    Dann sind $i, \pi(i), \pi^2(i), \dotsc, \pi^{k-1}(i)$ paarweise
    verschieden.
    Die Menge $\{i, \pi(i), \pi^2(i), \dotsc, \pi^{k-1}(i)\}$ heißt
    \begriff{Bahn von $i$ unter $\pi$} oder \begriff{Zykel} und wird mit
    $i^{[\pi]}$ bezeichnet.
    Dabei ist $k$ die \begriff{Länge der Bahn}.
\end{Def}

\begin{Lemma}{Äquivalenzrelation auf $\matrixm$}
    Sei $\pi \in \perm_n$ und $\matrixm = \{1, \dotsc, n\}$.
    Sei die Relation $\sim_\pi$ auf $\matrixm$ definiert durch
    $s \sim_\pi t \;\Leftrightarrow\;
    \exists_{k \in \natural_0}\; \pi^k(s) = t$.
    Dann ist $\sim_\pi$ eine Äquivalenzrelation auf $\matrixm$ und die
    Äquivalenzklassen $[s]$ sind genau die Bahnen $s^{[\pi]}$ unter $\pi$.
\end{Lemma}

\begin{Kor}
    Sei $\pi \in \perm_n$.
    Dann zerlegen die Bahnen bzgl. $\pi$ die Menge $\matrixm$ disjunkt.
    Also existieren Elemente $x_i \in \matrixm$
    und $k_1, \dotsc, k_t \in \natural$ für $i = 1, \dotsc, t$, sodass
    $\matrixm$ disjunkte Vereinigung von den Bahnen
    $\{x_i, \pi(x_i), \dotsc, \pi^{k_i-1}(x_i)\}$ ist.
\end{Kor}

\begin{Notation}
    Für $\pi \in \perm_n$ schreibt man
    $\pi = (x_1, \pi(x_1), \dotsc, \pi^{k_1-1}(x_1)) \dotsm
    (x_t, \pi(x_t), \dotsc, \pi^{k_t-1}(x_t))$.
    Diese Schreibweise heißt \begriff{Zykelschreibweise}.
    Die Teile mit $k_i = 1$ kann man auch weglassen.
\end{Notation}

\begin{Bem}
    \matrixsize{$\pi = \begin{pmatrix} 1 & 2 & 3 & 4 & 5 & 6 & 7 \\
    2 & 4 & 7 & 1 & 5 & 6 & 3 \end{pmatrix}$} lautet in Zykelschreibweise\\
    $\pi = (124)(37)(5)(6) = (124)(37)$.
\end{Bem}

\begin{Def}{Zykel}
    Ein \begriff{Zykel} ist eine Permutation $\pi \in \perm_n$,
    die höchstens eine Bahn hat, die länger als $1$ ist, d.\,h.
    $\pi = (a_1, \dotsc, a_k)$.
    Es gilt $\pi(a_i) = a_{i+1}$ für $i = 1, \dotsc, k - 1$,
    $\pi(a_k) = a_1$ und $\pi(b) = b$ für
    $b \in \matrixm \setminus \{a_1, \dotsc, a_k\}$.
    Die Ordnung von $\pi$ ist $|\pi| = k$.
\end{Def}

\begin{Lemma}{disjunkte Zyklen kommutieren}
    Disjunkte Zyklen kommutieren, d.\,h. es ist z.\,B. \\
    $(124)(356) = (356)(124)$, aber
    $(123)(245) \not= (245)(123)$.
\end{Lemma}

\begin{Kor}
    Jede Permutation $\pi \in \perm_n$ kann bis auf die Reihenfolge eindeutig
    als Produkt von disjunkten Zyklen beschrieben werden.
    Die Zyklen entsprechen dabei den Bahnen der Länge größer $1$.
\end{Kor}

\begin{Satz}{$|\pi| = \kgV$}
    Sei $\pi \in \perm_n$. \\
    Dann ist $|\pi|$ das kleinste gemeinsame Vielfache der Längen aller
    Bahnen von $\pi$.
\end{Satz}

\begin{Def}{Transposition}
    Ein Zykel der Länge $2$ heißt \begriff{Transposition}. \\
    Eine Transposition der Form $(i, i + 1)$ heißt
    \begriff{Fundamentaltransposition}.
\end{Def}

\begin{Satz}{Permutation als Produkt von Transpositionen}
    Jede Permutation $\pi \in \perm_n$ kann als Produkt von Transpositionen
    geschrieben werden.
    Jede Transposition (und daher auch jede Permutation)
    kann als Produkt von Fundamentaltranspositionen geschrieben werden.
\end{Satz}

\begin{Def}{reduzierter Ausdruck}
    Sei $\pi \in \perm_n$.
    Ein \begriff{reduzierter Ausdruck} von $\pi$ ist ein Produkt von
    Fundamentaltranspositionen
    $\pi = (i_1, i_1 + 1)(i_2, i_2 + 1) \dotsm (i_l, i_l + 1)$, sodass
    $l$ minimal ist
    (d.\,h. $\pi$ lässt sich nicht als Produkt von weniger als $l$
    Fundamentaltranspositionen schreiben). \\
    Der reduzierte Ausdruck für $\id$ sei dabei der leere Ausdruck mit
    $l = 0$ Faktoren. \\
    $l(\pi) = l$ heißt die \begriff{Länge der Permutation} $\pi$.
\end{Def}

\begin{Def}{Fehlstände}
    Sei $\pi \in \perm_n$.
    Die Menge der Fehlstände von $\pi$ ist definiert als \\
    $\{[i,j] \;|\; 1 \le i < j \le n \text{ mit } \pi(i) > \pi(j)\}$.
\end{Def}

\begin{Lemma}{Fehlstände und Fundamentaltransposition}
    Seien $n(\pi)$ die Anzahl der Fehlstände von $\pi \in \perm_n$
    und $(k, k + 1)$ eine Fundamentaltransposition. \\
    Dann gilt $n(\pi \; (k, k + 1)) =$
    \matrixsize{$\begin{cases} n(\pi) + 1 & \pi(k) < \pi(k + 1) \\
    n(\pi) - 1 & \pi(k) > \pi(k + 1) \end{cases}$}.
\end{Lemma}

\begin{Satz}{Länge der Permutation gleich Anzahl Fehlstände} \\
    Sei $\pi \in \perm_n$.
    Dann ist $l(\pi)$ gleich der Anzahl der Fehlstände von $\pi$.
\end{Satz}

\begin{Kor}
    Kein Produkt einer geraden Anzahl von (Fundamental-)Transpositionen
    ist gleich einem Produkt einer ungeraden Anzahl von
    (Fundamental-)Transpositionen.
\end{Kor}

\begin{Def}{gerade/ungerade, Signum}
    Eine Permutation $\pi$ heißt \begriff{gerade/ungerade}, wenn
    $l(\pi)$ gerade/ungerade ist.
    $\sign(\pi) = (-1)^{l(\pi)}$ heißt \begriff{Signum} von $\pi$.
\end{Def}

\begin{Lemma}{$\sign$ als Gruppenhomomorphismus}
    Die Abbildung $\sign: \perm_n \rightarrow \{1, -1\}$ ist ein
    Gruppenhomomorphismus in die multiplikative Gruppe $\{1, -1\}$,
    d.\,h. $\sign(\sigma \pi) = \sign(\sigma) \sign(\pi)$.
\end{Lemma}

\begin{Kor}
    Ein Produkt von einer geraden Anzahl von Transpositionen multipliziert mit
    einer ebensolchen ist wieder ein
    Produkt einer geraden Anzahl von Transpositionen.
\end{Kor}

\begin{Def}{Konjugationsklasse}
    Zwei Elemente $x, y \in G$ einer Gruppe $G$ heißen \begriff{konjugiert},
    falls es ein $g \in G$ gibt, sodass $x = g y g^{-1}$. \\
    Die Äquivalenzklasse $x^G = \{g x g^{-1} \;|\; g \in G\}$ heißt
    \begriff{Konjugationsklasse} von $x \in G$.
\end{Def}

\begin{Lemma}{"`konjugiert"' als Äquivalenzrelation}
    Die Relation $\sim$ auf $G$ definiert durch \\
    $x \sim y \;\Leftrightarrow\; \exists_{g \in G}\; x = g y g^{-1}$
    ist eine Äquivalenzrelation.
    Die Äquivalenzklassen sind genau die Konjugationsklassen,
    also ist $G$ disjunkte Vereinigung seiner Konjugationsklassen.
\end{Lemma}

\begin{Lemma}{Zykel konjugieren}
    Seien $\pi, \sigma \in \perm_n$ und $\sigma = (a_1, \dotsc, a_k)$
    ein Zykel. \\
    Dann ist $\pi \sigma \pi^{-1} = (\pi(a_1), \dotsc, \pi(a_k))$.
\end{Lemma}

\begin{Def}{Partition}
    Sei $n \in \natural$.
    Eine \begriff{Partition} von $n$ ist eine Folge
    $\lambda = (\lambda_1, \dotsc, \lambda_k)$ von
    Zahlen $\lambda_i \in \natural$, sodass
    $\lambda_1 \ge \dotsb \ge \lambda_k$ und $\sum_{i=1}^k \lambda_i = n$.
\end{Def}

\begin{Def}{Zykeltyp}
    Sei $\pi \in \perm_n$.
    Der \begriff{Zykeltyp} von $\pi$ ist die Partition von $n$, die entsteht,
    wenn man $\pi$ als Produkt von disjunkten Zykeln schreibt und die
    Längen der Zykel (einschließlich der Zykel der Länge $1$)
    absteigend ordnet.
\end{Def}

\begin{Lemma}{Zykeltyp und konjugiert}
    Zwei Permutationen aus $\perm_n$ sind konjugiert genau dann, wenn
    sie vom selben Zykeltyp sind.
\end{Lemma}

\begin{Satz}{Bijektion}
    Es gibt eine Bijektion zwischen den Konjugationsklassen der $\perm_n$
    und den Partitionen von $n$,
    diese bildet eine Konjugationsklasse $\pi^{\sigma_n}$ auf
    den Zykeltyp von $\pi$ ab.
\end{Satz}

\section{%
    Multilinearformen%
}

\begin{Def}{Multilinearform}
    Seien $K$ ein Körper, $V_1, \dotsc, V_k, W$ $K$-Vektorräume und \\
    $f: V_1 \times \dotsb \times V_k \rightarrow W$ eine Abbildung.
    Dann heißt $f$ \begriff{multilinear} (oder \begriff{$k$-fach linear}),
    falls für alle $i = 1, \dotsc, k$ gilt, dass
    $f(v_1, \dotsc, v_i' + v_i'', \dotsc, v_k) =
    f(v_1, \dotsc, v_i', \dotsc, v_k) + f(v_1, \dotsc, v_i'', \dotsc, v_k)$
    und $f(v_1, \dotsc, \lambda v_i, \dotsc, v_k) =
    \lambda f(v_1, \dotsc, v_i, \dotsc, v_k)$ für
    $v_1 \in V_1$, \dots, $v_k \in V_k$, $v_i', v_i'' \in V_i$ und
    $\lambda \in K$. \\
    Eine multilineare Abbildung
    $f: V \overset{k\text{-fach}}{\times \dotsb \times} V \rightarrow K$ heißt
    \begriff{$k$-fache Multilinearform auf $V$}.
\end{Def}

\begin{Satz}{Menge der multilinearen Abbildungen als $K$-Vektorraum} \\
    Sei $M = \{f: V_1 \times \dotsb \times V_k \rightarrow W \;|\;
    f \text{ multilinear}\}$.
    Definiere auf $M$ eine Addition \\
    $f + g: V_1 \times \dotsb \times V_k \rightarrow W$,
    $(f + g)(v_1, \dotsc, v_k) = f(v_1, \dotsc, v_k) + g(v_1, \dotsc, v_k)$
    sowie eine skalare Multiplikation
    $\lambda f: V_1 \times \dotsb \times V_k \rightarrow W$,
    $(\lambda f)(v_1, \dotsc, v_k) = \lambda f(v_1, \dotsc, v_k)$
    mit $v_i \in V_i$ ($i = 1, \dotsc, k$), $f, g \in M$ und $\lambda \in K$.
    Dann wird $M$ mit diesen Operationen zum $K$-Vektorraum.
\end{Satz}

\begin{Bem}
    Das Nullelement von $M$ ist die Nullabbildung
    $0: V_1 \times \dotsb \times V_k \rightarrow W$,
    $0(v_1, \dotsc, v_k) = 0_W$.
\end{Bem}

\begin{Def}{Multiindex}
    Seien $I_1 = \{1, \dotsc, n_1\}$, \dots, $I_k = \{1, \dotsc, n_k\}$
    endliche Indexmengen.
    Ein Element $\mi{i} \in I_1 \times \dotsb \times I_k$ heißt
    \begriff{Multiindex} und $\mi{I} = I_1 \times \dotsb \times I_k$
    heißt \begriff{Menge der Multiindizes}. \\
    Sind $V_1, \dotsc, V_k$ Vektorräume und $\mi{i} = (i_1, \dotsc, i_k)$,
    dann sei $v_\mi{i} \in V_1 \times \dotsb \times V_k$ definiert durch
    $v_\mi{i} = (v_{i_1}^{(1)}, \dotsc, v_{i_k}^{(k)})$, wobei
    $v_1^{(\nu)}, \dotsc, v_{n_\nu}^{(\nu)} \in V_\nu$ für
    $\nu = 1, \dotsc, k$. \\
    Damit ist auch das Kronecker-Delta für Multiindizes definiert
    durch \matrixsize{$\delta_{\mi{i} \mi{j}} =
    \begin{cases} 1 & \mi{i} = \mi{j} \\ 0 & \mi{i} \not= \mi{j} \end{cases}$},
    da für $\mi{i}, \mi{j} \in \mi{I}$ mit $\mi{i} = (i_1, \dotsc, i_k)$ und
    $\mi{j} = (j_1, \dotsc, j_k)$ gilt, dass
    $\mi{i} = \mi{j} \;\Leftrightarrow\; (i_1 = j_1) \land \dotsb \land
    (i_k = j_k)$.
\end{Def}

\begin{Satz}{Dimension von $M$, Basis}
    Seien $V_1, \dotsc, V_k, W$ endlich-dimensionale Vektorräume. \\
    Dann ist $M$ ebenfalls endlich-dimensional und
    $\dim_K M = \dim_K V_1 \dotsm \dim_K V_k \dim_K W$. \\
    Seien $n_\nu = \dim_K V_\nu$,
    $\basis{B}_\nu = (v_1^{(\nu)}, \dotsc, v_{n_\nu}^{(\nu)})$ eine Basis
    von $V_\nu$ für $\nu = 1, \dotsc, k$ sowie $(w_1, \dotsc, w_m)$ eine
    Basis von $W$, dann ist
    $\basis{B} = \{f_{\mi{i},j} \;|\; \mi{i} \in \mi{I},\; 1 \le j \le m\}$
    eine Basis von $M$, wobei \\
    $f_{\mi{i},j}: V_1 \times \dotsb \times V_k \rightarrow W$,
    \matrixsize{$f_{\mi{i},j}(v_\mi{k}) = \begin{cases}
    w_j & \mi{i} = \mi{k} \\ 0 & \text{sonst} \end{cases}$} multilinear
    für $\mi{i}, \mi{k} \in \mi{I}$ und $j = 1, \dotsc, m$.
\end{Satz}

\begin{Def}{symmetrische Multilinearform} \\
    Sei $f: V^{\times k} \rightarrow K$ eine $k$-fache Multilinearform auf $V$
    (dabei ist $V^{\times k} =
    V \overset{k \text{ Faktoren}}{\times \dotsb \times} V$). \\
    $f$ heißt \begriff{symmetrisch}, falls
    $f(v_1, \dotsc, v_k) = f(v_{\pi(1)}, \dotsc, v_{\pi(k)})$ für
    alle $\pi \in \perm_k$ ist.
\end{Def}

\begin{Def}{alternierende Multilinearform (1. Versuch)} \\
    Sei $f: V^{\times k} \rightarrow K$ eine $k$-fache Multilinearform auf
    $V$. \\
    $f$ heißt \begriff{alternierend}, falls
    $f(v_1, \dotsc, v_k) = \sign(\pi) \cdot f(v_{\pi(1)}, \dotsc, v_{\pi(k)})$
    für alle $\pi \in \perm_k$ ist.
\end{Def}

\begin{Lemma}{alternierende Multilinearform ist $0$ bei gleichen Vektoren}
    Seien $\char K \not= 2$ (d.\,h. es ist $-1_K \not= 1_K$) und
    $f: V^{\times k} \rightarrow K$ eine $k$-fache alternierende
    Multilinearform auf $V$. \\
    Dann gilt $f(v_1, \dotsc, v_k) = 0$, falls
    $v_1, \dotsc, v_k \in V$ mit $v_i = v_j$ für bestimmte $i \not= j$ ist.
\end{Lemma}

\begin{Lemma}{alternierende Multilinearform ist $0$ bei linear abhängigen
              Vektoren} \\
    Seien $\char K \not= 2$ und
    $f: V^{\times k} \rightarrow K$ eine $k$-fache alternierende
    Multilinearform auf $V$. \\
    Dann gilt $f(v_1, \dotsc, v_k) = 0$, falls
    $v_1, \dotsc, v_k \in V$ linear abhängige Vektoren sind.
\end{Lemma}

\begin{Lemma}{Umkehrung}
    Sei $f: V^{\times k} \rightarrow K$ eine $k$-fache Multilinearform auf $V$.
    Dann ist $f$ alternierend, wenn
    $f(v_1, \dotsc, v_k) = 0$ für jede linear abhängige Menge
    $\{v_1, \dotsc, v_k\}$ ist.
\end{Lemma}

\begin{Bem}
    Also: Ist $\char(K) \not= 2$, dann ist $f$ alternierend genau dann,
    wenn $f(v_1, \dotsc, v_k) = 0$ für jedes linear abhängige Tupel
    $(v_1, \dotsc, v_k)$ ist.
    Für $\char(K) = 2$ gibt es alternierende Multilinearformen, die diese
    Bedingung nicht erfüllen.
    Sie ist daher stärker als die Definition "`alternierend"' von oben und
    deswegen wird die Definition verschärft.
\end{Bem}

\pagebreak

\begin{Def}{alternierende Multilinearform}
    Sei $f: V^{\times k} \rightarrow K$ eine $k$-fache Multilinearform auf
    $V$. \\
    $f$ heißt \begriff{alternierend}, falls $f(v_1, \dotsc, v_k) = 0$
    für jedes linear abhängige Tupel $(v_1, \dotsc, v_k)$ ist,
    wobei $v_i \in V$ für $i = 1, \dotsc, k$.
\end{Def}

\begin{Satz}{Basis und alternierende Multilinearform}
    Seien $n = \dim_K V$, $f: V^{\times n} \rightarrow K$ eine $n$-fache
    alternierende Multilinearform auf $V$ mit $f \not= 0$
    und $v_1, \dotsc, v_n \in V$. \\
    Dann ist $\basis{B} = (v_1, \dotsc, v_n)$ Basis von $V$ genau dann,
    wenn $f(v_1, \dotsc, v_n) \not= 0$ ist.
\end{Satz}

\begin{Satz}{alternierende Multilinearformen als Unterraum} \\
    Die Menge $\alt_k(V)$ der $k$-fachen alternierenden Multilinearformen
    auf $V$ ist ein Unterraum der Menge der $k$-fachen Multilinearformen
    auf $V$.
\end{Satz}

\begin{Satz}{Basis des Vektorraums aller (alternierenden) Multilinearformen
             auf $V$} \\
    Sei $\basis{B} = (v_1, \dotsc, v_n)$ Basis von $V$. \\
    $e_\mi{j}$ sei definiert durch $e_\mi{j}: V^{\times k} \rightarrow K$,
    $e_\mi{j}(v_\mi{\ell}) = \delta_{\mi{j} \mi{\ell}}$, wobei
    $\mi{j}, \mi{\ell} \in \{1, \dotsc, n\}^{\times k}$ ist. \\
    $\pi(\mi{i}) \in \mi{I}$ sei für $\mi{i} = (i_1, \dotsc, i_n)$ und
    $\pi \in \perm_k$ definiert durch
    $\pi(\mi{i}) = (i_{\pi(1)}, \dotsc, i_{\pi(k)})$.
    Dann gilt: \\
    1. Sind $u_1, \dotsc, u_k \in V$ und $\pi \in \perm_k$, dann ist
    $e_\mi{i}(u_{\pi(1)}, \dotsc, u_{\pi(k)}) =
    e_{\pi^{-1}(\mi{i})}(u_1, \dotsc, u_k)$. \\
    2. $\{e_\mi{j} \;|\; \mi{j} \in \{1, \dotsc, n\}^{\times k}\}$ ist
    Basis des Vektorraums aller $k$-fachen Multilinearformen auf $V$. \\
    3. Sei $a_\mi{i} = \sum_{\pi \in \perm_k} \sign(\pi) e_{\pi(\mi{i})}$.
    Dann ist $\{a_\mi{i} \;|\; \mi{i} =
    (i_1, \dotsc, i_k) \in \{1, \dotsc, n\}^{\times k},\; i_1 < \dotsb < i_k\}$
    Basis von $\alt_k(V)$.
\end{Satz}

\begin{Kor}
    Es gilt $\dim_K \alt_k(V) = \binom{n}{k} =
    \Big|\{(i_1, \dotsc, i_k) \in \{1, \dotsc, n\}^{\times k} \;|\;
    1 \le i_1 < \dotsb < i_k \le n\}\Big|$. \\
    Insbesondere gilt $\dim_K \alt_k(V) = 1$ für $k = n$ und
    $\dim_K \alt_k(V) = 0$ für $k > n$.
\end{Kor}

\begin{Satz}{alternierende Multilinearformen und Determinanten} \\
    Seien $\dim_K V = n$ und $f$ eine $n$-fache alternierende Multilinearform
    auf $V$. \\
    Ist $\basis{B} = (v_1, \dotsc, v_n)$ Basis von $V$ und ist
    $u_i = \sum_{j=1}^n \lambda_{i,j} v_j$ für $\lambda_{i,j} \in K$ und
    $i = 1, \dotsc, n$, dann ist
    $f(u_1, \dotsc, u_n) = f(v_1, \dotsc, v_n) \cdot
    \sum_{\pi \in \perm_n} \sign(\pi) \lambda_{1,\pi(n)} \dotsm
    \lambda_{n,\pi(n)} = f(v_1, \dotsc, v_n) \cdot \det(\lambda_{ij})$.
\end{Satz}

\section{%
    Determinanten%
}

\begin{Def}{Determinante}
    Seien $V$ ein $K$-Vektorraum mit $\dim_K V = n$ und
    $\phi \in \End_K(V)$. \\
    Dann ist die \begriff{Determinante} $D(\phi)$ des Endomorphismus
    $\phi$ von $V$ folgendermaßen definiert: \\
    Man wähle eine von der Nullform verschiedene $n$-fache alternierende
    Multilinearform $f$ von $V$ (existiert nach Folgerung oben)
    sowie eine beliebige Basis $\basis{B} = (v_1, \dotsc, v_n)$ von $V$. \\
    Dann ist \fracsize{$D(\phi) =
    \frac{f(\phi(v_1), \dotsc, \phi(v_n))}{f(v_1, \dotsc, v_n)}$}.
\end{Def}

\begin{Satz}{Determinante wohldefiniert}
    Sei $\phi \in \End_K(V)$.
    Dann ist $D(\phi) \in K$ unabhängig von der Wahl der Basis $\basis{B}$
    von $V$ und von der Wahl der Form $f \in \alt_n(V)$, $f \not= 0$
    definiert.
\end{Satz}

\begin{Satz}{Determinante stimmt mit bekannter Definition überein}
    Seien $\phi \in \End_K(V)$, \\
    $\basis{B} = (v_1, \dotsc, v_n)$ eine
    Basis von $V$ sowie $\phi(v_j) = \sum_{i=1}^n \lambda_{ij} v_i$
    für $j = 1, \dotsc, n$. \\
    Dann ist $D(\phi) = \sum_{\pi \in \perm_n} \sign(\pi)
    \lambda_{1 \pi(1)} \dotsm \lambda_{n \pi(n)}$
    und deswegen stimmen die Definitionen der Determinante überein.
\end{Satz}

\begin{Satz}{Rechenregeln}
    Seien $\phi, \psi \in \End_K(V)$.
    Dann gilt:
    1. $D(\phi) \not= 0 \;\Leftrightarrow\; \phi \in \Aut_K(V)$, \\\
    2. $D(\id_V) = 1$, \qquad
    3. $D(\phi \circ \psi) = D(\phi) D(\psi)$, \qquad
    4. $D(\phi^{-1}) = (D(\phi))^{-1}$ für $\phi \in \Aut_K(V)$.
\end{Satz}

\pagebreak

\begin{Bem}
    Man kann leicht auch folgende bekannte Regeln zeigen:
    Ist eine Spalte von $A$ der Nullvektor, so ist $\det A = 0$.
    Hat $A$ zwei identische Spalten, so ist $\det A = 0$.
    Addiert man zu einer Spalte von $A$ das $\lambda$-fache einer anderen,
    so ändert sich die Determinante nicht.
    Vertauscht man zwei Spalten von $A$, so ändert sich das Vorzeichen
    der Determinante.
    Wenn man eine Spalte mit $\lambda \in K$, $\lambda \not= 0$ multipliziert,
    so multipliziert sich die Determinante mit $\lambda$. \\
    Außerdem kann man mit der ursprünglichen Definition leicht
    $\det(A) = \det(A^t)$ zeigen.
    Daher gelten alle Behauptungen auch für Zeilen.
\end{Bem}

\begin{Satz}{Entwicklungssatz von \textsc{Laplace}}
    Seien $k \in \{1, \dotsc, n\}$ und $A = (\alpha_{ij})$. \\
    Dann gilt $\det A = \sum_{i=1}^n (-1)^{i+k} a_{ik} \det(A_{ik})$.
\end{Satz}

\section{%
    \emph{Zusatz}: Projekt 9 und 10 (projektive Geometrie)%
}

\begin{Def}{projektiver Raum}
    Ein \begriff{projektiver Raum} $P$ über einem Körper $K$ ist die Menge aller
    eindimensionaler Unterräume eines $K$-Vektorraums $V_P$.
\end{Def}

\begin{Def}{projektiver Unterraum}
    Eine Teilmenge $U \subseteq P$ heißt \begriff{projektiver Unterraum} von
    $P$, falls sie genau aus den eindimensionalen Unterräumen eines Unterraums
    $V_U \ur V_P$ besteht. \\
    Alternativ: $U \subseteq P$ ist projektiver Unterraum von $P$, falls $U$
    ein projektiver Raum ist.
\end{Def}

\begin{Def}{projektive Dimension}
    Die \begriff{projektive Dimension} eines projektiven Raums $P$ ist
    definiert durch $\pdim P = \dim_K V_P - 1$.
\end{Def}

\begin{Def}{Punkt, Gerade, Ebene}
    Für einen Punkt $p \in P$ gibt es ein $p' \in V_P$ mit $p' \not= 0$, sodass
    $p = \aufspann{p'}$.
    Die leere Menge ist ein Unterraum von $P$, wobei $V_\emptyset = (0)$ ist
    (daher gilt $\pdim \emptyset = -1$). \\
    \begriff{Punkte}, \begriff{Geraden} und \begriff{Ebenen} sind Unterräume
    der p-Dimension $0$, $1$ und $2$. \\
    Ein Unterraum $H$ von $P$ mit $\pdim P = n$ und $\pdim H = n - 1$ heißt
    Hyperebene.
\end{Def}

\begin{Def}{Fernhyperebene}
    Sei $P \not= \emptyset$ ein $n$-dimensionaler
    projektiver Raum und $H$ eine Hyperebene von $P$.
    Dann ist $A = P \setminus H$ der
    \begriff{zu $H$ gehörende af"|fine Raum} von $P$. \\
    Die Punkte von $A$ heißen \begriff{eigentliche Punkte},
    die Punkte von $H$ heißen \begriff{uneigentliche Punkte}. \\
    $H$ heißt \begriff{uneigentliche Hyperebene} oder
    \begriff{Fernhyperebene} von $P$.
\end{Def}

\begin{Satz}{Dimensionsformel}
    Seien $M$ und $N$ projektive Unterräume von $P$. \\
    Dann sind auch $M \cap N$ (\begriff{Schnittraum}) bzw.
    $M \lor N = \bigcap_{U \ur P,\; U \supseteq M,N} U$
    (\begriff{Verbindungsraum}) Unterräume von $P$ mit
    $V_{M \cap N} = V_M \cap V_N$ bzw. $V_{M \lor N} = V_M + V_N$. \\
    Es gilt $\pdim M + \pdim N = \pdim(M \lor N) + \pdim(M \cap N)$.
\end{Satz}

\begin{Def}{unabhängige Punkte}
    Seien $p_0, \ldots, p_k$ Punkte des projektiven Raums $P$. \\
    $p_0, \ldots, p_k$ heißen \begriff{unabhängig}, falls
    $\pdim(p_0 \lor \cdots \lor p_k) = k$ gilt. \\
    Die Punkte $p_0, \ldots, p_k \in P$ sind genau dann unabhängig, falls
    $p_0', \ldots, p_k'$ linear unabhängige Vektoren sind
    ($\aufspann{p_i'} = p_i$ für $i = 0, \ldots, k$).
\end{Def}

\begin{Def}{projektives Koordinatensystem}
    Ein geordnetes $n+2$-Tupel $K = (q_0, \ldots, q_n, e)$ heißt
    \begriff{projektives Koordinatensystem},
    falls je $n + 1$ Punkte aus $K$ unabhängig sind.
    Die Punkte $q_0, \ldots, q_n$ heißen \begriff{Grundpunkte} und $e$ heißt
    \begriff{Einheitspunkt} von $K$.
\end{Def}

\begin{Def}{homogene Koordinaten}
    Nach obigem Lemma gibt es $q_i' \in q_i$ und $e' \in e$
    mit $e = q_0 + \cdots + q_n$.
    Für jeden Punkt $x = \aufspann{x'} \in P$ hat $x' \not= 0$ die eindeutige
    Darstellung $x' = \lambda_0 q_0' + \cdots + \lambda_n q_n'$.
    Dabei sind die Skalare $\lambda_i \in K$ bis auf einen gemeinsamen Faktor
    durch $x$ eindeutig bestimmt.
    Die Skalare $\lambda_0, \ldots, \lambda_n \in K$ heißen die
    \begriff{homogenen Koordinaten} des Punktes $x \in P$ bzgl. des projektiven
    Koordinatensystems $K$.
    $(\lambda_0, \ldots, \lambda_n) \in K^{n+1}$ heißt
    \begriff{homogener Koordinatenvektor} und ist bis auf einen Faktor
    eindeutig bestimmt.
\end{Def}

\begin{Def}{projektive Abbildung, Projektivität}
    Seien $P_1, P_2$ ein projektiver Raum mit zugehörigen $K$-Vektorräumen
    $V_{P_1}, V_{P_2}$.
    Eine \begriff{projektive Abbildung} $f: P_1 \rightarrow P_2$ wird durch
    eine injektive lineare Abbildung $F: V_{P_1} \rightarrow V_{P_2}$
    mit $f(\aufspann{x}) = \aufspann{F(x)}$ induziert.
    $F$ muss injektiv sein, denn sonst gäbe es Elemente
    $x \in \ker F$, $x \not= 0$ mit
    $f(\aufspann{x}) = \aufspann{0} \notin P_2$. \\
    Ist $F$ bijektiv, so ist auch $f$ bijektiv und heißt
    \begriff{Projektivität}.
\end{Def}

\begin{Satz}{$P(V) \cong P(V^\ast)$}
    Sei $V$ endlich-dimensional.
    Dann ist $P(V)$ isomorph zu $P(V^\ast)$, wenn $P(V)$ der projektive Raum
    mit zugehörigem Vektorraum $V_{P(V)} = V$ ist.
\end{Satz}

\begin{Satz}{Dualitätsprinzip allgemein}
    Vertauscht man in einer wahren Aussage über Punkte, Geraden usw.
    eines projektiven Raums der p-Dimension $n$ die Begriffe
    "`Punkt"' mit "`Hyperebene"', "`Gerade"' mit
    "`$n - 2$-dimensionaler Unterraum"' usw.
    (also "`$i$-dimensionaler Unterraum"'
    mit "`$n - i - 1$ dimensionaler Unterraum"'),
    so erhält man wieder eine wahre Aussage.
\end{Satz}

\begin{Satz}{Dualitätsprinzip für projektive Ebenen}
    Vertauscht man in einer wahren Aussage über Punkte und Geraden einer
    projektiven Ebene die Begriffe "`Punkt"' mit
    "`Gerade"' sowie "`Verbindung"' mit "`Schnitt"' und umgekehrt,
    so erhält man wieder eine wahre Aussage.
\end{Satz}

\section{%
    \emph{Zusatz}: Projekt 11 (Tensorprodukte)%
}

\begin{Def}{freier Vektorraum über einer Menge}
    Sei $M$ eine Menge und $K$ ein Körper.
    Dann ist der \begriff{freie $K$-Vektorraum} $\mathcal{F}(M)$ über der
    Menge $M$ definiert durch \\
    $\mathcal{F}(M) = \{(k_m)_{m \in M} \;|\; k_m \in K \text{ fast alle } 0\}
    = \{k: M \rightarrow K \;|\; k(m) = 0 \text{ für fast alle } m \in M\}$. \\
    $\mathcal{F}(M)$ wird zum $K$-Vektorraum durch
    $(k + l): M \rightarrow K$, $(k + l)(m) = k(m) + l(m)$ und
    $(\lambda k): M \rightarrow K$, $(\lambda k)(m) = \lambda k(m)$ für
    $k \in \mathcal{F}(M)$.
\end{Def}

\begin{Def}{Tensorprodukt als Faktorraum}
    Seien $V, W$ $K$-Vektorräume.
    Dann ist das \begriff{Tensorprodukt} $V \otimes W$ definiert durch
    $V \otimes W = \mathcal{F}(V \times W)/R$ mit
    $R = \aufspann{S} \ur \mathcal{F}(V \times W)$ und \\
    $S = \{(v_1 + v_2, w) - (v_1, w) - (v_2, w),\;
    (v, w_1 + w_2) - (v, w_1) - (v, w_2),\;
    (\lambda v, w) - \lambda (v, w),\;
    (v, \lambda w) - (v, \lambda w) \;|\;
    v_1, v_2, v \in V,\; w_1, w_2, w \in W, \lambda \in K\} \subseteq
    \mathcal{F}(V \times W)$, wobei $(v, w) \in \mathcal{F}(V \times W)$ die
    Abbildung $f_{(v, w)}: V \times W \rightarrow K$, $f_{(v, w)}(x, y) = 1$
    für $(x, y) = (v, w)$ und $f_{(v, w)}(x, y) = 0$ sonst darstellt.
    $v \otimes w = (v, w) + R \in V \otimes W$ mit $v \in V$, $w \in W$ ist
    ein \begriff{einfacher Tensor}.
\end{Def}

\begin{Lemma}{Basis von $V \otimes W$}
    Ist $\basis{B} = (v_1, v_2, \dotsc)$ eine Basis von $V$ und
    $\basis{C} = (w_1, w_2, \dotsc)$ eine Basis von $W$, so ist
    $(v_1 \otimes w_1, v_1 \otimes w_2, \dotsc, v_2 \otimes w_1, v_2 \otimes w_2, \dotsc)$
    eine Basis von $V \otimes W$.
\end{Lemma}

\begin{Satz}{Tensorprodukt über universelle Eigenschaft}
    Seien $V$ und $W$ $K$-Vektorräume.
    Sei außerdem $A$ ein $K$-Vektorraum, der die folgenden Eigenschaften
    hat: \\
    1. Es gibt eine bilineare Abbilduing $j: V \times W \rightarrow A$. \qquad
    2. Ist $U$ ein $K$-Vektorraum und $f: V \times W \rightarrow U$ bilinear,
    so gibt es genau einen Homomorphismus $\widetilde{f}: A \rightarrow U$
    mit $\widetilde{f} \circ j = f$. \\
    Dann ist $A \cong V \otimes W$.
\end{Satz}

\begin{Bem}
    Man kann auch das Tensorprodukt über diesen Satz definieren, d.\,h.
    jeder $K$-Vektorraum $A$, der die \begriff{universelle Eigenschaft}
    erfüllt, heißt Tensorprodukt $V \otimes W$.
    Der Satz garantiert, dass so das Tensorprodukt bis auf Isomorphie eindeutig
    definiert ist.
\end{Bem}

\pagebreak

\chapter{%
    Die \textsc{Jordan}sche Normalform%
}

\section{%
    Der Satz von \textsc{Cayley}-\textsc{Hamilton}%
}

\begin{Satz}{Teilen von charakteristischen Polynomen}
    Seien $V$ ein endlich-dimensionaler $K$-Vektorraum, $f \in \End_K(V)$,
    $U$ ein $f$-invarianter Unterraum und $\widehat{f}$ die Einschränkung
    von $f$ auf $U$. \\
    Dann teilt das charakteristische Polynom der Einschränkung $\widehat{f}$
    das von $f$:
    $\chi_{\widehat{f}}(t) \;|\; \chi_f(t)$.
\end{Satz}

\begin{Bem}
    Man kann Endomorphismen in Polynome über $K$ einsetzen und erhält
    wieder Endomorphismen:
    Ist $p(t) = \sum \alpha_i t^i \in K[t]$ und $f \in \End_K(V)$,
    so ist $p(f) = \sum \alpha_i f^i \in \End_K(V)$. \\
    Für $p(t), q(t) \in K[t]$ gilt $(pq)(f) = p(f) \circ q(f)$.
\end{Bem}

\begin{Def}{zyklischer Unterraum}
    Sei $x \in V$. \\
    Dann heißt $W = \aufspann{x, f(x), f^2(x), \dotsc}$
    der \begriff{von $x$ erzeugte $f$-zyklische Unterraum} von $V$.
\end{Def}

\begin{Lemma}{über zyklische Unterräume}
    Es gilt $W = \{(p(f))(x) \;|\; p \in K[t]\}$. \\
    Der von $x$ erzeugte $f$-zyklische Unterraum $W$ ist $f$-invariant. \\
    $W$ ist der kleinste $f$-invariante Unterraum von $V$, der $x$ enthält.
\end{Lemma}

\begin{Satz}{Basis des zyklischen Unterraums}
    Seien $f \in \End_K(V)$, $W$ der von $x \in V$ erzeugte $f$-zyklische
    Unterraum von $V$ und $k = \dim_K W \ge 1$ (d.\,h. $x \not= 0$). \\
    Dann ist $\basis{B}_W = (x, f(x), f^2(x), \dotsc, f^{k-1}(x))$ eine
    Basis von $W$.
\end{Satz}

\begin{Bem}
    Es gibt $\alpha_0, \dotsc, \alpha_{k-1} \in K$, sodass
    $f^k(x) = -\alpha_0 x - \alpha_1 f(x) - \dotsb -
    \alpha_{k-1} f^{k-1}(x)$. \\
    Ist $\widetilde{f} = f|_W$, so ist
    $\matrixm_{\widetilde{f}}(\basis{B}_W) =$
    \matrixsize{$\begin{pmatrix}0 & \cdots & 0 & -\alpha_0 \\
    1 & & 0 & -\alpha_1 \\ \vdots & \ddots & \vdots & \vdots \\
    0 & \cdots & 1 & -\alpha_{k-1}\end{pmatrix}$} \\
    die \begriff{Begleitmatrix} des Polynoms
    $p(t) = \alpha_0 + \alpha_1 t + \dotsb + \alpha_{k-1} t^{k-1} + t^k$.
\end{Bem}

\begin{Satz}{charakteristisches Polynom der Einschränkung}
    Seien die Bezeichnungen wie eben und
    $f^k(x) = -\alpha_0 x - \alpha_1 f(x) - \dotsb - \alpha_{k-1}(x) f^{k-1}$.
    Dann ist das charakteristische Polynom von $\widetilde{f} = f|_W$
    gegeben durch
    $\chi_{\widetilde{f}}(t) = \alpha_0 + \alpha_1 t + \dotsb +
    \alpha_{k-1} t^{k-1} + t^k$.
\end{Satz}

\begin{Def}{erfüllt}
    Seien $f \in \End_K(V)$ und $p(t) \in K[t]$.
    Dann \begriff{erfüllt $f$ das Polynom $p(t)$}, falls
    $p(f) \equiv 0$.
\end{Def}

\begin{Satz}{\textsc{Cayley}-\textsc{Hamilton}}
    Seien $f \in \End_K(V)$ und $V$ endlich-dimensional. \\
    Dann erfüllt $f$ sein charakteristisches Polynom $\chi_f(t)$.
\end{Satz}

\pagebreak

\section{%
    Verallgemeinerte Eigenräume%
}

\begin{Def}{\textsc{Jordan}-Block/-Form}
    \matrixsize{$J_\lambda(k) = \begin{pmatrix}\lambda & 1 & 0 & \cdots & 0 \\
    0 & \lambda & 1 & & 0 \\ \vdots & & \ddots & \ddots & \vdots \\
    0 & 0 & & \lambda & 1 \\ 0 & 0 & \cdots & 0 & \lambda\end{pmatrix}$},
    \qquad
    \matrixsize{$\begin{pmatrix}J_1 & 0 & \cdots & 0 & 0 \\
    0 & J_2 & & 0 & 0 \\ \vdots & & \ddots & & \vdots \\
    0 & & & J_{k-1} & 0 \\ 0 & 0 & \cdots & 0 & J_k\end{pmatrix}$} \\
    Eine $k \times k$-Matrix der Form $J_\lambda(k)$
    heißt \begriff{\textsc{Jordan}-Block}. \\
    Eine Matrix heißt in
    \begriff{\textsc{Jordan}-Form} oder
    \begriff{\textsc{Jordan}sche Normalform}, wenn sie in der Form
    wie oben rechts ist mit $J_i = J_{\lambda_i}(k_i)$ für $i = 1, \dotsc, k$,
    wobei die $\lambda_i$ die (nicht notwendigerweise verschiedenen) Eigenwerte
    von $A$ sind und $k_i \in \natural$ ist.
\end{Def}

\begin{Def}{\textsc{Jordan}-Basis}
    Seien $V$ ein endlich-dimensionaler Vektorraum und $f \in \End_K(V)$,
    wobei das charakteristische Polynom $\chi_f(t)$ in Linearfaktoren
    zerfällt. \\
    Eine \begriff{\textsc{Jordan}-Basis} von $f$ ist eine Basis
    $\basis{B}_f$ von $V$, sodass $\matrixm_f(\basis{B}_f)$ in Jordanform
    ist.
\end{Def}

\begin{Def}{verallgemeinerter Eigenraum}
    Seien $f \in \End_K(V)$, $V$ endlich-dimensional und $\lambda \in K$.
    Dann ist $\ker(f - \ell_\lambda) \ur \ker(f - \ell_\lambda)^2 \ur \dotsb
    \ur \ker(f - \ell_\lambda)^i \ur \dotsb$ eine aufsteigende Kette
    von Unterräumen von $V$, die terminiert
    (d.\,h. es gibt $k \in \natural$, sodass
    $\ker(f - \ell_\lambda)^{n+i} = \ker(f - \ell_\lambda)^n$ für alle
    $i \in \natural$).
    Daher ist $\mathcal{V}_\lambda(f) = \bigcup_{i=1}^\infty
    \ker(f - \ell_\lambda)^i$ ein wohldefinierter Unterraum von $V$.
    $\mathcal{V}_\lambda(f)$ heißt \begriff{verallgemeinerter Eigenraum}
    zum Eigenwert $\lambda$ von $f$ und seine Elemente heißen
    \begriff{verallgemeinerte Eigenvektoren} von $f$.
    Also gilt $\mathcal{V}_\lambda(f) = \{v \in V \;|\;
    \exists_{p \in \natural}\; (f - \ell_\lambda)^p(v) = 0\}$. \\
    Analog kann man auch für quadratische Matrizen $\mathcal{V}_\lambda(A)$
    definieren.
\end{Def}

\begin{Bem}
    Sei $\matrixm_f(\basis{B}_f) = J_\lambda(n)$.
    Dann ist $V_\lambda(f)$ ein- und $\mathcal{V}_\lambda(f)$ $n$-dimensional.
    Ist $\basis{B}_f = (v_1, \dotsc, v_n)$, so ist $v_1 \in V_\lambda(f)$ der
    bis auf skalare Vielfache eindeutig bestimmte Eigenvektor von $f$ mit
    Eigenwert $\lambda$ und $\basis{B}_f$ ist die \begriff{zyklische Basis} des
    von $v_n$ erzeugten $f - \ell_\lambda$-zyklischen Unterraums von $V$.
\end{Bem}

\begin{Satz}{$\mathcal{V}_\lambda(f)$ ist $f$-invarianter Unterraum}
    Sei $\lambda$ ein Eigenwert von $f \in \End_K(V)$. \\
    Dann ist $\mathcal{V}_\lambda(f)$ ein $f$-invarianter Unterraum von $V$,
    der den Eigenraum $V_\lambda(f)$ enthält.
\end{Satz}

\begin{Def}{Zykel}
    Seien $\lambda$ ein Eigenwert von $f \in \End_K(V)$, $v$ ein
    verallgemeinerter Eigenvektor zu $\lambda$
    (d.\,h. $v \in \mathcal{V}_\lambda(f)$)
    und $p \in \natural$ die kleinste
    natürliche Zahl, sodass $(f - \ell_\lambda)^p(v) = 0$. \\
    Dann ist $\basis{B} = ((f - \ell_\lambda)^{p-1}(v),
    (f - \ell_\lambda)^{p-2}(v), \dotsc, (f - \ell_\lambda)(v), v)$
    eine Basis des von $v$ erzeugten $f - \ell_\lambda$-zyklischen Unterraums
    von $V$. \\
    $\basis{B}$ ist \begriff{der von $v$ erzeugte Zykel verallgemeinerter
    Eigenvektoren von $f$} oder kurz \begriff{$\lambda$-Zykel von $f$}.
    $v$ heißt der \begriff{Anfangsvektor} und $(f - \ell_\lambda)^{p-1}(v)$
    der \begriff{Endvektor} des Zykels.
\end{Def}

\begin{Satz}{Eigenschaften von Anfangs-/Endvektor}
    Sei $\basis{B}$ ein $\lambda$-Zykel von $f$. \\
    Dann ist $\basis{B}$ eine Basis des vom Anfangsvektor erzeugten
    $f - \ell_\lambda$-zyklischen Unterraums $W$ von $V$ und dieser
    ist $f$-invariant.
    Die Einschränkung von $f$ auf $W$ besitzt genau einen eindimensionalen
    Eigenraum und dieser wird vom Endvektor des Zykels $\basis{B}$ erzeugt.
    Es gilt $\enmatrix{f|_W}{B} = J_\lambda(p)$.
\end{Satz}

\begin{Satz}{Jordanbasis $\;\Leftrightarrow\;$
             disjunkte Vereinigung von Zykeln}
    Sei $\basis{B}$ eine geordnete Basis von $V$. \\
    Dann ist $\basis{B}$ eine Jordanbasis von $f$ genau dann, wenn
    $\basis{B}$ eine disjunkte Vereinigung von Zykeln verallgemeinerter
    Eigenvektoren von $f$ ist.
\end{Satz}

\begin{Satz}{$V$ ist direkte Summe der verallgemeinerten Eigenräume}
    Sei $f \in \End_K(V)$, wobei $\chi_f(t)$ in Linearfaktoren zerfällt.
    Dann ist $V$ die direkte Summe der verallgemeinerten Eigenräume
    $V = \bigoplus_{\lambda} \mathcal{V}_\lambda(f)$, wobei $\lambda$
    die Menge der Eigenwerte von $f$ durchläuft.
\end{Satz}

\begin{Kor}
    Seien $\lambda_1, \dotsc, \lambda_k$ die paarweise verschiedenen
    Eigenwerte von $f$,
    $\basis{B}_i$ eine Basis von $\mathcal{V}_{\lambda_i}(f)$,
    $\basis{B} = \bigcup_{i=1}^k \basis{B}_i$ und
    $f_i$ die Einschränkung von $f$ auf $\mathcal{V}_{\lambda_{i}}(f)$. \\
    Dann ist $\enmatrix{f}{B} =$ \matrixsize{%
    $\begin{pmatrix}A_1 & & 0 \\ & \ddots & \\ 0 & & A_k\end{pmatrix}$},
    wobei $A_i = \matrixm_{f_i}(\basis{B}_i)$ ist.
\end{Kor}

\section{%
    Die \textsc{Jordan}sche Normalform: Algorithmus%
}

\begin{Bem}
    Im Folgenden wird versucht, ein Algorithmus zur Bestimmung der
    JNF und der zugehörigen Jordanbasis eines
    Endomorphismus bzw. einer Matrix zu finden, wobei immer vorausgesetzt wird,
    dass das charakteristische Polynom vollständig in Linearfaktoren
    zerfällt. \\
    Zur Einfachheit kann dank obiger Folgerung angenommen werden, dass
    $\chi_f(t) = (t - \lambda)^n$, d.\,h. $f$ besitzt genau einen
    Eigenwert $\lambda$ mit Vielfachheit $n$.
\end{Bem}

\begin{Lemma}{Kern-Dimensionen eines Jordanblocks}
    Sei $J = J_\lambda(k)$ ein Jordanblock. \\
    Dann ist $\dim_K \ker(J - \lambda E)^i = i$ für $i = 1, \dotsc, k$ und
    $\dim_K \ker(J - \lambda E)^i = k$ für $i > k$.
\end{Lemma}

\begin{Lemma}{Bestimmung der Anzahl und
              Größen der Jordanblöcke einer Matrix} \\
    Seien $A$ eine Matrix in Blockdiagonalform, deren $s$ Diagonalblöcke
    Jordanblöcke $J_i = J_\lambda(i)$ sind ($\lambda \in K$ fest),
    sowie $n_i = \dim_K \ker(A - \lambda E)^i$ und $r \in \natural$,
    sodass $n_{r - 1} < n_r = n_{r+1}$. \\
    Sei außerdem $k_i \in \natural_0$ die Anzahl der vorkommenden Kästchen
    $J_i$. \\
    Dann ist $n_1 = k_1 + k_2 + k_3 + \dotsb + k_r$, \qquad
    $n_2 = n_1 + k_2 + k_3 + \dotsb + k_r$, \\
    $n_3 = n_2 + k_3 + \dotsb + k_r$, \quad \dots, \quad
    $n_r = n_{r-1} + k_r$. \\
    Insbesondere ist $n_i - n_{i-1} = k_i + k_{i+1} + \dotsb + k_r$ für
    $i = 2, \dotsc, r$. \\
    Daher lassen sich die $k_i$ rekursiv aus den $n_j$ ausrechnen.
\end{Lemma}

\begin{Prozedur}{Bestimmung der Jordanschen Normalform (1)} \\
    Sei $A \in M_n(K)$, sodass $\chi_A(t)$ in Linearfaktoren zerfällt. \\
    Dann kann folgendermaßen die Jordansche Normalform von $A$ bestimmt werden:
    \begin{enumerate}
        \item Man ermittelt die Eigenwerte von $A$.
        Für jeden Eigenwert $\lambda \in K$ von $A$ werden die folgenden
        Schritte durchgeführt:

        \item Man berechnet $n_i = \dim_K \ker(A - \lambda E)^i$ für
        $i = 1, 2, \dotsc$.
        Beim ersten $r$ mit $n_r = n_{r+1}$ bricht man ab, denn
        die Dimensionen bleiben dann konstant.

        \item Man berechnet $l_i = n_i - n_{i-1}$ für $i = 1, \dotsc, r$,
        wobei $n_0 = 0$.

        \item Man berechnet $k_i = l_i - l_{i+1}$ für $i = 1, \dotsc, r$,
        wobei $l_{r+1} = 0$.

        \item Der Block der Jordanform von $A$, der zum Eigenwert $\lambda$
        korrespondiert, ist die Blockdiagonalmatrix, bei der
        $J_\lambda(i)$ genau $k_i$-mal als Diagonalblock auftritt.
    \end{enumerate}
\end{Prozedur}

\begin{Prozedur}{Bestimmung der Jordanschen Normalform (2)} \\
    Gegeben seien die $n_i$ wie eben.
    Man malt ein Diagramm aus Kreuzen in der Ebene in einem Gitter
    und zwar in die erste Zeile $l_1 = n_1$ Kreuze, in die zweite
    $l_2 = n_2 - n_1$ und in die $i$-te Zeile
    $l_i = n_i - n_{i-1}$ Kreuze. \\
    Wegen $l_i = k_i + k_{i+1} + \dotsb + k_r$ erhält man eine abfallende
    Folge natürlicher Zahlen, die sich mit $l_1 + l_2 + \dotsb + l_r =
    (n_1 - 0) + (n_2 - n_1) + \dotsb + (n_r - n_{r-1}) = n_r$ gerade zu
    $n_r = \dim_K \mathcal{V}_\lambda(A)$ aufsummieren. \\
    Die Spalten des entstehenden Diagramms geben dann gerade
    die $\lambda$-Zyklen wieder:
    Eine Spalte mit $k$ Kreuzen entspricht einem Jordanblock $J_\lambda(k)$
    der Größe $k$ von $A$. \\
    Das Diagramm heißt \begriff{\textsc{Young}-Diagramm}
    zur Partition $l_1 \ge \dotsb \ge l_r$ von $n_r$ oder
    $\lambda$-Diagramm von $A$ und wird mit $\mathfrak{D}_\lambda$
    bezeichnet. \\
    Im Diagramm entsprechen den untersten/obersten Spitzen der Spalten
    die Anfangs-/Endvektoren der $\lambda$-Zykeln.
\end{Prozedur}

\begin{Def}{linear unabhängig modulo $U$}
    Seien $U \ur V$ und $y_1, \dotsc, y_s \in V$.
    Dann sind die $y_i$ linear unabhängig modulo $U$, falls die
    Nebenklassen $y_1 + U, \dotsc, y_s + U$ in $V/U$ linear unabhängig sind,
    d.\,h. ist $\sum_{i=1}^s \lambda_i y_i \in U$ mit
    $\lambda_1, \dotsc, \lambda_s \in K$, dann ist
    $\lambda_1 = \dotsb = \lambda_s = 0$. \\
    Sind $y_1, \dotsc, y_s$ linear unabhängig modulo $U$, so sind sie
    linear unabhängig in $V$.
    Die Umkehrung gilt nicht.
\end{Def}

\begin{Satz}{Vereinigung von Zykeln ist linear unabhängig}
    Seien $f \in \End_K(V)$ und $\lambda \in K$ ein Eigenwert von $f$.
    Für $i = 1, \dotsc, s$ seien $\lambda$-Zyklen $Z_i$ von $f$ mit derselben
    Länge $t$ gegeben, wobei $y_i$ der Anfangsvektor von $Z_i$ ist. \\
    Ist die Menge der Anfangsvektoren $\{y_1, \dotsc, y_s\}$ linear
    unabhängig modulo $\ker(f - \ell_\lambda)^{t-1}$, so ist
    $Z = \bigcup_{i=1}^s Z_i$ ebenfalls linear unabhängig. \\
    Insbesondere ist daher die Summe der von den $Z_i$ aufgespannten
    Unterräume direkt.
\end{Satz}

\begin{Kor}
    Seien wie eben $y_1, \dotsc, y_s \in \ker(f - \ell_\lambda)^t$, deren
    Restklassen im Faktorraum \\
    $\ker(f - \ell_\lambda)^t / \ker(f - \ell_\lambda)^{t-1}$ linear
    unabhängig sind. \\
    Dann sind die von den $y_i$ erzeugten $\lambda$-Zykel paarweise
    disjunkt.
\end{Kor}

\begin{Lemma}{höhere Kerne bleiben gleich}
    Sei $\mathcal{N}_i = \ker(f - \ell_\lambda)^i$. \\
    Gilt $\mathcal{N}_r = \mathcal{N}_{r+1}$, so gilt
    $\mathcal{N}_r = \mathcal{N}_{r+i}$ für alle $i \in \natural$.
\end{Lemma}

\begin{Prozedur}{Bestimmung der Jordanbasis} \\
    Sei $f \in \End_K(V)$, sodass $\chi_f(t)$ in Linearfaktoren zerfällt. \\
    Dann kann folgendermaßen die Jordansche Normalform von
    $f$ bestimmt werden:
    \begin{enumerate}
        \item
        Sei $r \in \natural$ minimal mit $\mathcal{N}_r = \mathcal{N}_{r+1}$
        (Anzahl der Zeilen im $\lambda$-Diagramm).
        Man ergänzt eine Basis von $\mathcal{N}_{r-1}$ mit
        $y_1, \dotsc, y_{k_r}$  zu einer Basis von $\mathcal{N}_r$.

        \item
        Im $\lambda$-Diagramm ordnet man der $i$-ten Spalte von unten nach
        oben den Kreuzen die Elemente
        $y_i, (f - \ell_\lambda)(y_i), \dotsc, (f - \ell_\lambda)^{r-1}(y_i)$
        für $i = 1, \dotsc, k_r$ zu.
        Die Vektoren einer Spalte bilden dann einen $\lambda$-Zykel von $f$.
        Sei $U_1$ die Summe der von diesen $\lambda$-Zykeln aufgespannten
        Unterräumen, dann bilden die $(f - \ell_\lambda)^k y_i$
        mit $i = 1, \dotsc, k_r$ und $k = 1, \dotsc, r$ eine Basis von $U_1$.

        \item
        Die nächste, also die $k_r + 1$-te Spalte ist kürzer als die
        vorherigen.
        Sei sie von der Länge $t$ und $k_t$ die Anzahl der Spalten dieser
        Länge.
        Es gibt $k_t$ Basiselemente in einem Komplement von
        $(U_1 \cap \mathcal{N}_t) + \mathcal{N}_{t-1}$ in $\mathcal{N}_t$
        und nehmen wie eben die davon erzeugten $\lambda$-Zyklen von $f$.
        Diese erzeugen $U_2$ und sind eine Basis von $U_2$.

        \item
        Die nächste, also die $k_r + k_t + 1$-te Spalte ist kürzer als die
        vorherigen.
        Sei sie von der Länge $w$ und $k_w$ die Anzahl der Spalten dieser
        Länge.
        Es gibt $k_w$ Basiselemente in einem Komplement
        von $((U_1 + U_2) \cap \mathcal{N}_w) + \mathcal{N}_{w-1}$
        in $\mathcal{N}_w$
        und nehmen wie eben die davon erzeugten $\lambda$-Zyklen von $f$.
        Diese erzeugen $U_3$ und sind eine Basis von $U_3$.

        \item
        Man fährt so fort, bis man eine Basis von ganz
        $\mathcal{V}_\lambda(f)$ konstruiert hat.
        Jedem Kreuz im $\lambda$-Diagramm ist nun genau ein Basiselement
        zugeordnet.
        Diese werden nun spaltenweise (und von oben nach unten)
        durchnummeriert und bilden dann die Jordanbasis.
    \end{enumerate}
\end{Prozedur}

\begin{Def}{Fahne, angepasst}
    Sei $V$ ein $K$-Vektorraum.
    Eine \begriff{Fahne} der Länge $k$ in $V$ ist eine aufsteigende
    Kette $\mathcal{F}: (0) = U_0 \ur U_1 \ur \dotsb \ur U_k \ur V$ von
    Unterräumen $U_i$ von $V$. \\
    Eine Basis $\basis{B} = (v_1, \dotsc, v_n)$ von $V$ heißt
    \begriff{an $\mathcal{F}$ angepasst}, falls
    $(v_1, \dotsc, v_{m_i})$ eine Basis von $U_i$ ist, wobei
    $m_i = \dim_K U_i$ ist.
\end{Def}

Die Unterräume $\ker(f - \ell_\lambda)^i$ von $\mathcal{V}_\lambda(f)$
sind ein Beispiel von Fahnen, wobei die zugehörige Jordanbasis angepasst ist.

\begin{Lemma}{Eigenwerte und charakteristisches Polynom von
             nilpotenten/unipotenten Matrizen} \\
    Eine nilpotente Matrix $A \in M_n(K)$ kann nur $0$ als Eigenwert haben,
    d.\,h. $\chi_A(t) = t^n$. \\
    Ist $A$ unipotent, dann muss für jeden Eigenwert $\lambda^k = 1$ gelten,
    d.\,h. $\lambda$ ist eine $k$-te Einheitswurzel.
    Also ist $\chi_A(t) = \prod_{i=1}^n (t - \zeta_i)$ mit $\zeta_i^k = 1$.
\end{Lemma}

\begin{Lemma}{binomischer Lehrsatz im Ring}
    Seien $R$ ein Ring und $a, b \in R$ mit $ab = ba$. \\
    Dann gilt $(a + b)^n = \sum_{i=0}^n \binom{n}{i} a^i b^{n-i}$.
    Ist zusätzlich eines der beiden Ringelemente nilpotent, so lässt
    sich die Summe einfach auswerten.
\end{Lemma}

\begin{Lemma}{Jordanform ist Summe einer Diagonalmatrix und einer
              nilpotenten Matrix} \\
    Sei $A \in M_n(K)$ in Jordanform.
    Dann ist $A = D + N$ mit $DN = ND$, wobei $D$ eine Diagonalmatrix und
    $N$ eine nilpotente Matrix ist.
\end{Lemma}

\begin{Lemma}{ähnliche Matrizen zu nilpotenter Matrix sind nilpotent} \\
    Seien $A, N \in M_n(K)$ ähnlich, wobei $N$ nilpotent (unipotent) ist. \\
    Dann ist $A$ ebenfalls nilpotent (unipotent).
\end{Lemma}

\begin{Satz}{Jordanzerlegung}
    Sei $A \in M_n(K)$, sodass $\chi_A(t)$ in Linearfaktoren zerfällt. \\
    Dann ist $A = S + N$ mit $SN = NS$, wobei $S$ eine diagonalisierbare
    und $N$ eine nilpotente Matrix ist.
    Diese Zerlegung heißt \begriff{Jordanzerlegung} von $A$.
\end{Satz}

\section{%
    Das Minimalpolynom%
}

\begin{Def}{Ideal}
    Sei $R$ ein Ring (oder eine $K$-Algebra).
    Eine nicht-leere Teilmenge $I \subseteq R$ heißt \begriff{Rechtsideal},
    falls $a - b \in I$ und $ar \in I$ für alle $a, b \in I$, $r \in R$ ist.
    Gilt $a - b \in I$ und $ra \in I$ für alle $a, b \in I$, $r \in R$,
    so heißt I \begriff{Linksideal}. \\
    Ein \begriff{(zweiseitiges) Ideal} ist eine nicht-leere Teilmenge
    $I \subseteq R$, die zugleich Links- und Rechtsideal ist.
    In diesem Fall schreibt man $I \trianglelefteq R$.
\end{Def}

\begin{Bem}
    Es gilt $0 \cdot i = 0 \in I$ für jedes Ideal.
    Sind $a, b \in I$, so ist auch $a + b \in I$, da $0 - b = -b \in I$ ist.
    Jedes Ideal $I \trianglelefteq R$ ist auch ein Ring, indem man die
    Addition und Multiplikation von $R$ auf $I$ einschränkt.
    Ist $J \trianglelefteq R$ und $J \subseteq I$, so ist
    $J \trianglelefteq I$.
    Ist $R$ ein kommutativer Ring, so sind Ideale, Links- und Rechtsideale
    dasselbe.
\end{Bem}

\begin{Def}{Faktorring}
    Seien $R$ ein Ring und $I \trianglelefteq R$ ein Ideal. \\
    Dann wird durch $r \sim s \;\Leftrightarrow\; r - s \in I$
    für $r, s \in R$ auf $R$ eine Äquivalenzrelation definiert. \\
    Die Äquivalenzklasse von $r \in R$ heißt $r + I$ und
    die Menge der Äquivalenzklassen mit \\
    $R/I = \{r + I \;|\; r \in R\}$.
    $R/I$ wird zum Ring durch
    $(r + I) + (s + I) = (r + s) + I$ und
    $(r + I) \cdot (s + I) = (r \cdot s) + I$
    und heißt \begriff{Faktorring}. \\
    Die natürliche Projektion $\pi: R \rightarrow R/I$, $\pi(r) = r + I$
    ist ein Ringhomomorphismus.
\end{Def}

\begin{Lemma}{Kern von Ringhomomorphismen}
    Sei $f: R \rightarrow S$ ein Ringhomomorphismus. \\
    Dann ist $\ker f \trianglelefteq R$ und $f$ ist injektiv genau dann,
    wenn $\ker f = (0)$.
\end{Lemma}

\begin{Satz}{Isomorphiesätze für Ringe}
    \begin{enumerate}
        \item
        Seien $f: R \rightarrow S$ ein Ringhomomorphismus
        und $I \trianglelefteq R$ ein Ideal mit $I \subseteq \ker f$.
        Dann gibt es genau einen Ringhomomorphismus $\widetilde{f}$, sodass
        $f = \widetilde{f} \circ \pi$.
        Es gilt $\widetilde{f}: R/I \rightarrow S$,
        $\widetilde{f}(r + I) = f(r)$.
        Mit $I = \ker f$ gilt insbesondere, dass $R/\ker f$ isomorph
        zu $\im f$ ist.

        \item
        Seien $R$ ein Ring und $I, J \trianglelefteq R$ zwei Ideale.
        Dann sind $I \cap J$ und \\
        $I + J = \{i + j \;|\; i \in I\;, j \in J\}$
        ebenfalls Ideale von $R$ und es gilt
        $I/(I \cap J) \cong (I + J)/J$.

        \item
        Seien $R$ ein Ring und $I, J, K \trianglelefteq R$ drei Ideale
        mit $K \subseteq J \subseteq I$. \\
        Dann ist $I/J \cong (I/K)/(J/K)$.
    \end{enumerate}
\end{Satz}

\begin{Bem}
    Jeder Kern eines Ringhomomorphismus ist ein Ideal.
    Jedes Ideal $I \trianglelefteq R$ ist Kern eines Ringhomomorphismus,
    nämlich der von $\pi: R \rightarrow R/I$.
    Also sind Ideale genau die Kerne von Ringhomomorphismen.
\end{Bem}

\pagebreak

\begin{Def}{Verschwindungsideal}
    Sei $V$ ein endlich-dimensionaler Vektorraum und $f \in \End_K(V)$.
    Dann ist
    $\mathcal{I}_f = \{p(t) \in K[t] \;|\; p(f) \equiv 0\}$
    ein Ideal von $K[t]$ und wird \begriff{Verschwindungsideal} genannt.
\end{Def}

\begin{Satz}{Polynomdivision}
    Seien $h, g \in K[t]$ Polynome mit $\deg g \le \deg h$.
    Dann gibt es Polynome $q, r \in K[t]$ mit $\deg r < \deg g$, sodass
    $h = gq + r$ ist.
    Das Polynom $r$ ist der Rest bei der Polynomdivision.
\end{Satz}

\begin{Def}{normiert}
    Ein Polynom $g(t) \in K[t]$ heißt \begriff{normiert}, falls der
    \begriff{führende Koef"|fizient}
    (also der nicht-verschwindende Koef"|fizient
    bei der höchsten Potenz) gleich $1$ ist.
\end{Def}

\begin{Satz}{Ideale des Polynomrings}
    Seien $I \trianglelefteq K[t]$ ein Ideal mit $I \not= (0)$ und
    $p \in I$ ein nicht-triviales Polynom minimalen Grades in $I$.
    Dann ist $I = p K[t]$ und es gilt
    $I = r K[t] \;\Leftrightarrow\; r = \beta p$, wenn $r \in K[t]$ und
    $\beta \in K$ mit $\beta \not= 0$ ist.
\end{Satz}

\begin{Def}{Erzeuger, Hauptideal}
    Es gibt genau ein normiertes Polynom $q \in I$, sodass $I = q K[t]$ ist.
    $q$ heißt \begriff{normierter Erzeuger} von $I$.
    Ideale, die von einem Element erzeugt werden, heißen \begriff{Hauptideale}.
\end{Def}

\begin{Bem}
    Der Satz sagt also aus, dass alle Ideale von $K[t]$ Hauptideale sind.
\end{Bem}

\begin{Def}{Minimalpolynom}
    Sei $f \in \End_K(V)$.
    Das eindeutig bestimmte normierte Polynom kleinsten Grades
    in $\mathcal{I}_f$ heißt \begriff{Minimalpolynom} von $f$ und wird
    mit $\mu_f(t)$ bezeichnet. \\
    Analog ist das Minimalpolynom $\mu_A(t)$ einer Matrix $A \in M_n(K)$
    definiert.
\end{Def}

\begin{Kor}
    Sei $p \in K[t]$ ein Polynom mit $p(f) \equiv 0$.
    Dann gibt es $q \in K[t]$, sodass \\
    $p(t) = q(t) \cdot \mu_f(t)$ ist,
    d.\,h. das Minimalpolynom $\mu_f(t)$ teilt $p$.
    Insbesondere teilt das Minimalpolynom das charakteristische Polynom von
    $f$.
\end{Kor}

\begin{Satz}{Minimalpolynome ähnlicher Matrizen gleich} \\
    Die Minimalpolynome ähnlicher Matrizen stimmen überein. \\
    Analog: Konjugierte Endomorphismen $f, g \in \End_K(V)$
    (d.\,h. $f = h^{-1} g h$ für ein $h \in \Aut_K(V)$)
    haben dasselbe Minimalpolynom.
\end{Satz}

\begin{Satz}{$\chi_f(t)$ und $\mu_f(t)$ haben dieselben Nullstellen}
    Sei $f \in \End_K(V)$.
    Dann ist $\lambda \in K$ eine Nullstelle von $\mu_f(t)$ genau dann,
    wenn er Eigenwert von $f$ ist.
    Also haben $\chi_f(t)$ und $\mu_f(t)$ dieselben Nullstellen.
\end{Satz}

\begin{Satz}{Minimalpolynome teilen sich}
    Seien $f \in \End_K(V)$, wobei $\chi_f(t)$ in Linearfaktoren zerfalle,
    $V = V_1 \oplus \dotsb \oplus V_k$ eine Zerlegung in $f$-invariante
    Unterräume $V_i$ sowie $\mu_i$ das Minimalpolynom von der
    Einschränkung $f_i$ von $f$ auf $V_i$ für $i = 1, \dotsc, k$. \\
    Dann teilt $\mu_f(t)$ das Polynom $\prod_{i=1}^k \mu_i(t)$ und jedes
    $\mu_i(t)$ teilt $\mu_f(t)$. \\
    Insbesondere gilt $\mu_f(t) = \prod_{i=1}^k \mu_i(t)$, falls die
    $\mu_i(t)$ paarweise teilerfremd sind.
\end{Satz}

\begin{Kor} \\
    Sei $A = \diag\{J_1, \dotsc, J_k\}$ eine Blockdiagonalmatrix und
    $\chi_A(t)$ zerfalle in Linearfaktoren. \\
    Dann ist $\mu_A(t) = \prod_{i=1}^k \mu_{J_i}(t)$,
    falls die $\mu_{J_i}(t)$ paarweise teilerfremd sind.
\end{Kor}

\begin{Satz}{Minimalpolynom bestimmen}
    Sei $f \in \End_K(V)$ mit
    $\chi_f(t) = (t - \lambda_1)^{n_1} \dotsm (t - \lambda_k)^{n_k}$,
    wobei die $\lambda_i$ paarweise verschieden sind. \\
    Dann ist $\mu_f(t) = (t - \lambda_1)^{m_1} \dotsm (t - \lambda_k)^{m_k}$,
    wobei $m_i$ für $i = 1, \dotsc, k$
    die kleinste natürliche Zahl $s \in \natural$ mit
    $\ker(f - \ell_{\lambda_i})^s = \ker(f - \ell_{\lambda_i})^{s+1}$ ist
    (d.\,h. die Größe des größten Jordanblocks zum Eigenwert $\lambda_i$). \\
    Insbesondere ist $f$ diagonalisierbar genau dann, wenn
    $\mu_f(t) = (t - \lambda_1) \dotsm (t - \lambda_k)$ ist.
\end{Satz}

\pagebreak

\section{%
    Ringe und Moduln%
}

\subsection{%
    Kommutative Ringe und \texorpdfstring{$K$}{K}-Algebren: \emph{Setting the Stage}%
}

\begin{Bem}
    Mit der Jordanschen Normalform kann man zu einer Matrix eine ähnliche
    Matrix (Jordansche Normalform) bzw. zu einem Endomorphismus eines
    endlich-dimensionalen Vektorraums eine Basis finden, die sich besonders
    "`gutartig"' verhalten.
    Damit dies jedoch für alle Matrizen/Endomorphismen gilt, muss der
    Grundkörper algebraisch abgeschlossen sein, damit das charakteristische
    Polynom immer in Linearfaktoren zerfällt.
    Man sucht nun nach Alternativen, wenn der Körper nicht algebraisch
    abgeschlossen ist.
    Dafür muss man etwas weiter ausholen und die endlich-erzeugten Moduln
    über Hauptidealringen klassifizieren. \\
    Da $\integer$ ein Hauptidealring ist und die $\integer$-Moduln
    genau die abelsche Gruppen sind, bekommt man dabei als Nebenprodukt
    eine Klassifikation aller endlichen abelschen Gruppen.
\end{Bem}

\begin{Bem}
    Beim \begriff{Klassifikationsproblem} ist eine Struktur durch Axiome
    gegeben (z.\,B. Vektorräume, Moduln, Gruppen usw.).
    Außerdem gibt es strukturerhaltende Abbildungen\\
    (\begriff{Morphismen}),
    mit denen man die Objekte vergleichen kann. \\
    Bei der Klassifizierung aller Objekte der Kategorie muss man dann eine
    Liste von Objekten (\begriff{Prototypen}) angeben, sodass \qquad
    1. die Prototypen paarweise nicht isomorph sind und \\
    2. jedes Objekt der Kategorie isomorph zu einem Prototyp ist. \\
    Beim \begriff{Wiedererkennungsproblem} geht es darum, dass eine Kategorie
    durch eine Liste von Prototypen klassifiziert wurde und nun
    ein Objekt der Kategorie gegeben ist.
    Zu welchem Prototyp aus der Liste ist das Objekt dann isomorph?
\end{Bem}

\begin{Bem}
    Im Folgenden seien $K$ ein Körper und $R$ ein kommutativer Ring
    bzw. eine $K$-Algebra mit Einselement $1 = 1_R$.
\end{Bem}

\begin{Def}{Unterring}
    Sei $S \subseteq R$ mit $S \not= \emptyset$ nicht-leere Teilmenge von $R$.
    Dann ist $S$ ein \begriff{Unterring} von $R$, falls
    $r - s \in S$ und $rs \in S$ für alle $r, s \in S$ gilt.
\end{Def}

\begin{Bem}
    Die erste Bedingung sagt aus, dass $(S, +)$ eine abelsche
    Untergruppe von $(R, +)$ ist. \\
    Ist $1_R \in S$, so ist $1_R = 1_S$ das Einselement von $S$.
    Unterringe müssen jedoch nicht notwendigerweise dasselbe Einselement
    wie $R$ haben, sie müssen nicht einmal ein Einselement besitzen.
    Bspw. ist $2 \integer$ ein Unterring von $\integer$, der kein
    Einselement besitzt.
\end{Bem}

\begin{Def}{Ringhomomorphismus}
    Seien $R$ und $S$ Ringe sowie $f: R \rightarrow S$ eine Abbildung.
    $f$ heißt \begriff{Ringhomomorphismus}, falls
    $f(a + b) = f(a) + f(b)$ und $f(ab) = f(a) f(b)$ für alle $a, b \in R$. \\
    Ist $f(1_R) = 1_S$, so \begriff{erhält $f$ das Einselement}.
    $\ker f = \{r \in R \;|\; f(r) = 0_S\}$ heißt \begriff{Kern}
    und $\im f = \{f(r) \;|\; r \in R\}$ heißt \begriff{Bild} von $f$. \\
    Mono-, Epi- und Isomorphismen sind analog zu Vektorräumen definiert.
\end{Def}

\begin{Lemma}{Kern und Bild}
    Sei $f: R \rightarrow S$ Ringhomomorphismus.
    Dann ist $\ker f$ ein Unterring von $R$ und $\im f$ ein Unterring von $S$.
    Ist $r \in \ker f$ sowie $x \in R$, dann ist $rx = xr \in \ker f$.
\end{Lemma}

\begin{Def}{Ideal}
    Ein Unterring $S$ von $R$ heißt \begriff{Ideal} von R, falls
    $rs \in S$ für alle $s \in S, r \in R$.
\end{Def}

\begin{Def}{Faktorring}
    Sei $I \trianglelefteq R$.
    Dann ist die Menge $R/I = \{a + I \;|\; a \in R\}$ der Restklassen modulo
    $I$ eine abelsche Gruppe bzgl. der Addition
    $(a + I) + (b + I) = (a + b) + I$ mit Nullelement $0 + I$.
    Durch $(a + I)(b + I) = ab + I$ für $a, b \in R$ ist eine Multiplikation
    auf $R/I$ definiert, die $R/I$ zum Ring macht
    (Einselement $1 + I$).
    $R/I$ heißt daher \begriff{Faktorring} von $R$ modulo $I$.
\end{Def}

\begin{Lemma}{kanonische Projektion}
    Sei $I \trianglelefteq R$.
    Dann ist die Abbildung $\pi: R \rightarrow R/I$, $\pi(a) = a + I$
    ein Ringepimorphismus, die sog. \begriff{kanonische Projektion}
    von $R$ auf $R/I$.
    Es gilt $\ker \pi = I$, d.\,h. jedes Ideal von $R$ kommt als Kern
    eines Ringhomomorphismus vor.
\end{Lemma}

\begin{Bem}
    $(0)$ und $R$ sind Ideale von $R$.
    Alle anderen Ideale heißen \begriff{nicht-trivial/echt}. \\
    Sei $f: R \rightarrow S$ Ringhomomorphismus, dann ist $f$ surjektiv
    genau dann, wenn $\im f = S$, und injektiv genau dann,
    wenn $\ker f = (0)$ ist. \\
    Sei $A \subseteq R$, $A \not= \emptyset$ eine nicht-leere Teilmenge
    von $R$.
    Dann ist das von $A$ \begriff{erzeugte Ideal}
    $\aufspann{A} = \bigcap_{I \trianglelefteq R,\; A \subseteq I} I$ das
    kleinste Ideal von $R$, das $A$ als Teilmenge enthält. \\
    Es gilt: $\aufspann{A} = \{\sum_{a \in A} r_a a \;|\;
    r_a \in R \text{ fast alle } 0\}$. \\
    Der Durchschnitt von Idealen von $R$ ist ein Ideal von $R$. \\
    Seien $I, J \trianglelefteq R$ Ideale von $R$.
    Dann ist $I + J = \{a + b \;|\; a \in I,\; b \in J\}$ ein Ideal
    von $R$ (die Summe der Ideale $I$ und $J$), wobei
    $I + J = \aufspann{I \cup J}$ gilt. \\
    Die drei Isomorphiesätze gelten wie oben.
\end{Bem}

\begin{Satz}{Ideal ist maximal $\;\Leftrightarrow\;$ Faktorring ist Körper}
    Sei $I \trianglelefteq R$ ein Ideal von $R$.
    Dann ist $I$ maximal genau dann
    (d.\,h. $I \not= R$ und aus
    $I \subsetneqq J \trianglelefteq R$ folgt $J = R$),
    wenn $R/I$ ein Körper ist.
\end{Satz}

\begin{Def}{endlich erzeugt, Hauptideal, noethersch}
    Ein Ideal $I \trianglelefteq R$ heißt \begriff{endlich erzeugt},
    falls $I = \aufspann{S}$ für eine endliche Teilmenge $S \subseteq R$ ist.
    $S$ heißt dann \begriff{endliches Erzeugendensystem von I}.
    Besteht $S$ aus genau aus einem Element $s$, so heißt $I$ Hauptideal.
    In diesem Fall ist $I = sR = \{sr \;|\; r \in R\}$.
    Ein Ring, in dem alle Ideale endlich erzeugt sind, heißt
    \begriff{noethersch}.
\end{Def}

\begin{Satz}{äquivalente Bedinungen für noethersch} \\
    Sei $R$ ein Ring.
    Dann sind folgende Bedingungen äquivalent: \\
    1. $R$ ist noethersch. \\
    2. Jede aufsteigende Kette $I_1 \subseteq I_2 \subseteq \dotsb$ von Idealen
    von $R$ wird stationär, d.\,h. es gibt ein $n \in \natural$ mit $I_k = I_n$
    für alle $k \ge n$. \\
    3. Jede nicht-leere Menge von Idealen von $R$ besitzt maximale
    Elemente bzgl. der Inklusion.
\end{Satz}

\begin{Def}{Produkt zweier Ideale}
    Seien $I, J \trianglelefteq R$ zwei Ideale.
    Das Produkt $I \cdot J$ ist das Ideal, das von der Menge
    $\{i \cdot j \;|\; i \in I,\; j \in J\}$ erzeugt wird.
    Es gilt $I \cdot J \subseteq I \cap J$.
\end{Def}

\begin{Def}{invertierbar, Einheit}
    Ein Element $a \in R$ heißt \begriff{invertierbar} oder \begriff{Einheit},
    falls es ein $b \in R$ mit $ab = 1$ gibt.
    Das Inverse $b = a^{-1} \in R$ ist dann eindeutig bestimmt und selbst
    invertierbar.
    Die Menge $U(R)$ der invertierbaren Elemente von $R$ bildet unter
    Multiplikation eine Gruppe, die \begriff{Einheitengruppe} von $R$.
\end{Def}

\begin{Def}{Polynomring}
    Sei $R$ ein kommutativer Ring mit Eins.
    Dann besteht der \begriff{Polynomring $R[x]$} aus formalen Summen
    $\sum_{i=0}^n \alpha_i x^i$ mit $n \in \natural_0$, $\alpha_i \in R$ und
    $x$ Unbekannte.
    Ist $p(x) = \sum_{i=0}^n \alpha_i x^i$ und $\alpha_k \not= 0$,
    aber $\alpha_m = 0$ für alle $m > k$, so heißt $k = \deg p(x)$ der
    \begriff{Grad} von $p(x)$. \\
    Die Addition und Multiplikation von zwei Polynomen ist wie
    die Multiplikation von Polynomen mit einem Skalar $\lambda \in R$
    wie üblich definiert
    (es gilt nicht mehr notwendigerweise
    $\deg(p(x)q(x)) = \deg p(x) + \deg q(x)$). \\
    Der \begriff{Polynomring $R[x_1, \dotsc, x_n]$} in den Unbestimmten
    $x_1, \dotsc, x_n$ ist induktiv durch \\
    $R[x_1, \dotsc, x_n] = (R[x_1, \dotsc, x_{n-1}])[x_n]$ definiert. \\
    Er besteht aus formalen Summen
    $\sum_{\mi{i} = (i_1, \dotsc, i_n) \in \natural^n}
    \alpha_{\mi{i}} x_1^{i_1} \dotsm x_n^{i_n}$
    mit $\alpha_{\mi{i}} \in R$ gleich $0$ für fast alle
    $\mi{i} \in \natural^n$.
    Terme der Form $x^{\mi{i}} = x_1^{i_1} \dotsm x_n^{i_n}$
    mit $\mi{i} = (i_1, \dotsc, i_n) \in \natural^n$ heißen
    \begriff{Monome}.
\end{Def}

\begin{Satz}{universelle Eigenschaft von $K[x_1, \dotsc, x_n]$}
    Seien $K$ ein Körper und
    $K[x_1, \dotsc, x_n]$ der Polynomring über $K$.
    Dann hat $K[x_1, \dotsc, x_n]$ folgende universelle Eigenschaft: \\
    Es gibt eine Abbildung
    $\iota: \{1, \dotsc, n\} \rightarrow K[x_1, \dotsc, x_n]$,
    nämlich die Abbildung gegeben durch $\iota(i) = x_i$. \\
    Sei $R$ eine kommutative $K$-Algebra mit Einselement und
    $f: \{1, \dotsc, n\} \rightarrow R$ eine Abbildung.
    Dann gibt es genau einen $K$-Algebrahomomorphismus
    $\widehat{f}: K[x_1, \dotsc, x_n] \rightarrow R$ mit
    $\widehat{f}(x_i) = f(i)$ für $i = 1, \dotsc, n$,
    d.\,h. $\widehat{f} \circ \iota = f$.
\end{Satz}

\begin{Bem}
    Sind also $s_1, \dotsc, s_n \in R$ beliebig, so kann man
    die Abbildung $x_i \mapsto s_i$ zu einem $K$-Algebrahomo\-morphismus
    $\sum_{\mi{i}} \alpha_{\mi{i}} x^{\mi{i}} \mapsto
    \sum_{\mi{i}} \alpha_{\mi{i}} s^{\mi{i}}$ fortsetzen.
\end{Bem}

\subsection{%
    Hauptidealringe (HIR)%
}

\begin{Bem}
    Im Folgenden sei $R$ ein kommutativer Ring oder $K$-Algebra mit Eins.
\end{Bem}

\begin{Def}{Nullteiler, Integritätsbereich}
    Ein Element $a \in R$ heißt \begriff{Nullteiler}, falls
    es ein $b \in R$, $b \not= 0$ gibt, sodass $ab = 0$ ist. \\
    Besitzt $R$ außer $0$ keinen Nullteiler, so heißt
    $R$ \begriff{Integritätsbereich} oder \begriff{nullteilerfrei}.
\end{Def}

\begin{Def}{Quotientenkörper}
    Sei $R$ ein Integritätsbereich. \\
    Auf der Menge $\{(a, b) \in R \times R \;|\; b \not= 0\}$ definiert
    man eine Äquivalenzrelation durch \\
    $(a, b) \sim (c, d) \;\Leftrightarrow\; ad = bc$.
    Die Äquivalenzklasse von $(a, b)$ wird mit $\frac{a}{b}$
    bezeichnet. \\
    Auf der obigen Menge kann man mit $a, b, c, d \in R$, $b, d \not= 0$
    eine Addition und Multiplikation definieren durch
    $\frac{a}{b} + \frac{c}{d} = \frac{ad + bc}{bd}$ und
    $\frac{a}{b} \cdot \frac{c}{d} = \frac{ac}{bd}$. \\
    Damit wird $K = \{\frac{a}{b} \;|\; a, b \in R,\; b \not= 0\}$ ein
    Körper, der sog. \begriff{Quotientenkörper} $Q(R)$ von $R$. \\
    Die Abbildung $R \rightarrow K$, $r \mapsto \frac{r}{1}$ ist ein
    injektiver Ringhomomorphismus, sodass man $R$ als Unterring des Körpers
    $K = Q(R)$ betrachten kann.
\end{Def}

\begin{Def}{Hauptidealring}
    Ein Integritätsbereich $R$ heißt \begriff{Hauptidealring (HIR)}, falls
    jedes Ideal von $R$ ein Hauptideal ist.
\end{Def}

\begin{Def}{\textsc{Euklid}ische Ringe}
    Ein Integritätsbereich $R$ heißt \begriff{\textsc{Euklid}ischer Ring},
    falls es eine Abbildung (Gradfunktion)
    $\deg: R \rightarrow \natural_0 \cup \{-1\}$ gibt, sodass \\
    1. für alle $r \in R$ mit $r \not= 0$ gilt, dass $\deg 0 < \deg r$, und \\
    2. für alle $f, g \in R$ mit $g \not= 0$ es $q, r \in R$ mit
    $\deg r < \deg g$ gibt, sodass $f = q \cdot g + r$ ist.
\end{Def}

\begin{Bem}
    Beispiele für Euklidische Ringe sind $\integer$ mit $\deg z = |z|$
    sowie $K[x]$.
\end{Bem}

\begin{Satz}{ERs sind HIRs}
    Euklidische Ringe sind Hauptidealringe.
\end{Satz}

\begin{Kor}
    Also sind auch $\integer$ und $K[x]$ Hauptidealringe.
\end{Kor}

\begin{Def}{Teilbarkeit}
    Seien $R$ ein Integritätsbereich und $a, b \in R$. \\
    Dann \begriff{teilt} $a$ $b$, d.\,h. $a|b$, falls es ein $c \in R$ mit
    $b = ca$ gibt.
    Es gilt $a|b \;\Leftrightarrow bR \subseteq aR$.
\end{Def}

\begin{Def}{assoziiert}
    Seien $R$ ein Integritätsbereich und $a, b \in R$. \\
    Dann heißen $a$ und $b$ \begriff{assoziiert}, falls es eine Einheit
    $u \in U(R)$ gibt mit $a = bu$.
\end{Def}

\begin{Lemma}{assoziiert}
    Sei $R$ ein Integritätsbereich.
    Dann ist "`assoziiert sein"' eine Äquivalenzrelation
    und $a, b \in R$ sind assoziiert $\;\Leftrightarrow\; aR = bR
    \;\Leftrightarrow\; a|b \;\land\; b|a$.
\end{Lemma}

\begin{Def}{ggT und kgV}
    Seien $R$ ein Integritätsbereich und $a, b \in R$. \\
    $c \in R$ heißt \begriff{größter gemeinsamer Teiler} von $a$ und $b$,
    falls $c|a$ und $c|b$ sowie
    für $d \in R$ mit $d|a$ und $d|b$ auch $d|c$ gilt.
    Der größte gemeinsame Teiler $\ggT(a, b)$ von $a$ und $b$ ist,
    falls er existiert, bis auf Assoziiertheit eindeutig bestimmt. \\
    $c \in R$ heißt \begriff{kleinstes gemeinsames Vielfaches} von $a$ und $b$,
    falls $a|c$ und $b|c$ sowie
    für $d \in R$ mit $a|d$ und $b|d$ auch $c|d$ gilt.
    Das kleinste gemeinsame Vielfache $\kgV(a, b)$ von $a$ und $b$ ist,
    falls es existiert, bis auf Assoziiertheit eindeutig bestimmt. \\
    Ist $R$ ein HIR, dann existieren $\ggT(a, b)$ und $\kgV(a, b)$ und es
    gilt \\
    $aR + bR = \ggT(a, b)R$, \qquad
    $aR \cap bR = \kgV(a, b)R$ \quad sowie \quad
    $(aR) \cdot (bR) = abR$.
\end{Def}

\begin{Bem}
    Teilbarkeit ist eine Ordnungsrelation auf der Menge der Assoziiertenklassen
    von $R$, nicht auf $R$ selbst.
\end{Bem}

\pagebreak

\begin{Def}{Primideal}
    Seien $R$ ein kommutativer Ring mit Eins und $P \trianglelefteq R$. \\
    Dann heißt $P$ \begriff{Primideal}, falls
    für alle $x, y \in R$ mit $xy \in P$ gilt, dass $x \in P$ oder $y \in P$
    ist.
\end{Def}

\begin{Satz}{Primideale}
    $R$ ist ein Integritätsbereich genau dann, wenn $(0)$ ein Primideal ist. \\
    Sei $P \trianglelefteq R$.
    Dann ist $P$ ein Primideal genau dann, wenn $R/P$ ein Integritätsbereich
    ist.
\end{Satz}

\begin{Kor}
    Sei $M$ ein maximales Ideal von $R$.
    Dann ist $M$ ein Primideal.
\end{Kor}

\begin{Def}{irreduzibel, Primelement}
    Seien $R$ ein kommutativer Ring mit Eins und $a \in R$ mit $a \not= 0$. \\
    $a \not= 0$ heißt \begriff{irreduzibel}, falls $a$ eine Nicht-Einheit und
    nicht als Produkt zweier Nicht-Einheiten darstellbar ist,
    d.\,h. $a \notin U(R)$ sowie für alle $x, y \in R$ mit
    $a = xy$ gilt $x \in U(R)$ oder $y \in U(R)$. \\
    $a \not= 0$ heißt \begriff{Primelement}, falls $aR$ ein Primideal ist,
    d.\,h. aus $a|xy$ folgt $a|x$ oder $a|y$.
\end{Def}

\begin{Satz}{in Integritätsbereichen sind Primelemente irreduzibel} \\
    Seien $R$ ein Integritätsbereich und $p \in R$ Primelement.
    Dann ist $p$ irreduzibel.
\end{Satz}

\begin{Def}{Zerlegung in irreduzible Faktoren}
    Seien $R$ ein kommutativer Ring mit Eins und $a \in R$ mit $a \not= 0$. \\
    Dann besitzt $a \not= 0$ eine \begriff{Zerlegung in irreduzible Faktoren},
    falls $a = \varepsilon \pi_1 \dotsm \pi_r$ mit $\varepsilon \in U(R)$,
    $\pi_i \in R$ irreduzible Elemente und
    $r \in \natural_0$. \\
    $a \not= 0$ besitzt eine
    \begriff{eindeutige Zerlegung in irreduzible Faktoren},
    falls zusätzlich gilt: \\
    Ist $a = \varepsilon' \pi_1' \dotsm \pi_s'$
    mit $\varepsilon' \in U(R)$, $\pi_i' \in R$ irreduzibel und
    $s \in \natural_0$ eine weitere solche Zerlegung, dann ist
    $s = r$ und nach Umnummerierung sind die $\pi_i$ und $\pi_i'$ assoziiert
    ($i = 1, \dotsc, r$), d.\,h. es gibt
    $\varepsilon_1, \dotsc, \varepsilon_r \in U(R)$ mit
    $\pi_i' = \varepsilon_i \pi_i$ für $i = 1, \dotsc, r$.
\end{Def}

\begin{Def}{faktoriell}
    Ein Integritätsbereich heißt \begriff{faktoriell (UFD,
    \emph{unique factorisation domain})}, falls jedes Element von $R$ ungleich
    $0$ eine eindeutige Zerlegung in irreduzible Faktoren besitzt.
\end{Def}

\begin{Satz}{in UFDs stimmen irreduzible und Primelemente überein} \\
    Sei $R$ faktoriell und $p \in R$ irreduzibel. \\
    Dann ist $p$ ein Primelement, d.\,h. für UFDs stimmen irreduzible und
    Primelemente überein.
\end{Satz}

\begin{Satz}{Kriterium für UFD}
    Sei $R$ ein Integritätsbereich.
    Dann ist $R$ UFD genau dann, wenn \\
    1. jede aufsteigende Kette von Hauptidealen stationär wird und \\
    2. jedes irreduzible Element von $R$ Primelement ist.
\end{Satz}

\begin{Satz}{in HIRs sind irreduzible Elemente Primelemente}
    Sei $R$ ein Hauptidealring. \\
    Dann ist jedes irreduzible Element von $R$ ein Primelement.
\end{Satz}

\begin{Satz}{HIRs sind UFDs}
    Hauptidealringe sind UFDs.
\end{Satz}

\begin{Satz}{Primideale von HIRs sind maximal}
    Sei $R$ ein Hauptidealring.
    Dann ist jedes Primideal $P \not= (0)$ von $R$ maximal und daher ist
    $R/P$ ein Körper.
\end{Satz}

\pagebreak

\subsection{%
    Moduln%
}

\begin{Def}{Modul}
    Sei $A$ ein Ring mit Einselement oder eine $K$-Algebra mit einem Körper
    $K$.
    Ein \begriff{$A$-Linksmodul} ist eine abelsche Gruppe $(M, +)$ zusammen mit
    einer äußeren binären Operation $A \times M \rightarrow M$,
    $(a, m) \mapsto am$, sodass \\
    M1) $1_A m = m$ \qquad\qquad\qquad\;\;
    M2) $a(bm) = (ab)m$ \\
    M3) $(a + b)m = am + bm$ \qquad
    M4) $a(m_1 + m_2) = am_1 + am_2$ \\
    für alle $a, b \in A$ und $m, m_1, m_2 \in M$ gilt. \\
    Analog wird ein \begriff{$A$-Rechtsmodul} definiert
    (Operation $M \times A \rightarrow M$, $(m, a) \mapsto ma$). \\
    Man kann auch Moduln für Ringe ohne Einselement betrachten oder
    Moduln, bei denen $1_A$ nicht notwendigerweise wie die Eins operiert,
    d.\,h. M1) entfällt.
    Will man betonen, dass M1) immer gilt, so spricht man von
    einem \begriff{unitalen Modul}. \\
    Im Folgenden ist ein $A$-Modul immer ein unitaler $A$-Linksmodul.
\end{Def}

\begin{Satz}{Linksmodul als Rechtsmodul und Vektorraum}
    Ist $R$ kommutativer Ring mit Eins und $M$ ein $R$-Linksmodul,
    so wird $M$ zum $R$-Rechtsmodul, indem man $M \times A \rightarrow M$,
    $(m, a) \mapsto am$ definiert.
    Bei nicht-kommutativen Ringen gilt dies i.\,A. nicht, da 
    dann M2) verletzt ist. \\
    Ist $A$ eine $K$-Algebra und $M$ ein $A$-Linksmodul,
    so wird $M$ zum $K$-Vektorraum mit \\
    $\lambda m = (\lambda \cdot 1_A) m$ für $\lambda \in K$, $m \in M$.
\end{Satz}

\begin{Satz}{abelsche Gruppe sind $\integer$-Moduln}
    Sei $(M, +)$ eine abelsche Gruppe. \\
    Dann wird $M$ zum $\integer$-Modul mit
    $z \cdot m = m + \dotsb + m$ ($z$-mal) für $z > 0$,
    $z \cdot m = -(m + \dotsb + m)$ ($-z$-mal) für $z < 0$
    und $z \cdot m = 0$ für $z = 0$.
    Umgekehrt ist jeder $\integer$-Modul eine abelsche Gruppe nach Definition.
    Macht man diese zu einem $\integer$-Modul, so erhält man die
    ursprüngliche $\integer$-Modulstruktur zurück.
    Daher sind die $\integer$-Moduln genau die abelschen Gruppen.
\end{Satz}

\begin{Def}{Darstellung}
    Homomorphismen $f: A \rightarrow \End(M,+)$ für Ringe
    und $f: A \rightarrow \End_K(M)$ für $K$-Algebren $A$ heißen
    \begriff{(lineare) Darstellungen} von $A$. \\
    Seien $A$ ein Ring mit Eins und $M$ ein $A$-Modul.
    Für $a \in A$ definiert man $f_a: M \rightarrow M$, $m \mapsto am$.
    Dann ist $f_a \in \End(M,+)$ und $F: A \rightarrow \End(M,+)$,
    $a \mapsto f_a$ ist ein Ringhomomorphismus, der die Eins enthält. 
    Ist $A$ eine $K$-Algebra, so ist $f_a \in \End_K(M)$ und $F$ ist
    $K$-Algebrahomomorphismus.
    $F$ heißt \begriff{die zum $A$-Modul $M$ gehörende Darstellung} von $A$. \\
    Darstellungen und Moduln sind völlig äquivalente Konzepte.
\end{Def}

\begin{Def}{trivialer Modul}
    Der Nullmodul $(0)$ ist ein $A$-Modul mit $A$-Operation
    $a \cdot 0 = 0$ für alle $a \in A$.
    Er heißt \begriff{trivialer Modul}.
\end{Def}

\begin{Def}{regulärer Modul}
    $A$ wird zum $A$-Modul ${}_A A$, wobei $a \in A$ auf $A$ durch
    die normale Linksmultiplikation operiert.
    Er heißt \begriff{regulärer Modul}.
\end{Def}

\begin{Kor}
    Jedes Linksideal und daher auch jedes Ideal von $A$ ist $A$-Modul.
\end{Kor}

\begin{Def}{Modulhomomorphismus}
    Seien $A$ ein Ring mit Eins und $M, N$ $A$-Moduln. \\
    Eine Abbildung $f: M \rightarrow N$ heißt
    \begriff{($A$-Modul-)Homomorphismus}, falls $f$ ein
    Homomorphismus der abelschen Gruppen $(M,+)$ und $(N,+)$ ist,
    der zusätzlich die $A$-Operation respektiert, d.\,h.
    $f(am) = af(m)$ für alle $a \in A$, $m \in M$. \\
    $\ker f = \{m \in M \;|\; f(m) = 0_N\}$ heißt \begriff{Kern},
    $\im f = \{f(m) \;|\; m \in M\}$ heißt \begriff{Bild} von $f$. \\
    Injektive/surjektive/bijektive Homomorphismen heißen
    Mono-/Epi-/Isomorphismen. \\
    $M$ und $N$ heißen \begriff{isomorph}
    ($M \cong N$), falls es einen Isomorphismus $f: M \rightarrow N$ gibt.
\end{Def}

\pagebreak

\begin{Bem}
    Seien $A$ ein Ring mit Eins und $M, N$ $A$-Moduln. \\
    \textbf{Untermodul}:
    Eine nicht-leere Teilmenge $U \subseteq M$, $U \not= \emptyset$
    heißt \begriff{Untermodul} von $M$, falls $(U,+)$ abelsche Untergruppe
    von $(M,+)$ ist und $a \cdot u \in U$ für alle $a \in A$, $u \in U$ ist.
    Man schreibt dann $U \ur M$. \\
    Die $A$-Untermoduln von ${}_A A$ sind genau die Linksideale von $A$. \\
    \textbf{Durchschnitt von Untermoduln}:
    Der Durchschnitt von Untermoduln von $M$ ist wieder ein Untermodul von $M$.
    Dabei handelt es sich um den größten Untermodul von $M$, der in allen
    geschnittenen Untermoduln enthalten ist. \\
    \textbf{Aufspann einer Teilmenge}:
    Sei $S \subseteq M$.
    Der von $S$ \begriff{erzeugte Untermodul} $U = \aufspann{S}$ ist definiert
    als $\bigcap_{U \ur M,\; U \supseteq S} U$, der eindeutig bestimmte,
    kleinste Untermodul von $M$, der $S$ als Teilmenge enthält.
    $S$ heißt \begriff{Erzeugendensystem} von $U$.
    $M$ heißt \begriff{endlich erzeugt}, falls es eine endliche Menge
    $S \subseteq M$ gibt mit $\aufspann{S} = M$.
    Es gilt $\aufspann{S} =
    \left.\left\{\sum_{s \in S} a_s s \;\right|\; a_s \in A
    \text{ fast alle } 0_A\right\}$. \\
    \textbf{Summe von Untermoduln}:
    Sei $U_i \ur M$ für $i \in I$. \\
    Die \begriff{Summe} $U = \sum_{i \in I} U_i$ ist definiert
    als $\aufspann{\bigcup_{i \in I} U_i}$.
    Es gilt $U = \left.\left\{\sum_{i \in I} u_i \;\right|\;
    u_i \in U_i \text{ fast alle } 0_A\right\}$. \\
    \textbf{Faktormodul}:
    Sei $U \ur M$.
    Man definiert eine Äquivalenzrelation auf $M$ mit
    $x \equiv y \mod U \;\Leftrightarrow\; x - y \in U$ für $x, y \in M$.
    Die Äquivalenzklasse von $x \in M$ ist dann die \begriff{Nebenklasse} \\
    $x + U = \{x + u \;|\; u \in U\}$.
    Auf der Menge der Äquivalenzklassen $M/U = \{x + U \;|\; x \in M\}$
    wird eine Addition $(x + U) + (y + U) = (x + y) + U$ sowie eine
    $A$-Operation durch $a(x + U) = ax + U$ definiert.
    Diese sind wohldefiniert und machen $M/U$ zu einem $A$-Modul,
    dem \begriff{Faktormodul}.
    Die Abbildung $\pi: M \rightarrow M/U$, $\pi(m) = m + U$ ist ein
    Epimorphismus (\begriff{Projektion} von $M$ auf $M/U$). \\
    \textbf{Modulhomomorphismus}:
    Sei $f: M \rightarrow N$ ein $A$-Homomorphismus.
    Dann ist $\ker f \ur M$ und $\im f \ur N$. \\
    Sei $f: M \rightarrow N$ ein Isomorphismus.
    Dann ist $f^{-1}: N \rightarrow M$ ebenfalls einer.
    Die Relation "`isomorph sein"' ist Äquivalenzrelation auf der Klasse
    der $A$-Moduln. \\
    \textbf{1. Isomorphiesatz}:
    Sei $f: M \rightarrow N$ eine $A$-lineare Abbildung und $U \ur M$ mit
    $U \subseteq \ker f$.
    Dann gibt es genau eine $A$-lineare Abbildung
    $\widetilde{f}: M/U \rightarrow N$ mit $\widetilde{f} \circ \pi = f$.
    Es gilt $\im \widetilde{f} = \im f$ und
    $\ker \widetilde{f} = (\ker f)/U$.
    $\widetilde{f}$ ist gegeben durch $\widetilde{f}(m + U) = f(m)$.
    Ist insbesondere $\ker f = U$, so ist $\widetilde{f}$ ein
    $A$-Modulisomorphismus von $M/(\ker f)$ auf $\im f$ und
    $M/(\ker f) \cong \im f$. \\
    \textbf{2. Isomorphiesatz}:
    Seien $U, V \ur M$.
    Dann ist $(U + V)/V \cong U/(U \cap V)$. \\
    \textbf{3. Isomorphiesatz}:
    Seien $U \ur V \ur M$.
    Dann ist $V/U \ur M/U$ und $(M/U)/(V/U) \cong (M/V)$. \\
    \textbf{Modul über $K$-Algebra als Vektorraum}:
    Ist $A$ eine $K$-Algebra, so ist $M$ ein $K$-Vektorraum mit
    $\lambda m = (\lambda \cdot 1_A) m$ für $\lambda \in K$, $m \in M$.
    Dabei sind Untermoduln von $M$ auch $K$-Unterräume und $A$-lineare
    Abbildungen zwischen $A$-Moduln sind auch $K$-linear. \\
    \textbf{direkte Summe}:
    Seien $M_i \ur M$ für $i \in I$.
    Die Summe $U = \sum_{i \in I} M_i$ heißt
    \begriff{(interne) direkte Summe} der $M_i$, falls
    $M_i \cap \sum_{j \in I,\; j \not= i} M_j = (0)$ für alle $i \in I$ ist.
    Dies gilt genau dann, wenn jedes $u \in U$ eindeutig
    als $u = \sum_{i \in I} x_i$ mit $x_i \in M_i$ fast alle $0$ dargestellt
    werden kann. \\
    Sind $M_i$ für $i \in I$ eine Menge von $A$-Moduln,
    so ist die \begriff{(äußere) direkte Summe} \\
    $\bigoplus_{i \in I} M_i = \{(x_i)_{i \in I} \;|\; x_i \in M_i
    \text{ fast alle } 0\}$
    mit Addition und $A$-Operation definiert durch
    $(x_i)_{i \in I} + (y_i)_{i \in I} = (x_i + y_i)_{i \in I}$ und
    $a (x_i)_{i \in I} = (ax_i)_{i \in I}$.
    Damit ist $\bigoplus_{i \in I} M_i$ ein $A$-Modul.
\end{Bem}

\begin{Def}{freier Modul}
    Ein $A$-Modul $M$ heißt \begriff{frei},
    falls er isomorph zu einer direkten Summe von Kopien des
    regulären $A$-Moduls ${}_A A$ ist.
\end{Def}

\begin{Def}{Basis}
    Sei $M$ ein $A$-Modul.
    Dann heißt eine Teilmenge $S \subseteq N$ \begriff{linear unabhängig},
    falls es nur die triviale Darstellung der $0$ gibt, d.\,h.
    aus $\sum_{s \in S} a_s s = 0$, $a_s \in A$ fast alle $0$ folgt,
    dass $a_s = 0$ für alle $s \in S$. \\
    Eine linear unabhängiges Erzeugendensystem von $N$ heißt
    \begriff{Basis} von $N$. \\
    $S$ ist eine Basis von $N$ genau dann, wenn
    sich jedes Element von $N$ eindeutig als Linearkombination
    $\sum_{s \in S} a_s s$, $a_s \in A$ fast alle $0$ darstellen
    lässt. \\
    In diesem Fall gilt dann $N = \bigoplus_{s \in S} A s$
    mit $A s = \{as \;|\; a \in A\}$.
\end{Def}

\pagebreak

\begin{Satz}{Modul ist frei $\;\Leftrightarrow\;$ Modul hat eine Basis}
    Sei $M$ ein $A$-Modul. \\
    Dann ist $M$ frei genau dann, wenn
    $M$ eine $A$-Basis besitzt.
\end{Satz}

\begin{Bem}
    Der Basissatz für Vektorräume sagt, dass alle $K$-Vektorräume
    für einen Körper $K$ frei sind.
    I.\,A. haben jedoch $A$-Moduln keine $A$-Basis.
    Ist $A$ eine $K$-Algebra, so ist ein $A$-Modul zugleich ein
    $K$-Vektorraum und muss daher eine $K$-Basis besitzen.
\end{Bem}

\begin{Def}{(kurze) exakte Folge}
    Seien $M_1, M_2, \dotsc, M_i, \dotsc$ $A$-Moduln und
    $\alpha_i: M_i \rightarrow M_{i+1}$ $A$-linear. \\
    $M_1 \xrightarrow{\alpha_1} M_2 \xrightarrow{\alpha_2} \dotsb
    \xrightarrow{\alpha_{i-1}} M_i \xrightarrow{\alpha_i} \dotsb$
    heißt \begriff{exakte Folge}, falls $\ker \alpha_{i+1} = \im \alpha_i$
    für alle $i \in \natural$ ist. \\
    Eine exakte Folge der Form $(0) \rightarrow N \xrightarrow{\alpha} M
    \xrightarrow{\beta} E \rightarrow (0)$
    heißt \begriff{kurze exakte Folge (keF)}.
\end{Def}

\begin{Bem}
    Es gibt genau einen $A$-Modulhomomorphismus $(0) \rightarrow N$
    und $E \rightarrow (0)$ (Nullabbildung). \\
    Die Folge $(0) \rightarrow N \xrightarrow{\alpha} M
    \xrightarrow{\beta} E \rightarrow (0)$
    ist exakt genau dann, wenn $\alpha$ injektiv, $\beta$ surjektiv
    sowie $\ker \beta = \im \alpha$ ist.
    In diesem Fall gilt nach dem 1. Isomorphiesatz $N/\im \alpha \cong E$. \\
    Ist $M$ ein $A$-Modul, $U \ur M$, so gibt es immer eine keF
    $(0) \rightarrow U \xrightarrow{\alpha} M
    \xrightarrow{\beta} M/U \rightarrow (0)$, wobei
    $\alpha$ die natürliche Einbettung von $U$ in $M$ und
    $\beta$ die natürliche Projektion von $M$ auf $M/U$ ist.
\end{Bem}

\begin{Satz}{Erzeugendensystem von epimorphen Bildern}
    Seien $M, N$ $A$-Moduln, $f: M \rightarrow N$ ein $A$-Epimorphismus und
    $S \subseteq M$ ein Erzeugendensystem für $M$.
    Dann wird $N$ von $f(S)$ erzeugt, d.\,h. insbesondere sind
    epimorphe Bilder von endlich erzeugten $A$-Moduln endlich erzeugt.
\end{Satz}

\begin{Satz}{$N, E$ endlich erzeugt $\Rightarrow M$ ebenfalls} \\
    Sei $(0) \rightarrow N \xrightarrow{\alpha} M
    \xrightarrow{\beta} E \rightarrow (0)$ keF von $A$-Moduln.
    Sind $N$ und $E$ endlich erzeugt, so auch $M$.
\end{Satz}

\begin{Satz}{$M$ als direkte Summe} \\
    Seien $(0) \rightarrow N \xrightarrow{\alpha} M
    \xrightarrow{\beta} E \rightarrow (0)$ keF von $A$-Moduln
    und $E$ freier $A$-Modul. \\
    Dann gibt es ein $U \ur M$ mit $U \cong E$ und $M = \im \alpha \oplus U$.
\end{Satz}

\begin{Satz}{Rang freier Moduln über noethersche Ringe ist wohldefiniert} \\
    Seien $R$ ein kommutativer, noetherscher Ring mit Eins und $M$ ein
    freier $R$-Modul. \\
    Sind $\{m_\alpha \;|\; \alpha \in \mathcal{A}\}$ und
    $\{v_\beta \;|\; \beta \in \mathcal{B}\}$ Basen von $M$ mit
    Indexmengen $\mathcal{A}$ und $\mathcal{B}$, so ist
    $|\mathcal{A}| = |\mathcal{B}|$.
\end{Satz}

\begin{Bem}
    Der Beweis des vorherigen Satzes funktioniert auch für Ringe $R$, die
    nicht kommutativ sind und kein Einselement haben, solange $R$ maximale
    Ideale besitzt. \\
    Hat $R$ ein Einselement, so kann man aus dem Zornschen Lemma die Existenz
    von maximalen Idealen folgern, d.\,h. auch hier ist der Rang eines freien
    $R$-Moduls wohldefiniert. \\
    Da Hauptidealringe noethersch sind, gilt der Satz insbesondere für HIRs
    (sogar ohne Zornsches Lemma).
\end{Bem}

\begin{Def}{Rang}
    Seien $R$ ein kommutativer noetherscher Ring mit Eins und $M$ ein
    freier $R$-Modul. \\
    Dann ist der \begriff{Rang} $\rg M$ definiert als Kardinalität einer
    Basis von $M$ (unabhängig von der Wahl der Basis).
\end{Def}

\begin{Lemma}{Annullator}
    Seien $A$ ein beliebiger Ring, $I \trianglelefteq A$ und $M$ ein
    $A$-Modul. \\
    Dann ist $I M$ ein $A$-Untermodul von $M$.
    Die Menge $\ann_A(M) = \{a \in A \;|\; \forall_{m \in M}\; am = 0\}$
    ist ein Ideal von $A$ und heißt \begriff{Annullator} von $M$ in $A$.
    Es gilt $I \subseteq \ann_A(M/IM)$.
    Ist $L \trianglelefteq A$ und $L \subseteq \ann_A(M)$, so ist
    $M$ ein $A/L$-Modul durch $(a + L)m = am$ für $a \in A$, $m \in M$. \\
    $M/IM$ ist $A/I$-Modul mit $A/I$-Operation
    $(a + I)(m + IM) = am + IM$.
\end{Lemma}

\begin{Satz}{freie Moduln über noethersche Ringe gleichen Rangs sind
             isomorph} \\
    Sei $R$ ein kommutativer, noetherscher Ring und seien $M$ und $N$
    freie $R$-Moduln mit $\rg M = \rg N$.
    Dann sind $M$ und $N$ isomorph.
    Für jede Kardinalität $\alpha$ gibt es daher einen bis auf Isomorphie
    eindeutigen freien $R$-Modul $\mathcal{F}_\alpha$ vom Rang $\alpha$,
    nämlich die direkte Summe von $\alpha$ vielen Kopien von ${}_R R$.
\end{Satz}

\pagebreak

\subsection{%
    \emph{Zusatz}: Projekt 12 (\texorpdfstring{$e$}{ℯ} hoch Matrix und
    lineare Dif"|ferentialgleichungen)%
}

\begin{Satz}{endlich-dim. normierte Vektorräume}
    Jeder endlich-dimensionale normierte Vektorraum ist vollständig.
    Zwei Normen auf einem endlich-dimensionalen Vektorraum sind äquivalent.
\end{Satz}

\begin{Def}{Algebranorm}
    Sei $\mathfrak{A}$ eine $K$-Algebra mit $K = \real$ oder $K = \complex$.
    Eine Vektorraum-Norm $\norm{\cdot}$ auf $\mathfrak{A}$ heißt Algebranorm,
    falls $\norm{AB} \le \norm{A} \cdot \norm{B}$ für alle
    $A, B \in \mathfrak{A}$ ist.
\end{Def}

\begin{Def}{$p$-Norm}
    Auf $M_n(K)$ ist mit $1 \le p \le \infty$ eine Norm definiert durch
    $\norm{A}_p = \left(\sum_{i,j=1}^n |\alpha_{ij}|^p\right)^{1/p}$
    für $A = (\alpha_{ij})_{ij} \in M_n(K)$.
    Für $1 \le p \le 2$ ist dies eine Algebranorm.
\end{Def}

\begin{Def}{$e$ hoch Matrix}
    Sei $S_k = \sum_{i=0}^k \frac{A^i}{i!}$ mit $A \in M_n(\complex)$.
    Dann existiert der Grenzwert der Folge $\{S_k\}_{k \in \natural}$ sowohl
    komponentenweise als auch bzgl. jeder Algebranorm auf $M_n(\complex)$. \\
    Der Grenzwert wird mit $e^A = \sum_{i=0}^\infty \frac{A^i}{i!}$
    bezeichnet.
\end{Def}

\begin{Satz}{Aussagen über $e$ hoch Matrix}
    Seien $A, B \in M_n(\complex)$ und $P \in \GL_n(\complex)$. \\
    Dann ist $P^{-1} e^A P = e^{P^{-1} A P}$, \qquad
    $e^A e^B = e^{A + B} = e^B e^A$ für $AB = BA$, \qquad
    $(e^A)^{-1} = e^{-A}$, \\
    $\det e^A = e^{\tr A}$ \quad und \quad
    $e^{\diag\{B_1, \dotsc, B_S\}} = \diag\{e^{B_1}, \dotsc, e^{B_s}\}$. \\
    Sind $\lambda_1, \dotsc, \lambda_n$ die Eigenwerte von $A$, so sind 
    $e^{\lambda_1}, \dotsc, e^{\lambda_n}$ die Eigenwerte von $e^A$.
\end{Satz}

\begin{Prozedur}{Berechnung von $e^A$}
    \begin{enumerate}
        \item
        Man bringt $A$ auf Jordanform, d.\,h. man bestimmt eine Matrix
        $P \in \GL_n(\complex)$ mit \\
        $P^{-1} A P = \diag\{J_1, \dotsc, J_s\}$,
        wobei $J_i$ ein Jordanblock ist.
        
        \item
        Es gilt nun $e^A = e^{P \diag\{J_1, \dotsc, J_s\} P^{-1}} =
        P e^{\diag\{J_1, \dotsc, J_s\}} P^{-1}$.
        
        \item
        Es ist $e^{\diag\{J_1, \dotsc, J_s\}} =
        \diag\{e^{J_1}, \dotsc, e^{J_s}\}$.
        
        \item
        Um $e^{J_i}$ zu berechnen, sei $J_i = J_\lambda(k)$ ein Jordanblock
        sowie $N = J_0(k)$. \\
        Dann ist $J_\lambda(k) = \lambda E + N$ sowie
        $\lambda E \cdot N = N \cdot \lambda E$. \\
        Es ist $e^{J_i} = e^{\lambda E + N} = e^{\lambda E} e^N$,
        da $\lambda E$ und $N$ kommutieren. \\
        Es gilt $e^{\lambda E} e^N = e^\lambda e^N$ sowie
        $e^N =$ \matrixsize{$\begin{pmatrix}
        1 & \frac{1}{1!} & \frac{1}{2!} & \frac{1}{3!} & \cdots &
        \frac{1}{(k - 1)!} \\
        0 & 1 & \frac{1}{1!} & \frac{1}{2!} & \cdots & \frac{1}{(k - 2)!} \\
        \vdots & \ddots & \ddots & \ddots & \ddots & \vdots \\
        0 & \cdots & 0 & 1 & \frac{1}{1!} & \frac{1}{2!} \\
        0 & \cdots & 0 & 0 & 1 & \frac{1}{1!} \\
        0 & \cdots & 0 & 0 & 0 & 1
        \end{pmatrix}$}.
        
        \item
        Also ist $e^A = P \diag\{e^{\lambda_1} e^{N_1}, \dotsc,
        e^{\lambda_s} e^{N_s}\} P^{-1}$.
    \end{enumerate}
\end{Prozedur}

\subsection{%
    \emph{Zusatz}: Projekt 13 (Beispiele von Ringen)%
}

\begin{Lemma}{Lemma von \name{Gauß}}
    Sei $R$ ein faktorieller Ring und $Q$ der Quotientenkörper von $R$.
    Außerdem sei $p \in R[x]$ ein Polynom, sodass die Koef"|fizienten in $R$
    den größten gemeinsamen Teiler $1$ haben. \\
    Ist $p = g \cdot h$ mit $g, h \in Q[x]$, so gibt es $g', h' \in R[x]$ mit
    $p = g'h'$ und $g'$ bzw. $h'$ unterscheiden sich von $g$ bzw. $h$ nur um
    ein Element aus $Q$.
\end{Lemma}

\begin{Satz}{Satz von \name{Gauß}}
    Sei $R$ ein faktorieller Ring.
    Dann ist $R[x]$ auch ein faktorieller Ring.
\end{Satz}

\pagebreak

\chapter{%
    Moduln über Hauptidealringen%
}

\section{%
    Torsionsmoduln%
}

\begin{Def}{Annullator}
    Sei $R$ ein kommutativer Ring mit Eins und $M$ ein $R$-Modul. \\
    Dann ist der \begriff{Annullator} $\ann_R(m)$ von $m \in M$ definiert durch
    $\ann_R(m) = \{r \in R \;|\; rm = 0\}$. \\
    Ähnlich ist für $S \subseteq M$ $\ann_R(S) =
    \{r \in R \;|\; \forall_{m \in S}\; rm = 0\} =
    \bigcap_{m \in S}\; \ann_R(m)$.
\end{Def}

\begin{Lemma}{Annullator ist Ideal}
    $\ann_R(m)$ und $\ann_R(S)$ sind Ideale von $R$.
\end{Lemma}

\begin{Lemma}{Annullator eines endlich erzeugten Moduls}
    Sei $M = \aufspann{m_1, \dotsc, m_k}$ endlich erzeugt.
    Dann ist $\ann_R(M) = \bigcap_{i=1}^k \ann_R(m_i)$.
\end{Lemma}

\begin{Def}{Torsionselement}
    Seien $R$ Integritätsbereich und $M$ ein $R$-Modul. \\
    Ein Element $m \in M$ heißt \begriff{Torsionselement}, falls
    $\ann_R(m) \not= 0$ ist, d.\,h. es gibt ein $r \in R$, $r \not= 0$
    mit $rm = 0$.
    Das Nullelement $0_M$ von $M$ ist immer ein Torsionselement.
    Ist es auch das einzige Torsionselement, so heißt $M$
    \begriff{torsionsfrei}.
\end{Def}

\begin{Def}{Torsionsmoduln und -untermoduln}
    Seien $R$ Integritätsbereich und $M$ ein $R$-Modul. \\
    Dann ist die Menge $T(M)$ der Torsionselemente von $M$ ein Untermodul
    von $M$, der \\
    \begriff{Torsionsuntermodul} von $M$.
    Ist $T(M) = M$, so heißt $M$ \begriff{Torsionsmodul}.
\end{Def}

\begin{Bem}
    Beispielsweise ist ${}_\integer \integer$ ein torsionsfreier Modul
    und $\integer/z\integer$ ist Torsionsmodul.
\end{Bem}

\begin{Satz}{Torsionsmoduln und torsionsfreie Moduln}
    Sei $R$ ein Integritätsbereich. \\
    Ist $M$ ein freier $R$-Modul, dann ist $M$ torsionsfrei. \\
    Ist $M$ ein $R$-Modul, dann ist $M/T(M)$ torsionsfrei. \\
    Epimorphe Bilder von Torsionsmoduln sind Torsionsmoduln. \\
    Sei $\{M_\alpha \;|\; \alpha \in \mathcal{A}\}$ eine Menge von $R$-Moduln.
    Dann ist $T\left(\bigoplus_{\alpha \in \mathcal{A}} M_\alpha\right) =
    \bigoplus_{\alpha \in \mathcal{A}} T(M_\alpha)$.
    Sind insbesondere die $M_\alpha$ Torsionsmoduln bzw. torsionsfrei,
    so ist auch ihre direkte Summe Torsionsmodul bzw. torsionsfrei. \\
    Untermoduln von Torsionsmoduln sind Torsionsmoduln. \\
    Untermoduln von torsionsfreien Moduln sind torsionsfrei.
\end{Satz}

\begin{Def}{zyklischer $R$-Modul}
    Seien $R$ ein kommutativer Ring mit Eins und $M$ ein $R$-Modul. \\
    $M$ heißt \begriff{zyklischer $R$-Modul}, falls
    $M$ einelementig erzeugt wird, d.\,h. $M = Rm$ für ein $m \in M$. \\
    In diesem Fall wird durch $f: {}_R R \rightarrow M$, $r \mapsto rm$
    ein $R$-Modulepimorphismus vom regulären $R$-Modul ${}_R R$ auf $M$
    definiert. \\
    Insbesondere ist $M$ isomorph zum Faktormodul $R/\ker f$ mit
    $\ker f \trianglelefteq R$. \\
    Umgekehrt ist $R/I$ zyklischer $R$-Modul ertezgt von der Nebenklasse
    $1 + I$, falls $I \trianglelefteq R$ ist.
\end{Def}

\begin{Lemma}{torsionsfreie, zyklische Moduln sind frei} \\
    Seien $R$ ein Integritätsbereich und
    $(0) \not= M = Rm$ ein torsionsfreier, zyklischer $R$-Modul. \\
    Dann ist $M \cong {}_R R$ frei mit Basis $\{m\}$.
\end{Lemma}

\begin{Satz}{Untermoduln von e.e. freien Moduln über HIR
             sind frei von kleinerem Rang} \\
    Seien $R$ ein Hauptidealring, $F$ ein endlich erzeugter,
    freier $R$-Modul mit $\rg F = n$ und $R$-Basis
    $\basis{B} = \{v_1, \dotsc, v_n\}$ sowie $M \ur F$.
    Dann ist $M$ ein freier $R$-Modul mit $\rg M = k \le n$.
\end{Satz}

\begin{Kor}
    Seien $R$ ein Hauptidealring und $M$ ein torsionsfreier, endlich erzeugter
    $R$-Modul mit Erzeugendensystem $S$ der Kardinalität $|S| = k$.
    Dann ist $M$ frei vom Rang $n \le k$.
\end{Kor}

\begin{Bem}
    Für Hauptidealringe $R$ sind also die torsionsfreien, endlich erzeugten
    $R$-Moduln genau die freien $R$-Moduln mit endlichem Rang.
    Für andere Integritätsbereiche ist dies i.\,A. falsch. \\
    Obige Folgerung besagt nicht, dass Erzeugendensysteme freier $R$-Moduln
    eine Basis enthalten.
    (Beispielsweise wird der freie $\integer$-Modul ${}_\integer \integer$
    von $\{2, 3\}$ erzeugt, $\{2, 3\}$ enthält aber keine Basis.)
\end{Bem}

\begin{Satz}{e.e. Modul über HIR als Summe von Torsionsmodul und
             freiem Modul} \\
    Seien $R$ ein Hauptidealring und $M$ ein endlich erzeugter $R$-Modul. \\
    Dann ist $M = T(M) \oplus U$ mit $U \ur M$ freier $R$-Modul von
    endlichem Rang mit $U \cong M/T(M)$.
\end{Satz}

Das Ziel, alle endlich erzeugten Moduln über Hauptidealringen $R$ zu
klassifizieren, kann man nun darauf reduzieren, alle endlich erzeugten
$R$-Torsionsmoduln zu klassifizieren. \\
Hat man nämlich eine Liste $\{M_\alpha \;|\; \alpha \in \mathcal{A}\}$ aller
paarweise nicht-isomorphen, endlich erzeugten $R$-Torsionsmoduln, bekommt man
eine aller paarweise nicht-isomorphen, endlich erzeugten $R$-Moduln als
$\{M_{\alpha,k} \;|\; \alpha \in \mathcal{A},\; k \in \natural_0\}$ mit
$M_{\alpha,k} = M_\alpha \oplus
R \oplus \overset{k\text{-mal}}{\dotsb} \oplus R$. \\
Nun will man eine Liste $\{M_\alpha \;|\; \alpha \in \mathcal{A}\}$
konstruieren.

\section{%
    Primärkomponenten%
}

\begin{Bem}
    Im Folgenden sei $R$ immer ein Hauptidealring und $M$ ein endlich erzeugter
    $R$-Modul.
\end{Bem}

\begin{Def}{$M_p$, Primärkomponente}
    Sei $p \in R$.
    Dann ist \begriff{$M_p$} der Untermodul \\
    $M_p = \{m \in M \;|\; \exists_{k \in \natural}\; p^k m = 0\}$ von $M$. \\
    Ist $p \not= 0$ ein Primelement, so heißt $M_p$ \begriff{Primärkomponente}.
\end{Def}

\begin{Lemma}{direkte Summe von $M_p$ und $M_q$}
    Seien $p, q \in R$, $p, q \not= 0$ mit $\ggT(p, q) = 1$. \\
    Dann ist $M_p \cap M_q = (0)$ und daher ist ihre Summe $M_p + M_q$ direkt.
\end{Lemma}

\begin{Def}{Ordnung}
    Seien $R$ ein Hauptidealring und $M$ ein endlich erzeugter
    $R$-Torsionsmodul. \\
    Dann ist der Annullator $\ann_R(M)$ ist nicht-trivial und wird von einem
    bis auf Einheiten eindeutig bestimmten $r \in R$ erzeugt, d.\,h.
    $\ann_R(M) = rR \not= (0)$. \\
    Ein Erzeuger von $\ann_R(M)$ wird \begriff{Ordnung} von $M$
    genannt und mit $r = \ordnung(M)$ bezeichnet.
\end{Def}

\begin{Satz}{Primärkomponentenzerlegung} \\
    Seien $R$ ein Hauptidealring und $M$ ein e.e.
    $R$-Torsionsmodul. \\
    Ist $\ordnung(M) = r$ und $r = p_1^{k_1} \dotsm p_n^{k_n}$ die
    Primfaktorzerlegung von $r$ in paarweise nicht-assoziierte
    Primelemente $p_1, \dotsc, p_n \in R$, $k_1, \dotsc, k_n \in \natural$
    (möglich da $R$ UFD), \\
    so zerlegt sich $M$ in die direkte Summe
    $M = M_{p_1} \oplus \dotsb \oplus M_{p_n}$
    seiner (eindeutig bestimmten) Primärkomponenten $M_{p_i}$ für
    $i = 1, \dotsc, n$. \\
    Diese Zerlegung heißt \begriff{Primärkomponentenzerlegung} des
    e.e. Torsionsmoduls $M$.
\end{Satz}

\begin{Kor}
    Seien $M$ und $r = p_1^{k_1} \dotsm p_n^{k_n}$ wie eben.
    Dann ist $\ordnung(M_{p_i}) = p_i^{k_i}$.
\end{Kor}

\begin{Def}{Ordnung}
    Seien $M$ ein e.e. $R$-Torsionsmodul, $m \in M$ und $\ann_R(m) = rR$. \\
    Dann heißt $r$ die \begriff{Ordnung} von $m$, die mit $\ordnung(m)$
    bezeichnet wird.
\end{Def}

\begin{Bem}
    Ist nun ein beliebiger e.e. $R$-Modul $M$ gegeben ($R$ HIR), so kann
    man zunächst mit \\
    $M = T(M) \oplus U$ den torsionsfreien Teil $U$ von $M$ abspalten.
    Der freie $R$-Modul $U \cong M/T(M)$ ist auch e.e. und bis auf Isomorphie
    eindeutig bestimmt.
    Der Rang von $U$ ist eindeutig bestimmt und endlich. \\
    Der Torsionsmodul $T(M) \cong M/U$ ist ebenfalls e.e. und hat eine
    eindeutige Zerlegung in Primärkomponenten
    $T(M) = T(M)_{p_1} \oplus \dotsc \oplus T(M)_{p_n}$, wobei die paarweise
    verschiedenen Primelemente $p_i \in R$, $i = 1, \dotsc, n$ gerade die
    Primfaktoren der Ordnung $\ordnung(M)$ sind, die in der
    Primfaktorzerlegung von $\ordnung(M)$ vorkommen.
    Nun muss man also die Moduln $T(M)_{p_i}$ weiter zerlegen und bestimmen.
\end{Bem}

\pagebreak

\section{%
    Elementarteiler und Prototypen%
}

\begin{Def}{zyklischer Modul}
    Seien $R$ ein Ring mit Einselement und $M$ ein $R$-Modul. \\
    Dann ist $M$ ein \begriff{zyklischer $R$-Modul}, falls $M$ von einem
    Element erzeugt wird, d.\,h. \\
    $M = Rm = \{rm \;|\; r \in R\}$ für ein $m \in M$.
\end{Def}

\begin{Satz}{$M$ zyklisch $\Leftrightarrow M$ epimorphes Bild von ${}_R R$}
    $M$ ist zyklischer $R$-Modul genau dann, wenn $M$ epimorphes Bild des
    regulären $R$-Moduls ${}_R R$ ist. \\
    In diesem Fall (sei $M = Rm$) ist $M \cong R/\ann_R(m)$.
\end{Satz}

\begin{Kor} \\
    Seien $R$ ein HIR, $M$ ein zyklischer $R$-Torsionsmodul mit Erzeuger
    $m \in M$ sowie $r = \ordnung(m)$. \\
    Dann ist $M \cong R/rR$ als $R$-Modul und $\ordnung(M) = r$.
\end{Kor}

\begin{Def}{unabhängig}
    Seien $R$ ein Ring mit Eins und $M$ ein $R$-Modul. \\
    Dann heißen $y_1, \dotsc, y_m \in M$ \begriff{unabhängig}, falls aus
    $\lambda_1 y_1 + \dotsb + \lambda_m y_m = 0$ mit
    $\lambda_1, \dotsc, \lambda_m \in R$ stets $\lambda_i y_i = 0$ für
    alle $i = 1, \dotsc, m$ folgt.
\end{Def}

\begin{Bem}
    \emph{Vorsicht}:
    Lineare Unabhängigkeit fordert mehr wie Unabhängigkeit, d.\,h.
    aus linearer Unabhängigkeit folgt immer Unabhängigkeit.
    Die Umkehrung gilt \emph{nicht}.
\end{Bem}

\begin{Satz}{Erzeugendensystem unabhängig $\Leftrightarrow M$ zerfällt in
             direkte Summe}
    Seien $R$ ein Ring mit Eins, $M$ ein $R$-Modul und $\{y_1, \dotsc, y_m\}$
    ein unabhängiges Erzeugendensystem. \\
    Dann ist $M = Ry_1 \oplus \dotsb \oplus Ry_m$. \\
    Ist umgekehrt $M = Ry_1 \oplus \dotsb \oplus Ry_m$, so ist
    $\{y_1, \dotsc, y_m\}$ unabhängig.
\end{Satz}

\begin{Kor}
    Sei $R$ ein HIR, $M$ ein $R$-Modul, $\{y_1, \dotsc, y_m\}$
    ein unabhängiges Erzeugendensystem und $s_i = \ordnung(y_i)$.
    Dann ist $M = Ry_1 \oplus \dotsb \oplus Ry_m \cong
    R/Rs_1 \oplus \dotsb \oplus R/Rs_m$.
\end{Kor}

\begin{Bem}
    Nun muss für e.e. $R$-Torsionsmoduln $M$ ($R$ HIR) ein unabhängiges
    Erzeugendensystem gefunden werden.
\end{Bem}

\begin{Lemma}{in Nebenklassen gibt es Elemente gleicher Ordnung}
    Seien $R$ ein HIR und $M$ ein e.e. $R$-Torsionsmodul, dessen Ordnung
    $\ordnung(M) = p^k$ für ein Primelement $p \in R$, $k \in \natural$ ist
    (d.\,h. es gilt $M = M_p$).
    Seien außerdem $m \in M$ mit $\ordnung(m) = \ordnung(M) = p^k$ und
    $\nk{M} = M/Rm$.
    Dann gibt es in jeder Nebenklasse $\nk{x} = x + Rm \in \nk{M}$ einen Vektor
    $y \in x + Rm$ mit $\ordnung(\nk{x}) = \ordnung(y)$.
\end{Lemma}

\begin{Lemma}{unabhängige Mengen}
    Seien $R$ ein HIR und $M$ ein e.e. $R$-Torsionsmodul mit \\
    $\ordnung(M) = p^k$ für ein Primelement $p \in R$, $k \in \natural$.
    Seien außerdem $m \in M$, sodass \\
    $\ordnung(m) = \ordnung(M) = p^k$ ist, und
    $y_1, \dotsc, y_n \in M$, sodass $\nk{y_i} = y_i + Rm \in M/Rm$
    unabhängig sind. \\
    Die Repräsentanten $y_i \in \nk{y_i}$ seien so gewählt, dass
    $\ordnung(\nk{y_i}) = \ordnung(y_i)$ ($i = 1, \dotsc, n$). \\
    Dann ist auch $\{m, y_1, \dotsc, y_n\} \subseteq M$ unabhängig.
\end{Lemma}

\begin{Satz}{Untermoduln des zyklischen Moduls}
    Sei $R$ ein HIR und $M = Rm$ ($m \in M$) ein zyklischer $R$-Modul mit
    $\ordnung(M) = p^k$ für ein Primelement $r \in R$, $k \in \natural$.
    Dann gilt: \\
    1. Für $\nu = 0, \dotsc, k$ sei $M_\nu = p^\nu M = Rp^\nu \cdot m$. \\
    Dann ist $M_\nu \in M$ und $\{M_\nu \;|\; \nu = 0, \dotsc, k\}$ ist genau
    die Menge der Untermoduln von $M$. \\
    2. $(0) = M_k \lneqq M_{k-1} \lneqq \dotsb \lneqq M_1 \lneqq M_0 = M$
    und $\ordnung(M_\nu) = p^{k - \nu}$ für $nu = 0, \dotsc, k$. \\
    $M_\nu$ ist zyklisch mit Erzeuger $p^\nu m$ der Ordnung $p^{k-\nu}$. \\
    3. Sei $x \in M$.
    Dann ist $M = Rx$ (d.\,h. $x$ Erzeuger von $M$) genau dann,
    wenn $x \notin M_1$ ist. \\
    4. Jedes Erzeugendensystem von $M$ enthält ein $x \notin M_1$ mit $M = Rx$.
\end{Satz}

\begin{Def}{minimales Erzeugendensystem}
    Seien $R$ ein Ring mit Einselement, $M$ ein $R$-Modul und
    $S \subseteq M$ Erzeugendensystem von $M$, d.\,h.
    $M = \aufspann{S} = \sum_{x \in S} Rx$. \\
    $S$ heißt \begriff{minimales Erzeugendensystem von $M$}, falls
    $\aufspann{T} \lneqq M$ für jede echte Teilmenge $T \subset S$.
\end{Def}

\pagebreak

\begin{Kor}
    Seien $R$ ein HIR, $M$ ein zyklischer $R$-Modul der Ordnung $p^k$ für
    ein Primelement $p \in R$, $k \in \natural$ sowie $S \subseteq M$
    minimales Erzeugendensystem von $M$. \\
    Dann ist $S = \{x\}$ für ein $x \in M$, aber $x \notin pM$.
\end{Kor}

\begin{Satz}{Modul zerfällt in Faktormoduln}
    Seien $R$ ein HIR, $M$ ein e.e. $R$-Torsionsmodul der Ordnung $p^k$ für
    ein Primelement $p \in R$, $k \in \natural$ sowie
    $S = \{m_1, \dotsc, m_n\} \subseteq M$ ein minimales
    Erzeugendensystem von $M$.
    Dann enthält jedes minimale Erzeugendensystem von $M$ exakt $n$ Elemente
    und es gibt eindeutig bestimmte natürliche Zahlen
    $k = e_1 \ge e_2 \ge \dotsb \ge e_n$, sodass
    $M \cong R/Rq_1 \oplus \dotsb \oplus R/Rq_n$ mit $q_i = p^{e_i}$,
    $i = 1, \dotsc, n$ ist.
    (Es gilt $q_n \;|\; \dotsb \;|\; q_2 \;|\; q_1 = p^k$.)
\end{Satz}

\begin{Satz}{Liste I von Prototypen} \\
    Seien $R$ ein HIR und $p_1, \dotsc, p_k \in R$ paarweise nicht-assoziierte
    Primelemente. \\
    Für $i = 1, \dotsc, k$ seien
    $e_1^{(i)} \ge e_2^{(i)} \ge \dotsb \ge e_{n_i}^{(i)} \ge 1$ natürliche
    Zahlen sowie $I_\nu^{(i)} = Rp^{e_\nu^{(i)}}$ für $\nu = 1, \dotsc, n_i$.
    Sei $\mi{e}_i := (e_1^{(i)}, \dotsc, e_{n_i}^{(i)})$ und
    $E(p_i, \mi{e}_i) := R/I_1^{(i)} \oplus \dotsb \oplus R/I_{n_i}^{(i)}$.
    Zusätzlich sei
    $M(p_1, \mi{e}_1, \dotsc, p_k, \mi{e}_k, \alpha) =
    E(p_1, \mi{e}_1) \oplus \dotsb \oplus E(p_k, \mi{e}_k) \oplus
    (R \oplus \overset{\alpha\text{-mal}}{\dotsb} \oplus R)$
    für $\alpha \in \natural_0$. \\
    %, es ist also
    %$M(p_1, \mi{e}_1, \dotsc, p_k, \mi{e}_k, \alpha) =
    %R/Rp^{e_1^{(1)}} \oplus \dotsb \oplus R/Rp^{e_{n_1}^{(1)}} \oplus \dotsb
    %\oplus R/Rp^{e_1^{(k)}} \oplus \dotsb \oplus R/Rp^{e_{n_k}^{(k)}} \oplus
    %(R \oplus \overset{\alpha\text{-mal}}{\dotsb} \oplus R)$. \\
    Dann ist
    $\{M(p_1, \mi{e}_1, \dotsc, p_k, \mi{e}_k, \alpha) \;|\;
    k \in \natural_0,\;
    p_1, \dotsc, p_k \in R \text{ Primelemente}$\\
    $\text{(bis auf Assoziierung)},
    \alpha \in \natural_0,\;
    n_i \in \natural \text{ und }
    \mi{e}_i = (e_1^{(i)}, \dotsc, e_{n_i}^{(i)}) \text{ mit }
    e_1^{(i)} \ge \dotsb \ge e_{n_i}^{(i)} \text{ für } i = 1, \dotsc, k\}$
    eine vollständige Liste von paarweise nicht-isomorphen, endlich erzeugten
    $R$-Moduln.
\end{Satz}

\begin{Bem}
    Nun ist zunächst das Klassifikationsproblem gelöst.
    Das Wiedererkennungsproblem ist damit noch nicht gelöst.
    Sei $M = R/Rr$ mit $r \in R$, $R$ HIR
    (d.\,h. $M$ ist zyklischer $R$-Modul der Ordnung $r$).
    Zu welchem der $R$-Moduln aus obiger Liste ist $M$ dann isomorph?
\end{Bem}

\begin{Satz}{Zerlegung von $R/Rr$ in teilerfremde Faktoren}
    Seien $R$ ein HIR sowie $r = s \cdot t$ eine Zerlegung von $r \in R$
    in Faktoren $s, t \in R$, $s, t \notin U(R)$, wobei
    $\ggT(s, t) = 1$ ist. \\
    Dann ist der zyklische $R$-Modul $M = R/Rr$ isomorph zu
    $R/Rs \oplus R/Rt$.
\end{Satz}

\begin{Kor}
    Seien $R$ ein HIR und $q = p_1^{e_1} \dotsm p_k^{e_k}$ Primfaktorzerlegung
    von $q \in R$. \\
    Dann ist $R/Rq \cong R/Rp_1^{e_1} \oplus \dotsb \oplus
    R/Rp_k^{e_k}$. \\
    Diese Zerlegung ist genau die Zerlegung von $R/Rq$ in Primärkomponenten
    $M = M_{p_1} \oplus \dotsb \oplus M_{p_k}$.
\end{Kor}

\begin{Bem}
    Seien $R$ ein HIR und $p_1, \dotsc, p_k \in R$ paarweise nicht-assoziierte
    Primelemente von $R$.
    Für $i = 1, \dotsc, k$ seien
    $e_1^{(i)}, \dotsc, e_{n_i}^{(i)} \in \natural$ mit
    $e_1^{(i)} \ge \dotsb \ge e_{n_i}^{(i)}$.
    Man setzt $e_\nu^{(i)} = 0$ für $\nu = n_i, \dotsc, n$ mit
    $n = \max\{n_1, \dotsc, n_k\}$.
    Außerdem seien wie oben $\mi{e}_i = (e_1^{(i)}, \dotsc, e_{n_i}^{(i)})$
    und $E_i = E(p_i, \mi{e}_i) = R/Rp_i^{e_1^{(i)}} \oplus \dotsb \oplus
    R/Rp_i^{e_n^{(i)}}$.
    (Beachte: Für $\nu > n_i$ ist $R/Rp_i^{e_\nu^{(i)}} = R/R = (0)$.) \\
    Sei $M = M(p_1, \mi{e}_1, \dotsc, p_k, \mi{e}_k) = E_1 \oplus \dotsb
    \oplus E_k$ aus der Liste oben. \\
    \begin{tabular}{p{5cm}p{11.0cm}}
    \matrixsize{$\begin{array}{ccccccc}
    e_1^{(1)} & \ge & \cdots & \ge & e_n^{(1)} & \ge & 0 \\
    \vdots & & & & \vdots & & \vdots \\
    e_n^{(k)} & \ge & \cdots & \ge & e_n^{(k)} & \ge & 0
    \end{array}$}
    &
    \begin{minipage}[c]{11.0cm}
    Betrachte linksstehendes Schema.
    Für $i = 1, \dotsc, n$ sei \\
    $q_i = p_1^{e_i^{(1)}} \dotsm p_k^{e_i^{(k)}}$.
    Dann ist $q_n \;|\; \dotsb \;|\; q_1$.
    Nach obiger Liste I ist $M = M_1 \oplus \dotsb \oplus M_n$ mit
    $M_i = R/Rp_1^{e_i^{(1)}} \oplus \dotsb \oplus
    R/Rp_k^{e_i^{(k)}}$.
    \end{minipage}\end{tabular}

    Es gilt $M_i \cong R/Rq_i$ nach obiger Folgerung.
    Die $q_i$ heißen dabei \begriff{Elementarteiler} von $M$. \\
    So kommt man auf folgende alternative Liste von e.e. $R$-Moduln.
\end{Bem}

\begin{Satz}{Liste II von Prototypen}
    Seien $R$ ein HIR, $q_1, \dotsc, q_n \in R$ Repräsentanten von
    Assoziierungsklassen von Elementen von $R$ mit $q_n \;|\; \dotsb \;|\; q_1$
    und $\alpha \in \natural_0$. \\
    Sei außerdem $M(q_1, \dotsc, q_n, \alpha) = R/Rq_1 \oplus \dotsb \oplus
    R/Rq_n \oplus R \oplus \overset{\alpha\text{-mal}}{\dotsb} \oplus R$. \\
    Ist $R_\alpha$ ein Repräsentantensystem der Assoziierungsklassen von $R$,
    dann ist \\
    $\{M(q_1, \dotsc, q_n, \alpha) \;|\;
    q_1, \dotsc, q_n \in R_\alpha,\;
    q_n \;|\; \dotsb \;|\; q_1,\;
    n \in \natural,\; \alpha \in \natural_0\}$
    ein vollständiges System paarweise nicht-isomorpher endlich erzeugter
    $R$-Moduln. \\
    Dabei ist $q_1 = \ordnung(M(q_1, \dotsc, q_n, 0))$ und
    $\ordnung(M(q_1, \dotsc, q_n, \alpha)) = 0$ für $\alpha \not= 0$.
\end{Satz}

\pagebreak

\chapter{%
    Anwendungen%
}

\section{%
    Endlich erzeugte \name{abel}sche Gruppen%
}

\begin{Satz}{zyklische Gruppen sind genau die zyklischen $\integer$-Moduln} \\
    Seien $G$ eine Gruppe und $x \in G$.
    Dann ist $\aufspann{x} = \{x^i \;|\; i \in \integer\}$
    eine abelsche Untergruppe von $G$, die \begriff{von $x$ erzeugte zyklische
    Untergruppe von $G$}.
    Die Abbildung $\rho: \integer \rightarrow \aufspann{x}$, $i \mapsto x^i$
    ist ein Gruppenepimorphismus von $(\integer, +)$ auf
    $(\aufspann{x}, \cdot)$.
    Ist $\ker \rho = (0)$, so ist $\integer \cong \aufspann{x}$ und die Ordnung
    $|\aufspann{x}|$ (d.\,h. die Anzahl der Elemente von $\aufspann{x}$) ist
    abzählbar unendlich. \\
    Ist $\ker \rho \not= (0)$ und $n \in \natural$ minimal mit $x^n = 1$, dann
    ist $\aufspann{x} = \{1, x, \dotsc, x^{n-1}\}$ und $|\aufspann{x}| = n$.
    Die Ordnung $|\aufspann{x}|$ von $\aufspann{x}$ heißt
    \begriff{Ordnung $|x|$ von $x$}.
    Ist $n = |x| \in \natural$, so ist
    $\aufspann{x} \cong (\integer/n\integer, +)$. \\
    Daher sind die zyklischen Gruppen genau die zyklischen $\integer$-Moduln
    und $(\integer, +)$ ist die einzige unendliche zykliche Gruppe.
\end{Satz}

\begin{Satz}{Klassifikation der endlich erzeugten abelschen Gruppen} \\
    Seien $q_1, \dotsc, q_k \in \natural$ (Elementarteiler) mit
    $q_k \;|\; \dotsb \;|\; q_1 \in \natural$ und $\alpha \in \natural_0$. \\
    Sei $M(q_1, \dotsc, q_k, \alpha) := C_{q_1} \times \dotsb \times C_{q_k}
    \times C_\infty \times \overset{\alpha\text{-mal}}{\dotsb} \times
    C_\infty$
    mit $C_n := (\integer/n\integer, +)$ und $C_\infty := (\integer, +)$. \\
    Dann ist $\{M(q_1, \dotsc, q_k, \alpha) \;|\;
    k \in \natural_0,\;
    q_1, \dotsc, q_k \in \natural,\;
    q_k \;|\; \dotsb \;|\; q_1,\;
    \alpha \in \natural_0\}$
    eine vollständige Liste paarweise nicht-isomorpher, endlich erzeugter
    abelscher Gruppen.
    Für $\alpha = 0$ erhält man mit
    $M(q_1, \dotsc, q_k) := M(q_1, \dotsc, q_k, \alpha)$ und
    $\{M(q_1, \dotsc, q_k) \;|\;
    k \in \natural_0,\;
    q_1, \dotsc, q_k \in \natural,\;
    q_k \;|\; \dotsb \;|\; q_1\}$
    eine vollständige Liste paarweise nicht-isomorpher, endlicher
    abelscher Gruppen. \\
    Dabei ist $|M(q_1, \dotsc, q_k)| = q_1 \dotsb q_k \in \natural$. \\
    Seien $p_1, \dotsc, p_k \in \natural$ Primzahlen,
    $e_1^{(i)} \ge \dotsb \ge e_n^{(i)} \ge 0$ ganze Zahlen für
    $i = 1, \dotsc, k$,
    $\mi{e}_i = (e_1^{(i)}, \dotsc, e_n^{(i)})$ und $\alpha \in \natural_0$.
    Dann erhält man durch \\
    $M(p_1, \mi{e}_1, \dotsb, p_k, \mi{e}_k, \alpha) = C_{p_1^{e_1^{(1)}}}
    \times \dotsb \times C_{p_1^{e_n^{(1)}}} \times \dotsb \times
    C_{p_k^{e_1^{(k)}}} \times \dotsb \times C_{p_k^{e_n^{(k)}}} \times
    C_\infty \times \overset{\alpha\text{-mal}}{\dotsb} \times C_\infty$
    eine vollständige Liste paarweise nicht-isomorpher, endlich erzeugter
    abelscher Gruppen.
\end{Satz}

\begin{Bem}
    Beispielsweise gibt es bis auf Isomorphie sieben abelsche Gruppen $A$
    mit $|A| = 32$ ($C_{32}$, $C_{16} \times C_2$, $C_8 \times C_4$,
    $C_8 \times C_2 \times C_2$, $C_4 \times C_4 \times 2$,
    $C_4 \times C_2 \times C_2 \times C_2$,
    $C_2 \times C_2 \times C_2 \times C_2 \times C_2$), aber nur eine
    mit $|A| = 15$ ($C_{15}$).
\end{Bem}

\begin{Bem}
    Das Wiedererkennungsproblem für abelsche Gruppen ist schwierig zu lösen,
    betrachtet man z.\,B. die $\integer$-Moduln
    $M_1 = \integer/2\integer \oplus \integer$ und
    $M_2 = \integer \oplus \integer$.
    Es gilt $M_1 = \aufspann{(1 + 2\integer, 1), (0, 1)}$ und
    $M_2 = \aufspann{(1, 0), (0, 1)}$, jedoch ist die Ordnungen aller
    Elemente der beiden Erzeugendensysteme $\infty$.
    Aus der Ordnung der Elemente von einem Erzeugendensystem kann man also
    nicht auf die abelsche Gruppe schließen.
\end{Bem}

\begin{Satz}{$\rational \otimes_\integer A = (0)$}
    Sei $A$ eine endliche abelsche Gruppe.
    Dann ist $\rational \otimes_\integer A = (0)$.
\end{Satz}

\begin{Satz}{$\rational \otimes_\integer \mathcal{F}$ als Vektorraum}
    Sei $\mathcal{F}$ ein freier $\integer$-Modul. \\
    Dann ist $\rational \otimes_\integer \mathcal{F}$ ein $n$-dimensionaler
    $\rational$-Vektorraum.
\end{Satz}

\begin{Satz}{Rangbestimmung des freien Anteils} \\
    Seien $M$ eine endlich erzeugte abelsche Gruppe und
    $n = \dim_\rational \rational \otimes_\integer M$. \\
    Dann ist $M = T(M) \oplus \mathcal{F}$, wobei der freie Anteil
    $\mathcal{F}$ von $M$ vom Rang $n$ ist.
\end{Satz}

\begin{Satz}{Anzahl abelscher Gruppen}
    Sei $k \in \natural$.
    Dann gibt es nur endlich viele paarweise nicht-isomorphe
    abelsche Gruppen $A$ mit $|A| = k$.
    Ist $k$ multiplizitätenfrei (in der Primfaktorzerlegung kommt jede
    Primzahl nur mit Exponent $1$ vor), so gibt es bis auf
    Isomorphie genau eine abelsche Gruppe $A$ mit $|A| = k$, nämlich die
    zyklische Gruppe $\integer/k\integer$ der Ordnung $k$.
\end{Satz}

\begin{Satz}{Kriterium für abelsche Gruppe zyklisch}
    Sei $A$ eine abelsche Gruppe.
    Dann ist $A$ zyklisch genau dann, wenn $A$ für jeden Teiler $d$ von $|A|$
    genau eine Untergruppe der Ordnung $d$ besitzt.
\end{Satz}

\section{%
    Die kanonisch rationale Form%
}

\begin{Bem}
    Seien $K$ ein Körper, $V$ ein endlich-dimensionaler $K$-Vektorraum und
    $f \in \End_K(V)$.
    Dann kann man den $K[t]$-Modul $V_f = V$ betrachtet, wobei die
    $K[t]$-Operation gegeben ist durch $p(t) \cdot v = (p(f))(v)$.
    Für das Verschwindungsideal $\mathcal{I}_f$ folgt sofort
    $\mathcal{I}_f = \ann_{K[t]}(V_f)$ sowie $\ordnung(V_f) = \mu_f(t)$.
\end{Bem}

\begin{Lemma}{$V_f$ e.e. Torsionsmodul}
    $V_f$ ist endlich-erzeugter Torsionsmodul.
\end{Lemma}

\begin{Lemma}{Unterraum von $V$ $f$-invariant $\Leftrightarrow$
              Unterraum Untermodul von $V_f$}
    Sei $U \ur V$ ein Untervektorraum.
    Dann ist $U$ $f$-invariant genau dann, wenn $U$ ein $K[t]$-Untermodul
    von $V_f$ ist.
\end{Lemma}

\begin{Satz}{$V_f$ und $V_g$ isomorph für $f, g$ konjugiert} \\
    Seien $f, g \in \End_K(V)$ konjugiert, d.\,h. es gibt ein
    $d \in \Aut_K(V)$ mit $f = d^{-1} g d$. \\
    Dann ist $d: V_f \rightarrow V_g$ ein $K[t]$-Isomorphismus und daher ist
    $V_f \cong V_g$ (als $K[t]$-Moduln).
\end{Satz}

\begin{Bem}
    Also ist die Modulstruktur von $V$ als $K[t]$-Modul für konjugierte
    Endomorphismen gleich, d.\,h. $V_f$ und $V_g$ sind zum selben Prototyp
    aus der obigen Liste isomorph.
    Weiter unten wird gezeigt:
    Dieser Prototyp bestimmt eine kanonisch rationale
    $K$-Basis von $V$, sodass konjugierte Endomorphismen dieselbe
    kanonisch rationale Form haben.
    Analog gilt dies für ähnliche Matrizen.
    Weil jede Matrix bzw. jeder Endomorphismus zu ihrer kanonisch rationalen
    ähnlich bzw. konjugiert ist, sind dann Matrizen/Endomorphismen mit der
    gleichen kanonisch rationalen Form ähnlich/konjugiert.
\end{Bem}

\begin{Kor}
    Seien $A, B \in M_n(K)$.
    Dann sind $A$ und $B$ ähnlich genau dann, wenn $A$ und $B$ dieselbe
    kanonisch rationale Form haben.
\end{Kor}

\begin{Bem}
    Das Minimalpolynom $\mu_f(t)$ als normiertes Polynom als Produkt normierter
    irreduzibler Polynome $\mu_f(t) = p_1(t)^{\nu_1} \dotsm p_k(t)^{\nu_k}$
    ($p_1, \dotsc, p_k \in K[t]$ paarweise verschieden, irreduzibel, normiert)
    dargestellt werden.
    So erhält man die Primärkomponentenzerlegung von
    $V_f = M_{p_1} \oplus \dotsb \oplus M_{p_k}$
    wegen $\ordnung(V_f) = \mu_f(t)$.
    Die Primärkomponenten kann man folgendermaßen ausrechnen:
\end{Bem}

\begin{Satz}{Primärkomponenten von $V_f$}
    Sei $\mu_f(t) = p_1(t)^{\nu_1} \dotsm p_k(t)^{\nu_k}$ wie eben. \\
    Dann ist $\ker(p_i(f)^{\nu_i-1}) \lneqq M_{p_i} =
    \ker(p_i(f)^{\nu_i}) \ur V$ für $i = 1, \dotsc, k$.
\end{Satz}

\begin{Satz}{Bestimmung der $\nu_i$}
    Die aufsteigende Kette
    $\ker p_i(f) \subseteq \dotsb \subseteq \ker p_i(f)^j \subseteq \dotsb$
    wird wegen $\dim_K V$ stationär.
    Sei $m$ die kleinste natürliche Zahl, sodass
    $\ker(p_i(f)^m) = \ker(p_i(f)^{m+1})$ ist.
    Dann ist $m = \nu_i$.
\end{Satz}

\begin{Lemma}{Primärkomponente von $t - \lambda$ ist verallg. Eigenraum}\\
    Seien $\mu_f(t) = p_1(t)^{\nu_1} \dotsm p_k(t)^{\nu_k}$ wie eben
    und $\lambda \in K$. \\
    Ist $p_i(t) = t - \lambda$ ein lineares Polynom,
    so ist $M_{p_i} = \mathcal{V}_f(\lambda)$.
\end{Lemma}

\begin{Satz}{Basis des zyklischen $K[t]$-Moduls}
    Seien $p \in K[t]$ ein Polynom mit $\deg p = n$ und \\
    $C_p = K[t]/K[t]p$ der zyklische $K[t]$-Modul.
    Dann ist $\dim_K(C_p) = n$ und $\{\nk{1}, \nk{t}, \dotsc, \nk{t}^{n-1}\}$
    ist $K$-Basis von $C_p$ (als $K$-Vektorraum), wobei
    $\nk{t}^i = t^i + K[t]p \in C_p$ ist.
\end{Satz}

\begin{Satz}{von $v$ erzeugter zyklischer Untermodul $K[t] \cdot v$}
    Sei $v \in V_f$.
    Dann ist der von $v$ erzeugte zyklische $K[t]$-Untermodul $K[t] \cdot v$
    der von $v$ erzeugte $f$-zyklische Unterraum von $V$. \\
    Dieser ist $f$-invariant und $f_{\aufspann{v}}$ sei die Einschränkung
    $f|_{K[t]v}$ von $f$ auf $K[t]v$. \\
    Sei $\mu_{f_{\aufspann{v}}}(t) = p(t) = \alpha_0 + \dotsb +
    \alpha_{k-1} t^{k-1} + t^k$ das normierte Minimalpolynom von
    $f_{\aufspann{v}}$. \\
    Dann ist $\ordnung(v) = p(t)$ und
    $\basis{B} = \{v, f(v), \dotsc, f^{k-1}(v)\}$ ist $K$-Basis von $K[t]v$. \\
    Die Matrix $\hommatrix{f_{\aufspann{v}}}{B}{B}$ ist die
    $k \times k$-Begleitmatrix von $p(t)$.
\end{Satz}

\pagebreak

\begin{Satz}{kanonisch rationale Form I}
    Seien $V$ ein $K$-Vektorraum mit $\dim_K V = n$, $f \in \End_K(V)$ und
    $\mu_f(t) = p_1(t)^{\nu_1} \dotsm p_k(t)^{\nu_k}$ die Primfaktorzerlegung
    von $\mu_f(t)$ in $K[t]$ mit paarweise verschiedenen, irreduziblen,
    normierten Polynomen.
    Sei außerdem $(p) \in M_{\deg p \times \deg p}(K)$ die Begleitmatrix von
    $p \in K[t]$. \\
    Dann gibt es eine Basis $\basis{B}$ von $V$ und natürliche Zahlen
    $e_1^{(i)} \ge \dotsb \ge e_m^{(i)} \ge 0$, $i = 1, \dotsc, m$,
    $m \in \natural$, sodass die $n \times n$-Matrix $\hommatrix{f}{B}{B}$
    die Blockdiagonalform \\
    $\diag\left\{\left(p_1^{e_1^{(1)}}\right), \dotsc,
    \left(p_1^{e_m^{(1)}}\right), \dotsc,
    \left(p_k^{e_1^{(k)}}\right), \dotsc,
    \left(p_k^{e_m^{(k)}}\right)\right\}$ hat, wobei
    $\sum_{i=1}^k \sum_{j=1}^m e_j^{(i)} \deg p_i = n$ ist. \\
    Diese Form heißt \begriff{kanonisch rationale Form} oder auch
    \begriff{\name{Frobenius}-Normalform} von $f$. \\
    Für $n \times n$-Matrizen ist sie analog definiert.
\end{Satz}

\begin{Satz}{kanonisch rationale Form II}
    $V$ hat eine $K$-Basis, sodass
    $\hommatrix{f}{B}{B} = \diag\{(q_1), \dotsc, (q_s)\}$ ist mit
    $q_i = p_1^{e_i^{(1)}} \dotsm p_k^{e_i^{(k)}}$, wobei
    $\mu_f(t) = q_1 = p_1^{\nu_1} \dotsm p_k^{\nu_k}$ ist mit
    $\nu_i = e_i^{(1)}$.
    Zusätzlich ist dann
    $\chi_f(t) = p_1^{e_1^{(1)} + \dotsb + e_m^{(1)}} \dotsm
    p_k^{e_1^{(k)} + \dotsb + e_m^{(k)}}$.
    Insbesondere ist $\mu_f(t) = \mu_f(t)$ genau dann, wenn $V_f$ nur einen
    Elementarteiler hat, d.\,h. zyklischer $K[t]$-Modul ist.
\end{Satz}

\pagebreak

