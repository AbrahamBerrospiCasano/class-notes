\chapter{%
    Ringe%
}

\section{%
    Ringe, Ideale, Restklassenringe und Körper%
}

\begin{Def}{Ring}
    Ein \begriff{Ring} $(R, +, \cdot)$ ist eine Menge $R$ mit zwei Abbildungen
    $+\colon R \times R \rightarrow R$, $(a, b) \mapsto a + b$ und
    $\cdot\colon R \times R \rightarrow R$, $(a, b) \mapsto a \cdot b$,
    sodass gilt:
    \begin{enumerate}[label=(R\arabic*)]
        \item
        $(R, +)$ ist eine abelsche Gruppe
        (mit neutralem Element $0$, zu $a$ inversem Element $-a$).

        \item
        \begriff{Assoziativität von $\cdot$}:
        $\forall_{a, b, c \in R}\; a(bc) = (ab)c$\\
        \begriff{Distributivität von $\cdot$ bzgl. $+$}:
        $\forall_{a, b, c \in R}\;
        a \cdot (b + c) = (a \cdot b) + (a \cdot c),\;
        (a + b) \cdot c = (a \cdot c) + (b \cdot c)$\\
        \begriff{neutrales Element von $\cdot$}:
        $\exists_{1 \in R} \forall_{a \in R}\; a \cdot 1 = a = 1 \cdot a$
    \end{enumerate}
    Im Folgenden wird zusätzlich $0 \not= 1$ verlangt
    (sonst wäre $R = \{0\}$ zugelassen).
\end{Def}

\begin{Def}{kommutativ}
    Ein Ring $(R, +, \cdot)$ heißt \begriff{kommutativ},
    falls $\forall_{a, b \in R}\; a \cdot b = b \cdot a$.
\end{Def}

\begin{Def}{Ringhomomorphismus}
    Seien $R$ und $S$ Ringe.
    Eine Abbildung\\
    $\varphi\colon R \rightarrow S$ heißt
    \begriff{Ringhomomorphismus}, falls gilt:
    \begin{itemize}
        \item
        $\varphi\colon (R, +) \rightarrow (S, +)$ ist ein Homomorphismus von
        abelschen Gruppen.

        \item
        $\forall_{a, b \in R}\;
        \varphi(a \cdot b) = \varphi(a) \cdot \varphi(b)$

        \item
        $\varphi(1_R) = 1_S$
    \end{itemize}
\end{Def}

\begin{Bem}
    $\{0\}$ ist kein Ring, da kein Einselement vorhanden ist.\\
    Für jeden Ring $R$ und $a \in R$ gilt
    $0 \cdot a = (0 + 0) \cdot a = 0 \cdot a + 0 \cdot a$, also
    $0 \cdot a = 0$.
\end{Bem}

\linie

\begin{Bsp}
    Beispiele für bekannte Ringe sind
    $(\integer, +, \cdot)$,
    $(\rational, +, \cdot)$,
    $(\real, +, \cdot)$ und
    $(\complex, +, \cdot)$.\\
    $(\Mat(n \times n, \rational), +, \cdot)$ ist ein Ring, der für
    $n \ge 2$ nicht kommutativ ist.\\
    $\rational[x] := \{f(x) = \sum_{i=0}^n a_i x^i \;|\; n \in \natural_0,
    a_0, \dotsc, a_n \in \rational\}$
    ist der \begriff{Polynomring} über $\rational$.\\
    Für $U \subset \real^n$ of"|fen ist bspw. die Menge aller stetigen
    Funktionen $f\colon U \rightarrow \real$ ein Ring, wobei Addition und
    Multiplikation im Bild erfolgen ($(f + g)(x) := f(x) + g(x)$,
    $(f \cdot g)(x) := f(x) \cdot g(x)$).\\
    Für eine abelsche Gruppe $G$ ist die Menge $\Hom(G, G)$ aller
    Gruppenhomomorphismen ein Ring.\\
    Die einzige Möglichkeit, $R = \{0, 1\}$ zu einem Ring zu machen, ist
    $1 \cdot 1 = 1$, $1 \cdot 0 = 0 \cdot 1 = 0 \cdot 0 = 0$,
    $1 + 0 = 0 + 1 = 1$, $0 + 0 = 1 + 1 = 0$.
    Dies entspricht dem Quotientenring $\integer/2\integer$ von
    $(\integer, +, \cdot)$.
\end{Bsp}

\linie

\begin{Def}{(Links-/Rechts-)Ideal}
    Seien $(R, +, \cdot)$ ein Ring und $I \subset R$ eine Untergruppe von
    $(R, +)$.\\
    $I$ heißt \begriff{Linksideal} von $R$, falls
    $\forall_{x \in I, a \in R}\; ax \in I$.
    $I$ heißt \begriff{Rechtsideal} von $R$, falls
    $\forall_{x \in I, a \in R}\; xa \in I$.\\
    $I$ heißt \begriff{Ideal} von $R$, falls $I$ Links- und Rechtsideal ist.
\end{Def}

\begin{Bsp}
    $I = \{0\}$ ist stets ein Ideal.
    Jede Untergruppe $I = n\integer \subset \integer$ von $\integer$ ist ein
    Ideal.
\end{Bsp}

\begin{Prop}{Restklassenring}
    Seien $R$ ein Ring und $I$ ein Ideal mit $I \not= R$.\\
    Dann ist die Faktorgruppe $R/I$ ein Ring mit der Multiplikation
    $(x + I) \cdot (y + I) := (xy) + I$.\\
    $R/I$ heißt \begriff{Restklassenring}.
\end{Prop}

\begin{Bem}
    Für $I = R$ wäre $R/I = \{0\}$ kein Ring (enthält kein Einselement).
\end{Bem}

\linie

\begin{Def}{Einheitengruppe}
    Sei $R$ ein Ring und\\
    $R^\ast := \{x \in R \;|\; x \text{ invertierbar bzgl. } \cdot\}$
    $= \{x \in R \;|\; \exists_{y \in R}\; xy = 1 = yx\}$.\\
    Die Elemente von $R^\ast$ heißen \begriff{Einheiten} und
    $R^\ast$ heißt \begriff{Einheitengruppe} von $R$.
\end{Def}

\begin{Bsp}
    $\integer^\ast = \{\pm 1\}$,
    $\rational^\ast = \rational \setminus \{0\}$,
    $\real^\ast = \real \setminus \{0\}$,
    $(\integer/6\integer)^\ast = \{\overline{1}, \overline{5}\}$
\end{Bsp}

\begin{Def}{Schiefkörper/Körper}
    $R$ heißt \begriff{Schiefkörper} oder \begriff{Divisionsring}, falls
    $R^\ast = R \setminus \{0\}$.\\
    $R$ heißt \begriff{Körper}, falls $R$ Schiefkörper und kommutativ ist.
\end{Def}

\begin{Bsp}
    Sei $K$ ein Körper und $R = K[x]$ der Polynomring.
    Was ist $R^\ast$?\\
    Für $f(x) \in R^\ast$ gibt es ein $g(x) \in R^\ast$ mit $f(x)g(x) = 1$.
    Ist $f(x) = a_0 + a_1 x + \dotsb + a_n x^n$ und
    $g(x) = b_0 + b_1 x + \dotsb + b_\ell x^\ell$ mit $a_j, b_j \in K$ und
    $a_n \not= 0$, $b_\ell \not= 0$, so gilt\\
    $1 = f(x) g(x) = a_n b_\ell x^{n+\ell} +
    \text{Terme echt kleineren Grades}$.
    Wegen $a_n b_\ell \not= 0$ muss $n + \ell = 0$ sein
    (Koef"|fizientenvergleich), d.\,h. $n = \ell = 0$ und
    $f(x) = a_0$.
    Also gilt $R^\ast = K \setminus \{0\}$.
\end{Bsp}

\section{%
    Kommutative Ringe%
}

\begin{Bem}
    Im Folgenden sei jeder Ring als kommutativ vorausgesetzt.
\end{Bem}

\begin{Prop}{Äquivalenzen zu Körper}
    Sei $R$ ein Ring.
    Dann sind äquivalent:
    \begin{enumerate}[label=(\alph*)]
        \item
        $R$ ist ein Körper.

        \item
        $R$ hat genau zwei Ideale ($\{0\}, R$).

        \item
        Für jeden Ring $S$ ist jeder Ringhomomorphismus $R \rightarrow S$
        injektiv.
    \end{enumerate}
\end{Prop}

\begin{Def}{Integritätsbereich}
    Sei $R$ ein Ring.\\
    $a \in R$ heißt \begriff{Nullteiler}, falls es ein
    $b \in R \setminus \{0\}$ gibt mit $ab = 0$.\\
    $R$ heißt \begriff{Integritätsbereich}, falls $0$ der einzige Nullteiler
    in $R$ ist.
\end{Def}

\begin{Bsp}
    In $R = \integer/6\integer$ sind die Nullteiler $\overline{0}$,
    $\overline{2}$, $\overline{3}$ und $\overline{4}$.\\
    $\integer$, $K$ und $K[x]$ sind Integritätsbereiche, falls
    $K$ ein Körper ist.
\end{Bsp}

\linie

\begin{Def}{Hauptideal(ring)/Primideal/max. Ideal}
    Sei $R$ ein Ring und $I$ ein Ideal in $R$.\\
    $I$ heißt \begriff{Hauptideal}, falls $\exists_{a \in R}\; I = Ra$.\\
    $R$ heißt \begriff{Hauptidealring}, falls jedes Ideal in $R$ ein Hauptideal
    ist.\\
    $I$ heißt \begriff{Primideal}, falls $I \not= R$ und
    $\forall_{a, b \in R,\; ab \in I}\; \{a, b\} \cap I \not= \emptyset$.\\
    $I$ heißt \begriff{maximales Ideal}, falls $I \not= R$ und
    $\forall_{J \text{ Ideal in } R,\; I \subset J}\; J \in \{I, R\}$.
\end{Def}

\begin{Bem}
    $I = \{0\}$ und $I = R$ sind Hauptideale.\\
    Ist $R$ ein Körper, so ist $R$ ein Hauptidealring.\\
    $R$ ist ein Körper genau dann, wenn $I = \{0\}$ maximales Ideal ist.\\
    $I$ ist maximal genau dann, wenn $R/I$ ein Körper ist.\\
    $I$ ist Primideal genau dann, wenn $R/I$ Integritätsbereich ist.\\
    Ist $I$ maximales Ideal, so ist $R/I$ ein Körper, also insb.
    Int.bereich und somit ist $I$ ein Primideal.\\
    (Die Umkehrung gilt nicht: $\{0\} \subset \integer$ ist Primideal,
    aber nicht maximal.)
\end{Bem}

\linie

\begin{Bsp}
    Im Beispiel $R = \integer$ sind Ideale genau die $n\integer$
    ($n \in \natural_0$), dies sind alles Hauptideale.\\
    Welche $n\integer$ sind Primideale, welche sind maximal?\\
    Sei zunächst $n = p$ Primzahl, dann ist $\integer/p\integer$ Körper,
    also ist $p\integer$ maximales Ideal und Primideal.\\
    Ist $n = ab$ mit $1 < a, b < n$, dann gilt in $\integer/n\integer$
    $\overline{0} = \overline{n} = \overline{ab} = \overline{a} \overline{b}$.
    Wegen $\overline{a}, \overline{b} \not= \overline{0}$ ist
    $\integer/n\integer$ kein Integritätsbereich, also ist $n\integer$
    weder Primideal noch maximales Ideal.\\
    Für $n = 0$ ist $0\integer = \{0\}$.
    $\integer/0\integer \simeq \integer$ ist ein Integritätsbereich, aber kein
    Körper,
    d.\,h. $0\integer$ ist Primideal, aber nicht maximal.\\
    Es gilt also: $n\integer$ ist ein Primideal genau dann, wenn $\pm n$
    eine Primzahl ist.
\end{Bsp}

\begin{Bem}
    Ein Beispiel für einen Ring, der kein Hauptidealring ist, ist
    $R = \integer[x]$.\\
    Sei dafür $I = \erzeugnis{2, x} =
    \{a_0 + a_1 x + \dotsb \;|\; a_i \in \integer,\; 2 \teilt a_0\}$.
    $I$ ist kein Hauptideal, denn andernfalls
    gäbe es ein $f(x) \in \integer[x]$ mit $I = \erzeugnis{f(x)} = Rf(x)$.
    Wegen $2 \in I$ gibt es dann ein $g(x) \in \integer[x]$ mit
    $f(x) g(x) = 2$.
    Da $\grad(f(x)g(x)) = \grad f(x) + \grad g(x) = 0$ sein muss,
    gilt $f(x) \in \integer$, d.\,h. $f(x) \in \{\pm 1, \pm 2\}$.
    Wegen $x \in I$ gibt es ein $h(x) \in \integer[x]$ mit $h(x) f(x) = x$,
    also $f(x) \not= \pm 2$.
    Daher gilt $f(x) = \pm 1$ und $I = Rf(x) = R$, ein Widerspruch zu
    $I \not= R$.
\end{Bem}

\linie

\begin{Def}{\name{euklid}isch}
    Ein Integritätsbereich $R$ heißt \begriff{\name{euklid}isch}, falls
    es eine \begriff{Gradabbildung}\\
    $\lambda\colon R \setminus \{0\} \rightarrow \natural_0$ gibt, sodass
    $\forall_{a \in R,\; b \in R \setminus \{0\}} \exists_{q, r \in R}\;
    a = qb + r$ und $r = 0$ oder $\lambda(r) < \lambda(b)$.
\end{Def}

\begin{Theorem}{euklidisch $\Rightarrow$ Hauptidealring}
    Sei $R$ euklidisch.
    Dann ist $R$ ein Hauptidealring.
\end{Theorem}

\linie

\begin{Prop}{Polynomring über Körper euklidisch}
    Sei $K$ ein Körper.\\
    Dann ist $K[x]$ ein euklidischer Ring, d.\,h. insbesondere Hauptidealring.
\end{Prop}

\begin{Bem}
    Man definiert dabei $\lambda(f(x)) := n$
    für $f(x) = a_0 + a_1 x + \dotsb + a_n x^n$, $a_n \not= 0$.
    Ist $I \not= \{0\}$ ein Ideal in $K[x]$, so ist
    $I = \erzeugnis{f(x)}$ mit $f(x)$ einem Polynom kleinsten Grades in $I$.
\end{Bem}

\begin{Def}{Ring der ganzen \name{Gauß}schen Zahlen}\\
    Der \begriff{Ring der ganzen \name{Gauß}schen Zahlen} ist
    $\integer[\i] := \{a + b\i \;|\; a, b \in \integer\} \subset \complex$.
\end{Def}

\begin{Prop}{$\integer[\i]$ euklidisch}
    Der Ring $\integer[\i]$ ist euklidisch, d.\,h. insbesondere Hauptidealring.
\end{Prop}

\begin{Bem}
    Die Norm $N(z)$ für $z \in \complex$ ist dabei definiert als
    $N(z) = |z|^2 = z \overline{z}$.
\end{Bem}

\section{%
    Irreduzible und Primelemente%
}

\begin{Bem}
    Im Folgenden sei jeder Ring als kommutativ vorausgesetzt.
\end{Bem}

\begin{Def}{irreduzibel/prim}
    Seien $R$ ein Integritätsbereich und $p \in R \setminus \{0\}$ mit
    $p \notin R^\ast$.\\
    $p$ heißt \begriff{irreduzibel}, falls
    $\forall_{x, y \in R,\; p = xy}\; \{x, y\} \cap R^\ast \not= \emptyset$.\\
    $p$ heißt \begriff{prim} oder \begriff{Primelement}, falls
    $\forall_{x, y \in R,\; p \teilt xy}\; (p \teilt x) \lor (p \teilt y)$.
    Eine äquivalente Definition ist, dass $\erzeugnis{p}$ ein Primideal ist.
\end{Def}

\begin{Lemma}{Primelemente sind irreduzibel}
    Jedes Primelement ist irreduzibel.
\end{Lemma}

\begin{Bem}
    Die Umkehrung gilt i.\,A. nicht.
\end{Bem}

\begin{Prop}{Äquivalenz in HIR}
    Seien $R$ ein Hauptidealring und $p \in R \setminus \{0\}$ mit
    $p \notin R^\ast$.\\
    Dann sind äquivalent:
    \begin{enumerate}[label=(\alph*)]
        \item
        $p$ ist irreduzibel.

        \item
        $p$ ist prim.

        \item
        $\erzeugnis{p}$ ist ein maximales Ideal.
    \end{enumerate}
\end{Prop}

\linie

\begin{Def}{\name{noether}sch}
    Ein Ring $R$ heißt \begriff{\name{noether}sch}, falls
    jede aufsteigende Kette von Idealen
    $I_1 \subset I_2 \subset \dotsb \subset I_k \subset \dotsb$
    stationär wird, d.\,h. es gibt ein $N \in \natural$ mit
    $I_N = I_{N+1} = \dotsb$.
\end{Def}

\begin{Lemma}{HIRs sind noethersch}
    Sei $R$ ein Hauptidealring.
    Dann ist $R$ noethersch.
\end{Lemma}

\begin{Def}{faktorieller Ring}
    Ein Integritätsbereich $R$ heißt \begriff{faktorieller Ring},
    falls jedes $a \in R \setminus \{0\}$ mit $a \notin R^\ast$
    als endliches Produkt von Primelementen darstellbar ist.\\
    Das ist äquivalent dazu, dass
    jedes $a \in R \setminus \{0\}$ mit $a \notin R^\ast$
    als endliches Produkt von irreduziblen Elementen darstellbar
    und diese Zerlegung bis auf Reihenfolge und Einheiten eindeutig ist.\\
    In faktoriellen Ringen sind Primelemente genau die irreduziblen Elemente.
\end{Def}

\begin{Def}{Repräsentanten der Primelemente}
    $\Prim(R)$ ist eine Menge von Repräsentanten von Primelementen von $R$,
    d.\,h. aus jeder Assoziiertheitsklasse
    $\{\varepsilon p \;|\; \varepsilon \in R^\ast\}$ für $p \in R$ prim
    wählt man genau ein Element aus.
\end{Def}

\begin{Theorem}{HIRs sind UFDs}
    Sei $R$ ein Hauptidealring.
    Dann ist $R$ faktoriell.
\end{Theorem}

\linie
\pagebreak

\begin{Bsp}
    In $K[x]$ ist z.\,B. $(x - \lambda)$ irreduzibel.\\
    $(x^2 + 1)$ ist irreduzibel in $\real[x]$ und
    $(x^2 - 2)$ ist irreduzibel in $\rational[x]$.\\
    $5$ ist nicht prim in $\integer[\i]$, da
    $5 = (1 + 2\i)(1 - 2\i)$, d.\,h.
    $5$ teilt das Produkt, aber $5$ teilt keinen der Faktoren
    (sonst wäre $5a = 1 \pm 2\i$, aber
    $N(5a) = 25 |a|^2 = 5 = N(1 \pm 2\i)$, d.\,h.
    $|a|^2 = \frac{1}{5}$, es gibt aber kein solches $a \in \integer[\i]$).\\
    $\integer[\sqrt{-5}]$ ist nicht faktoriell.
    Dazu zeigt man, dass z.\,B. $2$ irreduzibel, aber nicht prim ist.\\
    $2$ ist irreduzibel, denn aus $2 = ab$ folgt
    $N(a) = N(x + y\sqrt{-5}) = x^2 + 5y^2 \teilt 4 = N(2)$ und
    $N(b) = N(u + v\sqrt{-5}) = u^2 + 5v^2 \teilt 4 = N(2)$,
    somit gilt $y = v = 0$ und $a, b \in \integer$.
    Dann muss aber $a = 1$, $b = 2$ oder $a = 2$, $b = 1$ gelten.\\
    $2$ ist nicht prim, denn $2 \cdot 3 = 6 = (1 + \sqrt{-5})(1 - \sqrt{-5})$.
    Wäre $2$ prim, dann würde gelten, dass $2 \teilt (1 + \sqrt{-5})$ oder
    $2 \teilt (1 - \sqrt{-5})$.
    Aus $2 \teilt (1 \pm \sqrt{-5})$ folgt aber, dass
    $2z = 1 \pm \sqrt{-5}$ für ein $z \in \integer[\sqrt{-5}]$, also
    $z = \frac{1}{2} \pm \frac{1}{2} \sqrt{-5} \notin \integer[\sqrt{-5}]$,
    ein Widerspruch.
\end{Bsp}

\section{%
    Der Satz von \name{Gauss}%
}

\begin{Bem}
    Im Folgenden sei jeder Ring als kommutativ vorausgesetzt.
\end{Bem}

\begin{Theorem}{Satz von \upshape\,\!\name{Gauss}}
    Sei $R$ ein faktorieller Ring.
    Dann ist auch $R[x]$ faktoriell.
\end{Theorem}

\linie

\begin{Bem}
    Für den Beweis dieses Satzes benötigt man einige Vorarbeit.
\end{Bem}

\begin{Def}{Quotientenkörper}
    Sei $R$ ein Integritätsbereich.
    Definiere eine Äquivalenzrelation $\sim$ auf
    $M = \{(a, b) \in R \times R \;|\; b \not= 0\}$ mit
    $(a, b) \sim (c, d)$, falls $ad = bc$.
    Die Äquivalenzklasse von $(a, b) \in M$ wird mit $\frac{a}{b}$
    bezeichnet.
    Die Menge aller Äquivalenzklassen heißt \begriff{Quotientenkörper}
    $Q(R) := \{\frac{a}{b} \;|\; a, b \in R,\; b \not= 0\}$.
    Man definiert Addition und Multiplikation analog wie in $\rational$
    ($\frac{a}{b} + \frac{c}{d} := \frac{ad + bc}{bd}$,
    $\frac{a}{b} \cdot \frac{c}{d} := \frac{ac}{bd}$).
    Mit diesen Operationen wird $Q(R)$ zum Körper, der $R$ als Teilring
    enthält (mittels dem injektiven Ringhomomorphismus
    $R \rightarrow Q(R)$, $r \mapsto \frac{r}{1}$).
\end{Def}

\begin{Bem}
    Ist $R$ faktoriell und $a, b \in R$, so kann man $a$ und $b$ eindeutig
    bis auf Einheiten in Primelemente zerlegen, d.\,h.
    $a = \varepsilon p_1^{a_1} \dotsm p_n^{a_n}$ und
    $b = \varepsilon' p_1^{b_1} \dotsm p_n^{b_n}$
    für $a_i, b_i \in \natural_0$ und $\varepsilon, \varepsilon' \in R^\ast$.\\
    Daher ist
    $\frac{a}{b} = \widetilde{\varepsilon} p_1^{c_1} \dotsm p_n^{c_n}$
    mit $c_i = a_i - b_i$.
    Man kann also jedes Element $\frac{a}{b} \in Q(R)$ schreiben als
    $\frac{a}{b} = \varepsilon \prod_{p \in \Prim(R)} p^{\nu_p}$ mit
    eindeutigen Exponenten $\nu_p = \nu_p(\frac{a}{b}) \in \integer$.
    Formal setzt man $\nu_p(0) := \infty$, um die Regel
    $\nu_p(ab) = \nu_p(a) + \nu_p(b)$ auch auf $0$ anwenden zu können.
\end{Bem}

\begin{Bem}
    Auch für $Q(R)[x]$ kann man diese Schreibweise anwenden:\\
    Für $f(x) = \sum_{i=0}^n a_i x^i$ definiert man
    $\nu_p(f) := \min_{i=0,\dotsc,n} \nu_p(a_i)$.\\
    Mit obigem Formalismus gilt $f = 0 \iff \nu_p(f) = \infty$ und
    $f \in R[x] \iff \nu_p(f) \ge 0$.
\end{Bem}

\linie

\begin{Prop}{Lemma von \upshape\,\!\name{Gauss}}\\
    Seien $R$ ein faktorieller Ring, $p \in \Prim(R)$ und $f, g \in Q(R)[x]$.\\
    Dann gilt $\nu_p(fg) = \nu_p(f) + \nu_q(g)$.
\end{Prop}

\begin{Def}{normiertes Polynom}
    Ein Polynom $f(x) = \sum_{i=0}^n a_i x^i$ heißt \begriff{normiert},
    falls $a_n = 1$.
\end{Def}

\begin{Def}{primitives Polynom}
    Ein Polynom $f \in R[x]$ mit $\nu_p(f) = 0$ für alle $p \in \Prim(R)$
    heißt \begriff{primitiv}.
\end{Def}

\begin{Bem}
    Jedes normierte Polynom $f \in R[x]$ ist primitiv.
    Für primitive Polynome $f \in R[x]$ sind die Primfaktorzerlegungen über
    $R[x]$ und über $Q(R)[x]$ identisch.
\end{Bem}

\begin{Kor}
    Seien $R$ ein faktorieller Ring, $h \in R[x]$ normiert
    und $h = fg$ mit $f, g \in Q(R)[x]$.\\
    Dann gilt $f, g \in R[x]$.
\end{Kor}

\begin{Bem}
    $h \in R[x]$ primitiv ist irreduzibel in $R[x]$ $\iff$
    $h$ ist irreduzibel in $Q(R)[x]$.\\
    Für $g \in Q(R)[x]$ ist $g = af$ mit $f$ primitiv und
    $a = \prod_{p \in \Prim(R)} p^{\nu_p(g)} \in Q(R)$.
\end{Bem}

\pagebreak
