\section{%
    Körper%
}

\subsection{%
    Motivation und Beispiele%
}

\begin{Bem}
    Im Folgenden werden Methoden gesucht, mit dem man einen gegebenen
    Körper $K$ so zu einem Körper $L$ erweitern kann,
    sodass ein in $K[x]$ irreduzibles Polynom $f(x)$ in $L[x]$ reduzibel ist.
    Dies ist gleichwertig zur Suche von Nullstellen.
    $L$ sollte dabei in irgendeiner Art kleinstmöglich sein.
\end{Bem}

\begin{Bsp}
    Beispielsweise ist für $K = \real$ das Polynom $f(x) = x^2 + 1$
    irreduzibel, denn\\
    $f(x) = (x - \i)(x + \i)$ ist reduzibel in $\complex[x]$,
    d.\,h. aufgrund der Eindeutigkeit der Primfaktorzerlegung in $\complex[x]$
    müsste $f(x)$ in $\real[x]$ dieselbe Primfaktorzerlegung besitzen.
    Da $x \pm \i \notin \real[x]$ gilt, ist $f(x)$ irreduzibel in $\real[x]$.
\end{Bsp}

\begin{Bsp}
    In $K = \rational$ ist $f(x) = x^2 - 2$ mit gleicher Argumentation
    irreduzibel.
    Definiert man $L = \rational[\sqrt{2}] :=
    \{a + b\sqrt{2} \;|\; a, b \in \rational\}$, so sieht man, dass $L$
    ein Ring ist
    (es gilt $(a + b\sqrt{2}) + (c + d\sqrt{2}) \in L$ und
    $(a + b\sqrt{2})(c + d\sqrt{2}) = (ac + 2bd) + (ad + bc)\sqrt{2} \in L$).
    Man kann sogar zeigen, dass $L$ ein Körper ist:
    Für $a + b\sqrt{2} \in L$ muss ein multiplikativ
    Inverses gefunden werden.\\
    Falls $a = 0$, $b \not= 0$, so ist $\frac{1}{2b} \sqrt{2} \in L$ invers.
    Falls $b = 0$, $a \not= 0$, so ist $\frac{1}{a} \in L$ invers.\\
    Falls $a, b \not= 0$ ist, so muss für ein Inverses $c + d\sqrt{2}$ gelten,
    dass $ac + 2bd = 1$ und $ad + bc = 0$.
    Aus der zweiten Gleichung folgt $c = -\frac{ad}{b}$, d.\,h.
    $-\frac{a^2 d}{b} + 2bd = 1$, also
    $-a^2 d + 2b^2 d = b$ bzw.
    $d = \frac{b}{2b^2 - a^2}$.
    Man hat also $c, d \in \rational$ bzw. das Inverse $c + d\sqrt{2} \in L$
    gefunden.\\
    Der Nenner kann nicht $0$ werden, denn sonst ist
    $a^2 = 2b^2$ für $a, b \in \rational$.
    Setzt man $\frac{a}{b} = \frac{p}{q}$ mit ganzen Zahlen $p, q \in \integer$
    und $(p, q) = 1$, so gilt $p^2 = 2q^2$.
    Dann würde $2 \teilt p^2$ gelten, also $2 \teilt p$.
    Daraus folgt $4 \teilt p^2 = 2q^2$, also $2 \teilt q$,
    ein Widerspruch, denn $p$ und $q$ sind teilerfremd.\\
    Somit ist $\rational[\sqrt{2}]$ ein solcher Erweiterungskörper.
    Analog gilt $\real[\i] = \complex$ mit $\i = \sqrt{-1}$.
\end{Bsp}

\subsection{%
    Körpererweiterungen%
}

\begin{Def}{Teilkörper}
    Sei $L$ Körper.
    Ein Teilring $K \subset L$ heißt \begriff{Teilkörper} von $L$,
    falls
    $\forall_{a \in K \setminus \{0\}}\; a^{-1} \in K$.\\
    $L$ heißt dann \begriff{Erweiterungskörper} von $K$
    und die Inkl. $K \subset L$ heißt \begriff{Körpererweiterung} $L/K$.
\end{Def}

\begin{Def}{Zwischenkörper}
    Ein Körper $K'$ mit $K \subset K' \subset L$ heißt \begriff{Zwischenkörper}
    von $L/K$.
\end{Def}

\begin{Def}{erzeugter Teilkörper}
    Sei $M \subset L$ eine Teilmenge.\\
    Dann heißt $T(M) := \bigcap_{T \text{ Teilkörper von } L,\; T \supset M} T$
    der \begriff{von $M$ erzeugte Teilkörper} von $L$.\\
    $T(M)$ ist der kleinste Teilkörper von $L$, der $M$ enthält.
\end{Def}

\begin{Def}{Adjunktion}
    Sei $M \subset L$ eine Teilmenge und $K \subset L$ ein Teilkörper.\\
    Dann entsteht $K(M)$ aus $K$ durch \begriff{Adjunktion}, d.\,h.
    $K(M) := T(M \cup K)$.\\
    Für $M = \{a_1, \dotsc, a_n\}$ schreibt man
    $K(a_1, \dotsc, a_n) := K(\{a_1, \dotsc, a_n\})$.
\end{Def}

\begin{Def}{endlich erzeugt}
    $L/K$ heißt \begriff{endlich erzeugt}, falls
    $\exists_{a_1, \dotsc, a_n \in L}\; L = K(a_1, \dotsc, a_n)$.
\end{Def}

\begin{Def}{einfach}
    $L/K$ heißt \begriff{einfach} oder \begriff{einfache Erweiterung}, falls
    $\exists_{a \in L}\; L = K(a)$.
\end{Def}

\linie

\begin{Def}{Grad einer Körpererweiterung}
    Sei $L/K$ eine Körpererweiterung.\\
    Die Vektorraumdimension $\dim_K L$ heißt der \begriff{Grad} $[L:K]$
    der Körpererweiterung.
\end{Def}

\begin{Def}{endlich}
    Eine Körpererweiterung $L/K$ heißt
    \begriff{endlich}, falls $[L:K] < \infty$.
\end{Def}

\begin{Bsp}
    $[\rational[\sqrt{2}]:\rational] = 2 =
    [\rational[\sqrt{3}]:\rational] =
    [\complex:\real]$
\end{Bsp}

\begin{Lemma}{Produkt}
    Es gilt $[M:K] = [M:L] \cdot [L:K]$ für $M/L$ und $L/K$
    Körpererweiterungen.
\end{Lemma}

\subsection{%
    Auswertungshomomorphismus%
}

\begin{Bem}
    Im Folgenden soll versucht werden, zu gegebenen polynomialen Gleichungen,
    die in einem gegebenen Körper nicht lösbar sind, einen kleinstmöglichen
    größeren Körper zu konstruieren, in dem die Gleichung lösbar wird.\\
    Es ist also $K$ ein Körper und $f(x) \in K[x]$ gegeben.
    Existiert eine Körpererweiterung $L/K$, sodass $f(x)$ in $L$ eine
    Nullstelle hat?\\
    Bei z.\,B. $\rational[\sqrt{2}]$ oder $\real[\i]$ kannte man die Lösung
    schon, bevor man diese Körper konstruiert hat.
    Was ist, wenn man die Lösung nicht kennt?
\end{Bem}

\begin{Bem}
    Die Idee ist, $L$ als Quotient von $K[x]$ zu produzieren.
    Sinnvoll ist dabei, sich eine Abbildung
    $\varphi\colon K[x] \rightarrow L = K(a)$ zu definieren, wobei
    $\varphi(\lambda) = \lambda$ und $\varphi(x) = a$ für $\lambda \in K$
    gelten soll.
    Falls $\varphi$ existiert, so ist $K(a)$ Quotient von $K[x]$.
\end{Bem}

\linie

\begin{Prop}{Auswertungshomomorphismus}\\
    Seien $R$ und $S$ Ringe,
    $\alpha\colon R \rightarrow S$ ein Ringhomomorphismus und
    $a \in S$.\\
    Dann gibt es genau einen Ringhomomorphismus
    $\varphi\colon R[x] \rightarrow S$ mit $\varphi|_R = \alpha$ und
    $\varphi(x) = a$.\\
    $\varphi$ heißt \begriff{Auswertungshomomorphismus}.
    \begin{align*}
        \begin{xy}
            \xymatrix{
                R \ar@{^{(}->}[d] \ar[r]^{\alpha} & S \ni a \\
                R[x] \ar@{-->}[ur]_{\quad\exists!\varphi,\;
                \varphi|_R = \alpha,\; \varphi(x) = a}
            }
        \end{xy}
    \end{align*}
\end{Prop}

\linie

\begin{Bsp}
    Oft wird als $\alpha$ die Inklusion verwendet.
    Im Beispiel $\rational[\sqrt{2}]$ gibt es einen Ringhomomorphismus
    $\varphi\colon \rational[x] \rightarrow \rational[\sqrt{2}]$, wobei
    $f(x) \mapsto f(a)$ gilt
    (daher der Name Auswertungshomomorphismus).
    Wählt man $a = \sqrt{2}$, dann gilt $\varphi(c + dx) = c + d\sqrt{2}$,
    d.\,h. $\varphi$ ist surjektiv (i.\,A. ist dies nicht so).
    Da $\rational[x]$ ein Hauptidealring ist, ist
    $\Kern(\varphi) = \erzeugnis{f(x)}$ für $f(x) \in \Kern(\varphi)$ mit
    minimalem Grad.
    Man kann z.\,B. $f(x) = x^2 - 2 \in \Kern(\varphi)$ wählen
    (es gilt $\Kern(\varphi) = \erzeugnis{f(x)}$, da $f(x)$ irreduzibel ist,
    denn wenn $f(x)$ nicht minimalen Grad hätte, wäre $f(x)$ reduzibel).\\
    Nach dem Isomorphiesatz gilt
    $\rational[\sqrt{2}] \simeq \rational[x]/\erzeugnis{x^2 - 2}$.
    $\rational[x]/\erzeugnis{x^2 - 2}$ ist dabei "`unabhängig von $\sqrt{2}$"',
    d.\,h. man hat nun den Körper ohne Kenntnis der Lösung konstruiert.
    Die Lösung $\sqrt{2}$ entspricht dabei $\overline{x}$,
    denn für den Isomorphismus gilt
    $\overline{x} \mapsto \varphi(x) = \sqrt{2}$.
\end{Bsp}

\begin{Bsp}
    Analog gilt $\real[\i] \simeq \complex \simeq
    \real[x]/\erzeugnis{x^2 + 1}$.
\end{Bsp}

\begin{Bsp}
    Ein Beispiel, in dem der Auswertungshomomorphismus nicht surjektiv ist,
    ist $K = \rational$ mit $a = \pi$.
    Der Auswertungshomomorphismus
    $\varphi\colon \rational[x] \rightarrow \real$ kann nicht surjektiv sein,
    denn $\rational[x]$ ist abzählbar und $\real$ ist überabzählbar.
    Alternativ kann man auch $\pi^{-1} \notin \Bild(\varphi)$ zeigen:
    Sonst wäre $\pi^{-1} = \varphi(f(x)) = \sum_{i=0}^n r_i \pi^i$, also
    $1 = \sum_{i=0}^n r_i \pi^{i+1}$.
    Damit wäre $\pi$ Lösung einer algebraischen Gleichung, was nicht sein kann.
\end{Bsp}

\pagebreak

\subsection{%
    Algebraische Elemente und Minimalpolynom%
}

\begin{Def}{algebraisch/transzendent}
    Seien $L/K$ eine Körpererweiterung und
    $\varphi\colon K[x] \rightarrow L$ der Auswertungshomomorphismus mit
    $\alpha$ als Inklusion $K \rightarrow L$ und $\varphi(x) = a \in L$.\\
    Dann heißt $a$ \begriff{transzendent} über $K$,
    falls $\varphi$ injektiv ist,
    sonst \begriff{algebraisch (abhängig)} über $K$.
\end{Def}

\begin{Bem}
    $a$ ist transzendent genau dann, wenn $a$ keine algebraische Gleichung
    in $K$ erfüllt.
\end{Bem}

\linie

\begin{Def}{Minimalpolynom}
    Seien $L/K$ eine Körpererweiterung und $a \in L$ algebraisch.
    Ein Polynom\\
    $f(x) \in K[x] \setminus \{0\}$ minimalen Grades mit
    $f(a) = 0$ heißt \begriff{Minimalpolynom} von $a$ über $K$.\\
    Das \begriff{normierte Minimalpolynom} zu $a$ bezeichnet man mit
    $m_a = m_{a,K}$.
\end{Def}

\begin{Bem}
    Ist $f(x) \in K[x] \setminus \{0\}$ ein Minimalpolynom, so gilt
    $\Kern(\varphi) = \erzeugnis{f(x)}$, d.\,h. das Minimalpolynom ist
    eindeutig bis auf skalare Vielfache bestimmt.\\
    Insbesondere ist das normierte Minimalpolynom $m_{a,K}$ eindeutig
    bestimmt.
\end{Bem}

\begin{Bem}
    Jedes andere Polynom $p(x) \in K[x] \setminus \{0\}$
    mit $p(a) = 0$ wird von $m_a$ geteilt,
    d.\,h. es gibt ein Polynom $q(x) \in K[x]$ mit $m_a(x) q(x) = p(x)$.
    Ist $p(x)$ irreduzibel, so muss $q(x)$ eine Einheit sein, also
    $\grad q(x) = 0$ und $\grad m_a(x) = \grad p(x)$.
    Somit ist jedes normierte und irreduzible Polynom
    $p(x) \in K[x] \setminus \{0\}$ mit $p(a) = 0$ gleich $m_a$.\\
    Wäre umgekehrt $m_a$ reduzibel, so wäre $m_a(x) = p(x) q(x)$ mit
    $p(x), q(x) \in K[x]$ und\\
    $0 < \grad p(x), \grad q(x) < \grad m_a(x)$.
    Wegen $m_a(a) = 0$ ist $p(a) = 0$ oder $q(a) = 0$, d.\,h. $m_a$ hätte nicht
    minimalen Grad.
\end{Bem}

\begin{Lemma}{Kriterium für Minimalpolynom}\\
    Seien $L/K$ eine Körpererweiterung, $a \in L$ algebraisch
    und $p(x) \in K[x] \setminus \{0\}$ ein Polynom.\\
    Dann ist $p = m_a$ genau dann, wenn
    $p(a) = 0$ sowie $p$ normiert und irreduzibel ist.\\
    ($p$ ist Minimalpolynom genau dann,
    wenn $p(a) = 0$ und $p$ irreduzibel ist.)
\end{Lemma}

\linie

\begin{Def}{Polynome ausgewertet in $a$}
    Sei $L/K$ eine Körpererweiterung und $a \in L$.\\
    Dann ist $K[a] := \{\sum_{i=0}^n r_i a^i \;|\;
    n \in \natural_0,\; r_i \in K\}$ die Menge aller
    \begriff{Polynome ausgewertet in $a$}.
\end{Def}

\begin{Bem}
    Für $L/K$ und $a \in L$ ist $K(a)$ der kleinste Teilkörper von $L$, der
    $K \cup \{a\}$ enthält.\\
    Für den Auswertungshomomorphismus $\varphi\colon K[x] \rightarrow L$
    mit Inklusion $\alpha$ gilt $\Bild(\varphi) = K[a]$.\\
    $K[a]$ ist i.\,A. kein Körper.
\end{Bem}

\begin{Prop}{Äquivalenzen zu algebraisch}
    Seien $L/K$ eine Körpererweiterung und $a \in L$.\\
    Dann sind äquivalent:
    \begin{enumerate}[label=(\alph*)]
        \item
        $K[a] = K(a)$
        
        \item
        $a \in L$ ist algebraisch abhängig über $K$.
        
        \item
        $[K(a) : K] < \infty$
    \end{enumerate}
    In diesem Fall gilt zusätzlich $\lambda(m_{a,K}) = [K(a) : K]$.
\end{Prop}

\begin{Def}{Grad eines algebraischen Elements}
    Seien $L/K$ eine Körpererweiterung und $a \in L$ algebraisch.
    Dann heißt $\lambda(m_{a,K}) = [K(a) : K]$ \begriff{Grad} von $a$ über $K$.
\end{Def}

\begin{Bem}
    Im Beweis wird zusätzlich gezeigt:
    Falls $f(x) \in K[x] \setminus \{0\}$ irreduzibel ist,
    so ist $\erzeugnis{f(x)}$ maximales Ideal (siehe oben).
    Insbesondere ist das Ideal $\erzeugnis{m_{a,K}}$ ist maximal in $K[x]$.\\
    Der Körper $K[x]/\erzeugnis{f(x)}$ hat als $K$-Vektorraum die Basis
    $1, \overline{x}, \dotsc, \overline{x}^{n-1}$.\\
    Ist $a$ transzendent, dann ist $a^{-1} \notin K[a]$, d.\,h.
    $K[a]$ ist ein Körper $\iff$ $a$ ist algebraisch.
\end{Bem}

\pagebreak

\subsection{%
    Das Kriterium von \name{Eisenstein}%
}

\begin{Theorem}{Kriterium von \upshape\,\!\name{Eisenstein}}
    Sei $R$ ein faktorieller Ring,
    $K = Q(R)$ der Quotientenkörper von $R$ und
    $f(x) = \sum_{i=0}^n a_i x^i \in R[x]$ mit $n \ge 1$.\\
    Sei außerdem $p \in R$ irreduzibel mit $p \teilt a_i$ für
    $i = 0, \dotsc, n - 1$, aber
    $p \notteilt a_n$ und $p^2 \notteilt a_0$.\\
    Dann ist $f(x)$ irreduzibel in $K[x]$.\\
    Falls zusätzlich $f(x)$ primitiv ist (z.\,B. $a_n = 1$), so ist
    $f(x)$ irreduzibel in $R[x]$.
\end{Theorem}

\begin{Bsp}
    Für $f(x) = x^n - pq$ mit $p \in R$ prim und $q \in R$ mit $p \notteilt q$
    erfüllt $p$ das Kriterium, d.\,h. $x^n - pq$ ist irreduzibel in $R[x]$.\\
    Auf $g(x) = (x^p - 1)/(x - 1) = x^{p-1} + \dotsb + x + 1$
    mit $p \in R$ prim lässt sich das Kriterium nicht direkt anwenden.
    Substituiert man aber $x \rightarrow x + 1$ und nimmt an, dass
    $g(x) = g_1(x) g_2(x)$ reduzibel ist mit $\grad(g_1), \grad(g_2) \ge 1$,
    so ist $g(x + 1) = g_1(x + 1) g_2(x + 1)$ ebenfalls reduzibel.
    Es gilt aber $g(x + 1) = ((x + 1)^p - 1) / (x + 1 - 1) =
    \left(\sum_{j=0}^p \binom{p}{j} x^j - 1\right) / x$\\
    $= \sum_{j=1}^p \binom{p}{j} x^{j-1} =
    x^{p-1} + p x^{p-2} + \dotsb + p$
    und $p \teilt \binom{p}{j}$ für alle $j < p$.
    Daher ist das Kriterium von Eisenstein anwendbar und $g(x + 1)$
    irreduzibel, ein Widerspruch.
\end{Bsp}

\subsection{%
    Beispiel für eine Körpererweiterung%
}

\begin{Bsp}
    Ein Beispiel für eine einfache Körpererweiterung ist
    $(\rational(\sqrt{2}))(\sqrt{3})$.
    
    Dazu stellt man zunächst fest, dass $\sqrt{3} \notin \rational(\sqrt{2})$:
    Sonst wäre nämlich $\sqrt{3} = a + b\sqrt{2}$ mit $a, b \in \rational$.
    Quadrieren ergibt $3 = a^2 + 2b^2 + 2ab\sqrt{2}$ bzw.
    $\sqrt{2} = \frac{3 - a^2 - 2b^2}{2ab}$, d.\,h. $\sqrt{2} \in \rational$
    für $ab \not= 0$.
    Da dies nicht stimmt, ist $a = 0$
    (für $b = 0$ wäre $a^2 = 3$, das dies für $a \in \rational$ nicht geht,
    zeigt man analog wie für $\sqrt{2}$).
    Für $a = 0$ ist $\frac{3}{2} = \frac{p^2}{q^2}$ mit $b = \frac{p}{q}$,
    $p \in \integer$, $q \in \natural$ und $(p, q) = 1$.
    Daraus folgt $3q^2 = 2p^2$, d.\,h. $2 \teilt q^2$,
    $2 \teilt q$, $4 \teilt 2p^2$, $2 \teilt p^2$ und $2 \teilt p$.\\
    Das ist ein Widerspruch, daher ist $\sqrt{3} \notin \rational(\sqrt{2})$.
    
    Somit muss $\dim_\rational (\rational(\sqrt{2}))(\sqrt{3}) > 2$ sein
    ($\sqrt{3}$ ist nicht als $\rational$-Linearkombination von
    $1$ und $\sqrt{2}$ darstellbar).
    Es gilt außerdem $\dim_\rational (\rational(\sqrt{2}))(\sqrt{3}) \le 4$,
    da $x^2 - 3$ das Minimalpolynom von $\sqrt{3}$ über $\rational$ ist.\\
    $x^2 - 3$ ist auch irreduzibel über $\rational(\sqrt{2})$
    (sonst wäre $x^2 - 3$ das Produkt von zwei linearen Faktoren,
    wegen $x^2 - 3 = (x + \sqrt{3})(x - \sqrt{3})$ und $\complex[x]$
    faktoriell wären dies die gesuchten Faktoren, das steht allerdings im
    Widerspruch zu $\sqrt{3} \notin \rational(\sqrt{2})$, siehe oben).
    Daher ist $x^2 - 3$ das Minimalpolynom von
    $\sqrt{3}$ über $\rational(\sqrt{2})$.
    
    Somit ist $[(\rational(\sqrt{2}))(\sqrt{3}) : \rational(\sqrt{2})] = 2$,
    mit $[\rational(\sqrt{2}) : \rational] = 2$ gilt also
    $[(\rational(\sqrt{2}))(\sqrt{3}) : \rational] = 4$.
    
    Ist diese Erweiterung einfach?
    Dazu versucht man nun $b := \sqrt{2} + \sqrt{3}$.
    Es gilt $b^2 = 5 + 2\sqrt{2}\sqrt{3}$ und $(b^2 - 5)^2 = 24$.
    $b$ ist also Nullstelle von $x^4 - 10x^2 + 1 \in \rational[x]$.
    Es gilt $\rational(\sqrt{2} + \sqrt{3}) \subset
    (\rational(\sqrt{2}))(\sqrt{3})$.
    Umgekehrt gilt wegen $b^3 = 11\sqrt{2} + 9\sqrt{3}$, dass
    $b^3 - 9b = 2\sqrt{2}$, d.\,h.
    $\sqrt{2} \in \rational(\sqrt{2} + \sqrt{3})$ und daher auch
    $\sqrt{3} \in \rational(\sqrt{2} + \sqrt{3})$.
    Somit gilt $\rational(\sqrt{2} + \sqrt{3}) =
    (\rational(\sqrt{2}))(\sqrt{3})$ und
    $x^4 - 10x^2 + 1$ ist das Minimalpolynom von $\sqrt{2} + \sqrt{3}$
    (da Grad $4$ und\\
    $\lambda(m_{b,\rational}) =
    [\rational(\sqrt{2} + \sqrt{3}) : \rational] =
    [(\rational(\sqrt{2}))(\sqrt{3}) : \rational] = 4$).
\end{Bsp}

\pagebreak

\subsection{%
    Der Satz von \name{Kronecker}%
}

\begin{Def}{algebraische Körpererweiterung}
    Eine Körpererweiterung $L/K$ heißt \begriff{algebraisch}, falls alle
    $a \in L$ algebraisch abhängig über $K$ sind.
\end{Def}

\begin{Prop}{Körpererweiterungen}
    Seien $M/L$ und $L/K$ Körpererweiterungen. Dann gilt:
    \begin{enumerate}[label=(\alph*)]
        \item
        $L/K$ ist endlich genau dann, wenn $L/K$ algebraisch
        und endlich erzeugt ist.
        
        \item
        Sind $M/L$ und $L/K$ algebraisch, so ist auch $M/K$ algebraisch.
    \end{enumerate}
\end{Prop}

\begin{Theorem}{Satz von \upshape\,\!\name{Kronecker}}\\
    Seien $K$ ein Körper und $f(x) \in K[x]$ ein irreduzibles Polynom.\\
    Dann existiert eine einfache, algebraische Körpererweiterung $L/K$ mit
    $[L:K] = \grad(f(x))$, sodass $f(x)$ in $L$ eine Nullstelle hat.
\end{Theorem}

\begin{Bem}
    Alle polynomialen Gleichungen sind also lösbar!\\
    Der Beweis ist konstruktiv ($L := K[x]/\erzeugnis{f(x)}$,
    Nullstelle $\overline{x} = x + \erzeugnis{f(x)} \in L$).
\end{Bem}

\subsection{%
    \emph{Einschub}: Auswahlaxiom und \name{Zorn}sches Lemma%
}

\begin{Bem}
    Existiert für alle Körper $K$ der algebraische Abschluss $\overline{K}$
    (siehe unten)?
    Wenn ja, ist dieser eindeutig?
    Für den Existenzbeweis wird das Auswahlaxiom benötigt, das unabhängig
    vom Axiomensystem von Zermelo-Fraenkel ist.
\end{Bem}

\begin{Def}{Auswahlaxiom}
    Das \begriff{Auswahlaxiom} garantiert die Richtigkeit
    der folgenden Aussage:\\
    Seien $I \not= \emptyset$ eine Menge und
    $\{M_i \;|\; i \in I\}$ eine Menge von Mengen mit
    $M_i \not= \emptyset$ für alle $i \in I$.\\
    Dann existiert eine Funktion (\begriff{Auswahlfunktion})
    $f \colon I \rightarrow \bigcup_{i \in I} M_i$ mit
    $f(i) \in M_i$ für alle $i \in I$,
    d.\,h. es gibt eine Folge $(x_i)_{i \in I} \in \prod_{i \in I} M_i$.
\end{Def}

\linie

\begin{Def}{partielle Ordnung}
    Seien $M$ eine Menge und $\le$ eine Relation auf $M$.
    Dann heißt $\le$ \begriff{partielle Ordnung} auf $M$, falls
    $\forall_{x \in M}\; x \le x$(\begriff{reflexiv}),
    $\forall_{x, y, z \in M}\;
    (x \le y \land y \le z) \Rightarrow (x \le z)$ (\begriff{transitiv}) und
    $\forall_{x, y \in M}\; (x \le y \land y \le x) \Rightarrow (x = y)$
    (\begriff{antisymmetrisch}).
\end{Def}

\begin{Def}{Totalordnung}
    Seien $M$ eine Menge und $\le$ eine partielle Ordnung auf $M$.\\
    Dann heißt $\le$ \begriff{Totalordnung}, falls
    $\forall_{x, y \in M}\; (x \le y \lor y \le x)$.
\end{Def}

\begin{Def}{obere Schranke}
    Sei $N \subset M$.
    $a \in M$ heißt \begriff{obere Schranke für $N$}, falls
    $\forall_{x \in N}\; x \le a$.
\end{Def}

\begin{Def}{maximales Element}
    $a \in M$ heißt \begriff{maximales Element in $M$}, falls
    $\forall_{x \in M}\; (a \le x \Rightarrow x = a)$.
\end{Def}

\begin{Prop}{\name{Zorn}sches Lemma}
    Folgende Aussage ist äquivalent zum Auswahlaxiom:\\
    Sei $M \not= \emptyset$ partiell geordnet durch $\le$, sodass
    für jede total geordnete Teilmenge $N \subset M$ eine obere Schranke
    $a \in M$ existiert.
    Dann gibt es ein maximales Element in $M$.
\end{Prop}

\begin{Bsp}
    Mit dem Auswahlaxiom kann man zum Beispiel beweisen
    (sogar äquivalent):
    \begin{itemize}
        \item
        Jeder Vektorraum hat eine Basis.
        
        \item
        Es gibt nicht-messbare Mengen.
        
        \item
        Das Produkt von kompakten Mengen ist kompakt.
    \end{itemize}
\end{Bsp}

\pagebreak

\subsection{%
    Algebraischer Abschluss%
}

\begin{Bem}
    Um alle Nullstellen eines irreduziblen Polynoms zu erzeugen, kann
    man den Satz von Kronecker iterativ anwenden.\\
    Gibt es für beliebige Körper $K$ eine Körpererweiterung $L/K$, sodass
    \emph{alle} polynomialen Gleichungen lösbar sind?
    (Für $K = \real$ wählt man z.\,B. $L = \complex$.)
\end{Bem}

\begin{Def}{algebraisch abgeschlossen}
    Ein Körper $K$ heißt \begriff{algebraisch abgeschlossen}
    ($K = \overline{K}$), falls eine der folgenden äquivalenten
    Bedingungen erfüllt ist:
    \begin{enumerate}[label=(\alph*)]
        \item
        Jedes nicht-konstante Polynom $f(x) \in K[x] \setminus K$ hat eine
        Nullstelle in $K$.
        
        \item
        Jedes nicht-konstante Polynom $f(x) \in K[x] \setminus K$ zerfällt in
        ein Produkt von Linearfaktoren $f = f_1 \dotsm f_n$ mit
        $f_i(x) \in K[x]$ und $\grad f_i(x) = 1$ für $i = 1, \dotsc, n$.
        
        \item
        Jedes irreduzible normierte Polynom $f(x) \in K[x]$ ist von der Form
        $f(x) = x - a$, $a \in K$.
        
        \item
        Für jede algebraische Körpererweiterung $L/K$ gilt $L = K$.
    \end{enumerate}
\end{Def}

\begin{Def}{algebraischer Abschluss}
    Sei $K$ ein Körper.
    Dann heißt ein Erweiterungskörper $\overline{K}$, der
    algebraisch abgeschlossen und für den $\overline{K}/K$
    algebraisch ist,
    \begriff{algebraischer Abschluss} von $K$.
\end{Def}

\begin{Bsp}
    $\overline{\real} = \complex$
\end{Bsp}

\linie

\begin{Theorem}{Existenz von maximalen Idealen}
    Seien $R$ ein Ring.\\
    Dann existiert ein maximales Ideal $I_0$ in $R$
    ($I_0 \not= R$ und für jedes Ideal $J$ in $R$ mit $J \supset I_0$
    gilt $J = I_0$ oder $J = R$),
    falls das Auswahlaxiom vorausgesetzt wird.
\end{Theorem}

\begin{Bem}
    Somit kann man jeden Ring $R$ surjektiv auf einen Körper $R/I_0$ abbilden.
\end{Bem}

\begin{Theorem}{Existenz vom algebraischen Abschluss}\\
    Jeder Körper $K$ hat einen algebraischen Abschluss,
    falls das Auswahlaxiom vorausgesetzt wird.
\end{Theorem}

\linie

\begin{Bem}
    Für die Eindeutigkeit des algebraischen Abschlusses definiert man
    Eindeutigkeit als Eindeutigkeit bis auf $K$-Isomorphie, d.\,h. der
    Grundkörper soll elementweise festgehalten werden.
\end{Bem}

\begin{Def}{$K$-Homomorphismus}
    Seien $L_1/K$ und $L_2/K$ Körpererweiterungen über demselben Körper $K$ und
    $\varphi\colon L_1 \rightarrow L_2$ ein Ringhomomorphismus.\\
    $\varphi$ heißt \begriff{$K$-Homomorphismus}, falls $\varphi(x) = x$ für
    alle $x \in K$ (d.\,h. $\varphi|_K = \id_K$).\\
    $\varphi$ heißt \begriff{$K$-Isomorphismus}, falls
    $\varphi$ ein bijektiver $K$-Homomorphismus ist.\\
    $\varphi$ heißt \begriff{$K$-Automorphismus}, falls
    $\varphi$ ein $K$-Isomorphismus mit $L_1 = L_2$ ist.
\end{Def}

\begin{Def}{Gruppe der $K$-Automorphismen}
    Sei $L/K$ eine Körpererweiterung.\\
    Dann ist $\Aut_K(L)$ die \begriff{Gruppe der $K$-Automorphismen von $L$}
    unter Komposition.
\end{Def}

\begin{Bsp}
    Die komplexe Konjugation in $\complex/\real$, d.\,h.
    $a + b\i \mapsto a - b\i$ für $a, b \in \real$, ist ein
    $\real$-Automorphismus von $\complex$.
    Analog ist in $\rational[\sqrt{2}]/\rational$ die Abbildung\\
    $\rational[\sqrt{2}] \rightarrow \rational[\sqrt{2}]$,
    $a + b\sqrt{2} \mapsto a - b\sqrt{2}$ ein
    $\rational$-Automorphismus von $\rational[\sqrt{2}]$.
\end{Bsp}

\begin{Bem}
    Beide Automorphismen bilden eine Nullstelle des Minimalpolynoms
    ($x^2 + 1$ bzw. $x^2 - 2$) auf eine Nullstelle des Minimalpolynoms ab
    ($\i \mapsto -\i$ bzw. $\sqrt{2} \mapsto -\sqrt{2}$).
    Das ist kein Zufall:
    Ist $L = K(a)$, $m_a(x) = f(x) = \sum_{i=0}^n \lambda_j x^j$,
    $\alpha\colon L \rightarrow L$ ein $K$-Automorphismus und
    $x_0$ Nullstelle von $m_a(x)$, so gilt
    $0 = \alpha(0) = \alpha(\sum_{i=0}^n \lambda_j x_0^j) =
    \sum_{i=0}^n \alpha(\lambda_j) \alpha(x_0)^j =
    \sum_{i=0}^n \lambda_j \alpha(x_0)^j$, also ist $\alpha(x_0)$ Nullstelle
    von $m_a(x)$.
    Insbesondere gilt das für die Nullstelle $a$, es gilt sogar:\\
    Für einen $K$-Automorphismus $\varphi\colon L \rightarrow L$,
    $f(x) \in K[x]$ und $a \in L$ mit $f(a) = 0$ gilt
    $f(\varphi(a)) = 0$.
\end{Bem}

\linie
\pagebreak

\begin{Prop}{Anzahl an $K$-Isomorphismen}\\
    Seien $K, K'$ Körper, $\sigma\colon K \rightarrow K'$ ein
    Isomorphismus, $\sigma^\ast\colon K[x] \rightarrow K'[x]$,
    $\sum \lambda_i x^i \mapsto \sum \sigma(\lambda_i) x^i$\\
    der \begriff{induzierte Isomorphismus} und
    $L/K$ und $L'/K'$ algebraische Körpererweiterungen.\\
    Dann gilt:
    \begin{enumerate}[label=(\alph*)]
        \item
        Für $a \in L$ und $a' \in L'$ mit
        $m_{a',K'} = \sigma^\ast(m_{a,K})$ gibt es genau einen Isomorphismus\\
        $\varphi\colon K(a) \rightarrow K'(a')$ mit $\varphi|_K = \sigma$ und
        $\varphi(a) = a'$.
        
        \item
        Für $a \in L$ gilt
        $\#\{\varphi\colon K(a) \rightarrow L' \text{ Homom.}
        \;|\; \varphi|_K = \sigma\} =
        \#\{x \in L' \;|\; \sigma^\ast(m_{a,K})(x) = 0\}$.
    \end{enumerate}
\end{Prop}

\begin{Bsp}
    Für $K = K'$ ist $\sigma = \id_K$ ein Isomorphismus.
    Es gilt dann $\sigma^\ast = \id_{K[x]}$.\\
    Für $K = K' = \rational$,
    $L = \rational(\sqrt[3]{2})$, $L' = \complex$ und $a = \sqrt[3]{2}$
    ist $m_{a,\rational}(x) = x^3 - 2$.
    Nach (b) gilt daher
    $\#\{\varphi\colon \rational(\sqrt[3]{2}) \rightarrow \complex \;|\;
    \varphi|_\rational = \id_\rational\} =
    \#\{x \in \complex \;|\; x^3 - 2 = 0\}$.
    Die Menge der rechten Seite ist
    $\{\sqrt[3]{2}, \sqrt[3]{2} e^{2\pi\i/3}, \sqrt[3]{2} e^{4\pi\i/3}\}$,
    d.\,h. es gibt drei Abbildungen
    $\varphi\colon \rational(\sqrt[3]{2}) \rightarrow \complex$,
    die $\rational$ elementweise festlassen.
\end{Bsp}

\linie

\begin{Theorem}{Eindeutigkeit des algebraischen Abschlusses}\\
    Setzt man das Auswahlaxiom voraus, so gilt:
    \begin{enumerate}[label=(\alph*)]
        \item
        Seien $L/K$ algebraisch, $M$ algebraisch abgeschlossen und
        $\sigma\colon K \rightarrow M$ Homomorphismus.\\
        Dann existiert ein Homomorphismus $\varphi\colon L \rightarrow M$
        mit $\varphi|_K = \sigma$.
        \begin{align*}
            \begin{xy}
                \xymatrix{
                    K \ar[r]^\sigma \ar@{_{(}->}[d] &
                    M = \overline{M} \\
                    L \ar@{-->}[ru]_{\exists \varphi} &
                }
            \end{xy}
        \end{align*}
        
        \item
        Seien $K \simeq K'$ isomorph durch $\sigma$ und
        $\overline{K}, \overline{K'}$ algebraische Abschlüsse von $K, K'$.\\
        Dann existiert ein Isomorphismus $\varphi\colon \overline{K}
        \xrightarrow{\sim} \overline{K'}$ mit $\varphi|_K = \sigma$.
        \begin{align*}
            \begin{xy}
                \xymatrix{
                    K \ar[r]^\sigma_\sim \ar@{_{(}->}[d] &
                    K' \ar@{^{(}->}[d] \\
                    \overline{K} \ar@{-->}[r]_{\exists \varphi}^\sim &
                    \overline{K'}
                }
            \end{xy}
        \end{align*}
        
        \item
        Seien $K$ ein Körper und $L_1, L_2$ algebraische Abschlüsse von $K$.\\
        Dann existiert ein $K$-Isomorphismus
        $\varphi\colon L_1 \xrightarrow{\sim} L_2$.
        \begin{align*}
            \begin{xy}
                \xymatrix{
                    & K \ar@{_{(}->}[dl] \ar@{^{(}->}[dr] \\
                    L_1 \ar@{-->}[rr]_{\exists \varphi}^\sim & & L_2
                }
            \end{xy}
        \end{align*}
    \end{enumerate}
\end{Theorem}

\begin{Bem}
    Also ist der algebraische Abschluss eindeutig bis auf $K$-Isomorphie.
    Alle algebraischen Erweiterungen $L/K$ finden in $\overline{K}$ statt
    (bis auf $K$-Isomorphie).
\end{Bem}

\pagebreak
