\chapter{%
    Anwendungen%
}

\section{%
    Konstruktionen mit Zirkel und Lineal%
}

\begin{Bem}
    Die Aufgabe in diesem Abschnitt ist es,
    geometrische Konstruktionen durch Körpererweiterungen zu modellieren.
    Gegeben ist dabei eine Menge von "`Startpunkten"'
    $M \subset \real^2 = \complex$, ein Lineal (ohne Markierungen) und
    ein Zirkel.\\
    Das Ziel ist der Beweis der Unlösbarkeit von klassischen Problemen
    wie der Würfelverdopplung und der Winkeldreiteilung.
    Die Würfelverdopplung findet zwar im Dreidimensionalen statt, auf dort
    lassen sich die hier vorgestellten Aussagen jedoch leicht übertragen
    (beispielsweise ist es im Zweidimensionalen nicht möglich, die Kante
    eines verdoppelten Würfels zu konstruieren).
\end{Bem}

\linie

\begin{Def}{Gerade}
    Seien $M \subset \real^2$ und $p, q \in M$ mit $p \not= q$.\\
    Dann bezeichnet $p \lor q$ die \begriff{Gerade} durch $p$ und $q$.
\end{Def}

\begin{Def}{Kreis}
    Seien $M \subset \real^2$ und $p, q_1, q_2 \in M$ mit $q_1 \not= q_2$.\\
    Dann bezeichnet $K(p, \varrho)$ mit $\varrho := |q_1 - q_2|$
    den \begriff{Kreis} um $p$ mit Radius $\varrho$.
\end{Def}

\begin{Def}{elementare Konstruktion}
    Sei $M \subset \real^2$.\\
    Eine \begriff{elementare Konstruktion} aus $M$ ist eine der folgenden
    Konstruktionen:
    \begin{enumerate}[label=\Roman*.]
        \item
        \begriff{Schnitt von zwei Geraden}:\\
        Seien $p_1, p_2, q_1, q_2 \in M$,
        mit $p_1 \not= p_2$, $q_1 \not= q_2$ und
        $(p_1 \lor p_2) \not= (q_1 \lor q_2)$.\\
        Dann ist der Schnittpunkt
        $\widehat{p} := (p_1 \lor p_2) \cap (q_1 \lor q_2)$
        konstruiert (falls er existiert).

        \item
        \begriff{Schnitt einer Geraden mit einem Kreis}:\\
        Seien $p_1, p_2, q, q_1, q_2 \in M$,
        mit $p_1 \not= p_2$ und $q_1 \not= q_2$.\\
        Dann sind die Schnittpunkte
        $\{\widehat{r}, \widehat{s}\} := (p_1 \lor p_2) \cap K(q, |q_1 - q_2|)$
        konstruiert.

        \item
        \begriff{Schnitt von zwei Kreisen}:\\
        Seien $p, p_1, p_2, q, q_1, q_2 \in M$,
        mit $p_1 \not= p_2$ und $q_1 \not= q_2$.\\
        Dann sind die Schnittpunkte
        $\{\widehat{r}, \widehat{s}\} :=
        K(p, |p_1 - p_2|) \cap K(q, |q_1 - q_2|)$
        konstruiert.
    \end{enumerate}
\end{Def}

\begin{Def}{konstruierbare Punkte}
    Sei $M \subset \real^2 = \complex$.
    Ein Punkt $p = (x, y) \in \complex$ heißt
    \begriff{aus $M$ (mit Zirkel und Lineal) konstruierbar}, falls
    es ein $n \in \natural$ und
    $M = M_0 \subset M_1 \subset \dotsb \subset M_n$
    gibt mit $p \in M_n$, sodass jedes $M_i$ aus $M_{i-1}$ durch eine
    elementare Konstruktion entsteht.\\
    Die Menge $\Kon(M) := \{p \in \real^2 \;|\;
    p \text{ aus } M \text{ konstruierbar}\}$ ist die Menge aller aus $M$\\
    \begriff{konstruierbaren Punkte}.
\end{Def}

\linie
\pagebreak

\begin{Bem}
    Im Folgenden wird angenommen, dass $M$ stets zwei Punkte enthält,\\
    nämlich $0 := (0, 0)$ und $1 := (1, 0)$.
\end{Bem}

\begin{Theorem}{$\Kon(M)$ als Erweiterungskörper}
    Seien $M \subset \complex$ mit $0, 1 \in M$. Dann gilt:
    \begin{enumerate}[label=(\alph*)]
        \item
        $\Kon(M)$ ist ein Teilkörper von $\complex$.

        \item
        $\Kon(M) = \overline{\Kon(M)} := \{\overline{z} \;|\; z \in \Kon(M)\}$

        \item
        $\rational(M \cup \overline{M})$ ist ein Teilkörper von $\Kon(M)$.

        \item
        Für $b \in \complex$ gilt:
        Falls $b^2 \in \Kon(M)$ ist, so ist auch $b \in \Kon(M)$\\
        (d.\,h. $\Kon(M)$ ist \begriff{quadratisch abgeschlossen}).
    \end{enumerate}
\end{Theorem}

\begin{Bem}
    Man kann also mit Zirkel und Lineal addieren, subtrahieren, multiplizieren,
    dividieren und Quadratwurzeln ziehen.
    $\Kon(\rational)$ ist echt kleiner als $\complex$, hat
    aber unendlichen Grad über $\rational$,
    da $\sqrt{2}, \sqrt[4]{2}, \sqrt[8]{2}, \dotsc \in \Kon(\rational)$
    linear unabhängig sind.
\end{Bem}

\linie

\begin{Theorem}{Körpererweiterung $\Kon(M)/\rational(M \cup \overline{M})$}
    Seien $M \subset \complex$ und $0, 1 \in M$. Dann gilt:
    \begin{enumerate}[label=(\alph*)]
        \item
        $\Kon(M)/\rational(M \cup \overline{M})$ ist algebraisch.

        \item
        Für $z \in \complex$ gilt $z \in \Kon(M)$ genau dann, wenn
        es eine Kette von Körperweiterungen
        $\rational(M \cup \overline{M}) = L_0 \subset L_1 \subset \dotsb
        \subset L_r$ gibt mit $z \in L_r$ und
        $\forall_{j=1,\dotsc,r}\; [L_j:L_{j-1}] \le 2$.\\
        Für $z \in \Kon(M)$ ist also
        $[L_0(z):L_0]$ eine Potenz von $2$.
    \end{enumerate}
\end{Theorem}

\begin{Bem}
    Ist also $[L_0(z):L_0]$ keine Potenz von $2$, so ist $z$ nicht
    konstruierbar\\
    (z.\,B. für $M = \{0, 1\}$ ist $L_0 = \rational$).
\end{Bem}

\pagebreak

\section{%
    Unmöglichkeit bestimmter geometrischer Konstruktionen%
}

\begin{Bem}
    Die bisher entwickelte Theorie lässt sich nun für
    Unmöglichkeitsbeweise von geometrischen Konstruktionen verwenden:
    \begin{enumerate}
        \item
        \begriff{Würfelverdopplung (Delisches Problem)}:
        Konstruiere die Seitenlänge eines Würfels vom Volumen $2$.
        Aufgrund $[\rational(\sqrt[3]{2}):\rational] = 3$ ist
        $\sqrt[3]{2}$ nicht aus $0, 1$ konstruierbar, d.\,h.
        die Aufgabe ist unlösbar.

        \item
        \begriff{Dreiteilung eines Winkels}:
        Gegeben ist $z = e^{\i\alpha}$, konstruiere $e^{\i\alpha/3}$.\\
        Wähle $\alpha = 120^\circ = \frac{2\pi}{3}$.
        In diesem Fall ist
        $z = e^{2\pi\i/3} = -\frac{1}{2} + \frac{\i}{2} \sqrt{3}$ gegeben,
        gesucht ist $\xi = e^{2\pi\i/9}$.
        Es gilt $[\rational(z):\rational] = 2$
        ($x^2 + x + 1$ Minimalpolynom von $x$ über $\rational$).

        Wenn man zeigt, dass $[\rational(\xi):\rational] = 6$,
        dann folgt aufgrund
        $\rational(z) \subset \rational(z, \xi) = \rational(\xi)$ und\\
        $[\rational(\xi):\rational] =
        [\rational(\xi):\rational(z)] \cdot [\rational(z):\rational]$,
        dass $[\rational(\xi):\rational(z)] = 3$,
        d.\,h. $\xi$ ist nicht aus $z$ konstruierbar.
        Es gilt $[\rational(\xi):\rational] \le 6$, da
        das Minimalpolynom von $\xi$ über $\rational$
        das Polynom $\frac{x^9 - 1}{x^3 - 1} = x^6 + x^3 + 1$ teilen muss.
        Außerdem gilt $2 < [\rational(\xi):\rational]$
        und $2 \teilt [\rational(\xi):\rational]$.\\
        Es bleiben also nur die Möglichkeiten
        $[\rational(\xi):\rational] = 4$ und
        $[\rational(\xi):\rational] = 6$.

        Nun wird gezeigt, dass $[\rational(\xi):\rational] = 6$.
        Ein $\rational$-Automorphismus von $\rational(\xi)$ bildet
        jede Nullstelle von $x^6 + x^3 + 1$ wieder auf eine Nullstelle
        ab, d.\,h. $e^{2\pi\i/9}$ wird abgebildet
        $e^{2\pi\i\ell/9}$ mit $\ell \in \{1, 2, 4, 5, 7, 8\}$.
        Jeder Automorphismus
        $\sigma\colon \rational(\xi) \rightarrow \rational(\xi)$ ist bestimmt
        durch $\sigma(\xi) = e^{2\pi\i\ell/9}$, d.\,h.
        $\ell \in (\integer/9\integer)^\ast$.
        Die Zuordnung $\sigma \mapsto \ell \in (\integer/9\integer)^\ast$
        definiert einen Gruppenhomomorphismus
        $\Aut_\rational(\rational(\xi)) \rightarrow (\integer/9\integer)^\ast$,
        dieser ist injektiv.
        Somit ist $\Aut_\rational(\rational(\xi))$ isomorph zu einer
        Untergruppe von $(\integer/9\integer)^\ast$,
        daraus folgt $|\Aut_\rational(\rational(\xi))| \teilt 6$.
        $\rational(\xi)/\rational$ ist eine Galoiserweiterung
        (separabel und normal), also
        $|\Aut_\rational(\rational(\xi))| =
        [\rational(\xi):\rational] \teilt 6$.

        Somit muss $[\rational(\xi):\rational] = 6$ gelten und
        die Winkeldreiteilung ist nicht möglich.

        \item
        \begriff{Quadratur des Kreises}:
        Gegeben ist der Einheitskreis,
        gesucht ist ein Quadrat mit derselben Fläche, d.\,h.
        man muss $\sqrt{\pi}$ oder $\pi$ konstruieren.
        Die Zahlentheorie besagt allerdings, dass $\pi$ transzendent ist,
        also nicht konstruierbar.
        Somit ist die Quadratur des Kreises unmöglich.

        \item
        \begriff{Konstruktion von regelmäßigen $n$-Ecken}:
        Es müssen die $n$-ten Einheitswurzeln $\xi = e^{2\pi\i/n}$
        konstruiert werden.
        Das Minimalpolynom von $\xi$ über $\rational$ ist ein Teiler von
        $x^n - 1$, sein Grad ist $\varphi(n) := |(\integer/n\integer)^\ast| =
        \{j \in \{1, \dotsc, n\} \;|\; \ggT(j, n) = 1\}$
        (\begriff{\name{Euler}sche $\varphi$-Funktion}).
        Es gilt nun $\xi$ konstruierbar $\iff$
        $\varphi(n)$ ist eine Potenz von $2$ $\iff$
        $n = 2^\ell \cdot p_1 \dotsm p_r$ mit
        paarweise verschiedenen \begriff{\name{Fermat}schen Primzahlen}
        $p_1, \dotsc, p_r$
        (d.\,h. eine Primzahl der Form $p_i = 2^{2^a} + 1$).
        Für $a = 0, 1, 2, 3, 4$ ist das prim
        (man erhält $3$, $5$, $17$, $257$, $65537$),
        für $a = 5$ gilt allerdings $641 \teilt 4294967297$.
        Es ist ein ungelöstes Problem, ob weitere Fermatsche Primzahlen
        existieren
        (man vermutet, dass dies nicht zutrifft).
        Somit ist auch die Konstruktion von regelmäßigen $n$-Ecken
        für allgemeine $n$ ein ungelöstes Problem.
    \end{enumerate}
\end{Bem}

\pagebreak

\section{%
    Polynomiale Gleichungen%
}

\begin{Bem}
    Sei $K$ ein Körper und $f(x) \in K[x]$ ein Polynom vom Grad $n$.
    Gesucht ist eine Formel, die die Nullstellen von $f(x)$ berechnet.
    Beispielsweise geht dies für $n = 2$ und\\
    $f(x) = ax^2 + bx + c$
    mit der Mitternachtsformel $\frac{-b \pm \sqrt{b^2 - 4ac}}{2a}$
    für $\Char K \not= 2$.
    Für $n = 3$ und $f(x) = x^3 + ax^2 + bx + c$ ergeben sich schon
    kompliziertere Formeln, man formt zunächst um zu $x^3 + px + q$ und
    erhält Lösungen wie
    $\sqrt[3]{-\frac{q}{2} + \sqrt{\left(\frac{p}{3}\right)^3 +
    \left(\frac{q}{2}\right)^2}} +
    \sqrt[3]{-\frac{q}{2} - \sqrt{\left(\frac{p}{3}\right)^3 +
    \left(\frac{q}{2}\right)^2}}$
    für $\Char K \not= 2, 3$ -- hier werden schon verschiedene Wurzeln
    benötigt.\\
    Im Folgenden wird gezeigt, dass es für $n \ge 5$ keine solche allgemeine
    Formel gibt, die Lösungen aus den Koef"|fizienten berechnet
    (erlaubt sind $+, -, \cdot, /$ und beliebige Wurzeln).
    Dabei reicht es, ein Polynom anzugeben, das eine Nullstelle besitzt,
    die nicht mit diesen Operationen berechnet werden kann.\\
    Die Strategie ist, Körpererweiterungen $K(\sqrt[n]{a})/K$ zu
    Galoiserweiterungen zu vergrößern,
    sodass die Galoisgruppen spezielle Eigenschaften haben.
    Dann wird ein $f(x)$ angegeben, dessen Zerfällungskörper diese
    Eigenschaften nicht hat.
\end{Bem}

\linie

\begin{Bem}
    Im Folgenden ist $\Char K = 0$ (oder sogar $K = \rational$),
    d.\,h. Körpererweiterungen sind automatisch separabel.
\end{Bem}

\begin{Def}{Radikal}
    Seien $K$ ein Körper, $n \in \natural$, $a \in K$ und
    $E/K$ eine Körpererweiterung, sodass $b^n = a$ für ein $b \in E$.
    Dann heißt $b$ \begriff{Radikal} von $a$ über $K$
    (Schreibweise $b = \sqrt[n]{a}$).\\
    $b$ ist eindeutig bis auf Multiplikation mit \begriff{Einheitswurzeln}
    ($\sqrt[n]{1}$).
\end{Def}

\begin{Def}{Körpererw. durch Radikale auf"|lösbar}
    Eine Körpererweiterung $L/K$ heißt
    \begriff{(durch\\
    Radikale) auf"|lösbar}, falls
    es eine Kette von Körpererweiterungen
    $K = K_0 = K_1 \subset \dotsb \subset K_\ell$ gibt mit
    $\ell \in \natural$, $L \subset K_\ell$ und
    $K_{j+1} = K_j(b_j)$ mit $b_j = \sqrt[n_j]{a_j}$ für ein $a_j \in K_j$
    für alle $j = 0, \dotsc, \ell - 1$.
\end{Def}

\begin{Def}{Polynom durch Radikale auf"|lösbar}
    Ein Polynom $f(x) \in K[x]$ heißt
    \begriff{(durch Radikale) auf"|lösbar}, falls
    es sein Zerfällungskörper $L$ über $K$ durch Radikale auf"|lösbar ist.
\end{Def}

\linie

\begin{Bem}
    Im Folgenden sei $K$ ein Körper mit $\rational \subset K$
    und $K_n$ der Zerfällungskörper von $x^n - 1$ über $K$.
    Wegen $\rational \subset K$ haben daher die Einheitswurzeln
    $\sqrt[n]{1}$ die Werte $e^{2\pi\i j/n}$ für $j = 0, \dotsc, n - 1$.
\end{Bem}

\begin{Lemma}{$\Gal(K_n/K)$ abelsch}
    Es gibt einen injektiven Grp.homom.
    $\Gal(K_n/K) \rightarrow (\integer/n\integer)^\ast$,
    d.\,h. $\Gal(K_n/K)$ ist isomorph zu einer Untergruppe von
    $(\integer/n\integer)^\ast$ und daher abelsch.
\end{Lemma}

\begin{Lemma}{$\Gal(K(\sqrt[n]{a})/K)$ abelsch}
    Seien $e^{2\pi\i/n} \in K$ und $L := K(\sqrt[n]{a})$ für ein $a \in K$.\\
    Dann ist $L/K$ eine Galoiserweiterung und $\Gal(L/K)$ ist zyklisch mit
    $|\Gal(L/K)| \teilt n$.
\end{Lemma}

\begin{Bem}
    Es gilt auch die Umkehrung:
    Ist $L/K$ eine endliche Galoiserw. mit $\Gal(L/K)$ zyklisch und
    $n := [L:K]$, dann ist $L$ der Zerfällungskörper von $x^n - a$ für ein
    $a \in K$.
\end{Bem}

\linie
\pagebreak

\begin{Bem}
    Man erhält also in beiden Erweiterungen
    $K \subset K(\sqrt[n]{1}) \subset K(\sqrt[n]{a})$ abelsche Gruppen.
    Allerdings geht die Eigenschaft "`abelsch"' beim Iterieren verloren,
    wie folgendes Gegenbeispiel zeigt:
    Sei $K = \rational$, $n = 3$ und $a = 2$.
    Dann ist $\Gal(K(\sqrt[n]{1})/K) =
    \Gal(\rational(e^{2\pi\i/3})/\rational) \simeq \integer/2\integer$ und
    $\Gal(K(\sqrt[n]{a})/K) =
    \Gal(\rational(\sqrt[3]{2}, e^{2\pi\i/3})/\rational) \simeq \Sigma_3$.
    $\Sigma_3$ ist jedoch nicht abelsch.
\end{Bem}

\begin{Def}{Normalreihe}
    Sei $G$ eine Gruppe.
    Eine endliche Kette $\{1\} = G_0 \nt G_1 \nt \dotsb \nt G_n = G$
    von Untergruppen mit $G_j \nt G_{j+1}$ für $j = 0, \dotsc, n - 1$
    heißt \begriff{Normalreihe}.\\
    Die Normalreihe heißt \begriff{abelsch}, falls
    $G_{j+1}/G_j$ für $j = 0, \dotsc, n - 1$ abelsch ist.
\end{Def}

\begin{Def}{Gruppe auf"|lösbar}\\
    Eine Gruppe $G$ heißt \begriff{auf"|lösbar}, falls
    $G$ eine abelsche Normalreihe besitzt.
\end{Def}

\begin{Bem}
    Das Ziel ist zu zeigen, dass ein Polynom durch Radikale auf"|lösbar ist
    genau dann, wenn sein Zerfällungskörper eine auf"|lösbare Galoisgruppe
    besitzt.
    Dann muss man noch zeigen, dass es Galoisgruppen gibt, die nicht
    auf"|lösbar sind.
\end{Bem}

\begin{Bsp}
    Auf"|lösbare Gruppen sind z.\,B.
    abelsche Gruppen ($\{1\} \nt G$),\\
    $\Sigma_3$ ($\{1\} \nt \erzeugnis{(123)} \nt \Sigma_3$, da
    $[\Sigma_3:\erzeugnis{(123)}] = 2$, und
    $\Sigma_3/\erzeugnis{(123)}$ ist zyklisch, da
    $|\Sigma_3/\erzeugnis{(123)}| = 2$)\\
    und $G$ mit $|G| = p^n$ mit $p$ prim und $n \in \natural_0$
    (für $n \not= 0$ gilt $Z(G) \not= \{e\}$,
    $Z(G) \nt G$ mit $|G/Z(G)| = p^\ell$ für ein $\ell < n$,
    induktiv ist also $G$ auf"|lösbar).
\end{Bsp}

\linie

\begin{Def}{Kommutator}
    Seien $G$ eine Gruppe und $a, b \in G$.\\
    Dann heißt $[a, b] := aba^{-1}b^{-1}$ der \begriff{Kommutator}
    von $a$ und $b$.\\
    Die von allen Kommutatoren erzeugte Untergruppe
    $D(G) := \erzeugnis{[a, b] \;|\; a, b \in G}$ heißt\\
    \begriff{Kommutatoruntergruppe} (oder \begriff{derivierte Gruppe})
    von $G$.\\
    Mit $D^n(G) := D(\dotsb(D(G))\dotsb)$ bezeichnet man die
    \begriff{iterierte Kommutatoruntergruppe}.
\end{Def}

\begin{Bem}
    Es gilt $[a, b] = 1$ genau dann, wenn $ab = ba$.
    $\{[a, b] \;|\; a, b \in G\}$ ist i.\,A. keine Gruppe.
    Es gilt $D(G) \nt G$, da
    $g [a,b] g^{-1} = gaba^{-1}b^{-1}g^{-1} =
    (gag^{-1})(gbg^{-1})(ga^{-1}g^{-1})(gb^{-1}g^{-1})$ ein Kommutator ist.
    $G$ ist abelsch genau dann, wenn $D(G) = \{1\}$.
\end{Bem}

\begin{Bem}
    Durch iterierte Anwendung der Kommutatoruntergruppe kann man
    eine Normalreihe $G > D(G) > D^2(G) > \dotsb$ herstellen
    (die Untergruppen sind alle normal).\\
    $G/D(G)$ ist abelsch, denn für $a, b \in G$ ist
    $\overline{a} \overline{b} = \overline{b} \overline{a}$,
    da $\overline{1} = \overline{[a,b]} =
    \overline{a}\overline{b}\overline{a}^{-1}\overline{b}^{-1}$.\\
    Also ist $G \vartriangleright D(G) \vartriangleright
    D^2(G) \vartriangleright \dotsb$ eine abelsche Normalreihe, falls
    $D^n(G) = \{1\}$ für ein $n \in \natural$.
    Allerdings muss diese Bedingung nicht immer erfüllt sein:
    Ist $G$ einfach, aber nicht abelsch, dann besitzt $G$ keine Normalteiler
    außer $\{1\}$ und $G$.
    Wegen $D(G) \not= \{1\}$ ($G$ nicht abelsch) und $D(G) \nt G$ gilt
    also $G = D(G)$ (und $D(G) = D^2(G) = \dotsb$).\\
    Es kann also passieren, dass diese Reihe stehen bleibt.
    Im Folgenden wird das ausgenutzt, indem die Aussage getrof"|fen wird,
    dass dann $G$ nicht auf"|lösbar ist
    (man muss nur diese "`Testreihe"' prüfen).
\end{Bem}

\begin{Prop}{Testreihe der Kommutatoruntergruppen}\\
    Eine Gruppe $G$ ist auf"|lösbar genau dann, wenn
    $D^n(G) = \{1\}$ für ein $n \in \natural$.
\end{Prop}

\begin{Prop}{$\Sigma_n$ für $n \ge 5$ nicht auf\,\!lösbar}
    Sei $n \ge 5$.
    Dann ist $D(\Sigma_n) = D(A_n) = A_n$.\\
    Insbesondere sind $\Sigma_n$ und $A_n$ für $n \ge 5$
    nicht auf"|lösbar.\\
    ($A_n < \Sigma_n$ ist die Untergruppe der geraden Permutationen.)
\end{Prop}

\begin{Bem}
    Man kann zeigen, dass $A_n$ sogar einfach für $n \ge 5$ ist.
\end{Bem}

\linie
\pagebreak

\begin{Theorem}{Körpererw. auf\,\!lösbar $\Rightarrow$
                Galoisgrp. auf\,\!lösbar}\\
    Sei $L/K$ eine endliche Körpererweiterung mit $\Char K = 0$.
    Dann gilt (a) $\Rightarrow$ (b), wobei:
    \begin{enumerate}[label=(\alph*)]
        \item
        $L/K$ ist durch Radikale auf"|lösbar.

        \item
        Es gibt eine endliche Galoiserweiterung $M/K$ mit $M \supset L$,
        sodass $\Gal(M/K)$ auf"|lösbar ist.
    \end{enumerate}
\end{Theorem}

\begin{Bem}
    Es gilt auch die Umkehrung (b) $\Rightarrow$ (a), wobei aber die
    erwähnte Umkehrung des obigen Lemmas benötigt wird.
\end{Bem}

\begin{Bem}
    Wie wendet man dieses Theorem bei unbekanntem $M$ an?\\
    Gegeben seien $f(x) \in K[x]$ und
    $L$ der Zerfällungskörper von $f(x)$ über $K$.
    Aufgrund $\Char K = 0$ ist $L/K$ separabel, also galoissch.\\
    Angenommen, $L/K$ ist durch Radikale auf"|lösbar.
    Dann folgt aus dem Hauptsatz der Galoistheorie und obigem Satz, dass
    $\Gal(L/K) \simeq \Gal(M/K)/\Gal(M/L)$.\\
    Ist $\Gal(M/K)$ auf"|lösbar, so ist auch $\Gal(L/K)$ auf"|lösbar
    (allgemein gilt:
    gibt es einen surjektiven Gruppenhomomorphismus
    $G \rightarrow \overline{G}$ mit $G$ auf"|lösbar, so ist auch
    $\overline{G}$ auf"|lösbar,
    da aus $D^n(G) = \{e\}$ folgt, dass
    $D^n(\overline{G}) = \{\overline{e}\}$, weil
    $[\overline{g}, \overline{h}] = \overline{[g, h]}$).\\
    Ist also $L/K$ durch Radikale auf"|lösbar, so muss
    $\Gal(L/K)$ auf"|lösbar sein.
    Im Umkehrschluss kann eine Gleichung mit nicht auf"|lösbarer Galoisgruppe
    nicht durch Radikale auf"|lösbar sein.
\end{Bem}

\linie

\begin{Prop}{bestimmte Polynome in $\rational[x]$ sind nicht auf\,\!lösbar}
    Sei $f(x) \in \rational[x]$ irreduzibel vom Grad $5$, sodass
    $f(x)$ in $\complex$ genau drei reelle Nullstellen besitzt.\\
    Dann ist die Galoisgruppe von $f(x)$
    (d.\,h. die Galoisgruppe des Zerfällungskörpers von $f(x)$ über
    $\rational$)
    nicht auf"|lösbar,
    insbesondere ist $f(x)$ nicht durch Radikale auf"|lösbar.
\end{Prop}

\begin{Bem}
    Ein Beispiel für ein solches Polynom ist
    $f(x) = x^5 - 4x + 2 \in \rational[x]$
    (irreduzibel nach Eisenstein).
    Das Polynom $f(x) - 2 = x^5 - 4x = x(x^2 - 2)(x^2 + 2)$ hat
    drei reelle Nullstellen (nämlich $0$ und $\pm\sqrt{2}$) und
    zwei komplexe.
    Um die Frage zu beantworten, ob dies für $f(x)$ auch gilt, können die
    Extrempunkte bestimmt werden.
    $\frac{d}{dx}(f(x) - 2) = f'(x) = 5x^4 - 4 = 0$ gilt für
    $x = \pm\sqrt[4]{\frac{4}{5}}$.
    Der Wert von $f(x) - 2$ für diese $x$ ist größer bzw. kleiner als
    $\pm 2$,
    d.\,h. auch $f(x)$ hat $3$ reelle und zwei komplexe Nullstellen
    (der Abstand der Extrempunkte zur $x$-Achse ist größer als die
    Verschiebung).
    Somit ist $f(x)$ nach der Proposition nicht auf"|lösbar und
    es gibt keine allgemeine Formel für Lösungen polynomialer Gleichungen.
\end{Bem}

\section{%
    Der Fundamentalsatz der Algebra%
}

\begin{Bem}
    Man kann den Fundamentalsatz der Algebra tatsächlich algebraisch beweisen
    (zusätzlich z.\,B. zum naiv-analytischen, zum
    komplex-analytischen und zum topologischen Beweis).
    Dazu verwendet man nur ein wenig elementare Analysis:
    \begin{enumerate}[label=(\alph*)]
        \item
        Jedes Polynom $f(x) \in \real[x]$ mit ungeradem Grad besitzt
        eine reelle Nullstelle.

        \item
        Jede positive reelle Zahl besitzt eine Quadratwurzel
        (d.\,h. $f(x) = x^2$ hat das Bild $\real_{\ge 0}$).
    \end{enumerate}
    Die Aussagen folgen beide aus dem Zwischenwertsatz
    (Vollständigkeit von $\real$).
\end{Bem}

\begin{Theorem}{Fundamentalsatz der Algebra}\\
    Der Körper $\complex$ der komplexen Zahlen ist algebraisch abgeschlossen.
\end{Theorem}

\pagebreak
