\section{%
    Interferenz und Beugung%
}

\begin{itemize}
    \item
    \textbf{Strahlen-/Teilchenmodell}:
    Strahlenmodell/Teilchenmodell erklärt viele Dinge, aber manche
    Sachen können nur mit dem Wellenmodell beschrieben werden,
    manchmal müssen Wellen- und Teilcheneigenschaften des Lichts berücksichtigt
    werden, dies erfordert das Quantenmodell

    \item
    \textbf{Licht als elektromagnetische Welle}:
    Licht ist vektorielle Welle (nicht wie Schall skalar),
    Vektor des elektrischen und Vektor des magnetischen Felds stehen immer
    senkrecht zur Ausbreitungsrichtung (transversale Welle),
    jedoch wird $B$-Feld durch $E$-Feld fest bestimmt, d.\,h. man kann sich
    auf $E$-Feld konzentrieren,
    Licht wird letztlich durch die vier Parameter
    Amplitude (proportional zur Wurzel der Lichtintensität),
    Wellenlänge (als Maß für die "`Farbe"'),
    Phase und
    Polarisation
    beschrieben

    \item
    \textbf{Zusammenhang Wellen-/Strahlenmodell}:
    Lichtstrahlen stehen senkrecht auf den Wellenfronten
    (Flächen konstanter Phase), Wellenfronten stehen senkrecht auf den
    Lichtstrahlen, Kugelwelle/ebene Welle
\end{itemize}
\linie
\begin{itemize}
    \item
    \textbf{Interferenz}:
    Superposition (Addition) von Wellen

    \item
    \textbf{Seifenblase}:
    Licht wird einerseits direkt an Oberfläche reflektiert,
    andererseits an der Unterseite (Innenseite) der Seifenhaut reflektiert,
    beide Lichtanteile überlagern sich (interferieren),
    führt zu Abschwächung oder Verstärkung bestimmter Spektralanteile,
    unterschiedlicher Blickwinkel führt zu unterschiedlichem Farbeindruck

    \item
    \textbf{Lichtdetektoren}:
    detektieren das zeitliche Mittel der Intensität (Amplitude im Quadrat)

    \item
    \textbf{konstruktive/destruktive Interferenz}:
    je nach Phasenverschiebung überlagern sich die Wellen konstruktiv
    (verstärken sich) oder destruktiv (löschen sich annähernd aus)

    \item
    \textbf{kurz vor dem Zerplatzen wird Seifenblase schwarz}:
    Dicke geht gegen $0$, nur noch destruktive Interferenz möglich
    (Phasensprung von $\pi$)

    \item
    \textbf{Perlen, Perlmutt usw.}:
    schillert wegen Interferenz
    (Schichten aus Aragonit und Beugung an der feinen regelmäßigen
    Oberflächenstruktur),
    analog Interfenzpigmente in Lacken,
    Ölfilm auf Straße,
    Antireflexbeschichtung von entspiegelnden Gläser
    (destruktive Interferenz unterdrückt Reflexionen)

    \item
    \textbf{leuchtende Augen}:
    Interferenzspiegel erhöhen Lichtempfindlichkeit von Augen nachtaktiver
    Tiere (Katzen, Hunde, Wale, Pferde usw.)

    \item
    \textbf{Schmetterlinge}:
    komplizierte Interferenzef"|fekte an Schichtsystemen, teilweise auch mit
    Beugung und Streuung
    (zur Tarnung, Verwirrung der Gegner oder Partnersuche)
\end{itemize}
\linie
\begin{itemize}
    \item
    \textbf{Holografie}:
    Problem beim konventionellen Foto ist,
    dass die Richtung des einfallenden Lichts nicht aufgezeichnet wird,
    damit ist ein Umrunden der Szene nicht möglich,
    bei der Holografie wird zusätzlich zum üblichen Foto die Phase der
    Wellenfront aufgezeichnet, aus der die ursprüngliche Welle
    rekonstruiert werden kann

    \item
    \textbf{holografische Projektoren}:
    wie R2-D2 in Star Wars,
    würden nicht funktionieren, da Licht sich geradlinig ausbreitet
\end{itemize}
\linie
\pagebreak
\begin{itemize}
    \item
    \textbf{Beugung}:
    Überlagerung von vielen Wellen, geschieht aufgrund folgender Eigenschaften
    des Lichts:
    Licht breitet sich nur näherungsweise geradlinig aus
    (Lichtstrahl hat endliche Breite und Divergenz),
    Fokussierung nur auf Lichtfleck von der Größe der halben Wellenlänge
    möglich,
    fällt Licht auf kleine Begrenzung, dann wird es gebeugt, d.\,h.
    das Licht fächert sich auf,
    wird das gebeugte Licht mit Schirm sichtbar gemacht, so
    gibt es Stellen mit lokalen Intensitätsmaxima und Stellen mit
    Intensität $0$
    (Verteilung hängt ab von Wellenlänge, Begrenzung und Abstand des Schirms
    von Begrenzung),
    an feinen Gittern wird das Licht aufgespaltet in eine Ebene senkrecht
    zur Gitterorientierung (Anwendung bei Schmetterlingen oder Geldscheinen),
    Inferferenz tritt nur auf, falls intereferiende Lichtanteile zueinander
    kohärent sind,
    Kohärenzlänge = wie stark variieren unterschiedliche Wege
    von der Lichtquelle bis zum Beobachtungspunkt in der (optischen) Länge,
    ohne dass Interferenzfähigkeit verloren geht
    (Sonne: wenige Mikrometer, Laser: mehrere Kilometer)

    \item
    \textbf{\name{Poisson}scher Fleck}:
    helle Stelle im Zentrum eines Schattenbereichs

    \item
    \textbf{Wellenausbreitung in der Nähe von Hindernissen}:
    einfallendes Licht führt zu Anregung vieler kleiner Kugelwellen,
    die sich dann weiter ausbreiten

    \item
    \textbf{Beugung/Streuung}:
    was ist der Unterschied?
    Frage falsch, da es nur \emph{eine} Wechselwirkung des Lichts mit dem
    Objekt gibt, Beugung und Streuung sind nur Modelle zur
    näherungsweisen Beschreibung der Auswirkung
\end{itemize}
\linie
\begin{itemize}
    \item
    \textbf{Spalt/Doppelspalt/Mehrfachspalt/Gitter}:
    punktförmige, helle Lichtquelle (Halogenbirne/LED) durch
    dünnen Spalt betrachten,
    Resultat ist farbiger Streifen mit Maxima und Minima
    (statt punktförmiger Quelle), auch mit engmaschigem Giter möglich
    (sehr feine Gardinen, Strumpfhosen, Regenschirme, Vogelfeder)

    \item
    \textbf{\name{Fraunhofer}-Beugung}:
    vereinfachte mathematische Beschreibung der Beugung für große Abstände
    (Schirm idealerweise unendlich weit entfernt aufgestellt),
    Begründung der Beugungserscheinungen mit konstruktiver bzw. destruktiver
    Interferenz der von den Spalten ausgesandten Kugelwellen
\end{itemize}
\linie
\begin{itemize}
    \item
    \textbf{beugungsbegrenzte Auflösung}:
    $r_A = 1.22 \lambda K$ mit der Blendenzahl $K = \frac{f'}{D}$
    ($f'$ Brennweite und $D$ Durchmesser der Eintrittspupille der Optik,
    bei Einzellinsensystem ist dies der Linsendurchbesser), zwei Objektpunkte,
    deren Bilder gerade noch um $r_A$ separiert liegen, werden noch als
    getrennte Punkt wahrgenommen,
    $K$ kann nicht beliebig klein gemacht werden, maximal kann eine Auflösung
    von ca. $\lambda/2$ erzielt werden
    (Rayleighsche Auflösungsgrenze),
    hinzu kommen Fehler des optischen Systems/in der Fokussierung, die
    die Auflösung verringern

    \item
    \textbf{Spionagesatellit}:
    um Nummernschilder aus dem Weltraum zu erkennen, müsste das optische System
    einen Durchmesser von $\SI{8.1}{\meter}$ besitzen, also nicht möglich,
    zusätzlich kommen Wolken und Abberationen aufgrund Atmosphäre

    \item
    \textbf{ärgerlicher Mann/neutrale Frau}:
    begrenzte Auflösung optischer Abbildungen führt dazu, dass hohe Frequenzen
    eines Objekts (feine Strukturen) aus großen Entfernungen nicht mehr gesehen
    werden können
\end{itemize}
\linie
\pagebreak
\begin{itemize}
    \item
    \textbf{Korona}:
    farbige Ringmuster um Lichtquelle aufgrund der Beugung an kleinen
    Objekten (z.\,B. Wassertropfen),
    dünne Wolken oder Nebel zwischen Sonne oder Mond,
    inkohärente Überlagerung der an verschiedenen Tropfen gebeugten Wellen,
    falls alle Tropfen eine ähnliche Größe haben,
    Tropfen haben aufgrund des Babinet-Prinzips die Wirkung einer Ringblende
    mit demselben Durchmesser, daher Ringmuster
    (mehr oder weniger rotierte Spaltbeugungsfunktion),
    für ausgedehnte Lichtquelle ergibt sich für jeden Punkt der Lichtquelle
    ein farbiges Ringmuster mit Rot außen,
    Überlagerung ergibt auf der Fläche der Lichtquelle Weiß, aber
    am Rand heben sich die Farben nicht komplett weg, sodass ein
    farbiges Ringmuster um die Lichtquelle beobachtbar ist,
    sind Wolken zu dick, dominieren Mehrfachstreuungen und Ef"|fekt ist nicht
    beobachtbar

    \item
    \textbf{\name{Babinet}-Prinzip}:
    Beugungsmuster einer Blende entspricht (weitgehend) dem
    Beugungsmuster der inversen Blende

    \item
    \textbf{Pollenkorona}:
    Korona analog auch an Eiskristallen oder Pollen möglich,
    oder einfach in Dunkelheit auf starke Lichtquelle (Autoscheinwerfer)
    schauen, Beugung an kleinen Unregelmäßigkeit in der Hornhaut
    (ciliare Korona), oder durch leicht beschlagene Scheibe

    \item
    \textbf{Glorie}:
    Korona aufgrund von rückgestreutem Licht durch Wassertropfen,
    ähnliches Beugungsmuster,
    wird am besten aus vom Flugzeug aus beobachtet,
    zeigt sich in Richtung des eigenen Schattens,
    somit kann man daraus ermitteln, wo man im Flugzeug sitzt

    \item
    \textbf{irisierende Wolken}:
    kurze Koronasegmente,
    analog Entstehung durch Beugung an Wassertropfen oder Eiskristallen,
    aber Ringstruktur nicht sichtbar, da nur ein Teil des Ringbereichs mit
    passenden Partikeln bedeckt ist
    (oder Größe der Streuer variieren, dann entstehen komplexere Farbmuster),
    findet sich oft am Rand von Wolken
    (breite Größenverteilung der Wassertröpfchen bei geringer mittlerer Größe,
    Wolke dünn, d.\,h. kaum Mehrfachstreuung),
    bei großen Winkelabstand von der Sonne sind in der Regel Eiskristalle
    verantwortlich, denn Wasertropfen ergeben nur nach vorne und hinten starke
    Intensität
\end{itemize}
\linie
\begin{itemize}
    \item
    \textbf{überzählige Bögen}:
    Interferenzef"|fekte beim Regenbogen, da
    an unterschiedlichen Positionen einfallende Strahlen in einen Tropfen
    unter demselben Winkel wieder austreten und daher interferieren können,
    treten vor allem dann auf, wenn alle Tropfen ungefähr die gleiche Größe
    haben (sonst Verschmierung der Maxima),
    Maxima in der Nähe der Haupt- und Nebenbögen

    \item
    \textbf{Nebelbogen}:
    breite Beugungsbögen für jeder Wellenlänge, Überlagerung aller Farben
\end{itemize}
\linie
\pagebreak
\begin{itemize}
    \item
    \textbf{\name{Michelson}-Interferometer}:
    von links Lichtstrahl auf halbdurchlässigen Spiegel, nach oben bzw. rechts
    zu Spiegel, dann wieder zurück auf halbdurchlässigen Spiegel,
    Hälfte des Lichts wird nach unten geschickt zu einem Detektor, \\
    bei exakt gleichem Weg ist konstruktive Interferenz vorhanden, \\
    bei um $\lambda/4$ vergrößertem Abstand ist destruktive Interferenz
    vorhanden, \\
    Grundlage für Entfernungsmessung mittels Optik (Interferometrie)

    \item
    \textbf{Kohärenz}:
    bei inkohärenten Lichtquellen gibt es keine unendlich ausgedehnten
    Kosinusschwingungen, sondern relativ kurze Wellenpakete
    (z.\,B. mit Länge $\SI{5}{\micro\meter}$), aber sehr schnell hintereinander,
    Wellenpakete haben keine Beziehung zueinander

    \item
    \textbf{Kohärenz und Interferenz}:
    wird Länge beim Interferometer um $\SI{1}{\milli\meter}$ verstellt, dann
    verschwindet Interferenz, da viele unterschiedliche Überlagerungen
    von Wellenpaketen stattfinden (mal konstruktiv, mal destruktiv),
    im zeitlichen Mittel verschwindet der Interferenzterm,
    Argumentation bricht zusammen, wenn Wegdif"|ferenz kleiner als
    Wellenpaketlänge ist,
    hier gibt es lauter Überlagerungen von Wellen mit derselben
    Phasendif"|ferenz

    \item
    \textbf{Kohärenzlänge/-zeit}:
    räumliche/zeitliche Wellenpaketlänge,
    spektral schmalbandige \\
    Lichtquellen (Laser) haben große Kohärenzlängen,
    Sonne dagegen nur wenige $\si{\micro\meter}$

    \item
    \textbf{Speckles}:
    raue Oberfläche wird mit kohärentem Licht beleuchtet (Laser),
    dann gehen von der Oberfläche viele Kugelwellen mit zufälliger Phase
    aus, Resultat auf Netzhaut ist zufällige Interferenz,
    Voraussetzungen sind Unebenheiten kleiner Kohärenzlänge
    (aber nicht zu klein, sonst kaum Interferenz)
\end{itemize}
\linie
\begin{itemize}
    \item
    \textbf{Funktionsweise eines Lasers}:
    stimulierte Emission, ankommendes Licht regt andere Atome an,
    ebenfalls Licht auszusenden (sogar gleichphaseig),
    in sich kohärenter Wellenzug wird länger,
    zusätzliche Spiegel sorgen dafür, dass das Rohr nicht zu lang sein muss

    \item
    \textbf{warum sind Laser so gefährlich}:
    Leistung konzentriert auf einen kleinen Punkt (im Gegensatz z.\,B.
    zur Glühbirne) und
    Lichtquelle hat eine sehr geringe Ausdehnung (annäherend punktförmig),
    sodass die Lichtquelle auf einen kleinen Punkt in der Netzhaut abgebildet
    wird
\end{itemize}

\pagebreak
