\section{%
    Schatten und Perspektive%
}

\begin{itemize}
    \item
    \textbf{Licht} ist sehr kompliziert (bspw. \name{Einstein}),
    noch viele Fragen of"|fen

    \item
    \textbf{\name{Euklid}sches Strahlenmodell}:
    Strahl geht von Auge aus, trifft Objekt, so wird das Objekt gesehen

    \item
    Licht bereitet sich \textbf{geradlinig} aus
    und hat eine \textbf{endliche Geschwindigkeit}
\end{itemize}
\linie
\begin{itemize}
    \item
    \textbf{Definition Schatten} schwierig (z.\,B. Tokyo-Baumodell),
    Versuch: \\
    Schatten sind Regionen relativer Dunkelheit in beleucheteten Gebieten,
    die durch eine vollständige oder unvollständige Verdeckung der
    Lichtquelle(n) durch Objekte hervorgerufen werden.

    \item
    \textbf{Schatten sind wichtig}: Erfassung von Szenen, realistische
    Darstellung unabdingbar bei Fotografie und Rendern

    \item
    \textbf{Oppositionsef"|fekt}:
    bei Fotografie bei sonnigem Wetter in Richtung des Kamera-Schat\-tens
    sieht um den Schatten herum alles heller aus

    \item
    \textbf{Schatten haben meist einen kontinuierlichen Helligkeitsverlauf}:
    Kern- und Halbschatten (Umbra und Penumbra)
    entstehen durch ausgedehente Lichtquellen

    \item
    \textbf{Schatten sind kompliziert}:
    weiche Schatten sind nicht nur Summe der Helligkeitsverteilungen,
    seltsame Schattenmuster bei Überlagerungen entstehen durch komplizierte
    Abhängigkeit des sichtbaren Teils der Lichtquelle von der Position
    (Integral)

    \item
    \textbf{Schatten und Lochkamera}:
    Lichtflecken unter Bäumen sind rund, da Bilder der Sonne
    (extrem bemerkbar bei Sonnenfinsternissen), Begründung mit
    Lochkamera-Prinzip (z.\,B. Verfolgung der Strahlen)

    \item
    \textbf{Abbildungsgleichung}: $\beta := \frac{y'}{y} = \frac{a'}{a}$
    Abbildungsmaßstab mit
    $y$/$y'$ die Höhe des Objekts/Bilds und
    $a$/$a'$ die Entfernung des Objekts/Bilds zum Loch

    \item
    \textbf{Lochkamera}:
    nach ihrem Prinzip funktionieren vereinfacht auch Auge und Kamera,
    Probleme sind lange Belichtungszeit, großes Loch führt zu Unschärfe,
    Linse schafft Abhilfe, wiederum mit Nachteil, dass
    $\frac{1}{a'} = \frac{1}{f'} + \frac{1}{a}$ erfüllt werden muss, d.\,h.
    für jede Objektdistanz $a$ eine andere Linse (oder Neufokussierung)
\end{itemize}
\linie
\begin{itemize}
    \item
    \textbf{Perspektive}:
    Lage (Entfernung) von Objekten hat Einfluss auf das Abbild, z.\,B.
    Verschwörungstheorien zur Apollo 11

    \item
    \textbf{Perspektive ist eine direkte Folge aus Abbildungsgleichung}:
    parallele Gerade schneiden sich im Bild in einem Punkt (Fluchtpunkt),
    z.\,B. Gleise bei Bahnschienen: Bohlen werden immer kleiner

    \item
    \textbf{Bedeutungsperspektive}:
    wichtige Objekte sind größer gemalt (früher häufig oder heute bei Kindern)

    \item
    \textbf{Bilder von \name{Escher}}: zwei Fluchtpunkte werden
    zusammengelegt, \\
    perspektivisches Zeichnen wichtig

    \item
    \textbf{Schatten existieren im Volumen}:
    Schatten sind nicht zweidimensional, sondern überstreichen
    ein dreidimensionales Volumen (Brockengespinst/spectre of the Brocken --
    andere Schatten sieht man kaum, nur den eigenen)

    \item
    \textbf{Searchlight-Effekt}:
    Lichtstrahl in den Nachthimmel hört scheinbar plötzlich auf

    \item
    \textbf{farbige Schatten}:
    entstehen durch anderes farbiges Licht (z.\,B. blauer Himmel)
\end{itemize}

\pagebreak
