\chapter{%
    Auge und Wahrnehmung%
}

\begin{itemize}
    \item
    \textbf{menschliches Auge}:
    besteht aus Hornhaut, Iris, Pupille, Linse, Ziliarmuskeln, Glaskörper,
    Netzhaut, Aderhaut, Lederhaut und Sehnerv

    \item
    \textbf{Leistungsfähigkeit}:
    $1'' = \frac{1}{60}^\circ$ Auflösung im zentralen Bereich der Netzhaut,
    Wellenlängenempfindlichkeit schwankt individuell, liegt aber
    zwischen $\SI{400}{\nano\meter}$ und $\SI{700}{\nano\meter}$, \\
    mäßige Bildqualität hinsichtlich Auflösung, Kontrast usw., aber
    sehr gute Verarbeitung durch Sehnerven,
    technisch ist es zwar möglich, in Teilbereichen besser zu sein als das
    Auge, aber insgesamt gesehen ist das Auge unerreicht

    \item
    \textbf{autonomes Fahren}:
    Beispiel für schwierige technische Umsetzung des Auges

    \item
    \textbf{Evolution}:
    (alles Annahme)
    1. lichtempfindliche Flecken auf der Oberfläche,
    2. Sensoren in Gruben um dif"|fuse Beleuchtung abzuschirmen,
    3. durchsichtige Membran zum Schutz gegen Schmutz,
    4. Linsenwirkung zur Verstärkung der Sensitivität,
    5. mehrere Sensortypen

    \item
    \textbf{warum sieht man unter Wasser schlecht}:
    andere Brechzahl, Brechzahlunterschied deutlich geringer,
    daher geringerer Ablenkwinkel und Fokussierung des Objekts erst hinter der
    Netzhaut

    \item
    \textbf{Akkomodation}:
    Fähigkeit zur Fokussierung auf unterschiedlich entfernte Objekte,
    wird mit zunehmenden Alter schwächer

    \item
    \textbf{Gründe für größere Augen}:
    Empfindlichkeit und beugungsbegrenzte Auflösung

    \item
    \textbf{warum können Adler so scharf sehen}:
    kleines Gehirn, daher mehr Raum für Augen, und
    Negativlinse direkt vor der Netzhaut (Funktion eines Teleobjektivs)

    \item
    \textbf{Iris}:
    Regulation der einfallenden Lichtintensität und Tiefenschärfenregelung,
    außerdem kann man erkennen, wo man gerade hinschaut

    \item
    \textbf{Augenbewegung}:
    ruckartige Bewegungen auf die momentan interessanteste Objektposition
    (damit das Bild dieser Region auf Bereich der Netzhaut mit der höchsten
    Auflösung fällt),
    außerdem laufendes, hochfrequentes Zittern, um eine gute Auflösung zu
    erhalten (Sehzellen reagieren nur auf Veränderungen, würde man dies
    ausgleichen, so würde man nur noch graue Fläche sehen)

    \item
    \textbf{\name{Purkinje}-Reflex}:
    Augen bewegen sich bei Körperbewegung, um diese Bewegung auszugleichen

    \item
    \textbf{Floaters}:
    abgelöste Retinazellen im Kammerwasser
\end{itemize}
\linie
\pagebreak
\begin{itemize}
    \item
    \textbf{\name{Purkinje}-Ef"|fekt}:
    bei wenig Licht ist man für blaues Licht emfindlicher, weil dann das Sehen
    im Wesentlichen durch die Stäbchen erfolgt, die eine andere spektrale
    Empfindlichkeit als die Zapfen haben

    \item
    \textbf{Hell-/Dunkel-Adaption}:
    rotes Licht im Cockpit dient dazu, dass die Zapfen für die Detailsicht
    aktiv bleiben, während gleichzeitig die Stäbchen an die Dunkelheit
    adaptiert werden,
    Auge passt sich an die Helligkeit an (innerhalb von $\SI{30}{\minute}$)

    \item
    \textbf{\name{Weber}-\name{Fechner}-Gesetz}:
    Reaktion auf einfallende Lichtintensität ist beim Mensch nicht-linear,
    der geringste noch wahrnehmbare Helligkeitsunterschied $\Delta I$
    ist proportional zur Gesamthelligkeit $I$,
    d.\,h. $\frac{\Delta I}{I} = \text{const.}$,
    daher erscheinen Kerzen in dunklen Räumen heller wie in hellen
    und tagsüber sind keine Sterne sichtbar (analog Gardinenef"|fekt),
    Fechner-Gesetz gilt nicht für geringe Helligkeiten

    \item
    \textbf{\name{Pulfrich}-Ef"|fekt}:
    werden Sinneszellen mit wenig Licht gereizt, dann geben sie den
    entsprechenden Reiz etwas verzögert weiter,
    z.\,B. sich zweidimensional bewegendes Pendel, ein Auge mit
    Sonnenbrillenglas abdecken ergibt dreidimensionalen Ef"|fekt
    (Anwendung: 3D-Fernsehen und MS-Diagnose)
\end{itemize}
\linie
\begin{itemize}
    \item
    \textbf{nachts sind alle Katzen grau}:
    da die Zapfen abgeschaltet werden

    \item
    \textbf{\name{Benham}-Scheibe}:
    unterschiedlich große schwarze Striche auf weißem Grund, bei Drehung
    erscheinen die Striche farbig (Farbrezeptoren arbeiten bei gepulstem Licht
    anders, Empfindlichkeitskurven in der Frequenz verschieben sich)

    \item
    \textbf{Nachbilder}:
    helle Lichtquelle betrachten,
    zunächst positives Nachbild (Überregung der Sinneszellen, feuern auch noch
    ohne Lichtreiz nach), dann negatives Nachbild
    (Ausbleichung der angeregten Sinneszellen, Komplementärfarben, da nur
    die entsprechenden Zapfen ausgebleicht sind)
\end{itemize}
\linie
\begin{itemize}
    \item
    \textbf{Realität und Wahrnehmung}:
    die Bilder, die wir wahrnehmen, sind nicht die Bilder, die auf die
    Netzhaut fallen, stattdessen bildet unser Gehirn (mit anderen Informationen
    zusammen, z.\,B. Gehör, Vorwissen usw.) fortlaufend ein Modell der
    Realität, diese Realtität nehmen wir bewusst wahr,
    optische Täuschungen sind falsche Modellbildungen,
    die allermeistens allerdings berechtigt sind

    \item
    \textbf{Modell zur optischen Wahrnehmung}:
    die Realität wird zunächst vom Auge mit\\
    $100\,\text{MPixel}$ wahrgenommen,
    dort werden schon einfache Merkmal (wie Kante) herausgearbeitet,
    das optische Signal auf ca. $1\,\text{MPixel}$ komprimiert und zum
    Gehirn geschickt, das aufgrund von anderen Informationen eine
    Hypothesenbildung durchführt, die unsere wahrgenommene Realität
    erzeugt, aufgrund derer wir unsere Reaktionen planen

    \item
    \textbf{Kantenerkennung/Erkennung von Gesichtern}:
    sehr wichtig

    \item
    \textbf{\name{Thomson}-Ef"|fekt}:
    wenn bei einem Porträt nur Augen und Mund gedreht werden,
    sieht es merkwürdig aus, wenn das ganze Bild dann noch einmal
    auf den Kopf gestellt wird, sieht es wieder fast normal aus,
    d.\,h. Augen und Mund sind wichtig bei Gesichtserkennung

    \item
    \textbf{Hohlmaske}:
    binokulares Sehen sagt, Objekt ist konkav,
    während Erfahrung sagt, Objekt ist konvex,
    Erfahrung überwiegt und wir nehmen das Objekt konvex wahr

    \item
    \textbf{\name{Ames}-Raum}:
    Größentäuschung, analog Mondillusion (Mond am Horizont)
\end{itemize}
\linie
\pagebreak
\begin{itemize}
    \item
    \textbf{mehrdeutige Bilder}:
    Gehirn wählt wahrscheinlichste Realität aus
    (Beispiele sind Mann im Mond, alte Frau/junge Frau, spinning dancer,
    Barber-Pole-Illusion, Foot-Step-Illusion, Ente/Kaninchen)

    \item
    \textbf{Necker-Würfel}:
    dass Würfel von genau schräg oben gemalt wurde, ist für das
    Gehirn unwahrscheinlich

    \item
    \textbf{Regeln beim Sehen}:
    sich im Bild tref"|fende Linien tref"|fen sich auch in der Realität,
    im Bild benachbarte Elemente sind auch in der Realität benachbartm,
    konvexe/konkave Abschnitte im Bild entsprechen konvexen/sattelförmigen
    Abschnitten/Objektteilen in der Realität
\end{itemize}
\linie
\begin{itemize}
    \item
    \textbf{Caféhaus-Illusion}:
    waagerechte Geraden (grauer Mörtel) zwischen versetzten hellen und dunklen
    Blöcken erscheinen schief, verschiedene Theorien,
    aber wahrscheinlich aufgrund der Kantendetektion
    (lokale Steigungen verleiten das Gehirn eine Gesamtsteigung festzustellen)

    \item
    \textbf{Pyramiden-Illusion}:
    gestapelte, kleiner werdende Quadrate mit aufsteigender Helligkeit,
    Diagonalen scheinen heller wie der Rest,
    Grund wieder Kantendetektion
\end{itemize}
\linie
\begin{itemize}
    \item
    \textbf{\name{Hermann}-Gitter}:
    an Kreuzungsstellen gibt es dunkle Punkte, die verschwinden, sobald man
    seinen Blick darauf wirft,
    Grund ist vielleicht die Kantendetektion,
    denn eine Erregung benachbarter Sehzellen führt zu einer Dämpfung
    der zentralen Sehzelle (laterale Inhibition), daher wirken die
    Kreuzungspunkte dunkel,
    im Bereich des gelben Flecks ist die Auflösung größer, daher funktioniert
    die laterale Inhibition nicht und der Fleck verschwindet,
    allerdings ist es nicht so einfach, da Größe des Felds variabel
    (wahrscheinlich auch Kantendetektion in höheren Verarbeitungsregionen)

    \item
    \textbf{Mach-Effekt}:
    Überschwinger an Kanten,
    z.\,B. vertikale graue Rechtecke von links nach rechts, die
    von Schwarz nach Weiß gehen,
    an den Kanten tritt ein leicht heller bzw. dunkler Rand auf,
    Grund wieder Kantenerkennung
    (Überschwinger bei Laplace-Filter)
\end{itemize}
\linie
\begin{itemize}
    \item
    \textbf{Bewegungsillusionen}:
    treten besonders leicht auf, weil der Mensch besonders stark auf
    Bewegungen reagiert (Gefahren, Futter usw.)

    \item
    \textbf{Enigma-Illusion}:
    scheinbare Rotation im Bild,
    keine plausible Erklärung vorhanden

    \item
    \textbf{Ouchi-Illusion}:
    zentrales Quadrat bewegt sich scheinbar gegenüber dem Rand,
    menschliches Sehsystem reagiert auf senkrechte Bewegungen zur
    Grundausrichtung eines Körpers stärker, da die unterschiedlichen
    Bereiche der Illusion unterschiedliche Ausrichtungen haben,
    scheint die Bewegung von beiden Teilen unterschiedlich zu sein
    (daher scheinbare relative Bewegung des inneren Teils)

    \item
    \textbf{Wasserfall-Ef"|fekt}:
    Starren auf Wasserfall (mehrere Minuten) bewirkt, dass bei Betrachten
    einer unbewegten Szene direkt danach die Szene nach oben zu laufen scheint,
    Hypothese, dass Nervenzellen erschöpfen

    \item
    \textbf{periphere Drift-Illusionen}:
    Kreise scheinen sich zu bewegen, es gibt keine überzeugende Erklärung

    \item
    \textbf{Chronostasis}:
    ruhende Uhr, bei Blick auf Uhr mit Sekundenzeiger scheint die erste
    Sekunde länger zu dauern als eine Sekunde (ca. $\SI{1.2}{\second}$),
    Modellbildung des Gehirns verantwortlich
    (während Blick zur Uhr wandert, wird die Netzhaut nicht ausgewertet bzw.
    Nervensignale werden unterdrückt, da nur unscharfes Bild vorhanden,
    Realität muss aber auch für diese Zeit existieren, daher wird sie
    mit dem nächsten empfangenen Signal gefüllt),
    analog Blinzeln
\end{itemize}
\linie
\pagebreak
\begin{itemize}
    \item
    \textbf{Helligkeitstäuschungen}:
    Nachbarschaft und Schatten beeinflussen Realitätskonstruktion des Gehirns,
    z.\,B. auch Schneelandschaft und weißer Himmel,
    Schnee erscheint heller

    \item
    \textbf{simultaner Helligkeitskontrast}:
    zwei graue Kreise, jeweils umgeben von einem schwarzen oder weißen
    Rechteck, erscheinen unterschiedlich hell

    \item
    \textbf{Muffin-Bleck}:
    Interpretation konkav/konvex aufgrund Erfahrung mit Beleuchtung \\
    (kommt meistens von oben: Himmel, Lampen usw.)
\end{itemize}
\linie
\begin{itemize}
    \item
    \textbf{Neon-Color-Illusion}:
    Gehirn sucht sich wahrscheinlichste Deutung heraus

    \item
    \textbf{Kaniza-Illusion}:
    analog mit Dreiecken
\end{itemize}
\linie
\begin{itemize}
    \item
    \textbf{Antigravitationshügel}:
    auch magnetische Hügel,
    Gegenstände (Autos im Leerlauf, Kugeln etc.) scheinen sich nach oben
    zu bewegen,
    Grund ist falsche Einschätzung der Gravitationsrichtung aufgrund
    "`schiefem"' Horizont (seltene Situation),
    Fluchtpunkt scheint überhalb der Straße zu liegen,
    es gibt auch noch andere Einflüsse, die die Einschätzung beeinflussen
\end{itemize}
\linie
\begin{itemize}
    \item
    \textbf{Perspektivenef"|fekte}:
    Längen werden falsch eingeschätzt aufgrund der Perspektive
    (Entfernung wird in Zusammenhang mit Objektgröße gebracht)

    \item
    \textbf{\name{Müller}-\name{Lyer}-Pfeiltäuschung}:
    Pfeile scheinen unterschiedlich lang zu sein,
    da perspektivische Darstellung Tiefe suggeriert

    \item
    \textbf{\name{Zöllner}-Illusion}:
    diagonale Linien mit horizontalen bzw. vertikalen Linien
    sehen nicht diagonal aus,
    spitze Winkel werden als weniger spitz wahrgenommen als sie im Bild sind,
    analog stumpfe Winkel,
    Grund ist die Gewöhnung an rechte Winkel
    (diese sind in zweidimensionalen Bildern aufgrund der Perspektive
    nämlich oft spitz oder stumpf)

    \item
    \textbf{\name{Hering}-Illusion}:
    zwei horizontale Linien, von Mittelpunkt ausgehend viele Strahlen,
    horizontale Linien scheinen gebogen zu sein,
    Erklärung analog mit obiger Regel

    \item
    \textbf{\name{Poggendorf}-Illusion}:
    Fortsetzung einer Linie durch Unterbrechung (Rechteck) wird falsch
    eingeschätzt
\end{itemize}

\pagebreak
