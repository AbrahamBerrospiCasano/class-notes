\section{%
    Polarisation und Moiré%
}

\begin{itemize}
    \item
    \textbf{Polarisation}:
    Licht ist transversale Welle, d.\,h. Schwingungsebenen stehen senkrecht zur
    Ausbreitungsrichtung des Lichts,
    Polarisation = Schwingungsebene des elektrischen Felds,
    Polarisationen lassen sich zerlegen in eine gewichtete Summe
    (Linearität der elektromagnetischen Wellen)
    
    \item
    \textbf{lineare Polarisation}:
    Schwingung in einer konstanten Ebene, kann vernachlässigt werden
    
    \item
    \textbf{zirkulare Polarisation}:
    andere Lösungen der Maxwell-Gleichungen,
    Rotation der momentanen Schwingungsebene des elektrischen Felds
    mit Lichtfrequenz
\end{itemize}
\linie
\begin{itemize}
    \item
    \textbf{unpolarisiertes Licht}:
    die Photonen des Lichts haben eine "`zufällige"' Polarisation,
    bei den meisten natürlichen und künstlichen Lichtquellen der Fall
    (Sonne, Glühbirne)
    
    \item
    \textbf{Polarisation durch Reflexion}:
    Fresnel-Gleichungen besagen, dass Reflexionsfaktor\\
    (z.\,B. an Glasscheibe)
    von Einfallswinkel, den Brechzahlen und der Polarisation abhängen,
    also ist im reflektierten Licht eine der beiden Polarisationsrichtungen
    stärker vertreten
    
    \item
    \textbf{Polarisationsfilter}:
    können Reflexionen vermindern,
    Anwendung Sonnenbrillen (Wassersportler), Bildaufnahmen vom Himmel,
    um Himmel abzudunkel, Kontrast zu Wolken größer (Licht des Himmels ist
    teilpolarisiert)
    
    \item
    \textbf{Polarisationsfolien}:
    Absorption von Lichtwellen durch Stromfluss,
    analog auch Drahtgitterpolarisator für Mikrowellen,
    elektrisches Feld bewirkt Wechselstrom im Gitter, dieses erzeugt ein
    Gegenfeld, das genau stark wie das einfallende Feld ist und dieses
    auslöscht,
    leider ist Licht zu kurzwellig für Gitterpolarisatoren,
    Edwin Land entwickelte 1938 aber eine preiswerte Herstellungsmethode
    für Gitter mit sehr kleinem Abstand, \\
    Anwendung in praktisch allen LCDs (Flüssigkristalldisplays),
    z.\,B. Armbanduhr, Taschenrechner, Laptop
    
    \item
    \textbf{gekreuzte Polarisatoren}:
    führen zu einer kompletten Auslöschung des Lichts,
    ein zusätzlicher (schiefer) Zwischenpolarisator führt dazu,
    dass ein Teil des Lichts die Anordnung passiert
\end{itemize}
\linie
\begin{itemize}
    \item
    \textbf{Polarisation des Himmels}:
    maximal zu ca. $\SI{75}{\percent}$ polarisiert
    (wenn Sonne tief steht und keine Bewölkung vorhanden ist),
    Grund: Rayleigh-Streuung an Luftmolekülen erhält die Schwingungsebene
    des elektromagnetischen Felds,
    Polarisation variiert mit Sonnenposition, beobachtetem Gebiet und
    Partikel in Atmosphäre,
    Anwendung in Fotografie (Verstärkung des Kontrasts Himmel -- Wolken),
    da Wolken aufgrund der Mehrfachstreuung stark depolarisieren
\end{itemize}
\linie
\pagebreak
\begin{itemize}
    \item
    \textbf{Doppelbrechung}:
    ein senkrecht auf die Grenzfläche eines doppelbrechenden Materials
    einfallender Lichtstrahl wird teilweise gebrochen (widerspricht dem
    Brechungsgesetz), d.\,h. für den außerordentlichen Strahl ergibt sich
    ein Brechungswinkel ungleich Null, die beiden Strahlen sind unterschiedlich
    polarisiert,
    Grundlage ist, dass sich unterschiedlich polarisiertes Licht in
    doppelbrechenden Medien (z.\,B. Kalkspat, Quarz) unterschiedlich schnell
    fortpflanzt und daher unterschiedliche Brechungsindizes gelten
    
    \item
    \textbf{Farbef"|fekte aufgrund Doppelbrechung}:
    laufen die Lichtwellen danach durch einen Polarisator, passieren zwar nur
    Teile der beiden Wellen den Polarisator, diese sind dann aber gleich
    polarisiert, sodass sie interferieren können,
    lokale Dickenschwankungen oder Änderungen der Geometrie führen zu
    unterschiedlichen Farben
    (z.\,B. Zuckerschicht auf Glas, Cockpitfenster, Plexiglasbox)
    
    \item
    \textbf{Anwendungen}:
    Spannungsdoppelbrechung (mechanische Spannungen sichtbar machen),
    Feststellen von Konstruktionsfehlern
\end{itemize}
\linie
\begin{itemize}
    \item
    \textbf{\name{Haidinger}-Schmetterling}:
    Auge kann (direkt) eigentlich keine Polarisation feststellen,
    mit ein wenig Übung ist das aber tatsächlich möglich,
    polarisiertes Licht zeigt sich dabei als eine leichte
    (kaum wahrnehmbare) schmetterlingsartige Figur,
    vier Kreise, zwei gelbe und zwei blau, wobei die gelben Teile in
    die Richtung senkrecht zur Polarisation zeigen,
    Ausdehnung von ca. $4^\circ$,
    Erklärung nicht vollständig geklärt,
    aber Lutein (ein Pigment im gelben Fleck) ist doppelbrechend und
    absorbiert Licht mit einer Polarisationsrichtung parallel zur Molekülachse
    besonders stark, aufgrund der kreisförmigen Anordnung der Moleküle
    auf der Netzhaut ergibt sich auf unterschiedlichen Bereichen der Netzhaut
    eine unterschiedlich starke Abschwächung des Lichts
    
    \item
    \textbf{Polarisation bei Tieren}:
    manche Tiere (Bienen, einige Käfer und Spinnen) können Polarisation sehen
    und benutzen sie zur Orientierung
    
    \item
    \textbf{3D-Kino}:
    Polarisationsfilter in den Brillengläsern sorgen dafür, dass jedes Auge
    das richtige Bild empfängt
    
    \item
    \textbf{Dichroismus}:
    unterschiedliche Absorption für unterschiedliche Polarisation
    (Turmalin)
    
    \item
    \textbf{Zuckerlösung}:
    rotiert die Schwingungsebene von einfallendem, linear polarisiertem Licht
\end{itemize}
\linie
\begin{itemize}
    \item
    \textbf{Moiré}:
    Überlagerung (multiplikativ oder additiv) zweier Gitter mit leicht
    anderer Frequenz
    (z.\,B. auch unterschiedliche Entfernung zum Beobachter),
    sowohl orts- als auch zeitbezogen möglich,
    bspw. wenn Räder eines Autos/Rotor eines Helikopters still zu stehen oder
    sogar rückwärts zu drehen scheinen,
    Farbmaske eines Digitalfotos kann zu farbigen Moirés führen,
    analog bei Scans
\end{itemize}

\pagebreak
