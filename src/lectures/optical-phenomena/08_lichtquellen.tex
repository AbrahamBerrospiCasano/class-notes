\chapter{%
    Lichtquellen%
}

\begin{itemize}
    \item
    \textbf{Lichtemission}:
    Atome und Moleküle senden Licht aus, indem Elektronen von einem energetisch
    hohen Niveau in ein niedrigeres Niveau fallen, die dabei frei werdende
    Energie, also die Energiedif"|ferenz der Niveaus, kann in Form eines
    Photons abgegeben werden,
    Energie muss aber vorher zugeführt werden
    (z.\,B. chemische Reaktion, Stromfluss oder Wärme)

    \item
    \textbf{Lumineszenz}:
    Erzeugung von Licht, ohne dass Materie erhitzt wird oder brennt

    \item
    \textbf{Möglichkeiten der Lichtaussendung}: \\
    thermisch (Sonne, Glühbirne, Blitze, Kerze), \\
    Stromfluss (LED, Neonröhre), \\
    chemisch (Chemolumineszenz, Biolumineszenz), \\
    optisch (Fluoreszenz, Phosphoreszenz), \\
    divers (magnetisch bei Polarlicht, akustisch bei Sonolumineszenz,
    radioaktiv, mechanisch bei Reibung, Mikrowellen)

    \item
    \textbf{Lumen}:
    Leistungsangabe in $\si{\watt}$ (wie bei Glühbirnen) ist für den Beobachter
    uninteressant, da viel Licht im Infraroten abgestrahlt wird,
    außerdem ist das Auge für unterschiedliche Wellenlängen unterschiedlich
    empfindlich (bei $\SI{550}{\nano\meter}$ ist das Auge am empfindlichsten),
    stattdessen integriert man,
    $\int S(\lambda) L(\lambda) d\lambda$,
    wobei $S(\lambda)$ die Empfindlichkeit und $L(\lambda)$ die Leistung
    bei Wellenlänge $\lambda$ ist,
    teilt man dann noch durch die Leistung, erhält man ein Maß für die
    Ef"|fizienz (Einheit $\si{\lumen\per\watt}$), z.\,B. hätte eine ideale
    Lampe $\SI{680}{\lumen\per\watt}$ (Glühlampe $\SI{15}{\lumen\per\watt}$,
    LEDs $\SI{50}{\lumen\per\watt}$)
\end{itemize}
\linie
\begin{itemize}
    \item
    \textbf{thermische Lichtquellen}:
    Sonne, Blitz, Feuer

    \item
    \textbf{schwarzer/\name{Planck}scher Strahler}:
    kleines Loch in einem beheizten Körper,
    absorbiert einfallende elektromagnetische Strahlung bei jeder Wellenlänge
    vollständig

    \item
    \textbf{\name{Planck}sches Strahlungsgesetz}:
    Gesetz für die Abstrahlung (Strahldichte $L$ in $\si{\watt}$ in Abhängigkeit
    von $\lambda$ und $T$) thermischer Quellen

    \item
    \textbf{\name{Stefan}-\name{Boltzmann}-Gesetz}:
    gesamte abgestrahlte Leistung in Abhängigkeit von der Temperatur $T$,
    $P(T) \sim T^4$

    \item
    \textbf{\name{Wien}sches Verschiebungsgesetz}:
    für das Maximum der Abstrahlung gilt \\
    $\lambda_{\text{max}} T = \SI{2898}{\micro\meter\kelvin}$

    \item
    \textbf{Sonne}:
    wichtigste Licht-/Energiequelle,
    im Inneren $15$ Millionen Grad heiß, außen $\SI{5800}{\kelvin}$

    \item
    \textbf{Kerzen}:
    Stearin der Kerze wird durch die Hitze geschmolzen, wandert im Docht
    durch Kapillarkräfte nach oben und wird verdampft

    \item
    \textbf{Gasbrenner}:
    höhere Temperaturen durch Verbrennung von Mischung aus Sauerstoff und
    Brennstoff

    \item
    \textbf{Glühbirne}:
    dünner Draht (Wolfram ist geeignet) wird durch Stromfluss erwärmt

    \item
    \textbf{Halogenlampe}:
    funktioniert wie Glühbirne, nur werden dem Füllgas Halogene beigemischt,
    damit Lebensdauer und Ef"|fizienz verbessert werden,
    höhere Temperatur der Glühwendelt ergibt natürlicheres Licht
    (Spektrum liegt näher beim Sonnenlicht)

    \item
    \textbf{Feuerwerkskörper}:
    Verbrennung von Mangan zu Manganoxid,
    Farbe durch Zusätze (Nitrate)
\end{itemize}
\pagebreak
\begin{itemize}
    \item
    \textbf{Blitze}:
    besonders im Sommer führt starke Sonneneinstrahlung dazu, dass
    feuchtwarme Luft nach oben steigt und kondensiert,
    in den entstehenden Wolken gibt es starke Aufwinde, sodass Regentropfen
    nach oben getragen werden und gefrieren, durch Reibungsprozesse erfolgt
    Ladungstrennung, Wolke wird negativ geladen, Boden hat positive
    Spiegelladung,
    Spannungen von bis zu $\SI{200000}{\volt}$ reichen allerdings nicht für
    Überbrückung der Luft aus ($\SI{2.5}{\mega\volt\per\meter}$
    Durchbruchfeldstärke),
    teilweise noch ungeklärt, \\
    zuerst bildet langsamer Leitblitz einen $\SI{1}{\meter}$ breiten Kanal
    mit ionisierter Luft, also hoher Leitfähigkeit, dann bewirken mehrere
    schnelle Hauptblitze den eigentlichen Ladungsausgleich,
    $\SI{30000}{\celsius}$, $\SI{100}{\bar}$,
    hohe Temperatur führt zu einer starken Stoßionisation in der Luft,
    die Anregung von Elektronen bewirkt, die ihre Energie teilweise in Form
    von Licht abgeben (nur $\SI{0.1}{\percent}$ ist Licht und Schall),
    Blitz flackert wegen den mehreren Hauptblitzen

    \item
    \textbf{Erdblitze/Wolkenblitze}:
    laufen von Erde zu Wolke/finden zwischen Wolken statt

    \item
    \textbf{Linienblitze/Flächenblitze}:
    ohne Verästelung/von Hauptast gehen fein verzweigte\\
    Blitzbahnen ab

    \item
    \textbf{Länge}:
    Vertikalblitze meistens $\SI{5}{\kilo\meter}$ bis $\SI{7}{\kilo\meter}$,
    Horizontalblitze können auch dutzende Kilometer überbrücken
    (Blitz aus heiterem Himmel)

    \item
    \textbf{Blitz als Energiequelle}:
    nicht sinnvoll, da nur $\SI{25}{\kilo\watt\hour}$ pro Blitz nutzbar
\end{itemize}
\linie
\begin{itemize}
    \item
    \textbf{Thermolumineszenz}:
    durch Erwärmung wird Licht ausgesendet,
    darf nicht mit thermischen Lichtquellen verwechselt werden,
    da bei Thermolumineszenz bereits gespeicherte Energie (durch das Erhitzen)
    als Licht freigesetzt wird

    \item
    \textbf{radioaktive Lichterzegung}:
    radioaktive Strahlung wird durch Lumineszenz in sichtbares Licht
    umgewandelt,
    auch ewiges Licht wegen der hohen Halbwertszeiten

    \item
    \textbf{Sonolumineszenz}:
    Ultraschallanregung einer Flüssigkeit führt zur Bildung mikroskopischer
    Blasen, die anschließend extrem schnell kollabieren,
    hohe frei werdende Leistung führt zur Photonenaussendung

    \item
    \textbf{Tribolumineszenz}:
    (teilweise) Umwandlung von Reibungsenergie zu Licht
    (z.\,B. schnelles Abrollen von Klebebändern erzeugt Röntgenstrahlen)
\end{itemize}
\linie
\begin{itemize}
    \item
    \textbf{Leuchtstoff"|lampen}:
    umgangssprachlich Neonröhren, auch Gasentladungslampen,
    ein von Strom durchflossenes Gas (z.\,B. Quecksilberdampf) in einer
    teilevakuierten Röhre gibt dabei (in der Regel ultraviolette)
    Strahlung ab,
    die durch eines fluoreszierende Leuchtschicht in sichtbares Licht
    umgewandelt wird,
    verschiedene Beschichtungen führen zu verschiedenen Farben,
    durch Glühemission erzeugte Elektronen werden beschleunigt und
    regen Atome an,
    Vorteile sind hohe Ef"|fizienz, lange Lebensdauer und variable
    Farbtemperatur,
    Nachteil Gefährlichkeit (Lücke in Beschichtung, oder Loch in Röhre)

    \item
    \textbf{Stromsparlampen}:
    sind eigentlich gewöhnliche Leuchtstoff"|lampen, nur mit gebogenen
    Röhren zur Verringern der Größe

    \item
    \textbf{Straßenlaternen}:
    ebenfalls meistens Entladungslampen,
    oft Natriumdampf"|lampen, die rosarot und dann gelb-orange leuchten
\end{itemize}
\linie
\pagebreak
\begin{itemize}
    \item
    \textbf{Fluoreszenz/Phosphoreszenz}:
    Anregung von Elektronen in Atomen durch Licht, \\
    die Energie (und damit Frequenz) der abgestrahlten Photonen
    kann geringer sein,
    wenn Elektronen über Zwischenniveaus zurückfallen, \\
    Unterscheidung anhand der Zeitdauer Anregung -- Reemission,
    für Zeiten kleiner $\SI{e-4}{s}$ spricht man von
    Fluoreszenz, ansonsten Phosphoreszenz (bis zu mehreren Stunden)

    \item
    \textbf{Fluoreszenz}:
    Leuchtfarben (z.\,B. in Rettungswesten, Textmarkern) erhalten ihre
    Leuchtkraft durch Fluoreszenz,
    kurzwelliges (blaues und ultraviolettes) Licht wird in orangefarbiges
    Licht konvertiert, d.\,h. mehr Licht im gelb-roten Spektralbereich wird
    emittiert als überhaupt im Anregungslicht vorhanden war, also leuchtet
    die Weste wesentlich heller als bspw. eine orangefarbig lackierte Fläche,
    analog Weißmacher in Waschmittel (UV-Licht wird umgewandelt)
    sowie Geldscheine

    \item
    \textbf{Fluoreszenz und Totalreflexion}:
    z.\,B. Eislöffel, transparente Stühle und CD-Hüllen
    (Kunststof"|fe mit fluoreszierenden Farbstof"|fen),
    Licht wird im Kunststoff unter beliebigen Winkel reemittiert,
    u.\,U. Totalreflexion, Licht ist gefangen (analog Glasfaser) und
    kann erst an einer Kante das Material verlassen, d.\,h. Kante leuchtet
\end{itemize}
\linie
\begin{itemize}
    \item
    \textbf{Chemolumineszenz}:
    chemische Reaktion sorgt für die Anregung der Elektronen, z.\,B.
    Leuchtstäbe zur Absicherung von Unfallstellen
    (zwei Chemikalien sind getrennt und kommen beim Knicken zusammen),
    Bezeichnung "`kaltes Licht"', da kaum Wärme entsteht, hoher Wirkungsgrad

    \item
    \textbf{Biolumineszenz}:
    Chemolumineszenz von Lebewesen, z.\,B. durch Glühwürmchen, Pilze,
    Tiefseebewohner, dient der Tarnung, Anlocken von Beute,
    Blendung von Angreifern und Kommunikation
\end{itemize}
\linie
\begin{itemize}
    \item
    \textbf{LEDs}:
    lichtemittierende Dioden,
    basiert auf einem pn-Übergang (Elektrolumineszenz),
    Vorteile robust, preiswert, lange Lebensdauer, hohe Ef"|fizienz und klein,
    Energiekonversion hat Ef"|fizienz von fast $\SI{100}{\percent}$, aber
    nicht alle Photonen können verlustfrei nach außen gelangen, daher
    Gesamtef"|fizienz von $\SI{30}{\percent}$ (immer noch hoch)

    \item
    \textbf{weiße LEDs}:
    abgestrahltes Licht ist sehr schmalbandig
    ($\SI{10}{\nano\meter}$ -- $\SI{50}{\nano\meter}$ Halbwertsbreite),
    für weiße LEDs gibt es verschiedene Möglichkeiten,
    am verbreiteten ist die Nutzung einer fluoreszierenden Konversionsschicht,
    die das kurzwellige Licht der LED in ein breitbandiges Fluoreszenzlicht
    umwandelt, z.\,B. blaue LED und gelbe Konversionsschicht
    (Blau + Gelb = Weiß, dichromatische Farbmischung),
    Nachteil schlechtere Energieef"|fizienz (Photonen gehen verloren),
    alternativ mehrere, unterschiedlich farbige LEDs
\end{itemize}
\linie
\begin{itemize}
    \item
    \textbf{Polarlichter}:
    Sonnenwind von der Sonne zur Ende
    (Sonne verliert $\SI{1}{\tera\gram\per\second}$ an Masse),
    geladene Teilchen tref"|fen auf Erdmagnetfeld, werden abgelenkt,
    treten in Atmosphäre ein und regen Sauerstoff- (grünes Licht) und
    Stickstoffmoleküle (rotes Licht) an,
    teilweise auch in niederen Breiten und auf anderen Planeten
\end{itemize}
\linie
\begin{itemize}
    \item
    \textbf{Kugelblitz}:
    Existenz ungesichert,
    viele Berichte, aber Häufigkeit sehr gering

    \item
    \textbf{Elmsfeuer}:
    bläuliches Leuchten an Kirchturmsspitzen, Schiffsmasten oder anderen
    metallischen Spitzen (Zäune),
    Grund lokale Entladung in starken elektrischen Feldern,
    hohe Wahrscheinlichkeit für Blitzeinschlag,
    widerspricht der Legende, dass Leuchten den Seefahren anzeigt, dass der
    Sturm nahezu überstanden ist

    \item
    \textbf{Irrlichter}:
    Existenz ebenfalls ungesichert,
    kleine, umherwandernde Flämmchen auf moorigem Boden,
    Annahme, dass die Ursache für die Flammen in einer Selbstentzündung von
    aus dem Boden austretenden Gas (Methan mit Phosphin) liegt
\end{itemize}

\pagebreak
