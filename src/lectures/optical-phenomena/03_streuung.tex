\section{%
    Streuung%
}

\begin{itemize}
    \item
    \textbf{Streuung}:
    Vorgang, bei dem Licht auf unterschiedliche Richtungen mehr oder weniger
    zufällig umverteilt wird,
    Lichtstrahl trifft auf kleine Teilchen/Oberfläce mit sehr feiner Struktur
    (Unebenheiten im Wellenlängenbereich) --
    Streuung erfolgt immer in der Ebene senkrecht zur Riefenrichtung, d.\,h.
    ein nur in einer Richtung geschlif"|fenes Werkstück zeigt eine starke
    Streuung senkrecht zur Bearbeitungsrichtung
    
    \item
    \textbf{\name{Rayleigh}-Streuung}:
    tritt bei Teilchen mit Durchmesser $< \lambda$ auf,
    Streu-Wahrschein"-lichkeit $p(\lambda)$ ist stark wellenlängenabhängig,
    es gilt $p(\lambda) \sim 1/\lambda^4 \sim \nu^4$,
    z.\,B. verantwortlich für Farbe des Sonnenuntergangs oder des
    blauen Himmels
    
    \item
    \textbf{\name{Mie}-Streuung}:
    tritt bei Teilchen mit Durchmesser $\ge \lambda$ auf,
    Streu-Wahrscheinlichkeit $p(\lambda)$ konstant, aber dafür
    richtungsabhängig:
    am meisten wird Licht nach hinten gestreut (in Lichtrichtung),
    am wenigsten nach vorne (entgegen Lichtrichtung),
    z.\,B. verantwortlich für Farbe der Wolken
\end{itemize}
\linie
\begin{itemize}
    \item
    \textbf{\name{Rayleigh}-Streuung an Atomen/Molekülen der Luft}:
    Licht regt in den Molekülen der Luft einen Dipol an
    (Ladungstrennung durch elektrisches Feld), Hertzscher Dipol wird zum
    Schwingen gebracht und strahlt Energie ab
    
    \item
    \textbf{Dimensionsanalyse}:
    wurde von Rayleigh 1871 durchgeführt, leitet
    $p(\lambda) \sim 1/\lambda^4$ her
\end{itemize}
\linie
\begin{itemize}
    \item
    \textbf{warum ist der Himmel blau}:
    Rayleigh-Streuung, da Luftmoleküle einen deutlich kleineren Durchmesser
    als $\lambda$ haben, Wellenlängenabhängigkeit führt dazu, dass Blau
    viel wahrscheinlicher gestreut wird als Rot
    
    \item
    \textbf{von Bergen erscheint der Himmel besonders blau}:
    weniger Luftmoleküle vorhanden, also weniger Streuung, Himmel wird dunkler
    (evtl. sogar Sterne am Mittag)
    
    \item
    \textbf{Abendrot}:
    Sonne steht tief, Licht muss durch viel mehr Luft, Blau wird eher
    weggestreut, rotes Licht bleibt übrig,
    besonders tiefes Rot bei hohen Anzahl an Partikel in Luft
    (Wassertröpfchen, Aerosole, Staub),
    analog tiefstehender Mond
    
    \item
    \textbf{gelbe Wolken bei Hagel}:
    Sonne steht tief und schweres Unwetter, Vermutung, dass roter Teil des
    Sonnenlichts (andere Anteile gibt es kaum) durch viel Wasser und Eis
    absorbiert wird, somit verbleiben nur mittlere Spektralanteile
    (Gelb/Grün), aber Phänomen ist nicht vollständig geklärt
    
    \item
    \textbf{Mars-Verschwörung}:
    Warum ist der Himmel auf dem Mars nicht ebenfalls blau?
    viele Eisenoxid-Partikel, Absorption führt zu einer rötlichen Färbung
    
    \item
    \textbf{Airlight}:
    Überlagerung von Streulicht mit dem eigentlichen
    (von einem Objekt ausgehenden) Licht,
    z.\,B. erscheinen weit entfernte Berge leicht bläulich, da viel Luft
    zwischen Beobachter und Berg,
    bei Sonnenuntergang kann man weiter sehen wie tagsüber
    (weniger Luft streut, Airlight geringer)
    
    \item
    \textbf{once in a blue moon}:
    Streuwahrscheinlichkeit wird für eine bestimmte Wellenlänge besonders
    stark, selektive Streuung
    
    \item
    \textbf{Alpenglühen}:
    reflektiertes Licht des Sonnenuntergangs kann zu einer rötlichen
    Färbung von Bergen führen
\end{itemize}
\linie
\pagebreak
\begin{itemize}
    \item
    \textbf{blaue Augen bei Babys}:
    noch keine Pigmente vorhanden, blaue Färbung kommt durch Rayleigh-Streuung
    zustande
    
    \item
    \textbf{blauer Dunst}:
    Rauch, der aus nicht zu großen Partikeln besteht und nicht zu dicht,
    erscheint leicht bläulich (z.\,B. blauer Dunst),
    ausgeatmeter Rauch enthält allerdings kleine Wassertröpfchen,
    die wellenlängenunabhängige Mie-Streuung verursachen, d.\,h. der Rauch
    erscheint dann weiß
\end{itemize}
\linie
\begin{itemize}
    \item
    \textbf{Wolken sind weiß}:
    kein Gegenstand, sondern ein Zustand, Ansammlung von kleinen
    Wassertröpfchen (Durchmesser 2 bis $\SI{100}{\micro\meter}$), daher
    wird Licht wellenlängenunabhängig gestreut (Mie-Streuung),
    weiß aufgrund Vielfachstreuung, rot wird zwar weniger häufig gestreut
    (Wasser), aber aufgrund der hohen Teilchenzahl wirkt sich das kaum auf
    die Gesamtstreuung aus (Absorption ist vernachlässigbar)
    
    \item
    \textbf{mächtige Wolken sind an der Unterseite schwarz}:
    Grund liegt in der hohen Dichte an der Unterseite, weniger Licht kommt
    durch
    
    \item
    \textbf{Cyphochilus}:
    kleiner, auf weißen Pilsen lebender Käfer nutzt Vielfachstreuung, um sich
    weiß zu tarnen (sehr kleine Fasern, $< \lambda$, Rayleigh-Streuung, aber
    Fasern sind so dicht, dass Käfer weiß erscheint)
    
    \item
    \textbf{farbige Wolken}:
    Beleuchtung mit farbigem Licht, z.\,B. rote Wolken bei Sonnenuntergang
    
    \item
    \textbf{leuchtende Nachtwolken}:
    Breite größer $55^\circ$ Nord oder Süd, bevorzugt im Sommer
    ein bis zwei Stunden nach Sonnenuntergang,
    hell leuchtende Wolken vor dem Nachthimmel nahe beim Horizont,
    geht nur, wenn Wolken sehr hoch ($\SI{80}{\kilo\meter}$ bis $\SI{90}{\kilo\meter}$) sind
    (Grund: Beleuchtung durch bereits untergegangene Sonne)
    
    \item
    \textbf{Helligkeit des Himmels}:
    der Himmel erscheint in der Nähe des Horizonts weiß,
    durch Mehrfachstreuung (Licht geht durch viel Luft) kommen in der Summe
    alle Farbanteile gleich an (blaue Farbanteile werden häufiger aus der
    "`korrekten"' Richtung weggestreut)
    
    \item
    \textbf{Reflexion einer Landschaft im See}:
    alle Farben sehen gleich aus,
    bloß der reflektierte Himmel sieht bläulicher aus
    (See sieht einen anderen/höheren Teil des Himmels wegen anderem Winkel)
    
    \item
    \textbf{Gegendämmerung}:
    entgegen der Sonne sieht man beim Sonnenuntergang (oder -auf"-gang)
    den Schatten der Erde als dunkles Band,
    damit dieser sichtbar wird, muss der Schatten auf Berg oder Atmosphäre
    fallen
    
    \item
    \textbf{Venusgürtel}:
    in Richtung der Gegendämmerung ist ein rosa-violetter Streifen am Horizont
    sichtbar, Grund liegt in der Überlagerung zweier Lichtanteile,
    zum einen starke Rückstreuung durch Sonnenuntergangslicht,
    zum anderen blaues Licht aus anderem Himmelsbereichen durch zweifache
    Streuung
    
    \item
    \textbf{Purpurlicht}:
    ähnlich wie Venusgürtel, nur in Richtung der Sonne,
    Überlagerung von blauen Photonen aus oberen Atmosphärensichten
    (Licht muss nicht durch viel Luft) mit roten Photonen aus tieferen
    Bereichen, vor allem sichtbar, wenn viel Staub in der Atmosphäre vorhanden
    ist (Vulkanausbrüche, Waldbrände oder Nähe zu Großstadt)
\end{itemize}

\pagebreak
