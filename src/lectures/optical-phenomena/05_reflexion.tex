\chapter{%
    Reflexion%
}

\enlargethispage{10mm}

\begin{itemize}
    \item
    \textbf{Reflexionsgesetz}:
    $\alpha = \alpha'$ (Einfallswinkel gleich Ausfallswinkel),
    lässt sich aus dem Fermatschen Prinzip herleiten

    \item
    \textbf{Spiegelung ist kompliziert}:
    bei Spiegelungen im Raum kann man nicht einfach das zweidimensionale
    Bild spiegeln,
    sondern man muss zunächst eine (mathematische) Spiegelung an der
    Spiegelebene (z.\,B. See) durchführen, die gespiegelten Objekte haben
    dann wieder eindeutige Position im dreidimensionalen Raum,
    dann zweidimensionales Abbild auf Netzhaut

    \item
    \textbf{Spiegel vertauscht nicht links und rechts}:
    rein psychologischer Ef"|fekt, liegt daran, dass man sich in die
    gespiegelte Position hinein versetzt

    \item
    \textbf{vom Spiegel weggehen hilft nichts}:
    bei senkrecht zum Boden aufgehängten Spiegeln sieht man immer gleich viel
    (z.\,B. gerade noch sein Knie), egal, wie weit man vom Spiegel entfernt ist

    \item
    \textbf{Spiegel als Signalgeber}:
    z.\,B. Optiker sendet mit Spiegel von einsamer Insel SOS,
    wäre Sonne punktförmig, dann könnte man sehr weit Signale senden
    (wegen Absorption und Streuung in Atmosphäre nicht unendlich weit),
    aber aufgrund der Ausdehnung der Sonne nimmt Bildgröße linear mit der
    Entfernung zu (Lochkamera-Prinzip), d.\,h. Bestrahlungsstärke einer
    Fläche nimmt quadratisch ab, also kann man nur über kurze Entfernungen
    Signale senden (mit quadratischem Spiegel mit $\SI{5}{\centi\meter}$ Seitenlänge
    über ca. $\SI{13}{\kilo\meter}$ weit)
\end{itemize}
\linie
\begin{itemize}
    \item
    \textbf{gekrümmte Spiegel, Kugelspiegel}:
    mit gekrümmten Spiegeln kann Licht fokussiert werden,
    ein Kugelspiegel hat als Brennweite den halben Krümmungsradius

    \item
    \textbf{spiegelnde Kugeln}:
    zeigen gesamten Raum mit Ausnahme des kleinen Bereichs hinter der Kugel
    (z.\,B. Christbaumkugeln), Raum wird verzerrt, am Rand unendlich starke
    Verzerrung, aber theoretisch komplette Information enthalten,
    Seifenblasen sind gleichzeitig Konvex- und Konkavspiegel

    \item
    \textbf{objects in mirrors are closer than they appear}:
    eigentlich müsste das Bild näher beim Betrachter liegen wie das
    Originalobjekt, Widerspruch löst sich auf, wenn man bedenkt, dass Gehirn
    Entfernung aus der Bildgröße bestimmt, Bildgröße ist hier reduziert
    (Abbildungsmaßstab kleiner $1$), daher die Warnung

    \item
    \textbf{Deflektometrie}:
    aus der Verzerrung des Bildes eines bekannten Objekts kann auf die Form
    des spiegelnden Elements zurückgeschlossen werden, z.\,B.
    Hochhaus in Fensterfassade oder Wasseroberfläche
    (dann können die Wellen berechnet werden)

    \item
    \textbf{Lichtkreuze}:
    aufgrund deformierten Fenstern, Druck auf Außenpunkte,
    Zylinderlinsen entstehen
\end{itemize}
\linie
\begin{itemize}
    \item
    \textbf{\name{Fresnel}-Gleichungen}:
    an einer Grenzfläche ergibt sich Brechung und Reflektion,
    d.\,h. nur ein Teil wird gebrochen, das restliche Licht wird reflektiert,
    Anteile können mit den Fresnel-Gleichungen bestimmt werden

    \item
    \textbf{von außen durch Fenster schauen}:
    bei Tag schlecht möglich, da es innen dunkler ist als außen,
    starke Reflektion verringert den Kontrast und verhindert das Hineinsehen,
    bei Nacht ist es innen heller als außen, kaum Reflektion, hoher Kontrast,
    man kann gut hineinsehen,
    analoger Ef"|fekt bei Gardinen

    \item
    \textbf{halbdurchlässiger Spiegel}:
    lässt in beide Richtungen gleich viel Licht durch, aber im hellen Raum
    kann man schlecht in den dunklen Raum hineinsehen
\end{itemize}
\linie
\pagebreak
\begin{itemize}
    \item
    \textbf{Reflexion und Wellen}:
    am Horizont ist Meer dunkler, wenn Wellen vorhanden sind
    (ein Teil des Lichts wird durch die Wellenberge am Horizont abgeschattet)

    \item
    \textbf{verschmiertes Bild}:
    in der Spiegelung einer Brücke kann man senkrechte Pfeiler gut sehen, aber
    diagonale Streben kaum, Grund liegt in der unebenen Wasseroberfläche,
    vertikale Verschmierung ändert kaum etwas am senkrechten Pfeiler,
    aber verschmiert diagonale Streben mit dem Hintergrund,
    auch schön sichtbar von tiefstehender Sonne/Mond
    (Verschmierung wird stärker, wenn Objekt tief steht)
\end{itemize}
\linie
\begin{itemize}
    \item
    \textbf{nass = dunkel}:
    sobald beliebiges Material nass wird (mit beliebiger Flüssigkeit),
    erscheint es dunkler, Grund liegt in der Totalreflektion eines
    Teil des von der Oberfläche gestreuten Lichts im dünnen Wasserfilm
    oberhalb des Materials, reflektiertes Licht hat erneut die Chance,
    von der Oberfläche absorbiert zu werden
    (beim Streuung treten beliebige Winkel auf)

    \item
    \textbf{dünne helle Stof"|fe}:
    wird auch dunkler, da Transmissionswahrscheinlichkeit erhöht wird
    (Absorption ist zu vernachlässigen),
    analog werden dünne weiße Kleidungsstücke durchsichtig, wenn sie nass
    sind (in Durchsicht erscheinen sie heller)
\end{itemize}
\linie
\begin{itemize}
    \item
    \textbf{Halos}:
    für Halos sind viele Eiskristalle in der Luft erforderlich,
    hexagonale Kristalle, vielfältige Formen, auch im Sommer möglich
    (in großen Höhen ist die Temperatur unter dem Gefrierpunkt)

    \item
    \textbf{Halo-Beobachtung}:
    regelmäßig Himmel inspizieren, Himmel in Reflexion beobachten, z.\,B. in
    Gartenkugel (stärkere Helligkeitsgradienten), Sonnenbrille benutzen
    (Abdunklung)

    \item
    \textbf{Lichtsäule (sun pillar)}:
    vertikale helle Linie oberhalb der tiefstehenden Sonne
    (Eiskristalle haben eher horizontale Ausrichtung, sind aber leicht
    gekippt, reflektieren Sonnenlicht an ihren Endflächen),
    geht auch mit künstlichen Lichtquellen (Straßenlaternen) und
    Lichtsäule nach unten

    \item
    \textbf{Nebensonnen (sun dogs)}:
    entstehen durch Brechung in hexagonalen Eiskristallen,
    die $120^\circ$-Innenwinkel führen zu einer (gehäuften, durchschnittlichen)
    Ablenkung von $22^\circ$ der Lichtstrahlen der Sonne, da die
    Eiskristalle häufig senkrecht stehen, scheinen unter dem Winkel von
    $22^\circ$ neben der Sonne links und rechts ebenfalls weitere
    (Neben-)Sonnen zu sein,
    da Brechung und daher Ablenkwinkel wellenlängenabhängig ist,
    gibt es oft eine farbliche Aufspaltung
    (man hat den Eindruck, ein kurzes Regenbogensegment zu sehen),
    analog Nebenmode (moon dogs), aber sehr selten, da Leuchtkraft schwach

    \item
    \textbf{$22^\circ$-Halo}:
    Kristalle nicht mehr alle vertikal ausgerichtet,
    sondern stark unterschiedlich, Lichtablenkung ist (durchschnittlich)
    $22^\circ$,
    Nebensonnen leicht außerhalb, da schräger Lichteinfall auf die Kristalle
    zu einer zusätzlichen Winkeldif"|ferenz führt,
    analog Mondhalo

    \item
    \textbf{$46^\circ$-Halo}:
    Kristalle mit $90^\circ$-Winkeln ergeben Ablenkung von $46^\circ$,
    $46^\circ$-Halos wesentlich seltener, da
    größerer Himmelsbereich voller Eiskristalle sein muss,
    da weniger Lichtanteile durch $90^\circ$-Winkel laufen
    und da der Ring breiter ist (Dispersion, geringerer Kontrast)

    \item
    \textbf{zirkumzenitaler Bogen}:
    Sonne tiefer als $32^\circ$, Eiskristalle stehen senkrecht,
    Ablenkung am $90^\circ$-Prismenwinkel der Kristalle,
    Zirkumzenitalbogen oder Bravaisbogen direkt über Beobachter,
    Berührungsbögen/Tangentialbögen grenzen oben und unten an $22^\circ$-Halo
    an

    \textbf{Horizontalkreis}:
    parallel zum Horizont durch die Sonne um den Beobachter herum,
    überall Eiskristalle, die senkrecht sind und Reflexionen an den
    Seitenwänden verursachen+
\end{itemize}

\pagebreak
