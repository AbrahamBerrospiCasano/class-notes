\section{%
    Ausgleichsprobleme%
}

\subsection{%
    Ausgleichsgerade und Normalengleichungen%
}

\textbf{Ausgleichsgerade (Methode der kleinsten Fehlerquadrate)}:
Gegeben seien Datenpunkte $(t_i, f_i)$ für $i = 1, \dotsc, n$.
Dann kann eine Ausgleichsgerade $p(t) = u + vt$ durch Minimierung
der Fehlerquadratsumme
$\sum_{i=1}^n (f_i - p(t_i))^2$ ermittelt werden.
Man kann explizite Formeln für $u$ und $v$ ausrechnen,
wenn mindestens zwei Abszissen $t_i$ verschieden sind:
\begin{align*}
    u = \frac{(\sum t_i^2) (\sum f_i) - (\sum t_i) (\sum t_i f_i)}
    {n (\sum t_i^2) - (\sum t_i)^2}, \qquad
    v = \frac{n(\sum t_i f_i) - (\sum t_i) (\sum f_i)}
    {n (\sum t_i^2) - (\sum t_i)^2}.
\end{align*}

\linie

\textbf{Normalengleichungen}:
Seien $A$ eine beliebige $m \times n$-Matrix und $x \in \mathbb{R}^n$.
Dann minimiert $x$ die Norm $\norm{Ax - b}_2$ des Residuums $r = Ax - b$
(Ausgleichsproblem) genau dann, wenn $x$ die Normalengleichungen
\begin{align*}
    A^t Ax = A^t b,
\end{align*}
erfüllt.
$A^t A$ ist eine quadratische $n \times n$-Matrix.
Sie ist invertierbar genau dann, wenn $\rg A = n$, d.\,h. wenn die Spalten
von $A$ linear unabhängig sind.
Die Normalengleichungen sind auch für $A^t A$ nicht invertierbar lösbar,
dann ist die Lösung jedoch nicht eindeutig.

\subsection{%
    \textsc{Cholesky}-Faktorisierung%
}

\textbf{positiv definite Matrix}:
Eine quadratische $n \times n$-Matrix heißt positiv definit, falls
$x^t A x > 0$ für alle $x \in \mathbb{R}^n$ mit $x \not= 0$ gilt.

\textbf{\textsc{Cholesky}-Faktorisierung}:
Eine symmetrische, positiv definite $n \times n$-Matrix $S$ besitzt eine
eindeutige Faktorisierung
\begin{align*}
    S = R^t R,
\end{align*}
wobei $R$ eine obere Dreiecksmatrix mit positiven Diagonaleinträgen ist. \\
Die Faktorisierung kann durch aufeinanderfolgendes Lösen der Gleichungen
\begin{align*}
    s_{jk} = r_{1j} r_{1k} + \dotsb + r_{jj} r_{jk}, \qquad k \ge j
\end{align*}
für $j = 1, \dotsc, n$ bestimmt werden.
Dafür wird für jedes $j$ zunächst $r_{jj}$
berechnet und dann können die $r_{jk}$ mit $k = j + 1, \dotsc, n$
bestimmt werden:
\begin{align*}
    r_{jj} = \sqrt{s_{jj} - \sum_{i < j} r_{ij}^2}, \qquad
    r_{jk} = \left(s_{jk} - \sum_{i < j} r_{ij} r_{ik}\right) / r_{jj}.
\end{align*}

\linie

\textbf{Lösung eines symmetrischen, positiv definiten LGS}:
Ein LGS $Sx = b$ mit einer symmetrischen, positiv definiten
(also quadratischen) Matrix $S$ kann mithilfe der Cholesky-Zerlegung
$S = R^t R$ gelöst werden:
\begin{align*}
    Sx = b \quad\Leftrightarrow\quad
    R^t Rx = b \quad\Leftrightarrow\quad
    R^t y = b \;\land\; Rx = y.
\end{align*}
Die Lösungen $y$ und $x$ der zwei LGS in Dreiecksform werden durch
Vorwärts- und Rückwärtseinsetzen bestimmt.
Die Cholesky-Zerlegung kann insbesondere zur Lösung der Normalengleichungen
$A^t Ax = A^t c$ bzw. $Sx = b$ mit $S = A^t A$ und $b = A^t c$ genutzt werden,
falls die $m \times n$-Matrix $A$ maximalen Rang $n$ hat.

\subsection{%
    Singulärwertzerlegung, Pseudoinverse und af"|fine Approximation%
}

\textbf{Singulärwertzerlegung}:
Zu jeder reellen $m \times n$-Matrix $A$ existieren orthogonale Matrizen $U$
und $V$, sodass
\begin{align*}
    U^t A V = S =
    \begin{pmatrix}s_1 & & 0 \\ & s_2 & \\ 0 & & \ddots\end{pmatrix}, \qquad
    S \text{ hat gleiche Größe wie } A.
\end{align*}
Dabei gilt $s_1 \ge \dotsb \ge s_k > s_{k+1} = \dotsb = 0$
mit $k = \rg A$,
die $s_i$ heißen \textbf{Singulärwerte} und sind die Wurzeln der Eigenwerte
von $A^t A$.
Die Spalten $u_j$ von $U$ bzw. $v_j$ von $V$ bilden ONBs aus Eigenvektoren
von $A A^t$ bzw. $A^t A$ und es gilt
$A v_j = s_j u_j$ für $j = 1, \dotsc, k$. \\
(Der Satz gilt für komplexe Matrizen entsprechend.)

Mithilfe der Singulärwertzerlegung lässt sich die lineare Abbildung
$x \mapsto y = Ax$ in der Form
\begin{align*}
    y = \sum_{i=1}^k s_i (v_i^t x) u_i
\end{align*}
darstellen.
Daraus folgt insbesondere, dass $\ker A = \aufspann{v_{k+1}, \dotsc, v_n}$ und
$\im A = \aufspann{u_1, \dotsc, u_k}$. \\
Außerdem gilt $\norm{A}_2 = s_1$ sowie
$\norm{A}_F^2 = \sum_{j,k} |a_{j,k}|^2 = s_1^2 + \dotsb + s_k^2$.

\linie

\textbf{Pseudoinverse}:
Mit der Singulärwertzerlegung $A = USV^t$ einer reellen $m \times n$-Matrix
$A$ lässt sich die Lösung des Ausgleichsproblems
$\norm{Ax - b}_2 \rightarrow \min$ mit minimaler Norm in der Form
\begin{align*}
    x = A^+ b, \quad
    A^+ = V S^+ U^t, \qquad
    S^+ = \diag\{1/s_1, \dotsc, 1/s_k, 0, \dotsc, 0\}, \quad
    k = \rg A
\end{align*}
schreiben.
Dabei heißt $A^+$ die \emph{Pseudoinverse} von $A$ und $S^+$ ist die
$n \times m$-Diagonalmatrix mit den Kehrwerten der positiven Singulärwerte.
(Der Satz gilt für komplexe Matrizen entsprechend.)

Bezeichnen $\{u_1, \dotsc, u_m\}$ und $\{v_1, \dotsc, v_n\}$ die ONBs aus
den Spalten von $U$ und $V$, so lässt sich die lineare Abbildung
$b \mapsto x = A^+ b$ in der faktorisierten Form
\begin{align*}
    x = \sum_{\ell=1}^k \frac{1}{s_\ell} (u_\ell^t b) v_\ell
\end{align*}
darstellen.
Daraus folgt insbesondere, dass $\ker A^+ = \aufspann{u_{k+1}, \dotsc, u_m}$,
$\im A^+ = \aufspann{v_1, \dotsc, v_k}$ sowie $\norm{A^+}_2 = 1/s_k$.

\linie

\textbf{af"|fine Approximation von Punktwolken}:
Eine beste Approximation von Punkten $p_i \in \mathbb{R}^n$, $i = 1, \dotsc, m$
durch einen $k$-dimensionalen af"|finen Unterraum
$H = a + \aufspann{v_1, \dotsc, v_k}$ lässt sich durch Minimierung der
Quadratsumme der Distanzen $e(P, H)^2 = \sum_{i=1}^m d(p_i, H)^2$ bestimmen.

$H$ kann folgendermaßen berechnet werden:
Der Aufpunkt $a$ ist der Mittelpunkt der Ortsvektoren $p_i$.
Die Richtungsvektoren können durch die Singulärwertzerlegung der
$m \times n$-Matrix der zentrierten Punkte
$(p_1 - a, \dotsc, p_m - a)$ bestimmt werden, d.\,h.
\begin{align*}
    a = \frac{1}{m} \sum_{i=1}^m p_i, \qquad
    (p_1 - a, \dotsc, p_m - a) = U S V^t.
\end{align*}
Dabei müssen die ersten $k$ Spalten von $V$ als Basisvektoren $v_j$ gewählt
werden. \\
Der Fehler kann dabei durch die Singulärwerte errechnet werden:
\begin{align*}
    e(P, H)^2 = \sum_{i=k+1}^m s_i^2.
\end{align*}

\pagebreak
