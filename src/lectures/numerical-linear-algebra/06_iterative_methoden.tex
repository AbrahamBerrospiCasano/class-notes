\chapter{%
    Iterative Methoden%
}

\section{%
    Lineare Iterationsverfahren%
}

\textbf{Lineares Iterationsverfahren}:
Um ein LGS $Ax = b$ zu lösen, kann man es mithilfe einer
invertierbaren Matrix $C$ in der Form
\begin{align*}
    (A - C)x + Cx = b
    \quad\Leftrightarrow\quad
    x = (E - C^{-1}A)x + C^{-1}b = Qx + p
\end{align*}
schreiben.
Die Rekursion
\begin{align*}
    x_{\ell+1} = Qx_\ell + p, \quad
    Q = E - C^{-1}A, \quad
    p = C^{-1}b
\end{align*}
definiert dann ein lineares Iterationsverfahren, das als Fixpunkt $x_\ast$
die Lösung des LGS hat.
Für eine schnelle Konvergenz sollte $C^{-1}$ die Inverse von $A$ möglichst
gut approximieren.
Gleichzeitig müssen für ein ef"|fizientes Verfahren Produkte mit $C^{-1}$
einfach berechenbar sein.

\linie

\textbf{Näherungsfehler eines linearen Iterationsverfahrens}:
Für den Fehler $\Delta x_\ell = x_\ell - x_\ast$ der $\ell$-ten Näherung
eines linearen Iterationsverfahrens mit der Rekursion
$x_{\ell+1} = Qx_\ell + p$ und dem Fixpunkt $x_\ast$ gilt
\begin{align*}
    \Delta x_\ell
    = Q \Delta x_{\ell-1}
    = \dotsb
    = Q^\ell \Delta x_0.
\end{align*}
Die Iteration konvergiert für jeden Startwert $x_0$ gegen den Fixpunkt
$x_\ast$ genau dann, wenn der \textbf{Spektralradius} von $Q$
\begin{align*}
    \varrho(Q) = \max\{|\lambda| \;\;|\;\; Qv = \lambda v,\; v \not= 0\}
\end{align*}
(also der Betrag des betragsmäßig größten Eigenwerts) kleiner als $1$ ist. \\
Der Spektralradius ist außerdem auch ein Maß für die Konvergenzrate, also für
die im Mittel zu erwartende Fehlerreduktion pro Iterationsschritt.
Je kleiner $\varrho(Q)$ ist, desto schneller konvergiert das Verfahren.

Für den Spektralradius gilt im Übrigen $\varrho(Q) \le \norm{Q}$ für jede
induzierte Matrixnorm \\
$\norm{Q} = \sup_{x \not= 0} \frac{\norm{Qx}}{\norm{x}}$
(man setze einfach einen Eigenvektor zum Spektralradius ein).

\section{%
    \textsc{Jacobi}-Verfahren%
}

\textbf{\textsc{Jacobi}-Verfahren}:
Ein elementares lineares Iterationsverfahren für ein LGS $Ax = b$ ist
das \textsc{Jacobi}-Verfahren.
Bei diesem wird die Diagonale $D$ von $A$ als Approximation von $A^{-1}$
verwendet.
Ein Schritt $x_\ell = y \rightarrow z = x_{\ell+1}$ des
\textsc{Jacobi}-Verfahrens hat also die Form
\begin{align*}
    z = y - D^{-1}Ay + D^{-1}b = Qy + p \qquad\text{bzw.}\qquad
    z_j = \left(b_j - \sum_{k\not=j} a_{jk}y_k\right) / a_{jj}, \quad
    j = 1, \dotsc, n
\end{align*}
mit $Q = E - D^{-1}A$ und $p = D^{-1}b$.
Ein hinreichendes Kriterium für die Konvergenz des \textsc{Jacobi}-Verfahrens
ist, dass die Koef"|fizientenmatrix $A$ diagonal dominant ist, d.\,h.
\begin{align*}
    |a_{jj}| > \sum_{k\not=j} |a_{jk}| \quad \text{für} \quad j= 1, \dotsc, n.
\end{align*}

\section{%
    \textsc{Gauss}-\textsc{Seidel}-Verfahren%
}

\textbf{\textsc{Gauss}-\textsc{Seidel}-Verfahren}:
Das \textsc{Gauss}-\textsc{Seidel}-Verfahren für ein LGS $Ax = b$ entsteht
aus dem \textsc{Jacobi}-Verfahren, indem man den Näherungsvektor elementweise
neu bestimmt und für die Berechnung der $k$-ten Komponente der nächsten
Näherung bereits die neuen Daten der ersten $k - 1$ Komponenten verwendet. \\
Dies entspricht einer Auteilung der Matrix $A$ in eine Diagonalmatrix $D$,
eine linke Dreiecksmatrix $L$ und eine rechte Dreiecksmatrix $R$. \\
Ein Iterationsschritt $x_\ell = y \rightarrow z = x_{\ell+1}$ hat somit die
Form
\begin{align*}
    z = -D^{-1}(Lz + Ry) + D^{-1}b \qquad\text{bzw. nach } z
    \text{ aufgel"ost}\qquad
    z = -(L + D)^{-1}Ry + (L + D)^{-1}b.
\end{align*}
Dabei muss die Iterationsmatrix $Q = -(L + D)^{-1}R$ nicht explizit neu
berechnet werden.
Für eine $n \times n$-Matrix $A$ ist nämlich
\begin{align*}
    z_j = \left(b_j - \sum_{k<j} a_{jk}z_k - \sum_{k>j} a_{jk}y_k\right)
    / a_{jj}, \quad j= 1, \dotsc, n.
\end{align*}
Bei der sukzessiven Ausführung der Operationen kann auch $z = y$ gesetzt
werden.
Die Vektorelemente werden dann automatisch in der gewünschten Weise
überschrieben.
Wie das \textsc{Jacobi}-Verfahren konvergiert auch das
\textsc{Gauss}-\textsc{Seidel}-Verfahren für diagonal dominante Matrizen $A$.
Darüber hinaus konvergiert es auch für symmetrische, positiv definite
Matrizen $A$.

\section{%
    (Über-)Relaxation%
}

\textbf{Relaxation}:
Bei einem Iterationsverfahren kann man versuchen, die Konvergenz durch
eine sogenannte Relaxation zu beschleunigen.
Dazu wird in der Iterationsvorschrift $x_{\ell+1} = f(x_\ell)$ ein zusätzlicher
Parameter $\omega$ eingeführt und das neue Folgenglied auf der durch
$x_\ell$ und $f(x_\ell)$ verlaufenden Gerade gewählt:
\begin{align*}
    x_{\ell+1} = (1 - \omega)x_\ell + \omega f(x_\ell).
\end{align*}
Für $\omega = 1$ erhält man das ursprüngliche Iterationsverfahren. \\
Für $\omega > 1$ spricht man von \textbf{Überrelaxation} und für $\omega < 1$
von \textbf{Unterrelaxation}.

\linie

\textbf{sukzessive Überrelaxation (SOR)}:
Berechnet man beim \textsc{Gauss}-\textsc{Seidel}-Verfahren die einzelnen
Komponenten sukzessive, so kann man die Relaxation in jedem Teilschritt
anwenden.
Das so entstehende Verfahren heißt sukzessive Überrelaxation oder SOR
(\emph{successive over-relaxation}).
Die Iterationsvorschrift hat die Form
\begin{align*}
    x_{\ell+1}
    = x_\ell + \omega D^{-1}(b - Lx_{\ell+1} - Dx_\ell - Rx_\ell),
\end{align*}
wobei $A = L + D + R$ die Aufteilung der Matrix in den linken, diagonalen
und rechten Anteil ist.

Führt man zwei SOR-Schritte durch, wobei beim ersten Schritt die Komponenten
in der Reihenfolge $1, \dotsc, n$ und beim zweiten Schritt in der umgekehrten
Reihenfolge berechnet werden, so erhält man das
\textbf{SSOR-Verfahren} (\emph{symmetric SOR}).
Dabei ist nur die Behandlung der Reihenfolge der Unbekannten symmetrisch,
die Iterationsmatrix jedoch im Allgemeinen nicht.
Auch die Konvergenzrate wird i.\,A. nicht besser.
Die Symmetrisierung wird in erster Linie bei der Vorkonditionierung des
Konjugierte-Gradienten-Verfahrens verwendet.
