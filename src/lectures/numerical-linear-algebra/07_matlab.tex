\chapter{%
    \emph{Zusatz}: Programmieren in \matlab{}%
}

\textbf{Allgemeines}:
\matlab{}s Programmiersprache arbeitet zeilenorientiert.
Zeilen können mit \code{;} abgeschlossen werden, andernfalls wird das
Ergebnis der Berechnung angezeigt.
Zeilenkommentare werden mit \code{\%} eingeleitet.

\textbf{Operatoren}:
Der Zuweisungsoperator ist \code{=}, der Vergleichsoperator \code{==}.
Außer den üblichen Operatoren \code{+}, \code{-}, \code{*}, \code{/} stehen
\code{^} (Potenzieren) und \code{\\} (LGS lösen wie in \code{x = A \\ b})
zur Verfügung.
Bei manchen Operatoren wie \code{*} oder \code{^} gibt es eine entsprechende
punktweise Operatoren \code{.*} und \code{.^}, die die ursprüngliche Operatoren
auf Matrizen und Vektoren punktweise anwendet (z.\,B. \code{A * B} vs.
\code{A .* B}).

\textbf{Vektor- und Matrixrechnung}:
\code{1:n} erzeugt einen Zeilenvektor mit Einträgen $1$ bis $n$. \\
Wie in \code{5.3:0.1:6} kann die Schrittweite vorgegeben werden. \\
Eine Matrix kann mit \code{[a1, a2, a3; a4, a5, a6]} zeilenweise eingegeben
werden.
Transponieren erfolgt durch angefügtes \code{'}.

\textbf{Strings}:
Strings werden durch einfache Hochkommata \code{'} begrenzt.
Konkatenation erfolgt durch Einfügen in einen Zeilenvektor:
\code{['Ich bin ', 'ein String.']}.

\textbf{Inline-Funktionen}:
Eine Inline-Funktion kann wie \code{rho = @(w) 2*w + 3} deklariert werden.
Danach kann durch \code{rho(42)} auf die Funktion zugegriffen werden.

\textbf{Kontrollfluss}:
Blöcke durch den Kontrollfluss werden durch \code{end} in einer eigenen
Zeile beendet.
Eine \code{for}-Schleife beginnt mit \code{for i = 1:n},
eine \code{while}-Schleife analog mit \\
\code{while error > tol} und
eine \code{if}-Abfrage mit \code{if k == 1}. \\
Verzweigungen werden dabei mit \code{elseif k == n} und \code{else}
realisiert. \\
In Schleifen sind \code{break} und \code{continue} möglich.

\textbf{Funktionen}:
Programme können skriptähnlich einfach in programmname.m-Dateien geschrieben
werden, sie werden dann durch \code{programmname} aufgerufen (falls \matlab{} im
aktuellen Verzeichnis ist).
Man kann auch Funktionen erstellen, die jedoch dann eine ganze Datei umfassen
(es gibt also nur eine Funktion pro Datei) und von anderen Dateien aufgerufen
werden müssen.
Funktionen beginnen mit \code{function [rueckgabe1, rueckgabe2] =} \\
\code{funktionsname(parameter1, parameter2)}, diese Zeile muss ganz am Anfang
der Datei stehen.
Ansonsten werden Funktionen wie Skripte als funktionsname.m
(Dateiname = Funktionsname!) gespeichert und durch
\code{[A, B] = funktionsname(42, [1, 2])} aufgerufen.

\textbf{eingebaute Funktionen}:
Zu den nützlichsten eingebauten \matlab{}-Funktionen gehören \\
\code{size(A, n)} ($n$-te Dimension der Matrix $A$),
\code{length(x)} (Größe des Vektors $x$), \\
\code{zeros(m, n)} ($m \times n$-Nullmatrix),
\code{eye(n)} ($n \times n$-Einheitsmatrix),
\code{repmat(A, m, n)} (Blockmatrix mit $m \times n$ Blöcken von $A$),
\code{norm(x, art)} (wobei \code{1}, \code{2}, \code{inf} und \code{'fro'}
für \code{art} möglich sind),
\code{inv(A)},
\code{eig(A)} (Eigenwerte von $A$),
\code{[P, D] = eig(A)} (Basiswechselmatrix $P$ und Diagonalmatrix $D$),
\code{abs(x)},
\code{max(x)}, \code{min(x)},
\code{sum(x)},
\code{disp(x)} (zeigt Zahl/String/Vektor/Matrix/\ldots{} $x$ an)

\pagebreak
