\chapter{%
    Lineare Optimierung%
}

\section{%
    Lineares Programm und Basislösungen%
}

\textbf{Standardform eines linearen Programms}:
Ein \emph{lineares Programm} in Standardform ist ein Optimierungsproblem
mit linearer Zielfunktion,
linearen Gleichungsnebenbedingungen und nicht-negativen Unbekannten:
\begin{align*}
    c^t x \rightarrow \min, \qquad
    Ax = b, \qquad
    x \ge 0.
\end{align*}
Man nimmt an, dass die $m \times n$-Matrix $A$ vollen Rang $m \le n$ hat.

\textbf{allgemeines Problem auf Standardform bringen}:
Sind nur allgemeine lineare Nebenbedingungen
$Py = q$, $Ry \le s$ mit den Unbekannten $y_i$ gegeben, so kann man das Problem
durch Einführung zusätzlicher Variablen auf die Standardform bringen: \\
Man schreibt $y = u - v$ als Differenz nicht-negativer Vektoren $u, v \ge 0$
und führt nicht-negative Schlupfvariablen $w$ ein, um
die Ungleichungsnebenbedingungen zu entfernen. \\
Genauer gesagt werden die Nebenbedingungen ersetzt durch
\begin{align*}
    Pu - Pv = q, \qquad
    Ru - Rv + w = s, \qquad
    (u, v, w) \ge 0.
\end{align*}
Ist $c^t x \rightarrow \max$ als Maximimierungsproblem gegeben, so kann man
mit $d^t x \rightarrow \min$, $d = -c$ ein äquivalentes Minimierungsproblem
erzeugen.

\linie

\textbf{Basislösung}:
Ein Vektor $x$ heißt \emph{Basislösung} eines linearen Programms
$c^t x \rightarrow \min$, $Ax = b$, $x \ge 0$ mit $m \times n$-Matrix $A$
vollen Ranges, falls es einen Indexvektor
$I \subset \{1, \dotsc, n\}$ der Länge $m$ gibt, sodass
$A_I = A(:, I)$ invertierbar ist und $A_I x_I = b$ mit $x_I = x(I)$
und $x_k = 0$ für $k \notin I$.

\textbf{zulässig/unzulässig}:
Die Basislösung $x$ heißt \emph{zulässig}, falls $x_I \ge 0$ ist.
Ansonsten heißt sie nicht \emph{nicht zulässig} oder \emph{unzulässig}.

Eine Basislösung ist demnach eine Lösung mit einer möglichst großen Anzahl
an Nullen.
Die Indexmenge legt die Basislösung eindeutig fest.
Umgekehrt kann es jedoch zu einer Basislösung mehrere Indexmengen mit derselben
Basislösung geben, falls $x_i = 0$ für bestimmte $i \in I$.

\linie

\textbf{Satz zur optimalen Lösung linearer Programme}: \\
Sei ein lineare Programm $c^t x \rightarrow \min$, $Ax = b$, $x \ge 0$
gegeben.
Ist die zulässige Lösungsmenge
$D = \{x \in \mathbb{R}^n \;|\; Ax = b,\; x \ge 0\}$
nicht-leer und ist die Zielfunktion $c^t x$ auf $D$ nach unten beschränkt,
so existiert eine optimale zulässige Basislösung $x^\ast$ mit
$c^t x^\ast = \inf_{x \in D} c^t x$.

Gibt es also Lösungen und die Zielfunktion ist auf $D$ nach unten beschränkt,
so ist die zulässige Basislösung mit dem kleinsten Zielfunktionswert optimal.

\pagebreak

\section{%
    Pivotschritt für ein lineares Programm und Rang-1-Modifikation einer
    inversen Matrix%
}

\textbf{Pivotschritt}:
Ein Pivotschritt für ein lineares Programm $c^t x \rightarrow \min$,
$Ax = b$, $x \ge 0$ benötigt die Indexmenge $I$ einer zulässigen Basislösung
$x$ und verändert diese zu einer Indexmenge $J$ einer neuen zulässigen
Basislösung $y$, sodass der Zielfunktionswert nicht ansteigt, d.\,h.
\begin{align*}
    I \rightarrow J = (I \setminus j) \cup k \quad\text{ mit }\quad
    c^t y \le c^t x.
\end{align*}
Der Pivotschritt tauscht also ein Index der Basislösung mit einem anderen aus,
sodass die Lösung nicht schlechter wird.

\textbf{Bestimmung der Indizes}:
\begin{itemize}
    \item
    \textbf{Bestimmung von $k$}:
    Sei $K = \{1, \dotsc, n\} \setminus I$ das Komplement von $I$. \\
    Dann berechnet man
    \begin{align*}
        d_K^t = c_K^t - c_I^t A_I^{-1} A_K
    \end{align*}
    und man wählt den kleinsten Eintrag von $d_K$.
    $k \in K$ ist dann der zugehörige Index. \\
    Für $d_K \ge 0$ ist $x$ eine optimale Lösung des linearen Programms.

    \item
    \textbf{Bestimmung von $j$}:
    Sei $A_k = A(:, k)$.
    Man berechnet nun
    \begin{align*}
        z_I = A_I^{-1} A_k.
    \end{align*}
    Ist $z_I \le 0$, so ist die Zielfunktion nach unten unbeschränkt
    und daher das lineare Programm unlösbar.
    Andernsfalls muss $j \in I$ so gewählt werden, sodass
    \begin{align*}
        \frac{x_j}{z_j} \quad\text{ mit }\quad z_j > 0
    \end{align*}
    minimal wird.
\end{itemize}
Bei Uneindeutigkeit wird für $k$ und $j$ jeweils der kleinstmögliche Index
gewählt.

\linie

\textbf{Rang-1-Modifikation einer inversen Matrix}:
Sei eine invertierbare Matrix $A$ mit ihrer Inversen $A^{-1}$ gegeben.
Die Matrix $B$ entsteht aus $A$, indem man die $j$-te Spalte durch den
Vektor $v$ ersetzt. \\
Gilt nun
\begin{align*}
    Au = v \quad\text{ mit }\quad u_j \not= 0,
\end{align*}
so ist $B$ invertierbar und für die inverse Matrix $B^{-1}$ gilt
\begin{align*}
    B^{-1} = QA^{-1} \quad\text{ mit }\quad
    Q = E + \frac{1}{u_j} (e_j - u) e_j^t.
\end{align*}

\pagebreak

\section{%
    Simplex-Tableau und Simplex-Algorithmus%
}

\textbf{Simplex-Tableau}:
Sei ein lineares Programm $c^t x \rightarrow \min$, $Ax = b$, $x \ge 0$
gegeben.
Dann kann ein Pivot-Schritt $I \rightarrow J = (I \setminus j) \cup k$
mithilfe des Tableaus $T_I = (A_I^{-1}, x_I)$ durchgeführt werden.
Um das Tableau $T_J = (A_J^{-1}, y_J)$ zu erhalten, wird $T_I$ wie folgt
verändert:
\begin{enumerate}
    \item
    Man berechnet $z_I = A_I^{-1} A_k$.

    \item
    Die Zeile mit Index $j$ von $T_I$ wird durch $z_j$ dividiert.

    \item
    Für alle $i \in I \setminus J$ wird die modifizierte Zeile mit Index $j$
    mit $z_i$ multipliziert und von der Zeile mit Index $i$ subtrahiert.
\end{enumerate}
Dabei sind die Matrizen und Vektoren entsprechend $I$ und $J$ indiziert,
d.\,h. die Zeile mit Index $i$ ist die $\ell$-te Zeile, wobei $i$ der
$\ell$-te Index in $I$ ist.

\linie

\textbf{Simplex-Algorithmus}:
Der Simplex-Algorithmus zur Lösung eines linearen Programms \\
$c^t x \rightarrow \min$, $Ax = b$, $x \ge 0$
modifiziert eine zulässige Basislösung $x_I$ mit aufeinanderfolgenden
Pivot-Operationen, bis ein optimaler Vektor $x$ gefunden ist.

Ein Schritt $x_I \rightarrow x_J$ mit $J = (I \setminus j) \cup k$,
der das Simplex-Tableau $T_I = (A_I^{-1}, x_I)$ verwendet, verläuft wie folgt:
\begin{enumerate}
    \item
    Man berechnet $d_K^t = c_K^t - c_I^t A_I^{-1} A_K$ mit
    $K = \{1, \dotsc, n\} \setminus I$ und wählt $k$ kleinstmöglich,
    sodass $d_K$ sein Minimum annimmt.
    Für $d_k \ge 0$ ist die aktuelle Basislösung $x_I$ optimal.

    \item
    Ist $z_I = A_I^{-1} A_k \le 0$, so ist $\inf_{x \in D} c^t x = -\infty$
    und das lineare Programm hat keine Lösung.
    Andernfalls wählt man $j$ kleinstmöglich, sodass
    $\frac{x_i}{z_i}$ mit $z_i > 0$ und $i \in I$ minimal wird.

    \item
    Das Tableau $T_I$ wird aktualisiert mit
    $T_I = (A_I^{-1}, x_I) \rightarrow T_J = (A_J^{-1}, y_J)$, indem
    die Zeile mit Index $j$ durch $z_j$ dividiert und
    für jedes $i \in I \setminus j$ von der Zeile mit Index $i$
    die modifizierte Zeile mit Index $j$ multipliziert mit $z_i$ subtrahiert
    wird.
\end{enumerate}

\linie

\textbf{Hilfsproblem für ein lineares Programm}: \\
Der Simplex-Algorithmus zur Lösung eines linearen Programms
$c^t x \rightarrow \min$, $Ax = b$, $x \ge 0$
benötigt anfangs eine zulässige Basislösung. \\
Ist $b = (b_1, \dotsc, b_m) \ge 0$, so kann eine solche
durch Lösen des Hilfsproblems
\begin{align*}
    y_1 + \dotsb + y_m \rightarrow \min, \qquad
    Ax + y = b, \qquad
    x, y \ge 0
\end{align*}
bestimmt werden.
Um dieses Hilfproblem wiederum mit dem Simplex-Algorithmus zu lösen,
benötigt man wieder eine zulässige Basislösung zum Start, die jedoch
durch $(x^t, y^t) = (0, b^t)$ einfach gegeben ist. \\
Ist nun $(x^t, y^t)$ eine Lösung des Hilfsproblems, so ist $x$ eine
zulässige Basislösung für das Ausgangsproblem, falls $y = 0$ ist. \\
$y \not= 0$ ist genau dann der Fall, wenn das Ausgangsproblem keine
zulässigen Punkte hat.

\pagebreak

\section{%
    \emph{Beispiel}: Polynomiale Approximierung einer Funktion%
}

Gegeben seien Messpunkte $(t_j, f_j)$ für $j = 1, \dotsc, m$.
Sie sollen durch ein Polynom $n$-ten Grades approximiert werden.
Als Approximationsfehler soll hier der größte Abstand des Polynoms
zu einem Punkt minimiert werden:
$\max_{j=1,\dotsc,m}
\left|f_j - \sum_{k=0}^n p_k t_j^k\right| \rightarrow \min$.

Dies lässt sich in Ungleichungen durch
$\sum_{k=0}^n p_k t_j^k - e \le f_j$ und
$\sum_{k=0}^n p_k t_j^k + e \ge f_j$ mit $e \rightarrow \min$ darstellen.
Die Einführung von Schlupfvariablen führt zu \\
$\sum_{k=0}^n p_k t_j^k - e + \delta_j^+ = f_j$ und
$\sum_{k=0}^n p_k t_j^k + e = f_j + \delta_j^-$.

Setzt man $p_k = p_k^+ - p_k^-$ mit $p_k^+, p_k^- \ge 0$ und trennt
Variablen und Konstanten, so erhält man \\
$-e + p_0^+ t_j^0 + \dotsb + p_n^+ t_j^n -
p_0^- t_j^0 - \dotsb - p_n^- t_j^n + \delta_j^+ = f_j$ und \\
$-e - p_0^+ t_j^0 - \dotsb - p_n^+ t_j^n +
p_0^- t_j^0 + \dotsb + p_n^- t_j^n + \delta_j^- = -f_j$
für $j = 1, \dotsc, m$.

So erhält man die Nebenbedingungen
\begin{align*}
    \underbrace{\left(\begin{array}{ccccccccccccc}
        -1 & t_1^0 & \dots & t_1^n & -t_1^0 & \dots & -t_1^n &
        1 & & 0 & 0 & \dots & 0 \\
        \vdots & \vdots & & \vdots & \vdots & & \vdots & & \ddots &
        & \vdots & & \vdots \\
        -1 & t_m^0 & \dots & t_m^n & -t_m^0 & \dots & -t_m^n &
        0 & & 1 & 0 & \dots & 0 \\
        -1 & -t_1^0 & \dots & -t_1^n & t_1^0 & \dots & t_1^n &
        0 & \dots & 0 & 1 & & 0 \\
        \vdots & \vdots & & \vdots & \vdots & & \vdots &
        \vdots & & \vdots & & \ddots & \\
        -1 & -t_m^0 & \dots & -t_m^n & t_m^0 & \dots & t_m^n &
        0 & \dots & 0 & 0 & & 1
    \end{array}\right)}_{A}
    \underbrace{\left(\begin{array}{c}
        e \\
        p_0^+ \\
        \vdots \\
        p_n^+ \\
        p_0^- \\
        \vdots \\
        p_n^- \\
        \delta_1^+ \\
        \vdots \\
        \delta_m^+ \\
        \delta_1^- \\
        \vdots \\
        \delta_m^-
    \end{array}\right)}_{x}
    =
    \underbrace{\left(\begin{array}{c}
        f_1 \\
        \vdots \\
        f_m \\
        - f_1 \\
        \vdots \\
        - f_m
    \end{array}\right)}_{b}.
\end{align*}

Eine Startlösung erhält man mit dem Polynom $p = 0$
($p_k^+ = p_k^- = 0$, $e = \max_{j=1,\dotsc,m} |f_j|$).

\pagebreak
