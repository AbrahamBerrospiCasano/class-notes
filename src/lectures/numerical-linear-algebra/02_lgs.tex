\section{%
    Lineare Gleichungssysteme%
}

\subsection{%
    Allgemeines, \textsc{Gauß}-\textsc{Jordan}-Algorithmus und
    Fehlerabschätzung%
}

Ein \textbf{LGS} besteht aus $m$ Gleichungen in $n$ Unbestimmten.
Es kann durch die Koef"|fizientenmatrix $A$, den Unbekannten $x_j$ und einer
rechten Seite $b$ dargestellt werden.
Für $b = 0$ heißt das LGS \textbf{homogen}, sonst \textbf{inhomogen}.
Ein LGS ohne Lösung heißt \textbf{überbestimmt}, ein LGS mit mehreren
Lösungen heißt \textbf{unterbestimmt}.

\linie

\textbf{\textsc{Gauß}-\textsc{Jordan}-Algorithmus}: \\
\begin{tabular}{p{6.5cm}p{9.65cm}}
\matrixsize{$\left(\begin{array}{cccccc|c}
1 & \cdots & 0 &
w_{1, \ell} & \cdots & w_{1, n} & w_{1, n + 1} \\
\vdots & \ddots & \vdots & \vdots & & \vdots & \vdots \\
0 & \cdots & 1 & & & & \\
0 & \cdots & 0 & w_{\ell, \ell} & \cdots & w_{\ell, n} & w_{\ell, n + 1} \\
\vdots & & \vdots & \vdots & & \vdots & \vdots \\
0 & \cdots & 0 & w_{n, \ell} & \cdots & w_{n, n} & w_{n, n + 1}
\end{array}\right)$}
&
\begin{minipage}[c]{9.65cm}
Durch Umformen wird die Matrix eines LGS $Ax = b$ in die Einheitsmatrix
überführt.
Vor dem $\ell$-ten Eliminationsschritt sieht die Matrix wie links angegeben
aus, dieser verläuft wie folgt:
\end{minipage}\end{tabular}

\begin{enumerate}
    \item Bestimmung des Maximums $|w_{i,\ell}|$ der Beträge von
    $w_{\ell,\ell}, \ldots, w_{n,\ell}$ und Vertauschen der Zeilen $\ell$ und
    $i$,

    \item Division der Zeile $\ell$ durch $w_{\ell,\ell}$ und

    \item Subtraktion des $w_{j, \ell}$-fachen der Zeile $\ell$ von allen
    Zeilen $j$ mit $j \not= l$, d.\,h. \\
    $w_{j,k} \leftarrow w_{j,k} - w_{j,\ell} \cdot w_{\ell,k}$
    für $k = \ell, \ldots, n + 1$.
\end{enumerate}

\linie

\textbf{Vektor- und Matrixnormen}:
Für zwei Matrizen oder Vektoren gilt stets
$\norm{A \cdot B} \le \norm{A} \cdot \norm{B}$.
\begin{align*}
    \norm{x}_2 = \sqrt{\sum_k |x_k|^2},
    \quad
    \norm{x}_\infty = \max_k |x_k|,
    \qquad\qquad
    \norm{A}_\infty = \max_j \sum_k |a_{jk}|,
    \quad
    \norm{A}_F = \sqrt{\sum_{i,j} |a_{jk}|^2}
\end{align*}
\begin{align*}
    \norm{A}_2 = \max_{\norm{x}_2 = 1} \norm{Ax}_2
    = \max\{\sqrt{\lambda} \;|\; \lambda \text{ EW von } A^\ast A\}
\end{align*}

\linie

\textbf{Fehler bei LGS}:
Ist $\widetilde{x}$ die numerische berechnete Lösung eines regulären linearen
LGS $Ax = b$ sowie $A\widetilde{x} = \widetilde{b}$, dann gilt für den
Fehler $\Delta x = \widetilde{x} - x$, dass
\begin{align*}
    \cond(A)^{-1} \cdot\frac{\Vert \Delta b \Vert}{\Vert b \Vert}
    \le \frac{\Vert \Delta x \Vert}{\Vert x \Vert}
    \le \cond(A) \cdot \frac{\Vert \Delta b \Vert}{\Vert b \Vert},
\end{align*}
wobei $\cond(A) = \Vert A \Vert \cdot \Vert A^{-1} \Vert$ die
\textbf{Kondition} der Matrix $A$ ist.
Beide Ungleichungen sind bestmöglich, d.\,h. Gleichheit kann durch
konkrete Beispiele erreicht werden.

\linie

\textbf{Rückwärtseinsetzen}:
Bei einem LGS $R = (r_{i,j})_{ij}$ in oberer Dreiecksform mit
$r_{1,1}, \dotsc, r_{n,n} \not= 0$ können die Unbekannten $x_n, \dotsc, x_1$
nacheinander bestimmt werden durch
\begin{align*}
    x_n = b_n / r_{n,n} \quad\text{ sowie }\quad
    x_\ell = (b_\ell - r_{\ell,\ell+1} x_{\ell+1} - \dotsb -
    r_{\ell,n} x_n) / r_{\ell,\ell},
\end{align*}
wobei die schon berechneten Werte
$x_{\ell+1}, \dotsc, x_n$ verwendet werden ($\ell = n - 1, \dotsc, 1$).

\linie

\textbf{\textsc{Cramer}sche Regel}:
Sei $A$ eine $n \times n$-Matrix und $b$ ein $n$-zeiliger Spaltenvektor.
Dann lässt sich die Lösung des LGS $Ax = b$ berechnen durch
\begin{align*}
    x_i = \frac{\det A_i}{\det A}, \qquad
    x = \begin{pmatrix}x_1 \\ \vdots \\ x_n\end{pmatrix}, \qquad
    i = 1, \dotsc, n,
\end{align*}
wobei $A_i$ die Matrix ist, die aus $A$ entsteht, wenn man die $i$-te
Spalte durch $b$ ersetzt.

\pagebreak

\subsection{%
    \textsc{Householder}-Transformation und QR-Zerlegung%
}

\textbf{Spiegelung eines Vektors an einer Hyperebene}:
Sei $d$ ein zu einer Hyperebene orthogonaler Vektor, dann ist
\fracsize{$Q = E - \frac{2}{\norm{d}_2^2} d d^t$} die Transformationsmatrix
der Spiegelung eines Vektors $x$ an der Hyperebene,
da \fracsize{$Qx = x - \frac{2}{\norm{d}_2^2} d d^t x =
x - 2 \frac{d}{\norm{d}_2} \cdot \sp{\frac{d}{\norm{d}_2}, x}$},
wobei \fracsize{$\sp{\frac{d}{\norm{d}_2}, x}$} der Abstand von $x$ zur
Hyperebene ist.

\linie

\textbf{\textsc{Householder}-Transformation}: Die Transformation
\begin{align*}
    x \mapsto Qx = x - \frac{1}{r} \left(d^t x\right) \cdot d, \qquad
    d = \begin{pmatrix}c_1 + \sigma \norm{c}_2 \\ c_2 \\ \vdots \\
    c_n\end{pmatrix}, \quad
    \sigma = \begin{cases}\sgn(c_1) & c_1 \not= 0 \\
    1 & c_1 = 0\end{cases}, \quad
    r = |d_1| \norm{c}_2
\end{align*}
ist eine Spiegelung, die den Vektor
$c$ auf $-\sigma \norm{c}_2 \cdot e_1$ (also auf ein Vielfaches des
ersten Einheitsvektors) abbildet.
Sie wird \emph{Householder-Transformation} genannt.

Normalerweise wird die Householder-Transformation durch
\begin{align*}
    A \mapsto A - d \cdot \frac{1}{r} \left(d^t A\right)
\end{align*}
gleichzeitig auf alle Spalten einer Matrix $A$ angewandt, wobei $c = A(:, 1)$
die erste Spalte von $A$ ist.
Dadurch werden die Einträge $a_{2,1}, a_{3,1}, \dotsc$ zu $0$.

Die Matrix $Q$ einer Householder-Transformation ist symmetrisch ($Q^t = Q$), \\
orthogonal ($Q^{-1} = Q^t$) und involutorisch ($Q^2 = E$).

\linie

\textbf{Permutation bei LGS}:
Bei einem LGS $Ax = b$ ändert eine Permutation
$(1, 2, \dotsc) \rightarrow I$ \\
($I$ Indexvektor) der Zeilen der Matrix die rechte Seite ($A(I,:) x = b(I)$)
und eine Permutation der Spalten die Lösung ($A(:,I) x(I) = b$).

\linie

\textbf{QR-Faktorisierung}:
Eine $m \times n$-Matrix $A$ kann, ggf. nach einer Permutation der Spalten,
als Produkt einer orthogonalen Matrix $Q$ und einer oberen Dreiecksmatrix
$R$ geschrieben werden:
\begin{align*}
    A(:,I) = QR \quad\text{ mit }\quad
    R = \begin{pmatrix}\widetilde{R} & S \\ 0 & 0\end{pmatrix},
\end{align*}
wobei $I$ ein Indexvektor und $\widetilde{R}$ eine quadratische invertierbare
obere Dreiecksmatrix mit Zei\-len"~/Spaltenzahl $\rg A$ ist. \\
Die QR-Zerlegung lässt sich mit maximal $\min\{m - 1, n\}$
Householder-Transformationen $Q_k$ konstruieren:
$Q = (\dotsc Q_2 Q_1)^{-1} = Q_1 Q_2 \dotsc$
($Q_i$ sind orthogonal und symmetrisch).

\textbf{Ablauf der QR-Faktorisierung}: \\
\begin{tabular}{p{4.5cm}p{11.65cm}}
\matrixsize{$\left(\begin{array}{ccc|ccc}
\ast & \cdots & \ast & \ast & \cdots & \ast \\
& \ddots & \vdots & \vdots & & \vdots \\
0 & & \ast & \ast & \cdots & \ast \\ \hline
& & & & & \\
& 0 & & & B & \\
& & & & & \\
\end{array}\right)$}
&
\begin{minipage}[c]{11.65cm}
Ist die Matrix $Q_{\ell - 1} \dotsm Q_1 A(:,I)$ vor dem $\ell$-ten
Transformationsschritt von der Form wie links angegeben, so verläuft der
nächste Schritt wie folgt:
\end{minipage}\end{tabular}
\begin{enumerate}
    \item Ist $B = 0$, so ist die Transformation abgeschlossen.
    Für $B \not= 0$ tauscht man die erste Spalte mit der Spalte, die
    die maximale $2$-Norm hat ($c = B(:,1)$).
    Die Permutation muss durch entsprechenden Tausch in I gespeichert werden.

    \item Falls $B$ mehr als eine Zeile hat, wendet man nun die
    Householder-Transformation auf $B$ an, sodass unterhalb von Position
    $(\ell, \ell)$ Nullen sind.
\end{enumerate}

\linie
\pagebreak

\textbf{Lösen eines LGS durch QR-Faktorisierung}:
Ein LGS $Ax = b$ mit einer $m \times n$-Matrix $A$ kann mithilfe der
QR-Zerlegung $A(:,I) = QR$ gelöst werden.
Nach Anwendung der Householder-Transformationen auf $A(:,I)$
(d.\,h. Multiplikation mit $Q^{-1} = Q^t$) hat das System
$A(:,I) x(I) = b$ die Form
\begin{align*}
    A(:,I) x(I) = b \;\Leftrightarrow\;
    Ry = Q^t b \;\Leftrightarrow\;
    \begin{pmatrix}\widetilde{R} & S \\ 0 & 0\end{pmatrix}
    \begin{pmatrix}u \\ v\end{pmatrix}
    = \begin{pmatrix}c \\ d\end{pmatrix}, \qquad
    R = \begin{pmatrix}\widetilde{R} & S \\ 0 & 0\end{pmatrix},
\end{align*}
wobei $\widetilde{R}$ eine quadratische, invertierbare, obere Dreiecksmatrix
(Zeilen-/Spaltenzahl $= \rg A$), $y = x(I)$,
$u = y(1:k)$, $v = y(k+1:n)$ und
$Q^t b = \begin{pmatrix}c \\ d\end{pmatrix}$ ist.

\textbf{Lösungen}:
Eine Lösung existiert genau dann, wenn $d = 0$ ist.
In diesem Fall kann sie durch Rückwärtseinsetzen bzw. Lösung des Systems
$\widetilde{R}u + Sv = c$ berechnet werden.
Für $k = n$ ist die Lösung ist eindeutig, für $k < n$ sind die Komponenten
$y(k+1:n)$ frei wählbar. \\
Für $d \not= 0$ ist das LGS nicht lösbar.

\textbf{Fehlerminimierung}:
Da orthogonale Transformationen (also auch Householder-Transforma\-tionen)
die $2$-Norm invariant lassen, kann man durch die ermittelte Faktorisierung und
den Fehler $e = \norm{Ax - b}_2 = \norm{Q^tAx - Q^tb}_2$ minimieren.
Der Vektor $x$, der den Fehler $e$ minimiert, ergibt sich durch obige Lösung
$\widetilde{R}u + Sv = c$.
Der minimale Fehler ist dann $e_{min} = \norm{d}_2$.

\textbf{reguläres System}:
Ist $Ax = b$ mit einer invertierbaren $n \times n$-Matrix $A$,
dann ist die Zei\-len"~/Spaltenzahl von $\widetilde{R}$ gleich $\rg A = n$.
Die QR-Zerlegung $A(:,I) = QR$ ergibt in diesem Fall $R = \widetilde{R}$
als invertierbare obere Dreiecksmatrix ($n \times n$).

\subsection{%
    \textsc{Padé}-Approximation%
}

\textbf{\textsc{Padé}-Approximation}:
Die Padé-Approximation einer Funktion $f(x) = f_0 + f_1 x + f_2 x^2 + \dotsb$
ist eine rationale Funktion
\begin{align*}
    \frac{p(x)}{q(x)} \approx f(x), \qquad
    p(x) = p_0 + p_1 x + \dotsb + p_m x^m, \quad
    q(x) = q_0 + q_1 x + \dotsb + q_m x^m
\end{align*}
mit Zählergrad $m$ und Nennergrad $n$,
die mit $f$ für $z \to 0$ in den Termen der Ordnung bis einschließlich $m + n$
übereinstimmt, d.\,h. $\frac{p(x)}{q(x)} - f(x) = \O(x^{m+n+1})$.

\textbf{Beispiel für $m = n = 2$}:
Zur Bestimmung der Koef"|fizienten von $p$ und $q$ kann man $f$ z.\,B. in
eine Taylor-Reihe $f(x) = f_0 + f_1 x + f_2 x^2 + \dotsb$ im Punkt $0$
entwickeln, damit $f$ eine Potenzreihe ist.
Durch Ausmultiplizieren von $f(x) q(x) \approx p(x)$ bzw. \\
$(f_0 + f_1 x + f_2 x^2 + \dotsb) (1 + q_1 x + q_2 x^2) =
(p_0 + p_1 x + p_2 x^2)$ ($q_0$ kann durch Kürzen immer auf $1$ gesetzt werden)
und Sammeln der Terme gleicher Ordnung auf der linken Seite entsteht ein
LGS mit den $m + n + 1$ Unbestimmten $p_0, p_1, p_2, q_1, q_2$
in $m + n + 1$ Gleichungen, wobei $f_0, f_1, f_2, f_3, f_4$ auf der
rechten Seite steht.

\pagebreak
