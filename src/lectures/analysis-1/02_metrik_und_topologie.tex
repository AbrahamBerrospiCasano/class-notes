\section{%
    Metrik und Topologie in den Räumen
    \texorpdfstring
    {$\mathbb{R}$, $\mathbb{C}$, $\mathbb{R}^n$, $\mathbb{C}^n$}%
    {ℝ, ℂ, ℝⁿ, ℂⁿ}%
}

\subsection{%
    \texorpdfstring{Konvergenz in $\mathbb{R}$}{Konvergenz in ℝ}%
}

\textbf{Intervalle} ($a, b \in \mathbb{R}$): \\
$\left]a,b\right[ = \left(a,b\right) =
\{x \in \mathbb{R} \;|\; a < x < b\} \quad$ \emph{of"|fen}, \\
$\left[a,b\right] =
\{x \in \mathbb{R} \;|\; a \le x \le b\} \quad$ \emph{abgeschlossen}, \qquad
$\left(a,b\right] = \left]a,b\right]$,
$\left[a,b\right) = \left[a,b\right[ \quad$ \emph{halbof"|fen}

\textbf{$\varepsilon$-Umgebung}: $U_\varepsilon(x) =
(x - \varepsilon, x + \varepsilon) =
\{y \in \mathbb{R} \;|\; |x - y| < \varepsilon\} \quad$
($x \in \mathbb{R}$, $\varepsilon > 0$)

\textbf{Konvergenz reeller Folgen}:
$\{a_n\}_{n \in \mathbb{N}} \xrightarrow{n \to \infty} a \;\Leftrightarrow\;
\forall_{\varepsilon > 0} \exists_{N(\varepsilon) \in \mathbb{N}}
\forall_{n \ge N(\varepsilon)}\; a_n \in U_\varepsilon(a)$

\textbf{Beschränktheit reeller Teilmengen}:
$\exists_{C > 0} \forall_{x \in M}\; |x| \le C \quad$ ($M \subset \mathbb{R}$)

\emph{Lemma}: Jede unbeschränkte Menge $M \subset \mathbb{R}$ ist
transfinit. \\
$\Rightarrow\;$ Jede endliche Menge ist beschränkt.

\emph{Satz}: Sei $\{a_n\}_{n \in \mathbb{N}}$ mit $a_n \in \mathbb{R}$,
$\lim_{n \to \infty} a_n = a \in \mathbb{R}$. \\
1. $\bigcup_{n \in \mathbb{N}} \{a_n\} = M$ ist beschränkt.
2. Für jede Teilfolge $\{a_j\}$ gilt ${a_j}_k \xrightarrow{n \to \infty} a$.

Es gelten die \textbf{Grenzwertsätze}
($\lim_{n \to \infty} a_n = a \in \mathbb{R}$,
$\lim_{n \to \infty} b_n = b \in \mathbb{R}$): \\
1. $\lim_{n \to \infty} (a_n + b_n) = a + b \qquad$
2. $\lim_{n \to \infty} (a_n \cdot b_n) = a \cdot b$ \\
3. $\lim_{n \to \infty} \frac{a_n}{b_n} = \frac{a}{b}$
($b_n, b \not= 0$) $\qquad$
4. $\lim_{n \to \infty} |a_n| = |a|$ \\
Außerdem gilt
$(\forall_{n \in \mathbb{N}}\; a_n \le b_n) \;\Rightarrow\; a \le b$.

\textbf{Satz der zwei Polizisten}: Seien $\{a_n\}$, $\{b_n\}$, $\{c_n\}$
reelle Folgen mit $a_n, b_n \xrightarrow{n \to \infty} a \in \mathbb{R}$. \\
Dann gilt
$(\forall_{n \ge N}\; a_n \le c_n \le b_n) \;\Rightarrow\;
\lim_{n \to \infty} c_n = a$.

\subsection{%
    \texorpdfstring{$\mathbb{R}$ als metrischer Raum}{ℝ als metrischer Raum}%
}

Sei $M$ Menge, $d: M \times M \rightarrow \mathbb{R}$ Funktion. \\
Dann heißt $d$ \textbf{Abstandsfunktion (Metrik)}, falls folgende Axiome
erfüllt sind:

\begin{itemize}
    \item[(1)] $d(x,y) \ge 0$, $\quad d(x,y) = 0 \;\Leftrightarrow\; x = y$
    \item[(2)] $d(x,y) = d(y,x)$
    \item[(3)] $d(x,z) \le d(x,y) + d(y,z)$
\end{itemize}

$(M,d)$ heißt \textbf{metrischer Raum}. Bspw. ist $(\mathbb{R},d_{|\cdot|})$
mit $d_{|\cdot|}(x,y) = |x - y|$ metrischer Raum.

\textbf{triviale Metrik}:
$M \not= \emptyset$, $d(x,y) = 0$ falls $x = y$, $d(x,y) = 1$ falls $x \not= y$

\textbf{$\varepsilon$-Umgebung}:
$U_\varepsilon(x) = \{y \in M \;|\; d(x,y) < \varepsilon\}$
($x \in M$, $\varepsilon > 0$)

\linie

\textbf{Konvergenz im Sinne der Metrik}: $x_n, x \in M$ \\
$x \overset{(M,d)}{=} \lim_{n \to \infty} x_n \; (x_n \xrightarrow{d} x)
\;\Leftrightarrow\;
\forall_{\varepsilon > 0} \exists_{N(\varepsilon) \in \mathbb{N}}
\forall_{n \ge N(\varepsilon)}\; x_n \in U_\varepsilon(x)$

\begin{enumerate}
    \item Wenn eine Folge $x_n \in M$
    konvergiert, dann hat sie genau einen Grenzwert.
    
    \item $a_n \xrightarrow{(M,d)} a \;\Leftrightarrow\;
    d(a_n, a) \xrightarrow{\mathbb{R}} 0$
    
    \item $M' \subset M$ heißt beschränkt $\;\Leftrightarrow\;
    \exists_{a \in M} \exists_{C \in \mathbb{R}} \forall_{a' \in M'}\;
    d(a, a') \le C$
\end{enumerate}

\textbf{Cauchy-Folge}:
$\{a_n\}_{n \in \mathbb{N}} \in \CF((M,d)) \;\Leftrightarrow\;
\forall_{\varepsilon > 0} \exists_{N(\varepsilon) \in \mathbb{N}}
\forall_{n,m \ge N(\varepsilon)}\; d(a_n, a_m) < \varepsilon$

Eine \textbf{konvergente Folge ist auch eine Cauchy-Folge}, d.\,h. \\
$a_n \xrightarrow{(M,d)} a \;\Rightarrow\;
\{a_n\}_{n \in \mathbb{N}} \in \CF((M,d))$.

\pagebreak

Die Umkehrung ist nicht immer wahr. Ein metrischer Raum $(M,d)$ heißt
\textbf{vollständig}, falls jede Cauchy-Folge $\{a_n\}_{n \in \mathbb{N}}$
aus $M$ auch einen Grenzwert $a$ in $M$ besitzt.

\linie

\textbf{Satz von \textsc{Cauchy}}: $(\mathbb{R},d_{|\cdot|})$ ist vollständig,
d.\,h. eine Folge reeller Zahlen $\{a_n\}_{n \in \mathbb{N}}$ konvergiert genau
dann gegen ein $a \in \mathbb{R}$, wenn
$\{a_n\}_{n \in \mathbb{N}} \in \CF((\mathbb{R},d_{|\cdot|}))$.

Der Beweis erfolgt basierend auf den Lemmas $\{r_{n+n_0}\} \in x$ und
$\{|r_n|\} \in |x|$ (wenn $x \in \mathbb{R}$, $\{r_n\} \in x$). Außerdem
gilt in diesem Fall $\lim_{n \to \infty} r_n \overset{\mathbb{R}}{=} x$. Der
Beweis des Satzes von \textsc{Cauchy} \\
($\{x_n\}_{n \in \mathbb{N}} \in \CF(\mathbb{R}) \;\Rightarrow\;
\exists_{y \in \mathbb{R}}\; y = \lim_{n \to \infty} x_n$)
wird anschließend in drei Schritte aufgeteilt:

\begin{itemize}
    \item \emph{Schritt 1}: Konstruktion eines "`Kandidaten"'
    $\{q_n\}_{n \in \mathbb{N}}$, $q_n \in \mathbb{Q}$
    
    \item \emph{Schritt 2}: $\{q_n\} \in \CF(\mathbb{Q})$, d.\,h.
    $\exists_{y \in \mathbb{R}}\; y \ni \{q_n\}$
    
    \item \emph{Schritt 3}: $\lim_{n \to \infty} x_n = y$
\end{itemize}

\linie

\textbf{Monotonie von reellen Folgen}:
$\{x_n\}$ wächst monoton, d.\,h. $\{x_n\}\!\!\uparrow$
$\;\Leftrightarrow\; \forall_{n \in \mathbb{N}}\; x_n \le x_{n+1}$ \\
$\{x_n\}$ wächst streng monoton, d.\,h. $\{x_n\}\!\!\upuparrows$
$\;\Leftrightarrow\; \forall_{n \in \mathbb{N}}\; x_n < x_{n+1}$,
analog $\{x_n\}\!\!\downarrow$, $\{x_n\}\!\!\downdownarrows$

\textbf{Beschränktheit von reellen Folgen}: $\{x_n\}$ ist beschränkt
$\;\Leftrightarrow\;
\exists_{C \in \mathbb{R}} \forall_{n \in \mathbb{N}}\; |x_n| \le C$

\textbf{Satz}: Jede monotone, beschränkte Folge reeller Zahlen besitzt einen
reellen Grenzwert.

\subsection{%
    Maximum, Minimum, Infimum, Supremum%
}

$M \subset \mathbb{R}$, $M \not= \emptyset$, $a \in \mathbb{R}$

$a = \max{M} \;\Leftrightarrow\;
(a \in M) \land (\forall_{x \in M}\; x \le a) \quad$ \textbf{Maximum} \\
$a = \min{M} \;\Leftrightarrow\;
(a \in M) \land (\forall_{x \in M}\; x \ge a) \quad$ \textbf{Minimum}

$c \in \mathbb{R}$ heißt \textbf{obere Schranke} von
$M \;\Leftrightarrow\; \forall_{x \in M}\; x \le c$ \\
$c \in \mathbb{R}$ heißt \textbf{untere Schranke} von
$M \;\Leftrightarrow\; \forall_{x \in M}\; x \ge c$

$M_{+}$ Menge aller oberen Schranken, $M_{-}$ Menge aller unteren Schranken \\
M ist \textbf{beschränkt nach oben}
$\;\Leftrightarrow\; M_{+} \not= \emptyset$,
M ist \textbf{beschränkt nach unten}
$\;\Leftrightarrow\; M_{-} \not= \emptyset$

$a = \sup{M} \;\Leftrightarrow\;
(M_{+} \not= \emptyset) \land (a = \min{M_{+}}) \quad$ \textbf{Supremum} \\
$a = \inf{M} \;\Leftrightarrow\;
(M_{-} \not= \emptyset) \land (a = \max{M_{-}}) \quad$ \textbf{Infimum}

\textbf{Satz}: Mengen, die nach oben/unten beschränkt sind, haben ein
Supremum/Infimum, d.\,h. \\
$M_{+} \not= \emptyset \;\Rightarrow\;
\exists_{a_{+} \in \mathbb{R}}\; a_{+} = \sup{M}\quad $ bzw.
$\quad M_{-} \not= \emptyset \;\Rightarrow\;
\exists_{a_{-} \in \mathbb{R}}\; a_{-} = \inf{M}$.

\subsection{%
    \texorpdfstring{Die Eulersche Zahl $e$}{Die Eulersche Zahl ℯ}%
}

\textbf{Fakultät}: $n! \overset{\text{def.}}{=}
1 \cdot 2 \cdot \cdots \cdot (n-1) \cdot n$,
$\quad 0! \overset{\text{def.}}{=} 1$

{\large $x_n = \sum_{k=0}^{n} \frac{1}{k!} \;=\; \frac{1}{0!} + \frac{1}{1!} +
\frac{1}{2!} + \cdots + \frac{1}{n!}$}

\begin{itemize}
    \item \emph{Satz 1}: $\exists \lim_{n \to \infty} x_n$
    in $\mathbb{R}$. \\
    \textbf{Definition der Eulerschen Zahl}: $e = \lim_{n \to \infty} x_n$
    
    \item \emph{Satz 2}: Für $n \ge 2$ gilt
    $x_n < e < x_n \;+ $ {\large $\frac{1}{n \cdot n!}$}.
    
    \item \emph{Satz 3}: $e$ ist irrational, d.\,h. $e \notin \mathbb{Q}$.
    
    \item \emph{Satz 4}: $y_n =$ {\large $(1 + \frac{1}{n})^n$},
    $n \in \mathbb{N}$  $\;\Rightarrow\; e = \lim_{n \to \infty} y_n$.
\end{itemize}

\subsection{%
    Einige wichtige Grenzwerte%
}

{\large
\begin{tabular}{p{3.8cm}p{3.8cm}p{3.8cm}p{3.8cm}}
    $\lim_{n \to \infty} \frac{n}{a^n} = 0$ & $a > 1$ &
    $\lim_{n \to \infty} \frac{n^k}{a^n} = 0$ & $a > 1,$ $k \in \mathbb{N}$ \\
    $\lim_{n \to \infty} \frac{1}{\sqrt[n]{n!}} = 0$ & &
    $\lim_{n \to \infty} \frac{a^n}{n!} = 0$ & $a > 0$ \\
    $\lim_{n \to \infty} n^k a^n = 0$ & $|a| < 1,$ $k \in \mathbb{N}$ &
    $\lim_{n \to \infty} \sqrt[n]{a} = 1$ & $a > 0$ \\
    $\lim_{n \to \infty} \frac{\log_a{n}}{n} = 0$ & $a > 1$ &
    $\lim_{n \to \infty} \sqrt[n]{n} = 1$
\end{tabular}
}

\vspace{3mm}
\linie

\textbf{bestimmte Divergenz}:

\begin{itemize}
    \item $\lim_{n \to \infty} x_n = +\infty \;\Leftrightarrow\;
    \forall_{C > 0} \exists_{N(C) \in \mathbb{N}} \forall_{n \ge N(C)}\;
    x_n \ge C$
    
    \item $\lim_{n \to \infty} x_n = -\infty \;\Leftrightarrow\;
    \forall_{C > 0} \exists_{N(C) \in \mathbb{N}} \forall_{n \ge N(C)}\;
    x_n \le -C$
    
    \item $\lim_{n \to \infty} x_n = \infty \;\Leftrightarrow\;
    \forall_{C > 0} \exists_{N(C) \in \mathbb{N}} \forall_{n \ge N(C)}\;
    |x_n| \ge C$
\end{itemize}

\subsection{%
    \texorpdfstring{Der euklidische Raum $\mathbb{R}^n$}%
    {Der euklidische Raum ℝⁿ}%
}

$\mathbb{R}^n = \mathbb{R} \times \cdots \times \mathbb{R}$,
$\qquad x = (x_1, \ldots, x_n) \quad$ $x_j \in \mathbb{R}$,
$j = 1, \ldots, n$

$x, y \in \mathbb{R}^n \;\rightarrow\; x + y = (x_1 + y_1, \ldots, x_n + y_n)$,
$\quad x \in \mathbb{R}^n$, $\alpha \in \mathbb{R} \;\rightarrow\;
\alpha \cdot x = (\alpha \cdot x_1, \ldots, \alpha \cdot x_n)$

\textbf{algebraische Struktur}: \\
$X = \mathbb{R}^n$, $\mathbb{K} = \mathbb{R}$,
$\quad \boldsymbol{+}: X \times X \rightarrow X$, $\;\boldsymbol{\cdot}:
\mathbb{K} \times X \rightarrow X$
erfüllen die \emph{Vektorraum-Axiome}:

\begin{tabular}{p{8cm}p{8cm}}
    (1) $x + y = y + x$ &
    (5) $1 \cdot x = x$ ($1 \in \mathbb{K}$) \\
    (2) $(x + y) + z = x + (y + z)$ &
    (6) $\alpha (\beta x) = (\alpha \beta) x$ \\
    (3) $\exists_{0 \in X}\; 0 + x = x$ für alle $x \in X$ &
    (7) $(\alpha + \beta) x = \alpha x + \beta x$ \\
    (4) $\forall_{x \in X} \exists_{-x \in X}\; x + (-x) = 0$ &
    (8) $\alpha (x + y) = \alpha x + \alpha y$
\end{tabular}

\vspace{12pt}
\linie

\textbf{euklidische Struktur (Skalarprodukt/inneres Produkt)}:
$X$ Vektorraum über $\mathbb{R}$ \\
$\langle \cdot, \cdot \rangle:
X \times X \rightarrow \mathbb{R}$ heißt \emph{(reelles) Skalarprodukt}, falls
folgende Eigenschaften erfüllt sind:

\begin{itemize}
    \item[(1)] $\langle x, x \rangle \ge 0$,
    $\qquad \langle x, x \rangle = 0 \Leftrightarrow x = 0$
    
    \item[(2)] $\langle x, y \rangle = \langle y, x \rangle$
    
    \item[(3)] $\langle \alpha' x' + \alpha'' x'', y \rangle =
    \alpha' \langle x', y \rangle + \alpha'' \langle x'', y \rangle$
    \qquad ($\alpha', \alpha'' \in \mathbb{R}$)
\end{itemize}

\emph{Kanonisches Skalarprodukt im $\mathbb{R}^n$}:
$\langle x, y \rangle = x_1 y_1 + \cdots + x_n y_n$ \\
$(X, \langle \cdot, \cdot \rangle)$ heißt \emph{euklidischer Raum}.

\linie

\textbf{Struktur des normierten Raumes}:
$X$ Vektorraum über $\mathbb{R}$ \\
$\Vert \cdot \Vert: X \rightarrow \mathbb{R}$ heißt \emph{Norm}, falls
folgende Eigenschaften erfüllt sind:

\begin{itemize}
    \item[(1)] $\Vert x \Vert \ge 0$,
    $\qquad \Vert x \Vert = 0 \Leftrightarrow x = 0$
    
    \item[(2)] $\Vert \alpha x \Vert = |\alpha| \Vert x \Vert$
    \qquad ($\alpha \in \mathbb{R}$)
    
    \item[(3)] $\Vert x + y \Vert \le \Vert x \Vert + \Vert y \Vert$
\end{itemize}

Falls auf $X$ ein (reelles) Skalarprodukt gegeben ist, so definiert
$\Vert x \Vert = \sqrt{\langle x, x \rangle} \ge 0$ die zum Skalarprodukt
\emph{kanonische Norm} und erfüllt somit automatisch die Normeigenschaften \\
(für $X = \mathbb{R}^n$ ist $\Vert x \Vert_{\mathbb{R}^n} =
\sqrt{x_1^2 + \cdots + x_n^2}$). \\
Für den Beweis ist die \emph{Ungleichung von
\textsc{Cauchy}-\textsc{Schwarz}-\textsc{Bunjakowskij}} (\textbf{CSB})
wichtig: \\
$|\langle x, y \rangle| \le \Vert x \Vert \cdot \Vert y \Vert$

$d_{\Vert \cdot \Vert}(x,y) = \Vert x - y \Vert$ ist eine Abstandsfunktion,
$(\mathbb{R}^n, d_{\Vert \cdot \Vert})$ metrischer Raum.

\subsection{%
    \texorpdfstring{Der Raum $\mathbb{C}^n$}{Der Raum ℂⁿ}%
}

$\mathbb{C}^n = \mathbb{C} \times \cdots \times \mathbb{C}$,
$\qquad z = (z_1, \ldots, z_n) \quad$ $z_j \in \mathbb{C}$,
$j = 1, \ldots, n$

$z, w \in \mathbb{C}^n \;\rightarrow\; z + w = (z_1 + w_1, \ldots, z_n + w_n)$,
$\quad z \in \mathbb{C}^n$, $\alpha \in \mathbb{C} \;\rightarrow\;
\alpha \cdot z = (\alpha \cdot z_1, \ldots, \alpha \cdot z_n)$

$X = \mathbb{C}^n$, $\mathbb{K} = \mathbb{C}$ Vektorraum über
$\mathbb{K} = \mathbb{C}$, Axiome (1) -- (8) erfüllt

\linie

\textbf{hermitesche Struktur (komplexes Skalarprodukt)}:
$X$ Vektorraum über $\mathbb{C}$ \\
$\langle \cdot, \cdot \rangle:
X \times X \rightarrow \mathbb{C}$ heißt \emph{(komplexes) Skalarprodukt},
falls folgende Eigenschaften erfüllt sind:

\begin{itemize}
    \item[(1)] $\langle z, z \rangle \ge 0$,
    $\qquad \langle z, z \rangle = 0 \Leftrightarrow z = 0$
    
    \item[(2)] $\langle z, w \rangle = \overline{\langle w, z \rangle}$
    
    \item[(3)] $\langle \alpha' z' + \alpha'' z'', w \rangle =
    \alpha' \langle z', w \rangle + \alpha'' \langle z'', w \rangle$
    \qquad ($\alpha', \alpha'' \in \mathbb{C}$)
\end{itemize}

\emph{Kanonisches Skalarprodukt im $\mathbb{C}^n$}:
$\langle z, w \rangle = z_1 \overline{w_1} + \cdots + z_n \overline{w_n}$

\linie

\textbf{normierter Raum}:
$X$ Vektorraum über $\mathbb{C}$ \\
$\Vert \cdot \Vert: X \rightarrow \mathbb{R}$ heißt \emph{Norm}, falls
folgende Eigenschaften erfüllt sind:

\begin{itemize}
    \item[(1)] $\Vert z \Vert \ge 0$,
    $\qquad \Vert z \Vert = 0 \Leftrightarrow z = 0$
    
    \item[(2)] $\Vert \alpha z \Vert = |\alpha| \Vert z \Vert$
    \qquad ($\alpha \in \mathbb{C}$)
    
    \item[(3)] $\Vert z + w \Vert \le \Vert z \Vert + \Vert w \Vert$
\end{itemize}

Für $\Vert z \Vert = \sqrt{\langle z, z \rangle}$ sind automatisch die
Normeigenschaften erfüllt, wobei die Dreiecksungleichung auf
$|\langle z, w \rangle| \le \Vert z \Vert \cdot \Vert w \Vert$ (CSB) basiert.

\linie

Vektoren des $\mathbb{C}^n$ können als $n$-Tupel komplexer Zahlen
$z_j = x_j + iy_j$ dargestellt werden: \\
$\mathbb{C}^n \ni z = (z_1, \ldots, z_n) =
(x_1 + iy_1, \ldots, x_n + iy_n) \quad$
(wobei $x_j, y_j \in \mathbb{R}$). \\
Nun können $x_j$, $y_j$ auch als Elemente von $\mathbb{R}^{2n}$ angesehen
werden:
$\mathbb{R}^{2n} \ni (x_1, y_1, x_2, y_2, \ldots, x_n, y_n)$ \\
$\Vert z \Vert^2 = \sum_{j=1}^n z_j \overline{z_j} =
\sum_{j=1}^n (x_j^2 + y_j^2) = \sum_{j=1}^n x_j^2 + \sum_{j=1}^n y_j^2$

Bzgl. der Addition von Vektoren und der Norm ist es unerheblich, ob man die
Vektoren als $n$-Tupel komplexer Zahlen oder als $2n$-Tupel reeller Zahlen
betrachtet ($\mathbb{C}^n$ und $\mathbb{R}^{2n}$ \textbf{isomorph}).
Dies gilt nicht mehr für die Multiplikation mit Skalaren
(dort sind $\mathbb{C}^n$ und $\mathbb{R}^{2n}$ verschieden).

\subsection{%
    \texorpdfstring{Konvergenz im $\mathbb{R}^n$ und $\mathbb{C}^n$}%
    {Konvergenz im ℝⁿ und ℂⁿ}%
}

$(X, \Vert \cdot \Vert)$ normierter Raum, z.\,B.
$(\mathbb{R}^n, \Vert \cdot \Vert_{\mathbb{R}^n})$ oder
$(\mathbb{C}^n, \Vert \cdot \Vert_{\mathbb{C}^n})$.

$d(x,y) = \Vert x - y \Vert \quad$ ($x, y \in X$) ist Abstandsfunktion auf $X$.
Daher ist $(X, d_{\Vert \cdot \Vert})$ ein \textbf{metrischer Raum} mit
induzierter Abstandsfunktion
$d_{\Vert \cdot \Vert}(x, y) = \Vert x - y \Vert$.

Damit lassen sich automatisch auch die Definitionen
\emph{$\varepsilon$-Umgebung}, \emph{Konvergenz} und \emph{Fundamentalfolge}
auf $X$ übertragen:

\textbf{$\varepsilon$-Umgebung}: $y \in X$;\; $U_\varepsilon(y) =
\{x \in X \;|\; \Vert x - y \Vert < \varepsilon\}$ \\
\textbf{Konvergenz}:
$x_m \xrightarrow{\Vert \cdot \Vert} y \quad\Leftrightarrow\quad
\forall_{\varepsilon > 0} \exists_{N_\varepsilon}
\forall_{n \ge N_\varepsilon}\;
x_m \in U_\varepsilon(y)$, \;d.\,h. $\Vert x_m - y \Vert < \varepsilon$ \\
\textbf{Fundamentalfolge}:
$\{x_m\}_{m \in \mathbb{N}} \in \CF(X, \Vert \cdot \Vert)
\quad\Leftrightarrow\quad \forall_{\varepsilon > 0} \exists_{N_\varepsilon}
\forall_{n, m \ge N_\varepsilon}\; \Vert x_m - x_n \Vert < \varepsilon$

\textbf{Beschränktheit}:
$\{x_m\}_{m \in \mathbb{N}}$ ($x_m \in X$) ist beschränkt
$\quad\Leftrightarrow\quad \exists_{C} \forall_{m \in \mathbb{N}}\;
\Vert x_m \Vert \le C$

\linie
\pagebreak

\textbf{Satz}: \quad
$\{x_m\}_{m \in \mathbb{N}}$, $x_m \in \mathbb{K}^n$,
$y', y'' \in \mathbb{K}^n$

\begin{enumerate}
    \item $\quad (x_m \xrightarrow{\Vert \cdot \Vert} y') \land
    (x_m \xrightarrow{\Vert \cdot \Vert} y'') \;\Rightarrow\; y' = y''$
    
    \item $\quad y' = \lim_{n \to \infty} x_m \;\Leftrightarrow\;
    y' = \lim_{n \to \infty} x_{m + m_0}$
    
    \item $\quad y' = \lim_{n \to \infty} x_m \;\Rightarrow\;
    \{x_m\}_{m \in \mathbb{N}}$ beschränkt
\end{enumerate}

\textbf{Grenzwertsätze im $\mathbb{K}^n$}: \quad
$\{x_m'\}_{m \in \mathbb{N}}$, $\{x_m''\}_{m \in \mathbb{N}}$,
$x_m', x_m'' \in \mathbb{K}^n$, \\
$\{\alpha_k\}_{k \in \mathbb{N}}$,
$\alpha_k \in \mathbb{K}$, \quad
$y', y'' \in \mathbb{K}^n$,
$\beta \in \mathbb{K}$, \quad
$x_m' \xrightarrow{\Vert \cdot \Vert} y'$,
$x_m'' \xrightarrow{\Vert \cdot \Vert} y''$,
$\alpha_k \xrightarrow{|\cdot|} \beta$

\begin{enumerate}
    \item $\quad \lim_{m \to \infty} (x_m' + x_m'')
    \;\overset{\mathbb{K}^n}{=}\; y' + y''$
    
    \item $\quad \lim_{m \to \infty} (\alpha_m x_m')
    \;\overset{\mathbb{K}^n}{=}\; \beta y'$
    
    \item $\quad \lim_{m \to \infty}
    \langle x_m', x_m'' \rangle_{\mathbb{K}^n}
    \;\overset{\mathbb{K}}{=}\; \langle y', y'' \rangle$
\end{enumerate}

\linie

\emph{Schreibweise}:
$x_m = (\xi_m^{(1)}, \ldots, \xi_m^{(n)}) \in \mathbb{K}^n$,
$\xi_m^{(j)} \in \mathbb{K}$ \\
\textbf{Projektion auf die $j$-te Komponente}:
$\pi_j: \mathbb{K}^n \rightarrow \mathbb{K}$,
$\pi_j(x) = \pi_j(\xi^{(1)}, \ldots, \xi^{(n)}) = \xi^{(j)}$ \\
es gilt:
$\pi_j(\alpha'x' + \alpha''x'') = \alpha'\pi_j(x') + \alpha''\pi_j(x'')$

\textbf{Basisvektoren}: $e_j = (0, \ldots, 0, 1, 0, \ldots, 0)$ mit der $1$
an der $j$-ten Stelle, sonst $0$, $\Vert e_j \Vert_{\mathbb{K}^n} = 1$ \\
mit $x = \sum_{j=1}^n \pi_j(x) e_j$ folgt
$|\xi^{(j)}| = |\pi_j(x)| \le \Vert x \Vert \le \sum_{j=1}^n |\xi^{(j)}|$

\textbf{Satz (Konvergenz)}:
$y \overset{\mathbb{K}^n}{=} \lim_{m \to \infty} x_m \quad\Leftrightarrow\quad
\forall_{j=1,\ldots,n}\; \pi_j(y) \overset{\mathbb{K}}{=}
\lim_{m \to \infty} \pi_j(x_m)$

\textbf{Satz (Cauchy-Folgen)}:
$\{x_m\}_{m \in \mathbb{N}} \in \CF(\mathbb{K}^n)
\quad\Leftrightarrow\quad
\forall_{j=1,\ldots,n}\; \{\pi_j(x_m)\}_{m \in \mathbb{N}} \in
\CF(\mathbb{K})$

\textbf{Folgerung}: $\mathbb{R}^n$ und $\mathbb{C}^n$ sind vollständig.

\subsection{%
    Of"|fene und abgeschlossene Mengen%
}

$(M,d)$ metrischer Raum, $X \subset M$

\begin{itemize}
    \item \textbf{Häufungspunkt}:
    $x_0 \in M$ heißt Häufungspunkt (HP) von $X$ bzw. $x_0 \in \acc(X)$ \\
    $\Leftrightarrow\quad \forall_{\varepsilon > 0}\;\;
    U_\varepsilon(x_0) \cap (X \setminus \{x_0\}) \not= \emptyset$
    
    \item \textbf{isolierter Punkt}:
    $x_0 \in X$ ist ein isolierter Punkt von $X$ bzw. $x_0 \in \iso(X)$ \\
    $\Leftrightarrow\quad \exists_{\varepsilon > 0}\;\;
    U_\varepsilon(x_0) \cap (X \setminus \{x_0\}) = \emptyset$
    \qquad d.\,h. $\iso(X) = X \setminus \acc(X)$
    
    \item \textbf{innerer Punkt}:
    $x_0 \in X$ heißt innerer Punkt von $X$ bzw. $x_0 \in \interior(X)$ \\
    $\Leftrightarrow\quad
    \exists_{\varepsilon > 0}\;\; U_\varepsilon(x_0) \subset X$
    
    \item \textbf{äußerer Punkt}:
    $x_0 \in M$ heißt äußerer Punkt zu $X$ bzw. $x_0 \in \exterior(X)$ \\
    $\Leftrightarrow\quad
    \exists_{\varepsilon > 0}\;\; U_\varepsilon(x_0) \subset X_M^c
    \quad\Leftrightarrow\quad
    \exists_{\varepsilon > 0}\;\; U_\varepsilon(x_0) \cap X = \emptyset$
    
    \item \textbf{Randpunkt}:
    $x_0 \in M$ heißt Randpunkt von $X$ bzw. $x_0 \in \partial X$ \\
    $\Leftrightarrow\quad
    (x_0 \notin \interior(X)) \land (x_0 \notin \exterior(X))
    \quad\Leftrightarrow\quad \forall_{\varepsilon > 0}\;\;
    (U_\varepsilon(x_0) \cap X_M^c \not= \emptyset) \land
    (U_\varepsilon(x_0) \cap X \not= \emptyset)$
\end{itemize}

\linie

$\interior(X)$, $\exterior(X)$, $\partial X$ sind paarweise disjunkt und
$M = \interior(X) \cup \partial X \cup \exterior(X)$. \\
Dabei gilt $X \subset \interior(X) \cup \partial X$,
$\quad X_M^c \subset \exterior(X) \cup \partial X$ \quad sowie
$\partial X = \partial X_M^c$, da $(X_M^c)_M^c = X$.

\emph{Lemma}: $\interior(X) = X \setminus \partial X$,
$\quad\exterior(X) = X_M^c \setminus \partial X$ \\
$X \cup \acc(X) = X \cup \partial X = \interior(X) \cup \partial X$

Sei $X \subset X_1 \subset M$, dann gilt auch $\acc(X) \subset \acc(X_1)$,
$\interior(X) \subset \interior(X_1)$ und
$\exterior(X) \supset \exterior(X_1)$
(über isolierte Punkte und den Rand ist keine Aussage möglich).

\linie
\pagebreak

\textbf{of"|fene und abgeschlossene Mengen}:

\begin{itemize}
    \item $X$ ist \emph{of"|fen} in $(M,d)$
    $\quad\Leftrightarrow\quad X = \interior(X) \quad\Leftrightarrow\quad
    X \cap \partial X = \emptyset$
    
    \item $X$ ist \emph{abgeschlossen} in $(M,d)$ \\
    $\Leftrightarrow\quad X \cup \partial X = X = 
    \partial X \cup \interior(X) = X \cup \acc(X) \quad\Leftrightarrow\quad
    \acc(X) \subset X$
\end{itemize}

\textbf{Satz}: $X$ of"|fen $\Leftrightarrow X_M^c$ abgeschlossen,
$\quad X$ abgeschlossen $\Leftrightarrow X_M^c$ of"|fen

\textbf{Familien von of"|fenen Mengen}:
$F_\alpha \subset M$ of"|fen, $\alpha \in A$ Indexmenge
$\;\Rightarrow\; F = \bigcup_{\alpha \in A} F_\alpha$ of"|fen \\
\emph{endlich viele Mengen}:
$F_k$ of"|fen, $k = 1, \ldots, n$ (endlich viele)
$\;\Rightarrow\; F = \bigcap_{k=1}^n F_k$ of"|fen

\textbf{Familien von abgeschlossenen Mengen}: \\
$G_\alpha \subset M$ abgeschlossen
$\;\Rightarrow\; G = \bigcap_{\alpha \in A} G_\alpha$ abgeschlossen \\
\emph{endlich viele Mengen}:
$G_k$ abgeschlossen, $k = 1, \ldots, n$
$\;\Rightarrow\; G = \bigcup_{k=1}^n G_k$ abgeschlossen

$\emptyset$ und $M$ sind sowohl abgeschlossen als auch of"|fen.

\linie

$\mathbb{R}^n$, $\mathbb{C}^n$ als \textbf{topologische Räume}: \\
$2^M$ Menge aller Teilmengen aus $M$, $T \subset 2^M$ nennt man
\textbf{Topologie}, falls

\begin{itemize}
    \item[(1)] $\emptyset \in T$, $M \in T$
    
    \item[(2)] $\{F_\alpha\}_{\alpha \in A}$, $F_\alpha \in T$
    $\;\Rightarrow\; \bigcup_{\alpha \in A} F_\alpha \in T$
    
    \item[(3)] $\{F_k\}_{k=1}^n$, $F_k \in T$
    $\;\Rightarrow\; \bigcap_{k=1}^n F_k \in T$
\end{itemize}

$(M,T)$ heißt dann \textbf{topologischer Raum}, $F \in T$
\textbf{Umgebungen/of"|fene Mengen}. \\
Mit $M = \mathbb{R}^n$ oder $M = \mathbb{C}^n$, $T \subset 2^M$ sowie
$F \in T \;\Leftrightarrow\; F$ of"|fen ist eine Topologie definiert.

\emph{Lemma}: $\interior(X)$ ist of"|fen. \qquad
\emph{Folgerung}: $\exterior(X)$ ist of"|fen.

\linie

\textbf{Abschluss}: $\overline{X} = X \cup \partial X = X \cup \acc(X)$ ist
der Abschluss der Menge $X$.

\textbf{Sätze über den Abschluss}:
$\overline{X}$ ist abgeschlossen. \\
$\overline{X}$ ist die kleinste abgeschlossene Menge, die $X$
enthält, d.\,h.
$\overline{X} = \bigcap_{Y \supset X,\; Y \text{abgeschlossen}} Y$. \\
$X$ ist abgeschlossen $\;\Leftrightarrow\; X = \overline{X} \qquad$ sowie
$\qquad \overline{\overline{X}} = \overline{X}$. \\
$\overline{X}$ ist die Menge aller möglichen Grenzwerte für Folgen
$\{x_n\}_{n \in \mathbb{N}}$, $x_n \in X$.

\subsection{%
    Grenzwerte von Funktionen%
}

$(M_1,d_1)$, $(M_2,d_2)$ metrische Räume, $X \subset M_1$, $Y \subset M_2$,
$f: X \rightarrow Y$ Funktion von $X$ nach $Y$

\textbf{$\varepsilon$-$\delta$-Definition}:
Sei $x_0 \in \acc(X)$, $y \in Y$. \\
$y = \lim_{x \to x_0} f(x) \quad\Leftrightarrow\quad
\forall_{\varepsilon > 0} \exists_{\delta > 0}
\forall_{x \in X \cap U_\delta(x_0),\; x \not= x_0}\;
f(x) \in U_\varepsilon(y)$

\textbf{Folgendefinition}:
Sei $x_0 \in \acc(X)$, $y \in Y$. \\
$y = \lim_{x \to x_0} f(x) \quad\Leftrightarrow\quad
\forall_{\{x_k\} \xrightarrow{k \to \infty} x_0,\;
x_k \in X \setminus \{x_0\}}\;
y_k = f(x_k) \xrightarrow{k \to \infty} y$

\linie

\textbf{Satz}: $f: X \rightarrow Y$,
$x_0 \in \acc(X)$, $y_0 = \lim_{x \to x_0} f(x)$

\begin{enumerate}
    \item $y_0$ ist eindeutig bestimmt.
    
    \item Existenz/Wahl des Grenzwertes hängt nicht vom Verhalten von
    $f(x)$ für $d(x,x_0) \ge \varepsilon$ ab.
    
    \item $\{f(x) \;|\; x \in U_\delta(x_0) \cap X\}$ ist für geeignetes
    $\delta > 0$ beschränkt.
\end{enumerate}

\linie
\pagebreak

\textbf{Grenzwertsätze bei vektorwertigen Funktionen (Spezialfall)}:
$\mathbb{K} = \mathbb{R}$ oder $\mathbb{K} = \mathbb{C}$, \\
$f, g: X \subset M_1 \rightarrow \mathbb{K}^n$, \quad
$\alpha: X \subset M_1 \rightarrow \mathbb{K}$, \quad
$x_0 \in \acc(X)$, \\
$y_0, z_0 \in \mathbb{K}^n$,
$\beta \in \mathbb{K}$, \quad
$y_0 = \lim_{x \to x_0} f(x)$,
$z_0 = \lim_{x \to x_0} g(x)$,
$\beta = \lim_{x \to x_0} \alpha(x)$

\begin{enumerate}
    \item $\lim_{x \to x_0} (f(x) + g(x)) = y_0 + z_0$
    
    \item $\lim_{x \to x_0} (\alpha(x) \cdot g(x)) = \beta \cdot z_0$
    
    \item $\lim_{x \to x_0} \langle f(x), g(x) \rangle =
    \langle y_0, z_0 \rangle$
    
    \item $\lim_{x \to x_0} f|_{X_0}(x) = y_0$ \qquad
    ($X_0 \subset X$, $x_0 \in \acc(X_0)$)
    
    \item $\lim_{x \to x_0} \frac{1}{\alpha(x)} = \frac{1}{\beta}$ \qquad
    ($\alpha(x) \not= 0$, $\beta \not= 0$)
\end{enumerate}

\linie

\textbf{links-/rechtsseitiger Grenzwert}:
$f: X \subset \mathbb{R} \rightarrow M_2$, $X \subset [a,b]$,
$x_0 \in \acc(X)$, $a < x_0 < b$ \\
falls $x_0 \in \acc(X_-)$ mit $X_- = X \cap \left[a,x_0\right[$, ist
$\lim_{x \to x_0 - 0} f(x) = \lim_{x \to x_0} f|_{X_-}(x)$ der
\emph{linkss. GW} \\
falls $x_0 \in \acc(X_+)$ mit $X_+ = X \cap \left]x_0,b\right]$, ist
$\lim_{x \to x_0 + 0} f(x) = \lim_{x \to x_0} f|_{X_+}(x)$ der
\emph{rechtss. GW}

es gilt: $(y = \lim_{x \to x_0} f(x)) \;\Leftrightarrow\;
(y = \lim_{x \to x_0 - 0} f(x)) \land (y = \lim_{x \to x_0 + 0} f(x))$

\textbf{Satz}: $f: \left]a,b\right[ \rightarrow \mathbb{R}$, $a < b$ \qquad
Ist $f$ monoton wachsend und beschränkt nach oben, dann gibt es den Grenzwert
$\lim_{x \to b} f(x)$
(analog für monoton fallende Funktionen).

\subsection{%
    Die komplexe Exponentialfunktion und die \textsc{Euler}sche Formel%
}

$z \in \mathbb{C}$; \quad $t_n(z) = 1 + \sum_{k=1}^n$
{\large $\frac{z^k}{k!}$}, \quad $n \in \mathbb{N}$

\textbf{Satz 1}: Die Folge $\{t_n(z)\}_{n \in \mathbb{N}}$ besitzt für jedes
$z \in \mathbb{C}$ einen Grenzwert
$\exp(z) \overset{\text{def.}}{=} \lim_{n \to \infty} t_n(z)$. \\
Es ist $\exp(0) = 1$ sowie $\exp(1) = e$.

\textbf{Satz 2} (Multiplikativität): Für $z, w \in \mathbb{C}$ ist
$\exp(z + w) = \exp(z) \cdot \exp(w)$.

\textbf{Folgerungen}: $\exp(n) = e^n$, $\exp(\frac{n}{m}) = e^{n/m}$
($n, m \in \mathbb{N}$), $\exp(q) = e^q$ ($q \in \mathbb{Q}$), \\
$\exp(z) \not= 0$, $\exp(-z) = \frac{1}{\exp(z)}$ ($z \in \mathbb{C}$)

\textbf{Satz 3}: $|\exp(z) - z - 1| \le |z|^2$ für $z \in \mathbb{C}$,
$|z| < 1$

\textbf{Satz 4}: $z = x + iy \in \mathbb{C}$, $x = \Re z$, $y = \Im z$

\begin{enumerate}
    \item $\exp(\overline{z}) = \overline{\exp(z)}$
    
    \item $|\exp(z)| = \exp(x)$ \quad (von $y$ unabhängig)
    
    \item $\arg(\exp(z)) = \arg(\exp(iy)) \mod 2\pi$ \quad (von $x$ unabhängig)
    
    \item $\arg(\exp(iy)) = y \mod 2\pi$
\end{enumerate}

\textbf{Folgerung}: $\exp(iy) = \cos y + i \sin y = e^{iy}$, da
$|\exp(iy)| = |\exp(0)| = 1$, $\arg(\exp(iy)) = y$, \\
d.\,h. für $z = x + iy$ gilt
$\exp(z) = \exp(x + iy) = \exp(x) \exp(iy) = e^x e^{iy}$

\textbf{Reihendarstellung von Sinus/Kosinus}: \\
$\sin z = \Im(\exp(iy)) =$ {\large $z - \frac{z^3}{3!} + \frac{z^5}{5!} -
\frac{z^7}{7!} \pm \cdots$}, \quad
$\cos z = \Re(\exp(iy)) =$ {\large $1 - \frac{z^2}{2!} + \frac{z^4}{4!} -
\frac{z^6}{6!} \pm \cdots$}

\subsection{%
    Stetige Funktionen%
}

$(M_1,d_1)$, $(M_2,d_2)$ metrische Räume, $f: X \subset M_1 \rightarrow M_2$ \\
\textbf{Stetigkeit}: $f$ ist stetig im Punkt $x_0 \in X$
$\;\Leftrightarrow\; (x_0 \in \iso(X)) \lor (\lim_{x \to x_0} f(x) = f(x_0))$
\\
$f$ ist auf $X$ stetig $\;\Leftrightarrow\;$
$f$ ist in allen $x_0 \in X$ stetig

\pagebreak

$f$ ist stetig in $x_0 \in X$
$\;\Leftrightarrow\; \forall_{\varepsilon > 0}
\exists_{\delta = \delta(\varepsilon, x_0)}
\forall_{x \in U_\delta(x_0) \cap X}\; f(x) \in U_\varepsilon(f(x_0))$ \\
$\;\Leftrightarrow\; \forall_{\{x_n\}_{n \in \mathbb{N}},\; x_n \in X,\;
x_n \xrightarrow{n \to \infty} x_0}\;
\lim_{n \to \infty} f(x_n) = f(x_0)$

\textbf{Stetigkeit bei vektorwertigen Funktionen (Spezialfall)}: \\
$f, g: X \subset M_1 \rightarrow \mathbb{K}^n$,
$\alpha: X \subset M_1 \rightarrow \mathbb{K}$, $f, g, \alpha$ stetig
in $x_0 \in X$ (auf $X$) \\
$\Rightarrow\;$
$f \pm g$, $\langle f, g \rangle_{\mathbb{K}^n}$, $\alpha \cdot f$ stetig
in $x_0 \in X$ (auf $X$),
$\frac{1}{\alpha(x)}$ stetig in $x_0 \in X$ bzw. auf $X$ ($\alpha(x) \not= 0$)
\\
$x_0 \in X_0 \subset X$, $f: X \rightarrow M_2$ stetig
$\;\Rightarrow\;$ $f|_{X_0}: X_0 \rightarrow M_2$ stetig

\textbf{Satz}: Polynome $P_n(z)$, der Betrag $|z|$ und $\exp(z)$ sind stetig
auf $\mathbb{C}$, \\
d.\,h. auch $\sin z$ und $\cos z$ sind stetig auf $\mathbb{C}$.

Ist $f$ in $x_0 \in X$ stetig, dann ist $f$ in einer geeigneten
$\delta$-Umgebung von $x_0$ beschränkt ($\delta > 0$).

\linie

\textbf{Formen der Unstetigkeit bei reellen Funktionen}:
$f: X \subset \left]a,b\right[ \rightarrow \mathbb{R}$,
$x_0 \in \left]a,b\right[$

\begin{itemize}
    \item \emph{Hebbare Unstetigkeit}: $x_0 \notin X$, d.\,h. $f$ ist im Punkt
    $x_0$ nicht definiert, aber \\
    $\exists \lim_{x \to x_0 - 0} f(x) =
    \lim_{x \to x_0 + 0} f(x) \;\Rightarrow\;$
    $\tilde{f}(x) =$ {\small
    $\begin{cases}
    \lim_{x \to x_0} f(x) & x = x_0 \\
    f(x) & x \not= x_0
    \end{cases}$} ist stetig in $x_0$.
    
    \item \emph{Unstetigkeit vom Typ 1}:
    $\exists \lim_{x \to x_0 - 0} f(x)$, $\exists \lim_{x \to x_0 + 0} f(x)$,
    aber $f(x_0 - 0) \not= f(x_0 + 0)$ \\
    $\Rightarrow$ Sprung der Funktion (verschiedene Grenzwerte)
    
    \item \emph{Unstetigkeit vom Typ 2}:
    $f(x_0 - 0)$ oder $f(x_0 + 0)$ existiert nicht
\end{itemize}

\linie

\emph{Lemma}: $f: [a,b] \rightarrow \mathbb{R}$ stetig, $x_0 \in [a,b]$,
$f(x_0) \not= 0 \;\Rightarrow\;
\exists_{\delta > 0} \forall_{x \in U_\delta(x_0) \cap [a,b]}\;
\sgn f(x) = \sgn f(x_0)$

\textbf{Satz von \textsc{Bolzano}-\textsc{Cauchy}}:
$f: [a,b] \rightarrow \mathbb{R}$ stetig mit $f(a)f(b) < 0$
$\quad\Rightarrow\quad \exists_{c \in \left]a,b\right[}\; f(c) = 0$

Ist $f\!\!\upuparrows$ oder $f\!\!\downdownarrows$, dann ist $c$ eindeutig
bestimmt. \\
Anwendung: eindeutige Lösungen $\sqrt[n]{g}$, $\ln g$, Existenz der
Umkehrfunktionen

\textbf{Folgerung (Zwischenwertsatz)}: $f: [a,b] \rightarrow \mathbb{R}$
stetig, $x_1, x_2 \in [a,b]$ mit $x_1 < x_2$, \\
$y_- = \min\{f(x_1), f(x_2)\}$, $y_+ = \max\{f(x_1), f(x_2)\}$
$\;\Rightarrow\; \forall_{\eta \in \left]y_-,y_+\right[}
\exists_{c(\eta) \in \left]x_1,x_2\right[}\; f(c(\eta)) = \eta$

\textbf{Umkehrung als Satz}: $f: [a,b] \rightarrow \mathbb{R}$,
$f\!\!\uparrow$ (wichtig!),
$f$ nimmt alle Werte $y \in [f(a), f(b)]$ an \\
$\Rightarrow\; f$ stetig auf $[a,b]$

\textbf{Umkehrfunktionen}:
$f: [a,b] \rightarrow [\alpha,\beta]$ monoton, bijektiv
$\;\Rightarrow\; f^{-1}$ stetig auf $[\alpha,\beta]$

\linie

\textbf{Stetigkeit mit of"|fenen Mengen}: $(M_1,d_1)$, $(M_2,d_2)$
metrische Räume, $f: M_1 \rightarrow M_2$ \\
$f$ ist auf $M_1$ stetig
$\;\Leftrightarrow\;$ das Urbild $V = f^{-1}(U)$ jeder in $M_2$ of"|fenen Menge
$U$ ist in $M_1$ of"|fen.

\textbf{Komposition von stetigen Funktionen}: Sind $f: M_1 \rightarrow M_2$ und
$g: M_2 \rightarrow M_3$ stetige Funktionen, so ist auch
$g \circ f: M_1 \rightarrow M_3$ stetig.

\linie

\textbf{dichte Menge}: $(M,d)$ metrischer Raum, $X \subset M$ \quad
$X$ ist dicht in $M \;\Leftrightarrow\; \overline{X} = M$.

\textbf{Satz}: Seien $f, g: M_1 \rightarrow M_2$ stetige Funktionen,
$X \subset M_1$ und $X$ dicht in $M_1$. \\
Ist $f|_X = g|_X$, dann ist auch $f(x) = g(x)$ für alle $x \in M_1$.

\textbf{links- und rechtsseitige Stetigkeit}:
$f: [a,b] \rightarrow M_2$, $x_0 \in [a,b]$ \\
$f$ ist in $x_0$ linksseitig stetig
$\;\Leftrightarrow\; f(x_0 - 0) = f(x_0)$ \\
$f$ ist in $x_0$ rechtsseitig stetig
$\;\Leftrightarrow\; f(x_0 + 0) = f(x_0)$ \\
$f$ ist stetig in $x_0$ genau dann, wenn $f$ in $x_0$ links- und rechtsseitig
stetig ist.

\textbf{Notation (Grenzwerte von Funktionen)}:
$f: \mathbb{R} \rightarrow M_2$, $y \in M_2$ \\
$y = \lim_{x \to \infty} f(x) \;\Leftrightarrow\;
\forall_{\varepsilon > 0} \exists_{C(\varepsilon)}
\forall_{x \in \mathbb{R},\; |x| \ge C(\varepsilon)}\;
|y - f(x)| < \varepsilon$ \\
$y = \lim_{x \to +\infty} f(x) \;\Leftrightarrow\;
\forall_{\varepsilon > 0} \exists_{C(\varepsilon)}
\forall_{x \in \mathbb{R},\; x \ge C(\varepsilon)}\;
|y - f(x)| < \varepsilon$ \\
$y = \lim_{x \to -\infty} f(x) \;\Leftrightarrow\;
\forall_{\varepsilon > 0} \exists_{C(\varepsilon)}
\forall_{x \in \mathbb{R},\; x \le C(\varepsilon)}\;
|y - f(x)| < \varepsilon$

\subsection{%
    Kompakte Mengen%
}

\textbf{Teilfolge}: Eine \emph{Teilfolge} entsteht durch "`Streichen"' von
endlich oder unendlich vielen Gliedern, sodass unendlich viele Folgenglieder
übrig bleiben.
Die Ordnung bleibt erhalten! \\
Wähle streng monotone Folge $\{n_k\}_{k \in \mathbb{N}}$,
$n_k \in \mathbb{N}$, dann ist $\{x_{n_k}\}_{k \in \mathbb{N}}$ eine Teilfolge
von $\{x_n\}_{n \in \mathbb{N}}$. \\
Wenn $\lim_{n \to \infty} x_n = y$, dann konvergieren auch alle Teilfolgen:
$\lim_{k \to \infty} x_{n_k} = y$.

\textbf{kompakte Menge}: Sei $(M,d)$ metrischer Raum, $X \subset M$. \\
$X$ heißt \emph{(folgen-)kompakt} $\;\Leftrightarrow\;$ aus jeder Folge
$\{x_n\}_{n \in \mathbb{N}}$, $x_n \in X$ kann man mindestens eine geeignete
Teilfolge $\{x_{n_k}\}_{k \in \mathbb{N}}$ auswählen, welche einen Grenzwert
$\lim_{k \to \infty} x_{n_k} = y \in X$ besitzt.

\textbf{Kompaktheitskriterium von \textsc{Bolzano}}: Sei $M = \mathbb{R}^d$
oder $M = \mathbb{C}^d$. \\
$X \subset \mathbb{R}^d$ bzw. $X \subset \mathbb{C}^d$ ist kompakt
$\;\Leftrightarrow\; X$ ist beschränkt und abgeschlossen.

\linie

\textbf{Satz}: Sei $X \subset \mathbb{R}$ eine nicht-leere, kompakte
Teilmenge von $\mathbb{R}$. \\
Dann besitzt $X$ ein Maximum $x_+ = \max X$ und ein Minimum $x_- = \min X$.

\textbf{Satz}: Sei $f: X \subset M_1 \rightarrow M_2$ stetig.
Ist $X$ kompakt, dann ist auch das Bild $f(X)$ kompakt.

\textbf{Satz von \textsc{Weierstraß} (Extremwertsatz)}:
Sei $f: X \subset M_1 \rightarrow \mathbb{R}$ stetig und $X$ kompakt. \\
Dann ist $f(X)$ beschränkt und es gibt Elemente $x_+, x_- \in X$, sodass
$y_+ = f(x_+) = \max f(X)$ und $y_- = f(x_-) = \min f(X)$.

\linie

\textbf{Verdichtungspunkt}: Sei $x_k \in M$ eine Folge.
$y \in M$ heißt \emph{Verdichtungspunkt} von $\{x_k\}$,
falls es eine Teilfolge $\{x_{k_j}\}$ aus $\{x_k\}$ gibt mit
$\lim_{j \to \infty} x_{k_j} = y$.

Jede beschränkte Folge $\{x_k\}$, $x_k \in \mathbb{R}^d$
($x_k \in \mathbb{C}^d$) besitzt mindestens einen Verdichtungspunkt.

\subsection{%
    Gleichmäßige Stetigkeit%
}

Seien $(M_1,d_1)$, $(M_2,d_2)$ metrische Räume und
$f: X \subset M_1 \rightarrow M_2$ Funktion. \\
\emph{Wiederholung}: $f$ heißt stetig auf $X$, falls
$\forall_{x_0 \in X} \forall_{\varepsilon > 0}
\exists_{\delta = \delta(\varepsilon, x_0) > 0}
\forall_{x \in U_\delta(x_0) \cap X}\; f(x) \in U_\varepsilon(f(x_0))$.

$f$ heißt \textbf{gleichmäßig stetig} auf $X$, falls
$\forall_{\varepsilon > 0} \exists_{\delta = \delta(\varepsilon) > 0}
\forall_{x_0 \in X} \forall_{x \in U_\delta(x_0) \cap X}\;
f(x) \in U_\varepsilon(f(x_0))$ \\
bzw. $\forall_{\varepsilon > 0} \exists_{\delta = \delta(\varepsilon) > 0}
\forall_{x, x_0 \in X,\; d(x, x_0) < \delta}\;
f(x) \in U_\varepsilon(f(x_0))$.

Eine auf $X$ gleichmäßig stetige Funktion ist auch auf $X$ stetig.
Die Umkehrung gilt nicht!

\textbf{Satz von \textsc{Cantor}}: Sei $f: X \subset M_1 \rightarrow M_2$
stetig auf $X$ sowie $X$ kompakt (\emph{wichtig}). \\
Dann ist $f$ gleichmäßig stetig auf $X$.

Bei einer vektorwertigen, stetigen Funktion
$f: X \subset M \rightarrow \mathbb{K}^n$, $f(x) = (f_1(x), \ldots, f_n(x))$
kann man also aus $X$ kompakt folgern, dass
$f$ beschränkt ist, $\Vert f(x) \Vert$ das Maximum/Minimum annimmt
sowie dass $f$ gleichmäßig stetig ist.

\pagebreak

\subsection{%
    Der Raum der stetigen Funktionen%
}

Seien $M_1$, $M_2$ metrische Räume und $X \subset M$. \\
$C(X, M_2)$ bezeichnet die \textbf{Menge aller stetigen Funktionen}
$f: X \rightarrow M_2$.

\linie

\textbf{Spezialfall}: $X \subset M_1$, $X$ kompakt (\emph{wichtig!}),
$M_2 = \mathbb{K}^d$
($\mathbb{K} = \mathbb{R}$ oder $\mathbb{K} = \mathbb{C}$) \\
Auf der Menge der stetigen Funktionen $C(X, \mathbb{K}^d)$ werden dann zwei
Operationen definiert: \\
$\boldsymbol{+}: C(X, \mathbb{K}^d) \times C(X, \mathbb{K}^d) \rightarrow
C(X, \mathbb{K}^d)$, $(f + g)(x) = f(x) + g(x)$ für $x \in X$ \\
$\boldsymbol{\cdot}: \mathbb{K} \times C(X, \mathbb{K}^d) \rightarrow
C(X, \mathbb{K}^d)$, $(\alpha \cdot f)(x) = \alpha \cdot f(x)$ für $x \in X$ \\
Mit diesen Operationen wird $C(X, \mathbb{K}^d)$ zu einem
$\mathbb{K}$-Vektorraum (Nullvektor ist Nullabbildung).

\textbf{$C(X, \mathbb{K}^d)$ als normierter Raum}:
Die Norm einer Funktion $f \in C(X, \mathbb{K}^d)$ wird definiert als
$\Vert f \Vert_C := \max_{x \in X} \Vert f(x) \Vert_{\mathbb{K}^d}$
(Maximum existiert nach \textsc{Weierstraß}).
Die so definierte Funktion erfüllt die Eigenschaften einer Norm,
d.\,h. $C(X, \mathbb{K}^d)$ ist normierter Raum.
Dadurch wird $C(X, \mathbb{K}^d)$ auch zum metrischen Raum mit
$d_c(f,g) = \Vert f - g \Vert_C =
\max_{x \in X} \Vert f(x) - g(x) \Vert_{\mathbb{K}^d}$.

\linie

\textbf{Konvergenz in $C(X, \mathbb{K}^d)$}:
$f_n, g \in C(X, \mathbb{K}^d)$, \quad
$f_n \xrightarrow{\Vert \cdot \Vert_C} g \;\Leftrightarrow\;
\forall_{\varepsilon > 0} \exists_{N(\varepsilon)}
\forall_{n \ge N(\varepsilon)}\; \Vert f_n - g \Vert_C < \varepsilon$ \\
$\Leftrightarrow\; \forall_{\varepsilon > 0} \exists_{N(\varepsilon)}
\forall_{n \ge N(\varepsilon)}\;
\max_{x \in X} \Vert f_n(x) - g(x) \Vert_{\mathbb{K}^d} < \varepsilon$

\textbf{punktweise Konvergenz}:
$\lim_{n \to \infty} f_n(x) = g(x)$ punktweise für $x \in X$ \\
$\Leftrightarrow\; \forall_{x \in X} \forall_{\varepsilon > 0}
\exists_{N(\varepsilon, x)} \forall_{n \ge N(\varepsilon, x)}\;
\Vert f_n(x) - g(x) \Vert_{\mathbb{K}^d} < \varepsilon$ \\
Die Grenzwert-Funktion bzgl. einer punktweisen Konvergenz muss nicht stetig
sein.

\textbf{gleichmäßige Konvergenz}:
$\lim_{n \to \infty} f_n(x) = g(x)$ gleichmäßig bzgl. $x \in X$ \\
$\Leftrightarrow\; \forall_{\varepsilon > 0} \exists_{N(\varepsilon)}
\forall_{n \ge N(\varepsilon)} \forall_{x \in X}\;
\Vert f_n(x) - g(x) \Vert_{\mathbb{K}^d} < \varepsilon$. \\
Damit ist gleichmäßige Konvergenz gleichbedeutend mit Konvergenz im
$C(X, \mathbb{K}^d)$. \\
Gleichmäßige Konvergenz impliziert punktweise Konvergenz.
Die Umkehrung gilt nicht!

\textbf{Satz}: $C(X, \mathbb{K}^d)$ ist vollständig
(bzgl. der gleichmäßigen Konvergenz).

\textbf{Folgerung}: Seien $f_n \in C(X, \mathbb{K}^d)$,
$g: X \rightarrow \mathbb{K}^d$ ($X$ kompakt) mit
$f_n(x) \xrightarrow{n \to \infty} g(x)$ gleichmäßig bzgl. $x \in X$.
Dann ist auch $g$ stetig, d.\,h. $g \in C(X, \mathbb{K}^d)$.

\pagebreak
