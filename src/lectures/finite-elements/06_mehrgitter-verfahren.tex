\section{%
    Mehrgitter-Verfahren%
}

Für uniforme B-Spline-Basen liegt es nahe, Mehrgitter-Verfahren zur Lösung der
Ritz-Galerkin-Systeme zu verwenden.
Solche Algorithmen stellen die ef"|fizientesten iterativen Löser für große
Ritz-Galerkin-Systeme dar.
Die Zeit, die zur Lösung benötigt wird, ist proportional zur Anzahl der Unbekannten,
daher asymptotisch optimal.
Während bei klassischen Iterationsverfahren wie SSOR oder CG die Konvergenzrate
für kleinere Gitterweiten immer schlechter wird,
reduzieren Mehrgitter-Verfahren den Fehler in jedem Schritt um einen vom Gitter unabhängigen
Faktor.

\subsection{%
    Idee der Mehrgitter-Verfahren%
}

Um die Mehrgitter-Idee zu erklären, betrachtet man Beispiel das univariate Modellproblem
\begin{align*}
    -u'' = f \text{ in } [0, 1],\quad
    u(0) = u(1) = 0.
\end{align*}
Mehrgitter-Verfahren berechnen zunächst eine Näherung $V_h \approx U_\ast$ der exakten Lösung
$U_\ast$.
Anschließend wird $R := GV_h - F_h = GW_h$ mit $W_h = V_h - U_\ast$ und  nach $W_h$ gelöst
und eine sog. \begriff{Grobgitter-Korrektur} $U_h := V_h - W_h$ durchgeführt.

\linie

\textbf{\name{Ritz}-\name{Galerkin}-Diskretisierung}:
Die Standard-FE-Approximation
\begin{align*}
    u \approx u_h = \sum_i u_i b_i
\end{align*}
mit Hut-Funktionen $b_i := b^1(\cdot/h - i)$ berechnet sich durch das tridiagonale
Ritz-Galerkin-System
\begin{align*}
    GU_\ast = h^{-1} \begin{pmatrix}2 & -1 & &\\-1 & 2 & -1 & \\ & \ddots & \ddots & \ddots
    \end{pmatrix} U_\ast = F.
\end{align*}

\linie

\textbf{\name{Richardson}-Iteration}:
Die \begriff{\name{Richardson}-Iteration} verbessert eine Approximation $U \approx U_\ast$
des Koef"|fizientenvektors durch Subtraktion eines Vielfachen des Residuums:
\begin{align*}
    U \leftarrow U - \gamma^{-1} (GU - F).
\end{align*}
Der Parameter $\gamma$ wird gewählt als $4/h = \norm{G}_\infty$,
sodass die tridiagonale Iterationsmatrix
\begin{align*}
    E - \gamma^{-1} G =
    \begin{pmatrix}1/2 & 1/4 & &\\1/4 & 1/2 & 1/4 & \\ & \ddots & \ddots & \ddots\end{pmatrix}
\end{align*}
Eigenwerte in $(0, 1)$ besitzt.
(Lineare Iterationsverfahren konvergieren genau dann, wenn der Spektralradius,
d.\,h. der Betrag des betragsmäßig größten Eigenwerts, kleiner als $1$ ist,
außerdem ist jede Matrixnorm größer oder gleich wie der größte Eigenwert.)

Die Richardson-Iteration dämpft hoch-oszillierende Fehlerkomponenten sehr stark.

\linie
\pagebreak

\textbf{Subdivision}:
Für die B-Splines $\widetilde{b}_i := b^1(\cdot/(2h) - i)$ und $b_i$ auf den Gitter mit
Gitterweiten $\widetilde{h} := 2h$ und $h$ gilt
\begin{align*}
    \widetilde{b}_i = \sum_\ell p_{\ell,i} b_\ell,\quad
    P^t := \begin{pmatrix}1/2 & 1 & 1/2 & & & &\\&&1/2 & 1 & 1/2 & &\\&&&&
    \dotsb&\dotsb&\dotsb\end{pmatrix},
\end{align*}
also $\widetilde{b}_i = \frac{1}{2} b_{2i} + b_{2i+1} + \frac{1}{2} b_{2i+2}$ bzw.
$p_{2i+1,i} = 1$ und $p_{2i,i} = p_{2i+2,i} = 1/2$.

\linie

\textbf{Einschränkung}:
Der geglättete Fehler $e_v$ einer Approximation $V$ zur exakten Lösung $U_\ast$
kann relativ genau auf einem gröberen Gitter dargestellt werden.
Daher kann die Residuumsgleichung
\begin{align*}
    GW_\ast = R,\quad
    R = GV - F,
\end{align*}
für die Dif"|ferenz $W_\ast = V - U_\ast$ zur exakten Lösung $U_\ast$ durch die doppelte
Gitterweite $\widetilde{h} = 2h$ approximiert werden durch
\begin{align*}
    \widetilde{G} \widetilde{W} = \widetilde{R},\quad
    \widetilde{g}_{k,i} := \int_0^1 \widetilde{b}_i' \widetilde{b}_k',\quad
    \widetilde{R} := P^t R.
\end{align*}

\linie

\textbf{Erweiterung}:
Die Grobgitter-Korrektur $\widetilde{W}$ kann anschließend wieder durch
\begin{align*}
    \widetilde{w}
    = \sum_i \widetilde{w}_i \widetilde{b}_i
    = \sum_i \sum_\ell p_{\ell,i} \widetilde{w}_i b_\ell
    = \sum_\ell (P\widetilde{W})_\ell b_\ell
\end{align*}
zurück auf das feinere Gitter erweitert werden.

\linie

\textbf{Zwei-Gitter-Algorithmus}:\\
Eine Iteration des \begriff{Zwei-Gitter-Algorithmus} besteht aus den folgenden Schritten:
\begin{enumerate}
    \item
    Durchführung von $\alpha$-vielen Richardson-Iterationen
    $U \leftarrow U - \gamma^{-1} (GU - F)$,
    um eine Approximation $V$ mit glattem Fehler zu erhalten
    
    \item
    Berechnung des Residuums $R = GV - F$
    
    \item
    Einschränkung $\widetilde{R} = P^t R$ auf das gröbere Gitter
    
    \item
    Lösung des Grobgitter-Systems $\widetilde{G} \widetilde{W} = \widetilde{R}$
    
    \item
    Erweiterung $W = P \widetilde{W}$ auf das feinere Gitter
    
    \item
    Korrektur $U = V - W$ der Feingitter-Approximation
\end{enumerate}

\linie

\textbf{Mehrgitter-Algorithmus}:
Beim Mehrgitter-Algorithmus wird in Schritt 4 wieder der Algorithmus angewandt.
Wenn man einen Iterationsschritt als
\begin{align*}
    W = \mathcal{M}(U, F, h)
\end{align*}
bezeichnet ($U$ Startvektor, $F$ rechte Seite, Gitterweite $h$),
dann ersetzt man das Lösen des Grobgitter-Systems $\widetilde{G} \widetilde{W} = \widetilde{R}$
durch
\begin{align*}
    \widetilde{W} = \mathcal{M}(0, \widetilde{R}, 2h)
\end{align*}
(Nullvektor als Startvektor, da Residuen meistens klein).
Nur wenn die Gitterweite $h$ zu groß ist, bricht man ab und berechnet die exakte Lösung
$\widetilde{W} = \widetilde{G}^{-1} \widetilde{R}$.

\pagebreak

\subsection{%
    Gittertransfer%
}

\textbf{multivariate Subdivision}:\\
Die relevanten B-Splines $\widetilde{b}_\ell$ mit Gitterweite $2h$ können als Linearkombinationen
\begin{align*}
    \widetilde{b}_\ell = \sum_{k \in K} s_{k-2\ell} b_k,\quad
    s_\alpha := 2^{-nm} \prod_{\nu=1}^m \binom{n+1}{\alpha_\nu}
\end{align*}
von $b_k = b_{k,h}^n$ dargestellt werden,
wobei $\binom{n+1}{\mu} := 0$ für $\mu < 0$ oder $\mu > n + 1$.

Eine Linearkombination von Grobgitter-B-Splines $\widetilde{b}_\ell$ wird auf dem feinen Gitter
dargestellt durch
\begin{align*}
    \sum_{\ell \in \widetilde{K}} \widetilde{u}_\ell \widetilde{b}_\ell
    = \sum_{\ell \in \widetilde{K}} \sum_{k \in K} s_{k-2\ell} \widetilde{u}_\ell b_k
    = \sum_{k \in K} u_k b_k,
\end{align*}
d.\,h. die Koef"|fizienten hängen zusammen durch die Beziehung
\begin{align*}
    U = P \widetilde{U},\quad
    p_{k,\ell} := s_{k-2\ell}.
\end{align*}
Der Transfer eines Residuums erfolgt durch
\begin{align*}
    \widetilde{r}_\ell
    = \lambda(\widetilde{b}_\ell)
    = \sum_{k \in K} s_{k-2\ell} \lambda(b_k)
    \iff
    \widetilde{R} = P^t R.
\end{align*}
Diese Formeln verändern sich nicht bei Multiplikation mit einer Gewichtsfunktion $w$ und sind
daher auch für den gewichteten Spline-Raum $w\BB$ gültig.

\linie

\textbf{Gittertransfer für WEB-Splines}:
Die Projektion eines Grobgitter-WEB-Splines ist
\begin{align*}
    P_h \widetilde{B}_\ell
    := \sum_{i \in I} p_{i,\ell} B_i,\quad
    p_{i,\ell} := \frac{w(x_i)}{w(\widetilde{x}_\ell)}
    \left(s_{i-2\ell} + \sum_{j \in \widetilde{J}(\ell)} \widetilde{e}_{\ell,j} s_{i-2j}\right).
\end{align*}
Daher gilt
\begin{align*}
    P_h \sum_{\ell \in \widetilde{I}} \widetilde{u}_\ell \widetilde{B}_\ell
    = \sum_{i \in I} u_i B_i,\quad
    U = P \widetilde{U},
\end{align*}
und $\widetilde{R} = P^t R$ ist die Approximation eines Residuums auf dem groben Gitter.

\subsection{%
    Grundlegender Algorithmus%
}

\textbf{Mehrgitter-Algorithmus}:
Ein Schritt
$U \rightarrow W = \mathcal{M}(U, F, h)$
des Mehrgitter-Algorithmus, der eine Approximation $U \approx U_\ast := G^{-1} F$ verbessert,
ist definiert durch das Programm
\begin{align*}
    &V = S^\alpha(U, F)\\
    &R = GV - F\\
    &\widetilde{R} = P^t R\\
    &\mathbf{if}\; 2h = h_{\text{max}}\\
    &\qquad\widetilde{W} = \widetilde{G}^{-1} \widetilde{R}\\
    &\mathbf{else}\\
    &\qquad\widetilde{W} = \mathcal{M}^\beta(0, \widetilde{R}, 2h)\\
    &\mathbf{end}\\
    &W = V - P\widetilde{W},
\end{align*}
wobei $\alpha$ und $\beta$ die Anzahl an Glättungs- bzw. groben Mehrgitter-Iterationen bezeichnet.

\linie
\pagebreak

\textbf{Mehrgitter-Heuristik}:
Der Fehler nach $\alpha$-vielen Richardson-Schritten ist
\begin{align*}
    V - U_\ast = (E - \gamma^{-1} G)^\alpha (U - U_\ast).
\end{align*}
Die langsam oszillierenden dominieren
gegenüber den hochfrequenten Komponenten, daher kann die
Residuumsgleichung $G(V - U_\ast) = R$ mit $R := GV - F$
gut auf dem groben Gitter approximiert werden, d.\,h. man betrachtet approximative Lösungen
der Form $P\widetilde{W} \approx V - U_\ast$, die man als Projektionen vom groben Gitter erhält.
Wegen $P_h \widetilde{B}_i = \sum_\alpha p_{\alpha,i} B_\alpha = \widetilde{B}_i$
(im Falle von WEB-Splines müsste das letzte Gleichheitszeichen durch $\approx$ ersetzt werden)
gilt
\begin{align*}
    \widetilde{g}_{k,i}
    = a(\widetilde{B}_i, \widetilde{B}_k)
    = \sum_{\alpha,\beta \in I} p_{\alpha,i} a(B_\alpha, B_\beta) p_{\beta,k}
\end{align*}
mit $a(B_\alpha, B_\beta) = g_{\beta,\alpha}$.
Daher ist $\widetilde{G} = P^t G P$, sodass
$\widetilde{G} \widetilde{W} = \widetilde{R}$ mit $\widetilde{R} = P^t R$
die angemessene Approximation der obigen Resiuumsgleichung ist
(wegen $GP\widetilde{W} \approx R$).
Durch Lösung dieses Grobgitter-Systems
(direkt oder approximativ mit $\beta$-vielen Schritten einer Mehrgitter-Iteration)
folgt aus $\widetilde{W} \approx \widetilde{G}^{-1} \widetilde{R}$,
dass $P\widetilde{W} \approx G^{-1} R$.
Deswegen sollte die Grobgitter-Korrektur $V \rightarrow W = V - P\widetilde{W}$ zu einer
substanziellen Verbesserung führen.

\linie

\textbf{Wahl von $\alpha$ und $\beta$}:
Die Fälle $\beta = 1$ und $\beta = 2$ werden \begriff{v- bzw. w-Zyklus} genannt.
Eine feste Wahl von $\alpha$ und $\beta$ ist für theoretische Zwecke praktisch,
jedoch ist für die Implementierung die dynamische Kontrolle des Gittertransfers viel ef"|fizienter.
Man hört bei den Glättungsschritten auf, wenn die Konvergenz langsam wird, und
transferiert die Korrektur zurück auf das feine Gitter, wenn der Fehler ausreichend
reduziert wurde.

\linie

\textbf{Anzahl der Operationen}:
Die Anzahl der vom Mehrgitter-Algorithmus durchgeführten Iterationen ist gleich
$\sigma(h) \preceq h^{-m}$, wenn $\beta < 2^m$.
Daher ist der rechnerische Aufwand für einen Mehrgitter-Schritt äquivalent zu dem einer festen
Anzahl an Richardson-Iterationen.
Wegen der Reduktion der Gittergröße verursachen die rekursiven Aufrufe nur einen moderaten
Zuwachs der Komplexität.

\subsection{%
    Glättung und Grobgitter-Approximation%
}

Im Folgenden wird der Glättungsef"|fekt der Richardson-Iteration und die Genauigkeit der
Grob"-gitter-Korrektur analytisch erklärt.
Die beiden Lemmas werden für den Beweis der Konvergenz im nächsten Abschnitt benötigt.
$D$ sei dazu ein glattes Gebiet und die Approximation erfolgt durch WEB-Splines mit einer
Standard-Gewichtsfunktion.
Außerdem betrachtet man wieder das Poisson-Problem $-\Delta u = f$ in $D$,
$u = 0$ auf $\partial D$ als typisches Modellproblem.

Für den Richardson-Iterationsfehler gilt
$V = U - \gamma^{-1} (GU - F)$ sowie $U_\ast = U_\ast - \gamma^{-1} (GU - F)$.
Die Dif"|ferenz der Gleichungen ist $V - U_\ast = S(U - U_\ast)$ mit
der Iterationsmatrix $S = E - \gamma^{-1} G$.
Nach $\alpha$-vielen Iterationen erhält man den Fehler
$V - U_\ast = S^\alpha (U - U_\ast)$
(wie bei jeder linearen Iteration).
Man kann zeigen, dass $\norm{GS^\alpha} \le \const \cdot \frac{h^{m-2}}{\alpha+1}$,
dabei entspricht die Multiplikation mit $G$ die "`Bildung der 2. Ableitung"',
d.\,h. wie stark der Fehler variiert.
Man erhält daher einen Faktor $h^{-2}$,
der Faktor $h^m$ kommt von der Normalisierung der WEB-Splines.
Das Wichtige ist die Division durch $\alpha + 1$, was den Glättungsef"|fekt der Iteration
quantifiziert.

\textbf{Glättung der Richardson-Iteration}:\\
Der Fehler nach $\alpha$-vielen Richardson-Schritten $U \rightarrow V$ erfüllt
\begin{align*}
    \norm{G (V - U_\ast)} \le \const(D, w, n) \frac{h^{m-2}}{\alpha+1} \norm{U - U_\ast},
\end{align*}
wobei $U, V$ Approximationen von $U_\ast = G^{-1} F$ sind.

\linie
\pagebreak

Der folgende Satz zeigt, dass es nur einen kleinen Unterschied zwischen den Lösungen
auf aufeinanderfolgenden Gittern gibt.

\textbf{Fehler der Grobgitter-Korrektur}:
Wenn
\begin{align*}
    \widetilde{G} \widetilde{U}_\ast = \widetilde{R},\quad
    R := G(V - U_\ast),
\end{align*}
mit $\widetilde{R} := P^t R$,
dann gilt
\begin{align*}
    \norm{(v - u_\ast) - \widetilde{u}_\ast}_0 \le \const(D, w, n) h^{2-m} \norm{r}_0,
\end{align*}
wobei $v, u_\ast, \widetilde{u}_\ast, r$ die WEB-Splines sind, die zu den Koef"|fizientenvektoren
$V, U_\ast, \widetilde{U}_\ast, R$ gehören.

\subsection{%
    Konvergenz%
}

\textbf{Mehrgitter-Konvergenz}:
Für $\beta = 2$ Grobgitter-Iterationen gilt
\begin{align*}
    \norm{W - U_\ast} \le \frac{\const(D, w, n)}{\alpha + 1} \norm{U - U_\ast}
\end{align*}
für einen Mehrgitter-Schritt $U \rightarrow W$.
Daher ist die Konvergenzrate $\varrho$ des w-Zyklus kleiner als $1$
(gleichmäßig bzgl. der Gitterweite $h$), falls die Anzahl $\alpha$ an Glättungsschritten
genügend groß ist.

Für diesen Satz benötigt man u.\,a. die Stabilität der WEB-Basis
und die Beschränktheit des Standard-Projektors, d.\,h.
\begin{align*}
    \norm{q}_0 \asymp h^{m/2} \norm{Q},\quad
    \norm{P_h \varphi}_0 \preceq \norm{\varphi}_0.
\end{align*}

\pagebreak
