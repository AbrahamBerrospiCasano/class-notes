\chapter{%
    Randwertprobleme%
}

Im Folgenden geht es um die Approximation der typischen Randwertprobleme.
Im Allgemeinen hat dabei die DGL für die Lösung $u$ eine schwache Formulierung
als Variationsproblem $a(u, v) = \lambda(v)$, $v \in H$,
wobei $H$ ein Hilbertraum ist, in dem die Randbedingungen verarbeitet sind.
Die Ritz-Galerkin-Approximation $u_h = \sum_{i \in I} u_i B_i$ erhält man einfach durch
Ersetzung von $u$ durch $u_h$ und von $v$ durch die Basisfunktionen $B_k$.

\section{%
    Wesentliche Randbedingungen%
}

Wesentliche Randbedingungen sind Randbedingungen, die in die FE-Unterräume eingearbeitet werden
müssen und daher eine Gewichtsfunktion benötigen.
Dagegen sind natürliche Randbedingungen automatisch durch die Lösungen der Variationsprobleme
erfüllt und erlauben eine einfachere Approximation.

\textbf{\name{Poisson}-Problem mit inhomogenen \name{Dirichlet}-Randbedingungen}:
Das typische Modellproblem für wesentliche Randbedingungen ist das
\begriff{\name{Poisson}-Problem mit inhomogenen\\
\name{Dirichlet}-Randbedingungen}:
\begin{align*}
    -\Delta \varphi = f \text{ in } D,\quad
    \varphi = g \text{ auf } \partial D.
\end{align*}

Durch Setzen von $\varphi = u + \widetilde{g}$ können die inhomogenen Randbedingungen eliminiert
werden, wobei $u \in H_0^1(D)$ und $\widetilde{g}$ eine Erweiterung von $g$ auf $D$ ist.
Durch Multiplikation der DGL mit $v \in H_0^1(D)$ und partielle Integration erhält man
mit dem Satz von Lax-Milgram folgende Aussage.

\textbf{\name{Dirichlet}-Problem}:
Das \begriff{inhomogene \name{Dirichlet}-Problem}
mit zugehöriger Bilinearform und zugehörigem linearen Funktional
\begin{align*}
    a(u, v) := \int_D \grad u \grad v,\quad
    \lambda(v) := \int_D (fv - \grad \widetilde{g} \grad v)
\end{align*}
hat eine eindeutige Lösung $\varphi = u + \widetilde{g}$ mit $u \in H_0^1(D)$,
wenn $f$ und $\grad \widetilde{g}$ quadrat-integrierbar sind.

\linie

Für die Ritz-Galerkin-Approximation $a(u_h, B_i) = \lambda(B_i)$, $i \in I$,
kann jeder der Räume $w\BB_h$, $w^\e\BB_h$ und $w\BB_h(\DD)$ verwendet werden, wobei
$w$ eine Gewichtsfunktion der Ordnung $1$ ist, die auf dem ganzen Rand $\partial D$ verschwindet.
Die einfacheren gewichteten Splines $w\BB_h$ sind für kleine Systeme gut geeignet,
wo Stabilität keine große Rolle spielt.
Hierarchische Verfeinerung wird bei Lösungen mit Singularitäten empfohlen.

Für die Berechnung der rechten Seite des Ritz-Galerkin-Systems ist eine Erweiterung $\widetilde{g}$
der Randdaten nötig.
Wenn kein exakter analytischer Ausdruck verfügbar ist, kann eine Linearkombination von B-Splines
\begin{align*}
    \widetilde{g} \approx \widetilde{g}_h := \sum_{k \in \partial K} g_k b_k
\end{align*}
verwendet werden, die durch die Minimierung von $\int_{\partial D} |g - \widetilde{g}_h|^2$
definiert ist.
Die Menge $\partial K$ enthält all die Indizes $k$, für die $b_k$ auf $\partial D$ nicht
verschwindet.
Bei Gebieten, die durch die R-Methode definiert sind, kann sogenannte
\begriff{transfinite Interpolation} verwendet werden.

\linie
\pagebreak

\textbf{Approximationsfehler für das \name{Dirichlet}-Problem}:
Wenn der Rand, die Gewichtsfunktion und die Daten $f, g$ glatt sind,
approximieren WEB-Splines mit optimaler Approximationsordnung.
Aus Céas Ungleichung und der Ungleichung von Jackson folgt
\begin{align*}
    \norm{e_h}_1 \preceq h^n \norm{u}_{n+1}.
\end{align*}
Für den $L^2$-Fehler gewinnt man durch das Aubin-Nitsche-Dualitätsprinzip einen Faktor $h$
und erhält
\begin{align*}
    \norm{e_h}_0 \preceq h^{n+1} \norm{u}_{n+1}
\end{align*}
(die Konstanten hängen von $D$, $w$ und $n$ ab).

\section{%
    Natürliche Randbedingungen%
}

\textbf{\name{Neumann}-Problem}:
Das \begriff{\name{Neumann}-Problem} lautet
\begin{align*}
    -\Delta u = f \text{ in } D,\quad
    \partial^\perp u = g \text{ auf } \partial D,
\end{align*}
wobei $\partial^\perp$ die Normalenableitung bezeichnet, d.\,h. $\partial^\perp u = \xi \grad u$
mit $\xi$ der nach außen zeigenden Einheitsnormalen zu $\partial D$.

\textbf{Kompatibilitätsbedingungen}:
Integriert man die Gleichung $-\Delta u = f$, so erhält man mit der Formel
$\int_D \partial_\nu u = \int_{\partial D} \xi_\nu u$
\begin{align*}
    \int_D f
    = -\int_D \Delta u
    = -\int_{\partial D} \partial^\perp u,
\end{align*}
d.\,h. damit überhaupt eine Lösung existiert, müssen die Daten die
\begriff{Kompatibilitätsbedingungen}
\begin{align*}
    \int_D f
    = -\int_{\partial D} g
\end{align*}
erfüllen.

\linie

\textbf{Herleitung der schwachen Formulierung}:
Durch Multiplikation der Gleichung mit einer Testfunktion $v$ und partielle Integration
\begin{align*}
    -\int_D (\Delta u) v
    = \int_D \grad u \grad v
    -\int_{\partial D} (\partial^\perp u) v
    = \int_D fv
\end{align*}
($v$ verschwindet hier nicht auf dem Rand) erhält man die schwache Formulierung
\begin{align*}
    \int_D \grad u \grad v
    = \int_D fv + \int_{\partial D} gv,
\end{align*}
wobei die Randbedingungen auf "`natürliche"' Weise auf der rechten Seite eingesetzt wurden.
Die rechte Seite ist ein beschränktes, lineares Funktional $\lambda$,
was man mit $\norm{u}_{0,\partial D} \preceq \norm{u}_1$
(die Beschränkung auf den Rand ist eine beschränkte Abbildung von $H^1(D)$ nach $L^2(\partial D)$)
und der Ungleichung von Cauchy-Schwarz leicht zeigen kann.
Die Bilinearform $a$ auf der linken Seite ist auch beschränkt auf $H^1$ (Cauchy-Schwarz).

\pagebreak

Um eine Eindeutigkeitsaussage für die Lösbarkeit des Variationsproblems zu erhalten,
muss man jedoch einen anderen Raum verwenden, da in $H^1$ die Summe $u + c$ ebenfalls eine
Lösung ist, wenn $u$ eine Lösung ist --
die Bilinearform ist nicht nach unten beschränkt, weil sie auf Konstanten verschwindet
(beim Dirichlet-Problem hat $H_0^1$ keine Konstanten ungleich null zugelassen).
Dazu geht man zum Unterraum
\begin{align*}
    H_\perp^1 := \left\{u \in H^1 \;\left|\; \int_D u = 0\right.\right\}
\end{align*}
über.
Mit der Projektion
\begin{align*}
    P_0 u := \left(\int_D u\right)\left/\left(\int_D 1\right)\right.
\end{align*}
auf Konstanten, die auf $H_\perp^1$ gleich null ist,
erhält man mit dem Bramble-Hilbert-Lemma
\begin{align*}
    \norm{u}_0^2 = \norm{u - P_0 u}_0^2 \preceq |u|_1^2 = a(u, u),
\end{align*}
was die Elliptizität von $a$ zeigt.
Somit kann man den Satz von Lax-Milgram anwenden.

\linie

\textbf{\name{Neumann}-Problem}:\\
Das \begriff{\name{Neumann}-Problem} mit zugehöriger Bilinearform und zugehörigem linearen
Funktional
\begin{align*}
    a(u, v) := \int_D \grad u \grad v,\quad
    \lambda(v) := \int_D fv + \int_{\partial D} gv
\end{align*}
hat eine eindeutige Lösung $u \in H_\perp^1$, wenn die Kompatibilitätsbedingung
\begin{align*}
    \int_D f = -\int_{\partial D} g
\end{align*}
erfüllt ist sowie $f$ und $g$ quadrat-integrierbar sind.

\linie

\textbf{schwache und klassische Lösungen des \name{Neumann}-Problems}:\\
Für $f, g \in L^2(D)$ haben die Variationsgleichungen
\begin{align*}
    a(u, v) = \int_D \grad u \grad v = \int_D fv + \int_{\partial D} gv = \lambda(v),\quad
    v \in H_\perp^1(D),
\end{align*}
eine eindeutige schwache Lösung $u \in H_\perp^1(D)$.
Ist zusätzlich $u$ glatt und die Kompatibilitätsbedingung
$\int_D f = -\int_{\partial D} g$ erfüllt, so gilt
\begin{align*}
    -\Delta u = f \text{ in } D,\quad
    u = g \text{ auf } \partial D,
\end{align*}
d.\,h. $u$ ist die klassische Lösung.

\linie

\textbf{\name{Ritz}-\name{Galerkin}-Approximation des \name{Neumann}-Problems}:
Sei die Kompatibilitätsbedingung $\int_D f = -\int_{\partial D} g$ erfüllt und $u \in H_\perp^1(D)$
glatt, wobei $u$ die Lösung des Neumann-Problems
$\forall_{v \in H_\perp^1(D)}\; a(u, v) = \lambda(v)$.
Dann hat der Fehler einer beliebigen Ritz-Galerkin-Approxi"-mation $u_h$ aus den Räumen $\BB_h$ oder
${}^\e\BB_h$ (WEB-Splines mit $w = 1$) optimale Fehlerordnung:
\begin{align*}
    \norm{u - \widetilde{u}_h}_{\nu,D} = \O(h^{n+1-\nu}),\quad
    \nu = 0, 1,
\end{align*}
wobei $\widetilde{u}_h$ die Projektion von $u_h$ auf $H_\perp^1(D)$ bezeichnet.

\pagebreak

\section{%
    Gemischte Probleme mit variablen Koef"|fizienten%
}

Das folgende Problem ist eine Verallgemeinerung der beiden Probleme, die weiter oben
beschrieben wurden.
Die Beschränkung auf homogene Dirichlet-Randbedingungen stellt keine Einschränkungen dar
(siehe oben).

\textbf{allgemeines elliptisches Problem 2. Ordnung}:\\
Ein \begriff{allgemeines elliptisches Problem 2. Ordnung} lautet
\begin{align*}
    -\div(A \grad u) + a_0 u &= f \quad\text{in } D,\\
    u &= 0 \quad\text{auf } \Gamma,\\
    A \grad u \xi + \alpha u &= g \quad\text{auf } \partial D \setminus \Gamma.
\end{align*}
Dabei ist $A(x)$ eine symmetrische, positiv definite Matrix, die auf $\overline{D}$ glatt ist,
$a_0$ und $\alpha$ sind beschränkte, nicht-negative Funktionen, $\Gamma \subset \partial D$ ist
eine Teilmenge des Rands mit nicht-verschwindendem $(m - 1)$-dimensionalem Maß und
$\xi$ ist die Einheitsaußennormale für $\partial D$.

\linie

\textbf{Beispiel}:
Ein häufig verwendeter Fall ist $A = a(x) \cdot E$, d.\,h. $A$ ist ein Vielfaches der
Einheitsmatrix.
In diesem Fall ist nämlich $A \grad u = a \grad u$,
d.\,h. $\div(A \grad u) = \sum_\nu \partial_\nu (a \grad u)_\nu =
a \Delta u + \grad a \grad u$.

\textbf{Gegenbeispiel}:
Die Annahmen über die Vorzeichen von $a_0$ und $\alpha$ sind für die Existenz einer eindeutigen
Lösung notwendig.
Betrachtet man $-u'' + a_0 u = 0$ mit $u(0) = 0$ und $u'(1) + \alpha u(1) = 1$,
so sieht man, dass es für bestimmte Werte von $a_0$ und $\alpha$ keine Lösung gibt,
wenn man $a_0, \alpha \ge 0$ nicht voraussetzt.

\linie

\textbf{Herleitung der schwachen Formulierung}:
Durch Multiplikation mit $v$ mit $v = 0$ auf $\Gamma$ und partielle Integration erhält man
\begin{align*}
    \int_D (-\div(A \grad u)v + a_0 u v) = \int_D fv.
\end{align*}
Das Integral des ersten Summanden ist gleich
$-\int_D \div(A \grad u) v
= -\sum_\nu \int_D \partial_\nu (A \grad u)_\nu v
= \sum_\nu \int_D (A \grad u)_\nu \partial_\nu v -
\sum_\nu \int_{\partial D} (A \grad u)_\nu v \xi_\nu
= \int_D A \grad u \grad v - \int_{\partial D} (A \grad u) v \xi$.\\
Wegen $(A \grad u) v \xi = gv - \alpha uv$ erhält man folgende schwache Formulierung.

\linie

\textbf{Lösung von allgemeinen elliptischen Problemen 2. Ordnung}:
Ein allgemeines elliptisches Problem 2. Ordnung besitzt die Bilinearform
und das lineare Funktional
\begin{align*}
    a(u, v) := \int_D (A \grad u \grad v + a_0 uv) +
    \int_{\partial D \setminus \Gamma} \alpha uv,\quad
    \lambda(v) := \int_D fv + \int_{\partial D \setminus \Gamma} gv.
\end{align*}
Es besitzt eine eindeutige schwache Lösung $u \in H_\Gamma^1$,
wenn $f$ und $g$ quadrat-integrierbar sind.\\
Dabei ist
\begin{align*}
    H_\Gamma^1 := \{u \in H^1 \;|\; u = 0 \text{ auf } \Gamma\}.
\end{align*}

\pagebreak

\section{%
    Biharmonische Gleichung%
}

\textbf{eingespannte Platte}:
Das folgende Problem tritt zum Beispiel bei einer horizontal eingespannten Platte auf,
auf die eine transversale Kraft $f$ wirkt.
Die Position $u$ der Platte bestimmt sich durch Minimierung der potentiellen Energie
\begin{align*}
    \Q(u) = \frac{1}{2} \int_D \left(|\Delta u|^2 + 2(1 - \nu) (|\partial_1 \partial_2 u|^2 -
    (\partial_1^2 u) (\partial_2^2 u)\right) - \int_D fu,\quad u \in H_0^2,
\end{align*}
wobei $H_0^2 = \{u \in H^2 \;|\; u = \partial^\perp u = 0 \text{ auf } \partial D\}$ und
$\nu = \lambda/(2(\lambda + \mu)) \in (0, 1/2)$ der
\begriff{\name{Poisson}"-Koef"|fizient} der Platte ist,
der durch die \begriff{\name{Lamé}-Konstanten} $\lambda$ und $\mu$ bestimmt ist.
Es gilt
\begin{align*}
    \int_D (\partial_1 \partial_2 u) (\partial_1 \partial_2 u)
    = -\int_D (\partial_1^2 \partial_2 u) (\partial_2 u)
    = \int_D (\partial_1^2 u) (\partial_2^2 u),
\end{align*}
da aus $u = \partial^\perp u = 0$ auf $\partial D$ folgt, dass $\grad u = 0$ auf $\partial D$.
Dadurch vereinfacht sich das Funktional zu
\begin{align*}
    \Q(u) = \frac{1}{2} \int_D |\Delta u|^2 - \int_D fu.
\end{align*}

\linie

\textbf{biharmonisches Randwertproblem}:
Das \begriff{biharmonische Randwertproblem} lautet
\begin{align*}
    \Delta^2 u = f \text{ in } D,\quad
    u = \partial^\perp u = 0 \text{ auf } \partial D.
\end{align*}

\linie

\textbf{Herleitung der schwachen Formulierung}:
Durch Multiplikation mit $v$ mit $v = \partial^\perp v = 0$ auf $\partial D$ und
partielle Integration erhält man
\begin{align*}
    \int_D (\Delta^2 u) v = \int_D fv.
\end{align*}
Das erste Integral ist gleich $\int_D (\Delta (\Delta u)) v =
\sum_\nu \int_D (\partial_\nu^2 (\Delta u)) v$\\
$= -\sum_\nu \int_D (\partial_\nu (\Delta u)) (\partial_\nu v) +
\sum_\nu \int_{\partial D} (\partial_\nu (\Delta u)) v \xi_\nu$\\
$= \int_D (\Delta u) (\Delta v) -
\int_{\partial D} (\Delta u) \grad v \xi +
\int_{\partial D} \grad(\Delta u) v \xi = \int_D (\Delta u) (\Delta v)$
wegen $v = \grad v \xi = 0$.

Die Bilinearform $a(u, v) = \int_D (\Delta u) (\Delta v)$ ist elliptisch, da
$|a(u, v)| \le \norm{\Delta u}_0 \norm{\Delta v}_0 \le \norm{u}_2 \norm{v}_2$
nach der Ungleichung von Cauchy-Schwarz --
für die untere Schranke benutzt man
\begin{align*}
    a(u, u) = \int_D |\partial_1^2 u + \partial_2^2 u|^2
    = \int_D \left(|\partial_1^2 u|^2 + 2|\partial_1 \partial_2 u|^2 + |\partial_2^2 u|^2\right)
    = |u|_2^2
\end{align*}
wegen $\int_D (\partial_1^2 u)(\partial_2^2 u) = \int_D |\partial_1 \partial_2 u|^2$ (siehe oben).
Aufgrund der Poincaré-Friedrichs-Ungleichung
\begin{align*}
    |u|_0^2 \preceq |u|_1^2,\quad
    |u|_1^2 = \sum_\nu |\partial_\nu u|_0^2 \preceq \sum_\nu |\partial_\nu u|_1^2
    = \sum_{\nu,\mu} \int_D |\partial_\mu \partial_\nu u|^2 = |u|_2^2,
\end{align*}
folgt $\norm{u}_2^2 = |u|_0^2 + |u|_1^2 + |u|_2^2 \preceq |u|_2^2 = a(u, u)$.

\linie

\textbf{Lösung des biharmonischen Randwertproblems}:
Das biharmonische Randwertproblem besitzt die Bilinearform und das lineare Funktional
\begin{align*}
    a(u, v) := \int_D (\Delta u) (\Delta v),\quad
    \lambda(v) := \int_D fv.
\end{align*}
Es besitzt eine eindeutige schwache Lösung $u \in H_0^2$, wenn $f$ quadrat-integrierbar ist.

\pagebreak

\section{%
    Lineare Elastizität%
}

Elastizitäts-Simulationen waren einer der Auslöser für die Entwicklung der FE-Methode und
stellen heute noch einen wichtigen Zweig der FE-Analysis dar.
Dabei wird ein elastischer Körper, der ein Volumen $\overline{D}$ belegt,
an einem Teil $\Gamma$ des Rands fixiert und einer Volumenkraft auf $D$ bzw. einer
Randkraft auf $\partial D \setminus \Gamma$ mit Dichten $(f_1, f_2, f_3)$ bzw.
$(g_1, g_2, g_3)$ ausgesetzt.
Diese Kräfte bewirken kleine Deformationen des Körpers, die durch eine Verschiebung
$u(x) \in \real^3$ der Materialpunkte $x \in D$ beschrieben werden.
Normalerweise ist $u$ sehr klein,
größere Verschiebungen zeigen das Vorhandensein riesiger Kräfte an.

\textbf{\name{Lamé}-\name{Navier}-Gleichungen}:
Die \begriff{\name{Lamé}-\name{Navier}-Gleichungen der linearen Elastizität} lauten
\begin{align*}
    -\div \sigma(u) &= f \quad\text{in }D,\\
    u &= 0 \quad\text{auf } \Gamma,\\
    \sigma(u) \xi &= g \quad\text{auf } \partial D \setminus \Gamma.
\end{align*}
Dabei ist
\begin{align*}
    \varepsilon_{k,\ell}(u) &:= \frac{1}{2} (\partial_k u_\ell + \partial_\ell u_k),\\
    \sigma_{k,\ell}(u) &:= \lambda \trace \varepsilon(u) \delta_{k,\ell} +
    2\mu \varepsilon_{k,\ell}(u)
\end{align*}
für $k, \ell = 1, 2, 3$, wobei
$\trace \varepsilon := \varepsilon_{1,1} + \varepsilon_{2,2} + \varepsilon_{3,3}$.

Die Divergenz $\div \sigma$ der Matrix $\sigma$ ist zeilenweise definiert, d.\,h.
$(\div \sigma)_k := \sum_\nu \partial_\nu \sigma_{\nu,k}$.
Die Konstanten $\lambda$ und $\mu$ sind die \begriff{\name{Lamé}-Koef"|fizienten},
die die Elastizitätseigenschaften des Materials beschreiben.
Die zweite Gleichung, die den \begriff{Spannungstensor} $\sigma$ mit dem
\begriff{Verzerrungsten"-sor} $\varepsilon$ (symmetrisierter Gradient)
in Verbindung bringt, ist auch als
\begriff{\name{hooke}sches Gesetz} bekannt.

\linie

\textbf{Herleitung der schwachen Formulierung}:
Durch Multiplikation mit $v$ mit $v_\ell = 0$ auf $\Gamma$ ($\ell = 1, 2, 3$) erhält man
\begin{align*}
    -\int_D (\div \sigma(u)) v = \int_D fv.
\end{align*}
Ausgeschrieben ist die linke Seite gleich
$-\sum_\ell \int_D (\div \sigma(u))_\ell v_\ell
= -\sum_{k,\ell} \int_D (\partial_k \sigma_{k,\ell}(u)) v_\ell$.
Durch partielle Integration erhält man
$\sum_{k,\ell} \int_D \sigma_{k,\ell}(u) (\partial_k v_\ell)
- \sum_{k,\ell} \int_{\partial D} \sigma_{k,\ell}(u) v_\ell \xi_k$.
Der erste Summand ist gleich
$\frac{1}{2}
\big(\sum_{k,\ell} \int_D \sigma_{k,\ell}(u) (\partial_k v_\ell) +
\sum_{k,\ell} \int_D \sigma_{k,\ell}(u) (\partial_\ell v_k)\big)
= \sum_{k,\ell} \int_D \sigma_{k,\ell}(u) \varepsilon_{k,\ell}(v)$
wegen $\sigma$ symmetrisch ($\sigma_{k,\ell} = \sigma_{\ell,k}$).
Der zweite Summand ist gleich
$-\int_{\partial D \setminus \Gamma} \sigma(u) \xi v =
-\int_{\partial D \setminus \Gamma} gv$ wegen $v_\ell = 0$ auf $\Gamma$.
Damit erhält man folgende Variationsform.

\linie

\textbf{Elastizitätsproblem}:
Die Lamé-Navier-Gleichungen besitzen die Variationsform mit der
Bilinearform und dem linearen Funktional
\begin{align*}
    a(u, v) := \int_D \sigma(u):\varepsilon(v),\quad
    \lambda(v) := \int_D fv + \int_{\partial D \setminus \Gamma} gv,
\end{align*}
wobei
\begin{align*}
    \sigma:\varepsilon := \sum_{k,\ell=1}^3 \sigma_{k,\ell} \varepsilon_{k,\ell}.
\end{align*}
Die schwache Formulierung besitzt eine eindeutige Lösung $(u_1, u_2, u_3) \in H_\Gamma^1(D)^3$,
wenn\\
$f \in L^2(D)^3$ und $g \in L^2(\partial D \setminus \Gamma)^3$.

\linie
\pagebreak

Weil der Integrand von $a(u, v)$ nur Ableitungen erster Ordnung beinhaltet,
ist die Beschränktheit von $a$ im Raum $(H_\Gamma^1)^3$ einfach nachzuweisen,
wobei der Raum mit der Produktnorm
$\norm{u}_1 := \left(\sum_{\nu=1}^3 \norm{u_\nu}_1^2\right)^{1/2}$ ausgestattet ist.
Die Abschätzung nach unten ist allerdings viel schwieriger,
für sie wird die sog. \begriff{\name{Korn}-Ungleichung} benötigt.

\linie

\textbf{\name{Ritz}-\name{Galerkin}-Approximiation des \name{Lamé}-\name{Navier}-Systems}:\\
Wenn man WEB-Splines als Basis für den FE-Teilraum verwendet, dann wird jede Komponente $u_\nu$
des Vektors $u$ separat durch eine Linearkombination
\begin{align*}
    (u_h)_\nu = \sum_{i \in I} u_{i,\nu} B_i
\end{align*}
approximiert, wobei $B_i|_\Gamma = 0$.
Man kann dies auch äquivalent schreiben als
\begin{align*}
    u_h = \sum_{i,\nu} u_{i,\nu} B_{i,\nu},\qquad
    B_{i,1} := (B_i, 0, 0),\quad
    B_{i,2} := (0, B_i, 0),\quad
    B_{i,3} := (0, 0, B_i).
\end{align*}
Bis auf die Doppelindizes erhält man also die übliche Form $u_h = \sum_j u_j B_j$.
Daher hat das Ritz-Galerkin-System $GU = F$ Blockstruktur.
Der Block $(k, i)$ von $G$ ist die $(3 \times 3)$-Matrix mit Einträgen
\begin{align*}
    \int_D \sigma(B_{i,\nu}):\varepsilon(B_{k,\ell}),\quad
    \ell, \nu = 1, 2, 3,
\end{align*}
und der $k$-te Block des Vektors $F$ ist der $3$-Vektor mit Komponenten
\begin{align*}
    \int_D f B_{k,\ell} + \int_{\partial D \setminus \Gamma} g B_{k,\ell}
    = \int_D f_\ell B_k + \int_{\partial D \setminus \Gamma} g_\ell B_k,\quad
    \ell = 1, 2, 3.
\end{align*}
Weil die Basisfunktionen $B_{i,\nu}$ nur in einer Komponente nicht null sein können,
vereinfachen sich die Tensoren $\sigma$ und $\varepsilon$ etwas.
Beispielsweise gilt
\begin{align*}
    \varepsilon(B_{k,1})
    = \begin{pmatrix}\partial_1 B_k & \frac{1}{2} \partial_2 B_k & \frac{1}{2} \partial_3 B_k\\
    \frac{1}{2} \partial_2 B_k & 0 & 0\\
    \frac{1}{2} \partial_3 B_k & 0 & 0\end{pmatrix},\quad
    \sigma(B_{i,1}) = \lambda (\partial_1 B_i) I_3 + 2\mu \varepsilon(B_{i,1})
\end{align*}
mit $I_3$ der $(3 \times 3)$-Einheitsmatrix.

\pagebreak

\section{%
    Plane-Strain- und Plane-Stress-Modell%
}

Die folgenden zwei Modelle sind Spezialfälle der oben vorgestellten Elastizitätsgleichung.

\textbf{Herleitung des Plane-Strain-Modells}:
Beim \begriff{Plane-Strain-Modell} (Modell der ebenen Verzerrung)
geht man von einem konstanten, horizontalen Schnitt $D$ aus,
der horizontalen Kräften ausgesetzt wird (d.\,h. keine vertikale Verschiebung).
Unter diesen Voraussetzungen ist $\varepsilon_{3,\ell} = \varepsilon_{\ell,3} = 0$
für $\ell = 1, 2, 3$.
Das hookesche Gesetz vereinfacht sich zu
\begin{align*}
    \begin{pmatrix}\sigma_{1,1} & \sigma_{1,2} & 0\\\sigma_{2,1} & \sigma_{2,2} & 0\\
    0 & 0 & \sigma_{3,3}\end{pmatrix}
    = \lambda (\varepsilon_{1,1} + \varepsilon_{2,2})
    \begin{pmatrix}1&0&0\\0&1&0\\0&0&1\end{pmatrix} +
    2\mu \begin{pmatrix}\varepsilon_{1,1} & \varepsilon_{1,2} & 0\\
    \varepsilon_{2,1} & \varepsilon_{2,2} & 0\\0 & 0 & 0\end{pmatrix}.
\end{align*}
Durch die Einführung von
\begin{align*}
    \underline{\varepsilon} := (\varepsilon_{1,1}, \varepsilon_{2,2}, \varepsilon_{1,2}),\quad
    \underline{\sigma} := (\sigma_{1,1}, \sigma_{2,2}, \sigma_{1,2}),
\end{align*}
sowie durch Umschreiben
\begin{align*}
    \lambda = \frac{E\nu}{(1 + \nu)(1 - 2\nu)},\quad
    \mu = \frac{E}{2(1 + \nu)},
\end{align*}
der Lamé-Koef"|fizienten $\lambda$ und $\mu$
in Abhängigkeit von dem \begriff{\name{Poisson}-Verhältnis} $\nu \in (0, 1/2)$ und
dem \begriff{\name{Young}-Modulus} $E > 0$ erhält man
die zweidimensionale Spannungs-/Verzerrungsrelation
\begin{align*}
    \underline{\sigma} = Q_{\text{strain}} \underline{\varepsilon},\quad
    Q_{\text{strain}} := \frac{E}{(1 + \nu)(1 - 2\nu)}
    \begin{pmatrix}1 - \nu&\nu&0\\\nu&1 - \nu&0\\0&0&1-2\nu\end{pmatrix}.
\end{align*}
Im elementweisen Produkt $\sigma:\varepsilon$ taucht der gemischte Term
$\sigma_{1,2} \varepsilon_{1,2}$ doppelt auf, was in der Bilinearform berücksichtigt werden muss.

\linie

\textbf{Plane-Strain-Modell}:
Wenn $f$ und $g$ quadrat-integrierbare Dichten von horizontalen Kräften sind,
die auf ein elastisches Objekt mit konstantem Querschnitt $D$ und vertikaler Verschiebung
$u_3(x_1, x_2) = 0$ angewendet werden,
dann ist
\begin{align*}
    u = (u_1, u_2) \in (H_\Gamma^1(D))^2,\quad
    D \subset \real^2,
\end{align*}
bestimmt durch
\begin{align*}
    \int_D \underline{\varepsilon}'(u) Q_{\text{strain}} \underline{\varepsilon}(v)
    = \int_D fv + \int_{\partial D \setminus \Gamma} gv,\quad
    v \in (H_\Gamma^1)^2,
\end{align*}
wobei $\underline{\varepsilon}' := (\varepsilon_{1,1}, \varepsilon_{2,2}, 2\varepsilon_{1,2})$.

\linie
\pagebreak

\textbf{Herleitung des Plane-Stress-Modells}:
Für das \begriff{Plane-Stress-Modell} (Modell der ebenen Spannung) geht man von einem Objekt mit
gleichmäßiger vertikaler Dicke aus, die verglichen mit der horizontalen Größe relativ klein ist.
Man nimmt an, dass $\sigma$ nicht von $x_3$ abhängt und dass
$\sigma_{3,\ell} = \sigma_{\ell,3} = 0$ für $\ell = 1, 2, 3$.
Wie beim Plane-Strain-Modell erhält man eine zweidimensionale Version des hookeschen Gesetzes.
Zunächst bemerkt man
$\varepsilon_{1,3} = \varepsilon_{3,1} = 0 = \varepsilon_{2,3} = \varepsilon_{3,2}$.
Im Allgemeinen ist $\varepsilon_{3,3} \not= 0$, d.\,h. kleine vertikale Deformationen
$u_3(x) = \varepsilon_{3,3} x_3$ sind möglich.
Eingesetzt in die Gleichung $\sigma_{3,3} = 0$ ergibt dies
\begin{align*}
    \varepsilon_{3,3} = -\frac{\lambda}{\lambda + 2\mu} (\varepsilon_{1,1} + \varepsilon_{2,2}).
\end{align*}
Wenn man die Lamé-Koef"|fizienten wieder durch das Poisson-Verhältnis und den Young-Modu"-lus
ersetzt, erhält man die Identität
\begin{align*}
    \underline{\sigma} = Q_{\text{stress}} \underline{\varepsilon},\quad
    Q_{\text{stress}} := \frac{E}{1 - \nu^2} \begin{pmatrix}
    1&\nu&0\\\nu&1&0\\0&0&1-\nu\end{pmatrix}.
\end{align*}

\linie

\textbf{Plane-Stress-Modell}:
Für das Plane-Stress-Modell gilt dasselbe wie für das Plane-Strain-Mo"-dell,
wenn man $Q_{\text{strain}}$ durch $Q_{\text{stress}}$ ersetzt.
Wenn $f$ und $g$ quadrat-integrierbare Dichten von horizontalen Kräften sind,
die auf ein elastisches Objekt mit konstanter vertikaler Dicke angewendet werden,
und $\sigma$ nur horizontale Komponenten besitzt, dann
minimieren die ersten beiden Komponenten $(u_1, u_2)$ der Verschiebung das Energie-Funktional
\begin{align*}
    \frac{1}{2} \int_D \underline{\varepsilon}'(u) Q_{\text{stress}} \underline{\varepsilon}(u) -
    \int_D fu - \int_{\partial D \setminus \Gamma} gu
\end{align*}
über alle $u \in (H_\Gamma^1)^2$, wobei
$\underline{\varepsilon}' := (\varepsilon_{1,1}, \varepsilon_{2,2}, 2\varepsilon_{1,2})$.

\pagebreak
