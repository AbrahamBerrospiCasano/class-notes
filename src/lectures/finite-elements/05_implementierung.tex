\chapter{%
    Implementierung%
}

Der normale Ablauf einer FE-Simulation ist
die Beschreibung des Rands,
die Generierung des Netzes,
die Auswahl von relevanten Elementen,
die Assemblierung des Systems und
schließlich die Lösung des Systems.
Bei der WEB-Methode wird die
meist sehr schwierige und zeitaufwendige Netzgenerierung sowie die Elementewahl
ersetzt durch die Konstruktion der WEB-Basis.
Dazu müssen zunächst die Zelltypen bestimmt werden,
dann werden die B-Splines klassifiziert,
anschließend werden die Erweiterungen berechnet und
die Gewichtsfunktion definiert.

\section{%
    Darstellung des Rands%
}

\textbf{rationale \name{Bézier}-Kurve}:
Eine \begriff{rationale \name{Bézier}-Kurve} mit Kontrollpunkten $c_\nu \in \real^m$ und
Gewichten $\omega_\nu > 0$ ist parametrisiert durch
\begin{align*}
    p(t) = \frac{\sum_{\nu=0}^n c_\nu \omega_\nu \beta_\nu^n(t)}
    {\sum_{\nu=0}^n \omega_\nu \beta_\nu^n(t)},\quad
    t \in [0, 1],
\end{align*}
wobei $\beta_\nu^n(t) = \binom{n}{\nu} (1 - t)^{n-\nu} t^\nu$ die Bernstein-Polynome vom Grad $n$
sind.

\textbf{Eigenschaften von rationalen \name{Bézier}-Kurven}:
\begin{itemize}
    \item
    \textbf{Endpunktinterpolation}:
    $p(0) = c_0$,
    $p(1) = c_n$ und
    das Kontrollpolygon bestimmt durch $c_0, \dotsc, c_n$
    ist tangential zu $p$.

    \item
    \textbf{konvexe Hülle}:
    Die Kurve $p$ liegt in der konvexen Hülle der Kontrollpunkte $c_0, \dotsc, c_n$.

    \item
    \textbf{Einfluss der Gewichte}:
    Die Vergrößerung eines Gewichts $\omega_\nu$ zieht die Kurve in Richtung des Kontrollpunkts
    $c_\nu$.
    Wenn $\omega_\nu = 1$ für alle $\nu$,
    dann ist der Nenner $\sum_\nu \omega_\nu \beta_\nu^n(t)$ identisch gleich eins und
    $p$ ist eine polynomiale Parametrisierung.
\end{itemize}

\textbf{rationale \name{Bézier}-Fläche}:
Eine \begriff{rationale \name{Bézier}-Fläche} mit Kontrollpunkten $c_{\nu,\mu} \in \real^3$ und
Gewichten $\omega_{\nu,\mu} > 0$ ist parametrisiert durch
\begin{align*}
    p(s, t) = \frac{\sum_{\nu,\mu=0}^n c_{\nu,\mu} \omega_{\nu,\mu} \beta_\nu^n(s) \beta_\mu^n(t)}
    {\sum_{\nu,\mu=0}^n \omega_{\nu,\mu} \beta_\nu^n(s) \beta_\mu^n(t)},\quad
    s, t \in [0, 1].
\end{align*}

\section{%
    Klassifikation der Gitterzellen%
}

Die Gitterzellen müssen in innere, äußere und Randzellen $Q$ eingeteilt werden, je nachdem
ob $Q \subset \overline{D}$, $Q \cap D = \emptyset$ oder das Innere von $Q$ den Rand schneidet.
Am schwierigsten ist es, die Randzellen zu bestimmen.
Die Unterscheidung zwischen inneren und äußeren Zellen erfolgt anschließend durch Anwendung
eines Standard-Tests auf einen einzigen Punkt in jeder Zelle.

\textbf{Charakterisierung planarer Gitterzellen}:
Das Innere einer planaren Randzelle enthält mindestens ein lokales Extremum von $\partial D$
oder ein Segment zwischen zwei aufeinanderfolgenden Schnittpunkten von $\partial D$ mit
Gitterlinien.

Ein Algorithmus sieht daher wie folgt aus:
Zuerst werden die horizontalen und vertikalen (linearen) Randsegmente bestimmt
und für jedes solches Segment ein Punkt einer Liste von Testpunkten hinzugefügt.
Anschließend werden die isolierten Extrempunkte der Liste hinzugefügt.
Nun werden die Schnitte mit Gitterlinien bestimmt und jeweils ein Punkt zwischen
aufeinanderfolgenden Schnitten als weitere Testpunkte gewählt.
Man muss nur überprüfen, zu welchen Zellen die Testpunkte gehören,
wobei Punkte auf Gitterlinien ignoriert werden.

\linie
\pagebreak

Die Verallgemeinerung auf drei Dimensionen ist möglich, aber nicht trivial,
weil mehr topologische Möglichkeiten bestehen.
Daher wird ein anderer Ansatz gewählt, der besonders für Bézier-Darstellungen geeignet ist:
Die Bézier-Fläche wird in sog.
\begriff{Bounding-Boxen} eingebettet, die zu einer uniformen Unterteilung
des Parameterraums gehören.
Die Mittelpunkte $p(s_\nu, t_\mu)$ der Boxes liegen auf der Fläche,
die Breite bestimmt sich durch die Hilfe von Schranken der Ableitungen der Parametrisierung.

\textbf{Bounding-Box}:
Wenn $d_{\ell,s}$ und $d_{\ell,t}$ ($\ell = 1, 2, 3$) Schranken für den Betrag der Ableitungen
$\partial_s p_\ell$ und $\partial_t p_\ell$ sind, dann ist das Flächenstück
\begin{align*}
    p(s + \sigma, t + \tau),\quad
    |\sigma| \le \delta_s/2,
    |\tau| \le \delta_t/2,
\end{align*}
vollständig in der Box mit Mittelpunkt $p(s, t)$ und Breite
$d_{\ell,s}\delta_s + d_{\ell,t}\delta_t$ in der $\ell$-ten Koordinatenrichtung enthalten.

Wenn $\delta_s$ und $\delta_t$ klein genug gewählt sind,
dann enthalten die meisten Gitterzellen, die die Bézier-Fläche schneiden,
einer der Punkte $p(s_\nu, t_\mu)$ und sind somit leicht identifizierbar.
Nur wenige Gitterzellen werden eine Bounding-Box schneiden, aber keinen der Punkte
$p(s_\nu, t_\mu)$ enthalten.
Für die Zellen kann man z.\,B. einen Optimierungsalgorithmus verwenden oder
einfach $\delta_s$ und $\delta_t$ kleiner wählen.

\section{%
    Auswertung von Gewichtsfunktionen%
}

Für die Berechnung der Ritz-Galerkin-Integrale ist es notwendig, beteiligte Gewichtsfunktionen
auswerten und ableiten zu können.

\textbf{Auswertung und Ableitung von Gewichtsfunktionen}:
Die Evaluation und Dif"|ferentiation von Gewichtsfunktionen, die mithilfe von R-Funktionen
konstruiert wurden, kann rekursiv erfolgen.
Ausgehend von vordefinierten Gewichtsfunktionen
\begin{align*}
    w_\ell,\quad
    \ell = 1, \dotsc, \alpha,
\end{align*}
berechnet man
\begin{align*}
    w_\ell = r_\ell(w_\nu, w_\mu),\quad
    \ell = \alpha + 1, \dotsc, \beta,
\end{align*}
wobei man zum Schluss $w = w_\beta$ erhält.
Die R-Funktionen $r_\ell$ gehören dabei zu booleschen Operationen und haben ein
oder zwei vorher definierte Gewichtsfunktionen als Argument
(im Falle eines Arguments wird die Abhängigkeit von $w_\mu$ ignoriert).

Der Gradient von $w$ kann simultan durch Dif"|ferentiation der Rekursion berechnet werden.
Durch die Kettenregel erhält man
\begin{align*}
    \grad w_\ell
    = (\partial_1 r_\ell) \grad w_\nu + (\partial_2 r_\ell) \grad w_\mu,
\end{align*}
wobei $\partial_k$ die Ableitung bzgl. der $k$-ten Variable darstellt.
Dadurch erhält man sukzessive
\begin{align*}
    (w_{\alpha+1}, \grad w_{\alpha+1}), \dotsc, (w_\beta, \grad w_\beta).
\end{align*}

\linie
\pagebreak

Für Gewichtsfunktionen, die nur durch R-Funktionen aufgebaut wurden, erhält man so explizite
Ausdrücke.
Übergeblendete Gewichtsfunktionen müssen numerisch ausgewertet werden,
dazu geht man wie weiter oben beschrieben mithilfe der Abstandsfunktion $d(x) = \dist(x, \Gamma)$
vor.
Die Ableitung dieser Funktion erhält man durch die negative Außeneinheitsnormale
(intuitiv klar):

\textbf{Abstandsfunktion}:
Für Punkte $x$ in einem genügend kleinen Streifen $\Gamma_\delta$ nahe des Randes ist der Abstand
\begin{align*}
    d(x) = \dist(x, \Gamma) = \norm{x - p(t)}
\end{align*}
von $x$ zu einem Kurvensegment $\Gamma$ mit regulärer Parametrisierung
(z.\,B. in Bézier-Form)\\
$t \mapsto p(t) = (p_1(t), p_2(t))$, $\norm{p'(t)} \not= 0$,
bestimmt durch die Orthogonalitätsbeziehung
\begin{align*}
    (x - p(t)) p'(t) = 0.
\end{align*}
Dabei gilt $\grad d(x) = -\xi(t)$.

Für den Abstand zu einer Fläche $\Gamma$ geht man analog vor.

\section{%
    Numerische Integration%
}

Für die Assemblierung der Ritz-Galerkin-Systeme müssen Integrale über Teilmengen des Gebiets $D$
und seines Rands $\partial D$ berechnet werden.
Dies wird durch Summation über die Beiträge jeder Gitterzelle $Q$ erledigt, d.\,h. die Integrale
haben die Form $\int_{Q \cap D} \varphi$ oder $\int_{Q \cap \partial D} \psi$,
wobei $\varphi$ und $\psi$ von den Basisfunktionen etc. abhängen.
Weil nur in sehr wenigen Fällen exakte analytische Lösungen vorhanden sind, müssen
numerische Verfahren benutzt werden.

Bei Integration von glatten Funktionen über kleine Mengen liefert
\begriff{\name{Gauß}-Quadratur} die ef"|fizientesten Approximationen.
Die Knoten $t_\nu$ der $\ell$-Punkt-Gauß-Formel sind die Nullstellen der Legendre-Polynome
$\ell$-ten Grades und die Gewichte $\gamma_\nu$ sind die Integrale der zugehörigen
Lagrange-Polynome.

Integrale der Form $\int_{Q \cap D} \varphi$ können für Gitterzellen $Q = \ell h + [0, 1]^m h$,
die den Rand nicht schneiden, einfach durch die Tensorprodukt-Gauß-Formel berechnet werden.
Zum Beispiel ist für $m = 3$
\begin{align*}
    \int_Q \varphi
    \approx h^3 \sum_{\nu,\mu,\sigma} \gamma_\nu \gamma_\mu \gamma_\sigma
    \varphi(t_\nu', t_\mu', t_\sigma'),
\end{align*}
wobei $t' = \ell h + (t_\nu, t_\mu, t_\sigma) h$ die transformierten Gauß-Knoten sind.
Für kleines $h$ fallen die meisten Integrale in diese Kategorie.
Dennoch gibt es eine Anzahl von Rand-Gitterzellen, bei denen man anders verfahren muss.
Wenn man naiverweise einfach $\varphi = 0$ auf $Q \setminus D$ setzt und die Integrationsformel
anwendet, würde man viel Glattheit verlieren und die Lösung würde nur wenig genau sein.
Durch Unterteilung von $Q \cap D$ kann man abschnittsweise die Formeln anwenden.
Dies ist in zwei Dimensionen schon kompliziert, wird in dreien aber noch komplizierter.

\textbf{Unterteilung für Gebiets-Integrale}:
Durch Schnitte parallel zu den Koordinatenrichtungen an Kantenschnittpunkten,
lokalen Extrema und Ecken von $Q \cap \partial D$ kann die Menge $Q \cap D$ in
glatt deformierte Rechtecke unterteilt werden.

\pagebreak

\section{%
    Matrix-Assemblierung%
}

\textbf{\name{Ritz}-\name{Galerkin}-System für gewichtete B-Splines}:\\
Die Matrix $G$ und die rechte Seite $F$ des Ritz-Galerkin-Systems für die Räume $w\BB$
können durch folgenden Algorithmus assembliert werden.
\begin{align*}
    &G = 0,\, F = 0\\
    &\mathbf{for}\; Q = \alpha h + [0, 1]^m h \text{ mit } Q \cap D \not= \emptyset\\
    &\qquad\mathbf{for}\; k \in \alpha - \{0, \dotsc, n\}^m\\
    &\qquad\qquad f_k = f_k + \lambda_Q(w b_k)\\
    &\qquad\qquad\mathbf{for}\; \ell \in \alpha - \{0, \dotsc, n\}^m\\
    &\qquad\qquad\qquad g_{k,\ell} = g_{k,\ell} + a_Q(w b_\ell, w b_k)\\
    &\qquad\qquad\mathbf{end}\\
    &\qquad\mathbf{end}\\
    &\mathbf{end}
\end{align*}

\textbf{\name{Ritz}-\name{Galerkin}-System für WEB-Splines}:\\
Die Ritz-Galerkin-Systeme $GU = F$ und ${}^\e G{}^\e U = {}^\e F$ für die gewichteten Spline-Räume
$w\BB$ und $w^\e\BB$ hängen zusammen durch
\begin{align*}
    {}^\e G = \widetilde{E} G \widetilde{E}^t,\quad
    {}^\e F = \widetilde{E} F,
\end{align*}
wobei
\begin{align*}
    \widetilde{e}_{i,k}
    := \frac{1}{w(x_i)} \cdot \begin{cases}1 & k = i,\\e_{i,j} & k = j \in J(i),\\0 & \text{sonst}.
    \end{cases}
\end{align*}

\pagebreak
