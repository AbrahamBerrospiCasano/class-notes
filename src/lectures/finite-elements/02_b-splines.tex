\chapter{%
    B-Splines%
}

\section{%
    Das Spline-Konzept%
}

Polynome stellen zwar gute lokale Approximationen für glatte Funktionen dar,
allerdings kann die Genauigkeit auf großen Intervallen sehr klein sein.
Außerdem haben lokale Änderungen einen globalen Einfluss.
Daher ist der Übergang zu stückweise Polynomen sozusagen "`natürlich"'.

\textbf{Spline}:
Ein \begriff{Spline} vom Grad $\le n$ mit Gitterweite $h$ ist $(n - 1)$-fach stetig
dif"|ferenzierbar und stimmt auf jedem Gitterintervall $[i, i+1]h$ des Parameterintervalls $D$ mit
einem Polynom vom Grad $\le n$ überein.

Diese Definition eignet sich natürlich nicht für numerische Berechnungen,
daher muss eine lokale Basis konstruiert werden.
Die Basis, die hier verwendet wird, kann durch den linearen Fall (Hut-Funktionen) motiviert werden.

\section{%
    Definition und grundlegende Eigenschaften%
}

\textbf{B-Spline}:
Der uniforme \begriff{B-Spline} vom Grad $n$ ist definiert durch die Rekursion
\begin{align*}
    b^n(x) := \int_{x-1}^x b^{n-1},
\end{align*}
beginnend mit der charakteristischen Funktion $b^0$ des Einheitsintervalls $[0, 1)$.
Äquivalent ist die Rekursion
\begin{align*}
    \frac{d}{dx} b^n(x) := b^{n-1}(x) - b^{n-1}(x - 1)
\end{align*}
mit $b^n(0) = 0$.

\linie

\textbf{Eigenschaften von B-Splines}:
B-Splines erfüllen die folgenden Eigenschaften:
\begin{itemize}
    \item
    \emph{Positivität und lokaler Träger}:
    $b^n$ ist positiv auf $(0, n + 1)$ und verschwindet außerhalb dieses Intervalls
    (außer für $n = 0$, hier gilt $b^0(0) = 1$).

    \item
    \emph{Glattheit}:
    $b^n$ ist $(n - 1)$-fach stetig dif"|ferenzierbar, wobei die $n$-te Ableitung in den
    Knotenpunkten $0, \dotsc, n + 1$ unstetig ist.

    \item
    \emph{Struktur als stückweises Polynom}:
    $b^n$ ist auf jedem Intervall $[k, k + 1)$, $k = 0, \dotsc, n$, ein Polynom vom Grad $n$.
\end{itemize}

\textbf{Symmetrie und Monotonie}:
Der B-Spline vom Grad $n$ ist symmetrisch, d.\,h.
\begin{align*}
    b^n(x) = b^n(n + 1 - x),
\end{align*}
und auf $[0, (n + 1)/2]$ und $[(n + 1)/2, n + 1]$ strikt monoton.

\pagebreak

\section{%
    Rekursionsformel%
}

\textbf{Rekursionsformel}:
Der B-Spline $b^n$ ist eine gewichtete Summe von B-Splines vom Grad $n - 1$:
\begin{align*}
    b^n(x) = \frac{x}{n} b^{n-1}(x) + \frac{n + 1 - x}{n} b^{n-1}(x - 1).
\end{align*}

\textbf{\name{Taylor}-Koef"|fizienten}:
Die $n + 1$ polynomialen Segmente
\begin{align*}
    a_{k,0}^n + a_{k,1}^n t + \dotsb + a_{k,n}^n t^n,\quad
    t = x - k \in [0, 1),
\end{align*}
des B-Splines $b^n$ können mit der Rekursion
\begin{align*}
    a_{k,\ell}^n = \frac{k}{n} a_{k,\ell}^{n-1} + \frac{1}{n} a_{k,\ell-1}^{n-1} +
    \frac{n + 1 - k}{n} a_{k-1,\ell}^{n-1} - \frac{1}{n} a_{k-1,\ell-1}^{n-1}
\end{align*}
berechnet werden, wobei $a_{0,0}^1 := 1$ und $a_{k,\ell}^n := 0$ für $k \notin \{0, \dotsc, n\}$
oder $\ell \notin \{0, \dotsc, n\}$.

\section{%
    Darstellung von Polynomen%
}

\textbf{kardinale Splines}:
Für $h > 0$ und $k \in \integer$ sind
\begin{align*}
    b_{k,h}^n(x) := b^n(x/h - k)
\end{align*}
\begriff{B-Splines auf dem Gitter $h\integer$}.
Ihre Linearkombinationen $\sum_{k \in \integer} c_k b_{k,h}^n$ heißen
\begriff{kardinale Splines} vom Grad $\le n$ mit Gitterweite $h$.

\textbf{\name{Marsden}-Identität}:
Für $x, t \in \real$ gilt
\begin{align*}
    (x - t)^n = \sum_{k \in \integer} \psi_{k,h}^n(t) b_{k,h}^n(x),
\end{align*}
wobei $\psi_{k,h}^n(t) := h^n (k + 1 - t/h) \dotsm (k + n - t/h)$.

\textbf{lineare Unabhängigkeit}:
Für jedes Gitterintervall $[\ell, \ell + 1)h$ sind die B-Splines $b_{k,h}$,\\
$k = \ell - n, \dotsc, \ell$, die auf diesem Intervall nicht verschwinden,
linear unabhängig.

\section{%
    Subdivision%
}

\textbf{Gitterverfeinerung}:
Der B-Spline $b_{k,h}^n$ kann als Linearkombination von B-Splines mit Gitterweite $h/2$
geschrieben werden:
\begin{align*}
    b_{k,h}^n = 2^{-n} \sum_{\ell=0}^{n+1} \binom{n+1}{\ell} b_{2k+\ell,h/2}^n.
\end{align*}

\textbf{Subdivisionsalgorithmus}:
Die Koef"|fizienten $c_\ell'$ eines kardinalen Splines $\sum_k c_k b_{k,h}^n$ bzgl. der halben
Gitterweite $h/2$ können wie folgt berechnet werden:
\begin{enumerate}
    \item
    Zunächst setzt man $c_{2k}' := c_{2k+1}' := c_k$.

    \item
    Anschließend bildet man simultan Mittelwerte, d.\,h.
    $c_\ell' \leftarrow \frac{1}{2} (c_\ell' + c_{\ell-1}')$, $\ell \in \integer$.\\
    Dieser Schritt wird $n$-mal insgesamt wiederholt.
\end{enumerate}

\pagebreak

\section{%
    Skalarprodukte%
}

\textbf{Faltung}:
Die Faltung zweier B-Splines ist ein B-Spline höheren Grades:
\begin{align*}
    b^{m+n+1}(x) = \int_\real b^m(x - y) b^n(y) \dy.
\end{align*}

\textbf{Skalarprodukte}:
Die Skalarprodukte der B-Splines $b_{k,h}^n$ und $b_{\ell,h}^n$ und ihrer Ableitungen sind
\begin{align*}
    s_{k-\ell}^n &:= h b^{2n+1} (n + 1 + k - \ell),\\
    d_{k-\ell}^n &:= h^{-2} (2s_{k-\ell}^{n-1} - s_{k-\ell-1}^{n-1} - s_{k-\ell+1}^{n-1}).
\end{align*}

\textbf{Tabelle}: Skalarprodukte der B-Splines und ihrer Ableitungen für $h = 1$:
\renewcommand*{\arraystretch}{1.3}
\begin{align*}
    \begin{array}{c||cccc|cccc}
        n & s_0^n & s_1^n & s_2^n & s_3^n & d_0^n & d_1^n & d_2^n & d_3^n\\\hline
        1 & \frac{2}{3} & \frac{1}{6} & & & 2 & -1 & &\\
        2 & \frac{11}{20} & \frac{13}{60} & \frac{1}{120} & & 1 & -\frac{1}{3} & -\frac{1}{6} &\\
        3 & \frac{151}{315} & \frac{397}{1680} & \frac{1}{42} & \frac{1}{5040} & \frac{2}{3} &
        -\frac{1}{8} & -\frac{1}{5} & -\frac{1}{120}
    \end{array}
\end{align*}
Wegen Symmetrie gilt $s_i^n = s_{-i}^n$ und $d_i^n = d_{-i}^n$.

\textbf{Skalarprodukte von Ableitungen höherer Ordnung}:
Skalarprodukte mit höheren Ableitungen können durch die Dif"|ferentiationsformel
\begin{align*}
    \frac{d^\alpha}{dx^\alpha} b_{k,h}^n(x)
    = h^{-\alpha} \sum_{\nu=0}^\alpha (-1)^\nu \binom{\alpha}{\nu} b_{k+\nu,h}^{n-\alpha}
\end{align*}
errechnet werden.

\pagebreak
