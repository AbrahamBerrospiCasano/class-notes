\chapter{%
    Dynamisches Programmieren%
}

\section{%
    \emph{Longest Common Subsequence}%
}

Gegeben seien zwei Strings $A$ und $B$.
Eine \textbf{Subsequenz} von $A$ und $B$ ist eine Teilfolge der Buchstaben, die
in beiden Strings enthalten ist (Reihenfolge muss also beachtet werden).
Gesucht ist nun die längste gemeinsame Subsequenz
(\textbf{\emph{longest common subsequence}, LCS}).

\textbf{Anwendungen}:
Beispielsweise bei Korrektur und Erkennung einer eingegebenen Sequenz,
in der Biologie bei neu entdeckten DNA-Sequenzen (um schon bekannte ähnliche
Gene zu finden) und bei \emph{UNIX patch/diff}
(Unterschiede zwischen Software-Quellcode finden).

\linie

\textbf{Beobachtung}:
Schreibt man die beiden Strings übereinander und verbindet die zugehörigen
Buchstaben einer Subsequenz, so sieht man, dass sich die Linien nicht
überschneiden können.

\textbf{Folgerungen}:
(analog für letzte Buchstaben)
\begin{itemize}
    \item
    Wenn beide Strings mit demselben Buchstaben beginnen, dann gibt es eine
    LCS, welche diese Buchstaben einander zuordnet.

    \item
    Falls sich die ersten Buchstaben der beiden Strings unterscheiden,
    kann maximal einer von ihnen in einer bestimmten LCS sein.
\end{itemize}

\linie

\textbf{rekursive Lösung von LCS}:
\begin{lstlisting}
rekLCS(A[1 ... m], B[1 ... n])                              gibt die Länge der LCS von
    if m = 0 or n = 0                                       A[1 ... m] und B[1 ... n] zurück
        return 0
    if A[m] = B[n]
        return 1 + rekLCS(A[1 ... m - 1], B[1 ... n - 1])

    $l_1$ = rekLCS(A[1 ... m], B[1 ... n - 1])
    $l_2$ = rekLCS(A[1 ... m - 1], B[1 ... n])
    return max($l_1$, $l_2$)
\end{lstlisting}

\textbf{Laufzeit}: Ist $|A| = |B| = n$, so ist die Laufzeit von
\code{rekLCS} mindestens $\Omega(2^n)$.

\begin{Beweis}
    Anfangs wird rekLCS mit $(n, n)$ aufgerufen.
    Dies ruft im schlimmsten Fall (Buchstaben sind unterschiedlich)
    $(n - 1, n)$ und $(n, n - 1)$ auf.
    Diese rufen wiederum im schlimmsten Fall
    $(n - 2, n)$, $(n - 1, n - 1)$ sowie $(n - 1, n - 1)$, $(n, n - 2)$ auf.
    Man sieht, dass $(n - i, n - i)$ im schlimmsten Fall $2^i$ mal aufgerufen
    wird, also wird $(1, 1)$ $2^{n-1}$ aufgerufen.
\end{Beweis}

\linie

\textbf{bessere Version}:
Man sieht, dass es eigentlich nur $m \cdot n$ ($|A| = m$, $|B| = n$)
verschiedene Argumente gibt, mit denen \code{rekLCS} aufgerufen werden kann.
Als Verbesserung speichert man sich jedes Ergebnis
$(i, j)$ (d.\,h. $|$\code{rekLCS(A[1 ... i], B[1 ... j])}$|$)
in einer Matrix und kann es bei Bedarf nachschlagen.
Insgesamt werden $m \cdot n$ Ergebnisse gespeichert, daher ist die Laufzeit
$\O(m \cdot n)$.

Man erhält eine $(m + 1) \times (n + 1)$-Matrix, wobei der $(i, j)$-te
Eintrag ($i = 0, \dotsc, m$, $j = 0, \dotsc, n$)
$|$\code{LCS(A[1 ... i], B[1 ... j])}$|$ enthält.
Die Matrix wird von links oben aufgefüllt:

\begin{tabular}{p{6.0cm}p{10.1cm}}
    \begin{minipage}[c]{6.0cm}
\matrixsize{\begin{tabular}{l|cccc}
    & $0$ & $1$ & \dots & $n$ \\ \hline
    $0$ & $0$ & $0$ & \dots & $0$ \\
    $1$ & $0$ & & & \\
    \vdots & \vdots & & & \\
    $m$ & $0$ & & &
\end{tabular}}
    \end{minipage} &
    \begin{minipage}[c]{10.1cm}
\begin{lstlisting}
for i = 1 to m
    for j = 1 to n
        if A[i] = B[j]
            L[i][j] = 1 + L[i - 1][j - 1}
        else
            L[i][j] = max(L[i - 1][j], L[i][j - 1])
\end{lstlisting}
    \end{minipage}
\end{tabular}

\linie

Die Länge der LCS lässt sich im $(m, n)$-ten Eintrag ablesen.
Um auch eine tatsächliche LCS zu ermitteln, geht man die Matrix von rechts
unten nach links oben folgendermaßen durch:
Man schaut, wo der Eintrag in der aktuellen Zelle herkommt.
Sind die Buchstaben links und über der Zelle gleich, so kommt der Eintrag
von links oben, andernfalls von links oder oben.
Dann "`besucht"' man die entsprechende Zelle (bei Ungleichheit ist es
unerheblich welche).
Falls die Buchstaben gleich sind, hängt man den Buchstaben ganz vorne
an die aktuelle LCS an (anfangs leerer String).

\section{%
    Edit-/\textsc{Levenshtein}-Distanz%
}

Eine etwas komplexere Art, zwei Strings $A$ und $B$ zu vergleichen, erfolgt
durch die Betrachtung, wie viele Änderungen (Einfügen, Löschen, Änderungen)
nötig sind, um $A$ in $B$ umzuwandeln.
Verschiedene Änderungen können dabei je nach Art verschieden gewichtet werden.

Die folgenden Änderungsoperationen sind zugelassen:
Einfügen (füge Buchstabe in String ein),
Löschen (lösche Buchstabe aus String),
Ändern (ersetze Buchstabe durch anderen).

Gegeben seien nun Kosten für Einfügen, Löschen und Änderung sowie
zwei Strings $A$ und $B$.
Gesucht ist die billigste Sequenz
(\textbf{minimale Edit-Sequenz}) an Operationen, die aus $A$ $B$ macht.

\linie

Liegt eine minimale Edit-Sequenz vor, so sagt man, zwei Buchstaben in $A$
und $B$ sind assoziiert, falls der Buchstabe aus $A$ in den aus $B$ geändert
oder falls er überhaupt nicht geändert wurde.

Betrachte nun zwei Strings $A[1 ... m]$ und $B[1 ... n]$.
\begin{itemize}
    \item
    Wenn in der minimalen Edit-Sequenz $A[m]$ und $B[n]$ assoziiert sind,
    so gibt das Kosten von $0$ (für $A[m] = B[n]$) oder Ersetzungskosten für
    $A[m] \rightarrow B[n]$
    plus Edit-Distanz von $A[1 ... m - 1]$ und $B[1 ... n - 1]$.

    \item
    Wenn in der minimalen Edit-Sequenz $A[m]$ mit niemandem assoziiert ist,
    so gibt das Kosten für das Löschen von $A[m]$
    plus Edit-Distanz von $A[1 ... m - 1]$ und $B[1 ... n]$.

    \item
    Wenn in der minimalen Edit-Sequenz $B[n]$ mit niemandem assoziiert ist,
    so gibt das Kosten für das Einfügen von $B[n]$
    plus Edit-Distanz von $A[1 ... m]$ und $B[1 ... n - 1]$.
\end{itemize}

Seien $ccost(A[i], B[j])$ die Ersetzungskosten für $A[i] \rightarrow B[j]$,
$\dcost(A[i])$ die Löschkosten von $A[i]$ und
$\icost(B[j])$ die Einfügekosten von $B[j]$, dann definiert \\
$E[i][j] = \min\{\ccost(A[i], B[j]) + E[i - 1][j - 1],
\dcost(A[i]) + E[i - 1][j],
\icost(B[j]) + E[i][j - 1]\}$ \\
eine Matrix, wobei der $(i, j)$-te Eintrag die minimale Edit-Distanz
von $A[1 ... i]$ und $B[1 ... j]$ enthält.
Die Edit-Sequenz kann wieder durch Rückverfolgung der Minima ermittelt werden.

\section{%
    Rucksackproblem%
}

Gegeben seien ein Rucksack mit einer bestimmten Kapazität $G$ (Gewicht) und
$n$ Gegenständen, jeweils mit Gewicht $g_i$ und Wert $w_i$.
Gesucht ist nun eine Teilmenge $I \subseteq \{1, \dotsc, n\}$, sodass
$\sum_{i \in I} g_i \le G$ sowie $\sum_{i \in I} w_i$ maximal wird.

\linie

\textbf{Lösung mit dynamischer Programmierung}: \\
\begin{tabular}{p{6.0cm}p{10.1cm}}
    \begin{minipage}[c]{6.0cm}
\matrixsize{\begin{tabular}{l|cccc}
    & $0$ & $1$ & \dots & $n$ \\ \hline
    $0$ & $0$ & $0$ & \dots & $0$ \\
    $1$ & $\infty$ & & & \\
    \vdots & \vdots & & & \\
    $\sum w_i$ & $\infty$ & & &
\end{tabular}}
    \end{minipage} &
    \begin{minipage}[c]{10.1cm}
        Der $(j, i)$-te Eintrag $A(j, i)$ enthält das minimale Gewicht
        eines Rucksacks, der nur Gegenstände aus $\{1, \dotsc, i\}$ enthält
        sowie Wert genau $j$ hat. \\
        Es gilt
        $A(j, i) = \min\{A(j, i - 1),\; g_i + A(j - w_i, i - 1)\}$.
    \end{minipage}
\end{tabular}

Man kann abbrechen, falls in den letzten $\max_i w_i$ Zeilen jeweils
$g_i$ addiert nur Werte größer $G$ vorkommen.
Den Rucksack kann man analog durch Rückverfolgung bestimmen.

\pagebreak
