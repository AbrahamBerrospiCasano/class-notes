\chapter{%
    Suchbäume und (2, 4)-Bäume%
}

\section{%
    Allgemeines zu Suchbäumen%
}

Gegeben sei eine Teilmenge $S = \{a_1, \dotsc, a_n\}$ eines geordneten
Universums $U$. \\
Gesucht ist eine Organisation von $S$ in einem binären Baum, wobei
\begin{itemize}
    \item
    der Baum $n$ Knoten und eine ausgezeichnete Wurzel besitzt,

    \item
    jedes $a_i$ mit einem Knoten asoziiert ist,

    \item
    jeder innere Knoten maximal einen linken und maximal einen rechten
    Kindknoten hat sowie

    \item
    für jeden Knoten $a_v$ des Baums sind alle Knoten im Teilbaum eines
    linken/rechten Kindes von $a_v$ kleiner/größer als $a_v$.
\end{itemize}

\linie

Die Suche nach einem Schlüssel $x$ im Suchbaum verläuft wie folgt: \\
\code{search(}$x, a_{root}$\code{)} gibt das größte $a_\ell$ mit
$a_\ell \le x$ im Baum zurück, sonst $-\infty$.

\begin{lstlisting}
search(x, {}) = -unendlich

search(x, a)
    if x < a
        return search(x, a_L)
    else if x = a
        return a
    else
        return max(a, search(a, a_R))
\end{lstlisting}

Suchbäume können, falls der Baum ausbalanciert ist, dabei helfen, sehr
"`billig"' nach einem Knoten zu suchen.
In diesem Fall ist die Suchzeit für ein Knoten $\O(\log n)$.
Das Problem ist, dass das Löschen und Einfügen von Elementen den Baum
wieder unbalanciert machen kann -- der Extremfall ist ein
"`ausgearteter Baum"', bei dem in jeder Ebene nur ein Knoten ist.
Daher nutzt man spezielle Datenstrukturen, die den Baum automatisch
ausbalancieren.

\section{%
    (2, 4)-Bäume
}

Gegeben sei wieder eine linear geordnete Menge $S = \{a_1, \dotsc, a_n\}$.
\begin{itemize}
    \item
    $S$ soll nur in den Blättern des Baums gespeichert werden. \\
    Die Blätter müssen sortiert sein.

    \item
    Die Blätter des Baums sollen alle die gleiche Tiefe haben.

    \item
    Jeder innere Knoten hat zwischen zwei und vier Kinder.

    \item
    Jede innere Knoten mit $i$ Kindern enthält selbst $i - 1$ Schlüssel,
    dabei ist der $j$-te Schlüssel das größte Element des $j$-ten Teilbaums
    des Knotens ($j = 1, \dotsc, i - 1$).
\end{itemize}

\linie

\textbf{Suche nach einem Schlüssel $k$}:
\begin{lstlisting}
v := wurzel
while v kein Blatt do
    bestimme l mit k_{l-1}(v) < k <= k_l(v)      (wobei k_0(v) = -unendlich und k_grad(v) = +unendlich gesetzt wird)
    v := l-tes Kind von v
od
\end{lstlisting}

Die Suche in einem Blatt $v$, wobei
$Schl"ussel(linkerNachbar) < k \le Schl"ussel(v)$ ist.
Ist $k > \max S$, dann endet die Suche in dem Blatt, das am weitesten rechts
liegt.
Die Laufzeit ist $\O(h)$, wenn $h$ die Tiefe des Baums ist.

\pagebreak

\textbf{Lemma}: Sei $T$ ein (2, 4)-Baum der Höhe $h$ mit $n$ Blättern. \\
Dann gilt $2^h \le n \le 4^h$ und daher $\frac{1}{2} \log n \le h \le \log n$.

\linie

\textbf{Einfügen eines Elements $k$}: \\
Angenommen wird, dass ein Verweis auf das Blatt $v$ mit
$Schl"ussel(linkerNachbar) < k \le Schl"ussel(v)$ vorliegt
(kann durch Suchen nach $k$ erreicht werden).
\begin{enumerate}
    \item
    \emph{Einfügen}:
    Füge $k$ links von $v$ als neues Blatt hinzu und füge $k$ als Schlüssel
    in den Vaterknoten vor den Schlüssel von $v$ ein.

    \item
    \emph{Spalten}:
    Wenn der Vaterknoten nun fünf Knoten hat, muss er aufgespaltet werden,
    d.\,h. er wird in zwei Knoten aufgeteilt, wobei der linke
    Knoten die ersten zwei und der rechte Knoten die letzten drei
    Kinder enthält.
    Wenn die Wurzel gespalten werden muss, muss eine neue Wurzel erzeugt
    werden, sodass die Tiefe um $1$ steigt.
    \begin{lstlisting}
v := vater(v)
while v hat fuenf Kinder do
    spalte(v)
    v := vater(v)
od
    \end{lstlisting}
\end{enumerate}
Die Laufzeit ist $\O(1 + \text{Anzahl Spaltungen}) = \O(\log n)$.

\linie

\textbf{Löschen eines Elements $k$}:
\begin{enumerate}
    \item
    \emph{Löschen}:
    Ist $k$ kein am weitesten rechts liegendes Kind, so kann es einfach
    mit dem zugehörigen Schlüssel in seinem Vaterknoten gelöscht werden.
    Liegt $k$ am weitesten rechts, so muss der zugehörige Schlüssel
    verändert werden, der allerdings nicht im Vaterknoten liegt, sondern
    in einem darüber liegenden indirekten Vaterknoten.

    \item
    \emph{Verschmelzen}/\emph{Stehlen}:
    Hat der Vaterknoten von $k$ nach dem Löschen nur noch ein Kind, so
    muss entweder der Knoten mit einem Nachbarn verschmolzen werden oder
    er muss einen Knoten von einem Nachbarknoten stehlen.
    Verschmolzen wird, falls der Nachbarknoten $2$ Knoten hat, hat er
    $3$ oder $4$ Knoten, so wird gestohlen.
\end{enumerate}
Dies kann bis zur Wurzel fortgesetzt werden, daher beträgt die Laufzeit
$\O(\log n)$.

\linie

\textbf{Warum benutzt man nicht (2, 3)-Bäume?} \\
Hat ein Knoten drei Kinder, so würde dieser nach einer Einfügeoperation
gespalten werden.
Ist jedoch der Baum "`voll"', d.\,h. jeder Knoten hat drei Kinder,
so müsste jeder gespalten werden, sodass nun alle Knoten zwei Kinder haben.
Wird nun wieder ein Knoten gelöscht, so müssten wieder alle Kinder verschmolzen
werden und die Ausgangssituation wäre wiederhergestellt.
Der Zeitaufwand von (2, 3)-Bäumen ist also größer
(analog zum Binärzähler).

\textbf{Laufzeit}:
Jede beliebige Sequenz aus Einfügen und Löschen benötigt in einem
(2, 4)-Baum amortisiert $\O(1)$ Operationen.

\begin{Beweis}
    ?
\end{Beweis}

\textbf{Potential}:
Das Potential eines (2, 4)-Baums $T$ mit maximal einem Knoten vom Grad 1 oder 5
ist $\pot(T) = (2 \cdot \#1) + (1 \cdot \#2) + (0 \cdot \#3) + (2 \cdot \#4) +
(4 \cdot \#5)$, wobei
$\#i$ die Anzahl der Knoten vom Grad $i$ bedeutet.

\pagebreak

\section{%
    Anwendungen von (2, 4)-Bäumen%
}

\textbf{Sortieren "`leicht"' vorsortierter Folgen}: \\
Gegeben sei eine Folge $x_1, \dotsc, x_n$.
Die Anzahl der Inversionen (Fehlstände) ist dann \\
$F = \Big|\big\{(i, j) \;|\; i < j,\; x_i > x_j\big\}\Big|$,
es gilt $0 \le F \le \binom{n}{2}$.

\textbf{Behauptung}: Man kann mit (2, 4)-Bäumen in
Zeit $\O(n \max\{1, \log \frac{F}{n}\})$ sortieren. \\
Ist $F = n$ bzw. $F = n^2$, so kann man in $\O(n)$ bzw. $\O(n \log n)$
sortieren.

\begin{Beweis}
    Man sortiert durch Einfügen in einen (2, 4)-Baum.
    Angenommen, $x_1, \dotsc, x_i$ sind schon sortiert eingefügt
    und $x_{i+1}$ soll nun eingefügt werden.
    Der Abstand von rechts zur eigentlich richtigen Position von $x_{i+1}$
    ist $f_{i+1} = |\{j \;|\; j < i + 1,\; x_j > x_{i+1}\}|$, wobei
    $\sum_i f_i = F$ ist.

    $x_{i+1}$ kann nun in amortisierter Zeit
    $\O(1 + \max\{1, \log f_{i+1}\})$ eingefügt werden: \\
    Man läuft vom rechten Blatt (am weitesten rechts) bis ein Knoten
    $v$ erreicht wird mit $x_{i+1} >$ größter Schlüssel $k$ in $v$.
    Dafür wird $\O(1 + h)$ Zeit benötigt, wenn $h$ die Höhe von $v$
    von unten gesehen ist.
    Ist $v'$ das rechte Kind von $v$ und $k'$ der größte Schlüssel von
    $v'$, so ist $k < x_{i+1} < k'$ und $x_{i+1}$ wird in einem Kind
    von $v'$ eingefügt, aber nicht im rechten Knoten $v''$ von $v'$. \\
    Daher gilt $f_{i+1} \ge \text{Anzahl Bl"atter unterhalb von } v''
    \ge 2^{h-2}$, weil $v''$ Höhe $h - 2$ hat (Blätter Höhe $0$).
    Es folgt $h \le 2 + \log f_{i+1}$.
    Die Einfügung selbst (ohne Lokalisation) hat amortisierte Kosten
    $\O(1)$.
    Daher kann $x_{i+1}$ in $\O(1 + \max\{1, \log f_{i+1}\})$ eingefügt
    werden.

    Damit ist die Gesamtlaufzeit
    $\O(\sum_i (1 + \max\{1, \log f_i\}))
    = \O(n + \sum_i \max\{1, \log f_i\}) \\
    = \O(n + \sum_i (1 + \log f_i))
    = \O(n + \sum_i \log f_i)
    = \O(n + n \log \frac{F}{n})
    = \O(n \max\{1, \log \frac{F}{n}\})$. \\
    Die vorletzte Gleichheit erhält man mit
    $(\prod_i f_i)^{1/n} \le \frac{\sum_i f_i}{n} \quad |\log$ \\
    (geometrisches Mittel ist kleiner/gleich arithmetisches Mittel).
\end{Beweis}

\linie

\textbf{Fingersuche}
(ein Finger ist ein Zeiger auf ein Blatt):

\textbf{Lemma}:
In niveau-verbundenen (2, 4)-Bäumen kann man Fingersuche in
$\O(\log \min\{d, n - d\})$ durchführen, wobei $d$ der Abstand des
Fingers zum Ziel der Suche ist. \\
Niveau-verbunden heißt, dass die Kanten jeder Ebene in einer zirkulären
Liste stehen, d.\,h. zu jedem Knoten ist der linke und rechte Nachbar bekannt
(auch wenn Vaterknoten anders ist) und zu einem Knoten ganz
rechts ist der rechte Nachbar der Knoten der Ebene ganz links.

Suche von $x$ von einem Finger aus:
Laufe von dem Finger in Richtung Wurzel, bis ein Knoten $v$ erreicht wird,
sodass $x$ unterhalb dem $v$, dem linken Nachbar oder dem rechten Nachbar
von $v$ liegt.
Dann dreht man um und sucht ganz normal. \\
Die Laufzeit ist $\O(\text{H"ohe des erreichten Knotens})$,
diese ist $\log(\min\{1 + d, n - d + 1\})$.

\linie

\textbf{schnelles Mischen und Sortieren durch Mischen}: \\
Gegeben seien sortierte Folgen $S_1, S_2, \dotsc$ als (2, 4)-Bäume. \\
Ziel: Mische $S_1$ und $S_2$ zu $S$ in einen (2, 4)-Baum,
wobei $|S_1| = n$ und $|S_2| = m$ mit $m \le n$ ist.

\textbf{naiv}:
Füge $S_2$ nacheinander in $S_1$ ein.
Die Laufzeit dafür ist $\O(m \cdot \log(m + n))$, dies ist schlecht für
$m \approx n$ (gut für $m \ll n$).

\textbf{Satz}:
Man kann $S_1$ und $S_2$ in Zeit
$\O(m \cdot \log \frac{m + n}{m}) = \O(\log \binom{m + n}{m})$ zu einem
(2, 4)-Baum mischen, der $S_1$ und $S_2$ enthält.

\begin{lstlisting}
finger := "linkestes" Blatt in S_1
i := 1
while i <= m do
    suche nach x_i von finger aus
    fuege x_i ein
    finger := Zeiger auf neues Blatt
    i++
od
\end{lstlisting}

\pagebreak

\section{%
    Alternativen für (2, 4)-Bäume%
}

\textbf{Rot-Schwarz-Bäume}, \textbf{AVL-Bäume} und \textbf{Skip Lists}
können immer dann benutzt werden, wenn Elemente mit einer Ordnung verwaltet
werden sollen.
Sie ermöglichen das Suchen, Einfügen und Löschen in logarithmischer Zeit.

Manchmal kann \textbf{Hashing} jedoch ef"|fizienter sein, denn so ist
Suchen, Einfügen und Löschen in $\O(1)$ möglich.
Dies geht aber nur, wenn die zuverwaltende Menge aus ganzen Zahlen besteht.
Außerdem können keine Anfragen der Art "`größtes Element kleiner $10$"'
beantwortet werden.

\section{%
    \emph{Einschub}: Amortisierte Analyse%
}

\textbf{Sinn und Zweck}:
Man möchte zeigen, dass nicht alle Operationen auf einer bestimmten
Datenstruktur teuer sind, d.\,h. im Durchschnitt sind die Operationen billig,
auch wenn eine einzelne Operation teuer sein kann.

\textbf{Intuition}:
Mit jeder Operation auf der Datenstruktur wird eine konstante Zahl von Euros
einbezahlt, die für den tatsächlichen Aufwand einer Operation bezahlt werden
müssen, dessen Rest aber bei billigen Operationen angespart werden kann, um
teurere Operationen zu bezahlen.

\linie

\textbf{Beispiel 1: Inkrementierung im Binärregister} \\
Hier entspricht der Aufwand der Anzahl der Überträge.
Im schlimmsten Fall müssen $\log_2 n$ Überträge gemacht werden
($n$ größte speicherbare Zahl). \\
Man kann zeigen: Wenn man bei Null anfängt und bei jeder Inkrementierung immer
$1$ Euro einbezahlen, so hat die Datenstruktur immer genügend Geld, um
die Überträge zu bezahlen (ein Übertrag kostet $1$ Euro).

Hier kommt die \textbf{Potentialfunktion} ins Spiel:
Sie ist eine untere Schranke für den Kontostand und entspricht hier der
Anzahl Einsen in der aktuellen Zahl
(dies müsste man zeigen).

Nach $i$ Inkrementierungen hat man $i$ Euro eingezahlt, der Kontostand
ist nicht-negativ, d.\,h. man hat nicht mehr als $i$ Euro für Überträge
ausgegeben.
Im Durchschnitt/amortisiert wurden also $\le 1$ Überträge gemacht.

\linie

\textbf{Beispiel 2: Konstruktion eines Heaps in $\O(n)$ Zeit}

\linie

\textbf{Beispiel 3: Spalten und Vertauschen in (2, 4)-Bäumen} \\
(Stehlen ist uninteressant, da nicht propagierend.)

Die Potentialfunktion (untere Schranke für Kontostand) ist hier \\
$\phi = 2 \cdot \#1 + 1 \cdot \#2 + 0 \cdot \#3 + 2 \cdot \#4 + 4 \cdot \#5$.

Pro Einfügen und Löschen werden $5$ Euro auf das Konto des (2, 4)-Baums
bezahlt, die gespart werden können, aber auch für Spalt-/Verschmelzoperationen
ausgegeben werden müssen.
Behauptung: Der (2, 4)-Baum hat immer genügend Geld, um Spalten/Verschmelzen
zu bezahlen.

Hat man dies gezeigt, so hat man, wenn man mit einem leeren Baum anfängt,
nach $i$ Operationen einen Kontostand von höchstens $5i$ Euro.
Weil der Kontostand nicht-negativ ist, ist der Aufwand für Spalten und
Verschmelzen $\le 5i$.

\pagebreak

\begin{Beweis}
    \begin{itemize}
        \item
        Durch das bloße Einfügen bzw. Löschen eines Blattes erhöht sich
        das Potential um maximal $2$ bzw. $1$.
        Also ist $\phi$ weiterhin eine gültige untere Schranke
        ($5$ Euro wurden eingezahlt).

        \item
        Beim Spalten eines Knotens mit $5$ Kindern erhöht sich das
        Potential des Vaterknotens um max. $2$, aus dem Kind mit Potential
        $4$ entstehen zwei Kinder mit Potential $1$ und $0$.
        Also nimmt das Potential des Baums um mindestens $1$ ab.
        Mit diesem Euro kann die Operation bezahlt werden und $\phi$ bleibt
        untere Schranke für den Kontostand.

        \item
        Beim Verschmelzen zweier Knoten mit $1$ Kind und $2$ Kindern
        erhöht sich das Potential des Vaterknotens um max. $1$, das Potential
        der Kinder $2$ und $1$ ändert sich zu $0$, da ein Knoten mit
        drei Kindern entsteht.
        Also nimmt das Potential um mindestens $2$ ab.
        Die Operation kann bezahlt werden und $\phi$ bleibt
        untere Schranke für den Kontostand.

        \item
        Beim Stehlen sind zwei Knoten mit $1$ bzw. $3$ oder $4$ Kinder
        vorhanden.
        Deren Potential ändert sich von $2$ bzw. $0$ oder $2$ zu
        $1$ bzw. $1$ oder $0$.
        Sonst verändert sich kein Knoten, daher auch nicht das Potential.
        Also nimmt bleibt das Potential gleich oder sinkt um $3$.
        $\phi$ ist weiterhin untere Schranke für Kontostand, denn
        es muss beim Stehlen nichts bezahlt werden.
    \end{itemize}
    Also bleibt $\phi$ durchgängig untere Schranke für den Kontostand. \\
    Beim Verschmelzen/Spalten ist immer Geld vorhanden, um die Operation
    zu bezahlen.
\end{Beweis}

\pagebreak
