\chapter{%
    Sortieren%
}

\section{%
    Sortierproblem und Aufwandsanalyse%
}

Gegeben sei eine Menge $S = \{A[1], \ldots, A[n]\}$ aus einem total geordneten
Universum.
Gesucht ist eine Permutation $\pi$ von $\{1, \ldots, n\}$, sodass
$A[\pi(1)] \le \dotsb \le A[\pi(n)]$ ist.

Zum Beispiel ist für $S = \{2, 7, 1, 3, 5\} \subseteq \mathbb{N}$ das
gesuchte $\pi$ gegeben durch {\footnotesize $\begin{pmatrix}1 & 2 & 3 & 4 & 5\\
3 & 1 & 4 & 5 & 2\end{pmatrix}$}.

Den "`lexikalischen Vergleich"' kann man definieren durch
$w_1 x r < w_1 y r \;\Leftrightarrow\; x < y$ mit $x, y \in \Sigma$,
$w_1, r \in \Sigma^\ast$.

\linie

\textbf{Aufwand}:
\emph{Platz}, der benötigt wird, um $\pi$ zu berechnen; \qquad
\emph{Anzahl der Arbeitsschritte}; \\
\emph{Zeit} für die Berechnung von $\pi$ auf einer Maschine mit $p$ Prozessoren
(= Anzahl der Arbeitsschritte für $p = 1$);
\emph{Anzahl der I/O-Operationen} (wichtig beim Sortieren großer Datenmengen).

\textbf{Bedingungen}, unter denen der Aufwand abgeschätzt werden soll:
\begin{itemize}
    \item \textbf{Worst-Case-Analyse}: Eingabe $S$ kann beliebig permutiert
    sein, interessant ist obere Schranke, die immer gilt

    \item \textbf{probabilistische Analyse}: Eingabe stammt aus einer
    Wahrscheinlichkeitsverteilung über alle Eingaben,
    Ziel ist Verfahren, das eine gute (erwartete) Laufzeit erzielt

    \item \textbf{randomisierte Algorithmen}: Es kann nützlich sein, dass
    Algorithmen den weiteren Fortgang vom Ergebnis eines Zufallsgenerators
    abhängig machen.
    Es interessiert uns die \emph{erwartete} Laufzeit bei beliebiger Eingabe.
\end{itemize}

\section{%
    Bubblesort%
}

\begin{tabular}{p{6.0cm}p{10.1cm}}
    \begin{minipage}[c]{6.0cm}
        \begin{lstlisting}
i := n
while (i > 1) do
    j := 1
    while (j < i) do
        if A[j] > A[j + 1]
            swap(A[j], A[j + 1])
        j := j + 1
    od
    i := i - 1
od
        \end{lstlisting}
    \end{minipage} &
    \begin{minipage}[c]{10.1cm}
        Im ersten Durchlauf wandert das größte Element ganz nach hinten,
        im zweiten Durchlauf wandert das zweitgrößte Element an die vorletzte
        Position usw. \\
        \emph{Beobachtung}: Die Menge der Elemente in $A[1], \ldots, A[n]$
        bleibt während des Algorithmus gleich. \\
        \emph{Lemma}: Für ein festes $i$ ist
        $A[i] = \max_{j = 1, \ldots, i} A[j]$ am
        Ende der äußeren Schleife. \\
        \emph{Satz}: Nach der Durchführung liegt $A[\;]$ in sortierter
        Reihenfolge vor.
    \end{minipage}
\end{tabular}

\vspace{12pt}
\linie

\textbf{Problem bei der Laufzeitanalyse}:
Die \emph{Implementierungssprache} sowie der \emph{Rechner}, auf dem der
Algorithmus ausgeführt wird, haben erheblichen Einfluss auf die Zeit, die
dieser zur Ausführung braucht.
Daher ist die Zeitmessung nicht geeignet, um die Laufzeit/Qualität eines
Algorithmus zu bestimmen. \\
Besser scheint es, die Anzahl der aufgeführten \textbf{Instruktionen} beim
Lösen eines bestimmten Problems zu zählen.
Dabei nimmt man an, dass eine Instruktion \emph{konstante Zeit}
($1$ Zeiteinheit) benötigt.
Was ist jedoch eine Instruktion?
Ist \code{swap} eine oder drei Instruktionen, oder noch mehr in Assembler?

Die Anzahl der Instruktionen hängt zudem von der CPU-Architektur ab.
Zur Analyse eines Algorithmus will man eine invariante Größe bzgl. Sprache und
CPU-Architektur wählen. \\
Dazu zählt man nur die Anzahl der \textbf{Vergleiche}, die durchgeführt werden.

Man nimmt an, dass die Beschreibung (insbesondere die Länge) des Algorithmus
unabhängig von der Eingabe ist.
Sei $C$ die \textbf{Anzahl an Instruktionen} in der Beschreibung
(nicht im Ablauf) des Algorithmus. \\
$C$ hängt zwar von Sprache/CPU-Architektur ab, ist aber konstant.

\textbf{Behauptung}: Wenn der Algorithmus terminiert, tritt bei der Ausführung
spätestens nach jeweils $C$ Instruktionen ein Vergleich auf.

\begin{Beweis}
    Sobald $> C$ Instruktionen ausgeführt wurden, wurde mindestens
    eine Instruktion mehrfach ausgeführt.
    Falls zwischen der ersten und zweiten Ausführung kein Vergleich ausgeführt
    wurde, gibt es keine Möglichkeit den Kontrollfluss dazwischen zu ändern
    und es kommt zu einer Endlosschleife.
\end{Beweis}

Wenn man nur die Vergleiche zählt, kann man also die "`Laufzeit"' (Anzahl der
ausgeführten Instruktionen) bis auf einen konstanten Fehler abschätzen, denn
es gilt $n_{\text{Ins}} \le C \cdot n_{\text{Vgl}}$.

\linie

Bei Bubblesort beträgt die Gesamtzahl an Vergleichen $\le n^2 + n$.
Daher beträgt die Anzahl ausgeführter Instruktionen $\le C \cdot (n^2 + n)$,
wobei $C$ von Sprache/Implementierung abhängt.

Die \textbf{$\O$-Notation} erlaubt es nun, Konstanten und dominierte Terme
wegzulassen: \\
$\O(f(n)) = \{g: \mathbb{N} \rightarrow \mathbb{R} \;|\;
\exists_{C > 0} \exists_{n_0 \in \mathbb{N}} \forall_{n \ge n_0}\;
g(n) \le C \cdot f(n)\}$.
Bspw. ist $\O(n^2)$ die Menge der Funktionen, die für hinreichend große $n$
nicht schneller wachsen als $n^2$.

Bubblesort hat also Worst-Case-Laufzeit $\O(n^2)$
(bzw. keine schlechtere Laufzeit).
Es macht einen großen Unterschied, ob Algorithmen Laufzeiten mit
$\O(n)$, $\O(n \log n)$ oder $\O(n^2)$ haben.

\section{%
    Mergesort%
}

Mergesort sortiert eine Datenreihe, indem sie so weit halbiert wird, bis sie
nur noch aus ein- und zweielementigen Tupeln besteht.
Diese werden sortiert und dann wieder in sortierter Reihenfolge verschmolzen
(engl. \emph{merge}). \\
Um eine Sequenz $a_1, \ldots, a_{\lceil n/2 \rceil},
a_{\lceil n/2 \rceil + 1}, \ldots, a_n$ zu sortieren, werden zunächst
$a_1, \ldots, a_{\lceil n/2 \rceil}$ und
$a_{\lceil n/2 \rceil + 1}, \ldots, a_n$ sortiert und dann miteinander
vermischt. \\
Mergesort handelt nach dem \textbf{Divide-\&-Conquer-Paradigma}
(\emph{teile und herrsche}).

\linie

\textbf{Laufzeitaufwand von Mergesort}: Der Gesamtlaufzeit $T(n)$, um
eine Liste mit $n$ Elementen zu mischen, lässt sich ausdrücken als
$T(n) = 2 \cdot T(\frac{n}{2}) + n$, wobei $T(2) = 1$.
Eine solche rekursive Formel würde sich mit dem \emph{Master-Theorem}
analytisch lösen lassen.

Intuitiv nimmt man an, dass $n = 2^k$ (sonst erweitert man die Eingabe um
Dummyzahlen, was die Problemgröße nur um konstanten Faktor verändert).
Um zwei Folgen der Länge $\frac{n}{2^i}$ zu mischen, sind
$2 \cdot \frac{n}{2^i}$ Vergleiche nötig.
Im Laufe des Algorithmus tauchen $2^i$ Folgen der Länge $\frac{n}{2^i}$ auf,
also $\frac{1}{2} \cdot 2^i$ Paare.
Daher ist der Aufwand zum Mischen aller Folgen der Länge $\frac{n}{2^i}$
gleich $\frac{1}{2} \cdot 2^i \cdot 2 \cdot \frac{n}{2^i} = n$.
Es treten $\sim \log_2 n$ viele verschiedene Teilfolgenlängen auf,
daher ist der Gesamtaufwand $\O(n \log n)$.

Mergesort ist ein \textbf{optimales Sortierverfahren}. \\
Man kann zeigen, dass das Sortierproblem nicht schneller als
$\O(n \log n)$ zu lösen ist
(zumindest nicht mit vergleichsbasierten, deterministischen Algorithmen,
siehe unten).

\pagebreak

\section{%
    Insertionsort%
}

\begin{lstlisting}
Insertionsort(A, n)
    for j = 1 to n - 1 do
        key := A[j]
        i := j - 1
        while (i >= 0 and A[i] > key) do
            A[i + 1] := A[i]
            i := i - 1
        od
        A[i + 1] := key
    od
\end{lstlisting}

\textbf{Beschreibung}: \\
Um eine Liste $A[0], \ldots, A[n - 1]$ mit $n$ Elementen zu sortieren, geht
Insertionsort im $j$-ten Schritt davon aus, dass die
Liste $A[0], \ldots, A[j - 1]$
schon sortiert ist ($1 \le j \le n - 1$). \\
Der \code{key} $= A[j]$ wird dann an der richtigen Stelle in dieser Liste
eingefügt, sodass die Liste
$A[0], \ldots, A[j - 1], A[j]$ sortiert ist.
Dazu werden die größeren Elemente (als der \code{key}) allesamt
"`nach rechts geschoben"' und \code{key} eingefügt (engl. \emph{insert}).

\linie

\textbf{Best-Case}:
Insertionsort hat ein asymptotisches Laufzeitverhalten von
$\O(n)$ im Best-Case.
Dieser tritt ein, falls die Liste anfangs schon sortiert ist.

\textbf{Worst-Case}:
Insertionsort hat ein asymptotisches Laufzeitverhalten von
$\O(n^2)$ im Worst-Case.
Dieser tritt ein, falls die Liste anfangs falsch herum sortiert ist.

\section{%
    Heapsort%
}

\textbf{Heapsort} basiert auf der Datenstruktur \emph{Heap} und funktioniert
wie folgt:
Füge zunächst alle $n$ Elemente in den Heap ein,
entferne dann $n$-mal das Maximum aus dem Heap und gebe es aus.

\textbf{Heap}:
Ein Heap (organisierter Haufen) ist ein Baum mit ausgezeichneter Wurzel, wobei
die zu organisierenden Elemente in den Knoten des Baums stehen. \\
Dabei gilt die sog. \textbf{Heap-Eigenschaft}: Das Element jedes Knotens ist
immer größer/gleich den Elementen der Kinder des Knotens.
%Heaps sind nützliche Datenstrukturen, die nicht nur zum Sortieren geeignet
%sind:
%Man kann sie z.\,B. brauchen, wenn Ereignisse in einer bestimmten Reihenfolge
%abgearbeitet werden müssen, sich diese Reihenfolge aber dynamisch ändert
%oder neue Elemente hinzukommen.

\linie

Wir fordern \textbf{binäre, balancierte Heaps}, bei denen nur "`rechts
unten"' Blätter fehlen.
Man kann solche Heaps mit $n$ Knoten in einem Array $A[0], \ldots, A[n - 1]$
schichtweise in einem Array speichern, welches die vollständige Stuktur des
Heaps widerspiegelt.
Dabei steht die Wurzel an Stelle $0$ des Arrays.
Der Vaterknoten eines Knotens mit Position $i$ steht an Position
$\lfloor \frac{i - 1}{2} \rfloor$.
Der linke bzw. rechte Kindknoten eines Knotens mit Position $i$ steht an
Position $2i + 1$ bzw. $2i + 2$.
Nur Knoten mit Position $i \le \lfloor \frac{n}{2} \rfloor - 1$ und
$i \le \lfloor \frac{n}{2} \rfloor - 2$ haben ein linkes oder rechtes Kind.

Umgekehrt repräsentiert ein Array mit $n$ Elementen $V[0], \ldots, V[n - 1]$
einen Heap, falls \\
$V[i] \ge V[2i + 1]$ für alle $i = 0, \ldots, \lfloor \frac{n}{2} \rfloor - 1$
und
$V[i] \ge V[2i + 2]$ für alle $i = 0, \ldots, \lfloor \frac{n}{2} \rfloor - 2$
\\
(d.\,h. Heap-Eigenschaft ist erfüllt).
Dabei steht in $V[0]$ das größte Element und jede Folge von Werten von einem
Knoten absteigend zu einem Blatt ist monoton fallend.

\linie

\textbf{\code{heapify}}:
\code{heapify} kann mit einer Voraussetzung die Heap-Eigenschaft eines Baums
von einem gewissen Index an wiederherstellen.

\textbf{Aufbau von \code{heapify}}:
Als Eingabe erwartet \code{heapify} ein Array $V[\;]$ und einen Index
$top \in \{0, \ldots, n - 1\}$ mit der Annahme,
dass für alle $i = top + 1, \ldots, n - 1$ mit $2i + 1 < n$ bzw.
$2i + 2 < n$ gilt, dass $V[i] \ge V[2i + 1]$ bzw. $V[i] \ge V[2i + 2]$
(d.\,h. die Heap-Eigenschaft ist für alle Indizes $i = top + 1, \ldots, n - 1$
erfüllt). \\
Die Ausgabe ist ein nur in den Indizes $top, \ldots, n - 1$ verändertes
Array, bei dem die Heap-Eigenschaft für alle Indizes $i = top, \ldots, n - 1$
erfüllt ist.

\textbf{Funktionsweise von \code{heapify}}:
Betrachte die Kinder des Knotens.
Sind beide kleiner/gleich dem Knoten, dann ist \code{heapify} beendet.
Ansonsten tausche den Inhalt des Knotens mit dem größten Inhalt seiner beiden
Kinder und betrachte dieses Kind rekursiv.

\textbf{Laufzeit von \code{heapify}}:
Eine mögliche Verletzung der Heap-Eigenschaft wandert immer eine Tiefe nach
unten.
Somit ergibt sich eine Laufzeit von $\O(\log n)$.

\linie

\textbf{Operationen des Heaps}:
Hinzufügen eines Elements zum Heap (\code{insert}),
Entfernen des Maximums aus dem Heap, welches immer in der Wurzel steht
(\code{remove_max}),
Ändern eines Elements im Heap (\code{change_key}).

\textbf{Funktionsweise von \code{remove_max}}:
Entferne zunächst das Element aus der Wurzel und gebe es zurück.
Danach stelle durch Kopieren des Inhalts des "`letzten"' Blatts in die Wurzel
und anschließendes Anwenden von \code{heapify} auf der Wurzel die
Heap-Eigenschaft wieder her.

\textbf{Funktionsweise von \code{insert}}:
Füge neues Blatt am "`Ende"' des Heaps ein.
Danach prüfe, ob die Heap-Eigenschaft zum Vaterknoten verletzt ist.
Falls ja, tausche mit Vaterknoten und überprüfe diesen rekursiv,
falls nein, ist die Prozedur beendet und der Baum wieder ein Heap.

\textbf{Kosten von \code{remove_max} und \code{insert}}:
$\O(\log n)$

\textbf{Funktionsweise \code{change_key}}:
\code{change_key} ändert den Wert eines Knotens im Heap.
Wird der Schlüssel erhöht, so muss mit dem Vaterknoten verglichen,
ggf. getauscht und rekursiv der Vaterknoten überprüft werden.
Wird der Schlüssel verringert, so muss \code{heapify} auf den Knoten aufgerufen
werden.
Die Laufzeit von \code{change_key} beträgt also in jedem Fall $\O(\log n)$.

\textbf{Theorem}: Ein binärer Heap unterstützt \code{insert}, \code{remove_max}
sowie \code{change_key} jeweils in $\O(\log n)$.
Ein Heap mit $n$ Elementen kann auch in $\O(n)$ konstruiert werden.

\textbf{Anmerkung}:
Es gibt spezialisierte Heaps, die manche Operationen besser können.
Ist z.\,B. bekannt, dass bei \code{change_key} der Wert immer nur erhöht wird
und die Maxima während der Lebenszeit des Heaps monoton fallen, so gibt
es spezielle Fibonacci-Heaps, die \code{change_key} in amortisiert $\O(1)$
ausführen können.

\linie

\textbf{Möglichkeiten für Konstruktion des Heaps}:
Entweder führt man $n$ \code{insert}-Operationen aus oder man schreibt die
zu organisierenden Daten zuerst beliebig in $V$ und stellt dann die
Heap-Struktur wieder her.
Die erste Variante hat eine Laufzeit von $\O(n \log n)$.

\textbf{Konstruktion des Heaps in $\O(n)$}:
Mit der zweiten Variante kann man den Heap in $\O(n)$ konstruieren.
Zunächst schreibt man die Daten in beliebiger Reihenfolge in den Baum.
Dann ruft man \code{heapify} für jeden Knoten auf, von hinten nach vorne
beginnend mit dem "`letzten"'. \\
Eine simple Laufzeitanalyse ergibt ein $\O(n \log n)$-Verhalten
($n$-mal \code{heapify}).
Man kann jedoch beobachten, dass \code{heapify} für untere Knoten erheblich
schneller ist wie für obere.

\textbf{amortisierte Laufzeitanalyse}:
Bei dieser Art von Laufzeitanalyse von Operationenfolgen betrachtet man nicht
die maximalen Kosten jedes einzelnen Schritts, sondern man berücksichtigt
verschiedene Laufzeiten bei unterschiedlichen Aufrufen.
Somit kann sich im gesamten Worst-Case-Verhalten eine bessere Laufzeitschranke
ergeben.

Ein Knoten der Höhe $h$ hat Kosten $h$ (max. Aufrufe aller \code{heapify}s
für den Knoten).
Lege auf jeden Knoten seine Kosten in Form von Münzen.
Die Gesamtzahl an Münzen im Baum entspricht dann der Gesamtlaufzeit aller
\code{heapify}s.
Geschickte Zählung: Verteile die Münzen jedes Knotens auf dem Pfad zu
einem Blatt, der zunächst einmal "`links"', dann immer "`rechts"' führt
(auf jede Kante eine Münze legen).
Man kann beobachten, dass die Pfade disjunkt sind.
Somit liegt auf jeder Kante maximal eine Münze und die Gesamtanzahl an Münzen
im Baum ist kleiner/gleich wie die Anzahl an Kanten $n - 1$
(falls der Baum $n$ Knoten hat).
Also ist die Gesamtlaufzeit aller \code{heapify}s $\O(n)$.

\section{%
    Quicksort%
}

\textbf{Quicksort} funktioniert wie Mergesort gemäß "`Teile \& Herrsche"'.
Der große Unterschied besteht jedoch darin, dass Quicksort hier randomisiert
ist, d.\,h. der Algorithmus "`würfelt"' und macht das weitere Vorgehen
vom Ergebnis des Zufallsexperiments abhängig.
Man will allerdings garantieren, dass am Ende immer das richtige Ergebnis
herauskommt.
Die Laufzeitanalyse ergibt dabei einen Erwartungswert für die Laufzeit,
der unabhängig von der Eingabe ist.

\begin{lstlisting}
Quicksort(A[1 ... n])
    waehle ein A[p] mit p in {1, ..., n} zufaellig gleichverteilt (u.a.r.)

    rearrangiere A in A_L A[p] A_R, wobei fuer alle a in A_L gilt, dass
    a <= A[p], sowie fuer alle a in A_R gilt, dass a > A[p]

    Quicksort(A_L)
    Quicksort(A_R)
\end{lstlisting}

Dabei steht u.a.r. für \emph{uniformly at random} und bedeutet
"`gleichverteilt"'.
\code{A[p]} heißt auch \textbf{Pivotelement}.
Für die Rearrangierung sind $n - 1$ Vergleiche notwendig.

\linie

\textbf{Laufzeitanalyse}:
Angenommen, es wird zufällig immer das kleinste Element als Pivotelement
gewählt.
Dann ist $A_L$ immer leer und der nächste Aufruf muss $n - 1$ Elemente
sortieren.
Dies ergibt eine Laufzeit von $\O(n^2)$.
Jedoch sollte dieser Fall fast nie eintreten, weil die $A[p]$ immer zufällig
gewählt werden.

Für die randomisierte Laufzeitanalyse benötigt man ein paar grundlegende
Definitionen:

\textbf{reelle Zufallvariable}: Eine Funktion, die jedem Ergebnis eines
Zufallsexperiments eine reelle Zahl zuweist.
Beispiel Würfeln mit zwei Würfel:
Dann ist $x_{ij} = i + j$ eine Zufallsvariable, wobei $ij$ das Ergebnis
bedeutet, bei dem der erste Würfel $i$ Augen und der zweite $j$ Augen zeigt.

\textbf{Erwartungswert}: Ein gewichteter Durchschnitt aller auftretenden
Werte der Zufallsvariable gemäß der Wahrscheinlichkeit, wobei der
Erwartungswert einer bestimmten Zufallsvariable zugewiesen wird.
Somit gibt der Erwartungswert die durchschnittlich zu erwartenden Kosten
etc. an.
Beispiel Würfeln mit zwei Würfel:
Sei $X$ die Summe der Augenzahlen beider Würfel, dann ist
der Erwartungswert $E(X) = 2 \cdot \frac{1}{36} + 3 \cdot \frac{2}{36} +
\dotsb + 11 \cdot \frac{2}{36} + 12 \cdot \frac{1}{36} = 7$. \\
Der Erwartungswert der Summe von Zufallsvariablen ist die Summe
der Erwartungswerte.

\linie

Im Folgenden wird gezeigt, dass die erwartete Laufzeit unabhängig von der
Eingabe $\O(n \log n)$ ist.
Man kann auch zeigen, dass es sehr unwahrscheinlich ist, dass die Laufzeit
stark vom Erwartungswert abweicht.

\begin{Beweis}
    Seien $s_1, \dotsc, s_n$ die zu sortierenden Elemente gemäß der Ordnung,
    d\,h. $s_i \le s_{i+1}$ für $i = 1, \dotsc, n - 1$.
    Definiere die Zufallvariable
    {\tiny $x_{ij} = \begin{cases}1 & s_i, s_j
    \text{ werden während des kompletten Quicksort miteinander verglichen} \\
    0 & \text{sonst}\end{cases}$}.
    Beachte, dass $s_i$ und $s_j$ höchstens einmal miteinander verglichen
    werden können.
    Dann beträgt die Gesamtlaufzeit $\sum_{i < j} x_{ij}$ (Gesamtzahl der
    Vergleiche, $x_{ij} = x_{ji}$ nicht doppelt zählen), wobei über
    $i, j = 1, \dotsc, n$ summiert wird. \\
    Die erwartete Laufzeit beträgt somit $E(\sum_{i < j} x_{ij})
    = \sum_{i < j} E(x_{ij})$.

    Was ist $E(x_{ij})$?
    Sei $p_{ij}$ die Wahrscheinlichkeit, dass $s_i$ und $s_j$
    während des kompletten Quicksort miteinander verglichen werden,
    dann ist $E(x_{ij}) = 1 \cdot p_{ij} + 0 \cdot (1 - p_{ij}) = p_{ij}$
    (nach Wahrscheinlichkeit gewichteter Durchschnitt der Werte, die $x_{ij}$
    annehmen kann).

    Was ist nun $p_{ij}$?
    Man kann den Ablauf von Quicksort als Binärbaum darstellen, wobei jeder
    Knoten ein Pivotelement darstellt und das linke bzw. rechte Kind dem
    Pivotelement von $A_L$ bzw. $A_R$ entspricht.
    Schreibe nun die Knoten in diesem Baum als Permutation in Levelorder
    (Breitensuche: Ebene für Ebene von oben nach unten, dort links nach rechts)
    auf.

    \pagebreak

    Wenn $s_i$ mit $s_j$ verglichen wird, dann befindet sich kein Element $s_k$
    mit $s_i < s_k < s_j$ vor $s_i$ und $s_j$ in der Permutation, da
    sonst dieses $s_k$ als Pivotelement $s_i$ in $A_L$ und $s_j$ in $A_R$
    sortiert hätte (somit wären die beiden nicht verglichen worden).
    Umgekehrt verhält es sich genau so. \\
    Betrachtet man die Elemente $s_i, s_{i+1}, \dotsc, s_{j-1}, s_j$, so tritt
    jedes mit gleicher Wahrscheinlichkeit als erstes dieser Elemente in der
    Permutation auf.
    Die Wahrscheinlichkeit, dass kein $s_k$ mit $s_i < s_k < s_j$ vor $s_i$ und
    $s_j$ auftritt, ist gleichbedeutend mit der Wahrscheinlichkeit, dass
    $s_i$ oder $s_j$ als erstes Element auftritt.
    Also ist $p_{ij} = \frac{2}{j - i + 1}$, da es $j - i + 1$ Elemente in
    dieser Liste gibt.

    Also ist $\sum_{i < j} E(x_{ij}) = \sum_{i < j} p_{ij}
    = \sum_{i < j} \frac{2}{j - i + 1}
    = \sum_{i=1}^n (1 + \sum_{j=i+2}^n \frac{2}{j - i + 1})
    = \sum_{i=1}^n (1 + \sum_{j=1}^{n-i-1} \frac{2}{j})$ \\
    $= n + 2 \cdot \sum_{i=1}^n \sum_{j=1}^{n-i-1} \frac{1}{j}
    \le n + 2 \cdot \sum_{i=1}^n \sum_{j=1}^{n} \frac{1}{j}
    \le n + 2 \cdot \sum_{i=1}^n \log n \in \O(n \log n)$.
\end{Beweis}


\section{%
    Grenze von vergleichsbasiertem Sortieren%
}

Gibt es Sortieralgorithmen, die eine bessere Schranke als $\O(n \log n)$
besitzen?
Zunächst muss ein Sortierverfahren stets alle Elemente der Eingabe betrachten.
Andernfalls könnte man in einem nicht betrachteten Element eine Zahl
"`verstecken"', die der berechneten Sortierung widerspricht.
Daher benötigt jeder Sortieralgorithmus mindestens $\Omega(n)$.

\textbf{Behauptung}: Jeder vergleichsbasierte deterministische
Sortieralgorithmus muss im Worst-Case $\Omega(n \log n)$ Zeit aufwenden.

\begin{Beweis}
    Man kann die Ausführung eines deterministischen Algorithmus als Folge
    von Vergleichen auf"|fassen.
    Wegen des Determinismus führt der Algorithmus je nach Ausgang eines
    Vergleichs einen bestimmten nächsten Vergleich aus (oder terminiert).
    Somit lässt sich der Ablauf als Binärbaum darstellen (wahrer/falscher
    Vergleich).
    Der Algorithmus stoppt nach einer gewissen Anzahl an Vergleichen und gibt
    eine Permutation der Eingabe aus.
    Dies entspricht einem Blatt in diesem Baum.
    Verschiedene Permutationen (derselben Eingabe) müssen in verschiedenen
    Blättern des Baums enden, sonst wäre für eine Eingabe die Ausgabe falsch.
    Es gibt $n!$ verschiedene Permutationen und ein Binärbaum der Höhe $h$ hat
    höchstens $2^h$ viele Blätter.
    Es muss wegen des vorherigen Satzes mindestens so viele Blätter wie
    Permutationen geben.
    Also gilt $2^h \ge n!$ bzw.
    $h \ge \log n! \ge \log \left(\frac{n}{e}\right)^n
    = n \cdot \log \frac{n}{e} \in \Omega(n \log n)$
    (\textsc{Stirling}-Formel).
    Die Höhe des Baums entspricht der Anzahl an Vergleichen im Worst-Case, also
    ist die Worst-Case-Laufzeit $\Omega(n \log n)$.
\end{Beweis}

Der Beweis gilt nur für deterministische Algorithmen (also eigentlich nicht
für randomisierte Algorithmen wie Quicksort).
Man kann allerdings zeigen, dass randomisiertes Sortieren ebenfalls
erwartet $\Omega(n \log n)$ Zeit braucht.

Nimmt man an, die zu sortierenden Objekte sind Zahlen beschränkter Größe,
so gibt es (nicht vergleichsbasierte) Sortierverfahren, die die
$\Omega(n \log n)$-Schranke schlagen (z.\,B. Countingsort, Radixsort).

\pagebreak
