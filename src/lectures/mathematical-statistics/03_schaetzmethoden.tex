\chapter{%
    Schätzmethoden%
}

\begin{Bem}
    Gegeben sind wieder ein reguläres statistisches Modell
    $\P = \{\PP_\theta \;|\; \theta \in \Theta\}$ und einen Vektor $x$ der Beobachtungen,
    der als Realisierung eines Zufallsvektors $X$ mit unbekannter Verteilung $P_\theta$
    interpretiert wird.

    Die Aufgabe ist nun, das unbekannte $\theta$ unter Verwendung der Beobachtung $x$ zu schätzen.

    Häufig ist man dabei nicht an $\theta$ selbst, sondern nur an $q(\theta)$ für eine
    fest vorgegebene, messbare Funktion $q\colon \Theta \rightarrow \real$ interessiert.

    Das Ziel ist es, $q(\theta)$ mittels einer geeigneten Statistik $T$ zu schätzen.
    $T(x)$ wird als \begriff{konkreter Schätzwert} für $q(\theta)$ verwendet.
    $T(X)$ ist dagegen der \begriff{zufallsabhängige Schätzer} für $q(\theta)$.
\end{Bem}

\begin{Bsp}
    Eine Anzahl von Messungen einer physikalischen Größe $\mu$ kann durch ein Messmodell
    mit $n$ Zufallsvariablen $X_i = \mu + \varepsilon_i$, $i = 1, \dotsc, n$,
    simuliert werden,
    wobei die Messfehler $\varepsilon_1, \dotsc, \varepsilon_n \sim \N(0, \sigma^2)$ unabhängig
    mit unbekannter Varianz $\sigma^2$ sein sollen.
    Natürlich sind dann auch die $X_1, \dotsc, X_n \sim \N(\mu, \sigma^2)$ unabhängig.
    $\theta$ ist hier gleich $\theta = (\mu, \sigma^2)$, gesucht ist $\mu$.
    Man schätzt $q(\theta) = \mu$ nun durch die Statistik
    $T(X) := \frac{1}{n} \sum_{i=1}^n X_i$ mit $X = (X_1, \dotsc, X_n)$.
\end{Bsp}

\section{%
    Substitutionsprinzip%
}

\begin{Bem}
    Die Idee des \begriff{Substitutionsprinzips} ist es,
    den unbekannten Parameter in Beziehung zu Größen zu setzen, die sich leicht schätzen lassen.
    Beispielsweise lässt sich die (wahre, aber unbekannte) Verteilungsfunktion durch die
    empirische Verteilungsfunktion oder Momente lassen sich durch empirische Momente schätzen.
\end{Bem}

\subsection{%
    Häufigkeitssubstitution%
}

\begin{Bem}
    Bei der Häufigkeitssubstitution werden bei diskreten Modellen die Wahrscheinlichkeiten
    der Elementarereignisse durch relative Häufigkeiten geschätzt.

    Im Folgenden bezeichnet $\int_\real f(x) dF(x)$ das
    \begriff{\name{Lebesgue}-\name{Stieltjes}-Integral}.
    Man sagt, $f(x)$ sei \begriff{bzgl. $F$ integrierbar}, falls $F$ monoton und
    $f$ bzgl. $\mu_F$ Lebesgue-integrierbar ist, in diesem Fall setzt man
    $\int_\real f(x) dF(x) := \int_\real f(x) d\mu_F$.
    Dabei ist $\mu_F$ ein Maß auf $\real$, das durch
    $\mu_F([a, b)) = F(b-0) - F(a-0)$ und
    $\mu_F((a, b]) = F(b+0) - F(a-0)$ eindeutig festgelegt ist.

    Ist $F$ stetig dif"|ferenzierbar, dann gilt $\int_a^b f(x) dF(x) = \int_a^b f(x)F'(x) \dx$.
    Insbesondere gilt:
    Wenn $F$ die Verteilungsfunktion einer Zufallsvariable $X$, die eine L.-B.-Dichte besitzt,
    und $f$ eine messbare Funktion mit $\EE(|f(X)|) < \infty$ ist,
    dann ist die Ableitung $F'$ die Dichte von $X$ und es gilt
    $\EE(f(X)) = \int_\real f(x)F'(x) \dx = \int_\real f(x) dF(x)$.
\end{Bem}

\linie

\begin{Def}{Häufigkeitssubstitution}
    Seien $X_1, \dotsc, X_n$ i.i.d. mit unbekannter Verteilungsfunktion $F$
    und $F_n$ die \begriff{empirische Verteilungsfunktion} mit
    $F_n(x) := \frac{1}{n} \sum_{i=1}^n \1_{\{X_i \in (-\infty, x]\}}$ für $x \in \real$
    (relative Häufigkeit der $X_1, \dotsc, X_n$ mit $X_i \le x$).\\
    Dann heißt die Schätzung des Funktionals $q := \int_\real f(x) dF(x)$
    mit einer bzgl. $F$ integrierbaren Funktion $f\colon \real \rightarrow \real$ durch
    $\widehat{q} := \int_\real f(x) dF_n(x) = \frac{1}{n} \sum_{i=1}^n f(X_i)$
    \begriff{Häufigkeitssubstitution}.
\end{Def}

\linie

\begin{Bsp}
    Im Fall $f(x) = x^2$ erhält man $q = \int_\real x^2 dF(x) = \EE(X_1^2)$ (zweites Moment).
    Ist $X_1, \dotsc, X_n$ eine i.i.d. Stichprobe mit Verteilungsfunktion $F$, so ist\\
    $\widehat{q} := \int_\real x^2 dF_n(x) = \frac{1}{n} \sum_{i=1}^n X_i^2$
    ein sinnvoller Schätzer für $\EE(X_1^2)$ (Gesetz der großen Zahlen).
\end{Bsp}

\pagebreak

\subsection{%
    Momentenmethode%
}

\begin{Bem}
    Die Momentenmethode ist ein Spezialfall der Häufigkeitssubstit. mit $f(x) = x^k$.
\end{Bem}

\begin{Def}{Momentenmethode}
    Seien $X_1, \dotsc, X_n$ i.i.d. Zufallsvariablen mit unbekannter Verteilung $P_\theta$.
    Dann heißt die Schätzung der $k$-ten Momente
    $m_k(\theta) = \EE_\theta(X_i^k) = \int_\real x^k dP_\theta$ der $X_i$ durch
    das \begriff{$k$-te Stichprobenmoment}
    $\widehat{m}_k := \frac{1}{n} \sum_{i=1}^n X_i^k = \int_\real x^k dF_n(x)$
    \begriff{Momentenmethode}.

    Allgemeiner:
    Ist $q(\theta)$ eine Funktion der ersten $r$ Momente, d.\,h.
    $q(\theta) = g(m_1(\theta), \dotsc, m_r(\theta))$ mit einer stetigen Funktion
    $g\colon \real^r \rightarrow \real$,
    so wird $q(\theta)$ nach der \begriff{Momentenmethode} durch\\
    $T(X) := g(\widehat{m}_1, \dotsc, \widehat{m}_r)$ geschätzt.
\end{Def}

\linie

\begin{Bsp}
    Seien $X_1, \dotsc, X_n \sim \N(\mu, \sigma^2)$ i.i.d. mit unbekanntem
    $\theta = (\mu, \sigma^2)$.\\
    Dann können $\mu = m_1$ und $\sigma^2 = m_2 - m_1^2$ durch
    $\widehat{\mu} = \widehat{m}_1 = \frac{1}{n} \sum_{i=1}^n X_i$ und\\
    $\widehat{\sigma}^2 = \widehat{m}_2 - \widehat{m}_1^2 =
    \frac{1}{n} \sum_{i=1}^n X_i^2 - \left(\frac{1}{n} \sum_{i=1}^n X_i\right)^2
    = \frac{1}{n} \sum_{i=1}^n (X_i - \overline{X})^2$ geschätzt werden.
\end{Bsp}

\begin{Bsp}
    Beim \begriff{Taxiproblem} gibt es $\theta$-viele Taxis, die mit den Nummern
    $1, \dotsc, \theta$ versehen sind, wobei $\theta$ unbekannt ist.
    Von einem festen Punkt aus werden die Nummern $X_1, \dotsc, X_n$ von $n$ vorbeifahrenden
    Taxis notiert (Ziehen mit Zurücklegen).
    Es gilt $X_1, \dotsc, X_n \sim \U(\{1, \dotsc, \theta\})$ i.i.d.,
    d.\,h. $\PP_\theta(X_i = r) = \frac{1}{\theta}$ für $r = 1, \dotsc, \theta$ und
    $i = 1, \dotsc, n$ und somit\\
    $m_1(\theta) = \EE_\theta(X_i) = \sum_{r=1}^\theta r \cdot \PP_\theta(X_i = r)
    = \frac{1}{\theta} \sum_{r=1}^\theta r = \frac{\theta + 1}{2}$.\\
    Also kann $\theta$ nach der Momentenmethode durch $\widehat{\theta} = 2\widehat{m}_1 - 1 =
    2\overline{X} - 1$ geschätzt werden.
    Dieser Schätzer liefert aber in bestimmten Situationen keine sinnvollen Ergebnisse, z.\,B.
    wenn $\max\{x_1, \dotsc, x_n\} > \widehat{\theta} = 2\overline{x} - 1$,
    dann gilt dennoch immer $\theta \ge \max\{x_1, \dotsc, x_n\}$.
\end{Bsp}

\section{%
    Methode der kleinsten Quadrate%
}

\begin{Def}{allgemeine Regression}
    Eine \begriff{allgemeine Regression} ist gegeben durch einen Parametervektor
    $\theta \in \Theta \subset \real^r$ und bekannte parametrische Funktionen
    $g_1, \dotsc, g_n\colon \Theta \rightarrow \real$.
    Das dazugehörige Modell lautet $Y_i = g_i(\theta) + \varepsilon_i$, $i = 1, \dotsc, n$.
    Für die Zufallsvariablen $\varepsilon_1, \dotsc, \varepsilon_n$ (\begriff{Beobachtungsfehler})
    gelte dabei für alle $i, j = 1, \dotsc, n$ mit $i \not= j$, dass
    \begin{enumerate}
        \item
        $\EE(\varepsilon_i) = 0$,

        \item
        $\Var(\varepsilon_i) = \sigma^2 > 0$ mit $\sigma^2$ unbekannt und

        \item
        $\Cov(\varepsilon_i, \varepsilon_j) = 0$.
    \end{enumerate}
\end{Def}

\begin{Bem}
    Man bezeichnet die Fehler $\varepsilon_i$ auch als \begriff{weißes Rauschen (white noise)}.
    Die letzte Bedingung heißt \begriff{Unkorreliertheit}.
    Stochastische Unabhängigkeit impliziert Unkorreliertheit (die Umkehrung gilt i.\,A. nicht).
    Die Bedingungen sind z.\,B. (aber nicht nur) erfüllt, wenn
    $\varepsilon_1, \dotsc, \varepsilon_n$ i.i.d. mit $\EE(\varepsilon_i) = 0$ und
    $\Var(\varepsilon_i) = \sigma^2 > 0$.
    Beispielsweise gilt dies für $\varepsilon_1, \dotsc, \varepsilon_n \sim \N(0, \sigma^2)$
    i.i.d., in diesem Fall kann man zeigen, dass der KQS ein MLS ist.
\end{Bem}

\linie

\begin{Def}{Kleinste-Quadrate-Schätzer}\\
    Sei $Q\colon \Theta \times \real^n \rightarrow \real_0^+$ definiert durch
    $Q(\theta, y) := \sum_{i=1}^n (y_i - g_i(\theta))^2$ für $y \in \real^n$.
    Gibt es eine messbare Funktion $\widehat{\theta}\colon \real^n \rightarrow \Theta$,
    sodass $Q(\widehat{\theta}(y), y) \le Q(\widetilde{\theta}, y)$ für alle
    $\widetilde{\theta} \in \Theta$ und $y \in \real^n$,
    so heißt $\widehat{\theta}(Y)$  \begriff{Kleinste-Quadrate-Schätzer (KQS)} für
    $g(\theta) = (g_1(\theta), \dotsc, g_n(\theta))$.
\end{Def}

\begin{Bem}
    $\widehat{\theta}$ ist wohldefiniert (d.\,h. eindeutig),
    wenn das Bild von $g = (g_1, \dotsc, g_n)$ in $\real^n$ abgeschlossen ist.
    Sind die $g_i$ nach $\theta_1, \dotsc, \theta_r$ dif"|ferenzierbar und $\Theta \subset \real^r$
    of"|fen, so muss $\widehat{\theta}$ notwendigerweise die sog.
    \begriff{Normalengleichungen} erfüllen:
    $\frac{\partial}{\partial \theta_j} Q(\theta, y)|_{\theta=\widehat{\theta}(y)} = 0$ für
    $j = 1, \dotsc, r$, d.\,h. für alle $j = 1, \dotsc, r$ gilt
    $\sum_{i=1}^n (y_i - g_i(\theta)) \cdot
    \left.\frac{\partial g_i(\theta)}{\partial \theta_j}\right|_{\theta=\widehat{\theta}(y)} = 0$.
\end{Bem}

\linie
\pagebreak

\begin{Bsp}
    Bei der \begriff{linearen Regression} gilt $g_i(\theta) = \theta$ mit
    $\theta \in \Theta \subset \real$ (also $r = 1$), d.\,h. das Messmodell lautet
    $Y_i = \theta + \varepsilon_i$, $i = 1, \dotsc, n$.
    Wegen $\frac{\partial g_i(\theta)}{\partial \theta} = 1$ lauten die Normalengleichungen
    $\sum_{i=1}^n (y_i - \theta) = 0$.
    Somit gilt $\widehat{y} = \frac{1}{n} \sum_{i=1}^n y_i = \overline{y}$.
    Die Methode der kleinsten Quadrate liefert also denselben Schätzer wie die Momentenmethode.
\end{Bsp}

\begin{Bsp}
    Angenommen, es liegen $n$ Beobachtungen $(x_1, y_1), \dotsc, (x_n, y_n)$ vor, wobei
    die $x_1, \dotsc, x_n$ deterministisch und bekannt seien
    (z.\,B. feste Parameter, an denen man eine physikalische Größe auswertet).
    Im einfachen, linearen Fall erhält man als statistisches Modell
    $Y_i = \theta_1 + \theta_2 x_i + \varepsilon_i = g_i(\theta) + \varepsilon_i$.
    Dabei heißen die $Y_i$ \begriff{Zielvariablen} und die $\varepsilon_i$ \begriff{Kovariablen}.
    Gesucht sind geeignete Schätzer $\widehat{\theta}_1, \widehat{\theta}_2$ für
    $\theta_1, \theta_2$.\\
    Wegen $\frac{\partial g_i(\theta)}{\partial \theta_1} = 1$ und
    $\frac{\partial g_i(\theta)}{\partial \theta_2} = x_i$ lauten die Normalengleichungen
    $\sum_{i=1}^n (y_i - \theta_1 - \theta_2 x_i) = 0$ und
    $\sum_{i=1}^n (y_i - \theta_1 - \theta_2 x_i) x_i = 0$.
    Wenn man dieses LGS löst, indem man die erste Gleichung schätzt mit
    $\widehat{\theta}_1 = \overline{y} - \widehat{\theta}_2 \overline{x}$ und in die zweite
    einsetzt, so erhält man
    $\widehat{\theta}_2(y) = \frac{\sum_{i=1}^n (x_i - \overline{x}) (y_i - \overline{y})}
    {\sum_{i=1}^n (x_i - \overline{x})^2}$.
\end{Bsp}


\section{%
    Maximum-Likelihood-Schätzung%
}

\begin{Def}{Likelihood-Funktion}
    Sei $\P = \{\PP_\theta \;|\; \theta \in \Theta\}$ ($\Theta \subset \real^k$) ein
    reguläres statistisches Modell.
    Dann heißt die Funktion $L\colon \Theta \times \real^n \rightarrow \real$ mit
    $L(\theta, x) := p(x, \theta)$ \begriff{Likelihood-Funktion} des Parameters $\theta \in \Theta$
    für die Beobachtung $x \in \real^n$.
\end{Def}

\begin{Bem}
    Die Maximum-Likelihood-Methode sucht jetzt denjenigen Schätzwert
    $\widehat{\theta} = \widehat{\theta}(x)$, unter welchem die Daten $x$ mit höchster
    Wahrscheinlichkeit oder W-Dichte erscheinen.
\end{Bem}

\begin{Def}{Maximum-Likelihood-Schätzer}
    Gibt es eine messbare Funktion $\widehat{\theta}\colon \real^n \rightarrow \Theta$ mit\\
    $L(\widehat{\theta}(x), x) = \max_{\theta \in \Theta} L(\theta, x)$ für alle $x \in \real^n$,
    dann heißt $\widehat{\theta}(X)$ \begriff{Maximum-Likelihood-Schätzer (MLS/MLE)}
    für $\theta$.
\end{Def}

\begin{Bem}
    Da der Logarithmus streng monoton wächst, liefert die in vielen Fällen einfachere
    \begriff{Log-Likelihood-Funktion} $\ell\colon \Theta \times \real^n \rightarrow \real$ mit
    $\ell(\theta, x) := \ln L(\theta, x)$ denselben ML-Schätzwert.\\
    Ist $L$ in $\Theta$ dif"|ferenzierbar, so sind Lösungen von
    $\frac{\partial}{\partial \theta_j} L(\theta, x) = 0$ bzw.
    $\frac{\partial}{\partial \theta_j} \ell(\theta, x) = 0$,\\
    $j = 1, \dotsc, k$, mögliche Kandidaten für den ML-Schätzwert.
    Ist zusätzlich $\Theta \subset \real^k$ of"|fen, so ist die Bedingung
    $\left.\frac{\partial}{\partial \theta_j}
    L(\theta, x)\right|_{\theta=\widehat{\theta}(x)} = 0$,
    $j = 1, \dotsc, k$, (\begriff{Likelihood-Gleichungen})
    notwendig für den ML-Schätzwert.
    Hinreichende Bedingungen können z.\,B. unter Verwendung von 2. Ableitungen oder
    Konkavitätsargumenten gefunden werden.\\
    Sind die Komponenten von $X = (X_1, \dotsc, X_n)$ stochastisch unabhängig mit Dichten
    $p_i(\cdot, \theta)$ von $X_i$, so gilt $\ell(\theta, x) =
    \ln\!\left(\prod_{i=1}^n p_i(x_i, \theta)\right) = \sum_{i=1}^n \ln p_i(x_i, \theta)$.
\end{Bem}

\linie

\begin{Bsp}
    Seien $X_1, \dotsc, X_n \sim \N(\mu, \sigma^2)$ i.i.d. mit $\sigma^2$ bekannt und
    $\theta = \mu$.
    Dann gilt $L(\theta, x) \propto
    \exp\!\left(-\sum_{i=1}^n \frac{(x_i - \theta)^2}{\sigma^2}\right)$, also
    $\ell(\theta, x) = -\sum_{i=1}^n \frac{(x_i - \theta)^2}{\sigma^2}$.
    Somit lautet die Log-Likelihood-Gleichung
    $\frac{1}{\sigma^2} \sum_{i=1}^n (x_i - \theta) = 0$,
    dies führt auf $\widehat{\theta}(x) = \frac{1}{n} \sum_{i=1}^n x_i = \overline{x}$.
    Wegen $\frac{\partial^2}{\partial \theta^2} \ell(\theta, x) = -\frac{n}{\sigma^2} < 0$
    ist $\widehat{\theta}(x) = \overline{x}$ ein globales Maximum der Likelihood-Funktion.
    Der MLS stimmt also mit dem KQS und dem Schätzer nach der Momentenmethode überein.
\end{Bsp}

\begin{Bsp}
    Beim Taxiproblem ist $X_1, \dotsc, X_n \sim \U(\{1, \dotsc, \theta\})$ i.i.d. mit
    $\theta \in \natural$ unbekannt.
    Es gilt $L(\theta, x) = \prod_{i=1}^n \frac{1}{\theta} \1_{\{1, \dotsc, \theta\}}(x_i) =
    \frac{1}{\theta^n} \1_{\{1, \dotsc, \theta\}^n}(x)$.
    Für $\theta < \max x_i$ ist also $L(\theta, x) = 0$ und für $\theta \ge \max x_i$ ist
    $L(\theta, x) = \frac{1}{\theta^n}$.
    Man erhält also den MLS $\widehat{\theta}(x) = \max_{i=1,\dotsc,n} x_i$.
\end{Bsp}

\begin{Bsp}
    Seien $X_1, \dotsc, X_n \sim \N(\mu, \sigma^2)$ i.i.d., wobei diesmal
    $\theta = (\mu, \sigma^2)$ unbekannt ist.
    Durch Nachrechnen lässt sich zeigen, dass
    $\widehat{\theta}(x) = \left(\overline{x},
    \frac{1}{n} \sum_{i=1}^n (x_i - \overline{x})^2\right)$ der ML-Schätzwert ist
    (\begriff{Mit"-telwert} und \begriff{unkorrigierte empirische Varianz}).
\end{Bsp}

\pagebreak
