\chapter{%
    Vergleich von Schätzern: Optimalitätstheorie%
}

\section{%
    Schätzkriterien%
}

\begin{Bem}
    Gegeben ist wieder ein reguläres statistisches Modell
    $\P = \{\PP_\theta \;|\; \theta \in \Theta\}$.
    Wie kann die Qualität eines Schätzers $T = T(X)$ für den Parameter $\theta$ beurteilt werden?\\
    Der Schätzfehler $E = |T(X) - q(\theta)|$ ist hierfür ungeeignet, da
    $E$ vom unbekannten Parameter $\theta$ und vom Zufall abhängt
    (d.\,h. $E$ kann erst nach Vorliegen der Stichprobe berechnet werden).
\end{Bem}

\begin{Def}{mittlerer quadratischer Fehler}
    Sei $T = T(X)$ ein Schätzer für $q(\theta) \in \real$.
    Der \begriff{mittlere quadratische Fehler (MQF)} von $T$ ist definiert durch
    $R(\theta, T) := \EE_\theta((T(X) - q(\theta))^2)$.
\end{Def}

\begin{Def}{Verzerrung/Bias}
    $b(\theta, T) = \EE_\theta(T(X)) - q(\theta)$ heißt \begriff{Verzerrung}
    oder \begriff{Bias} von $T$.
\end{Def}

\begin{Def}{unverzerrt/erwartungstreu}\\
    Gilt $b(\theta, T) = 0$ für alle $\theta \in \Theta$, so heißt
    $T$ \begriff{unverzerrt} oder \begriff{erwartungstreu}.
\end{Def}

\begin{Bem}
    Für den MQF gilt
    $R(\theta, T) = \EE_\theta((T(X) - \EE_\theta(T(X)) + \EE_\theta(T(X)) - q(\theta))^2)$\\
    $= \EE_\theta((T(X) - \EE_\theta(T(X)))^2) + \EE_\theta((\EE_\theta(T(X)) - q(\theta))^2)
    = \Var_\theta(T(X)) + b(\theta, T)^2$
    unabhängig vom Zufall, da $\EE_\theta((T(X) - \EE_\theta(T(X)))^2) = 0$.
    Diese Zerlegung heißt \begriff{Varianz-Bias-Zerlegung} des MQF.
\end{Bem}

\begin{Lemma}{Varianz-Bias-Zerlegung}
    Es gilt $R(\theta, T) = \Var_\theta(T(X)) + b(\theta, T)^2$.
\end{Lemma}

\linie

\begin{Bsp}
    Seien $X_1, \dotsc, X_n \sim \N(\mu, \sigma^2)$ i.i.d. mit
    $\theta = (\mu, \sigma^2) \in \real \times \real^+$ unbekannt.
    Weiter oben wurde erwähnt, dass
    $\widehat{\theta} = (\frac{1}{n} \sum_{i=1}^n X_i, \frac{1}{n} (X_i - \overline{X})^2)$
    der MLS für $\theta$ ist, wobei $\overline{X} \sim \N(\mu, \frac{\sigma^2}{n})$ gilt.\\
    Für $q(\theta) = \mu$ gilt daher
    $b(\theta, \overline{X}) = \EE_\theta(\overline{X}) - q(\theta) = 0$,
    d.\,h. $\overline{X}$ ist ein erwartungstreuer Schätzer für $\mu$.
    Für den MQF gilt $R(\theta, \overline{X}) = \Var_\theta(\overline{X}) + b(\theta, T)^2
    = \frac{\sigma^2}{n} \to 0$ ($n \to \infty$).\\
    Für $q(\theta) = \sigma^2$ ist $\widehat{\sigma}^2 = \sigma^2(X) :=
    \frac{1}{n} \sum_{i=1}^n (X_i - \overline{X})^2$ ein brauchbarer Schätzer für $\sigma^2$,
    da $\EE_\theta(\widehat{\sigma}^2) = \frac{n-1}{n} \EE_\theta(\frac{1}{n-1}
    \sum_{i=1}^n (X_i - \overline{X})^2) = \frac{n-1}{n} \sigma^2 \to \sigma^2$ ($n \to \infty$),
    d.\,h. $\widehat{\sigma}^2$ ist \begriff{asymptotisch unverzerrt}.
    Im Gegensatz zur Stichprobenvarianz $S^2(X) := \frac{n}{n-1} \sigma^2$ ist
    die empirische Varianz $\sigma^2(X)$ also kein erwartungstreuer Schätzer für $\sigma^2$.
    Mit $S := \frac{n\widehat{\sigma}^2}{\sigma^2} =
    \sum_{i=1}^n \left(\frac{X_i - \overline{X}}{\sigma}\right)^2 \sim \chi_{n-1}^2$ gilt
    $\EE(S) = n - 1$ und $\Var(S) = 2(n - 1)$, damit lässt sich der MQF berechnen als\\
    $R(\theta, \widehat{\sigma}^2) = \frac{\sigma^2}{n} \Var(S) +
    (\frac{\sigma^2}{n} \EE(S) - \sigma^2)^2 =
    \left(\frac{\sigma^2}{n}\right)^2 (2n-1) \to 0$ ($n \to \infty$).
\end{Bsp}

\begin{Bsp}
    Man kann zwei Mittelwertschätzer für $X_1, \dotsc, X_n \sim \N(\mu, \sigma^2)$ i.i.d.,
    $X = (X_1, \dotsc, X_n)$ und $\theta = (\mu, \sigma^2)$ unbekannt mittels MQF vergleichen.
    Dazu werden für $q(\theta) = \mu$
    die Schätzer $T_1(X) := \overline{X}$ und $T_2(X) := a \overline{X}$ für ein
    $a \in (0, 1)$ betrachtet.\\
    Für $T_1$ gilt $b(\mu, T_1) = 0$ und $R(\mu, T_1) = \frac{\sigma^2}{n}$ wie eben berechnet.\\
    Für $T_2$ gilt $b(\mu, T_2) = \EE_\theta(T_2(X)) - \mu = (a - 1)\mu$ und\\
    $R(\mu, T_2) = \Var_\theta(T_2(X)) + b(\mu, T_2)^2 = \frac{a^2 \sigma^2}{n} + (a-1)^2 \mu^2$.\\
    $T_2$ ist also nicht mehr erwartungstreu, aber für $\mu$ in einer hinreichend kleinen Umgebung
    von $0$ gilt $R(\mu, T_2) < R(\mu, T_1)$, d.\,h. für ein solches $\mu$ schätzt $T_2$ besser.
    Für $\mu$ hinreichend groß gilt natürlich $R(\mu, T_1) < R(\mu, T_2)$.
\end{Bsp}

\linie

\begin{Def}{unzulässig}
    Ein Schätzer $S = S(X)$ heißt \begriff{unzulässig}, falls es einen Schätzer $T = T(X)$ gibt
    mit $\forall_{\theta \in \Theta}\; R(\theta, T) \le R(\theta, S)$ und
    $\exists_{\theta \in \Theta}\; R(\theta, T) < R(\theta, S)$.
\end{Def}

\begin{Bem}
    Es kann keinen perfekten, "`besten"' Schätzer $T$ geben mit
    $\forall_{\theta \in \Theta}\; R(\theta, T) \le R(\theta, S)$ für jeden anderen Schätzer $S$.
    Wählt man nämlich ein festes $\theta_0 \in \Theta$ und setzt $S(X) := q(\theta_0)$, so gilt
    $R(\theta_0, S) = 0$.
    Für den "`besten"' Schätzer $T$ müsste also $\forall_{\theta \in \Theta}\; R(\theta, T) = 0$
    gelten, was of"|fensichtlich nicht geht.
    Daher müssen wir die Klasse der Vergleichskandidaten für einen guten Schätzer auf die Klasse
    der unverzerrten Schätzer einschränken ($S$ ist nicht unverzerrt).
\end{Bem}

\pagebreak

\section{%
    Unverzerrte Schätzer mit gleichmäßig minimaler Varianz%
}

\begin{Def}{UMVU-Schätzer}
    Ein unverzerrter Schätzer $T(X)$ für $q(\theta)$ heißt
    \begriff{UMVU-Schätzer\\
    (uniformly minimal variance unbiased, UMVUE)} für $q(\theta)$, falls
    für alle unverzerrten Schätzer $S(X)$ für $q(\theta)$ gilt, dass
    $\forall_{\theta \in \Theta}\; \Var_\theta(T(X)) \le \Var_\theta(S(X))$.
\end{Def}

\begin{Satz}{\name{Rao}-\name{Blackwell}}
    Seien $T(X)$ ein suf"|fizienter Schätzer für $\theta$ und $S(X)$ ein beliebiger Schätzer für
    $q(\theta)$ mit $\forall_{\theta \in \Theta}\; \EE_\theta(|S(X)|) < \infty$.
    Dann ist der Schätzer $T^\ast(X) := \EE(S(X)|T(X))$ für $q(\theta)$
    unabhängig von $\theta$ und es gilt
    $\forall_{\theta \in \Theta}\; R(\theta, T^\ast) \le R(\theta, S)$.
    Ist zusätzlich $\Var_\theta(S(X)) < \infty$, so gilt Gleichheit genau dann, wenn
    $\forall_{\theta \in \Theta}\; \PP_\theta(T^\ast(X) = S(X)) = 1$.
\end{Satz}

\begin{Bem}
    Der Schätzer $T^\ast(X)$ für $q(\theta)$ ist also mindestens so gut wie $S(X)$.\\
    $T^\ast(X)$ ist unabhängig von $\theta$, weil $T(X)$ ein suf"|fizienter Schätzer ist.
\end{Bem}

\linie

\begin{Bsp}
    Für $X_1, X_2 \sim \Bin(1, p)$ i.i.d. und $X := (X_1, X_2)$ ist $T(X) := X_1 + X_2$ eine
    suf"|fiziente Statistik für $p$.
    Wählt man $S(X) := X_1$, so gilt für $t = 0, 1, 2$, dass\\
    $\EE(S(X)|T(X)=t) = \EE(X_1|X_1+X_2 = t) =
    \frac{\PP(X_1 = 1, X_1 + X_2 = t)}{\PP(X_1 + X_2 = t)} =
    \frac{\PP(X_1 = 1, X_2 = t - 1)}{\PP(X_1 + X_2 = t)} =
    \frac{\PP(X_1 = 1) \cdot \PP(X_2 = t - 1)}{\PP(X_1 + X_2 = t)}$\\
    $= \frac{p \cdot p^{t-1} (1-p)^{1-(t-1)}}{\binom{2}{t} p^t (1 - p)^{2-t}} =
    \frac{1}{\binom{2}{t}} =
    \frac{t}{2}$.
    Somit ist $T^\ast(X) = \EE(S(X)|T(X)) = \frac{T(X)}{2} = \overline{X}$.\\
    Berechnet man den MQF von $T^\ast$, so erhält man\\
    $\EE_p((T^\ast(X) - p)^2) = \EE_p((\overline{X}-p)^2) = \Var_p(\overline{X}) =
    \frac{p(1-p)}{2}$.
    Dies ist echt kleiner als der MQF von $S$:\\
    $\EE_p((S(X) - p)^2) = \EE_p((X_1 - p)^2) = \Var_p(X_1) = p(1-p)$.
    Nach dem Satz von Rao-Blackwell muss dies auch so sein, denn $\Var_p(S(X)) = p(1-p) < \infty$
    und $\PP_p(\overline{X} = X_1) < 1$ für alle $p \in (0, 1)$.
\end{Bsp}

\linie

\begin{Bem}
    Für eine Eindeutigkeitsaussage benötigt man den Begriff der vollständigen Statistik.
    Er besagt, dass $\forall_{\theta \in \Theta}\;
    (\EE_\theta(g_1(T(X))) = \EE_\theta(g_2(T(X))) \;\Rightarrow\; g_1 = g_2)$.
    Äquivalent dazu ist folgende Definition.
    Vollständigkeit ist eigentlich eine Eigenschaft von
    $\P = \{\PP_\theta \;|\; \theta \in \Theta\}$ und verlangt eine gewisse Größe von $\Theta$,
    um die Implikation zu erzwingen.
\end{Bem}

\begin{Def}{vollständig}
    Eine Statistik $T(X)$ heißt \begriff{vollständig}, falls für jede messbare Abbildung
    $g$ gilt, dass aus $\forall_{\theta \in \Theta}\; \EE_\theta(g(T(X))) = 0$ folgt, dass
    $\forall_{\theta \in \Theta}\; \PP_\theta(g(T(X)) = 0) = 1$.
\end{Def}

\begin{Bsp}
    Seien $X_1, \dotsc, X_n \sim \Pois(\theta)$ i.i.d. mit $\theta \in \Theta := \real^+$.
    $T(X) = X_1 + \dotsb + X_n$ ist eine suf"|fiziente Statistik für $\theta$ mit
    $T(X) \sim \Pois(n\theta)$.
    Sei $g\colon \real \rightarrow \real$ messbar mit
    $\forall_{\theta \in \Theta}\; \EE_\theta(g(T(X))) = 0$.
    Dann gilt $\EE_\theta(g(T(X))) = \sum_{i=0}^\infty g(i) \cdot e^{-n\theta}
    \frac{(n\theta)!}{\i!}$.
    Dies ist eine Potenzreihe in $n\theta$, die in einer gewissen Umgebung von $0$ gleich
    null ist.
    Mit dem Eindeutigkeitssatz für Potenzreihen folgt, dass $g(i) = 0$ für alle $i \in \natural_0$,
    d.\,h. $g \equiv 0$ und $T(X)$ ist eine vollständige Statistik.
\end{Bsp}

\linie

\begin{Satz}{\name{Lehmann}-\name{Schef}\name{fé}}
    Seien $T(X)$ eine vollständige, suf"|fiziente Statistik für $\theta$ und
    $S(X)$ ein unverzerrter Schätzer für $q(\theta)$.\\
    Dann ist der Rao-Blackwell-Schätzer $T^\ast(X) := \EE(S(X)|T(X))$ ein UMVU-Schätzer für
    $q(\theta)$.\\
    Ist zusätzlich $\forall_{\theta \in \Theta}\; \Var_\theta(T^\ast(X)) < \infty$, so ist
    $T^\ast(X)$ der eindeutige UMVU-Schätzer für $q(\theta)$.
\end{Satz}

\begin{Bem}
    Ist $h(T(X))$ ein unverzerrter Schätzer für $q(\theta)$ und $T(X)$ eine vollständige,
    suf"|fiziente Statistik für $\theta$, so ist $h(T(X))$ ein UMVU-Schätzer für $q(\theta)$,
    da wegen\\
    $\EE(h(T(X))|T(X)) = h(T(X))$ der Schätzer $S(X) := h(T(X))$ gewählt werden kann.
\end{Bem}

\begin{Satz}{Konstruktion von vollständigen, suf{}fizienten Statistiken}
    Ist $\P = \{\PP_\theta \;|\; \theta \in \Theta\}$ eine $k$-parametrige Exponentialfamilie und
    enthält $c(\Theta) \subset \real^k$ mit $c = (c_1, \dotsc, c_k)$
    ein of"|fenes Rechteck in $\real^k$, so ist
    $T(X) := (T_1(X), \dotsc, T_k(X))$ eine vollständige, suf"|fiziente Statistik für
    $\theta \in \Theta \subset \real^k$.
\end{Satz}

\linie
\pagebreak

\begin{Bsp}
    Bei $X_1, \dotsc, X_n \sim \N(\mu, \sigma^2)$ i.i.d. mit $X = (X_1, \dotsc, X_n)$ und
    $\theta = (\mu, \sigma^2)$ unbekannt ist $T(X) := (\sum_{i=1}^n X_i, \sum_{i=1}^n X_i^2)$
    nach dem letzten Satz eine vollständige, suf"|fiziente Statistik für
    $\theta \in \Theta := \real \times \real^+$.
    $\overline{X} = \frac{1}{n} T_1(X) =: h(T(X))$ ist ein unverzerrter Schätzer für
    $\mu = q(\theta)$.
    Damit ist $\overline{X}$ nach dem Satz von Lehmann-Schef"|fé ein UMVU-Schätzer, der sogar
    eindeutig ist, da $\forall_{\theta \in \Theta}\; \Var_\theta(h(T(X))) =
    \Var_\theta(\overline{X}) = \frac{\sigma^2}{n} < \infty$.\\
    Da $S^2(X) = \frac{1}{n-1} \sum_{i=1}^n (X_i - \overline{X})^2$ ein erwartungstreuer Schätzer
    für $\sigma^2$ ist und in der Form
    $h(T(X)) := \frac{1}{n-1} \left(T_2(X) - \frac{1}{n} (T_1(X))^2\right)$ geschrieben werden
    kann, folgt analog, dass auch $S^2(X)$ ein (bzw. der eindeutige)
    UMVU-Schätzer für $\sigma^2 = q(\theta)$ ist.
\end{Bsp}

\section{%
    Die Informationsungleichung%
}

\begin{Bem}
    Im Folgenden sei $\P = \{\PP_\theta \;|\; \theta \in \Theta\}$ immer ein
    einparametriges, reguläres statistisches Modell, das folgende Regularitätsbedingungen
    (CR) erfülle.
\end{Bem}

\begin{Def}{\name{Cramér}-\name{Rao}-Regularitätsbedingungen (CR)}
    \begin{enumerate}
        \item
        $\Theta \subset \real$ sei of"|fen.

        \item
        $A := \{x \in \real^n \;|\; p(x, \theta) > 0\}$ sei unabhängig von $\theta$ und es gelte\\
        $\forall_{x \in A} \forall_{\theta \in \Theta}\; (\frac{\partial}{\partial \theta}
        \ln p(x, \theta) \text{ existiert und ist endlich})$.

        \item
        Hat $X$ eine L.-B.-Dichte und ist $T = T(X)$ eine Statistik mit
        $\forall_{\theta \in \Theta}\; \EE_\theta(|T(X)|) < \infty$, so gelte
        $\frac{\partial}{\partial \theta} \int_{\real^n} T(x) p(x, \theta) \dx =
        \int_{\real^n} T(x) \frac{\partial}{\partial \theta} p(x, \theta) \dx$.
    \end{enumerate}
\end{Def}

\begin{Bem}
    Ist $\P$ eine $1$-parametrige Exp.fam. mit
    $p(x, \theta) = \1_A(x) \exp(c(\theta)T(x) + d(\theta) + S(x))$, wobei
    $\forall_{\theta \in \Theta}\; \frac{\partial}{\partial \theta} c(\theta) \not= 0$,
    $\Theta \subset \real$ of"|fen und $c$ stetig ist, dann ist (CR) erfüllt.
\end{Bem}

\linie

\begin{Def}{\name{Fisher}-Information}
    Die \begriff{\name{Fisher}-Information} für einen Parameter $\theta$ ist gegeben durch\\
    $I(\theta) := \EE_\theta\!\left(\left(\frac{\partial}{\partial \theta} \ln p(X, \theta)
    \right)^2\right) \in [0, \infty]$.
    Die partielle Ableitung heißt \begriff{Score-Funktion}.
\end{Def}

\begin{Bem}
    Es gilt
    $I(\theta) = \int_{\real^n} \frac{1}{p(x, \theta)} \cdot \left(\frac{\partial}{\partial \theta}
    p(x, \theta)\right)^2 \dx$.\\
    Für die Score-Funktion gilt
    $\EE_\theta\!\left(\frac{\partial}{\partial \theta} \ln p(X, \theta)\right) = 0$.
    Damit ist also $I(\theta) =
    \Var_\theta\!\left(\frac{\partial}{\partial \theta} \ln p(X, \theta)\right)$.
\end{Bem}

\begin{Bem}
    Gilt $X = (X_1, \dotsc, X_n)$ mit $X_1, \dotsc, X_n$ i.i.d., dann gilt\\
    $I(\theta) = n \cdot \EE_\theta\!\left(\left(\frac{\partial}{\partial \theta}
    \ln p_1(X_1, \theta)\right)^2\right)$.
    Verdoppelt man also den Stichprobenumfang, so verdoppelt sich die Fisher-Information.
\end{Bem}

\begin{Bsp}
    Für $X \sim \N(\mu, \sigma^2)$ mit $\theta = \mu$ unbekannt und $\sigma^2$ bekannt gilt
    $I(\theta) = \frac{1}{\sigma^2}$.
\end{Bsp}

\linie

\begin{Satz}{Informationsungleichung von \name{Cramér}-\name{Rao}}\\
    Sei $T(X)$ eine Statistik mit $\forall_{\theta \in \Theta}\; \Var_\theta(T(X)) < \infty$.
    Außerdem sei (CR) erfüllt und es gelte $\forall_{\theta \in \Theta}\; 0 < I(\theta) < \infty$.
    Dann gilt für $\psi(\theta) := \EE_\theta(T(X))$, dass
    $\psi(\theta)$ für alle $\theta \in \Theta$ dif"|ferenzierbar ist und
    $\Var_\theta(T(X)) \ge \frac{(\psi'(\theta))^2}{I(\theta)}$.
\end{Satz}

\begin{Kor}
    Ist $T = T(X)$ ein unverzerrter Schätzer für $\theta$, so gilt unter den Voraussetzungen
    von eben die \begriff{\name{Cramér}-\name{Rao}-Schranke}
    $\Var_\theta(T(X)) \ge \frac{1}{I(\theta)}$ für alle $\theta \in \Theta$.
\end{Kor}

\begin{Kor}
    Sind $X_1, \dotsc, X_n$ i.i.d. mit $X = (X_1, \dotsc, X_n)$, so gilt unter den Voraussetzungen
    von eben $\Var_\theta(T(X)) \ge \frac{(\psi'(\theta))^2}{n \cdot I_1(\theta)}$ mit
    $I_1(\theta) = \EE_\theta\!\left(\left(\frac{\partial}{\partial \theta} \ln p_1(X_1, \theta)
    \right)^2\right)$ der Fisher-Information der einzelnen Beobachtung.
\end{Kor}

\begin{Bem}
    Dafür benötigt man die C.-S.-Ungleichung
    $|\Cov(X,Y)| \le (\Var(X))^{1/2} (\Var(Y))^{1/2}$.
    Man kann zeigen, dass die Cramér-Rao-Schranke nur für Exponentialfamilien erfüllt sein kann.
\end{Bem}

\pagebreak

\section{%
    Asymptotische Theorie%
}

\begin{Def}{(schwach) konsistent}
    Eine Folge von Schätzern $T_n := T_n(X_1, \dotsc, X_n)$ für $q(\theta)$ heißt\\
    \begriff{(schwach) konsistent}, falls
    $T_n \xrightarrow{\PP_\theta} q(\theta)$ für $n \to \infty$ und alle $\theta \in \Theta$.
\end{Def}

\begin{Def}{stark konsistent}
    Eine Folge von Schätzern $T_n := T_n(X_1, \dotsc, X_n)$ für $q(\theta)$ heißt\\
    \begriff{stark konsistent}, falls
    $T_n \xrightarrow{n \to \infty} q(\theta)$ $\PP_\theta$-f.s. für alle $\theta \in \Theta$.
\end{Def}

\begin{Bem}
    Aus starker Konsistenz folgt immer schwache Konsistenz.
    Umgekehrt kann man zeigen, dass ein schwach konsistenter Schätzer stark konsistent ist,
    wenn die stochastische Konvergenz schnell genug ist.\\
    UMVU-Schätzer sind immer konsistent und MLS sind in der Regel auch konsistent.
\end{Bem}

\linie

\begin{Satz}{Log-Likelihood-Funktion zum wahren Parameter besser}\\
    Sei $\P = \{p(\cdot, \theta) \;|\; \theta \in \Theta\}$
    ein reguläres statistisches Modell mit
    \begin{enumerate}
        \item
        $\forall_{\theta, \theta' \in \Theta}\; \EE_\theta(\ln p(X, \theta')) < \infty$ und

        \item
        $\forall_{\theta, \theta' \in \Theta,\; \theta \not= \theta'}\;
        \PP_\theta \not= \PP_{\theta'}$
        ($\P$ ist \begriff{identifizierbar}).
    \end{enumerate}
    Dann gilt $\forall_{\theta, \theta' \in \Theta,\; \theta \not= \theta'}\;
    \EE_\theta(\ln p(X, \theta)) > \EE_\theta(\ln p(X, \theta'))$,
    d.\,h. die Log-Likelihood-Funktion zum wahren Parameter $\theta$ ist im Mittel strikt besser
    als die Log-Likelihood-Funktion zu einem anderen Parameter $\theta'$.
\end{Satz}

\begin{Satz}{Konsistenz des MLS}\\
    Seien $\Theta \subset \real^k$ kompakt und $\P = \{p(\cdot, \theta) \;|\; \theta \in \Theta\}$
    ein reguläres statistisches Modell mit
    \begin{enumerate}
        \item
        $\forall_{\theta, \theta' \in \Theta}\; \EE_\theta(\ln p(X, \theta')) < \infty$ und

        \item
        $\forall_{\varepsilon > 0} \exists_{\delta > 0} \forall_{\theta, \theta' \in \Theta,\;
        \norm{\theta - \theta'} < \delta} \forall_{x \in \real^n}\;
        |\ln p(x, \theta) - \ln p(x, \theta')| < \varepsilon$\\
        ($\ln p(x, \cdot)\colon \Theta \rightarrow \real$ ist
        \begriff{gleichmäßig gleichgradig stetig}).
    \end{enumerate}
    Dann ist jeder MLS $\widehat{\theta}_n$ mit Likel.-Funktion
    $L(\theta, (X_1, \dotsc, X_n)) = \prod_{i=1}^n p(X_i, \theta)$
    stark konsistent.
\end{Satz}

\begin{Bem}
    Die Voraussetzungen $\Theta \subset \real^k$, $\Theta$ kompakt und
    gleichmäßige gleichgradige Stetigkeit der Log-Likelihood-Funktion werden für die
    starke Konsistenz des MLS nicht benötigt.
\end{Bem}

\linie

\begin{Bem}
    Ist ein MLS konsistent, so ist die Konvergenzgeschwindigkeit und die asymptotische Verteilung
    des MLS häufig von großem Interesse.
\end{Bem}

\begin{Satz}{asymptotische Normalität des MLS}\\
    Sei $\P = \{p(\cdot, \theta) \;|\; \theta \in \Theta\}$
    ein reguläres statistisches Modell mit $\Theta \subset \real$ of"|fen und
    \begin{enumerate}
        \item
        $B := \{x \in \real \;|\ p(x, \theta) > 0\}$ unabhängig von $\theta$,

        \item
        $\forall_{\theta, \theta' \in \Theta,\; \theta \not= \theta'}\;
        \PP_\theta \not= \PP_{\theta'}$
        ($\P$ ist \begriff{identifizierbar}),

        \item
        $\forall_{x \in \real}\; p(x, \cdot)\colon \Theta \rightarrow \real$ dreifach stetig
        dif"|ferenzierbar,

        \item
        $\frac{\partial^k}{\partial \theta^k} \int_B p(x, \theta) d\mu(x) =
        \int_B \frac{\partial^k}{\partial \theta^k} p(x, \theta) d\mu(x)$ für $k = 1, 2$
        (mit $\mu$ dem Zähl- oder L.-B.-Maß),

        \item
        $\forall_{\theta \in \Theta} \exists_{c_\theta > 0}
        \exists_{g_\theta\colon \real \rightarrow \real_0^+,\; \EE_\theta(g_\theta(X_1)) < \infty}
        \forall_{\theta^\ast \in \Theta,\; |\theta^\ast - \theta| < c_\theta}
        \forall_{x \in \real}\;
        \left|\frac{\partial^3}{\partial \theta^3}
        \left.\ln p(x, \theta)\right|_{\theta = \theta^\ast}\right| \le g_\theta(x)$,

        \item
        $\forall_{\theta \in \Theta}\; I(\theta) =
        \Var_\theta\!\left(\frac{\partial}{\partial \theta} \ln p(X, \theta)\right) \in
        (0, \infty)$ und

        \item
        $\forall_{\theta \in \Theta}\; \widehat{\theta}_n \text{ schwach konsistenter MLS für }
        \theta$.
    \end{enumerate}
    Dann gilt $\sqrt{n I(\theta)} (\widehat{\theta}_n - \theta)
    \xrightarrow{\text{(d)}} Z$ mit $Z \sim \N(0, 1)$ für alle $\theta \in \Theta$.
\end{Satz}

\pagebreak
