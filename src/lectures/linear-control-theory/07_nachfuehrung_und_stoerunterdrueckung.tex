\chapter{%
    Nachführung und Störunterdrückung%
}

\section{%
    Einführung: Nachführung eines Wasserkochers%
}

Bisher wurde versucht, den Zustands eines Systems zu einem Gleichgewicht (asymptotisch) zu steuern,
sogar in optimaler Weise bzgl. eines quadratischen Kostenkriteriums.
In der Praxis ist es aber oft notwendig, den Ausgang eines Systems zu einem bestimmten Punkt zu
steuern, der vom Benutzer vorgegeben wird.
Einige Beispiele sind:
\begin{itemize}
    \item
    \emph{Pilot}:
    Wie muss der Joystick nach vorne bewegt werden, damit sich das Flugzeug nach vorne um
    einen vorgegebenen Winkel neigt?

    \item
    \emph{Autofahrer}:
    Wie muss das Gaspedal nach unten gedrückt werden, damit das Auto mit einer vorgegebenen
    Geschwindigkeit fährt?

    \item
    \emph{Industrie}:
    Wie muss ein Schweißroboter gesteuert werden, damit er einem bestimmten Pfad auf einer
    Karosserie folgt?
\end{itemize}

\textbf{Nachführung}:
Gestalte einen Regler,
sodass ein vorgegebener Systemausgang $z$ einem \begriff{externen Referenzsignal} $r$
so gut wie möglich folgt.
Diese Aufgabe bezeichnet man als \begriff{Nachführung (tracking)}.

\linie

\textbf{klassischer Aufbau bei der Nachführung}:
Sei $G$ ein System mit Eingang $u$ und Ausgang $z$.
$z$ soll nun dem externen Referenzsignal $r$ folgen.
Berechne zunächst das \begriff{Fehlersignal (error signal)} $e = r - z$.
Wenn $e = 0$ ist, dann ist kein Handeln erforderlich.
Andernfalls erzeuge ein Steuersignal $u$, um $e$ wieder auf Null zu bringen.
Dies geschieht mit einem \begriff{Fehlerrückführungs\-Regler
(error feedback controller)} $u = Ke$.
Das Fehlersignal $e$ ist die einzige Information, die dem Regler zur Verfügung steht.

\linie

\textbf{Beispiel}:
Ein Wasserkocher kann durch $C\dot{T} + \frac{1}{R} (T - S) = \alpha u$ mit
der angelegten Spannung $u$,
der Wassertemperatur $T$,
der Umgebungstemperatur $S$,
der Wärmekapazität $C > 0$ von Wasser und
dem Wärmewiderstand $R > 0$ der Wand modelliert werden.
Für eine Nenntemperatur $S_0$ von $S$ (wie \SI{20}{\celsius}) sei $x = T - S_0$ und $d = S - S_0$.
Dies führt zum System $\dot{x} = -ax + bu + b_d d$ mit Konstanten $a, b, b_d > 0$.
Weil $x$ geregelt werden soll, führt man den Ausgang $z = cx$ mit $c = 1$ ein.
In diesen Bezeichnungen ist
$u$ der Steuereingang, der durch den Regler manipuliert wird,
$d$ ein Störeingang, der durch den Regler nicht beeinflusst werden kann, und
$z$ das zu regelnde Signal.

\textbf{konstante Verstärkung}:
Sei zunächst $d \equiv 0$ (die Umgebungstemperatur also konstant gleich $S_0$).
Das System lautet in diesem Fall $\dot{x} = -ax + bu$, $z = cx$.
Der einfachste Fehlerrückführungs-Regler ist $u = k(r - z)$
(\begriff{konstante Verstärkung (constant gain)}).\\
Man erhält $\dot{x} = -(a + bkc)x + bkr$, $z = cx$ als das geregelte System.
$k$ stabilisiert das System genau dann, wenn $k > -\frac{a}{bc}$.
Wenn $r \equiv r_0$ ein konstanter Referenzwert ist, dann konvergiert der Ausgang zu
$z_0 = \frac{cbk}{a + bkc} r_0$.
Es gilt $z_0 = r_0$ genau dann, wenn $\frac{cbk}{a + bkc} = 1$, was aber für kein $k$ möglich ist!

\pagebreak

\section{%
    Verschiedene Möglichkeiten zur Nachführung%
}

\textbf{Rückführung mit hoher Verstärkung}:
Wenn $k$ groß gemacht wird, dann wird der stationäre Fehler
$|r_0 - z_0| = \left|\frac{a}{a + bkc}\right| |r_0|$ beliebig klein
(die Stabilität wird dadurch nicht zerstört).
Man nennt dies auch \begriff{Rückführung mit hoher Verstärkung (high-gain feedback)}.
$z$ erreicht zwar $r_0$ nicht (auch nicht asymptotisch), allerdings gelten folgende Vorteile:
\begin{itemize}
    \item
    Stabilisierung und approximative Nachführung

    \item
    Robustheit dieser beiden Eigenschaften, da im Wesentlichen unabhängig von den Werten von
    $a, b, c$
\end{itemize}
Man erhält aber auch folgende Nachteile:
\begin{itemize}
    \item
    große Fehler führen zu großem Regelaufwand

    \item
    kleine Nachführungsfehler erfordern eine hohe Verstärkung

    \item
    hohe Verstärkungen destabilisieren oft kompliziertere Systeme
\end{itemize}

\linie

\textbf{PI-Regler}:
Ein \begriff{PI-Regler (proportional integral controller)}
mit den Verstärkungen $k_p$ und $k_i$ ist gegeben durch
$u(t) = k_p e(t) + k_i \int_0^t e(\tau)\d\tau$ mit $e(t) = r(t) - z(t)$.

Eine Zustandsraum-Darstellung ist $\dot{x}_K = 0x_K + 1(r - z)$,
$u = k_i x_K + k_p(r - z)$, $x_K(0) = 0$.
Das geregelte System kann beschrieben werden durch
$\smallpmatrix{\dot{x} \\ \dot{x}_K} = \smallpmatrix{-a - bk_pc & bk_i \\ -c & 0}
\smallpmatrix{x \\ x_K} + \smallpmatrix{b k_p \\ 1} r$,
$z = \smallpmatrix{c & 0} \smallpmatrix{x \\ x_K}$.

Wenn $k_p$ und $k_i$ das geregelte System asymptotisch stabil machen, dann erfüllt der
PI-Regler exakte Nachführung für konstante Referenzwerte $r_0$:
Die stationäre Antwort ist in diesem Fall nämlich
$z_0 = -\smallpmatrix{c & 0} \smallpmatrix{-a - bk_pc & bk_i \\ -c & 0}^{-1}
\smallpmatrix{b k_p \\ 1} r_0 = r_0$.

\linie

\textbf{Regler mit zwei Freiheitsgraden ($2$-DOF-Regler)}:
Wenn $z$ und $r$ beide dem Regler zur Verfügung stehen (kann in der Praxis auch anders sein),
dann ist ein \begriff{Regler mit zwei Freiheitsgraden ($2$-DOF controller)} mit einer
\begriff{statischen Rückführungsverstärkung (static feed"-back gain)} $k_p$ und einer
\begriff{statischen Vorsteuerungsverstärkung (static feedforward gain)} $k_f$ definiert durch
$u = k_p (r - z) + k_f r$.

Die Vorgehensweise bei der Gestaltung eines solchen Reglers ist:
\begin{enumerate}
    \item
    Wähle $k_p$, sodass $u = k_p (r - z)$ das System stabilisiert.

    \item
    Passe $k_f$ so an, dass die stationäre Verstärkung des Systems gleich $1$ ist.
\end{enumerate}
Allerdings ist unklar, unter welchen Bedingungen an das System diese Strategie anwendbar ist und
wie sie zu multivariaten System höherer Ordnung verallgemeinert werden kann.

Man erhält das geregelte System $\dot{x} = -(a + bk_p c)x + b (k_p + k_f) r$, $z = cx$.
Zunächst wählt man $k_p$ mit $a + bk_p c > 0$.
Man erhält nun für die stationäre Antwort $z_0 = \frac{cb (k_p + k_f)}{a + bk_p c} r_0$,
wenn $r_0$ ein konstanter Referenzwert ist.
Wegen $cb \not= 0$ kann man $k_f$ so wählen, sodass der Bruch gleich $1$ ist.

Die Wahl von $k_f$ hängt stark von den Systemparametern ab.
Wenn der Regler auf einem System mit anderen Werten von $a, b, c$ implementiert wird, dann wird
$k_f$ höchstwahrscheinlich nicht die richtige Vorsteuerungsverstärkung sein, um Nachführung zu
erzielen.

\pagebreak

\section{%
    Das Nachführungsproblem%
}

\textbf{System beim Nachführungsproblem}:\\
Sei ein System $\dot{x} = Ax + Bu$, $y = Cx + Du$, $z = C_z x + D_z u$ gegeben.
Dabei ist $y$ ein Messausgang, der zur Regelung zur Verfügung steht, und $z$ ein zu regelnder
Ausgang, der der Referenz nachgeführt werden soll
($z$ kann auch gleich $y$ sein, muss aber nicht).

\textbf{Nachführungsproblem}:
Die Aufgabe ist es, einen Regler zu gestalten, der das System stabilisiert und für den
$z$ asymptotisch alle konstanten Referenzsignale $r$ nachführt, d.\,h.\\
$\lim_{t \to \infty} [r - z(t)] = 0$
(für alle Anfangsbedingungen).
Dieses Problem heißt \begriff{Nachführungspro"-blem (tracking problem)}.

Um das System stabilisieren zu können, wird von nun an angenommen, dass $(A, B)$ stabilisierbar
und $(A, C)$ entdeckbar ist.

Zunächst muss die Struktur des Reglers festgelegt werden, mit dem das Ziel erreicht werden soll.
Dabei wird ein Regler mit zwei Freiheitsgraden gewählt, d.\,h.
der Regler kennt sowohl das nachzuführende Signal $r$ als auch den Ausgang $y$ des Systems.

\linie

\textbf{Regler mit vollständiger Information}:\\
Zunächst wird angenommen, dass $y = x$ ist (d.\,h. $C = I$ und $D = 0$).
Ein linearer, statischer \begriff{Regler mit vollständiger Information
(full-information controller)} ist gegeben durch\\
$u = -Fx + Gr = \smallpmatrix{-F & G} \smallpmatrix{x \\ r}$,
wobei die Verstärkungsmatrizen $F$ und $G$ noch gewählt werden müssen.
Der geschlossene Regelkreis ist
$\dot{x} = (A - BF)x + BGr$, $z = (C_z - D_z F)x + D_z Gr$.
Weil das System stabilisiert werden soll, wird $F$ so gewählt, dass $A - BF$ eine Hurwitz-Matrix
ist.
Dann ergibt sich die stationäre Antwort für konstantes $r$ durch\\
$z = [D_z - (C_z - D_z F)(A - BF)^{-1} B] Gr$.
Für asymptotische Nachführung muss $z = r$ für alle möglichen konstanten Referenzeingänge $r$
gelten, d.\,h. man wählt $G$, sodass\\
$[D_z - (C_z - D_z F)(A - BF)^{-1} B] G = I$.

\linie

\textbf{Nachführung durch Ausgangsrückführung}:
Wenn $x$ nicht messbar ist, also $y \not= x$ gilt, dann wählt man $L$, sodass $A - LC$ eine
Hurwitz-Matrix ist, und verwendet den Beobachter
$\dotwidehat{x} = A\widehat{x} + Bu + L(y - \widehat{y})$, $\widehat{y} = C\widehat{y} + Du$,
um den Zustand asymptotisch rekonstruieren zu können.
Das Separationsprinzip motiviert dann die Regelung des Systems durch
$u = -F\widehat{x} + Gr$, wobei $F$ und $G$ die Verstärkungsmatrizen vom Regler mit
vollständiger Information sind.

\textbf{Satz (Nachführung durch das Separationsprinzip)}:
Der so konstruierte Ausgangsrückfüh"-rungs-Regler stabilisiert das System und
erzielt asymptotische Nachführung.

% Tatsächlich stabilisiert der so konstruierte Ausgangsrückführungs-Regler das System und
% erzielt Nachführung:
% Wenn man den Fehler $\widetilde{x} = x - \widehat{x}$ definiert, so gilt
% $\dotwidetilde{x} = (A - LC) \widetilde{x}$ unabhängig von $r$, d.\,h.
% der geschlossene Regelkreis ist $\smallpmatrix{\dot{x} \\ \dotwidetilde{x}} =
% \smallpmatrix{A - BF & BF \\ 0 & A - LC} \smallpmatrix{x \\ \widetilde{x}} +
% \smallpmatrix{BG \\ 0} r$,\\
% $z = \smallpmatrix{C_z - D_z F & D_z F} \smallpmatrix{x \\ \widetilde{x}} + D_z Gr$.
% Dieses System ist stabil und die stationäre Verstärkung ist für konstante $r$ gleich
% $D_z G - (C_z - D_z F) (A - BF)^{-1} (BG) = I$.

\linie

\textbf{Existenz von $G$}:
Allerdings kann man diesen Regler nur konstruieren, falls ein $G$ existiert, sodass
$[D_z - (C_z - D_z F)(A - BF)^{-1} B] G = I$.
Diese Gleichung ist äquivalent zu\\
$(C_z - D_z F) \Pi + D_z G = I$ mit $\Pi := -(A - BF)^{-1} BG$.
Diese beiden Gleichungen sind wiederum äquivalent zu
$(A - BF)\Pi + BG = 0$ und $(C_z - D_z F) \Pi + D_z G = I$.
Durch Umordnung erhält man
$A\Pi + B(G - F\Pi) = 0$ und $C_z \Pi + D_z (G - F\Pi) - I = 0$.
Mit $\Gamma := G - F\Pi$ erhält man die sog. \begriff{Regulatorgleichung (regulator equation)}
$\smallpmatrix{A & B \\ C_z & D_z} \smallpmatrix{\Pi \\ \Gamma} + \smallpmatrix{0 \\ -I} = 0$.

Wenn ein $G$ existiert, das obige Gleichung $[\dotsb]G = I$ erfüllt,
dann gibt es eine Lösung $(\Pi, \Gamma)$ der Regulatorgleichung.
Wenn umgekehrt $(\Pi, \Gamma)$ eine Lösung der Regulatorgleichung und $F$ die Matrix
$A - BF$ zu einer Hurwitz-Matrix macht, dann setze $G := \Gamma + F\Pi$.
Durch Umkehrung der obigen Argumente erkennt man, dass $G$ die Gleichung $[\dotsb]G = I$ erfüllt.
Daher gibt es genau dann obigen Nachführungsregler, wenn die Regulatorgleichung erfüllt ist.

\pagebreak

\section{%
    Das Regulationsproblem%
}

\textbf{verallgemeinerte Anlage}:\\
Das System $\dot{x} = Ax + Bu + B_d d$, $y = Cx + Du + D_d d$, $e = C_e x + D_e u + D_{ed} d$
heißt \begriff{verallgemeinerte Anlage (generalized plant)} mit
\begin{itemize}
    \item
    $u$ dem \begriff{Steuereingang (control input)},

    \item
    $d$ der \begriff{verallgemeinerten Störung (generalized disturbance)},

    \item
    $y$ dem \begriff{Messausgang (measurement output)} und

    \item
    $e$ dem \begriff{Leistungsausgang (performance output)}.
\end{itemize}

\textbf{Regulationsproblem}:\\
Die Aufgabe ist es, einen Regler $\dot{x}_K = A_K x_K + B_K y$, $u = C_K x_K$
zu gestalten, der
\begin{itemize}
    \item
    das System stabilisiert\\
    (d.\,h. $\lim_{t \to \infty} x(t) = 0$ für $d = 0$ und alle $x(0)$) und

    \item
    \begriff{Regulation}
    für alle konstanten verallgemeinerten Störungen $d$ erzielt\\
    (d.\,h. $\lim_{t \to \infty} e(t) = 0$ für alle $d \equiv \const$ und $x(0)$).
\end{itemize}
Dieses Problem heißt \begriff{Regulationsproblem (regulation problem)}.
Man sagt auch, dass $d$ von $e$
\begriff{asymptotisch unterdrückt (asymptotically rejected)} wird.

\linie

\textbf{Spezialfall 1 (Nachführung)}:
Das Nachführungsproblem von oben für das System\\
$\dot{x} = Ax + Bu$, $\widetilde{y} = \widetilde{C}x + \widetilde{D}u$,
$z = C_z x + D_z u$ ist ein Spezialfall des Regulationsproblems für die verallgemeinerte Anlage
$\dot{x} = Ax + Bu + 0d$,
$y = \smallpmatrix{\widetilde{C} \\ 0} x + \smallpmatrix{\widetilde{D} \\ 0} u +
\smallpmatrix{0 \\ I} d$,
$e = C_z x + D_z u + (-I)d$.
Man interpretiert also $d$ als Referenzsignal und packt den Ausgang $\widetilde{y}$ des
ursprünglichen Systems und das Referenzsignal $r$ in den Messausgang $y$ hinein.
Damit ist dann $e = C_z x + D_z u - r$ der Nachführungsfehler, der gegen Null geht genau dann,
wenn das $r$ asymptotisch nachgeführt wird.

\linie

\textbf{Spezialfall 2 (Störunterdrückung)}:
$d$ heißt verallgemeinerte Störung, weil $d$
sowohl aus bekannten Komponenten (wie Referenzsignalen)
als auch aus echten, unbekannten Störungen (wie Messrauschen) bestehen kann.
Im letzten Fall bestimmen die Matrizen $B_d, D_d, D_{ed}$ jeweils, wie die Störung den Zustand,
den Messausgang und den Leistungsausgang beeinflussen.
Man spricht bei $B_d d$ von der \begriff{Prozessstörung (process disturbance)},
bei $D_d d$ vom \begriff{Messrauschen (measurement noise)} und
bei $D_{ed} d$ von der \begriff{Laststörung (load disturbance)}.
Wenn man diese Matrizen verschieden wählt, kann man verschiedene Situationen modellieren.
Zum Beispiel würde man $B_d = 0$, $D_{ed} = 0$ wählen, wenn man nur Messrauschen modellieren
wollte.
%(die Komponenten von $D_d \not= 0$ sind dann Gewichte, die angeben, wie jeder Eintrag von
%$y$ durch das Rauschen $y$ beeinflusst wird).

\linie

\textbf{Spezialfall 3 (Nachführung und Störunterdrückung)}:
Wenn man beim ersten Spezialfall
$\widetilde{y} = \widetilde{C} x + \widetilde{D} u + d_{\widetilde{y}}$ setzt, so erhält man
die verallgemeinerte Anlage\\
$\dot{x} = Ax + Bu$,
$y = \smallpmatrix{\widetilde{C} \\ 0} x + \smallpmatrix{\widetilde{D} \\ 0} u +
\smallpmatrix{I & 0 \\ 0 & I} \smallpmatrix{d_{\widetilde{y}} \\ r}$,
$e = C_z x + D_z u + \smallpmatrix{0 & -I} \smallpmatrix{d_{\widetilde{y}} \\ r}$.\\
Der Messausgang hat also die Komponenten $\widetilde{C} x + \widetilde{D} u + d_{\widetilde{y}}$
und $r$,
während die verallgemeinerte Störung die Komponenten $d_{\widetilde{y}}$ und $r$ hat.

\pagebreak

\section{%
    Lösungen des Regulationsproblems%
}

Obwohl das neue Modell substanziell allgemeiner ist, ist der vorherige Lösungsansatz ohne viel
Aufwand ebenfalls anwendbar.

\textbf{Satz (nominale Regulation bei vollständiger Information)}:\\
Seien $(A, B)$ stabilisierbar und die Regulatorgleichung (R0)
$\smallpmatrix{A & B \\ C_e & D_e} \smallpmatrix{\Pi \\ \Gamma} +
\smallpmatrix{B_d \\ D_{ed}} = 0$
lösbar.\\
Dann gibt es Matrizen $F$ und $G$, sodass der Regler mit vollständiger
Information gegeben durch $u = -Fx + Gd$ das Regulationsproblem löst.

Lösbarkeit der Regulatorgleichung ist auch notwendig, d.\,h. wenn sie unlösbar ist,
dann gibt es solche Matrizen $F$ und $G$ nicht.

$F$ und $G$ wählt man wie folgt:
\begin{enumerate}
    \item
    Wähle ein beliebiges $F$, sodass $A - BF$ eine Hurwitz-Matrix ist.

    \item
    Wenn $(\Pi, \Gamma)$ eine Lösung der Regulatorgleichung ist,
    dann setze $G := \Gamma + F\Pi$.
\end{enumerate}

\linie

\textbf{Satz (Lösbarkeit durch Ausgangsrückführung)}:
Seien $(A, B)$ stabilisierbar und $(A, C)$ entdeckbar.
Außerdem sei (R0) lösbar und es gelte (D): $\smallpmatrix{A & B_d \\ C & D_d}$ habe vollen
Spaltenrang.\\
Dann ist das Regulationsproblem durch Ausgangsrückführung lösbar.

Die Stabilisierbarkeit von $(A, B)$ und Entdeckbarkeit von $(A, C)$ sind notwendige Voraussetzungen
für die Konstruktion von stabilisierenden Reglern.
Man beachte, dass für vollständige Information und für Ausgangsrückführung dieselbe
Regulatorgleichung zu lösen ist.
Die Bedingung (D) impliziert, dass die Matrix $\smallpmatrix{A & B_d \\ C & D_d}$ höher wie
breit ist, d.\,h. also auch $D_d$.
Insbesondere muss $\dim(y) \ge \dim(d)$ gelten, was anschaulich klar ist:
Man braucht mindestens so viele Messungen, wie die Störung Komponenten besitzt.

Weil nur $y$ verfügbar ist, versucht man, das Separationsprinzip für den Entwurf eines
Ausgangs"-rückführungs-Reglers anzuwenden.
Dazu erstellt man einen Beobachter, der sowohl $x$ als auch $d$ aus $y$ asymptotisch rekonstruiert.
Man bemerkt, dass konstante Störungen $d$ durch die DGL $\dot{d} = 0$ beschrieben werden.\\
Wenn man $e$ weglässt, erhält man so das \begriff{erweiterte System (extended system)}\\
$\smallpmatrix{\dot{x} \\ \dot{d}} = A_E \smallpmatrix{x \\ d} + B_E u$,
$y = C_E \smallpmatrix{x \\ d} + Du$ mit
$A_E := \smallpmatrix{A & B_d \\ 0 & 0}$,
$B_E := \smallpmatrix{B \\ 0}$ und
$C_E := \smallpmatrix{C & D_d}$.\\
Der gesuchte Beobachter existiert, wenn $(A_E, C_E)$ entdeckbar ist.

\textbf{Lemma (Entdeckbarkeit von $(A_E, C_E)$)}:\\
$(A_E, C_E)$ ist entdeckbar genau dann, wenn
$(A, C)$ entdeckbar ist und (D) gilt.

\textbf{Lemma (asymptotisch verschwindende Störungen)}:
Seien $A$ eine Hurwitz-Matrix und\\
$\lim_{t \to \infty} v(t) = 0$.
Dann gilt für jede Lösung von $\dot{x} = Ax + v$, dass $\lim_{t \to \infty} x(t) = 0$.

\linie

\textbf{Lösung des Regulationsproblems}:\\
Wie eben motiviert, entwirft man einen Regulator wie folgt.
Man prüft zunächst, ob
\begin{itemize}
    \item
    $(A, B)$ stabilisierbar und
    $\left(\smallpmatrix{A & B_d \\ 0 & 0}, \smallpmatrix{C & D_d}\right)$ entdeckbar und

    \item
    die Regulatorgleichung (R0)
    $\smallpmatrix{A & B \\ C_e & D_e} \smallpmatrix{\Pi \\ \Gamma} +
    \smallpmatrix{B_d \\ D_{ed}} = 0$
    lösbar ist.
\end{itemize}
In diesem Fall wählt man
\begin{enumerate}
    \item
    $F$ und $L$, sodass $A - BF$ und $\smallpmatrix{A & B_d \\ 0 & 0} - L\smallpmatrix{C & D_d}$
    Hurwitz-Matrizen sind, und

    \item
    $G := \Gamma + F\Pi$ mit $(\Pi, \Gamma)$ einer Lösung der Regulatorgleichung (R0).
\end{enumerate}
Dann löst der folgende Regler das Regulationsproblem:\\
$\smallpmatrix{\dotwidehat{x} \\ \dotwidehat{d}}
= \smallpmatrix{A & B_d \\ 0 & 0} \smallpmatrix{\widehat{x} \\ \widehat{d}} +
\smallpmatrix{B \\ 0} u + L(y - \widehat{y})$,\quad
$u = \smallpmatrix{-F & G} \smallpmatrix{\widehat{x} \\ \widehat{d}}$,\quad
$\widehat{y} = \smallpmatrix{C & D_d} \smallpmatrix{\widehat{x} \\ \widehat{d}} + Du$.

\pagebreak

\section{%
    Signalmodelle und nicht-konstante Störungen%
}

\textbf{Signalmodell}:
Bisher wurden nur konstante verallgemeinerte Störungen betrachtet.
Solche Störungen sind genau die Lösungen der DGL $\dot{d} = 0$.
Man nennt solche Systeme \begriff{Signalmodelle (signal models)} oder
\begriff{Signalgeneratoren (signal generators)}.

Dies lässt sich relativ einfach verallgemeinern.
Zum Beispiel erhält man sinusförmige Signale der Frequenz $\omega$ durch
$\dot{d} = Sd$ mit
$S = \smallpmatrix{0 & \omega \\ -\omega & 0}$,
denn damit gilt $e^{St} =
\smallpmatrix{\cos(\omega t) & \sin(\omega t) \\ -\sin(\omega t) & \cos(\omega t)}$.\\
Noch allgemeiner kann man $S$ als Blockdiagonalmatrix
$S = \diag\left(\smallpmatrix{0 & 1 \\ 0 & 0}, \smallpmatrix{0 & \omega_1 \\ -\omega_1 & 0},
\dotsc, \smallpmatrix{0 & \omega_s \\ -\omega_s & 0}\right)$ wählen, um Konstanten,
\begriff{Rampen (ramps)}, d.\,h. Ursprungsgeraden, und sinusförmige Signale der Frequenzen
$\omega_1, \dotsc, \omega_s$ zu erzeugen.
Sinusförmige Signale sind deshalb wichtig, weil sich andere Signale mittels der Fourier-Synthese
durch sinusförmige Signale approximieren lassen.

\linie

\textbf{Lösung des Regulationsproblems für Störungen $d$ mit $\dot{d} = Sd$}:\\
In der Herleitung des Ausgangsrückführungs-Reglers, der das Regulationsproblem löst,
kann man $\dot{d} = 0$ konsequent durch $\dot{d} = Sd$ ersetzen.
Es reicht, nur $S$ mit $\Eig(S) \subset \complex^0 \cup \complex^+$ zu betrachten,
da alle asymptotischen verschwindenden Störungen wegen obigem Lemma sowieso keine
Auswirkungen haben.
Daher kann man stabilisierende Regler, die Regulation für alle Signale
$d(t) = e^{St} d_0$ mit beliebigem, unbekanntem $d_0 \in \real^{\dim(S)}$ erzielen,
unter folgenden Bedingungen erstellen:
\begin{itemize}
    \item
    \begriff{andauernde Störung (persistent disturbance)}:\\
    $S$ hat keine Eigenwerte in $\complex^-$.

    \item
    Existenz einer stabilisierender Zustandsrückführungs-Verstärkung:\\
    $(A, B)$ ist stabilisierbar.

    \item
    Existenz eines Zustands- und Störungsschätzers:\\
    $(A_E, C_E) := \left(\smallpmatrix{A & B_d \\ 0 & S}, \smallpmatrix{C & D_d}\right)$
    ist entdeckbar.

    \item
    Lösbarkeit der Regulatorgleichung (R):\\
    Es gibt $\Gamma$ und $\Pi$ mit
    $\smallpmatrix{A & B \\ C_e & D_e} \smallpmatrix{\Pi \\ \Gamma} -
    \smallpmatrix{\Pi \\ 0} S + \smallpmatrix{B_d \\ D_{ed}} = 0$.
\end{itemize}
In diesem Fall wählt man
\begin{enumerate}
    \item
    $F$ und $L$, sodass $A - BF$ und $\smallpmatrix{A & B_d \\ 0 & S} - L\smallpmatrix{C & D_d}$
    Hurwitz-Matrizen sind, und

    \item
    $G := \Gamma + F\Pi$ mit $(\Gamma, \Pi)$ einer Lösung der Regulatorgleichung (R).
\end{enumerate}
Der Regler
$\smallpmatrix{\dotwidehat{x} \\ \dotwidehat{d}}
= \smallpmatrix{A & B_d \\ 0 & S} \smallpmatrix{\widehat{x} \\ \widehat{d}} +
\smallpmatrix{B \\ 0} u + L(y - \widehat{y})$,\quad
$u = \smallpmatrix{-F & G} \smallpmatrix{\widehat{x} \\ \widehat{d}}$,\quad
$\widehat{y} = \smallpmatrix{C & D_d} \smallpmatrix{\widehat{x} \\ \widehat{d}} + Du$\\
stabilisiert dann das System und erfüllt Regulation ($\lim_{t \to \infty} e(t) = 0$) für alle
verallgemeinerten Störungen $d$, die $\dot{d} = Sd$ erfüllen.

Der Beweis geht analog wie beim Fall $S = 0$.

\pagebreak

\section{%
    Verallgemeinerte Eigenräume und unentdeckbarer Unterraum%
}

\textbf{verallgemeinerte Eigenräume}:\\
Sei $A \in \real^{n \times n}$.
Wenn man $\chi_A = \alpha \beta$ faktorisiert, sodass $\alpha$ und $\beta$ nur Nullstellen
in $\complex^-$ bzw. in $\complex^0 \cup \complex^+$ haben, dann
sind $\E_-(A) := N(\alpha(A))$ und $\E_{0+}(A) := N(\beta(A))$
die verallgemeinerten Eigenräume von $A$ bzgl. $\complex^-$ bzw. $\complex^0 \cup \complex^+$:
Gilt nämlich $\alpha(A) = (A - \lambda_1 I)^{n_1} \dotsm (A - \lambda_k I)^{n_k}$,
so gilt $\E_-(A) = N((A - \lambda_1 I)^{n_1}) \oplus \dotsb \oplus N((A - \lambda_k I)^{n_k})$
nach dem Lemma von Bézout.

Man kann zeigen: $\E_-(A)$ (bzw. $\E_{0+}(A)$)
ist der größte $A$-invariante Unterraum $V \subset \real^n$,
sodass $\Eig(A|_V) \subset \complex^-$ (bzw. $\Eig(A|_V) \subset \complex^0 \cup \complex^+$).

$\E_-(A)$ heißt auch \begriff{stabiler Unterraum (stable subspace)} von $A$.
Wenn $S_1$ eine Basismatrix von $\E_-(A)$ ist und mit $S_2$ zu einer invertierbaren Matrix
$S = \smallpmatrix{S_1 & S_2}$ ergänzt wird, dann gilt
$S^{-1} AS = \smallpmatrix{A_{11} & A_{12} \\ 0 & A_{22}}$ mit
$\Eig(A_{11}) \subset \complex^-$ und $\Eig(A_{22}) \subset \complex^0 \cup \complex^+$.
Eine orthogonale Ähnlichkeits"-transformation $S$ kann man ef"|fektiv mit der
reellen geordneten Schur-Zerlegung berechnen.

$\E_-(A)$ erlaubt auch eine dynamische Interpretation,
wenn man Trajektorien $\varphi(\cdot, \xi)$ des Systems $\dot{x} = Ax$, $x(0) = \xi$ betrachtet:
Es gilt $\E_-(A) = \{\xi \in \real^n \;|\;
\varphi(t, \xi) = e^{At}\xi \xrightarrow{t \to \infty} 0\}$.

\linie

\textbf{Blockdiagonalisierung}:
Seien nun $\Eig(A_{11}) \subset \complex^-$ und
$\Eig(A_{22}) \subset \complex^0 \cup \complex^+$.\\
Dann gilt $\E_-\!\left(\smallpmatrix{A_{11} & A_{12} \\ 0 & A_{22}}\right)
= R\!\left(\smallpmatrix{I \\ 0}\right)$.
Wegen $\Eig(A_{11}) \cap \Eig(A_{22}) = \emptyset$ existiert eine eindeutige Lösung $X$
der \begriff{\name{Sylvester}-Gleichung} $A_{11} X - X A_{22} + A_{12} = 0$.
Damit gilt dann\\
$\E_{0+}\!\left(\smallpmatrix{A_{11} & A_{12} \\ 0 & A_{22}}\right)
= R\!\left(\smallpmatrix{X \\ I}\right)
= \left\{\left. \smallpmatrix{x_1 \\ x_2} \in \real^n \;\right|\; x_1 - Xx_2 = 0\right\}$.
Außerdem überführt die Transformationsmatrix $S = \smallpmatrix{I & X \\ 0 & I}$
die Blockdreiecksmatrix in Blockdiagonalform:\\
$S^{-1} \smallpmatrix{A_{11} & A_{12} \\ 0 & A_{22}} S
= \smallpmatrix{A_{11} & A_{11} X - X A_{22} + A_{12} \\ 0 & A_{22}}
= \smallpmatrix{A_{11} & 0 \\ 0 & A_{22}}$.

\textbf{Ausgangsstabilität}:
Weiterhin ist $\Eig(A_{11}) \subset \complex^-$ und
$\Eig(A_{22}) \subset \complex^0 \cup \complex^+$.\\
Man betrachtet das System $\dot{x} = \smallpmatrix{A_{11} & A_{12} \\ 0 & A_{22}} x$,
$y = \smallpmatrix{C_1 & C_2} x$.
Wenn $X$ eine Lösung der Sylvester-Gleichung $A_{11} X - X A_{22} + A_{12} = 0$ ist,
dann kann man das System transformieren in\\
$\dot{z} = \smallpmatrix{A_{11} & 0 \\ 0 & A_{22}} z$,
$y = \smallpmatrix{C_1 & C_1 X + C_2} z$.
Für $z(0) = \smallpmatrix{\xi_1 \\ \xi_2}$ ist der Ausgang des Systems gleich\\
$y(t) = C_1e^{A_{11} t} \xi_1 + (C_1 X + C_2) e^{A_{22} t} \xi_2$.
Der erste Summand geht gegen $0$, da $\Eig(A_{11}) \subset \complex^-$.
Außerdem geht der zweite Summand gegen $0$ genau dann, wenn $C_1 X + C_2 = 0$ ist.\\
Daher gilt $\lim_{t \to \infty} y(t) = 0$ für alle Trajektorien genau dann, wenn $C_1 X + C_2 = 0$.

\textbf{Lemma (Ausgangsstabilität)}:
Alle Trajektorien des Systems $\dot{x} = Ax$, $y = Cx$ erfüllen\\
$\lim_{t \to \infty} y(t) = 0$ genau dann, wenn $\E_{0+}(A) \subset N(C)$.

\linie

\textbf{unentdeckbarer Unterraum}:
Der \begriff{unentdeckbare Unterraum (undetectable subspace)} von $(A, C)$ ist der größte
$A$-invariante Unterraum $V \subset \real^n$ in $N(C)$, sodass
$\Eig(A|_V) \subset \complex^0 \cup \complex^+$.

\textbf{Lemma (geometrische Charakterisierung des unentdeckbaren Unterraums)}:\\
Seien $\U$ der unbeobachtbare Unterraum von $(A, C)$ und $\E_{0+}$ der verallgemeinerte
Eigenraum von $A$ bzgl. $\complex^0 \cup \complex^+$.
Dann ist der unentdeckbare Unterraum von $(A, C)$ gleich $\U \cap \E_{0+}$.

Jedes System $\dot{x} = Ax + Bu$, $y = Cx + Du$ kann in Beobachtbarkeits-Normalform geschrieben
werden mit Matrizen $\left(\smallpmatrix{A_{11} & 0 \\ \widetilde{A}_{21} & \widetilde{A}_{22}},
\smallpmatrix{B_1 \\ \widetilde{B}_2}, \smallpmatrix{C_1 & 0}, D\right)$
und $(A_{11}, C_1)$ beobachtbar.
Mit obiger Methode kann $\widetilde{A}_{22}$ blockdiagonalisiert werden mit Blöcken
$A_{22}$ und $A_{33}$, wobei $\Eig(A_{22}) \subset \complex^-$ und
$\Eig(A_{33}) \subset \complex^0 \cup \complex^+$.
Daher kann man das System transformieren zu\\
$\smallpmatrix{\dot{z}_1 \\ \dot{z}_2 \\ \dot{z}_3} =
\smallpmatrix{A_{11} & 0 & 0 \\ A_{21} & A_{22} & 0 \\ A_{31} & 0 & A_{33}}
\smallpmatrix{z_1 \\ z_2 \\ z_3} + \smallpmatrix{B_1 \\ B_2 \\ B_3} u$,
$y = \smallpmatrix{C_1 & 0 & 0} \smallpmatrix{z_1 \\ z_2 \\ z_3} + Du$.

\textbf{Lemma (explizite Darstellung des unentdeckbaren Unterraums)}:\\
Für dieses System ist der unentdeckbare Unterraum gleich
$R\!\left(\smallpmatrix{0 \\ 0 \\ I}\right)$.

\section{%
    Notwendige Bedingungen%
}

Sei (P) die verallgemeinerte Anlage mit Signalgenerator\\
$\smallpmatrix{\dot{x} \\ e \\ y} = \smallpmatrix{A & B_d & B \\ C_e & D_{ed} & D_e \\ C & D_d & D}
\smallpmatrix{x \\ d \\ u}$, $\dot{d} = Sd$.
Die Anlage
$\smallpmatrix{\dot{x} \\ \dot{d} \\ e \\ y} =
\smallpmatrix{A & B_d & B \\ 0 & S & 0 \\ C_e & D_{ed} & D_e \\ C & D_d & D}
\smallpmatrix{x \\ d \\ u}$
ist die \begriff{erweiterte verallgemei"-nerte Anlage}.

Es wird angenommen, dass $(A, B)$ stabilisierbar und $(A, C)$ entdeckbar ist.
Außerdem soll $\Eig(S) \subset \complex^0 \cup \complex^+$ gelten, d.\,h. es werden nur
andauernde Störungen betrachtet.
Gesucht sind notwendige Bedingungen für die Lösbarkeit des Regulationsproblems mit einem
Regler (C)\\
$\smallpmatrix{\dot{x}_K \\ u} = \smallpmatrix{A_K & B_K \\ C_K & 0} \smallpmatrix{x_K \\ y}$.

Wenn die verallgemeinerte Anlage mit dem Regler verbunden wird,
erhält man das geregelte System
mit Signalgenerator (CL)\\
$\smallpmatrix{\dot{\xi} \\ e} = \smallpmatrix{\A & \B \\ \C & \D} \smallpmatrix{\xi \\ d}$,
$\dot{d} = Sd$
und das erweiterte geregelte System
$\smallpmatrix{\dot{\xi} \\ \dot{d} \\ e} = \smallpmatrix{\A & \B \\ 0 & S \\ \C & \D}
\smallpmatrix{\xi \\ d}$,\\
wobei $\xi = \smallpmatrix{x \\ x_K}$ und
$\A = \smallpmatrix{A & BC_K \\ B_K C & A_K + B_K D C_K}$,
$\B = \smallpmatrix{B_d \\ B_K D_d}$,
$\C = \smallpmatrix{C_e & D_e C_K}$,
$\D = D_{ed}$.

\textbf{Regulationsproblem (mit Signalgenerator)}:\\
Für die verallgemeinerte Anlage (P) ist ein Regler (C) gesucht, sodass die geregelte
verallgemeinerte Anlage (CL) die folgenden Eigenschaften hat:
\begin{itemize}
    \item
    Für $d(t) \equiv 0$ für $t \ge 0$ erfüllen alle Trajektorien von (CL)
    $\lim_{t \to \infty} x(t) = 0$ und\\
    $\lim_{t \to \infty} x_K(t) = 0$
    (äquivalent dazu: $\A$ ist eine Hurwitz-Matrix).

    \item
    Für alle Trajektorien von (CL) gilt $\lim_{t \to \infty} e(t) = 0$.
\end{itemize}

\linie

Sei $(A_E, C_E) := \left(\smallpmatrix{A & B_d \\ 0 & S}, \smallpmatrix{C & D_d}\right)$
nicht entdeckbar.
In diesem Fall gilt folgende notwendige Bedingung für die Lösbarkeit des Regulationsproblems.

\textbf{Lemma (notwendige Bedingung)}:
Wenn das Regulationsproblem lösbar ist, dann ist der unentdeckbare Unterraum von
$(A_E, C_E)$ in $N(\smallpmatrix{C_e & D_{ed}})$ enthalten.

Wenn diese notwendige Bedingung erfüllt ist, dann wird im Beweis gezeigt, dass es möglich ist,
den Signalgenerator oBdA so zu reduzieren, dass obige Entdeckbarkeits-Bedingung für das
resultierende erweiterte System erfüllt ist.
Daher kann man ohne Einschränkung annehmen, dass $(A_E, C_E)$ entdeckbar ist
(wenn der unentdeckbare Unterraum von
$(A_E, C_E)$ in $N(\smallpmatrix{C_e & D_{ed}})$ enthalten ist,
sonst ist das Problem ja ohnehin nicht lösbar).

\linie

\textbf{Satz (Hauptresultat)}:\\
Seien $(A, B)$ stabilisierbar und
$\left(\smallpmatrix{A & B_d \\ 0 & S}, \smallpmatrix{C & D_d}\right)$ entdeckbar,
wobei $\Eig(S) \subset \complex^0 \cup \complex^+$.\\
Dann ist das Regulatorproblem lösbar genau dann, wenn die Regulatorgleichung (R) lösbar ist.

Die Regulatorgleichung (R) ist gegeben durch
$\smallpmatrix{A & B \\ C_e & D_e} \smallpmatrix{\Pi \\ \Gamma} -
\smallpmatrix{\Pi \\ 0} S + \smallpmatrix{B_d \\ D_{ed}} = 0$.

\pagebreak

\section{%
    Prinzip des internen Modells%
}

Sei nun eine verallgemeinerte Anlage mit Signalgenerator mit $e = y$ gegeben, d.\,h.\\
$\smallpmatrix{\dot{x} \\ e \\ y} = \smallpmatrix{A & B_d & B \\ C & D_d & D \\ C & D_d & D}
\smallpmatrix{x \\ d \\ u}$, $\dot{d} = Sd$.

\textbf{Satz (Prinzip des internen Modells)}:\\
Seien $(A, B)$ stabilisierbar und
$\left(\smallpmatrix{A & B_d \\ 0 & S}, \smallpmatrix{C & D_d}\right)$ entdeckbar,
wobei $\Eig(S) \subset \complex^0 \cup \complex^+$.\\
Außerdem sei (C) ein Regler, der das Regulationsproblem löst.\\
Dann gibt es eine Zustandskoordina"-ten-Transformation von (C) in\\
$\dotwidetilde{x}_K = \smallpmatrix{S & \widetilde{C}_K \\ 0 & \widetilde{A}_K} \widetilde{x}_K
+ \smallpmatrix{\widetilde{D}_K \\ \widetilde{B}_K} y$,
$u = \smallpmatrix{\Gamma & \widehat{C}_K} \widetilde{x}_K$.

In diesem Sinne muss $A_K$ also das Modell $S$ der zu regulierenden Signale
notwendigerweise beinhalten
(\begriff{Prinzip des internen Modells (internal model principle)}).
Insbesondere muss $\dim(A_K) \ge \dim(S)$ gelten, eine untere Schranke für die erforderliche
Dimension jedes Regulators.

\linie

Den Regler aus dem obigen Satz kann man als Reihenschaltung von\\
$\smallpmatrix{\dot{\xi}_1 \\ u} = \smallpmatrix{S & I & 0 \\ \Gamma & 0 & I}
\smallpmatrix{\xi_1 \\ u_1 \\ u_2}$ und
$\smallpmatrix{\dot{\xi}_2 \\ u_1 \\ u_2} = \smallpmatrix{\widetilde{A}_K & \widetilde{B}_K \\
\widetilde{C}_K & \widetilde{D}_K \\ \widehat{C}_K & 0} \smallpmatrix{\xi_2 \\ y}$ schreiben
(das zweite System zuerst).
Wenn man nun den Regler (C) konstruieren will, so sind zwar $S$ und $\Gamma$ bekannt, nicht
allerdings die Matrizen $\widetilde{A}_K, \widetilde{B}_K, \widetilde{C}_K, \widetilde{D}_K$ und
$\widehat{C}_K$.
Man kann aber unter obigen Voraussetzungen sogar für verallgemeinerte Anlage vom Typ (P)
wie folgt vorgehen:
\begin{enumerate}
    \item
    Löse die Regulatorgleichung und erhalte $\Gamma$.

    \item
    Vorkompensiere die verallgemeinerte Anlage (P), indem
    $\smallpmatrix{\dot{\xi}_1 \\ u} = \smallpmatrix{S & I & 0 \\ \Gamma & 0 & I}
    \smallpmatrix{\xi_1 \\ u_1 \\ u_2}$
    der Anlage vorgeschaltet wird.

    \item
    Entwerfe einen stabilisierenden Regler für die vorkompensierte Anlage.
\end{enumerate}
Dann löst der resultierende Regler für (P) das Regulationsproblem.

\pagebreak
