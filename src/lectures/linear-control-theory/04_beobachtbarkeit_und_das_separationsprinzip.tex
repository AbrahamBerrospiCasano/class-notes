\chapter{%
    Beobachtbarkeit und das Separationsprinzip%
}

\section{%
    Beobachtbarkeit und Dualität%
}

\textbf{Rekonstruktion des Zustands}:
Bei linearen System $\dot{x} = Ax + Bu$ kann in der Praxis eigentlich nie davon ausgegangen werden,
dass alle Komponenten des Zustands messbar sind und zur Verfügung stehen.
Daher kennt man normalerweise nur den Ausgang $y = Cx + Du$
als dem Regler verfügbare Information.
Ist es möglich, nur aus dem Wissen von $u$ und $y$ den Zustand $x$ zu rekonstruieren?
Kann mit dem rekonstruierten Zustand ein Regler implementiert werden?

\textbf{beobachtbar}:
Das lineare System $\dot{x} = Ax + Bu$, $y = Cx + Du$ heißt \begriff{beobachtbar (observable)},
falls es für jedes $T > 0$ möglich ist, $x(t)$ für $t \in [0, T]$ aus
$u(t)$ und $y(t)$ für $t \in [0, T]$ zu rekonstruieren.

\linie

$y$ hat normalerweise viel weniger Komponenten als $x$.
Daher ist die Rekonstruktion direkt aus $y$ unmöglich.
Allerdings kann man $y$ ableiten, um $y(t), \dot{y}(t), \dotsc, y^{(n-1)}(t)$ zu erhalten
(zumindest in der Theorie, in der Praxis ist das kaum möglich).
Es gilt $Y(t) = W x(t) + \D U(t)$ mit\\
$Y(t) := \smallpmatrix{y(t) \\ \dot{y}(t) \\ \vdots \\ y^{(n-1)}(t)}$,
$W := \smallpmatrix{C \\ CA \\ \vdots \\ CA^{n-1}}$,
$\D := \smallpmatrix{D & & & 0 \\ CB & D & & \\ \vdots & \ddots & \ddots & \\
C A^{n-2} B & \cdots & CB & D}$ und
$U(t) := \smallpmatrix{u(t) \\ \dot{u}(t) \\ \vdots \\ u^{(n-1)}(t)}$.

\textbf{Beobachtbarkeitsmatrix}:
$W$ heißt \begriff{Beobachtbarkeitsmatrix (observability matrix)} des Systems
$\dot{x} = Ax + Bu$, $y = Cx + Du$
oder des Paars $(A, C)$.

\linie

Wenn $W$ vollen Spaltenrang hat, dann gibt es eine Matrix $W^+$ mit $W^+ W = I$,
d.\,h.\\
$W^+ Y(t) = x(t) + W^+ \D U(t) \iff x(t) = W^+ Y(t) - W^+ \D U(t)$,
also kann man $x(t)$ aus $Y(t)$ und $U(t)$ rekonstruieren.
Es gilt aber auch die Umkehrung.

\textbf{Satz (\name{Kalman}-Test zur Beobachtbarkeit)}:
Das lineare System $\dot{x} = Ax + Bu$, $y = Cx + Du$ ist beobachtbar genau dann,
wenn die Beobachtbarkeitsmatrix $W$ vollen Spaltenrang hat.

$W^T = \smallpmatrix{C^T & A^T C^T & \cdots & (A^T)^{n-1} C^T}$ ist die Kalman-Matrix
von $(A^T, C^T)$.
Daher gilt $(A, C)$ beobachtbar $\iff$ $W$ hat vollen Spaltenrang $\iff$ $W^T$ hat vollen
Zeilenrang $\iff$ $(A^T, C^T)$ ist regelbar.
Man kann also alle Sätze und Eigenschaften über die Regelbarkeit von $(A^T, C^T)$ auf
die Beobachtbarkeit von $(A, C)$ übertragen.

\textbf{Dualitätsprinzip}:
Das \begriff{Dualitätsprinzip (duality principle)}
der linearen Kontrolltheorie ist die Übersetzung von Fragen
der Beobachtbarkeit von $(A, B, C, D)$ in Fragen der Regelbarkeit von $(A^T, C^T, B^T, D^T)$
(oder umgekehrt).

\textbf{Satz (\name{Hautus}-Test zur Beobachtbarkeit)}:
$(A, C)$ ist beobachtbar genau dann, wenn für jeden Eigenvektor $e$ von $A$ gilt,
dass $Ce \not= 0$.
Äquivalent dazu ist, dass die Matrix $\smallpmatrix{A - \lambda I \\ C}$
vollen Spaltenrang für alle $\lambda \in \complex$ besitzt.

\pagebreak

\section{%
    Unbeobachtbarer Unterraum und Eigenwert%
}

Wenn $W$ nicht vollen Spaltenrang hat, dann ist $N(W) \not= \{0\}$.
Nicht-verschwindende Trajektorien in diesem Raum werden von $W$ "`verschluckt"'
und können daher im Ausgang nicht beobachtet werden.

\textbf{unbeobachtbarer Unterraum}:\\
$N(W)$ heißt \begriff{unbeobachtbarer Unterraum (unobservable subspace)} von $(A, C)$.

\textbf{unbeobachtbarer Eigenwert}:
Ein \begriff{unbeobachtbarer Eigenwert (unobservable mode)} von $(A, C)$ ist
$\lambda \in \complex$, sodass $\smallpmatrix{A - \lambda I \\ C}$ nicht vollen Spaltenrang hat.

\linie

\textbf{Satz (geometrische Charakterisierung von $N(W)$)}:
Der unbeobachtbare Unterraum von $(A, C)$ ist der größte $A$-invariante Unterraum,
der in $N(C)$ enthalten ist.

\textbf{Satz (Beobachtbarkeits-Normalform)}:
Es gibt eine Zustandskoordinaten-Transformation $z = Tx$ mit $T$ invertierbar,
die $\dot{x} = Ax + Bu$, $y = Cx + Du$ in
$\smallpmatrix{\dot{z}_1 \\ \dot{z}_2} = \smallpmatrix{A_{11} & 0 \\ A_{21} & A_{22}}
\smallpmatrix{z_1 \\ z_2} + \smallpmatrix{B_1 \\ B_2} u =: \widetilde{A}z + \widetilde{B}u$,
$y = \smallpmatrix{C_1 & 0} \smallpmatrix{z_1 \\ z_2} + Du =: \widetilde{C} z + Du$ transformiert,
wobei $(A_{11}, C_1)$ beobachtbar ist.
Diese Form heißt \begriff{Beobachtbarkeits-Normalform (observability normal form)}
oder \begriff{BNF}.

Ausgeschrieben bedeutet das $\dot{z}_1 = A_{11} z_1 + B_1 u$,
$\dot{z}_2 = A_{21} z_1 + A_{22} z_2 + B_2 u$, $y = C_1 z_1 + Du$,
d.\,h. $z_1$ und daher auch $y$ werden von $z_2$ nicht beeinflusst.
Beispielsweise lässt sich eine Veränderung der Anfangsbedingung $z_2(0)$ nicht in $y$ beobachten.

\textbf{Folgerung}:
Der unbeobachtbare Unterraum von $(\widetilde{A}, \widetilde{C})$ ist
$\{(0, z_2) \;|\; z_2 \in \real^{\dim(z_2)}\}$ und die unbeobachtbaren Eigenwerte sind
die Eigenwerte von $A_{22}$.

Es gibt viele Gründe für Unbeobachtbarkeit.
Wenn man z.\,B. zwei identische, beobachtbare Systeme $(A_S, B_S, C_S, D_S)$ durch
Parallelschaltung (mit unterschiedlichen Eingängen) verknüpft,
erhält man $\dot{x} = Ax + Bu$, $y = Cx + Du$ mit
$A := \smallpmatrix{A_S & 0 \\ 0 & A_S}$,
$B := \smallpmatrix{B_S & 0 \\ 0 & B_S}$,
$C := \smallpmatrix{C_S & C_S}$ und
$D := \smallpmatrix{D_S & D_S}$.
Die Beobachtbarkeitsmatrix von $(A, C)$ hat
zwei identische Blockspalten und kann daher keinen vollen Spaltenrang haben.
Wenn $A_S$ die Dimension $n$ hat, dann ist der unbeobachtbare Unterraum von $(A, C)$ gleich
$\left\{\left.\smallpmatrix{x \\ -x} \;\right| x \in \real^n\right\}$.

\linie

\textbf{Satz (beobachtbar-kanonische Form)}:
Wenn $\dot{x} = Ax + Bu$, $y = Cx + Du$ nur einen Ausgang hat (d.\,h. $y$ ist skalar)
und beobachtbar ist, dann gibt es einen Koordinatenwechsel $z = Tx$ mit $T$ invertierbar,
der das System in
$\dot{z} = \smallpmatrix{-\alpha_1 & 1 & & & 0 \\ -\alpha_2 & 0 & 1 & & \\
\vdots & & \ddots & \ddots & \\ -\alpha_{n-1} & & & 0 & 1 \\ -\alpha_n & & & & 0} z
+ \widetilde{B} u := \widetilde{A} z + \widetilde{B} u$,
%+ \smallpmatrix{b_1 \\ b_2 \\ \vdots \\ b_{n-1} \\ b_n} u := \widetilde{A} z + \widetilde{B} u$,
$y = \smallpmatrix{1 & 0 & 0 & \cdots & 0} z + Du = \widetilde{C} z + Du$ transformiert.\\
Diese Form heißt \begriff{beobachtbar-kanonische Form (observability canonical form)}
oder \begriff{BKF}.

\pagebreak

\section{%
    Beobachter und Entdeckbarkeit%
}

Die sofortige Rekonstruktion des Zustands ist praktisch nicht möglich, da Rauschen durch die
Dif"|ferentiation verstärkt wird.
Außerdem kann die Beobachtbarkeitsmatrix schlecht konditioniert sein.
Daher versucht man, den Zustand asymptotisch zu rekonstruieren.

\textbf{Beobachter}:
Ein \begriff{Beobachter (observer)} für das lineare System $\dot{x} = Ax + Bu$, $y = Cx + Du$
ist das dynamische System
$\dotwidehat{x} = A\widehat{x} + Bu + L(y - \widehat{y})$, $\widehat{y} = C\widehat{x} + Du$.

Ein Beobachter ist eine Kopie des Originalsystems mit einem Korrekturterm $L(y - \widehat{y})$,
der dazu dient, dass der geschätzte Zustand $\widehat{x}$ in Richtung $x$ geregelt wird
(für den Fall, dass sich der gemessene Ausgang $y$ vom geschätzten Ausgang $\widehat{y}$
unterscheidet).

\textbf{Bestimmung von $L$}:
Natürlich sollte der \begriff{Schätzfehler} $\widetilde{x} := x - \widehat{x}$ schnell gegen
Null konvergieren.
Für seine Dynamik gilt
$\dotwidetilde{x} = \dot{x} - \dotwidehat{x} =
Ax + Bu - A\widehat{x} - Bu - L(y - \widehat{y})$\\
$= A\widetilde{x} - L(Cx + Du - C\widehat{x} - Du)
= (A - LC) \widetilde{x}$ (\begriff{Fehlerdynamik}).
Daher sollte $L$ so gewählt werden, dass $A - LC$ eine Hurwitz-Matrix ist, damit
$\lim_{t \to \infty} \widetilde{x}(t) = 0$.
Die Konvergenzgeschwindigkeit und die Art der Antwort (z.\,B. das Überschwingen)
hängt von den Eigenwerten von $A - LC$ ab und von $e^{(A - LC)t}$.

\textbf{Satz (Polvorgabe für Beobachter)}:\\
Seien $(A, C)$ beobachtbar und $\alpha$ ein reelles, normiertes
Polynom vom Grad $n$.\\
Dann gibt es eine reelle Matrix $L$ mit $\chi_{A - LC} = \alpha$.

\linie

\textbf{entdeckbar}:
Das System $\dot{x} = Ax + Bu$, $y = Cx + Du$ (oder das Paar $(A, C$) heißt
\begriff{entdeckbar (detectable)},
falls es eine Matrix $L$ gibt, sodass $A - LC$ eine Hurwitz-Matrix ist.

\textbf{Satz (\name{Hautus}-Test zur Entdeckbarkeit)}:
$(A, C)$ ist entdeckbar genau dann, wenn
die unbeobachtbaren Eigenwerte alle negative Realteile besitzen.
Äquivalent dazu ist, dass $\smallpmatrix{A - \lambda I \\ C}$ vollen Spaltenrang für alle
$\lambda \in \complex$ mit $\Re(\lambda) \ge 0$ besitzt.

\textbf{Satz (Trajektorien-basierte Charakterisierung)}:
Das System $\dot{x} = Ax + Bu$, $y = Cx + Du$ ist entdeckbar genau dann, wenn
$u(t) \equiv 0$ und $y(t) \equiv 0$ für $t \ge 0$ implizieren, dass\\
$\lim_{t \to \infty} x(t) = 0$.

\begin{landscape}
\section{%
    \emph{Zusatz}: Zusammenfassung der Dualität%
}

\newcommand*{\tablelinespace}{2mm}
\small
\begin{tabular}{p{35mm}p{105mm}p{105mm}}
    \toprule
    &
    \textbf{Regelbarkeit} &
    \textbf{Beobachtbarkeit}\\

    \midrule
    \emph{Traj.-Definition} &
    für jedes $x_f \in \real^n$ existiert $u$ stetig, sodass $x(T) = x_f$\newline
    für $x(0) = 0$ und $T > 0$ fest &
    %von jedem $\xi \in \real^n$ für $t = 0$ kann in jedes $x_f \in \real^n$ für $t = T > 0$\newline
    %gesteuert werden &
    für jedes $T > 0$ ist die Rekonstruktion von $x(t)$ für $t \in [0, T]$\newline
    aus $u(t)$ und $y(t)$ für $t \in [0, T]$ möglich\\

    \addlinespace[\tablelinespace]
    \emph{Dualität} &
    \multicolumn{2}{c}{$(A^T, C^T)$ regelbar $\iff$ $(A, C)$ beobachtbar}\\

    \addlinespace[\tablelinespace]
    \emph{\name{Kalman}-Test} &
    $K = \smallpmatrix{B & AB & \cdots & A^{n-1}B}$ hat vollen Zeilenrang &
    $W = \smallpmatrix{C \\ CA \\ \vdots \\ CA^{n-1}}$ hat vollen Spaltenrang\\

    \addlinespace[\tablelinespace]
    \emph{Unterraum} &
    $R(K)$ regelbarer Unterraum &
    $N(W)$ unbeobachtbarer Unterraum\\

    \addlinespace[\tablelinespace]
    \emph{geom. Charakter.} &
    kleinster $A$-invarianter Unterraum, der $R(B)$ enthält &
    größter $A$-invarianter Unterraum, der in $N(C)$ enthalten ist\\

    \addlinespace[\tablelinespace]
    \emph{\name{Hautus}-Test} &
    $\smallpmatrix{A - \lambda I & B}$ hat vollen Zeilenrang für alle $\lambda \in \complex$ &
    $\smallpmatrix{A - \lambda I \\ C}$ hat vollen Spaltenrang für alle $\lambda \in \complex$\\

    \addlinespace[\tablelinespace]
    \emph{Eigenwerte} &
    $\lambda \in \complex$ mit Rangverlust unregelbare Eigenwerte &
    $\lambda \in \complex$ mit Rangverlust unbeobachtbare Eigenwerte\\

    \addlinespace[\tablelinespace]
    \emph{Polvorgabe} &
    für regelbare Systeme für $A - BF$ möglich &
    für beobachtbare Systeme für $A - LC$ möglich\\

    \addlinespace[\tablelinespace]
    \emph{kanonische Form\newline($m = 1$ bzw. $k = 1$)} &
    $\dot{z} = \smallpmatrix{-\alpha_1 & -\alpha_2 & \cdots & -\alpha_n \\
    1 & 0 & & & \\ & \ddots & \ddots & & \\ 0 & & 1 & 0} z +
    \smallpmatrix{1 \\ 0 \\ \vdots \\ 0} u$
    &
    $\dot{z} = \smallpmatrix{-\alpha_1 & 1 & & & 0 \\ -\alpha_2 & 0 & 1 & & \\
    \vdots & & \ddots & \ddots & \\ -\alpha_{n-1} & & & 0 & 1 \\ -\alpha_n & & & & 0} z
    + \widetilde{B} u$,
    $y = \smallpmatrix{1 & 0 & 0 & \cdots & 0} z + Du$\\

    \addlinespace[\tablelinespace]
    \emph{Normalform} &
    $\smallpmatrix{\widetilde{A} & \widetilde{B} \\ \widetilde{C} & \widetilde{D}}
    = \smallpmatrix{A_{11} & A_{12} & B_1 \\ 0 & A_{22} & 0 \\ C_1 & C_2 & D}$,
    $(A_{11}, B_1)$ regelbar &
    $\smallpmatrix{\widetilde{A} & \widetilde{B} \\ \widetilde{C} & \widetilde{D}}
    = \smallpmatrix{A_{11} & 0 & B_1 \\ A_{21} & A_{22} & B_2 \\ C_1 & 0 & D}$,
    $(A_{11}, C_1)$ beobachtbar\\

    \addlinespace[\tablelinespace]
    \midrule
    &
    \textbf{Stabilisierbarkeit} &
    \textbf{Entdeckbarkeit}\\

    \midrule
    \emph{Traj.-Definition} &
    für jedes $\xi \in \real^n$ existiert $u$ stückweise stetig, sodass\newline
    $\lim_{t \to \infty} x(t) = 0$ für $x(0) = \xi$ &
    aus $u(t) \equiv 0$ und $y(t) \equiv 0$ für $t \ge 0$ folgt $\lim_{t \to \infty} x(t) = 0$\\

    \addlinespace[\tablelinespace]
    \emph{Dualität} &
    \multicolumn{2}{c}{$(A^T, C^T)$ stabilisierbar $\iff$ $(A, C)$ entdeckbar}\\

    \addlinespace[\tablelinespace]
    \emph{Verallgemeinerung} &
    regelbar impliziert stabilisierbar &
    beobachtbar impliziert entdeckbar\\

    \addlinespace[\tablelinespace]
    \emph{\name{Hautus}-Test} &
    $\smallpmatrix{A - \lambda I & B}$ hat vollen Zeilenrang für alle
    $\lambda \in \complex$ mit $\Re(\lambda) \ge 0$ &
    $\smallpmatrix{A - \lambda I \\ C}$ hat vollen Spaltenrang für alle $\lambda \in \complex$
    mit $\Re(\lambda) \ge 0$\\

    \addlinespace[\tablelinespace]
    \emph{alg. Charakter.} &
    es gibt eine Matrix $F$ mit $A - BF$ Hurwitz-Matrix &
    es gibt eine Matrix $L$ mit $A - LC$ Hurwitz-Matrix\\

    \addlinespace[\tablelinespace]
    \bottomrule
\end{tabular}
\end{landscape}

\section{%
    Das Separationsprinzip%
}

Es wurde schon gezeigt, wie man ein System durch lineare Zustandsrückführung stabilisieren kann.
Allerdings benötigt diese Regelung den kompletten Zustand zu jeder Zeit.
Es wurde ebenfalls schon gezeigt, wie man den Zustand durch den gemessenen Ausgang
asymptotisch rekonstruieren kann.
Diese Techniken lassen sich enorm ef"|fizient miteinander verbinden:

Für ein stabilisierbares und entdeckbares LTI-System $\dot{x} = Ax + Bu$, $y = Cx + Du$
seien $F$ und $L$, sodass $A - BF$ und $A - LC$ Hurwitz-Matrizen sind.
Dann stabilisiert $u = -Fx$ das System und der Beobachter mit Verstärkung $L$ erzeugt eine
Zustandsschätzung $\widehat{x}$, die $x$ asymptotisch rekonstruiert.
Die Schlüsselidee ist es nun, das nicht verfügbare $x$ durch die verfügbare Schätzung
$\widehat{x}$ zu ersetzen, d.\,h. $u = -F\widehat{x}$.

\linie

\textbf{beobachterbasierter Ausgangsrückführungs-Regler}:
Für die Design-Parameter $F$ und $L$ ist das lineare System
$\dotwidehat{x} = A\widehat{x} + Bu + L(y - \widehat{y})$,
$\widehat{y} = C\widehat{x} + Du$, $u = -F\widehat{x}$
ein\\
\begriff{beobachterbasierter Ausgangsrückführungs-Regler
(observer-based output-feedback controller)}.

Äquivalente Implementierungen sind
$\dotwidehat{x} = (A - LC) \widehat{x} + (B - LD)u + Ly$, $u = -F\widehat{x}$ und
$\dotwidehat{x} = (A - LC - BF + LDF) \widehat{x} + Ly$, $u = -F\widehat{x}$.

\textbf{Satz (geschlossener Regelkreis)}:
Die Verbindung des beobachterbasierten Ausgangsrück"-führungs-Regler mit dem ursprünglichen
System führt zum geschlossenen Regelkreis\\
$\dot{x}  = Ax - BF\widehat{x}$,
$\dotwidehat{x} = (A - LC - BF) \widehat{x} + LCx$.
Dieses System ist asymptotisch stabil genau dann, wenn $A - BF$ und $A - LC$ Hurwitz-Matrizen sind.

Den Satz sieht man sehr schnell durch die Transformation
$\smallpmatrix{x \\ \widetilde{x}} = T \smallpmatrix{x \\ \widehat{x}}$ mit
$T := \smallpmatrix{I & 0 \\ I & -I} = T^{-1}$.
Damit ergibt sich $T \smallpmatrix{A & -BF \\ LC & A - LC - BF} T^{-1} = \A
:= \smallpmatrix{A - BF & BF \\ 0 & A - LC}$, d.\,h. die Eigenwerte des Systems
$\smallpmatrix{\dot{x} \\ \dotwidetilde{x}} = \A \smallpmatrix{x \\ \widetilde{x}}$
sind gleich denen von $A - BF$ vereinigt mit denen von $A - LC$.

\linie

\textbf{statische Ausgangsrückführung}:\\
Bei der \begriff{statischen Ausgangsrückführung (static output-feedback)} des Systems
$\dot{x} = Ax + Bu$, $y = Cx$ setzt man $u = -Ky$.
Man erhält also $\dot{x} = (A - BKC)x$ als geschlossenen Regelkreis.
Bis heute existiert kein einfacher Test,
unter welchen Bedingungen $A - BKC$ Hurwitz ist.

\linie

\textbf{Zusammenfassung}:
\begin{itemize}
    \item
    Überprüfe, ob $(A, B)$ stabilisierbar und $(A, C)$ entdeckbar ist.
    Falls nicht, so kann man zeigen, dass kein linearer, stabilisierender Regler existiert.

    \item
    Falls ja, bestimme $F$ und $L$, sodass $A - BF$ und $A - LC$ Hurwitz-Matrizen sind.

    \item
    Der beobachterbasierte Regler führt zu einem System mit Eigenwerten\\
    $\Eig(A - BF) \cup \Eig(A - LC)$.

    \item
    Wenn $(A, B)$ sogar regelbar und $(A, C)$ beobachtbar ist, dann kann man alle Eigenwerte
    des geschlossenen Regelkreises an beliebige (symmetrische) Stellen setzen.
\end{itemize}

\textbf{Separationsprinzip}:
Weil man $F$ und $L$ unabhängig voneinander konstruieren
(und somit die Zustandsrückführung und den Beobachter getrennt gestalten) kann,
spricht man davon, dass der entstehende Regler auf dem
\begriff{Separationsprinzip (separation principle)} basiert.

Allerdings sind die Eigenwerte von $A$ (System), $A - LC - BF + LDF$ (Regler) und $\A$ i.\,A.
verschieden, d.\,h. der Regler selbst ist evtl. instabil.
In diesem Fall muss man in der Praxis bei der Implementation eines solchen Reglers vorsichtig
sein.

Es ist naiv, die Eigenwerte von $L$ sehr "`schnell"' zu wählen:
Zum einen verstärkt sich dann Messrauschen, zum anderen verringern hohe Beobachter-Verstärkungen
die Robustheit.

\pagebreak

\section{%
    Rauschen und \name{Bode}-Plots%
}

Schnellere Eigenwerte von $L$ führen zu einer schnelleren Konvergenz des Fehlers durch den
Beobachter gegen Null.
Allerdings vergrößern zu große Einträge von $L$ die Empfindlichkeit gegenüber Rauschen,
was man mithilfe der Übertragungsmatrix erkennt.

\textbf{Einfluss von Rauschen}:
Sei $v$ ein Rauschsignal und $\dot{x} = Ax + Bu$, $y = Cx + Du + v$ das durch das Rauschen
gestörte System.
Der beobachterbasierte Ausgangsrückführungs-Regler soll aber gleich bleiben, weil er keinen
Zugriff auf $v$ hat, d.\,h.
$\dotwidehat{x} = A_c \widehat{x} + Ly$, $u = -F\widehat{x}$
mit $A_c := A - LC - BF + LDF$.
Mit der Übertragungsmatrix $G(s) = C (sI - A)^{-1} B + D$ des Systems und
$K(s) = -F(sI - A_c)^{-1} L$ der Übertragungsmatrix des Reglers erhält man
$\widehat{y}(s) = G(s) \widehat{u}(s) + \widehat{v}(s)$ und
$\widehat{u}(s) = K(s) \widehat{y}(s)$
(wenn man Null-Anfangsbedingungen annimmt).

Man kann die 2. Gleichung in die 1. einsetzen,
man bekommt dann $\widehat{y}(s) = G(s)K(s) \widehat{y}(s) + \widehat{v}(s)$.
Nach $\widehat{y}(s)$ aufgelöst ergibt dies
$\widehat{y}(s) = (I - G(s)K(s))^{-1} \widehat{v}(s)$.
Setzt man das in die 2. Gleichung ein, so erhält man
$\widehat{u}(s) = K(s) (I - G(s)K(s))^{-1} \widehat{v}(s)$.

Mit dieser Gleichung kann man den Einfluss des Messrauschens auf die Steuergröße im geschlossenen
Regelkreis analysieren, wenn das System ein SISO-System ist
(z.\,B. mit dem sog. Bode-Amplitudengang des Bode-Plots, siehe unten).

\textbf{Invertierbarkeit von $I - G(s)K(s)$}:
$G(s) = C(sI - A)^{-1} B + D$ ist ($|s|$ hinreichend groß) beschränkt und
$K(s) = -F(sI - A_c)^{-1} L \to 0$ für $|s| \to \infty$,
damit auch $G(s)K(s) \to 0$ bzw. $I - G(s)K(s) \to I$ für $|s| \to \infty$.
Weil $\det(I - G(s)K(s))$ eine rationale Funktion ist, ist sie entweder konstant gleich Null
oder sie hat sie nur endlich viele Nullstellen.
Allerdings ist wegen $\det(I - G(s)K(s)) \to \det(I) = 1$ der erste Fall nicht möglich, d.\,h.
$(I - G(s)K(s))^{-1}$ existiert für fast alle $s \in \complex$.

\linie

\textbf{\name{Bode}-Plot}:
Sei $H\colon D \subset \real \rightarrow \real$ eine reellwertige, rationale Funktion.
Dann heißt der Plot von $|H(\iu\omega)|$ über die Frequenz $\omega \ge 0$
\begriff{\name{Bode}-Amplitudengang (\name{Bode} magnitude plot)} von $H(s)$.
Der Plot von $\arg(H(\iu\omega))$ über die Frequenz $\omega \ge 0$ heißt
\begriff{\name{Bode}-Phasengang (\name{Bode} phase plot)} von $H(s)$.
Beide Plots zusammen werden \begriff{\name{Bode}-Plot} genannt,
mit ihnen wird $\omega \mapsto H(\iu\omega)$ für $\omega \ge 0$ vollständig dargestellt.

Es ist üblich, beim Bode-Plot die Frequenzachsen logarithmisch darzustellen
(Einteilung in 10er-Logarithmen).
Zusätzlich wird die Amplitude auch logarithmisch dargestellt.
Manchmal erfolgt eine Umrechnung in Dezibel (dB) durch die Formel
$|H|_{\text{dB}} := 20 \log_{10} |H|$,
in diesem Fall erfolgt die Darstellung der Amplitude in Dezibel natürlich linear.

Mithilfe des Bode-Amplitudengangs von $K(s) (I - G(s)K(s))^{-1}$
kann man erkennen, dass schnellere Eigenwerte zu einer größeren Verstärkung von Rauschen führen.

\pagebreak
