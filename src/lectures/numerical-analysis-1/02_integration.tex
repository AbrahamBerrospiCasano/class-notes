\section{%
    Integration%
}

\subsection{%
    Quadrataturformeln%
}

Im Folgenden werden Formeln gesucht, mit denen Integrale der Form
$\int_a^b f(x)\dx$ möglichst gut approximiert werden können.
Die vorgestellten Formeln haben alle die Form
$\int_a^b f(x)\dx \approx \sum_{k=1}^n w_k f(x_k)$.
Nun ist die Wahl der Stützstellen $x_k$ und Gewichte $w_k$ maßgeblich, d.\,h.
in diesen Parametern unterscheiden sich die verschiedenen Methoden.
Allen gemeinsam sind dabei folgende Forderungen, die man an die
Approximationsformeln stellt:
\begin{itemize}
    \item
    Die Formel sollte exakt für $f \equiv 1$ sein, d.\,h.
    $b - a= \sum_{k=1}^n w_k$.
    
    \item
    Die Gewichte $w_k > 0$ sollten positiv sein, da man sonst Funktionen
    definieren könnte, die nur bei einem negativen Gewicht größer Null
    und sonst überall Null sind.
    Die Approximation würde einen negativen Wert ergeben, was of"|fensichtlich
    sinnlos ist.
    
    \item
    Die Formel sollte exakt für Polynome $f$ von möglichst hohem Grad sein.
\end{itemize}

\subsubsection{%
    \name{Gauß}-Formel%
}

Die \textbf{\name{Gauß}-Formel der Ordnung $n$} approximiert das Integral einer
Funktion $f$ durch das Integral des Interpolationspolynoms an den Nullstellen
$x_1 < \dotsb < x_n$ des Legendre-Polynoms vom Grad $n$, d.\,h.
\begin{align*}
    \int_{-1}^1 f(x)\dx \approx \sum_{i=1}^n w_i f(x_i)
\end{align*}
mit $w_i$ den Integralen der Lagrange-Polynome über das Intervall $[-1, 1]$:
\begin{align*}
    w_i := \int_{-1}^1 p_i(x)\dx, \quad
    p_i(x) := \prod_{j \not= i} \frac{x - x_j}{x_i - x_j}.
\end{align*}

Die Formel ist exakt für Polynome vom Grad $< 2n$ und vor allem für analytische
Funktionen sehr genau.
Alle Gewichte $w_i$ sind positiv und
die Stützstellen $x_i$ liegen im Integrationsintervall $(-1, 1)$.

Gauß-Parameter $x'$ und $w'$ für ein beliebiges Integrationsintervall $[a, b]$
erhält man durch lineare Transformation:
\begin{align*}
    x_k' = a + \frac{b - a}{2} (x_k + 1), \quad
    w_k' = \frac{b - a}{2} w_k.
\end{align*}

\subsubsection{%
    Konvergenz der \name{Gauß}-Quadratur%
}

Für eine stetige Funktion $f$
\textbf{konvergieren die Approximationen der Gauß-Quadratur}
für ein Integrationsintervall $[a, b]$ mit wachsender Zahl $n$
der Knoten gegen $\int_a^b f(x)\dx$, d.\,h.
\begin{align*}
    s_n f := \sum_{i=1}^n w_i^n f(x_i^n) \xrightarrow{n \to \infty}
    \int_a^b f(x)\dx.
\end{align*}

\subsubsection{%
    Fehler der \name{Gauß}-Quadratur%
}

Der \textbf{Fehler der \name{Gauß}-Quadratur} besitzt die Darstellung
\begin{align*}
    \sum_{i=1}^n w_i f(x_i) - \int_a^b f(x)\dx =
    -\gamma_n f^{(2n)}(\xi)(b - a)^{2n + 1}
\end{align*}
mit
\begin{align*}
    \gamma_n := \frac{(n!)^4}{(2n + 1)((2n)!)^3},\quad \xi \in [a, b].
\end{align*}
Der Kehrwert der Fehlerkonstanten $\gamma_n$ besitzt für $n = 1, 2, 3, 4$
die Werte $24, 4320, 2016000,$
$1778112000$, d.\,h. $\gamma_n$ wird schnell sehr klein.

\subsubsection{%
    Gewichtete \name{Gauß}-Quadratur%
}

Seien $x_i$ die Nullstellen des Orthogonalpolynoms vom Grad $n$ zu einer
Gewichtsfunktion $w$ auf einem Intervall $[a, b]$.
Dann ist die auf polynomialer Interpolation basierende Quadraturformel
\begin{align*}
    \int_a^b f(x)w(x)\dx \approx \sum_{i=1}^n w_i f(x_i)
\end{align*}
für Polynome vom Grad $< 2n$ exakt
(\textbf{gewichtete \name{Gauss}-Quadratur}).
Die Gewichte $w_i$ sind positiv und können als Integrale der Lagrange-Polynome
berechnet werden:
\begin{align*}
    w_i := \int_a^b p_k(x)w(x)\dx, \quad
    p_i(x) := \prod_{j \not= i} \frac{x - x_j}{x_i - x_j}
\end{align*}
Der Fehler ist gleich $\gamma_n f^{(2n)}(\xi)$ für ein $\xi \in [a, b]$
mit einer von der Gewichtsfunktion abhängigen Konstanten $\gamma_n$.

Die folgende Tabelle zeigt die Parameter und Gewichtsfunktionen für die
klassischen Orthogonalpolynome:

\begin{tabular}{llll}
    \toprule
    
    \textbf{Typ} & \textbf{$[a, b]$} & \textbf{$w(t)$} & \textbf{$\gamma_n$} \\
    
    \midrule
    
    \name{Legendre} & $[-1, 1]$ & $1$ &
    $\frac{2^{2n + 1} (n!)^4}{(2n + 1)((2n)!)^3}$ \vspace{2mm}\\
    
    \name{Tschebyscheff} & $[-1, 1]$ & $\sqrt{1 - t^2}$ &
    $\frac{\pi}{2^{2n - 1} (2n)!}$, $n > 0$ \vspace{2mm}\\
    
    \name{Jacobi} & $[-1, 1]$ & $(1 + t)^r (1 - t)^s$ &
    $\frac{2^{2n + r + s + 1} n! \Gamma(n + r + 1) \Gamma(n + s + 1)
    \Gamma(n + r + s + 1)}{(2n + r + s + 1)
    \Gamma(2n + r + s + 1)^2 (2n)!}$ \vspace{2mm}\\
    
    \name{Laguerre} & $[0, \infty)$ & $\exp(-t)$ &
    $\frac{(n!)^2}{(2n)!}$ \vspace{2mm}\\
    
    \name{Hermite} & $(-\infty, \infty)$ & $\exp(-t^2)$ &
    $\frac{\sqrt{\pi} n!}{2^n (2n)!}$ \\
    
    \bottomrule
\end{tabular}

\pagebreak

\subsubsection{%
    Trapezregel%
}

Die Näherung
\begin{align*}
    \int_a^b f(x)\dx \approx s_h f :=
    h \left(\frac{f(a)}{2} + f(a + h) + \dotsb +
    f(b - h) + \frac{f(b)}{2}\right)
\end{align*}
approximiert das Integral durch \textbf{Summe von Trapezflächen (Trapezregel)}.

Für eine zweimal stetig dif"|ferenzierbare Funktion gilt für den Fehler
\begin{align*}
    s_h f - \int_a^b f(x)\dx = \frac{b - a}{12} f''(\xi) h^2
\end{align*}
für ein $\xi \in [a, b]$.

Genauer besitzt der Fehler für glatte Funktionen die asymptotische Entwicklung
\begin{align*}
    s_h f - \int_a^b f(x)\dx =
    c_1 (f'(b) - f'(a)) h^2 +
    c_2 (f'''(b) - f'''(a)) h^4 + \dotsb
\end{align*}
mit von $f$ und $h$ unabhängigen Konstanten $c_j$.
Daraus folgt, dass die Trapezregel für $(b - a)$-periodische Funktionen
sehr genau ist.
Der Fehler strebt schneller als jede $h$-Potenz gegen Null.

Diese Fehlerformel besagt insbesondere, dass für glatte periodische Integranden
schnell eine hohe Genauigkeit erzielt werden kann.

\subsubsection{%
    \name{Bernoulli}-Polynome%
}

Die \textbf{normalisierten \name{Bernoulli}-Polynome} sind definiert
durch die Rekursion
\begin{align*}
    p_i' := p_{i-1}, \quad
    p_0(x) := 1 \quad \text{mit} \quad
    \int_0^1 p_i(x)\dx = 0, \quad
    i \in \natural.
\end{align*}

Die normalisierten Bernoulli-Polynome sind symmetrisch bzgl. $t = \frac{1}{2}$,
d.\,h. $p_{2i-1}(x - \frac{1}{2})$ und $p_{2i}(x - \frac{1}{2})$
sind ungerade bzw. gerade Funktionen ($i \in \natural$).
Für $i \ge 2$ gilt außerdem
\begin{align*}
    & p_{2i-1}(0) = p_{2i-1}(1/2) = p_{2i-1}(1) = 0, \\
    & p_{2i-1}(t) \not= 0 \quad\text{für}\quad
    t \in (0, 1) \setminus \left\{\tfrac{1}{2}\right\}
\end{align*}
und für $i \ge 1$ ist
\begin{align*}
    \gamma_{2i} := p_{2i}(0) = p_{2i}(1)
\end{align*}
entweder ein Minimum oder ein Maximum von $p_{2i}$ auf $[0, 1]$. \\
Die Werte $\gamma_{2i}$ heißen \textbf{normierte \name{Bernoulli}-Zahlen}.

\pagebreak

\subsubsection{%
    \name{Euler}-\name{Maclaurin}-Entwicklung%
}

Für eine glatte Funktion $f$ hat der Fehler $e_h f := s_h f - \int_a^b f(x)\dx$
der Trapezregel die Entwicklung
\begin{align*}
    \sum_{i=1}^{m-1} \gamma_{2i} (f^{(2i-1)}(b) - f^{(2i-1)}(a)) h^{2i}
\end{align*}
mit dem Restglied
\begin{align*}
    \gamma_{2m} f^{(2m)}(\xi) (b - a) h^{2m}
\end{align*}
für ein $\xi \in [a, b]$ und $\gamma_{2i}$ den normierten Bernoulli-Zahlen \\
(\textbf{\name{Euler}-\name{Maclaurin}-Entwicklung}).

Aus der Entwicklung folgt insbesondere, dass die Trapezregel für unendlich oft
dif"|ferenzierbare $(b - a)$-periodische Funktionen sehr genau ist.
Der Fehler strebt schneller als jede $h$-Potenz gegen Null.
Für nicht-periodische Funktionen bildet die Entwicklung die Grundlage für
Extrapolationsverfahren, mit denen ebenfalls beliebige
Approximationsordnungen erzielt werden können. 

\subsubsection{%
    \name{Romberg}-Algorithmus%
}

Die Genauigkeit der Trapezregel
\begin{align*}
    s_h^1 := h \left(\frac{f(a)}{2} + f(a + h) + \dotsb +
    f(b - h) + \frac{f(b)}{2}\right)
\end{align*}
lässt sich durch Extrapolation verbessern
(\textbf{\name{Romberg}-Algorithmus}). \\
Die rekursiv definierten Approximationen
\begin{align*}
    s_h^{j+1} := \frac{4^j s_{h/2}^j - s_h^j}{4^j - 1}
\end{align*}
haben die Fehlerordnung $\O(h^{2j+2})$ und können in einem Dreiecksschema
berechnet werden:
\begin{align*}
    \begin{array}{ccccc}
        s_h^1 & \rightarrow & s_h^2 & \rightarrow & s_h^3 \\
        & \nearrow & & \nearrow \\
        s_{h/2}^1 & \rightarrow & s_{h/2}^2 \\
        & \nearrow \\
        s_{h/4}^1
    \end{array}
\end{align*}
Es werden solange sukzessive Diagonalen
$s_{2^{-m}}^1, s_{2^{1-m}}^2, \dotsc, s_h^{m+1}$ hinzugefügt, bis mit dem
zuletzt generierten Wert die gewünschte Genauigkeit erreicht ist.
Bei den Trapezsummen können bereits berechnete Funktionswerte genutzt werden:
\begin{align*}
    s_{h/2}^1 = \frac{1}{2} \left(s_h^1 +
    h \left(f(a + \tfrac{h}{2}) + \dotsb +
    f\left(b - \tfrac{h}{2}\right)\right)\right).
\end{align*}

\pagebreak

\subsubsection{%
    Numerische Integration mit \matlab{}%
}

Das Integral $s = \int_a^b f(x)\dx$ kann in \matlab{} mit dem Befehl
\code{s = quad(f, a, b, tol);} berechnet werden,
wobei die zu integrierende Funktion \code{f} als Funktionshandle, Funktionsname
(String) oder Inlinefunktion übergeben wird.
\code{tol} ist optional (standardmäßig $10^{-6}$) und gibt die absolute
Genauigkeit vor.

\code{quad} basiert dabei auf der Simpson-Regel
(abschnittsweise Interpolation mit quadratischen Polynomen und anschließende
Integration) mit adaptiver Unterteilung des Integrationsintervalls, d.\,h.
die Intervalllängen bzw. die Anzahl an Funktionsauswertungen werden an
die lokale Komplexität der Funktion angepasst.

Alternativ zu \code{quad} gibt es noch die Befehle \code{quadl} und
\code{dblquad}.
\code{quadl} erzielt eine höhere Approximationsordnung als \code{quad},
sollte aber nur bei hohen Genauigkeiten und glatten Integranden verwendet
werden.
Bei niedrigen Genauigkeiten oder nicht-glatten Integranden empfiehlt sich daher
die Verwendung von \code{quad}.
\code{dblquad} berechnet ein bivariates Integral, d.\,h.
\code{s = dblquad(f, xmin, xmax, ymin, ymax, tol);} berechnet
das Integral der bivariaten Funktion \code{f} über das Rechteck
$[$\code{xmin}$,\;$\code{xmax}$] \times [$\code{ymin}$,\;$\code{ymax}$]$.

\pagebreak

\subsection{%
    Mehrfachintegrale%
}

\subsubsection{%
    Tensorprodukt von Integrationsformeln%
}

\textbf{Integrationsformeln für Rechteck-Gebiete}
\begin{align*}
    Q := [a_1, b_1] \times \dotsb \times [a_m, b_m]
\end{align*}
erhält man durch Bilden von Tensorprodukten eindimensionaler Quadraturformeln.

Sind die Formeln $\sum_k w_{k,\nu} f(t_{k,\nu})$ zur Approximation von
$\int_{a_\nu}^{b_\nu} f$ exakt für Polynome vom Grad $\le n_\nu$,
$\nu = 1, \dotsc, m$, so ist die Produktformel
\begin{align*}
    \int_Q f \approx
    \sum_{k_1} \dotsb \sum_{k_m} (w_{k_1,1} \dotsb w_{k_m,m})
    f(t_{k_1,1}, \dotsc, t_{k_m,m})
\end{align*}
exakt für Polynome vom Koordinatengrad $\le (n_1, \dotsc, n_m)$.

\linie

Für die \textbf{Trapezregel} mit Schrittweite $h = \frac{b - a}{n}$
sind die Gewichte an den inneren Knoten gleich $h$ und an den äußeren
Knoten $a$ und $b$ gleich $\frac{h}{2}$.
Für ein Rechteck $[a_1, b_1] \times [a_2, b_2]$ und den Schrittweiten $h_1$
und $h_2$ erhält man somit drei verschiedene Gewichte:
$h_1 h_2$ für innere Knoten, $\frac{h_1 h_2}{2}$ für Knoten auf den Kanten
(außer den Ecken) und $\frac{h_1 h_2}{4}$ für die Ecken.

Allgemein gilt für ein $m$-dimensionales Rechteck mit Schrittweiten
$h_\nu$, $\nu = 1, \dotsc, m$, dass
$w_{k_1, \dotsc, k_m} = h^m 2^{-\alpha_1} \dotsm 2^{-\alpha_m}$
mit $\alpha_\nu = 0$ für einen inneren Index bzw.
$\alpha_\nu = 1$ für einen äußeren Index $k_\nu$, $\nu = 1, \dotsc, m$.

Der Fehler der multivariaten Trapezregel besitzt eine quadratische Entwicklung,
sodass wie für die univariate Formel die Romberg-Extrapolation anwendbar ist.

\pagebreak

\subsubsection{%
    Transformation von Integrationsformeln%
}

Für eine bijektive, stetig dif"|ferenzierbare Transformation $g$ eines
regulären Bereiches $U \subset \real^n$ mit
$\det g'(x) \not= 0$ für $x \in U$
gilt für stetige Funktionen $f$
\begin{align*}
    \int_U f \circ g |\det g'| dU = \int_V f dV, \quad V = g(U),
\end{align*}
wobei $\det g'$ als Funktionaldeterminante der Transformation bezeichnet wird.

\linie

Eine Integrationsformel
\begin{align*}
    \int_D f \approx \sum_k w_k f(x_k)
\end{align*}
kann man durch Variablensubstitution
\textbf{auf andere Gebiete transformieren}. \\
Ist $\varphi\colon D \rightarrow \widetilde{D}$ eine bijektive Abbildung,
erhält man durch
\begin{align*}
    \widetilde{w}_k = \left|\det \varphi'(x_k)\right| w_k, \quad
    \widetilde{x}_k = \varphi(x_k)
\end{align*}
Gewichte und Punkte für eine Integrationsformel auf $\widetilde{D}$.
Dabei ist $\varphi'(x_k)$ die Jacobi-Matrix von $\varphi$ im Punkt $x_k$.
$\det \varphi'(x_k)$ heißt \textbf{Funktionaldeterminante} der Transformation.

Speziell gilt bei einer af"|finen Abbildung
$\varphi(x) = Ax + b$ für die Gewichte
$\widetilde{w}_i = |\det A| w_i$.

Die Konvergenz der Integrationsformeln bleibt bei glatten Transformationen
$\varphi$ erhalten.
Allerdings werden Polynome auf $\widetilde{D}$ nicht mehr exakt integriert, da
der Integrand die Funktionaldeterminante der Transformation enthält.

\linie

Beispielsweise ist ein durch eine Funktion $h$ berandeter
Integrationsbereich \\
$D = \{(x, y) \in \real^2 \;|\; x \in [a, b],\; y \in [0, h(x)]\}$
als Bild des Rechtecks $Q = [a, b] \times [0, 1]$ unter der Abbildung
$\varphi\colon Q \rightarrow D$, $(u, v) \mapsto (u, v h(u))$ mit der
Funktionaldeterminante \\
$\det \varphi'(u, v) =
\det\begin{pmatrix}1 & 0\\vh'(u) & h(u)\end{pmatrix} = h(u)$
darstellbar.
Damit transformieren sich Punkte und Gewichte einer Produktformel für $Q$
gemäß $(u_i, v_j) \rightarrow (u_i, v_j h(u_i))$ und
$w_{i,j} \rightarrow w_{i,j} h(u_i)$.

\subsubsection{%
    Integrationsformeln für Simplizes%
}

Für einen $m$-dimensionalen Simplex $S$ mit den Ecken $v_0, \dotsc, v_m$ lässt
sich eine Integrationsformel durch Interpolation mit Polynomen vom
\textbf{totalen Grad}
(d.\,h. die Summe der Koordinatengrade) $\le n$ konstruieren.
Das interpolierende Polynom ist eindeutig durch die Werte an den Punkten
\begin{align*}
    x_k = \frac{1}{n} \sum_{\nu=0}^n k_\nu v_\nu, \quad
    k_0 + \dotsb + k_n = n
\end{align*}
bestimmt, die ein regelmäßiges Gitter bilden.
\begin{align*}
    \int_S f \approx
    \vol S \sum_{k_0 + \dotsb + k_n = n} w_k f(x_k)
\end{align*}
ist eine Approximation der Ordnung $n + 1$, d.\,h.
die Integrationsformel ist exakt für alle Polynome vom totalen Grad $\le n$
(\textbf{Integrationsformel für Simplizes}).
Dabei sind $\vol S \cdot w_k$ die Integrale über die Lagrange-Polynome
zu $x_k$.

Für allgemeine Gebiete kann die Integrationsformel auf den Simplizes einer
Triangulierung angewendet werden.
Der Fehler hat dann die Ordnung $\O(h^{n+1})$, wobei $h$ den maximalen
Durchmesser der Teilsimplizes bezeichnet.

\subsection{%
    Monte-Carlo-Verfahren%
}

\subsubsection{%
    Lineare Kongruenzmethode%
}

Die \textbf{lineare Kongruenzmethode} definiert durch
\begin{align*}
    n_\ell & := \alpha n_{\ell-1} \mod \beta \\
    x_\ell & := n_\ell / \beta
\end{align*}
mit $\alpha \in \natural$, $1 < \alpha < \beta$ und $\beta$
einer sehr großen Primzahl kann bei geeigneter Wahl der Parameter zur
numerischen Simulation von Zufallszahlen $x_\ell \in \left[0, 1\right)$
benutzt werden.

Eine minimale Anforderung ist, dass die maximale Periode $\beta - 1$ erreicht
wird.
Darüber hinaus soll die Folge $x_0, x_1, \dotsc$ bei möglichst vielen
statistischen Tests gute Ergebnisse liefern.

\subsubsection{%
    Satz von \name{Fermat}%
}

Für jede Primzahl $\beta$ und $\alpha \not= 0 \mod \beta$ gilt
der \textbf{Satz von \name{Fermat}}
\begin{align*}
    \alpha^{\beta-1} = 1 \mod \beta.
\end{align*}

\subsubsection{%
    Maximale Periode bei der linearen Kongruenzmethode%
}

Für eine Primzahl $\beta$ hat die Folge $\alpha^\ell \mod \beta$,
$\ell = 0, 1, \dotsc$ keine kleinere Periode als $\beta - 1$ genau dann, wenn
\begin{align*}
    \alpha^{(\beta-1)/m} \not= 1 \mod \beta
\end{align*}
für alle Primteiler $m$ von $\beta - 1$.
(Die Periode ist dabei die Anzahl der vorkommenden verschiedenen Zahlen.)

Mithilfe dieses Kriteriums lassen sich geeignete Multiplikatoren $\alpha$
für die Simulation von Zufallszahlen mit der linearen Kongruenzmethode
bestimmen.

\linie

Gebräuliche Parameter bei der Generierung von Pseudo-Zufallszahlen mit der
linearen Kongruenzmethode sind die \name{Mersenne}-Primzahl
$\beta = 2147483647 = 2^{31} - 1$ und der Multiplikator $\alpha = 16807$.
Aufgrund der Größe dieser Zahlen ist das Testen der Periodenlänge
nicht ganz einfach.

Aufgrund der Primfaktorzerlegung
$\beta - 1 = 2 \cdot 3^2 \cdot 7 \cdot 11 \cdot 31 \cdot 151 \cdot 331$
erhält man als mögliche Perioden $m = (\beta - 1)/p_k$:
$1073741823, 715827882, 306783378, 195225786, 69273666, 14221746,$
$6487866$.

Zur Berechnung der Potenzen $\alpha^m$ geht man zur Dualdarstellung
$m = m_0 + 2m_1 + 4m_2 + \dotsb$, $m_k \in \{0, 1\}$ über.
Man kann dann zunächst rekursiv die Potenzen
$\alpha_k := \alpha^{2^k} \mod \beta$ berechnen, indem man
$\alpha_{k+1} \mod \beta = \alpha_k^2 \mod \beta$ rechnet.
Damit ist
$\alpha^m \mod \beta = (\prod_{m_k=1} \alpha_k) \mod \beta$.

Zum Beispiel ist $195225786 = (001011101000101110100010111010)_2$
und man erhält \\
$16807^{195225786} \mod \beta = 997852928 \not= 1$.
Ebenso sind alle sechs anderen zu testenden Potenzen ungleich $1$ modulo
$\beta$.
Damit ist die Periode des \name{Mersenne}-Generators maximal.

\pagebreak

\subsubsection{%
    Spektraltest für die lineare Kongruenzmethode%
}

Für eine Primzahl $\beta$ lässt sich die durch
\begin{align*}
    u_\ell := (\alpha^{\ell m} a \mod \beta) / \beta, \qquad
    a := (1, \alpha, \dotsc, \alpha^{m-1}) \in \real^m
\end{align*}
definierte Folge von Vektoren $u_\ell$ durch parallele Hyperebenen im Abstand
\begin{align*}
    d := (\min\{\norm{n}_2 \;|\; n \in \integer^m,\; \norm{n}_2 \not= 0,\;
    a^t n = 0 \mod \beta\})^{-1}
\end{align*}
überdecken.

Der Abstand $d$ dient zur Beurteilung der Güte der Folge der
Pseudo-Zufallsvektoren $u_\ell$ (\textbf{Spektraltest}).
Je kleiner $d$ ist, um so besser sind im Allgemeinen die statistischen
Eigenschaften der Folge.

\subsubsection{%
    Gleichverteilte Folgen%
}

Eine Folge $x_0, x_1, \dotsc$ in einem Quader
\begin{align*}
    Q := [a_1, b_1] \times \dotsb \times [a_n, b_n]
\end{align*}
heißt \textbf{gleichverteilt}, falls
\begin{align*}
    \lim_{\ell \to \infty} \frac{|\{x_k \in Q' \;|\; k < \ell\}|}{\ell}
    = \frac{\vol Q'}{\vol Q}
\end{align*}
für alle Teilquader $Q' \subseteq Q$.

Allgemeiner heißt die Folge \textbf{$m$-verteilt}, falls
\begin{align*}
    \lim_{\ell \to \infty} \frac{|\{x_{k+1} \in Q'_1, \dotsc, x_{k+m} \in Q'_m
    \;|\; k < \ell\}|}{\ell}
    = \frac{\vol Q'_1 \dotsm \vol Q'_m}{\vol Q}
\end{align*}
für alle Teilquader $Q'_1, \dotsc, Q'_m \subseteq Q$.

Eine Folge, die für alle $m \in \natural$ $m$-verteilt ist, heißt
\textbf{$\infty$-verteilt}.

\linie

\name{Franklin} hat gezeigt, dass die Folge
\begin{align*}
    x_\ell := r^\ell \mod 1,\quad \ell = 0, 1, \dotsc
\end{align*}
für fast alle $r > 1$ $\infty$-verteilt in $[0, 1)$ ist. 

\subsubsection{%
    Konvergenz der Monte-Carlo-Integration%
}

Für eine gleichverteilte Folge $x_0, x_1, \dotsc$ in $[0, 1)$ gilt
\begin{align*}
    \lim_{\ell \to \infty} \frac{1}{\ell} \sum_{k<\ell} f(x_k) = \int_0^1 f
\end{align*}
für jede Riemann-integrierbare Funktion $f$.
Das entsprechende Approximationsverfahren wird aufgrund der quasi zufälligen
Wahl der Punkte $x_k$ als \textbf{Monte-Carlo-Integration} bezeichnet.

\pagebreak

\subsubsection{%
    Transformation gleichverteilter Zahlenfolgen%
}

Ausgehend von einer gleichverteilten Folge $x_0, x_1, \dotsc$ in dem
Standardintervall $[0, 1)$ können gleichverteilte Zahlen und Vektorfolgen
auf allgemeineren Mengen konstruiert werden. \\
Für $k = 0, 1, \dotsc$ ist
\begin{itemize}
    \item
    $a + (b - a)x_k$ eine Folge in $[a, b)$,
    
    \item
    $\lfloor m + (n - m)x_k \rfloor$ eine Folge in $\{m, \dotsc, n - 1\}$,
    
    \item
    $U_k := (x_{mk}, \dotsc, x_{mk + m - 1})$ eine Folge in $[0, 1)^m$ und
    
    \item
    $A U_k + b$ mit einer $m \times m$-Matrix $A$ und einem $m$-Vektor $b$
    eine Folge in $A[0,1)^m + b$.
\end{itemize}
Außerdem ist für eine gleichverteilte Folge in einem Quader $Q$ die in einer
Teilmenge $D \subseteq Q$ liegende Teilfolge gleichverteilt in $D$.

\linie

Sei $x_0, x_1, \dotsc$ eine gleichverteilte Folge in $[0, 1)$.

Zur Simulation von Würfeln kann die Folge
$n_k = 1 + 6 \lfloor x_k \rfloor$ verwendet werden.

Eine Folge im abgeschlossenen Einheitskreis $D\colon u^2 + v^2 \le 1$
erhält man durch die Transformation $(u_k, v_k) = 2(x_{2k}, x_{2k+1}) - (1, 1)$
und Auswahl der Teilfolge, die in $D$ liegt.

Gleichverteilte Permutationen von $\{1, \dotsc, n\}$, etwa zum Simulieren
des Mischens von Karten, können durch die Reihenfolge der Indizes beim
Sortieren der Komponenten gleichverteilter $n$-Vektoren generiert werden.

\subsubsection{%
    Multivariate Monte-Carlo-Integration%
}

Ist $u_0, u_1, \dotsc$ eine gleichverteilte Folge in einem Quader $Q$, so lässt
sich ein Integral über ein Gebiet $D \subseteq Q$ durch
\begin{align*}
    \int_D f = \lim_{\ell \to \infty} \frac{\vol Q}{\ell}
    \sum_{k < \ell,\; u_k \in D} f(u_k)
\end{align*}
approximieren (\textbf{multivariate Monte-Carlo-Integration}). \\
Im Spezialfall $f = 1$ erhält man ein Verfahren zur Volumenbestimmung:
\begin{align*}
    \vol D = \lim_{\ell \to \infty} \frac{\vol Q}{\ell}
    |\{u_k \in D \;|\; k < \ell\}|.
\end{align*}

\linie

Der Vorteil multivariater Monte-Carlo-Integrationen ist die weitgehende
Unabhängigkeit der Konvergenzrate von der Dimension.
Monte-Carlo-Verfahren sind daher besonders in hohen Dimension gut geeignet
(z.\,B. besser als Trapezregel).

\pagebreak
