\chapter{%
    Reelle Vektorräume%
}

\section{%
    \texorpdfstring{Der $n$-dimensionale reelle Raum}%
    {Der n-dimensionale reelle Raum}%
}

\begin{Def}{Vektorraum $\real^n$}
    Sei $n \in \natural$.
    Auf $\real^n$ ist eine \begriff{Addition} definiert durch \\
    $(\alpha_1, \ldots, \alpha_n) + (\beta_1, \ldots, \beta_n)
    = (\alpha_1 + \beta_1, \ldots, \alpha_n + \beta_n)$ und eine
    \begriff{skalare Multiplikation} definiert durch
    $\lambda (\alpha_1, \ldots, \alpha_n) =
    (\lambda \alpha_1, \ldots, \lambda \alpha_n)$,
    wobei $\alpha_i, \beta_i, \lambda \in \real$ für $i = 1, \ldots, n$.
\end{Def}

\begin{Satz}{Vektorraum-Axiome im $\real^n$}
    Die Addition im $\real^n$ ist assoziativ, es gibt einen Nullvektor
    $0 = (0, \ldots, 0)$ (neutrales Element), für jeden Vektor $v \in \real^n$
    gibt es ein eindeutig bestimmtes additiv Inverses $-v \in \real^n$
    und die Addition ist kommutativ. \\
    $1 \in \real$ ist bzgl. der skalaren Multiplikation ein neutrales Element,
    die skalare Multiplikation ist skalar assoziativ
    ($(\lambda \mu) v = \lambda (\mu v)$),
    skalar distributiv ($\lambda (u + v) = \lambda u + \lambda v$) sowie
    vektoriell distributiv ($(\lambda + \mu) v = \lambda v + \mu v$). \qquad
    Außerdem gilt $0 \cdot v = 0$ sowie $(-1) \cdot v = -v$.
\end{Satz}

\section{%
    Linearkombinationen und Unterräume%
}

\begin{Def}{skalare Vielfache}
    Seien $V = \real^n$ und $v \in V$. \\
    Man schreibt $\real v = \{\lambda v \;|\; \lambda \in \real\}$,
    die Elemente von $\real v$ heißen \begriff{skalare Vielfache}.
\end{Def}

\begin{Def}{Linearkombination}
    Seien $V = \real^n$ und $T \subseteq V$ mit $T \not= \emptyset$.
    Eine \begriff{Linearkombination} von $T$ ist ein Ausdruck der Form
    $\lambda_1 v_1 + \cdots + \lambda_k v_k$, wobei $v_i \in T$ und
    $\lambda_i \in \real$ für $i = 1, \ldots, k$ ist.
\end{Def}

\begin{Def}{linearer Aufspann}
    Seien $V = \real^n$ und $T \subseteq V$ mit $T \not= \emptyset$.
    Die Menge aller Linearkombinationen von $T$ heißt
    \begriff{linearer Aufspann} und wird mit $\aufspann{T}$ bezeichnet. \\
    Also ist $\aufspann{T} = \left\{\left. \sum_{i=1}^k \lambda_i v_i
    \;\right|\; v_i \in T,\; \lambda_i \in \real,\;
    i = 1, \ldots, k \right\}$ \\
    $= \left\{\left. \sum_{t \in T} \lambda_t t
    \;\right|\; \lambda_t \in \real,\;
    \lambda_t = 0 \text{ f"ur fast alle } t \in T \right\}$. \qquad
    Es ist $\aufspann{\{v_1, \ldots, v_j\}} = \aufspann{v_1, \ldots, v_j}$.
\end{Def}

\begin{Satz}{Aufspann einer Teilmenge ist abgeschlossen}
    Sei $T \subseteq \real^n$ mit $T \not= \emptyset$ sowie
    $U = \aufspann{T}$. \\
    Dann gilt $u + v \in U$ und $\lambda v \in U$ für $u, v \in U$.
    Die Teilmenge $U$ ist also abgeschlossen bzgl. Addition und skalarer
    Multiplikation.
\end{Satz}

\begin{Def}{reeller Vektorraum}
    Ein \begriff{reeller Vektorraum} ($\real$-Vektorraum) ist eine Menge $V$,
    für die zwei Abbildungen
    $\boldsymbol{+}: V \times V \rightarrow V$, $(u, v) \mapsto u + v$ sowie
    $\boldsymbol{\cdot}: \real \times V \rightarrow V$,
    $(\lambda, v) \mapsto \lambda \cdot v$ definiert sind.
    Dabei bildet $V$ mit der Addition eine abelsche Gruppe und
    die skalare Multiplikation besitzt das neutrale Element $1 \in \real$
    ($1 \cdot v = v$), ist skalar assoziativ sowie skalar und vektoriell
    distributiv über der Vektoraddition.
\end{Def}

\begin{Kor}
    Nullvektor und additiv inverse Elemente sind eindeutig. \\
    Es gilt $0 \cdot v = 0$, \quad
    $\alpha \cdot 0 = 0$ \quad und \quad
    $(-1) \cdot v = -v$.
\end{Kor}

\section{%
    \emph{Zusätzliches}: Polynome%
}

\begin{Def}{Polynom}
    Sei $K$ ein Körper. \\
    $K[x]$ ist die Menge der Ausdrücke der Form
    $f(x) = \sum_{i=0}^n \alpha_i x^i$ ($n \in \natural_0$,
    $\alpha_i \in K$). \\
    $f(x)$ heißt \begriff{Polynom} vom Grad $n = \deg f$ ($a_n \not= 0$).
    Für $f = 0$ ist $\deg f = -1$. \\
    Eine \begriff{polynomiale Funktion} ist eine Funktion
    $f: K \rightarrow K$, $f(x) = \sum_{i = 0}^n \alpha_i x^i$.
\end{Def}

\pagebreak
