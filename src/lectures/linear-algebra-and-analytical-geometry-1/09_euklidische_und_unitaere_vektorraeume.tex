\chapter{%
    Euklidische und unitäre Vektorräume%
}

\begin{Bem}
    Geometrie hat mit "`Messen"' zu tun.
    Wichtige Hilfsmittel sind Längen- und Winkelmessung.
    Die Länge eines Vektors kann man mit Hilfe einer Norm definieren.
\end{Bem}

\begin{Def}{Norm}
    Sei $V$ ein $K$-Vektorraum, wobei $K = \real$ oder $K = \complex$.
    Eine \begriff{Norm} $\norm{\cdot}$ auf $V$ ist eine Abbildung
    $\norm{\cdot}: V \rightarrow \real$, falls für alle $v, v_1, v_2 \in V$
    und $\alpha \in K$ gilt:
    \begin{enumerate}
        \item[(1)] $\norm{v} \ge 0$ \quad sowie \quad
        $\norm{v} = 0 \;\Leftrightarrow\; v = 0$

        \item[(2)] $\norm{\alpha v} = |\alpha| \norm{v}$

        \item[(3)] $\norm{v_1 + v_2} \le \norm{v_1} + \norm{v_2}$
    \end{enumerate}
    V zusammen mit einer Norm $\norm{\cdot}$ heißt
    \begriff{normierter Vektorraum}.
    $V$ wird mit der von der Norm induzierten Metrik
    $d(v_1, v_2) = \norm{v_1 - v_2}$ ($v_1, v_2 \in V$) zum
    \begriff{metrischen Raum}.
\end{Def}

\section{%
    Skalarprodukte%
}

\begin{Bem}
    Zur Winkelmessung werden innere Produkte bzw. Skalarprodukte benötigt.
\end{Bem}

\begin{Def}{Bilinearform}
    Sei $K$ ein Körper und $V$ ein $K$-Vektorraum.
    Eine Abbildung $\innerproduct{\cdot, \cdot}: V \times V \rightarrow K$ heißt
    \begriff{bilinear}, falls für alle $x, y, z \in V$ und $\alpha \in K$ gilt:
    \begin{enumerate}
        \item[(1)] $\innerproduct{x + y, z} = \innerproduct{x, z} + \innerproduct{y, z}$

        \item[(2)] $\innerproduct{x, y + z} = \innerproduct{x, y} + \innerproduct{x, z}$

        \item[(3)] $\innerproduct{\alpha x, y} = \innerproduct{x, \alpha y} = \alpha \innerproduct{x, y}$
    \end{enumerate}
\end{Def}

\begin{Def}{Eigenschaften einer reellen Bilinearform}
    Sei $\innerproduct{\cdot, \cdot}$ eine Bilinearform auf dem reellen Vektorraum $V$.
    Dann heißt $\innerproduct{\cdot, \cdot}$ \begriff{symmetrisch},
    falls $\innerproduct{x, y} = \innerproduct{y, x}$ für alle $x, y \in V$.
    $\innerproduct{\cdot, \cdot}$ heißt \begriff{positiv semidefinit} und
    \begriff{positiv definit}, falls $\innerproduct{x, y} \ge 0$ und
    $\innerproduct{x, x} = 0 \;\Leftrightarrow\; x = 0$ für alle $x \in V$.
\end{Def}

\begin{xDef}{reelles Skalarprodukt}{Skalarprodukt!reelles}
    Eine positiv definite symmetrische Bilinearform auf einem reellen
    Vektorraum $V$ heißt
    \xbegriff{(reelles) Skalarprodukt}{Skalarprodukt!reelles} auf $V$. \\
    $V$ zusammen mit einem reellen Skalarprodukt heißt
    \begriff{euklidischer Vektorraum}.
\end{xDef}

\begin{Def}{hermitesche Form}
    Sei $V$ ein komplexer Vektorraum.
    Eine Abbildung $\innerproduct{\cdot, \cdot}: V \times V \rightarrow \complex$
    heißt \begriff{hermitesche Form}, falls für alle $x, y, z \in V$ und
    $\lambda \in \complex$ gilt:
    \begin{enumerate}
        \item[(1)] $\innerproduct{x + y, z} = \innerproduct{x, z} + \innerproduct{y, z}$

        \item[(2)] $\innerproduct{\lambda x, y} = \lambda \innerproduct{x, y}$

        \item[(3)] $\innerproduct{x, y} = \kk{\innerproduct{y, x}}$
    \end{enumerate}
\end{Def}

\begin{Lemma}{Eigenschaften der hermiteschen Form}
    Für eine hermitesche Form $\innerproduct{\cdot, \cdot}$ gilt für alle
    $x, y, z \in V$ und $\lambda \in \complex$, dass
    $\innerproduct{x, y + z} = \innerproduct{x, y} + \innerproduct{x, z}$,
    $\innerproduct{x, \lambda y} = \kk{\lambda} \innerproduct{x, y}$ und
    $\innerproduct{x, x} \in \real$.
\end{Lemma}

\begin{Def}{Eigenschaften einer hermiteschen Form}
    Eine hermitesche Form heißt \begriff{positiv semi"-definit}, falls
    $\innerproduct{x, x} \ge 0$ für alle $x \in V$, und \begriff{positiv definit}, falls
    $\innerproduct{x, x} > 0$ für $x \in V$, $x \not= 0$.
\end{Def}

\begin{xDef}{komplexes Skalarprodukt}{Skalarprodukt!komplexes}
    Eine positiv definite hermitesche Form auf einem komplexen Vektorraum $V$
    heißt \xbegriff{(komplexes) Skalarprodukt}{Skalarprodukt!komplexes}
    auf $V$. \\
    $V$ zusammen mit einem komplexen Skalarprodukt heißt
    \begriff{unitärer Vektorraum}.
\end{xDef}

\begin{Bem}
    Im Folgenden sei $K = \real$/$K= \complex$, $V$ ein $K$-Vektorraum
    und $\innerproduct{\cdot, \cdot}$ eine bilineare/hermitesche Form auf $V$,
    sodass $V$ mit dieser Form einen euklidischen/unitären Raum bildet.
\end{Bem}

\begin{Def}{vom Skalarprodukt induzierte Norm}
    Sei $x \in V$.
    Die \begriff{Norm} oder \begriff{Länge} von $x$ ist die reelle Zahl
    $\norm{x} = \sqrt{\innerproduct{x, x}}$.
    Insbesondere ist $\norm{x} = 0 \;\Leftrightarrow\; x = 0$.
\end{Def}

\begin{Def}{Einheitsvektoren}
    Sei $x \in V$, $x \not= 0$.
    Dann hat $\frac{x}{\norm{x}}$ die Länge 1. \\
    Vektoren der Länge $1$ heißen \begriff{normiert} oder
    \begriff{Einheitsvektoren}.
\end{Def}

\begin{Satz}{Ungleichung von \name{Cauchy}-\name{Schwarz}}
    Seien $V$ ein euklidischer oder unitärer Vektorraum und $x, y \in V$.
    Dann ist $|\innerproduct{x, y}| \le \norm{x} \cdot \norm{y}$.
\end{Satz}

\begin{Kor}
    Seien $V$ ein euklidischer oder unitärer Vektorraum und $\norm{\cdot}$
    die vom Skalarprodukt induzierte Norm.
    Dann ist $V$ zusammen mit $\norm{\cdot}$ ein normierter Vektorraum
    im Sinne der Definition der Norm, d.\,h. insbesondere ist
    die Dreiecksungleichung $\norm{x + y} \le \norm{x} + \norm{y}$ für
    $x, y \in V$ erfüllt.
\end{Kor}

\begin{Kor}
    Seien $V$ ein euklidischer Vektorraum und $\norm{\cdot}$ wie oben.
    Dann ist der Betrag von
    $\cos(x, y) =$ \fracsize{$\frac{\innerproduct{x, y}}{\norm{x} \cdot \norm{y}}$}
    eine reelle nicht-negative Zahl $\le 1$.
    Insbesondere ist $\cos(x, y) \in [-1, 1]$.
\end{Kor}

\begin{Def}{Winkel}
    Seien $V$ ein euklidischer Vektorraum und $x, y \in V$.
    Dann ist der Winkel $\alpha$ zwischen $x$ und $y$ (nicht eindeutig) gegeben
    durch $\cos \alpha = \cos(x, y) =$
    \fracsize{$\frac{\innerproduct{x, y}}{\norm{x} \cdot \norm{y}}$}.
\end{Def}

\begin{Bem}
    Wie herum soll der Winkel orientiert werden, was macht man für unitäre
    Vektorräume (dort kann $\cos(x, y)$ auch komplex sein)?
    Für viele Anwendungen benötigt man keine echten Winkel, sondern nur das
    Prinzip der Orthogonalität.
\end{Bem}

\begin{Def}{Orthogonalität}
    Sei $V$ ein euklidischer/unitärer Vektorraum. \\
    $x$ und $y$ sind \begriff{orthogonal} ($x \orth y$), falls
    $\innerproduct{x, y} = 0$ ist (wobei $x, y \in V$). \\
    $M$ und $N$ sind \begriff{orthogonal} ($M \orth N$), falls $x \orth y$ für
    alle $x \in M$, $y \in N$ (wobei $\emptyset \not= M, N \subseteq V$). \\
    $M^\bot = \{v \in V \;|\; \forall_{m \in M}\; \innerproduct{v, m} = 0\}$ ist die
    \begriff{Menge der zu $M$ orthogonalen Vektoren}.
\end{Def}

\begin{Lemma}{$M^\bot$ als Unterraum}
    $M^\bot$ ist ein Unterraum von $V$.
\end{Lemma}

\begin{Def}{orthogonales/orthonormales System}
    Sei $M \subseteq V$ mit $M \not= \emptyset$.
    Dann heißt $M$ \\
    \begriff{orthogonales System} in $V$, falls
    $M$ aus paarweise orthogonalen Elementen $\not=$ Nullvektor besteht. \\
    $M$ heißt \begriff{orthonormales System}, falls zusätzlich alle Vektoren
    normiert sind.
\end{Def}

\begin{Satz}{orthogonales System ist linear unabhängig} \\
    Ein System orthogonaler Vektoren in $V$ ist linear unabhängig.
\end{Satz}

\begin{Def}{Orthonormalbasis (ONB)} \\
    Ein orthonormales Erzeugendensystem von $V$ heißt
    \begriff{Orthonormalbasis (ONB)} von $V$.
\end{Def}

\begin{Prozedur}{Orthonormalisierungsverfahren nach
                 \name{Gram}-\name{Schmidt}} \\
    Seien $V$ ein euklidischer/unitärer Vektorraum,
    $\basis{B} = (v_1, v_2, \ldots)$ eine endliche oder abzählbar unendliche,
    linear unabhängige Teilmenge von $V$.
    Dann ist $\basis{E} = (e_1, e_2, \ldots)$ von derselben Mächtigkeit
    folgendermaßen definiert: \\
    1.\quad $e_1 =$ \fracsize{$\frac{v_1}{\norm{v_1}}$} \qquad\qquad
    2.\quad $e_k =$ \fracsize{$\frac{x_k}{\norm{x_k}}$} mit
    $x_k = v_k - \sum_{i=1}^{k-1} \innerproduct{v_k, e_i}e_i$
\end{Prozedur}

\begin{Satz}{\name{Gram}-\name{Schmidt}}
    Seien $\basis{E}$ definiert wie oben,
    $\basis{B}_k = (v_1, \ldots, v_k)$, $U_k = \aufspann{\basis{B}_k}$
    für ein $k \le |\basis{B}|$ und
    $\basis{B}$ Basis von $V$.
    Dann ist $\basis{E}_k = (e_1, \ldots, e_k)$ eine ONB von $U_k$ und
    $\basis{E}$ ist eine ONB von $V$. \\
    Die Basiswechselmatrix $M_k = \matrixm_{\id_V}(\basis{E}_k, \basis{B}_k)$
    ist eine obere Dreiecksmatrix mit $\det(M_k) > 0$.
\end{Satz}

\begin{Satz}{Koef"|fizienten eines Vektors bzgl. ONB} \\
    Seien $\basis{E} = (e_1, e_2, \ldots)$ eine ONB von $V$ und $x \in V$.
    Dann ist $x = \innerproduct{x, e_1}e_1 + \innerproduct{x, e_2}e_2 + \cdots$.
\end{Satz}

\begin{Satz}{Skalarprodukt zweier Vektoren bzgl. ONB}
    Seien $x, y \in V$ mit $x = \alpha_1 e_1 + \alpha_2 e_2 + \cdots$
    und $y = \beta_1 e_1 + \beta_2 e_2 + \cdots$.
    Dann ist
    $\innerproduct{x, y} = \alpha_1 \kk{\beta_1} + \alpha_2 \kk{\beta_2} + \cdots$.
\end{Satz}

\pagebreak

\begin{Satz}{Orthogonalisierung} \\
    Seien $V$ ein euklidischer/unitärer Vektorraum,
    $\basis{B} = (v_1, v_2, \ldots)$ eine linear unabhängige Teilmenge von $V$.
    Dann ist $\basis{E} = (x_1, x_2, \ldots)$ ein orthogonales System in $V$,
    wobei: \\
    1.\quad $x_1 = v_1$ \qquad\qquad
    2.\quad $x_k = v_k - \sum_{i=1}^{k-1}$
    \fracsize{$\frac{\innerproduct{v_k, x_i}}{\norm{x_i}^2}$} $x_i$
\end{Satz}

\begin{xDef}{\name{Fourier}koef"|fizienten}%
{Fourierkoeffizienten@\name{Fourier}koef"|fizienten}
    Seien $V$ Vektorraum mit Skalarprodukt, $\basis{B}$ orthonormales
    System und $x \in V$.
    Dann heißen die Skalare $\innerproduct{x, b}$ mit $b \in \basis{B}$
    \xbegriff{\name{Fourier}koef"|fizienten}%
    {Fourierkoeffizienten@\name{Fourier}koef"|fizienten}
    von $x$ bzgl. $\basis{B}$.
\end{xDef}

\begin{xDef}{\name{Schauder}basis}{Schauderbasis@\name{Schauder}basis}
    Ein abzählbares orthonormales System mit der Eigenschaft, dass sich jeder
    Vektor als unendliche Linearkombination schreiben lässt, heißt
    \xbegriff{Schauderbasis}{Schauderbasis@\name{Schauder}basis}.
\end{xDef}

\begin{Satz}{$M^\bot$ als Komplement}
    Seien $V$ ein Vektorraum mit Skalarprodukt und $M, N \ur V$. \\
    Ist $M \orth N$, dann ist $M \cap N = (0)$ und daher die Summe $M + N$
    direkt (insbesondere ist $M + M^\bot$ direkt).
    Ist $M$ endlich-dimensional, dann ist $V = M \oplus M^\bot$. \\
    Der Unterraum $M^\bot$ ist das eindeutig bestimmte
    \begriff{orthogonale Komplement} von $M$ in $V$. \\
    Jeder zu $M$ orthogonale Unterraum von $V$ ist in $M^\bot$ enthalten.
\end{Satz}

\begin{Bem}
    Ein Unterraum kann also viele Komplemente haben, aber nur ein orthogonales
    Komplement.
\end{Bem}

\begin{Kor}
    Seien $V$ ein euklidischer/unitärer Vektorraum,
    $W \ur V$ endlich-dimensional mit ONB $(e_1, ..., e_k)$ und $y \in V$.
    Dann gibt es genau ein $z \in W^\bot$ mit
    $y = \sum_{i=1}^k \innerproduct{y, e_i}e_i + z$. \\
    Der Vektor $y_1 = y - z = \sum_{i=1}^k \innerproduct{y, e_i}e_i$ ist der
    eindeutig bestimmte Vektor von $W$, der $y$ am nächsten ist, d.\,h.
    $\forall_{u \in W}\; \norm{y - y_1} \le \norm{y - u}$.
\end{Kor}

\begin{Kor}
    Seien $V$ endlich-dimensional und $(e_1, \ldots, e_k)$ ein orthonormales
    System in $V$. \\
    Dann kann es zu einer ONB $(e_1, \ldots, e_k, e_{k+1}, \ldots, e_n)$
    von $V$ ergänzt werden und $(e_{k+1}, \ldots, e_n)$ ist ONB vom
    orthogonalen Komplement zu $W = \aufspann{e_1, \ldots, e_k}$.
\end{Kor}

\begin{Kor}
    Es gilt $\dim_K(V) = \dim_K(W) + \dim_K(W^\bot)$ für alle Unterräume
    $W$ von $V$.
\end{Kor}

\begin{Kor}
    Es gilt $(M^\bot)^\bot = M$ für jeden Unterraum $M$ von $V$.
\end{Kor}

\section{%
    Euklidische Vektorräume, orthogonale Abbildungen%
}

\begin{Def}{orthogonale Abbildung} \\
    Seien $V, W$ euklidische Vektorräume und $f: V \rightarrow W$ eine
    $\real$-lineare Abbildung. \\
    Dann ist $f$ eine \begriff{orthogonale Abbildung}, falls
    $\innerproduct{f(x), f(y)} = \innerproduct{x, y}$ für alle $x, y \in V$.
\end{Def}

\begin{Def}{Isometrie}
    Ein orthogonaler Isomorphismus heißt auch \begriff{Isometrie}. \\
    Euklidische Vektorräume heißen \begriff{isometrisch}, falls es eine
    Isometrie zwischen ihnen gibt.
\end{Def}

\begin{Satz}{äquivalente Aussagen}
    Seien $V$ ein euklidischer Vektorraum endlicher oder abzählbar unendlicher
    Dimension, $W$ ein euklidischer Vektorraum und $f: V \rightarrow W$ ein
    Homomorphismus.
    Dann sind folgende Aussagen äquivalent: \\
    1. $f$ ist orthogonale Abbildung. \qquad
    2. Für $x \in V$ gilt
    $\norm{x} = 1 \;\Rightarrow\; \norm{f(x)} = 1$. \\
    3. Für $x \in V$ gilt $\norm{x} = \norm{f(x)}$. \qquad
    4. Für jedes orthonormale System $\basis{E} = (e_1, e_2, \ldots)$ in $V$
    ist $\basis{E}^f = (f(e_1), f(e_2), \ldots)$ ebenfalls eines in $W$. \\
    5. Es gibt eine ONB $\basis{B}$ von $V$, sodass $\basis{B}^f$
    Orthonormalsystem ist.
\end{Satz}

\begin{Satz}{orthogonale Abbildungen injektiv}
    Eine orthogonale Abbildung ist injektiv. \\
    Ist insbesondere $\dim_\real(V) = \dim_\real(W)$ und $f: V \rightarrow W$
    orthogonal, so ist $f$ eine Isometrie.
\end{Satz}

\begin{Def}{orthogonale Gruppe}
    Die Menge der Isometrien eines euklidischen Vektorraum in sich ist eine
    Untergruppe der linearen Gruppe $\GL_\real(V)$ und wird
    \begriff{orthogonale Gruppe} $O_\real(V)$ genannt. \\
    Für $V = \real^n$ mit dem natürlichen Skalarprodukt ist $O_n(\real)$
    die Menge der reellen, orthogonalen $n \times n$-Matrizen.
\end{Def}

\begin{Satz}{Abbildung auf den $\real^n$}
    Seien $V$ ein euklidischer Vektorraum, $\basis{B} = (x_1, \ldots, x_n)$
    eine ONB von $V$ und
    $f: V \rightarrow \real^n$,
    $\sum_{i=1}^n \alpha_i x_i \mapsto (\alpha_1, \ldots, \alpha_n)$. \\
    Dann ist $f$ eine Isometrie sowie $V$ und $\real^n$ isometrisch.
    $f(\basis{B})$ ist die natürliche Basis des $\real^n$.
\end{Satz}

\begin{Def}{orthogonale Matrix}
    Eine invertierbare Matrix $A$ mit $A^{-1} = A^t$ nennt man
    \begriff{orthogonal}.
\end{Def}

\begin{Lemma}{Matrizen und ONB von $\real^n$}
    Die Spalten- bzw. Zeilenvektoren einer reellen Matrix $A \in M_n(\real)$
    bilden genau dann eine ONB von $\real^n$, wenn
    $A$ orthogonal ist.
\end{Lemma}

\begin{Kor}
    Seien $A \in M_n(\real)$ und
    $f_A: \real^n \rightarrow \real^n$, $f_A(x) = Ax$.
    Dann ist die natürliche Basis von $\real^n$ orthonormal und
    $f_A$ ist orthogonal genau dann, wenn $A$ orthogonal ist. \\
    So ist $O_\real(V) \cong O_n(\real) =
    \{A \in \GL_n(\real) \;|\; A^{-1} = A^t\}$.
\end{Kor}

\begin{Kor}
    Seien $f$ ein Endomorphismus des euklidischen Vektorraums $V$,
    $\basis{B}$ eine ONB von $V$ und $A = \hommatrix{f}{B}{B}$.
    Dann ist $f$ orthogonal genau dann, wenn $A$ orthogonal ist.
\end{Kor}

\begin{Kor}
    Sei $\basis{E}$ eine endliche ONB des euklidischen Vektorraums $V$.
    Dann ist eine Basis $\basis{B}$ von $V$ orthonormal genau dann, wenn
    $A = \hommatrix{\id_V}{E}{B}$ orthogonal ist.
\end{Kor}

\begin{Kor}
    Die Determinante einer orthogonalen Abbildung $f_A$ ist $\pm 1$.
\end{Kor}

\section{%
    Hauptachsentheorem%
}

\begin{Def}{symmetrische Matrix}
    Eine Matrix $A$ mit $A^t = A$ nennt man \begriff{symmetrisch}.
\end{Def}

\begin{Def}{orthogonal-äquivalent}
    Zwei Endomorphismen $f$ und $g$ eines euklidischen Vektorraums $V$
    heißen \begriff{orthogonal-äquivalent}, falls es einen orthogonalen
    Automorphismus $p$ von $V$ mit $g = p^{-1} \circ f \circ p$ gibt.
    Analog sind orthogonal-äquivalente quadratische reelle Matrizen definiert.
\end{Def}

\begin{Satz}{Hauptachsentheorem 1}
    Symmetrische reelle Matrizen sind diagonalisierbar.
\end{Satz}

\begin{Satz}{Hauptachsentheorem 2}
    Jede reelle symmetrische Matrix ist orthogonal-äquivalent zu einer
    Diagonalmatrix.
\end{Satz}

\begin{Satz}{$x^t y = 0$ bei symmetrischen Matrizen}
    Seien $A \in M_n(K)$ eine symmetrische Matrix und $\lambda, \mu \in K$
    verschiedene Eigenwerte ($\lambda \not= \mu$) mit Eigenvektoren
    $x =$ \matrixsize{$\begin{pmatrix}x_1 \\ \vdots \\ x_n\end{pmatrix}$} bzw.
    $y =$ \matrixsize{$\begin{pmatrix}y_1 \\ \vdots \\ y_n\end{pmatrix}$}. \\
    Dann ist $x^t y = \sum_{i=1}^n x_i y_i = 0$.
\end{Satz}

\begin{Kor}
    Sei $A \in M_n(\real)$ eine symmetrische reelle Matrix.
    Dann sind Eigenvektoren von $A$ zu verschiedenen Eigenwerten orthogonal
    bzgl. des Standardskalarprodukts des $\real^n$. \\
    Eigenräume zu verschiedenen Eigenwerten sind paarweise orthogonal.
\end{Kor}

\begin{Satz}{komplexes Konjugat als EV/EW}
    Seien $A \in M_n(\real)$ und $\lambda \in \complex$ ein komplexer Eigenwert
    mit Eigenvektor $x \in \complex^n$ von $A$.
    Dann ist $\kk{x}$ Eigenvektor von $A$ zum Eigenwert $\kk{\lambda}$.
\end{Satz}

\begin{Satz}{symmetrische, reelle Matrizen haben nur reelle EW} \\
    Die Eigenwerte symmetrischer reeller $n \times n$-Matrizen sind alle reell.
\end{Satz}

\begin{Kor}
    Das charakteristische Polynom einer symmetrischen reellen
    $n \times n$-Matrix zerfällt über den reellen Zahlen in Linearfaktoren.
\end{Kor}

\begin{Satz}{$\real^n$ besitzt eine Basis aus EV}
    Sei $A$ eine symmetrische reelle $n \times n$-Matrix. \\
    Dann besitzt $\real^n$ eine Basis aus Eigenvektoren.
\end{Satz}

\pagebreak

\begin{Prozedur}%
{Symmetrische reelle Matrix mit orthogonaler Matrix diagonalisieren} \\
    Sei $A$ eine symmetrische reelle $n \times n$-Matrix.
    \begin{enumerate}
        \item Man berechnet das charakteristische Polynom $\chi_A(t)$ von $A$.
        Es zerfällt in reelle Linearfaktoren
        $\chi_A(t) = (t - \lambda_1)^{\nu_1} \cdots (t - \lambda_k)^{\nu_k}$,
        wobei die $\lambda_i$ die paarweise verschiedenen Eigenwerte sind.

        \item Für jeden Eigenwert $\lambda_i$ berechnet man eine Basis
        $\basis{B}_i$ des zugehörigen Eigenraums $V_{\lambda_i}$ durch Lösen
        des LGS $(A - \lambda_i E_n)x = 0$.
        Für die Dimension gilt dann $\dim V_{\lambda_i} = \nu_i$ und die
        $V_{\lambda_i}$ sind paarweise orthogonal.

        \item Jede Basis $\basis{B}_i$ wird mithilfe des Verfahrens
        von \name{Gram}-\name{Schmidt} zu einer ONB $\basis{C}_i$ von
        $V_{\lambda_i}$ orthonormalisiert.

        \item $\basis{C} = \bigcup_{i=1}^k \basis{C}_i$ ist eine ONB von
        $\real^n$, da die $\basis{C}_i$ paarweise orthogonal sind.
        Die Basiswechselmatrix $P = \matrixm_{\id}(\basis{E}_n, \basis{C})$
        ist eine orthogonale Matrix, daher ist
        $P^{-1} = \matrixm_{\id}(\basis{C}, \basis{E}_n) = P^t$. \\
        Es gilt $P^{-1} A P = D$, wobei in
        $D = \diag\{\lambda_1, \ldots, \lambda_1, \lambda_2, \ldots,
        \lambda_2, \ldots, \lambda_k, \ldots, \lambda_k\}$ die $\lambda_i$
        jeweils $\nu_i$ oft vorkommen.
    \end{enumerate}
\end{Prozedur}

\section{%
    Unitäre Abb. und Hauptachsentheorem für normale Endom.%
}

\begin{Def}{unitäre Abbildung} \\
    Seien $V, W$ unitäre Vektorräume und $f: V \rightarrow W$ eine
    $\complex$-lineare Abbildung. \\
    Dann ist $f$ eine \begriff{unitäre Abbildung}, falls
    $\innerproduct{f(x), f(y)} = \innerproduct{x, y}$ für alle $x, y \in V$.
\end{Def}

\begin{Satz}{äquivalente Aussagen}
    Folgende Aussagen sind äquivalent: \\
    1. $f$ ist unitäre Abbildung. \qquad
    2. Für $x \in V$ gilt
    $\norm{x} = 1 \;\Rightarrow\; \norm{f(x)} = 1$. \\
    3. Für $x \in V$ gilt $\norm{x} = \norm{f(x)}$. \qquad
    4. Für jedes orthonormale System $\basis{E} = (e_1, e_2, \ldots)$ in $V$
    ist $\basis{E}^f = (f(e_1), f(e_2), \ldots)$ ebenfalls eines in $W$. \\
    5. Es gibt eine ONB $\basis{B}$ von $V$, sodass $\basis{B}^f$
    Orthonormalsystem ist.
\end{Satz}

\begin{Satz}{Aussagen über unitäre Abbildungen}
    Unitäre Abbildungen sind injektiv. \\
    Jeder $n$-dimen\-sionale unitäre Raum ist durch Wahl einer ONB isometrisch
    zum $\complex^n$ mit Standardskalarprodukt.
    Die Menge der unitären Isometrien von $V$ in sich bildet eine Untergruppe
    von $\Aut_\complex(V)$ und heißt \begriff{unitäre Gruppe}
    $U_\complex(V)$. \\
    Die \begriff{konjugiert komplexe Matrix} $\kk{A}$ einer Matrix
    $A = (\alpha_{ij}) \in M_n(\complex)$ ist $\kk{A} = (\kk{\alpha_{ij}})$.
\end{Satz}

\begin{Def}{Adjungierte Matrix}
    $A^\ast = \kk{A}^t$ ist die \begriff{adjungierte Matrix} von $A$.
\end{Def}

\begin{Lemma}{$^\ast$ als semilinearer $\complex$-Algebraantiautomorphismus}
    Seien $A, B \in M_n(\complex)$ und $\lambda \in \complex$. \\
    Dann gilt ${A^\ast}^\ast = A$, \qquad
    $(A + B)^\ast = A^\ast + B^\ast$, \qquad
    $(\lambda A)^\ast = \kk{\lambda} A^\ast$ \quad sowie \quad
    $(AB)^\ast = B^\ast A^\ast$.
\end{Lemma}

\begin{Def}{unitäre, hermitesche und normale Matrizen}
    Sei $A \in M_n(\complex)$. \\
    $A$ ist \begriff{unitär}, falls $A^{-1} = A^\ast$ \qquad
    ($\mathrel{\widehat{=}}$ orthogonale Matrix im Reellen). \\
    $A$ ist \begriff{hermitesch} oder \begriff{selbstadjungiert}, falls
    $A = A^\ast$ \qquad
    ($\mathrel{\widehat{=}}$ symmetrische Matrix im Reellen). \\
    $A$ ist \begriff{normal}, falls $A A^\ast = A^\ast A$.
\end{Def}

\begin{Lemma}{unitäre/hermitesche Matrizen normal}
    Unitäre und hermitesche Matrizen sind normal.
\end{Lemma}

\begin{Satz}{Matrizen und ONB von $\complex^n$}
    Die Spalten- bzw. Zeilenvektoren einer Matrix $A \in M_n(\complex)$ bilden
    genau dann eine ONB von $\complex^n$, wenn $A$ unitär ist.
\end{Satz}

\begin{Satz}{$f$ unitär $\Leftrightarrow \hommatrix{f}{B}{B}$ unitär}
    Seien $V$ endlich-dimensional, $\basis{B}$ eine ONB von $V$ und \\
    $f \in \End_\complex(V)$.
    Dann ist $f$ unitär genau dann, wenn $\hommatrix{f}{B}{B}$ unitär ist.
\end{Satz}

\begin{Def}{adjungierter Endomorphismus}
    Seien $f \in \End_\complex(V)$, $\basis{B} = (v_1, \ldots, v_n)$ eine
    ONB von $V$ und $A = (\alpha_{ij}) = \hommatrix{f}{B}{B}$.
    Dann heißt $f^\ast: V \rightarrow V$ definiert durch
    $f^\ast(v_j) = \sum_{i=1}^n \kk{\alpha_{ji}} v_i$ der
    \begriff{zu $f$ adjungierte Endomorphismus}.
    Es gilt $\hommatrix{f^\ast}{B}{B} = A^\ast =
    \left(\hommatrix{f}{B}{B}\right)^\ast$.
\end{Def}

\begin{Kor}
    Seien $f \in \End_\complex(V)$ und $x, y \in V$.
    Dann ist $\innerproduct{f(x), y} = \innerproduct{x, f^\ast(y)}$.
\end{Kor}

\begin{Def}{hermitesche und normale Endomorphismen}
    Sei $f \in \End_\complex(V)$.
    Dann ist $f$ \begriff{hermitesch}, falls $f = f^\ast$,
    und \begriff{normal}, falls $f \circ f^\ast = f^\ast \circ f$.
    Hermitesche/normale Endomorphismen sind genau die Endomorphismen, deren
    Matrizen bzgl. einer ONB hermitesch/normal sind.
\end{Def}

\begin{Kor}
    Seien $f \in \End_\complex(V)$ hermitesch und $x, y \in V$.
    Dann ist $\innerproduct{f(x), y} = \innerproduct{x, f(y)}$.
\end{Kor}

\begin{Satz}%
{$f$ normal $\Leftrightarrow \innerproduct{f(x), f(y)} = \innerproduct{f^\ast(x), f^\ast(y)}$}
    Sei $f \in \End_\complex(V)$. \\
    Dann ist $f$ normal genau dann, wenn
    $\innerproduct{f(x), f(y)} = \innerproduct{f^\ast(x), f^\ast(y)}$ für alle $x, y \in V$ gilt.
\end{Satz}

\begin{Kor}
    Ist $f$ normal, dann gilt insbesondere $\norm{f(x)} = \norm{f^\ast(x)}$
    und $\ker f = \ker f^\ast$ für alle $x \in V$.
    Ist $x \in \ker f$, dann ist $\innerproduct{f^\ast(x), f^\ast(x)} =
    \innerproduct{f(x), f(x)} = 0$ und daher $f^\ast(x) = 0$.
\end{Kor}

\begin{Satz}{Eigenvektoren des adjungierten Endomorphismus}
    Seien $f \in \End_\complex(V)$ normal und \\
    $x \in V$ ein EV von $f$ zum EW $\lambda \in \complex$.
    Dann ist $f^\ast(x) = \kk{\lambda}x$, d.\,h. $x$ ist EV von $f^\ast$ zum
    EW $\kk{\lambda}$.
    Insbesondere ist $V_\lambda(f) = V_{\kk{\lambda}}(f^\ast)$.
\end{Satz}

\begin{Satz}{Hauptachsentheorem für normale Abbildungen}
    Seien $V$ ein endlich-dimensionaler uni"-tärer Raum und
    $f \in \End_\complex(V)$.
    Ist $f$ normal, so besitzt $V$ eine ONB aus Eigenvektoren von $f$.
    Sind $\basis{B}$ irgendeine ONB von $V$ und $A = \hommatrix{f}{B}{B}$, so
    ist $A$ \begriff{unitär-äquivalent} zu einer Diagonalmatrix,
    d.\,h. es gibt eine
    unitäre Matrix $P$, sodass $P^{-1} A P$ Diagonalmatrix ist.
\end{Satz}

\begin{Satz}{Spezialfall hermitesche Endomorphismen}
    Sei $f \in \End_\complex(V)$ hermitesch.
    Dann sind alle Eigenwerte von $f$ reell und $V$ hat eine ONB bestehend
    aus Eigenvektoren von $f$. \\
    Ist $A = \hommatrix{f}{B}{B}$ bzgl. einer ONB $\basis{B}$ von $V$, so ist
    $A$ unitär-äquivalent zu einer reellen Diagonalmatrix.
\end{Satz}

\pagebreak
