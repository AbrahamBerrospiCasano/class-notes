\chapter{%
    Analytische Geometrie der Ebene und des Raums%
}

\section{%
    Vektoren in der Ebene und im Raum%
}

\begin{Def}{Vektorgrößen}
    Viele physikalische Größen (wie Kraft oder Geschwindigkeit) haben nicht nur
    einen \begriff{Betrag}, sondern auch eine \begriff{Richtung}.
    Solche Größen nennt man \begriff{Vektorgrößen}.
    Sie können zusammengesetzt werden, sind frei beweglich und werden durch
    einen Pfeil in der Ebene oder im Raum dargestellt, wobei die Länge
    den Betrag der Größe angibt.
    Pfeile derselben Länge und Richtung, die sich nur im Anfangspunkt
    unterscheiden, repräsentieren denselben Vektor.
    Im Folgenden sei $\eukl = \euklebene \text{ oder } \euklraum$
    die Menge der Punkte der Ebene oder des Raums.
\end{Def}

\begin{Notation}
    $d(A, B)$ Abstand der Punkte $A$ und $B$, \qquad
    $(A, B)$ Verbindungsgerade durch $A$, $B$, \qquad
    Geraden $g, h$ sind parallel, falls sie in einer Ebene liegen und keinen
    Punkt gemeinsam haben oder falls sie gleich sind, \qquad
    $\strecke{AB}$ gerichtete Strecke mit Anfangspunkt $A$ und Endpunkt $B$
\end{Notation}

\begin{Def}{verschiebungsgleich}
    Die gerichteten Strecken $\strecke{AB}$ und $\strecke{CD}$ heißen
    \begriff{verschiebungsgleich}, falls es eine Parallelverschiebung gibt,
    die $A$ in $C$ und $B$ in $D$ überführt.
\end{Def}

\begin{Lemma}{Äquivalenzrelation der Translationen}
    Die Relation "`verschiebungsgleich sein"' ist eine Äquivalenzrelation
    auf der Menge der gerichteten Strecken in $\eukl$.
\end{Lemma}

\begin{Def}{Vektoren}
    Die Äquivalenzklassen der Relation "`verschiebungsgleich sein"'
    auf der Menge der gerichteten Strecken in $\eukl$ heißen
    \begriff{Vektoren}.
    Ist $\strecke{AB}$ eine gerichtete Strecke, so wird die Äquivalenzklasse
    $[\strecke{AB}] = \{\strecke{CD} \;|\; \strecke{CD} \text{ und }
    \strecke{ AB } \text{ sind verschiebungsgleich}\}$ mit $\vektor{AB}$
    oder $\vec{a}$ bezeichnet. \\
    Die Länge $|\vec{a}|$ des Vektors $\vec{a} = \vektor{AB}$ ist durch
    $|\vec{a}| = d(A, B)$ definiert. \\
    Im Folgenden sei $V$ die Menge der Vektoren in $\eukl$.
\end{Def}

\begin{Def}{Nullvektor}
    Sei $A$ ein Punkt von $\eukl$.
    Dann ist $\vec{o} = \vektor{AA}$ der \begriff{Nullvektor}.
    Er hat die Länge $0$ und es ist $\vec{o} = \vektor{BB}$ für alle
    $B \in \eukl$.
    Die zugehörige Translation ist die Identität $\id_\eukl$.
\end{Def}

\begin{Def}{Addition von Vektoren}
    Seien $\vec{a}, \vec{b} \in V$.
    Wähle $O \in \eukl$ und $A, C \in \eukl$, sodass $\vec{a} = \vektor{OA}$
    und $\vec{b} = \vektor{AC}$ ist.
    Die Summe $\vec{a} + \vec{b} \in V$ ist dann definiert als
    $\vec{a} + \vec{b} = \vektor{OC} \in V$.
\end{Def}

\begin{Satz}{Rechenregeln für die Vektoraddition}
    Die Addition geometrischer Vektoren ist assoziativ.
    Es gibt ein Nullelement $\vec{o}$ und zu jedem Element
    $\vec{a} \in V$ ein additiv Inverses $-\vec{a} \in V$.
    Außerdem ist die Addition kommutativ.
\end{Satz}

\begin{Def}{skalare Multiplikation}
    Seien $\vec{a} \in V$ und $\lambda \in \real$.
    Dann ist $\lambda a \in V$ der Vektor, der die Länge
    $|\lambda| |\vec{a}|$ hat und die dieselbe (bzw. entgegengesetzte)
    Richtung wie $\vec{a}$ hat, wenn $\lambda > 0$ (bzw. $\lambda < 0$)
    ist.
    Ist $\lambda = 0$, so ist
    $\lambda \vec{a} = 0 \cdot \vec{a} = \vec{o}$ der Nullvektor.
\end{Def}

\begin{Satz}{Rechenregeln für die skalare Multiplikation}
    $1 \in \real$ ist ein Einselement bzgl. der skalaren Multiplikation.
    Die skalare Multiplikation ist assoziativ und skalar sowie vektoriell
    distributiv über der Addition von Skalaren.
\end{Satz}

\begin{Notation}
    Man kann eine Basis von $\euklebene$ oder $\euklraum$ wählen.
    Ist $\vec{a} = \lambda \vektor{n_1} + \mu \vektor{n_2} \in V$
    eindeutige Linearkombination der linear unabhängigen Vektoren
    $\vektor{n_1}$ und $\vektor{n_2}$ in $\euklebene$, dann schreibt man
    oft $\vec{a} = (\lambda, \mu)$.
    Analog schreibt man in $\euklraum$ dann $\vec{a} = (\lambda, \mu, \nu)$.
\end{Notation}

\section{%
    Die euklidische Ebene%
}

\begin{Bem}
    Im Folgenden sei $V$ die Menge der Vektoren in $\euklebene$,
    $\vec{a}, \vec{b} \in V$ sowie
    $\varphi = \sphericalangle(\strecke{OA}, \strecke{OB})$ der Winkel zwischen
    den Strecken $\strecke{OA}$ und $\strecke{OB}$.
\end{Bem}

\pagebreak

\begin{Def}{Skalarprodukt}
    $\vec{a} \vec{b} = |\vec{a}| |\vec{b}| \cos \varphi \in \real$
    ist das \begriff{Skalarprodukt} von $\vec{a}$ und $\vec{b}$.
\end{Def}

\begin{Satz}{Rechenregeln für das Skalarprodukt}
    Das Skalarprodukt ist i.\,A. nicht assoziativ.
    Ist $\vec{a} = \vec{o}$ oder $\vec{b} = \vec{o}$, dann ist
    $\vec{a} \vec{b} = 0$.
    Das Skalarprodukt ist distributiv über der Vektoraddition und
    es gilt $\vec{a} (\lambda \vec{b}) =
    (\lambda \vec{a}) \vec{b} = \lambda (\vec{a} \vec{b})$.
\end{Satz}

\begin{Def}{orthogonal}
    Seien $\vec{a}, \vec{b} \in V$ mit
    $\vec{a}, \vec{b} \not= \vec{o}$. \\
    Dann ist $\vec{a} \orth \vec{b}$, falls $\vec{a} \vec{b} = 0$,
    d.\,h. falls die Vektoren \begriff{senkrecht} aufeinander stehen. \\
    Es gilt $\vec{o} \orth \vec{a}$ für jeden Vektor $\vec{a} \in V$
    sowie $\vec{a} \vec{a} = \vec{a}^2 =
    |\vec{a}| |\vec{a}| \cos 0 = |\vec{a}|^2$.
\end{Def}

\begin{Def}{Orthogonalbasis}
    Seien $\vektor{n_1}, \vektor{n_2} \in V$ mit
    $\vektor{n_1}, \vektor{n_2} \not= \vec{o}$. \\
    Ist $\vektor{n_1} \orth \vektor{n_2}$, dann ist
    $\basis{B} = \{\vektor{n_1}, \vektor{n_2}\}$ eine Basis von $V$,
    eine \begriff{Orthogonalbasis}. \\
    Ist zusätzlich $|\vektor{n_1}| = |\vektor{n_2}| = 1$, so heißt
    $\basis{B}$ \begriff{Orthonormalbasis (ONB)} von $V$.
\end{Def}

\begin{Lemma}{Skalarprodukt komponentenweise}
    Seien $\{\vektor{n_1}, \vektor{n_2}\}$ ONB und $\vec{a} = (a_1, a_2)$,
    $\vec{b} = (b_1, b_2)$ bzgl. dieser Basis.
    Dann ist $\vec{a} \vec{b} = a_1 b_1 + a_2 b_2$.
    Außerdem ist
    $|\vec{a}| = \sqrt{\vec{a} \vec{a}} = \sqrt{a_1^2 + a_2^2}$.
\end{Lemma}

\begin{Def}{Gerade}
    Seien $\vec{a}, \vec{b} \in V$. \\
    Die \begriff{Gerade} durch $\vec{a}$ in Richtung $\vec{b}$ ist die Menge
    $g = \{\vec{x} \;|\; \vec{x} = \vec{a} + \lambda \vec{b},\;
    \lambda \in \real\}$. \\
    \begriff{Parameterdarstellung}:
    $g = \{(x, y) \in V \;|\; x = x_1 + \lambda x_2,\; y = y_1 + \lambda y_2,\;
    \lambda \in \real\}$ mit $\vec{a} = (x_1, y_1)$, $\vec{b} = (x_2, y_2)$
    bzgl. einer ONB $\{\vektor{n_1}, \vektor{n_2}\}$ von $\euklebene$ \\
    \xbegriff{\name{Hesse}sche Normalform}%
    {Hessesche Normalform@\name{Hesse}sche Normalform}:
    $\vec{x} \vec{a} = d$, wobei $d \in \real$ und $\vec{a}$ senkrecht zur
    Gerade und $|\vec{a}| = 1$ ist, \\
    alternativ $ax + by = d$ mit $\vec{a} = (a, b)$ und $\vec{x} = (x, y)$.
\end{Def}

\begin{Satz}{\name{Hesse}sche Normalform}
    Zu jeder Geraden $g$ in $\euklebene$ existieren $a, b, d \in \real$,
    sodass \\
    $g = \{(x, y) \in V \;|\; ax + by = d\}$ ist, wobei $a, b \not= 0$.
    Die Konstanten $a, b, d$ sind bis auf einen gemeinsamen Faktor eindeutig
    bestimmt.
    Ist $\sqrt{a^2 + b^2} = 1$, so ist $|d|$ der Abstand von $g$ zum Ursprung.
    Andernfalls ist $|d|$ gleich diesem Abstand multipliziert
    mit $\sqrt{a^2 + b^2}$.
\end{Satz}

\begin{Kor}
    Seien $ax + by = d$ \name{Hesse}sche Normalform der Gerade $g$ mit
    $\sqrt{a^2 + b^2} = 1$ und $P = (x_0, y_0) \in V$ ein Punkt.
    Dann ist $e = |ax_0 + by_0 - d|$ der Abstand von $g$ zu $P$.
\end{Kor}

\begin{Satz}{Schnittpunkt zweier Geraden}
    Der Schnittpunkt $P = (x, y)$ zweier Geraden $g_1, g_2$ mit den Gleichungen
    $g_1: a_1 x + b_1 y = d_1$ und $g_2: a_2 x + b_2 y = d_2$ ist
    die Lösungsgesamtheit des LGS dieser zwei Gleichungen, falls Lösungen
    existieren.
    Andernfalls sind $g_1$ und $g_2$ parallel.
\end{Satz}

\section{%
    Der euklidische Raum%
}


\begin{Bem}
    Im Folgenden sei $V$ die Menge der Vektoren in $\euklraum$ und
    $\vec{a}, \vec{b} \in V$.
    In $\euklraum$ kann man ebenfalls ein Skalarprodukt analog zu $\euklebene$
    definieren, dieses erfüllt dann dieselben Rechenregeln. \\
    $\basis{B} = \{\vektor{n_1}, \vektor{n_2}, \vektor{n_3}\}$ heißt analog
    Orthogonalbasis von $V$, falls
    $\vektor{o} \not= \vektor{n_1}, \vektor{n_2}, \vektor{n_3} \in V$ sowie
    $\vektor{n_1} \orth \vektor{n_2}$, $\vektor{n_1} \orth \vektor{n_3}$ und
    $\vektor{n_2} \orth \vektor{n_3}$.
    Ist zusätzlich $|\vektor{n_1}| = |\vektor{n_2}| = |\vektor{n_3}| = 1$,
    so heißt $\basis{B}$ Orthonormalbasis (ONB). \\
    Ist $\vec{a} = (a_1, a_2, a_3)$, $\vec{b} = (b_1, b_2, b_3)$ bzgl. einer
    ONB von $V$, dann ist
    $\vec{a} \vec{b} = a_1 b_1 + a_2 b_2 + a_3 b_3$. \\
    Außerdem ist dann $|\vec{a}| = \sqrt{a_1^2 + a_2^2 + a_3^2}$.
\end{Bem}

\begin{Def}{Gerade im Raum}
    Seien $\vec{a}, \vec{b} \in V$.
    Die \begriff{Gerade} $g$ durch $\vec{a}$ in Richtung $\vec{b}$ ist die
    Menge $g = \{\vec{x} \;|\; \vec{x} = \vec{a} + \lambda \vec{b},\;
    \lambda \in \real\}$.
    Ist $\vec{a} = (x_1, y_1, z_1)$, $\vec{b} = (x_2, y_2, z_2)$ bzgl. einer
    ONB von $V$, dann ist die \begriff{Parameterdarstellung} von $g$ gegeben
    durch $g = \{(x, y, z) \in V \;|\; x = x_1 + \lambda x_2,\;
    y = y_1 + \lambda y_2,\; z = z_1 + \lambda z_2,\; \lambda \in \real\}$.
\end{Def}

\begin{Lemma}{parallele Geraden}
    Zwei Geraden sind parallel genau dann, wenn die Richtungsvektoren Vielfache
    voneinander sind.
\end{Lemma}

\pagebreak

\begin{Bem}
    Will man den Schnittpunkt zweier Geraden berechnen, so muss man ein LGS
    lösen.
    Dabei gibt es verschiedene Möglichkeiten:
    Entweder gibt es eine eindeutige Lösung (Schnittpunkt), unendlich
    viele Lösungen (Geraden sind gleich), keine Lösung und Geraden sind
    parallel oder keine Lösung und Geraden sind nicht parallel
    (\begriff{windschief}).
\end{Bem}

\begin{Def}{Ebene}
    Seien $P_i = (x_i, y_i, z_i)$ drei Punkte ($i = 1, 2, 3$), die nicht auf
    einer Geraden $g$ liegen.
    Dann ist die Ebene durch $P_1$, $P_2$, $P_3$ definiert durch \\
    $e = \{\vec{x} \;|\; \vec{x} = \vektor{x_1} +
    \lambda (\vektor{x_1} - \vektor{x_2}) +
    \mu (\vektor{x_1} - \vektor{x_3}),\; \lambda, \mu \in \real\}$, wobei
    $\vektor{x_i} = \vektor{OP_i}$ für $i = 1, 2, 3$ ist. \\
    Entsprechend ist die Parameterdarstellung von $e$ gegeben.
\end{Def}

\begin{Bem}
    Schneidet man zwei Ebenen, dann ist der Schnitt entweder eine Gerade
    (Schnittgerade), leer (parallele Ebenen) oder die Ebene selbst (Ebenen
    sind gleich).
\end{Bem}

\begin{Satz}{Schnitt von Ebenen}
    Zwei verschiedene Ebenen schneiden sich entweder in einer Geraden oder gar
    nicht.
    Zwei Ebenen sind parallel genau dann, wenn die Richtungsvektoren
    der Ebenen dieselbe Ebene aufspannen.
    Eine Ebene geht durch den Ursprung genau dann, wenn der Aufpunkt der
    Ursprung ist.
\end{Satz}

\begin{Bem}
    \xbegriff{\name{Hesse}sche Normalform}%
    {Hessesche Normalform@\name{Hesse}sche Normalform}: $e: ax + by + cz = d$
    mit $\sqrt{a^2 + b^2 + c^2} = 1$, $|d|$ ist der Abstand der Ebene zum
    Ursprung
\end{Bem}

\section{%
    Das vektorielle Produkt%
}

\begin{Def}{Vektorprodukt}
    Seien $\vec{a} = (a_1, a_2, a_3)$, $\vec{b} = (b_1, b_2, b_3)$ bzgl.
    einer ONB.
    Das Vektorprodukt $\vec{a} \times \vec{b}$ von $\vec{a}$ und $\vec{b}$ ist
    der Vektor $\vec{c} = (c_1, c_2, c_3)$ mit
    $c_1 = a_2 b_3 - a_3 b_2$, $c_2 = a_3 b_1 - a_1 b_3$ und
    $c_3 = a_1 b_2 - a_2 b_1$.
\end{Def}

\begin{Satz}{Rechenregeln für das Vektorprodukt}
    Seien $\vec{a}, \vec{b}, \vec{c}, \vec{d} \in V$ und $r \in \real$.
    Dann ist \\
    $\vec{a} \times \vec{b} = -\vec{b} \times \vec{a}$, \qquad
    $(\vec{a} + \vec{b}) \times \vec{c} =
    \vec{a} \times \vec{c} + \vec{b} \times \vec{c}$, \qquad
    $r(\vec{a} \times \vec{b}) = (r\vec{a}) \times \vec{b} =
    \vec{a} \times (r\vec{b})$, \\
    $\vec{a} \times \vec{b} = \vec{o} \;\Leftrightarrow\; \vec{a}, \vec{b}$
    sind linear abhängig, \qquad
    $\vec{a} (\vec{a} \times \vec{b}) = 0 =
    \vec{b} (\vec{a} \times \vec{b})$, \\
    $\vec{a} (\vec{b} \times \vec{c}) = \vec{b} (\vec{c} \times \vec{a}) =
    \vec{c} (\vec{a} \times \vec{b})$, \qquad
    $\vec{a} \times (\vec{b} \times \vec{c}) =
    (\vec{a} \vec{c}) \vec{b} - (\vec{a} \vec{b}) \vec{c}$, \\
    $\vec{a} \times (\vec{b} \times \vec{c}) +
    \vec{b} \times (\vec{c} \times \vec{a}) +
    \vec{c} \times (\vec{a} \times \vec{b}) = \vec{o}$, \qquad
    $(\vec{a} \times \vec{b})(\vec{c} \times \vec{d}) =
    (\vec{a} \vec{c}) (\vec{b} \vec{d}) - (\vec{a} \vec{d}) (\vec{b} \vec{c})$.
\end{Satz}

\begin{Satz}{Vektorprodukt}
    Seien $\vec{a}, \vec{b} \in V$ mit $\vec{a}, \vec{b} \not= 0$.
    Dann ist $\vec{a} \times \vec{b} \in V$ ein Vektor senkrecht zu $\vec{a}$
    und $\vec{b}$, sodass $(\vec{a}, \vec{b}, \vec{a} \times \vec{b})$ ein
    Rechtssystem bilden, falls $(\vektor{n_1}, \vektor{n_2}, \vektor{n_3})$ ein
    Rechtssystem bilden.
    Dabei gilt $|\vec{a} \times \vec{b}| = |\vec{a}| |\vec{b}| \sin \varphi$,
    wobei $\varphi = \sphericalangle(\vec{a}, \vec{b})$ der (gerichtete) Winkel
    zwischen $\vec{a}$ und $\vec{b}$ ist.
    Außerdem ist $|\vec{a} \times \vec{b}|$ der Flächeninhalt des von
    $\vec{a}$ und $\vec{b}$ aufgespannten Parallelogramms.
\end{Satz}

\begin{Kor}
    Sei $e = \{\vec{x} \;|\; \vec{x} = \vektor{x_1} +
    \lambda (\vektor{x_1} - \vektor{x_2}) +
    \mu (\vektor{x_1} - \vektor{x_3}),\; \lambda, \mu \in \real\}$ eine Ebene,
    die in Parameterform gegeben ist.
    Dann ist $\vec{a} = (\vektor{x_1} - \vektor{x_2}) \times
    (\vektor{x_1} - \vektor{x_3})$
    Normalenvektor der Ebene.
    Ist $\vec{b} = \vec{a} \cdot \frac{1}{|\vec{a}|}$ der normierte
    Normalenvektor, dann ist
    $e: \vec{x} \vec{b} = \vektor{x_1} \vec{b}$ die \name{Hesse}sche Normalform
    der Ebene $e$.
\end{Kor}

\pagebreak
