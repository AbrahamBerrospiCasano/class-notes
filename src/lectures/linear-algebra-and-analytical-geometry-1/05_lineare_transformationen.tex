\section{%
    Lineare Transformationen%
}

\subsection{%
    Grundlagen%
}

\begin{Def}{Homomorphismus}
    Seien $V, W$ $K$-Vektorräume.
    Eine Abbildung $f: V \rightarrow W$ heißt \\
    \xbegriff{$K$-line\-ar}{K-linear|see{Homomorphismus}}/%
    \xbegriff{lineare Transformation}%
    {lineare Transformation|see{Homomorphismus}}/%
    \xbegriff{Homomorphismus}{linear|see{Homomorphismus}}, falls für
    alle $x, y \in V$, $\lambda \in K$ gilt, dass
    $f(x + y) = f(x) + f(y)$ und $f(\lambda x) = \lambda f(x)$.
\end{Def}

\begin{Def}{Mono-/Epi-/Iso-/Endo-/Automorphismus}
    Ein Homomorphismus $f: V \rightarrow W$ heißt
    \begriff{Monomorphismus}, falls $f$ injektiv,
    \begriff{Epimorphismus}, falls $f$ surjektiv, und
    \begriff{Isomorphismus}, falls $f$ bijektiv ist.
    Ein Homomorphismus $f: V \rightarrow V$ heißt \begriff{Endomorphismus}
    von $V$ und \begriff{Automorphis\-mus} von $V$, falls $f$ bijektiv ist.
\end{Def}

\begin{Def}{isomorph}
    Gibt es einen Isomorphismus zwischen den $K$-Vektorräumen $V$ und $W$,
    so heißen $V$ und $W$ \begriff{isomorph}.
    Man schreibt dann $V \cong W$.
\end{Def}

\begin{Satz}{Umkehrabbildung als Isomorphismus}
    Sei $f: V \rightarrow W$ Isomorphismus. \\
    Dann ist $f^{-1}: W \rightarrow V$ ebenfalls ein Isomorphismus.
\end{Satz}

\begin{Satz}{Komposition von Homomorphismen}
    Die Komposition von Mono-/Epi-/Iso-/Homomor\-phismen ist ebenfalls ein
    Mono-/Epi-/Iso-/Homomorphismus.
\end{Satz}

\begin{Satz}{Erzeugendensystem}
    Ein Homomorphismus $f: V \rightarrow W$ ist vollständig durch die Werte
    auf einem Erzeugendensystem bestimmt.
    Es gilt also:
    Seien $f, g: V \rightarrow W$ Homomorphismen und $T$
    Erzeugendensystem von $V$.
    Gilt $f(t) = g(t)$ für alle $t \in T$, dann ist $f = g$.
\end{Satz}

\begin{Satz}{Basis}
    Seien $\basis{B} = (v_1, \ldots, v_n)$ eine Basis von $V$
    und $w_1, \ldots, w_n$ (nicht notwendig verschiedene) Vektoren aus $W$.
    Dann gibt es genau eine lineare Transformation $T: V \rightarrow W$
    mit $T(v_i) = w_i$ für $i = 1, \ldots, n$.
    Es gilt $T\left(\sum_{i=1}^n \lambda_i v_i\right) =
    \sum_{i=1}^n \lambda_i w_i$ für alle
    $\lambda_1, \ldots, \lambda_n \in K$. \\
    (erweiterbar auf unendlich-dimensionale Vektorräume)
\end{Satz}

\begin{Satz}{$\im f$ Unterraum}
    Sei $f: V \rightarrow W$ Homomorphismus.
    Dann ist $\im f$ Unterraum von $W$.
\end{Satz}

\begin{Satz}{$f(\basis{B})$ erzeugt $\im f$}
    Seien $f: V \rightarrow W$ Homomorphismus und $\basis{B}$ Basis von $V$.
    Dann ist $\aufspann{f(\basis{B})} = \im f$, d.\,h. die Bilder
    $f(\basis{B}) = \{f(b) \;|\; b \in \basis{B}\}$ der Elemente
    einer beliebigen Basis von $V$ bilden ein Erzeugendensystem von $\im f$.
\end{Satz}

\begin{Satz}{Monomorphismus $\Leftrightarrow f(\basis{B})$ Basis von $\im f$}
    Sei $\basis{B}$ eine Basis von $V$ und $f: V \rightarrow W$
    Homomorphismus.
    Dann ist $f$ Monomorphismus genau dann, wenn $f(\basis{B})$ Basis von
    $\im f$ ist.
\end{Satz}

\begin{Kor}
    Sei $f: V \rightarrow W$ Monomorphismus, dann wird $f$ durch Einschränkung
    des Wertevorrats ein Isomorphismus ($f': V \rightarrow \im f$).
\end{Kor}

\begin{Kor}
    Ist $f: V \rightarrow W$ ein Isomorphismus, so ist $f(\basis{B})$ eine
    Basis von $W$.
    Insbesondere haben isomorphe Vektorräume dieselbe Dimension.
\end{Kor}

\begin{Lemma}{Pigeon-Hole-Principle}
    Seien $\mathfrak{M}, \mathfrak{N}$ endliche Mengen derselben Mächtigkeit
    und \\
    $f: \mathfrak{M} \rightarrow \mathfrak{N}$ Abbildung.
    Dann ist $f$ injektiv genau dann, wenn $f$ surjektiv ist.
\end{Lemma}

\begin{Satz}{gleichdimensionale Vektorräume}
    Seien $V, W$ endliche Vektorräume derselben Dimension
    und $f: V \rightarrow W$ Homomorphismus.
    Dann ist $f$ genau dann ein Isomorphismus, wenn $f$ ein Mono- oder
    Epimorphismus ist.
\end{Satz}

\begin{Satz}{isomorphe Vektorräume} \\
    Zwei $K$-Vektorräume sind isomorph genau dann, wenn sie dieselbe Dimension
    haben.
\end{Satz}

\begin{Satz}{Faktorräume}
    Seien $U \ur V$ und $W = V/U$. \\
    Dann ist die Abbildung $T: V \rightarrow W: v \mapsto \overline{v} = v + U$
    ein Epimorphismus.
\end{Satz}

\begin{Satz}{Isomorphismus bei Faktorräumen und Komplementen} \\
    Sei $W$ ein Komplement von $U \ur V$.
    Für $x \in V$ sei $w_x \in W$ das eindeutig bestimmte Element in
    $\overline{x}$.
    Dann ist $\varphi: V/U \rightarrow W: \overline{x} \mapsto w_x$ ein
    Isomorphismus.
\end{Satz}

\begin{Def}{Kern}
    Sei $f: V \rightarrow W$ Homomorphismus.
    Dann ist der \begriff{Kern} von $f$ \\
    $\ker f = f^{-1}(0) = \{v \in V \;|\; f(v) = 0_W\}$.
    Der Kern von $f$ ist ein Unterraum von $V$.
\end{Def}

\begin{Satz}{$f$ injektiv $\Leftrightarrow \ker f = (0)$}
    Sei $f: V \rightarrow W$ Homomorphismus. \\
    Dann ist $f$ injektiv genau dann, wenn $\ker f = (0)$ ist.
\end{Satz}

\subsection{%
    Matrizen%
}

\begin{xDef}{Matrix (eines Homomorphismus)}{Matrix!eines Homomorphismus}
    Seien $f: V \rightarrow W$ $K$-linear und
    $\basis{B} = (v_1, \ldots, v_n)$, \\
    $\basis{C} = (w_1, \ldots, w_m)$ Basen von $V$ bzw. $W$.
    Für $1 \le j \le n$ sei $f(v_j) = \sum_{i=1}^m \alpha_{ij} w_i$. \\
    Das Rechteckschema
    $\hommatrix{f}{C}{B} = (\alpha_{ij})_{ij} =$
    \matrixsize{$\begin{pmatrix}
        \alpha_{11} & \alpha_{12} & \cdots & \alpha_{1n} \\
        \alpha_{21} & \alpha_{22} & \cdots & \alpha_{2n} \\
        \vdots & \vdots & \ddots & \vdots \\
        \alpha_{m1} & \alpha_{m2} & \cdots & \alpha_{mn}
    \end{pmatrix}$} heißt \begriff{Matrix} der linearen Abbildung $f$
    bzgl. der Basen $\basis{B}$ und $\basis{C}$.
\end{xDef}

\begin{Lemma}{Umrechnung von Koef"|fizienten}
    Seien $f$ und $\hommatrix{f}{C}{B}$ wie oben und
    $x = \sum_{j=1}^n \lambda_j v_j$.
    Dann ist $f(x) = \sum_{i=1}^m \mu_i w_i$ mit
    $\mu_i = \sum_{j=1}^n \alpha_{ij} \lambda_j$.
\end{Lemma}

\begin{Bem}
    Man kann den $i$-ten Koef"|fizienten $\mu_i$ in
    $f(x) = \sum_{i=1}^m \mu_i w_i$ berechnen, indem man die $i$-te Zeile
    der Koef"|fizientenmatrix nimmt, auf
    $(\lambda_1, \ldots, \lambda_n)$ legt und die Einträge komponentenweise
    multipliziert und dann addiert.
    Man schreibt
    \matrixsize{$\begin{pmatrix}
        \alpha_{11} & \alpha_{12} & \cdots & \alpha_{1n} \\
        \alpha_{21} & \alpha_{22} & \cdots & \alpha_{2n} \\
        \vdots & \vdots & \ddots & \vdots \\
        \alpha_{m1} & \alpha_{m2} & \cdots & \alpha_{mn}
    \end{pmatrix}
    \begin{pmatrix}\lambda_1 \\ \vdots \\ \lambda_n\end{pmatrix} =
    \begin{pmatrix}\mu_1 \\ \vdots \\ \mu_m\end{pmatrix}$}.
\end{Bem}

%\begin{Notation}
%    Für $v = \sum_{j=1}^n \lambda_j v_j$ schreibt man
%    \matrixsize{$\begin{pmatrix}\lambda_1 \\ \vdots \\ \lambda_n
%    \end{pmatrix}_\basis{B}$}
%    ($V$ $K$-VR mit Basis $\basis{B} = (v_1, \ldots, v_n)$).
%\end{Notation}

\begin{Kor}
    Das "`Kochrezept"' kann man umschreiben:
    Für $v \in K^n$ ist
    $(\hommatrix{f}{C}{B}v)_\basis{C} = f(v_\basis{B})$.
\end{Kor}

\begin{Kor}
    Für $v_j$ gilt $f(v_j) = {s_j}_\basis{C}$
    ($s_j$ ist die $j$-te Spalte von $\hommatrix{f}{C}{B}$).
\end{Kor}

\begin{Def}{Basiswechsel}
    Mit $f = \id_V$ ist ein \begriff{Basiswechsel} möglich.
    So ist $(\hommatrix{\id_V}{C}{B}v)_\basis{C} = v_\basis{B}$. \\
    $\hommatrix{\id_V}{C}{B}$ heißt dann \begriff{Basiswechselmatrix}.
\end{Def}

\begin{Def}{Menge der Homomorphismen}
    Seien $V, W$ $K$-Vektorräume.
    Dann wird die Menge aller $K$-linearen Abbildungen von $V$ nach $W$ mit
    $\Hom_K(V, W)$ bezeichnet. \\
    Für $V = W$ ist $\Hom_K(V, V) = \End_K(V)$ die Menge aller Endomorphismen.
\end{Def}

\begin{xDef}{Matrix (allgemein)}{Matrix!allgemein}
    Eine \xbegriff{$m \times n$-Matrix}%
    {mxn-Matrix@$m \times n$-Matrix|see{Matrix!allgemein}}
    $A$ über dem Körper $K$ ist ein
    rechteckiges Schema mit $m \cdot n$ Einträgen $\alpha_{ij} \in K$
    ($1 \le i \le m$, $1 \le j \le n$). \\
    Man schreibt $A = (\alpha_{ij})_{ij} = (\alpha_{ij}) =$
    \matrixsize{$\begin{pmatrix}
        \alpha_{11} & \alpha_{12} & \cdots & \alpha_{1n} \\
        \alpha_{21} & \alpha_{22} & \cdots & \alpha_{2n} \\
        \vdots & \vdots & \ddots & \vdots \\
        \alpha_{m1} & \alpha_{m2} & \cdots & \alpha_{mn}
    \end{pmatrix}$}.
    Für $m = n$ heißt $A$ \begriff{quadratisch}.
    $M_{m \times n}(K)$ ist die Menge aller $m \times n$-Matrizen über $K$.
\end{xDef}

\begin{Satz}{Aufspann der Spaltenvektoren}
    Seien $f: V \rightarrow W$ Homomorphismus und $A = \hommatrix{f}{C}{B}$.
    Sind $s_1, \ldots, s_n \in K^m$ die Spaltenvektoren von $A$, dann ist
    $\aufspann{{s_1}_\basis{C}, \ldots, {s_n}_\basis{C}} = \im f$.
\end{Satz}

\begin{Satz}{Zuordnung Matrix -- Homomorphismus}
    $\hommatrix{-}{C}{B}: \Hom_K(V, W) \rightarrow M_{m \times n}(K)$, \\
    $f \mapsto \hommatrix{f}{C}{B}$ ist eine bijektive Abbildung mit
    Umkehrabbildung \\
    $f_-(\basis{C}, \basis{B}): M_{m \times n}(K) \rightarrow \Hom_K(V, W)$,
    $A \mapsto f_A(\basis{C}, \basis{B})$, wobei
    $f_A(\basis{C}, \basis{B})\Bigg($%
    \matrixsize{$\begin{pmatrix}\lambda_1 \\ \vdots \\
    \lambda_n\end{pmatrix}_\basis{B}$}%
    $\Bigg) = \Bigg(A$%
    \matrixsize{$\begin{pmatrix}\lambda_1 \\ \vdots \\
    \lambda_n\end{pmatrix}$}%
    $\Bigg)_\basis{C}$.
\end{Satz}

\subsection{%
    Homomorphismen sind selbst Vektoren!%
}

\begin{Def}{Addition und skalare Multiplikation von Homomorphismen}
    Seien $V, W$ $K$-VR'e, $f, g \in \Hom_K(V, W)$ und $\lambda \in K$.
    Die Summe $f + g$ ist definiert als
    $(f + g)(v) = f(v) + g(v)$ und das skalare Vielfache $\lambda f$ ist
    definiert als $(\lambda f)(v) = \lambda \cdot f(v)$,
    wobei $v \in V$.
\end{Def}

\begin{Satz}{Homomorphismen als Vektorraum}
    Seien $f, g \in \Hom_K(V, W)$ und $\lambda \in K$.
    Dann sind $f + g$ und $\lambda f$ ebenfalls Homomorphismen
    von $V$ nach $W$.
    Die Menge der Homomorphismen $\Hom_K(V, W)$ bildet mit diesen
    Operationen ein $K$-Vektorraum.
    Das Nullelement ist die Null\-abbildung
    $0_{VW}: V \rightarrow W$, $v \mapsto 0_W$, für $f$ ist das additiv Inverse
    $-f: V \rightarrow W$, $v \mapsto -f(v)$.
\end{Satz}

\begin{Def}{Addition und skalare Multiplikation von Matrizen}
    Seien $A = (\alpha_{ij})_{ij}$ und $B = (\beta_{ij})_{ij}$ mit
    $A, B \in M_{m \times n}(K)$ sowie $\lambda \in K$.
    Die Summe $A + B$ ist definiert als \\
    $(\alpha_{ij})_{ij} + (\beta_{ij})_{ij} = (\alpha_{ij} + \beta_{ij})_{ij}$
    und das skalare Vielfache $\lambda A$ ist definiert als
    $\lambda (\alpha_{ij})_{ij} = (\lambda \alpha_{ij})_{ij}$.
\end{Def}

\begin{Satz}{Matrizen als Vektorraum}
    Der Menge der $m \times n$-Matrizen über $K$ $M_{m \times n}(K)$ wird mit
    diesen beiden Operationen ein $K$-Vektorraum.
    Das Nullelement ist die Nullmatrix $0 = (0)_{ij}$ und die zu
    $A = (\alpha_{ij})_{ij}$ inverse Matrix ist
    $-A = (-1) \cdot A = (-\alpha_{ij})_{ij}$.
\end{Satz}

\begin{Satz}{Isomorphismus zwischen Homomorphismen und Matrizen} \\
    Sind $V$ und $W$ endlich erzeugt mit Basen $\basis{B} = (v_1, \ldots, v_n)$
    und $\basis{C} = (w_1, \ldots, w_m)$, so ist \\
    $\hommatrix{-}{C}{B}: \Hom_K(V, W) \rightarrow M_{m \times n}(K)$
    ein Isomorphismus von $K$-Vektorräumen.
\end{Satz}

\begin{Def}{natürliche Basis von Matrizen}
    $\basis{E}(m, n) = \{E_{ij} \in M_{m \times n}(K) \;|\;
    1 \le i \le m,\; 1 \le j \le n\}$ ist eine Basis von $M_{m \times n}(K)$
    und wird als \begriff{natürliche Basis} bezeichnet.
    Dabei ist $E_{ij} = (\alpha_{kl})_{kl}$ mit $\alpha_{kl} = 1$
    für $(k, l) = (i, j)$ und $\alpha_{kl} = 0$ sonst
    (für $1 \le i \le m$ und $1 \le j \le n$). \\
    $E_{ij}$ ist also die $m \times n$-Matrix, die nur Nullen hat, außer
    beim $(i, j)$-ten Eintrag, dort ist $\alpha_{ij} = 1$.
\end{Def}

\begin{Satz}{zugehöriger Homomorphismus}
    Sei für $1 \le i \le m$ und $1 \le j \le n$ mit
    $\varepsilon_{ij}: V \rightarrow W$,
    $\varepsilon_{ij}(v_k) =$
    \matrixsize{$\begin{cases}w_i & \text{für } j = k \\
    0_W & \text{für } j \not= k\end{cases}$}
    ein Homomorphismus definiert.
    Dann ist $\hommatrix{\varepsilon_{ij}}{C}{B} = E_{ij}$.
\end{Satz}

\begin{Kor}
    Für $\dim V = n$ und $\dim W = m$ gilt
    $\dim M_{m \times n}(K) = \dim \Hom_K(V, W) = m \cdot n$.
\end{Kor}

\begin{Def}{transponierte Matrix}
    Sei $A = (\alpha_{ij})_{ij} \in M_{m \times n}(K)$.
    Die \begriff{transponierte Matrix} \\
    $A^t \in M_{n \times m}(K)$ ist
    $A^t = (\alpha_{ji})_{ij}$, d.\,h. Zeilen und Spalten von $A$ werden
    vertauscht. \\
    Das Transponieren ist $K$-linear, d.\,h. $(A + B)^t = A^t + B^t$ und
    $(\lambda A)^t = \lambda A^t$.
\end{Def}

\subsection{%
    Komposition linearer Abbildungen%
}

\begin{Bem}
    Seien $U, V, W$ $K$-Vektorräume sowie $f: U \rightarrow V$ und
    $g: V \rightarrow W$ Homomorphismen.
    Dann ist die Komposition $g \circ f: U \rightarrow W: u \mapsto g(f(u))$
    ebenfalls ein Homomorphismus. \\
    $\circ$ ist nicht kommutativ ($f \circ g$ ist nur definiert für $U = W$),
    jedoch distributiv über $+$. \\
    Wie müssen die Matrizen $A = \hommatrix{f}{B}{A}$ und
    $B = \hommatrix{g}{C}{B}$ verrechnet werden, um die zugehörige Matrix
    der Komposition $C = \hommatrix{g \circ f}{C}{A}$ zu bestimmen?
\end{Bem}

\begin{Def}{Matrizenmultiplikation}
    Seien $B = (\beta_{rk})_{rk} \in M_{m \times p}(K)$ und
    $A = (\alpha_{kl})_{kl} \in M_{p \times n}(K)$ Matrizen.
    Dann ist mithilfe der Formel
    $\gamma_{rl} = \sum_{k=1}^p \beta_{rk} \alpha_{kl}$ eine Matrix
    $C = (\gamma_{rl})_{rl} \in M_{m \times n}$ gegeben.
    $C = B \cdot A = BA$ ist das \begriff{Produkt} der Matrizen $B$ und $A$.
\end{Def}

\begin{Bem}
    Man erhält also den $(r, l)$-ten Eintrag von $C = BA$, indem
    man die $r$-te Zeile von $B$ auf die $l$-te Spalte von $A$ legt,
    paarweise multipliziert und addiert.
    $B$ muss gleich viele Spalten wie $A$ Zeilen haben.
    Die Multiplikation ist nicht kommutativ, aber assoziativ.
\end{Bem}

\begin{Satz}{Komposition/Matrizenmultipl.}
    Seien $U, V, W$ $K$-Vektorräume mit endlichen Basen
    $\basis{A}$, $\basis{B}$, $\basis{C}$ und $f: U \rightarrow V$,
    $g: V \rightarrow W$ Homomorphismen.
    Dann ist $\hommatrix{g \circ f}{C}{A} = \hommatrix{g}{C}{B} \cdot
    \hommatrix{f}{B}{A}$.
\end{Satz}

\subsection{%
    Endomorphismenringe%
}

\begin{Bem}
    Zwei Endomorphismen von $V$ in sich können immer hintereinander ausgeführt
    werden und ergeben wieder einen Endomorphismus von $V$.
    Auf der Matrizenseite entspricht dies der Multiplikation von quadratischen
    Matrizen, das Produkt ist wieder eine quadratische Matrix derselben Größe.
\end{Bem}

\begin{Satz}{Komposition als Operation}
    Die Hintereinanderausführung von Endomorphismen eines Vektorraums $V$ ist
    eine binäre Operation auf $\End_K(V)$.
    Diese ist assoziativ und distributiv auf beiden Seiten über der Addition.
    $\id_V \in \End_K(V)$ ist das neutrale Element bzgl. der Komposition.
    Es gilt $\lambda (f \circ g) = (\lambda f) \circ g = f \circ (\lambda g)$
    für $f, g \in \End_K(V)$ und $\lambda \in K$.
\end{Satz}

\begin{Satz}{Matrizenmultiplikation als Operation}
    Sei $n \in \natural$.
    Dann definiert die Matrizenmultiplikation eine binäre Operation auf der
    Menge $M_n(K) = M_{n \times n}(K)$ der $n \times n$-Matrizen über $K$.
    Diese ist assoziativ und distributiv auf beiden Seiten über der Addition
    von Matrizen. \\
    Die \begriff{Einheitsmatrix} $E_n =$%
    \matrixsize{$\begin{pmatrix}1 & \cdots & 0 \\
    \vdots & \ddots & \vdots \\ 0 & \cdots & 1\end{pmatrix}$}
    ist das neutrale Element bzgl. der Addition
    (Matrix von $\id_V \in \End_K(V)$).
    Sind $A, B \in M_{n}(K)$ und $\lambda \in K$, so ist
    $\lambda (AB) = (\lambda A) B = A (\lambda B)$.
\end{Satz}

%\begin{Def}{Ring}
%    Ein \begriff{Ring} ist eine abelsche Gruppe $(R,+)$ mit einer binären
%    Operation (Multiplikation) $R \times R: (r,s) \mapsto r \cdot s$,
%    die assoziativ und distributiv auf beiden Seiten über der Addition ist.
%    Hat $R$ ein neutrales Element (\begriff{Einselement}) bzgl. der
%    Multiplikation, so ist $R$ ein \begriff{Ring mit Eins} und das Einselement
%    wird mit $1_R$ oder $1$ bezeichnet.
%    Ist die Multiplikation kommutativ, so heißt $R$
%    \begriff{kommutativer Ring}.
%\end{Def}

\begin{xDef}{$K$-Algebra}{Algebra|see{$K$-Algebra}}
    Eine \xbegriff{$K$-Algebra}{K-Algebra@$K$-Algebra} ist ein $K$-Vektorraum
    $A$, der zugleich ein Ring mit Eins ist,
    sodass für $a, b \in A$, $\lambda \in K$
    gilt, dass $\lambda (ab) = (\lambda a) b = a (\lambda b)$.
\end{xDef}

\begin{Satz}{Endomorphismenringe}
    Sei $V$ ein $K$-Vektorraum.
    Dann ist $\End_K(V)$ eine $K$-Algebra, wobei die Multiplikation zweier
    Endomorphismen von $V$ als Komposition definiert ist.
\end{Satz}

\begin{Satz}{Ringe von quadratischen Matrizen}
    Sei $n \in \natural$.
    Dann ist die Menge der $n \times n$-Matrizen $M_n(K)$ eine $K$-Algebra
    der Dimension $n^2$ mit der Matrizenmultiplikation.
\end{Satz}

\begin{Satz}{$\hommatrix{-}{B}{B}$ $K$-Algebraisomorphismus}
    Sei $V$ ein $K$-Vektorraum mit Basis $\basis{B}$. \\
    Dann ist $\hommatrix{-}{B}{B}: \End_K(V) \rightarrow M_n(K)$
    ein Isomorphismus von $K$-Algebren.
\end{Satz}

\subsection{%
    Automorphismen und invertierbare Matrizen%
}

\begin{Def}{invertierbar}
    Sei $A$ eine $K$-Algebra (Ring mit Eins) mit einem Einselement.
    Dann heißt $a \in A$ \begriff{invertierbar}/\begriff{Einheit}, falls es
    ein multiplikativ Inverses zu $a$ gibt, d.\,h. es gibt ein Element
    $b \in A$, sodass $ab = ba = 1$.
    Man schreibt $b = a^{-1}$.
    Die Menge der invertierbaren Elemente von $A$ ist multiplikativ
    abgeschlossen und bildet mit der Multiplikation eine Gruppe, die Gruppe
    $U(A)$ der Einheiten oder \begriff{Einheitengruppe} in $A$.
\end{Def}

\begin{Def}{Einheitengruppe von quadratischen Matrizen}
    Die Einheitengruppe $U(M_{n \times n}(K))$ der $K$-Algebra $M_{n \times n}(K)$
    wird mit $\GL_n(K)$ bezeichnet.
\end{Def}

\begin{Satz}{Homomorphismen und Einheiten}
    Seien $A$, $B$ $K$-Algebren (bzw. Ringe) sowie $f: A \rightarrow B$
    ein $K$-Algebrahomomorphismus (Ringhomomorphismus),
    dann ist $f(U(A)) \subseteq U(B)$ und $f|_{U(A)}$ von $A$ auf $U(A)$ ist
    ein Gruppenhomomorphismus von $U(A)$ in die Einheitengruppe $U(B)$ von $B$.
    Ist $f$ ein Isomorphismus, so auch $f|_{U(A)}$.
\end{Satz}

\begin{Def}{Antihomomorphismus}
    Seien $A$, $B$ $K$-Algebren (bzw. Ringe).
    Eine $K$-lineare Abbildung $f: A \rightarrow B$ heißt
    \begriff{Antihomomorphismus},
    falls $f(ab) = f(b)f(a)$ für alle $a, b \in A$.
    Analog sind Antimono-/epi-/isomorphismen und Antimorphismen für Gruppen
    definiert.
\end{Def}

\begin{Satz}{Transponieren}
    Sei $n \in \natural$.
    Dann ist das Transponieren
    $\_^t: M_{n \times n}(K) \rightarrow M_{n \times n}(K)$, $A \mapsto A^t$
    ein Antiautomorphismus.
    Seine Einschränkung auf invertierbare Matrizen ist ein Antiautomorphismus
    von $\GL_n(K)$ und es gilt $(A^t)^{-1} = (A^{-1})^t$ für alle
    $A \in \GL_n(K)$.
\end{Satz}

\subsection{%
    Der Rang einer Matrix%
}

\begin{Bem}
    $V$ und $W$ seien endliche $K$-Vektorräume mit Basen $\basis{A}$ und
    $\basis{B}$.
    Oft will man Basen von $V$ und $W$ finden, sodass die Matrix eines
    gegebenen Homomorphismus von $V$ nach $W$ bzgl. dieser Basen besonders
    "`schön"' wird.
    Welche Matrizen erhält man also, wenn man den Homomorphismus
    $f: V \rightarrow W$ festhält und die Basen $\basis{A}, \basis{B}$
    variiert?
\end{Bem}

\begin{Def}{Menge aller Matrizen mit beliebiger Basis}
    Sei $f: V \rightarrow W$ ein Homomorphismus.
    Dann ist $\hommatrix{f}{-}{A}$ die Menge aller $m \times n$-Matrizen
    der Form $\hommatrix{f}{B}{A}$, wobei $\basis{B}$ alle Basen von $W$
    durchläuft.
    Analog sind $\hommatrix{f}{B}{-}$ und $\hommatrix{f}{-}{-}$ definiert.
\end{Def}

\begin{Def}{invertierbare Matrizen}
    Eine $n \times n$-Matrix $A$ heißt \begriff{invertierbar}, falls es
    eine $n \times n$-Matrix $B$ gibt, sodass $AB = BA = E_n$.
    In diesem Fall ist $B$ (die \begriff{inverse Matrix} von $A$) durch $A$
    eindeutig bestimmt, man schreibt $B = A^{-1}$. \\
    $\GL_n(K)$ (\begriff{generelle lineare Gruppe}) ist die Menge aller
    invertierbaren $n \times n$-Matrizen.
\end{Def}

\begin{Lemma}{Basiswechselmatrizen}
    Basiswechselmatrizen aus $\hommatrix{\id_V}{-}{-}$ sind invertierbar
    und invertierbare Matrizen sind Matrizen eines
    Basiswechsels.
    Also ist $\hommatrix{\id_V}{-}{-} = \GL_n(K)$.
\end{Lemma}

\begin{Satz}{Komposition mittels invertierbaren Matrizen} \\
    Für zwei Teilmengen $A, B$ eines Rings $R$ definiert man
    $AB = \{ab \;|\; a \in A,\; b \in B\}$ und für $r \in R$ ist
    $rA = \{ra \;|\; a \in A\}$ bzw. $Ar = \{ar \;|\; a \in A\}$. \\
    Seien $f \in \Hom_K(V, W)$ und $\basis{A}, \basis{B}$ beliebige Basen
    von $V$ und $W$. \\
    Dann ist $\hommatrix{f}{-}{-} = \GL_m(K) \hommatrix{f}{B}{A} \GL_n(K)$.
\end{Satz}

\begin{Satz}{äquivalente Matrizen}
    Seien $f, g: V \rightarrow W$ Homomorphismen. \\
    Dann ist entweder
    $\hommatrix{f}{-}{-} \cap \hommatrix{g}{-}{-} = \emptyset$ oder
    $\hommatrix{f}{-}{-} = \hommatrix{g}{-}{-}$.
    So ist auf $M_{m \times n}(K)$ eine Äquivalenzrelation $\approx$ definiert
    durch $A \approx B \;\Leftrightarrow\;
    \exists_{f \in \Hom_K(V, W)}\; A, B \in \hommatrix{f}{-}{-}$.
\end{Satz}

\begin{xDef}{äquivalente Matrizen}{aquivalent@äquivalent!Matrizen}
    Seien $A, B \in M_{m \times n}(K)$.
    Dann ist $A \approx B$ genau dann, wenn es $X \in \GL_m(K)$ und
    $Y \in \GL_n(K)$ gibt, sodass $B = XAY$.
    $A$ und $B$ heißen dann
    \xbegriff{äquivalent}{aquivalent@äquivalent!Matrizen}.
\end{xDef}

\begin{Def}{Spalten-/Zeilenrang}
    Sei $A \in M_{m \times n}(K)$.
    Dann ist der \begriff{Spaltenrang} von $A$ die Dimension des von den
    Spaltenvektoren aufgespannten Unterraums des $K^m$.
    Analog ist der \begriff{Zeilenrang} die Dimension des von den
    Zeilenvektoren aufgespannten Unterraums des $K^n$.
\end{Def}

\begin{Lemma}{Spaltenrang gleich Dimension des Bildes}
    Sei $f: V \rightarrow W$ ein Homomorphismus. \\
    Dann ist der Spaltenrang von $\hommatrix{f}{B}{A}$ gleich $\dim_K(\im f)$.
\end{Lemma}

\begin{Kor}
    Sei $f: V \rightarrow W$ ein Homomorphismus. \\
    Dann haben alle Matrizen in $\hommatrix{f}{-}{-}$ denselben Spaltenrang
    gegeben durch $\dim_K(\im f)$.
\end{Kor}

\begin{Satz}{schöne Matrizen} \\
    \begin{tabular}{p{6.3cm}p{9.7cm}}
    $E_{m \times n}(k) =$%
    \matrixsize{$\begin{pmatrix}
    1 & \cdots & 0 & 0 & \cdots & 0 \\
    \vdots & \ddots & \vdots & \vdots & & \vdots \\
    0 & \cdots & 1 & 0 & \cdots & 0 \\
    0 & \cdots & 0 & 0 & \cdots & 0 \\
    \vdots & & \vdots & \vdots & & \vdots \\
    0 & \cdots & 0 & 0 & \cdots & 0 \end{pmatrix}$}
    &
    \begin{minipage}[c]{9.7cm}Sei $f: V \rightarrow W$ Homomorphismus und
    $k = \dim(\im f)$.
    Dann ist $E_{m \times n}(k) \in \hommatrix{f}{-}{-}$.
    Diese Matrix hat $k$ viele Einsen und den Spalten-/Zeilenrang $k$. \\
    Daher besteht $\hommatrix{f}{-}{-}$ genau aus allen\\$m \times n$-Matrizen
    mit Spaltenrang $k$.\end{minipage}
    \end{tabular}
\end{Satz}

\begin{Kor}
    Sei $f: V \rightarrow W$ Homom.
    Dann ist $\dim_K(\im f) + \dim_K(\ker f) = \dim_K(V)$.
\end{Kor}

\begin{Kor}
    Spalten- und Zeilenrang von Matrizen stimmen überein.
\end{Kor}

\begin{Def}{Rang}
    Der Spalten-/Zeilenrang $\rg(A)$ wird als \begriff{Rang} einer Matrix $A$
    bezeichnet.
\end{Def}

\pagebreak

\begin{Bem}
    $M_{m \times n}(K)$ hat genau $k + 1$ Äquivalenzklassen $\matrixm_i$
    bezüglich $\approx$, nämlich \\
    $\matrixm_i = \{A \in M_{m \times n}(K) \;|\;
    \rg(A) = i\}$ ($i = 0, \ldots, k$, $k = \min\{n, m\}$). \\
    Um den Rang einer Matrix auszurechnen, verändert man die Basen
    $\basis{A}, \basis{B}$, bis sie die Form $E_{m \times n}(k)$ hat.
    Dabei bleibt der Rang konstant und die Matrix hat dann den Rang $k$. \\
    d.\,h. man konstruiert Basen $\basis{A'}, \basis{B'}$, sodass
    $\hommatrix{f}{B'}{A'} = E_{m \times n}(k)$.
\end{Bem}

\begin{Lemma}{Modifikation von Basen}
    Seien $V$ ein $K$-Vektorraum und $\basis{A} = (v_1, \ldots, v_n)$ eine
    Basis von $V$.
    Dann ist $\basis{A'}$ ebenfalls eine Basis von $V$, wenn $\basis{A'}$
    durch folgende Modifikationen entsteht: \\
    a) Vertauschen zweier Vektoren, \qquad
    b) Multiplikation eines $v_i$ mit einem Skalar $\lambda \in K$
    ($\lambda \not= 0$), \\
    c) Ersetzen von $v_i$ durch $v_i' = v_i + \lambda v_j$ mit
    $1 \le j \le n$, $\lambda \in K$.
\end{Lemma}

\begin{Def}{elementare Operationen}
    Sei $A \in M_{m \times n}(K)$.
    Dann sind folgende Operationen \begriff{elementare Zeilenoperationen}:
    a) Vertauschen zweier Zeilen, \qquad
    b) Multiplikation einer Zeile mit einem Skalar $\lambda \not= 0$, \qquad
    c) Addition des Vielfachen einer Zeile zu einer anderen. \\
    Analog werden \begriff{elementare Spaltenoperationen} definiert. \\
    \begriff{Elementare Operationen} sind elementare
    Zeilen-/Spaltenoperationen.
\end{Def}

\begin{Def}{Elementarmatrizen}
    Die Anwendung einer elementaren Operation auf eine Matrix $A$ entspricht
    dem Produkt $AM$ bzw. $MA$ (für Spalten- bzw. Zeilenoperationen)
    mit einer geeigneten invertierbaren Matrix $M$, die als Basiswechselmatrix
    aufgefasst werden kann.
    Diese Matrizen heißen \begriff{Elementarmatrizen}.
\end{Def}

\begin{Satz}{Rang bleibt erhalten}
    Unter elementaren Operationen bleibt der Rang erhalten.
\end{Satz}

\begin{Satz}{Rang ausrechnen}
    Sei $A \in M_{m \times n}(K)$.
    Dann gibt es eine Reihe von elementaren Operationen, die auf $A$
    angewendet die Matrix $E_{m \times n}(k)$ ergeben, sodass $k = \rg(A)$.
\end{Satz}

\begin{Prozedur}{Rang einer Matrix ausrechnen}
    Der Rang einer Matrix kann ausgerechnet werden, indem man elementare
    Zeilen-/Spaltenoperationen verwendet, um $E_{m \times n}(k)$ zu erreichen.
    Dann ist $k$ der Rang der Matrix.
\end{Prozedur}

\begin{Kor}
    Ist $A \in M_{n \times n}(K)$, so ist $A$ genau dann invertierbar,
    wenn $\rg(A) = n$. \\
    $E_{n \times n}(n) = E_n$ ist die \begriff{Einsmatrix}.
    Jede invertierbare Matrix ist Produkt von Elementarmatrizen.
    Sei $A \in M_{m \times n}(K)$, dann ist $\rg(A) = \rg(A^t)$.
\end{Kor}

\begin{Def}{augmentierte Matrix} \\
    \begin{tabular}{p{5.8cm}p{10.2cm}}
    \matrixsize{$\left(\begin{array}{ccc|ccc}
    \alpha_{11} & \cdots & \alpha_{1n} & \beta_{11} & \cdots & \beta_{1p} \\
    \vdots & & \vdots & \vdots & & \vdots \\
    \alpha_{m1} & \cdots & \alpha_{mn} & \beta_{m1} & \cdots & \beta_{mp}
    \end{array}\right)$}
    &
    \begin{minipage}[c]{10.2cm}Seien $A = (\alpha_{ij}) \in M_{m \times n}(K)$
    und $B = (\beta_{ij}) \in M_{m \times p}(K)$.
    Dann entsteht die \begriff{augmentierte Matrix} $(A|B)$ durch
    Aneinanderfügen der Spalten von $A$ und $B$.\end{minipage}
    \end{tabular}
\end{Def}

\begin{Prozedur}{Matrix invertieren}
    Eine invertierbare $n \times n$-Matrix $A$ kann man invertieren, indem man
    $(A | E_n)$ durch eine Folge elementarer \emph{Zeilenoperationen} in
    $(E_n | A^{-1})$ umwandelt.
\end{Prozedur}

\subsection{%
    \emph{Zusätzliches}: Projekt 5 (Nilpotenz und Homomorphismen)%
}

\begin{Def}{nilpotent}
    Sei $A$ eine $K$-Algebra. \\
    Ein Element $x \in A$ heißt \begriff{nilpotent}, falls
    es ein $n \in \natural$ gibt, sodass $x^n = 0$.
\end{Def}

\begin{Satz}{nilpotente Elemente als Unterraum}
    Sei $A$ eine $K$-Algebra.
    Ist die Multiplikation in $A$ kommutativ, dann bilden die
    nilpotenten Elementen einen Unterraum vom $K$-Vektorraum $A$. \\
    Bei nicht-kommutativer Multiplikation stimmt dies i.\,A. nicht.
\end{Satz}

\begin{Satz}{nilpotentes Element zu Einheit}
    Seien $A$ eine $K$-Algebra und $x \in A$ nilpotent. \\
    Dann ist $1 + x$ eine Einheit (d.\,h. invertierbar).
\end{Satz}

\begin{Def}{Homomorphismen bei Gruppen, Ringen und
            $K$-Algebren} \\
    \emph{Gruppe}: $f: G \rightarrow H$ mit
    $f(x \circ y) = f(x) \bullet f(y)$ für alle $x, y \in G$ für Gruppen
    $(G, \circ)$ und $(H, \bullet)$ \\
    \emph{Ring}: $f: R \rightarrow S$ mit
    $f(x + y) = f(x) \boxplus f(y)$ und $f(x \cdot y) = f(x) \boxdot f(y)$
    für alle $x, y \in R$ für Ringe
    $(R, \boldsymbol{+}, \boldsymbol{\cdot})$ und $(S, \boxplus, \boxdot)$ \\
    \emph{Ring mit Eins}: wie Ring, aber zusätzlich $f(1_R) = 1_S$, wobei
    $1_R \in R$, $1_S \in S$ Einselemente sind \\
    \emph{$K$-Algebra}: Vektorraumhomomorphismus $f: A \rightarrow B$, der
    gleichzeitig Homomorphismus von Ringen mit Eins ist für $K$-Algebren $A$
    und $B$ \\
    \emph{Kerne} der Homomorphismen sind die Urbilder der Nullelemente.
    $\ker f$ und $\im f$ sind Unterstrukturen, jedoch ist $\ker f$ bei
    $K$-Algebren keine $K$-Algebra.
\end{Def}
