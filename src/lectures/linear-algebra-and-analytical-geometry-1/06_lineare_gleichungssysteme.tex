\section{%
    Lineare Gleichungssysteme%
}

\subsection{%
    Theoretisches%
}

\begin{Def}{lineares Gleichungssystem} \\
    \begin{tabular}{p{7.7cm}p{8.3cm}}
    \matrixsize{$\begin{array}{rcrcccrcl}
    \alpha_{11} x_1 & + & \alpha_{12} x_2 & + & \cdots &
    + & \alpha_{1n} x_n & = & \beta_1 \\
    \alpha_{21} x_1 & + & \alpha_{22} x_2 & + & \cdots &
    + & \alpha_{2n} x_n & = & \beta_2 \\
    \vdots & & \vdots & & & & \vdots & & \vdots \\
    \alpha_{m1} x_1 & + & \alpha_{m2} x_2 & + & \cdots &
    + & \alpha_{mn} x_n & = & \beta_m
    \end{array}$}
    &
    \begin{minipage}[c]{8.3cm}Ein lineares Gleichungssystem \lgs{G} besteht
    aus $m$ Gleichungen, $n$ Unbestimmten $x_j$ ($1 \le j \le n$),
    Koef"|fizienten $\alpha_{ij} \in K$ und $\beta_i \in K$ ($1 \le i \le m$)
    und hat die links angegebene Form.\end{minipage}
    \end{tabular}
\end{Def}

\begin{Def}{Matrixform}
    Ein LGS \lgs{G} kann in eine Matrixgleichung $Ax = b$ umgeschrieben
    werden. \\
    Dabei ist $x =$
    \matrixsize{$\begin{pmatrix}x_1 \\ \vdots \\ x_n\end{pmatrix}$}
    $ \in K^n$, $b = $
    \matrixsize{$\begin{pmatrix}\beta_1 \\ \vdots \\
    \beta_m\end{pmatrix}$} $ \in K^m$
    und $A = (\alpha_{ij}) \in M_{m \times n}(K)$. \\
    Der \begriff{zu \lgs{G} gehörende Homomorphismus}
    $f_A: K^n \rightarrow K^m$ wird definiert als $f_A(x) = Ax$ für $x \in K^n$
    (unabhängig von $b$).
    Es ist $\hommatrix{f_A}{\stdbasis{m}}{\stdbasis{n}} = A$.
    Die Lösungsgesamtheit \lgslsg{G} von \lgs{G} besteht aus allen Vektoren in
    der Faser $f_A^{-1}(b)$.
\end{Def}

\begin{Def}{homogenes LGS}
    Sei ein LGS \lgs{G} mit $Ax = b$ gegeben. \\
    Ist $b$ der Nullvektor, so heißt \lgs{G} \begriff{homogen},
    sonst \begriff{inhomogen}. \\
    Ist \lgs{G} homogen, so ist die Lösung $x = 0$ die
    \begriff{triviale Lösung}. \\
    Ist \lgs{G} inhomogen, so heißt das LGS \lgs{H} mit $Ax = 0$ das
    \begriff{zu \lgs{G} gehörige homogene System}.
\end{Def}

\begin{Satz}{Lösungen eines homogenen LGS}
    Die Lösungsgesamtheit eines homogenen LGS $Ax = 0$ besteht genau aus
    den Vektoren von $\ker f_A$.
\end{Satz}

\begin{Kor}
    Ein homogenes LGS \lgs{H} ($Ax = 0$) besitzt genau dann nur die triviale
    Lösung, wenn der zugehörige Homomorphismus $f_A$ injektiv ist. \\
    Die Menge der Lösungen von \lgs{H} ist ein Unterraum von $K^n$ mit der
    Dimension $n - \rg(A)$.
\end{Kor}

\begin{Satz}{Lösbarkeit}
    Sei $\lgs{G}: Ax = b$ ein LGS mit $A \in M_{m \times n}(K)$.
    Dann sind folgende Aussagen äquivalent: \qquad
    \lgs{G} besitzt eine Lösung, \qquad $b \in \im f_A$, \qquad
    $\rg(A) = \rg(A|b)$. \\
    \lgs{G} hat genau dann eine eindeutige Lösung, wenn
    $\rg(A) = \rg(A|b) = n$.
\end{Satz}

\begin{Satz}{Lösungen}
    Sei $x_0$ eine Lösung des LGS \lgs{G}.
    Dann ist $\lgslsg{G} = x_0 + \ker f_A$.
\end{Satz}

\begin{Kor}
    Sei $\lgs{G}: Ax = b$ ein LGS mit $n$ Gleichungen und $n$ Unbestimmten.
    Dann hat \lgs{G} eine eindeutige Lösung genau dann, wenn $A$ invertierbar
    ist.
    In diesem Fall ist die Lösung $x = A^{-1}b$.
\end{Kor}

\begin{Kor}
    Sei $m < n$ und $\lgs{H}: Ax = 0$ ein homogenes LGS mit $m$ Gleichungen
    und $n$ Unbestimmten.
    Dann hat \lgs{H} nichttriviale Lösungen.
\end{Kor}

\pagebreak

\subsection{%
    Konkretes%
}

\begin{Satz}{LGS-Umformungen}
    Sei $\lgs{G}: Ax = b$ ein LGS mit $A \in M_{m \times n}(K)$. \\
    Kann man $(A|b)$ in ($A'|b')$ mittels elementaren Zeilenoperationen
    umwandeln, so ist $\lgslsg{G} = \lgslsg{G'}$ \\
    (wobei $\lgs{G'}: A'x = b'$).
    Wandelt man $A$ in $A''$ durch eine Permutation $\pi$ der Spalten von $A$
    um, so erhält man $\lgslsg{G}$ aus $\lgslsg{G''}$, indem man auf die
    Komponenten jedes Lösungsvektors $x_0$ von $\lgslsg{G'}$ die inverse
    Permutation $\pi^{-1}$ anwendet (wobei $\lgs{G''}: A''x = b$).
\end{Satz}

\begin{Satz}{Lösen von LGS} \\
    \begin{tabular}{p{6.8cm}p{9.2cm}}
    \matrixsize{$\left(\begin{array}{ccccccc|c}
    1 & 0 & \cdots & 0 &
    \delta_{1, r+1} & \cdots & \delta_{1, n} & \beta'_1 \\
    0 & 1 & \cdots & 0 &
    \delta_{2, r+1} & \cdots & \delta_{2, n} & \beta'_2 \\
    \vdots & \vdots & \ddots & \vdots & \vdots & & \vdots & \vdots \\
    0 & 0 & \cdots & 1 &
    \delta_{r, r+1} & \cdots & \delta_{r, n} & \beta'_r \\
    0 & 0 & \cdots & 0 & 0 & \cdots & 0 & \beta'_{r+1} \\
    \vdots & \vdots & & \vdots & \vdots & & \vdots & \vdots \\
    0 & 0 & \cdots & 0 & 0 & \cdots & 0 & \beta'_m \end{array}\right)$}
    &
    \begin{minipage}[c]{9.2cm}
    Sei $\lgs{G}: Ax = b$ ein LGS mit $A \in M_{m \times n}(K)$.
    Dann kann $(A|b)$ durch Zeilenoperationen und Anwendung einer Permutation
    $\pi$ der Spalten von $A$ auf die Gestalt $(A'|b')$ (siehe links)
    gebrachten werden.
    Dabei ist $r = \rg(A)$ sowie $\delta_{kl}, \beta'_i \in K$
    ($1 \le k \le r < r+1 \le l \le n$, $1 \le i \le m$).
    \end{minipage}\end{tabular}
\end{Satz}

\begin{Satz}{Lösungen eines LGS} \\
    \begin{tabular}{p{5.9cm}p{10.1cm}}
    \matrixsize{$x_0 = \begin{pmatrix}\beta'_1 \\ \vdots \\ \beta'_r \\ 0 \\
    \vdots \\ 0\end{pmatrix}
    \in K^n,\; x_i = \begin{pmatrix}-\delta_{1, r+i} \\ \vdots \\
    -\delta_{r, r+i} \\ 0 \\ \vdots \\ 1 \\ \vdots \\
    0\end{pmatrix} \in K^n$}
    &
    \begin{minipage}[c]{10.1cm}
    Sei $\lgs{G'}: A'x = b'$ ein LGS mit der obigen Form.
    Dann ist \lgs{G} genau dann lösbar, falls
    $\beta'_{r+1} = \cdots = \beta'_m = 0$. \\
    In diesem Fall besteht $\lgslsg{G'}$ aus allen Vektoren der Form \\
    $x = x_0 + \lambda_1 x_1 + \cdots + \lambda_s x_s$
    mit $\lambda_1, \ldots, \lambda_s \in K$, $s = n - r$ sowie
    $r = \rg(A)$, wobei der Eintrag $1$ an der $r+i$-ten Stelle steht.
    Dabei ist $x_0$ eine spezielle Lösung und
    $\aufspann{x_1, \ldots, x_s} = \ker f_A$, genauer
    $\{x_1, \ldots, x_s\}$ eine Basis von $\ker f_A$.
    \end{minipage}\end{tabular}
\end{Satz}

\begin{Prozedur}{Lineares Gleichungssystem lösen}
    Gegeben sei ein LGS $\lgs{G}: Ax = b$.
    \begin{enumerate}
        \item Man bildet die augmentierte Matrix $(A|b)$.

        \item Ist die erste Spalte eine Nullspalte, so vertauscht man sie mit
        der ersten Spalte vom $A$-Teil, die keine Nullspalte ist.
        Ist $A$ die Nullmatrix, so ist man fertig (Schritt 6).

        \item Ist das Element an Position $(1, 1)$ Null, so vertauscht man die
        erste Zeile mit einer Zeile, an deren erster Position keine Null steht.

        \item Dann dividiert man die erste Zeile durch den ersten
        Eintrag, sodass dort eine $1$ steht. \\
        Durch Abziehen von Vielfachen der ersten Zeile kann man
        erreichen, dass in der ersten Spalte bis auf den ersten Eintrag nur
        Nullen stehen.

        \item Dann macht man mit der zweiten Spalte und Zeile genau so weiter.
        Der $(2, 2)$-te Eintrag ist dann eine Eins, durch Abziehen
        eines Vielfachen der zweiten Spalte kann man erreichen, dass der
        $(1, 2)$-te Eintrag $0$ ist.

        \item Mit den anderen Zeilen/Spalten fährt man auf diese Weise fort.
        Man endet in einer Matrix $(A'|b')$ der obigen Form.

        \item Ist $\beta'_i \not= 0$ für ein $i$ mit $r + i \le i \le m$, dann
        ist das LGS nicht lösbar.
        Andernfalls füllt man $(A'|b')$ auf ($m < n$) oder streicht Nullen
        ($m > n$), bis eine Matrix mit $n$ Zeilen entsteht.

        \item $x_0$ ist dann eine spezielle Lösung des modifizierten LGS
        und die $x_i$ ($1 \le i \le n - r$) eine Basis der Lösungsgesamtheit
        des zugehörigen homogenen LGS (\emph{siehe oben}).
        Dann werden die Spaltenpermutationen durch Anwendung der inversen
        Permutation auf $x_0, x_1, \ldots, x_{n-r}$ wieder rückgängig
        gemacht. \\
        Die Lösungsgesamtheit des LGS \lgs{G} ist dann
        $\lgslsg{G} = x_0 + \aufspann{x_1, \ldots, x_{n-r}}$.
    \end{enumerate}
\end{Prozedur}

\subsection{%
    Numerisches%
}

\begin{Def}{Treppenform}
    $A = (\alpha_{ji}) \in M_{m \times n}(K)$ mit Zeilenvektoren $z_j$ ist in
    \begriff{Treppenform}, falls $A = (0)$ oder es Indizes
    $1 \le i_1 < \cdots < i_r \le n$ ($r = \rg(A)$) gibt, sodass \\
    1. alle Zeilen $z_j$ mit $j > r$ Nullzeilen sind und \quad
    2. für $j \le r$ der erste von Null verschiedene Eintrag der Zeile $z_j$
    in Spalte $i_j$ ist, d.\,h. $\alpha_{ji_{j}} \not= 0$ sowie für $k < i_j$
    gilt $\alpha_{jk} = 0$.
\end{Def}

\begin{Beobachtung}
    Ein LGS in oberer Dreiecksgestalt (quadratisch, unter der Hauptdiagonalen
    Nullen) lässt sich durch Einsetzen von unten nach oben einfach auflösen.
    (Man kann immer annehmen, dass die Matrix quadratisch ist:
    sonst Nullzeilen/-spalten hinzufügen.)
\end{Beobachtung}

\begin{xDef}{\name{Gauß}-Algorithmus}%
{Gauss-Algorithmus@\name{Gauß}-Algorithmus}
    Mit dem
    \xbegriff{Gauß-Algorithmus}{Gauss-Algorithmus@\name{Gauß}-Algorithmus}
    lässt sich ein LGS sogar nur mit elementaren Zeilenoperationen
    (ohne Spaltenoperationen) lösen.
\end{xDef}

\subsection{%
    \emph{Zusätzliches}: Projekt 6 (Af"|fine Geometrie)%
}

\begin{Def}{af"|finer Raum}
    Ein \begriff{af"|finer Raum} $(\affin{A}, V, \boldsymbol{+})$ besteht aus
    einer nicht-leeren Menge $\affin{A}$ von Punkten, einem $K$-Vektorraum $V$
    und einer Abbildung
    $\boldsymbol{+}: \affin{A} \times V \rightarrow \affin{A}$,
    sodass $(P + v) + u = P + (v + u)$,
    $\exists!_{x \in V}\; Q = P + x$ sowie
    $P + 0_V = P$
    für alle $P, Q \in \affin{A}$ und $u, v \in V$. \\
    Der eindeutig bestimmte Vektor $x \in V$ in der Gleichung $Q = P + x$
    heißt $x = \vektor{PQ}$. \\
    Für $\affin{A} = \emptyset$ entfallen Vektorraum und Abbildung. \\
    Ist $K = \real$ oder $K = \complex$, so heißt $\affin{A}$
    \begriff{reeller oder komplexer af"|finer Raum}. \\
    Die \begriff{Dimension} $\dim \affin{A}$ eines af"|finen Raums ist
    $\dim V$.
    Für $\affin{A} = \emptyset$ ist $\dim \affin{A} = -1$.
\end{Def}

\begin{Def}{af"|finer Unterraum}
    Sei $(\affin{A}, V, \boldsymbol{+})$ ein af"|finer Raum.
    $(\affin{U}, V_\affin{U}, \boldsymbol{+})$ heißt
    \begriff{af"|finer Unterraum} von $\affin{A}$, falls
    $\affin{U} \subseteq \affin{A}$ und $V_\affin{U} \ur V$ mit
    $V_\affin{U} = \{v \in V \;|\;
    P + v \in \affin{U} \text{ für alle } P \in \affin{U}\}$. \\
    Zwei nicht-leere af"|fine Unterräume $\affin{U}, \affin{W}$ des af"|finen
    Raums $\affin{A}$ heißen \begriff{parallel}
    ($\affin{U} \parallel \affin{W}$),
    falls $V_\affin{U} \subseteq V_\affin{W}$ oder
    $V_\affin{W} \subseteq V_\affin{U}$.
    $\emptyset$ als af"|finer Unterraum ist parallel zu allen Unterräumen.
\end{Def}

\begin{Def}{af"|fine Abbildung}
    Seien $(\affin{A}, V_1, \boldsymbol{+})$,
    $(\affin{B}, V_2, \boldsymbol{+})$ zwei af"|fine Räume
    ($V_1, V_2$ $K$-Vektorräume).
    Eine Abbildung $f: \affin{A} \rightarrow \affin{B}$ heißt
    \begriff{af"|fine Abbildung}, falls es eine $K$-lineare Abbildung
    $f^\ast: V_1 \rightarrow V_2$ gibt,
    sodass $f^\ast(\vektor{PQ}) = \vektor{f(P) f(Q)}$ für alle
    $P, Q \in \affin{A}$ \\
    (alternativ $f(P + v) = f(P) + f^\ast(v)$ für alle
    $P \in \affin{A}$ und $v \in V_1$). \\
    Eine bijektive af"|fine Abbildung eines af"|finen Raums in sich heißt
    \begriff{Af"|finität}.
\end{Def}

\begin{Lemma}{Af{}finitäten und Isomorphismen}
    Eine af"|fine Abbildung $f$ ist genau dann eine Af"|finität, wenn
    $f^\ast$ ein Isomorphismus ist.
\end{Lemma}

\begin{Def}{Punkt, Gerade, Ebene}
    Sei $\affin{A}$ ein af"|finer Raum. \\
    Ein \begriff{Punkt} ist ein Element $P \in \affin{A}$
    (oder ein $0$-dimensionaler af"|finer Unterraum). \\
    Eine \begriff{Gerade} bzw. eine \begriff{Ebene} ist ein $1$- bzw.
    $2$-dimensionaler Unterraum von $\affin{A}$.
\end{Def}

\begin{Def}{Verbindungsraum}
    Für zwei af"|fine Unterräume $\affin{U}, \affin{W}$ von $\affin{A}$ ist
    $\affin{U} \lor \affin{W}$ der \begriff{Verbindungsraum} und zwar
    der kleinste af"|fine Unterraum von $\affin{A}$, der $\affin{U}$ und
    $\affin{W}$ als Teilmengen enthält.
\end{Def}

\begin{Lemma}{Verbindungsraum und Durchschnitt}
    Es gilt $V_{\affin{U} \lor \affin{W}} = V_\affin{U} + V_\affin{W}$ für
    $\affin{U} \cap \affin{W} \not= \emptyset$ bzw.
    $V_{\affin{U} \lor \affin{W}} = V_\affin{U} + V_\affin{W} +
    K (\vektor{PP'})$ für $\affin{U} \cap \affin{W} = \emptyset$,
    wobei $P \in \affin{U}$ und $P' \in \affin{W}$ fest gewählt sind. \\
    Es gilt $V_{\affin{U} \cap \affin{W}} = V_\affin{U} \cap V_\affin{W}$.
\end{Lemma}

\begin{Satz}{Dimensionssatz}
    Seien $\affin{U}, \affin{W}$ zwei endlich-dimensionale af"|fine Unterräume
    des af"|finen Raums $\affin{A}$.
    Dann gilt $\dim \affin{U} + \dim \affin{W} =
    \dim(\affin{U} \lor \affin{W}) + \dim(\affin{U} \cap \affin{W}) +
    \dim(V_\affin{U} \cap V_\affin{W})$ für den Fall
    $\affin{U}, \affin{W} \not= \emptyset$ und
    $\affin{U} \cap \affin{W} = \emptyset$.
    Andernfalls ist $\dim \affin{U} + \dim \affin{W} =
    \dim(\affin{U} \lor \affin{W}) + \dim(\affin{U} \cap \affin{W})$.
\end{Satz}

\pagebreak

\begin{Def}{kollinear}
    Drei Punkte $x, y, z \in \affin{A}$ eines af"|finen Raums $\affin{A}$
    heißen \begriff{kollinear}, falls sie auf einer gemeinsamen Gerade liegen.
\end{Def}

\begin{Def}{Teilverhältnis}
    Seien $P, Q, R \in \affin{A}$ drei kollineare Punkte eines af"|finen Raums
    $\affin{A}$. \\
    Ist $P \not= Q$, dann existiert ein Skalar $t \in K$, sodass
    $\vektor{PR} = t \cdot \vektor{PQ}$. \\
    $t = \TV(P, Q, R)$ heißt \begriff{Teilverhältnis} von $P, Q, R$.
\end{Def}

\begin{Satz}{af{}fine Abbildungen erhält Teilverhältnis}
    Seien $P, Q, R \in \affin{A}$ drei kollineare Punkte eines af"|finen Raums
    $\affin{A}$ und $f: \affin{A} \rightarrow \affin{B}$ eine af"|fine
    Abbildung. \\
    Dann sind $f(P), f(Q), f(R)$ ebenfalls kollinear.
    Ist $P \not= Q$ sowie $f(P) \not= f(Q)$, dann bleibt das Teilverhältnis
    erhalten, d.\,h. ist $\vektor{PR} = t \cdot \vektor{PQ}$ für ein $t \in K$,
    dann ist $\vektor{f(P)f(R)} = t \cdot \vektor{f(P)f(Q)}$.
\end{Satz}

\begin{Def}{Fixpunkt, Fixgerade, Fixpunktgerade}
    Ein \begriff{Fixpunkt} $P$ einer af"|finen Abbildung
    $f: \affin{A} \rightarrow \affin{A}$ ist ein Punkt $P \in \affin{A}$,
    sodass $f(P) = P$.
    Eine \begriff{Fixgerade} ist eine Gerade, die wieder auf sich selbst
    abgebildet wird, d.\,h. das Bild jeden Punktes der Gerade liegt wieder auf
    der Gerade.
    Eine \begriff{Fixpunktgerade} ist eine Gerade, die nur aus Fixpunkten
    besteht, d.\,h. jeder Punkt wird auf sich selbst abgebildet.
\end{Def}

\begin{Satz}{Bestimmung von af{}finen Abbildungen durch Fixpunkte}
    Sei $\affin{G}$ eine af"|fine Gerade und
    $f: \affin{G} \rightarrow \affin{G}$
    eine af"|fine Abbildung mit zwei verschiedenen Fixpunkten.
    Dann ist $f$ die Identität. \\
    Sei $\affin{E}$ eine af"|fine Ebene und
    $f: \affin{E} \rightarrow \affin{E}$
    eine af"|fine Abbildung mit zwei verschiedenen Fixpunktgeraden.
    Dann ist $f$ die Identität.
\end{Satz}

\begin{Def}{af"|fines Koordinatensystem}
    Sei $(\affin{A}, V, \boldsymbol{+})$ ein af"|finer Raum.
    Eine Menge $\{p_0, p_1, \ldots, p_n\}$ heißt
    \begriff{af"|fines Koordinatensystem}, falls
    $\{\vektor{p_0 p_i} \;|\; 1 \le i \le n\}$ eine Basis von $V$ bildet.
\end{Def}

\pagebreak
