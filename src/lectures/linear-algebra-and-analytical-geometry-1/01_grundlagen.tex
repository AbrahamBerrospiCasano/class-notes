\chapter{%
    Grundlagen%
}

\section{%
    Mengen und Relationen%
}

\begin{Def}{Menge (\name{Cantor})}
    Eine \begriff{Menge} ist eine Zusammenfassung von wohlunterschiedenen
    Objekten der (mathematischen) Anschauung und des (mathematischen) Denkens.
    Die Objekte von $M$ werden \begriff{Elemente} genannt.
    Ist $a$ ein Element der Menge $M$, so schreibt man $a \in M$, sonst
    $a \notin M$.
    Die \begriff{leere Menge} $\emptyset$ (oder $\{\}$) ist die Menge, die
    kein Element enthält.
\end{Def}

\begin{Def}{Teilmenge}
    Seien $A$ und $B$ Mengen.
    $A$ ist eine \begriff{Teilmenge} von $B$, wenn jedes Element von $A$ auch
    Element von $B$ ist, d.\,h. $x \in A \;\Rightarrow\; x \in B$.
    Man schreibt dann $A \subseteq B$.
\end{Def}

\begin{Def}{Aussagen}
    \begriff{Aussagen} sind entweder wahr oder falsch.
    Eine Aussage kann \begriff{negiert} ($\lnot$), zwei Aussagen können durch
    \begriff{Konjunktion} ($\land$), \begriff{Alternative} ($\lor$),
    \begriff{Implikation} ($\Rightarrow$) oder
    \xbegriff{Äquivalenz}{Aquivalenz@Äquivalenz} ($\Leftrightarrow$)
    miteinander verknüpft werden: \\
    \matrixsize{\begin{tabular}{c|c}
        $A$ & $\lnot A$ \\ \hline
        \wahr & \falsch \\
        \falsch & \wahr
    \end{tabular}
    \hspace{1cm}
    \begin{tabular}{c|c|c|c|c|c}
        $A$ & $B$ & $A \land B$ & $A \lor B$ & $A \Rightarrow B$ &
        $A \Leftrightarrow B$ \\ \hline
        \falsch & \falsch & \falsch & \falsch & \wahr & \wahr \\
        \falsch & \wahr & \falsch & \wahr & \wahr & \falsch \\
        \wahr & \falsch & \falsch & \wahr & \falsch & \falsch \\
        \wahr & \wahr & \wahr & \wahr & \wahr & \wahr
    \end{tabular}}
\end{Def}

\begin{Def}{Kontraposition}
    Es gilt $(A \Rightarrow B) \Leftrightarrow (\lnot B \Rightarrow \lnot A)
    \Leftrightarrow \lnot(\lnot B \land A)$, d.\,h. ist $A \Rightarrow B$ zu
    zeigen, kann man auch $\lnot B \Rightarrow \lnot A$ zeigen
    (\begriff{Kontraposition}).
    Bei einem \begriff{Widerspruchsbeweis} zeigt nimmt man an, dass $A$ und
    $\lnot B$ gelten.
    Ergibt sich ein Widerspruch, dann ist $\lnot B \land A$ falsch, d.\,h.
    es gilt $A \Rightarrow B$.
\end{Def}

\begin{Notation}
    Mengen kann man als Liste von Elementen $M = \{a, b, c\}$
    (auch unendlich: $\natural = \{1, 2, 3, \ldots\}$) schreiben oder sie
    können durch \begriff{Aussageformen} beschrieben werden.
    Eine Aussageform $A(x)$ wird zu einer Aussage, wenn man Variablen in $x$
    einsetzt.
    Man schreibt dann $M = \{x \;|\; A(x)\}$.
    Die \begriff{Quantoren} $\exists$ und $\forall$ sind Abkürzungen für
    "`es gibt"' und "`für alle"'.
\end{Notation}

\begin{Def}{Operationen mit Mengen}
    $A \subseteq B \;\Leftrightarrow\; a \in A \Rightarrow a \in B$
    \begriff{Teilmenge}, \\
    $A \subset B \;\Leftrightarrow\; A \subseteq B \land A \not= B$
    \begriff{echte Teilmenge}, \\
    $A \cap B = \{x \;|\; x \in A \land x \in B\}$ \begriff{Durchschnitt},
    $A \cup B = \{x \;|\; x \in A \lor x \in B\}$ \begriff{Vereinigung},
    $A \setminus B = \{x \in A \;|\; x \notin B\}$ \begriff{Differenz},
    $P(A) = \{B \;|\; B \subseteq A\}$ \begriff{Potenzmenge}
\end{Def}

\begin{Bem}
    Es gilt $A = B \;\Leftrightarrow\; A \subseteq B \land B \subseteq A$
    sowie $A \subseteq A$ für alle Mengen $A$, $B$. \\
    Außerdem gilt $\lnot(\forall_{x \in M}\; A(x)) \;\Leftrightarrow\;
    \exists_{x \in M}\; \lnot A(x)$ sowie
    $\lnot(\exists_{x \in M}\; A(x)) \;\Leftrightarrow\;
    \forall_{x \in M}\; \lnot A(x)$.
\end{Bem}

\begin{Def}{kartesisches Produkt}
    Das \begriff{kartesische Produkt} zweier Mengen $M$ und $N$ ist die Menge
    aller geordneten Paare $(m, n)$ und wird mit $M \times N$ bezeichnet: \\
    $M \times N = \{(m, n) \;|\; m \in M,\; n \in N\}$.
    Dabei wird das geordnete Paar $(m, n)$ mengentheoretisch als
    $(m, n) = \{m, \{m, n\}\}$ definiert.
    Im Allgemeinen gilt $A \times B \not= B \times A$
    sowie $(a, b) \not= (b, a)$.
\end{Def}

\begin{Def}{Indizes}
    Man kann Elemente, Mengen usw. mit \begriff{Indizes} versehen, um sie zu
    unterscheiden.
    Sei $I$ eine Indexmenge und für jedes $i \in I$ sei $A_i$ Menge.
    Dann ist $\prod_{i \in I} A_i = \{(a_i)_{i \in I} \;|\;
    \forall_{i \in I}\; a_i \in A_i\}$, \\
    $\bigcap_{i \in I} A_i = \{x \;|\; \forall_{i \in I}\; x \in A_i\}$ und
    $\bigcup_{i \in I} A_i = \{x \;|\; \exists_{i \in I}\; x \in A_i\}$.
\end{Def}

\begin{Def}{zweistellige Relation}
    Sei $A$ eine Menge.
    Eine Teilmenge $R \subseteq A \times A$ heißt
    \begriff{zweistellige Relation} auf $A$.
    Statt $(a,b) \in R$ schreibt man oft $aRb$ oder $a \sim_R b$.
\end{Def}

\begin{xDef}{Äquivalenzrelation}{Aquivalenzrelation@Äquivalenzrelation}
    Eine Relation $R \subseteq A \times A$ heißt
    \xbegriff{Äquivalenzrelation}{Aquivalenzrelation@Äquivalenzrelation},
    falls sie reflexiv, symmetrisch und transitiv ist.
    $R$ ist \begriff{reflexiv}, falls $\forall_{a \in A}\; aRa$.
    $R$ ist \begriff{symmetrisch}, falls
    $\forall_{a, b \in A}\; aRb \Rightarrow bRa$.
    $R$ ist \begriff{transitiv}, falls
    $\forall_{a, b, c \in A}\; aRb \land bRc \Rightarrow aRc$. \\
    Beispiele für Äquivalenzrelationen sind Gleichheit und
    "`Restrelation"' (gleicher Rest bei Division durch feste Zahl).
\end{xDef}

\begin{xDef}{Äquivalenzklasse}{Aquivalenzklasse@Äquivalenzklasse}
    Seien $\sim$ eine Äquivalenzrelation auf der Menge $A$ und $a \in A$.
    Dann ist die \xbegriff{Äquivalenzklasse}{Aquivalenzklasse@Äquivalenzklasse}
    $[a]$ definiert als $[a] = \{b \in A \;|\; b \sim a\}$.
\end{xDef}

\begin{Lemma}{Äquivalenzklassen}
    Seien $A$ eine Menge, $\sim$ Äquivalenzrelation auf $A$ und $a, b \in A$.
    Dann ist $[a] \cap [b] \not= \emptyset \;\Leftrightarrow\; a \sim b$
    und im Falle $a \sim b$ gilt $[a] = [b]$.
\end{Lemma}

\begin{Def}{Partition}
    Seien $I$ eine Indexmenge, $A$ eine Menge und $A_i \subseteq A$ mit
    $A_i \not= \emptyset$ für jedes $i \in I$.
    $A$ heißt \begriff{disjunkte Vereinigung} der $A_i$ bzw. das System
    $\{A_i \;|\; i \in I\}$ heißt \begriff{Partition} von $A$, falls
    $A = \bigcup_{i \in I} A_i$ und
    $A_i \cap \left(\bigcup_{j \in I,\; j \not= i} A_j\right) = \emptyset$.
\end{Def}

\begin{Satz}{Äquivalenzklassen als Partition}
    Seien $\sim$ eine Äquivalenzrelation auf der Menge $A$ und
    $\{[a] \;|\; a \in A\}$ die Menge aller Äquivalenzklassen auf $A$.
    Dann bilden diese eine Partition von $A$. \\
    \emph{Vorsicht:} Die "`Liste"' $\{[a] \;|\; a \in A\}$ ist redundant, eine
    Äquivalenzklasse kann auch für $a \not= b$ mehrfach vorkommen.
    Diese wird jedoch auch nur einmal "`gezählt"'.
\end{Satz}

\begin{Satz}{Äquivalenzrelation aus Partition}
    Sei $A = \bigcup_{i \in I} A_i$ eine Partition von $A$.
    Definiere $a \sim b$ für $a, b \in A$ durch
    $a \sim b \;\Leftrightarrow\; \exists_{i \in I}\; a, b \in A_i$.
    Dann ist $\sim$ eine Äquivalenzrelation und die Äquivalenzklassen sind
    genau die $A_i$.
\end{Satz}

\begin{Def}{Ordnungsrelation}
    Sei $A \not= \emptyset$ eine Menge.
    Eine Relation $R \subseteq A \times A$ heißt
    \xbegriff{(teilweise) Ordnung}{teilweise Ordnung@Ordnungsrelation},
    falls sie reflexiv, antisymmetrisch und transitiv ist.
    $R$ ist \begriff{antisymmetrisch}, falls
    $\forall_{a, b \in A}\; aRb \land bRa \Rightarrow a = b$.
    Beispiele für Ordnungsrelationen sind $\le$, die Teilbarkeitsrelation $|$
    und Mengeninklusion $\subseteq$ auf der Potenzmenge einer Menge.
\end{Def}

\begin{Def}{lineare Ordnung}
    Sei $\le$ eine teilweise Ordnung auf $A$.
    $\le$ heißt \begriff{lineare/totale Ordnung}, falls
    $\forall_{a, b \in A}\; (a \le b) \lor (b \le a)$.
\end{Def}

\begin{Def}{minimale/kleinste Elemente}
    Seien $\le$ eine teilweise Ordnung auf $A$ sowie $B \subseteq A$. \\
    Dann heißt $b \in B$ \begriff{minimales Element} von $B$, falls
    $\forall_{c \in B}\; c \le b \Rightarrow c = b$. \\
    $b \in B$ heißt \begriff{kleinstes Element} von $B$, falls
    $\forall_{c \in B}\; b \le c$
    (analog: \begriff{maximales/größtes Element}).
\end{Def}

\begin{Def}{untere Schranke}
    Seien $\le$ eine teilweise Ordnung auf $A$ sowie $B \subseteq A$.
    Ein Element $a \in A$ heißt \begriff{untere Schranke} von $B$,
    falls $\forall_{b \in B}\; a \le b$
    (analog: \begriff{obere Schranke}).
\end{Def}

\begin{Def}{Wohlordnung}
    Sei $\le$ eine teilweise Ordnung auf $A$.
    $\le$ heißt \begriff{Wohlordnung} (und $A$ heißt wohlgeordnet), falls jede
    nicht-leere Teilmenge von $A$ ein kleinstes Element besitzt.
\end{Def}

\begin{Def}{endliche/unendliche Mengen}
    Eine Menge heißt \begriff{endlich}, falls sie nur endlich viele Elemente
    besitzt, sonst \begriff{unendlich}.
\end{Def}

\begin{Satz}{Wohlordnungssatz}
    Jede Menge lässt sich wohlordnen.
\end{Satz}

\begin{Satz}{Auswahlaxiom}
    Seien $I$ eine Indexmenge und $\{A_\alpha \;|\; \alpha \in I\}$ ein System
    von nicht-leeren Mengen $A_\alpha$.
    Dann gibt es eine Auswahlfunktion von $I$ in
    $\bigcup_{\alpha \in I} A_\alpha$, die jedem $\alpha \in I$ ein
    $x_\alpha \in A_\alpha$ zuordnet.
\end{Satz}

\begin{Satz}{\name{Zorn}sches Lemma}
    Sei $\le$ eine teilweise Ordnung auf $A$.
    Eine Kette in $A$ ist eine Teilmenge $K \subseteq A$ so, dass $\le$
    eingeschränkt auf $K$ die Menge $K$ zur linear geordneten Teilmenge macht.
    Ist $A$ nicht-leer und besitzt jede Kette $K$ in $A$ eine obere Schranke
    in $A$, so hat $A$ selbst maximale Elemente.
\end{Satz}

\begin{Bem}
    Wohlordnungssatz, Auswahlaxiom und \name{Zorn}sches Lemma sind echte
    Axiome, d.\,h. ihre Aussage oder ihre Negation erzeugen keinen Widerspruch
    zu den Axiomen der Mengenlehre.
    Die drei Sätze sind äquivalent, d.\,h. sie gelten entweder alle gleichzeitig
    oder keines von ihnen gilt.
    Man sollte jedoch besser die Richtigkeit voraussetzen, da manche Beweise
    auf ihrer Gültigkeit beruhen.
    Speziell das Auswahlaxiom gibt keine explizite Auswahlfunktion an, sonst
    besagt nur, dass es eine gibt.
\end{Bem}

\section{%
    Vollständige Induktion%
}

\begin{Satz}{vollständige Induktion}
    Sei $A(n)$ eine Aussageform mit $n \in \natural$.
    Wenn $A(1)$ (\begriff{Induktionsan\-fang}) und
    $A(n) \;\Rightarrow\; A(n + 1)$ (\begriff{Induktionsschritt}) gilt,
    dann ist $\{m \in \natural \;|\; A(m) \text{ wahr}\} = \natural$. \\
    Dieses Beweisverfahren heißt \begriff{vollständige Induktion}.
\end{Satz}

\begin{Bem}
    Oft benutzt man als Induktionsvoraussetzung nicht nur $A(n)$, sondern
    mehrere der $A(m)$ mit $m \le n$.
    Der Induktionsanfang kann auch eine andere natürliche oder negative ganze
    Zahl $n_0$ sein.
    Die Aussage gilt dann entsprechend für alle $k \in \integer$ mit
    $k \ge n_0$.
\end{Bem}

\section{%
    Abbildungen%
}

\begin{Def}{Abbildung}
    Seien $A$ und $B$ Mengen.
    Eine \begriff{Abbildung} $f$ (auch Funktion) von $A$ nach $B$ ist eine
    Relation $f \subseteq A \times B$ mit den Eigenschaften
    $\forall_{a \in A} \exists_{b \in B}\; (a, b) \in f$ (Vorbereich ist $A$)
    und $\forall_{a \in A} \forall_{b_1, b_2 \in B}\;
    (a, b_1) \in f \land (a, b_2) \in f \Rightarrow b_1 = b_2$
    (Nacheindeutigkeit). \\
    Man schreibt dann $f: A \rightarrow B$ und anstatt $(a, b) \in f$ schreibt
    man $a \mapsto b$ oder $b = f(a)$. \\
    Die Teilmenge $f = \{(a, f(a)) \in A \times B\}$ von $A \times B$ heißt
    \begriff{Graph} der Abbildung $f$.
\end{Def}

\begin{Bem}
    Abbildungen können durch Graphen und durch Pfeildiagramme visualisiert
    werden.
    Entsprechend können Abbildungen als Teilmengen des kartesischen Produkts
    $A \times B$ z.\,B. als Mengenlisten (bei endlicher Menge $A$) oder als
    definierende Aussageform wie \\
    $f = \{(a, b) \in A \times B \;|\; \text{Aussageform f"ur } f(a)\}$
    festgelegt werden.
\end{Bem}

\begin{Bem}
    Seien $f, g: A \rightarrow B$ Abbildungen.
    Dann ist $f = g$ genau dann, wenn $f$ und $g$ als Teilmengen von
    $A \times B$ gleich sind, d.\,h. $f(a) = g(a)$ für alle $a \in A$ ist.
\end{Bem}

\begin{Def}{Definitions-/Bildbereich}
    Sei $f: A \rightarrow B$ eine Abbildung.
    Dann ist $A$ der \begriff{Definitionsbereich} von $f$
    und die Teilmenge $\im f = \{b \in B \;|\; \exists_{a \in A}\; f(a) = b\}$
    heißt \begriff{Bild} von $f$. \\
    Für $X \subseteq A$ ist die \begriff{Einschränkung} $f|_X$ von $f$ auf $X$
    definiert als $f|_X = \{(a, b) \in f \;|\; a \in X\}$. \\
    $f(X)$ (Bild der Teilmenge $X$ von $A$ unter $f$) ist definiert als
    $f(X) = \im f|_X$.
\end{Def}

\begin{Def}{Komposition}
    Seien $f: A \rightarrow B$, $g: B \rightarrow C$ Abbildungen.
    Die \begriff{Hintereinanderausführung}/
    \begriff{Komposition} $g \circ f = gf$ ist definiert durch
    $g \circ f: A \rightarrow C$, $a \mapsto g(f(a))$.
\end{Def}

\begin{Def}{injektiv/surjektiv/bijektiv}
    Sei $f: A \rightarrow B$ eine Abbildung. \\
    $f$ ist \begriff{injektiv}, falls
    $\forall_{a, b \in A}\; f(a) = f(b) \Rightarrow a = b$. \qquad
    $f$ ist \begriff{surjektiv}, falls $\im f = B$
    (bzw. $\forall_{b \in B} \exists_{a \in A}\; b = f(a)$). \qquad
    $f$ ist \begriff{bijektiv}, falls $f$ injektiv und surjektiv ist. \\
    Eine bijektive Abbildung $f: A \rightarrow A$ einer Menge $A$ in sich
    selbst heißt \begriff{Permutation} von $A$.
\end{Def}

\begin{Def}{Umkehrrelation}
    Sei $f: A \rightarrow B$ eine Abbildung.
    Die \begriff{Umkehrrelation} $f^{-1}$ ist gegeben durch
    $f^{-1} = \{(b, a) \in B \times A \;|\; f(a) = b\}$.
    Für $U \subseteq B$ ist $f^{-1}(U) = \{a \in A \;|\; f(a) \in U\}$
    das \begriff{Urbild} von $U$ unter $f$.
    Für $U = \{b\}$ ($b \in B$) schreibt man $f^{-1}(b) = f^{-1}(\{b\})$. \\
    $f^{-1}$ ist genau dann eine Abbildung, wenn $f$ bijektiv ist.
\end{Def}

\begin{Def}{Identität}
    Sei $A$ eine Menge.
    Die Abbildung $\id_A: A \rightarrow A$, $a \mapsto a$ heißt
    \begriff{Identität}.
\end{Def}

\begin{Lemma}{Identität als neutrales Element}
    Sei $f: A \rightarrow B$ eine Abbildung. \\
    Dann ist $\id_B \circ f = f \circ \id_A = f$, d.\,h. die Identität ist
    neutrales Element bzgl. der Komposition.
\end{Lemma}

\begin{Satz}{$f$ bijektiv $\Leftrightarrow$ es gibt eine Umkehrabbildung}
    Sei $f: A \rightarrow B$ eine Abbildung.
    Dann ist $f$ bijektiv genau dann, wenn es eine Abbildung
    $g: B \rightarrow A$ gibt mit $f \circ g = \id_B$ und
    $g \circ f = \id_A$. \\
    Die Abbildung $g$ ist dann eindeutig bestimmt und identisch mit der
    Umkehrrelation $f^{-1}$. \\
    $g$ heißt Umkehrabbildung und wird mit $f^{-1}$ bezeichnet.
    $f^{-1}$ ist ebenfalls bijektiv.
\end{Satz}

\begin{Satz}{Komposition}
    Die Komposition von injektiven, surjektiven bzw. bijektiven Abbildungen
    ist injektiv, surjektiv bzw. bijektiv.
\end{Satz}

\begin{Satz}{Kürzen von injektiven Abbildungen}
    Seien $f, g: A \rightarrow B$, $h: B \rightarrow C$ Abbildungen
    mit $h$ injektiv.
    Ist $h \circ f = h \circ g$, dann ist $f = g$ (injektive Abbildungen kann
    man links kürzen).
\end{Satz}

\begin{Satz}{Kürzen von surjektiven Abbildungen}
    Seien $f, g: A \rightarrow B$, $h: C \rightarrow A$ Abbildungen
    mit $h$ surjektiv.
    Ist $f \circ h = g \circ h$, dann ist $f = g$ (surjektive Abbildungen kann
    man rechts kürzen).
\end{Satz}

\begin{xDef}{Mächtigkeit}{Machtigkeit@Mächtigkeit}
    Sei $M$ eine Menge.
    Dann ist $|M|$ die \xbegriff{Mächtigkeit}{Machtigkeit@Mächtigkeit}
    von $M$ und wie folgt definiert: \\
    Gibt es eine Bijektion zwischen $M$ und
    $\{1, \ldots, n\}$, dann ist $|M| = n$ und $M$ ist
    \begriff{endliche Menge}. \\
    Gibt es eine Bijektion zwischen $M$ und $\natural$, dann ist
    $|M| = \aleph_0$ und $M$ ist
    \xbegriff{abzählbar unendlich}{abzahlbar unendlich@abzählbar unendlich}. \\
    Ist $M$ weder endliche noch abzählbar unendliche Menge, so ist
    $M$ \xbegriff{überabzählbar}{uberabzahlbar@überabzählbar}.
\end{xDef}

\begin{Bem}
    Die Elemente einer abzählbaren Menge lassen sich auflisten. \\
    Auf der "`Klasse"' aller Mengen kann man eine Äquivalenzrelation $\sim$
    definieren durch $A \sim B \;\Leftrightarrow\;
    \exists f: A \rightarrow B \text{ bijektiv}$.
    Die Äquivalenzklassen bilden die \begriff{Kardinalitäten} oder
    \begriff{Mächtigkeiten}.
\end{Bem}

\begin{Satz}{Mächtigkeiten}
    $\integer$ und $\rational$ sind abzählbar.
    $\real$ und $\complex$ sind überabzählbar.
    Die Vereinigung abzählbar vieler abzählbarer Mengen ist abzählbar.
    Für eine Menge $M$ gilt $|M| \not= |P(M)|$.
\end{Satz}

\section{%
    \emph{Zusätzliches}: Gruppen, Körper, Ringe%
}

\begin{Def}{binäre Operation}
    Eine \begriff{binäre Operation} $B$ auf einer Menge $M$ ist eine
    Abbildung  \\
    $B: M \times M \rightarrow M$.
    Sie wird gewöhnlich mit einem Symbol (z.\,B. $+$) bezeichnet
    und man schreibt $B(m_1, m_2) = m_1 + m_2$ mit $m_1, m_2 \in M$.
\end{Def}

\begin{Def}{Gruppe}
    Eine \begriff{Gruppe} besteht aus einer Menge $G$ und einer binären
    Operation $\circ: G \times G \rightarrow G$, sodass
    $\forall_{a, b, c \in G}\; (a \circ b) \circ c = a \circ (b \circ c)$
    (\begriff{Assoziativität}) und es ein Element $e \in G$ gibt, sodass
    $\forall_{a \in G}\; e \circ a = a$ (\begriff{neutrales Element}) und
    $\forall_{a \in G} \exists_{a' \in G}\; a' \circ a = e$
    (\begriff{inverses Element}). \\
    Gilt zusätzlich $\forall_{a, b \in G}\; a \circ b = b \circ a$
    (\begriff{Kommutativität}), so heißt die Gruppe eine
    \begriff{abelsche Gruppe}.
\end{Def}

\begin{Bem}
    In einer Gruppe $G$ gibt es genau ein neutrales Element und zu jedem
    $a \in G$ genau ein Inverses $a' \in G$.
    Außerdem ist $(a')' = a$.
\end{Bem}

\begin{Def}{Körper}
    Ein \begriff{Körper} besteht aus einer Menge $K$ und zwei binären
    Operationen $\boldsymbol{+}$ und $\boldsymbol{\cdot}$, sodass
    $K$ bzgl. $\boldsymbol{+}$ eine abelsche Gruppe mit Nullelement $0$ ist,
    $K^\ast = K \setminus \{0\}$ bzgl. $\boldsymbol{\cdot}$ eine abelsche
    Gruppe ist
    und $\forall_{a, b, c \in K}\; a \cdot (b + c) = a \cdot b + a \cdot c$
    sowie $\forall_{a, b, c \in K}\; (b + c) \cdot a = b \cdot a + c \cdot a$
    (\begriff{Distributivität} von $\boldsymbol{\cdot}$ über $\boldsymbol{+}$
    auf beiden Seiten).
\end{Def}

\begin{Def}{Ring}
    Ein \begriff{Ring} besteht aus einer Menge $R$ und zwei binären Operationen
    $\boldsymbol{+}$ und $\boldsymbol{\cdot}$, sodass
    $K$ bzgl. $\boldsymbol{+}$ eine abelsche Gruppe ist
    sowie $\boldsymbol{\cdot}$ assoziativ und auf beiden Seiten distributiv
    über $\boldsymbol{+}$ ist. \\
    Hat $R$ ein neutrales Element $1$ bzgl. $\boldsymbol{\cdot}$, dann
    ist $R$ ein \begriff{Ring mit Eins} ($1$ heißt \begriff{Einselement}). \\
    Ist $\boldsymbol{\cdot}$ kommutativ, so heißt $R$
    \begriff{kommutativer Ring}.
\end{Def}

\begin{Bem}
    In einem Ring $R$ gilt $0 \cdot a = a \cdot 0 = 0$ für jedes Element
    $a \in R$.
\end{Bem}

\section{%
    \emph{Zusätzliches}: Projekt 1 (Mengen und Abbildungen)%
}

\begin{Satz}{Menge aller Mengen}
    Es gibt keine Menge aller Mengen.
\end{Satz}

\begin{Bem}
    Ansonsten gäbe es eine surjektive Abbildung $f: M \rightarrow P(M)$,
    da jedes Element $T \in P(M)$ eine Teilmenge von $M$ ist, also eine Menge
    und daher auch ein Element von $M$ ($T \in M$).
    Definiere $f(T) = T$ für alle $T \in P(M)$ und $f(T) = \emptyset$ sonst.
\end{Bem}

\begin{Satz}{\name{Schröder}-\name{Bernstein}}
    Seien $A$ und $B$ Mengen und $f: A \rightarrow B$, $g: B \rightarrow A$
    injektive Abbildungen.
    Dann sind $A$, $B$ gleichmächtig (d.\,h. es gibt eine Bijektion zwischen
    $A$ und $B$).
\end{Satz}

\pagebreak
