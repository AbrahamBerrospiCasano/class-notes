\section{%
    PDE-Klassen und klassische Lösungen%
}

\begin{Bem}
    Das Ziel dieses Kapitels ist es, für die vier wichtigsten linearen PDEs die klassischen
    Lösungen und deren Eigenschaften zu bestimmen.
    Dabei werden Invarianzen ausgenutzt, um die PDE zu vereinfachen.
    Außerdem werden die PDEs zweiter Ordnung klassifiziert.
\end{Bem}

\subsection{%
    Advektionsgleichung%
}

\subsubsection{%
    Konstante Advektionsgeschwindigkeit%
}

\begin{Def}{\name{Cauchy}problem für Advektionsgleichung}
    Seien $\Omega := \real^d$,
    $T := \infty$,
    $\Omega_T := \Omega \times (0, T)$
    $b \in \real^d$ die konst. \begriff{Advektionsgeschwindigkeit} und
    $u_0 \in \C^1(\Omega)$ der \begriff{Anfangswert}.\\
    Das Problem, ein $u \in \C^1(\Omega_T) \cap \C^0(\overline{\Omega_T})$ zu bestimmen mit
    $\partial_t u + \div(bu) = 0$ in $\Omega_T$ und $u(\cdot, 0) = u_0$ in $\Omega$,
    heißt \begriff{\name{Cauchy}problem für die Advektionsgleichung}.
\end{Def}

\begin{Bem}
    Diese Form der Advektionsgleichung heißt \begriff{Divergenzform}/\begriff{Erhaltungsform}.\\
    In der Literatur findet man auch $\partial_t u + b \nabla u = 0$.
    Für $b\colon \Omega \to \real^d$ \begriff{divergenzfrei} (d.\,h. $\div b = 0$)
    sind beide Formen äquivalent, da
    $\div(bu) = \cancel{(\div b) u} + b \nabla u = b \nabla u$.\\
    Cauchyprobleme heißen auch \begriff{Anfangswertprobleme (AWPs)},
    im Gegensatz zu \begriff{Anfangs-Rand"-wertproblemen (ARWPs)} oder
    \begriff{Randwertproblemen (RWPs)}.
\end{Bem}

\linie

\begin{Satz}{Translationsinvarianz}
    Sei $u$ eine klassische Lösung des Cauchyproblems.\\
    Dann gilt $\forall_{(x, t) \in \Omega_T} \forall_{s \in (-t, T-t)}\;
    \frac{\d}{\ds} u(x + bs, t + s) = 0$.
\end{Satz}

\begin{Bem}
    Die Linien $\Gamma := \{(x_0 + bs, s) \;|\; s \in (0, T)\} \subset \Omega_T$,
    entlang denen eine klassische Lösung des Cauchyproblems konstant ist,
    heißen \begriff{charakteristische Kurven} oder \begriff{Charakteristiken}.
\end{Bem}

\begin{Satz}{Ex./Eind. der Traveling-Wave-Lösung}
    Die Funktion $u(x, t) := u_0(x - bt)$ ist die eindeutige Lösung des Cauchyproblems
    und heißt \begriff{Traveling-Wave-Lösung}.
\end{Satz}

\linie

\begin{Satz}{$L^\infty$-Stabilität}
    Für $u_0 \in \C^1(\Omega) \cap L^\infty(\Omega)$ gilt für die Lösung $u$ des Cauchyproblems\\
    $\forall_{t \in (0, T)}\; \norm{u(\cdot, t)}_{L^\infty} \le \norm{u_0}_{L^\infty}$.
\end{Satz}

\begin{Satz}{Maximum-/Minimumprinzip}
    Für $u_0 \in \C^1(\Omega) \cap L^\infty(\Omega)$ gilt für die Lösung $u$ des Cauchyproblems
    $\forall_{(x, t) \in \Omega_T}\;
    \inf_{\overline{x} \in \Omega} u_0(\overline{x}) \le u(x, t) \le
    \sup_{\overline{x} \in \Omega} u_0(\overline{x})$.
\end{Satz}

\begin{Bem}
    Die Lösung nimmt ihr Maximum/Minimum auf dem Rand $\partial\Omega_T$ an.
\end{Bem}

\begin{Satz}{st. Abh. von Anfangsdaten}
    Seien $u, u'$ zwei Lösungen des Cauchyproblems zu den Anfangsdaten
    $u_0, u_0' \in \C^1(\Omega) \cap L^\infty(\Omega)$ mit identischem $b$.
    Dann gilt\\
    $\forall_{t \in (0, T)}\;
    \norm{u(\cdot, t) - u'(\cdot, t)}_{L^\infty} \le \norm{u_0 - u_0'}_{L^\infty}$.
\end{Satz}

\linie

\begin{Satz}{keine stetige Abhängigkeit von $b$}
    Es existiert keine $t$-abhängige Konstante $C(t)$ mit
    $\forall_{t \in (0, T)}\;
    \norm{u(\cdot, t) - u'(\cdot, t)}_\infty \le C(t) \norm{b - b'}$
    für alle Anfangswerte $u_0 \in \C^1(\Omega) \cap L^\infty(\Omega)$ mit
    $\norm{u_0}_{L^\infty} \le 1$,
    wobei $u, u'$ zwei Lösungen des Cauchyproblems zu identischem $u_0$, aber unterschiedlichem
    $b, b' \in \real^d$ sind.
\end{Satz}

\begin{Bem}
    Damit existiert insbesondere keine $t$-unabhängige Konstante.
    Die Einschränkung $\norm{u_0}_{L^\infty} \le 1$ ist erforderlich, weil die linke Seite mit
    $\norm{u_0}_{L^\infty}$ skaliert.
    Der Satz zeigt, dass Transportprobleme ohne Dif"|fusion bzgl. der Analysis unschöne
    Eigenschaften haben können und spezielle analytische Werkzeuge erfordern,
    z.\,B. neue Normen (\begriff{Totalvariation}) und Räume (\begriff{BV-Räume},
    Räume von Funktionen beschränkter Variation).
\end{Bem}

\subsubsection{%
    Verallgemeinerung 1: Beschränktes Gebiet%
}

\begin{Bem}
    Ist $\Omega$ beschränkt, so betrachte das ARWP
    $\partial_t u + \div_x(bu) = 0$ in $\Omega_T$,
    $u(\cdot, 0) = u_0$ in $\Omega$ und
    $u = g$ auf $\Gamma_\text{in} \times (0, T)$
    mit Dirichlet-RBen (es sind aber auch Neumann-RBen oder gemischte RBen möglich).
    Dabei bezeichnet $\Gamma_\text{in}$ den sog. \begriff{Einflussrand}
    $\Gamma_\text{in} := \{x \in \partial\Omega \;|\; b^\tp n(x) < 0\}$
    (die Stellen auf dem Rand, auf denen $b$ in $\Omega$ hinein zeigt).
    
    Falls $x - bt \notin \Omega$ ist, so gibt es einen Schnittpunkt der Charakteristik durch
    $(x, t)$ und dem Zylinderrand $\partial\Omega \times [0, t)$.
    Der entsprechende Randwert wird in das Gebiet hineintransportiert, sodass man die Lösungsformel
    $u(x, t) := u_0(x - bt)$ für $\forall_{s \in (0, t)}\; x - bs \in \Omega$ und\\
    $u(x, t) := g(x - b\overline{t}, t - \overline{t})$ mit
    $\overline{t} := \min\{s \in (0, t) \;|\; x - bs \in \partial\Omega\}$ sonst
    ($s = \overline{t}$ ist die Zeitspanne, die man zurückgehen muss, damit $x - bs$ auf dem Rand
    liegt).
    
    Soll die Lösung zu allen Zeiten stetig auf $\overline{\Omega}$ sein
    (d.\,h. $u(\cdot, t) \in \C^0(\overline{\Omega})$), dann dürfen keine Randwerte auf
    $\partial\Omega \setminus \Gamma_\text{in}$ vorgegeben werden.
    Außerdem müssen Randwerte auf verschiedenen Zusammenhangskomponenten von $\Gamma_\text{in}$
    kompatibel sein, sonst ist keine Stetigkeit oder gar Diffb.keit zu erwarten.
    Zusätzlich müssen Rand- und Anfangswerte miteinander kompatibel sein, um Stetigkeit/Diffb.keit
    der Lösung zu ermöglichen, z.\,B. ist
    $\forall_{x_0 \in \Gamma_\text{in}}\; \lim_{x \to x_0} u_0(x) = g(x_0, 0)$
    notwendig für $u$ stetig.
\end{Bem}

\subsubsection{%
    Verallgemeinerung 2: Reaktions-/Quellterm%
}

\begin{Bem}
    Sei wieder $\Omega := \real^d$.
    Betrachte das AWP $\partial_t u + \div(bu) = q$ in $\Omega_T$ und $u(\cdot, 0) = u_0$ in
    $\Omega$ mit Quellterm $q \in \C^0(\Omega_T)$.
    Dann kann man die explizite Lösung durch Integration über die Charakteristik erhalten:
    $u(x, t) := u_0(x - bt) + \int_0^t q(x + (s-t)b, s)\ds$ für $(x, t) \in \Omega_T$.
\end{Bem}

\subsubsection{%
    Verallgemeinerung 3: Allgemeine Anfangsdaten%
}

\begin{Bem}
    Auch unstetige Daten wie $u_0(x) := \chi_{[-1,1]}(x)$ (für $d = 1$) sind physikalisch sinnvoll
    und die obige Lösungsformel ist auch wohldefiniert.
    Allerdings ist die resultierende Lösung nicht stetig diffb., d.\,h. keine klassische Lösung.
    Deswegen sind verallgemeinerte Lösungsbegrif"|fe wie der einer schwachen Lösung sinnvoll
    (siehe nächstes Kapitel).
\end{Bem}

\pagebreak

\subsubsection{%
    Verallgemeinerung 4: Nicht-lineare Konvektion%
}

\begin{Bem}
    Seien nun $\Omega := \real$ und $\Omega_T := \Omega \times (0, T)$.
    Betrachte die Konvektionsgleichung
    $\partial_t u + \partial_x(f(u)) = 0$ in $\Omega_T$ und
    $u(\cdot, 0) = u_0$ in $\Omega$
    mit $f \in \C^2(\real)$ nicht-linear.
    
    Zu einer Lösung $u \in \C^1(\overline{\Omega_T})$ sei
    $\gamma \in \C^1((0, T))$ mit $\gamma'(t) = f'(u(\gamma(t), t))$ und
    $\gamma(0) = x_0 \in \Omega$\\
    ($\gamma$ existiert nach dem Satz von Picard-Lindelöf, weil $f'$ L.-stetig in
    $u(\gamma(t), t) \in [u_\text{min}, u_\text{max}]$ ist).\\
    Dann kann man zu $u$ Charakteristiken $\Gamma := \{(\gamma(t), t) \;|\; t \in [0, T)\}$
    definieren.\\
    Die Lösung $u$ ist dann wieder konstant entlang Charakteristiken, da\\
    $\partial_t((\gamma(t), t)) \cdot \nabla_{(x,t)} u(\gamma(t), t)
    = (\gamma'(t), 1) \cdot \left.(\partial_x u, \partial_t u)^\tp\right|_{(\gamma(t), t)}
    = (\partial_t u + \partial_x f(u))|_{(\gamma(t), t)} = 0$.\\
    Wegen $\gamma'(t) = f'(u(\gamma(t), t))$ konstant (da $u(\gamma(t), t)$ konstant)
    sind die Charakteristiken wieder Geraden, allerdings
    haben die Geraden i.\,A. jeweils eine andere Steigung $\gamma'(0) = f'(u_0(x_0))$.
    
    Die Lösung $u(x, t)$ ist wieder vollständig durch $u_0$ definiert, falls die Charakteristiken
    $\Omega_T$ überdecken, sich selbst aber untereinander nicht schneiden.
    Sonst ist die klassische Lösung i.\,A. nur bis zu einer endlichen Zeit wohldefiniert.
    Zwei Charakteristiken ausgehend von $x_0, x_0'$ schneiden sich genau dann, wenn
    $x_0 + f'(u_0(x_0)) \overline{t} = x_0' + f'(u_0(x_0')) \overline{t}$
    für eine Zeit $\overline{t} \in (0, T)$.
    Durch Umformung bekommt man
    $\overline{t} = \frac{x_0' - x_0}{f'(u(x_0)) - f'(u(x_0'))} = -\frac{1}{f''(v)}$
    für ein $v$ zwischen $u(x_0)$ und $u(x_0')$.\\
    Weil $\overline{t} \in (0, T)$ gilt, ist es hinreichend, dass
    $T \le \inf_{v \in \real} \frac{1}{|f''(v)|} = (\norm{f''}_\infty)^{-1}$,
    damit sich keine Charakteristiken in $\Omega_T$ schneiden.
\end{Bem}

\linie

\begin{Satz}{lokale Existenz von klassischen Lösungen}\\
    Seien $f \in \C^2(\real)$ und $u_0 \in \C^1(\Omega)$ mit
    $\norm{f''}_\infty, \norm{u_0'}_\infty < \infty$.\\
    Dann gilt $\forall_{\overline{x} \in \real} \exists_{\varepsilon > 0} \exists_{T > 0}
    \exists_{u}\;
    \big[\text{$u$ klassische Lösung auf $B_\varepsilon(\overline{x}) \times (0, T)$}\big]$,\\
    wobei die Lösung $u(x, t) = u_0(x - tf'(u(x, t)))$ erfüllt.
\end{Satz}

\linie

\begin{Bsp}
    Betrachte die \begriff{\name{Burgers}gleichung}
    $\partial_t u + \partial_x (\frac{1}{2} u^2) = 0$, d.\,h. $f(u) := \frac{1}{2} u^2$.
    \begin{itemize}
        \item
        Verwendet man $u_0(x) := x$, so erhält man
        $u(x, t) = u_0(x - t u(x, t)) = x - t u(x, t)$\\
        $\iff u(x, t) = \frac{x}{t + 1}$ als Lösung, die sogar auf $\real \times (0, \infty)$
        definiert ist.
        
        \item
        Verwendet man $u_0(x) := -x$, so erhält man analog $u(x, t) = \frac{x}{t - 1}$.
        Diese Lösung ist nur für $T < 1$ wohldefiniert, weil sich alle Charakteristiken in
        $(x, t) = (0, 1)$ schneiden.
    \end{itemize}
    Trotz glatter Daten können sich also Unstetigkeiten entwickeln.
\end{Bsp}

\pagebreak

\subsection{%
    \name{Poisson}-Gleichung%
}

\subsubsection{%
    Gleichung%
}

\begin{Def}{\name{Poisson}-/\name{Laplace}-Gleichung}
    Für $\Omega \subset \real^d$ heißt $-\Delta u = 0$ in $\Omega$
    \begriff{\name{Laplace}-Gleichung} und für $f\colon \Omega \to \real$ in $\Omega$ heißt
    $-\Delta u = f$ in $\Omega$ \begriff{\name{Poisson}-Gleichung}.
\end{Def}

\begin{Bem}
    Lösungseindeutigkeit ist ohne weitere RBen nicht zu erwarten
    ($u(x) + (c + dx)$ ist Lösung, wenn $u$ Lösung ist).
    Lösungen der Laplace-Gleichung heißen auch \begriff{harmonisch}.
\end{Bem}

\linie

\begin{Satz}{Rotationsinvarianz}
    Seien $\Omega, f$ \begriff{rotationssymmetrisch},
    d.\,h. es gibt ein $O \in \real^{d \times d} \setminus \{I_d\}$ orthogonal mit
    $\Omega = O\Omega$ und $f = f \circ O$,
    und $u \in \C^2(\Omega)$ eine klassische Lösung der Poisson-Gleichung.
    Dann ist auch $v \in \C^2(\Omega)$ mit $v(x) := u(Ox)$ eine klassische Lösung.
\end{Satz}

\begin{Bem}
    Es gilt \begriff{Translationsinvarianz}, d.\,h. ist
    $t \in \real^d \setminus \{0\}$ mit $\Omega = \Omega + t$, $f(\cdot) = f(\cdot + t)$,
    dann ist auch $v(x) := u(x + t)$ eine klassische Lösung.
    Die Translations-/Rotationsinvarianz gilt insbesondere für die Laplace-Gleichung,
    weil $f \equiv 0$ translations-/rotationsinvariant ist.
\end{Bem}

\subsubsection{%
    Fundamentallösung der \name{Laplace}-Gleichung%
}

\begin{Bem}
    Es soll eine explizite Lsg. $u \in \C^2(\Omega)$
    für die Laplace-Gleichung hergeleitet werden, wobei $\Omega := \real^d \setminus \{0\}$.
    Sei $u$ rot.symm., d.\,h. es gibt $v \in \C^2((0, \infty))$ mit
    $u(x) = v(\norm{x})$ für alle $x \in \Omega$.
    Dann folgt mit $r := \norm{x}$, dass
    $\partial_{x_i} u(x) = v'(r) \cdot \frac{x_i}{r}$,
    also $\partial_{x_i}^2 u(x) =
    v''(r) \cdot \frac{x_i^2}{r^2} + v'(r) \cdot \frac{r - x_i^2/r}{r^2}$.
    Ist $u$ harmonisch, so gilt
    $0 = \Delta u(x) =
    v''(r) + v'(r) \cdot \left(\frac{d}{r} - \frac{1}{r}\right)
    = v''(r) + v'(r) \cdot \frac{d-1}{r}$,
    womit man die DGL $v''(r) + v'(r) \cdot \frac{d-1}{r} = 0$ für $v(r)$ erhält.
    Sei $v$ streng monoton, d.\,h. oBdA $v'(r) > 0$, dann bekommt man
    $(\ln(v'(r)))' = \frac{v''(r)}{v'(r)} = \frac{1-d}{r}$.
    Daraus folgt $\ln(v'(r)) = (1 - d)\ln(r) + \ln(a) = \ln(ar^{1-d})$
    mit $a > 0$.
    Man erhält die DGL $v'(r) = ar^{1-d}$ mit Lösung
    $v(r) = ar + b$ für $d = 1$,
    $v(r) = a\ln(r) + b$ für $d = 2$ und
    $v(r) = \frac{a}{(2-d) r^{d-2}} + b$ für $d \ge 3$ mit $b \in \real$.
\end{Bem}

\linie

\begin{Def}{Fundamentallösung}
    Sei $\Omega := \real^d \setminus \{0\}$ mit $d > 1$.
    Dann heißt die Funktion $\Phi \in \C^\infty(\Omega)$ mit
    $\Phi(x) := -\frac{1}{2\pi} \cdot \ln(\norm{x})$ für $d = 2$ und
    $\Phi(x) := \frac{1}{(d-2)\omega_d} \cdot \frac{1}{\norm{x}^{d-2}}$ für $d \ge 3$
    \begriff{Fundamentallösung} der Laplace-Gleichung
    mit $\omega_d := |\partial B_1(0)|$ der Oberfläche der Einheitssphäre in $\real^d$.
\end{Def}

\begin{Bem}
    $\Phi$ hat in $0$ eine Singularität und ist eine klassische Lösung der Laplace-Gleichung.
\end{Bem}

\linie

\begin{Lemma}{Eigenschaften von $\Phi$}
    \begin{enumerate}
        \item
        $\forall_{\varepsilon>0}\; \int_{B_\varepsilon(0)} \Phi(x)\dx < \infty$,\quad
        $\int_{B_\varepsilon(0)} \Phi(x)\dx \xrightarrow{\varepsilon \to 0} 0$
        
        \item
        $\Phi \in L^1_\loc(\real^d)$
        
        \item
        $\Phi(\varepsilon e_1) \varepsilon^{d-1} \xrightarrow{\varepsilon \to 0} 0$
        
        \item
        $\forall_{\varepsilon>0}\;
        \int_{\partial B_\varepsilon(0)} \nabla\Phi(x) \cdot n\dsigma(x) = -1$
    \end{enumerate}
\end{Lemma}

\subsubsection{%
    Faltungslösung der \name{Poisson}-Gleichung%
}

\begin{Satz}{Faltung und Dif{}ferentiation}
    Seien $u \in L^1_\loc(\real^d)$ und $\phi \in \C^m_0(\real^d)$.\\
    Dann gilt für die \begriff{Faltung} $u \ast \phi$ mit
    $(u \ast \phi)(x) := \int_{\real^d} u(x-y)\phi(y) \dy
    = \int_{\real^d} u(y)\phi(x-y) \dy$, dass
    $u \ast \phi \in \C^m(\real^d)$ mit
    $\forall_{|\beta| \le m}\; \partial^\beta (u \ast \phi) = u \ast \partial^\beta \phi$.
\end{Satz}

\begin{Satz}{Faltungslösung}
    Seien $\Omega := \real^d$ mit $d \ge 2$ und $f \in \C^2_0(\Omega)$.\\
    Dann ist $u := \Phi \ast f$ eine klassische Lösung der Poisson-Gleichung.
\end{Satz}

\pagebreak

\subsubsection{%
    Mittelwerteigenschaft/Maximumprinzip harm. Funktionen%
}

\begin{Def}{Mittelwert}
    Für $K \subset \real^d$ mit $0 < |K| < \infty$ und $u \in L^1(K)$ ist
    $\fint_K u(x) \dx := \frac{1}{|K|} \int_K u(x) \dx$
    der \begriff{Mittelwert} von $u$ auf $K$.
    Analog ist für $0 < |\partial K| < \infty$ und $u \in L^1(\partial K)$
    der Mittelwert von $u$ auf $\partial K$ definiert durch
    $\fint_{\partial K} u(x) \dsigma(x) :=
    \frac{1}{|\partial K|} \int_{\partial K} u(x) \dsigma(x)$.
\end{Def}

\begin{Satz}{Mittelwerte harm. Fkt.en}\\
    Seien $u \in \C^2(\Omega)$ harmonisch, $x \in \Omega$ und $r > 0$ mit
    $\overline{B_r(x)} \subset \Omega$.\\
    Dann ist $\fint_{B_r(y)} u(y) \dy = u(x) = \fint_{\partial B_r(x)} u(y) \dsigma(y)$.
\end{Satz}

\linie

\begin{Satz}{Maximumprinzip für harm. Fkt.en}\\
    Seien $\Omega \subset \real^d$ of"|fen und beschränkt sowie
    $u \in \C^2(\overline{\Omega})$ harmonisch.
    Dann gilt:
    \begin{enumerate}
        \item
        $u$ nimmt das Maximum auf dem Rand an, d.\,h.
        $\max_{x \in \overline{\Omega}} u(x) = \max_{x \in \partial\Omega} u(x)$.
        
        \item
        Wenn $\Omega$ zusammenhängend ist und $\exists_{x \in \Omega}\;
        u(x) = \max_{y \in \overline{\Omega}} u(y)$, dann ist $u$ konstant auf $\Omega$.
    \end{enumerate}
\end{Satz}

\begin{Bem}
    Analog gelten folgende Verallgemeinerungen.
    \begin{itemize}
        \item
        \begriff{verallg. Max.prinzip}:\\
        Für $u \in \C^2(\Omega) \cap \C^0(\overline{\Omega})$ mit $-\Delta u = f \le 0$
        nimmt $u$ das Maximum auf dem Rand an.
        
        \item
        \begriff{verallg. Min.prinzip}:
        wie eben mit $-\Delta u = f \ge 0$ und Minimum
        
        \item
        \begriff{Vergleichsprinzip}:
        Für $u, v \in \C^2(\Omega) \cap \C^0(\overline{\Omega})$ mit $-\Delta u \le -\Delta v$ in
        $\Omega$ und $u \le v$ auf $\partial\Omega$ gilt $u \le v$ in $\Omega$
        (wähle $w := u - v$ im verallg. Max.prinzip).
    \end{itemize}
\end{Bem}

\subsubsection{%
    Eindeutigkeit und stetige Abhängigkeit beim \name{Poisson}-RWP%
}

\begin{Satz}{Eindeutigkeit}
    Seien $\Omega \subset \real^d$ of"|fen und beschränkt, $g \in \C^0(\partial\Omega)$ und
    $f \in \C^0(\Omega)$.\\
    Dann gibt es höchstens eine Lösung $u \in \C^2(\Omega) \cap \C^0(\overline{\Omega})$ des
    \begriff{\name{Poisson}-RWPs}\\
    $-\Delta u = f$ in $\Omega$ und $u = g$ auf $\partial\Omega$.
\end{Satz}

\linie

\begin{Satz}{st. Abh. von Randdaten}
    Seien $u, u' \in \C^2(\Omega) \cap \C^0(\overline{\Omega})$ Lsg.en des Poisson-RWPs
    mit identischem $f \in \C^0(\Omega)$, aber unterschiedlichem $g, g' \in \C^0(\partial\Omega)$.
    Dann gilt $\norm{u - u'}_\infty \le \norm{g - g'}_\infty$.
\end{Satz}

\begin{Satz}{st. Abh. von rechter Seite}
    Seien $u, u' \in \C^2(\Omega) \cap \C^0(\overline{\Omega})$ Lösungen des Poisson-RWPs
    mit identischem
    $g \in \C^0(\partial\Omega)$, aber unterschiedlichem
    $f, f' \in \C^0(\Omega)$.\\
    Dann gilt $\norm{u - u'}_\infty \le C \norm{f - f'}_\infty$
    mit $C := \frac{R^2}{2}$ und $R := \sup_{x \in \Omega} \norm{x}$.
\end{Satz}

\subsubsection{%
    Regularität%
}

\begin{Satz}{$\C^\infty$-Regularität}
    Seien $\Omega := \real^d$ und $u \in \C^2(\Omega)$ harmonisch.
    Dann ist $u \in \C^\infty(\Omega)$.
\end{Satz}

\linie

\begin{Def}{$\varepsilon$-Glättungskern}
    Sei $\eta \in \C^\infty_0(\real^d)$ definiert durch
    $\eta(x) := c \exp\!\left(\frac{1}{\norm{x}^2 - 1}\right)$ für $\norm{x} < 1$ und
    $\eta(x) := 0$ sonst, wobei $c \in \real$ mit $\int_{\real^n} \eta(x) \dx = 1$.\\
    Dann ist für $\varepsilon > 0$ der \begriff{$\varepsilon$-Glättungskern}
    $\eta_\varepsilon \in \C^\infty_0(\real^d)$ definiert durch
    $\eta_\varepsilon(x) := \frac{1}{\varepsilon^d} \eta(x/\varepsilon)$.
\end{Def}

\begin{Bem}
    Es gilt $\int_{\real^d} \eta_\varepsilon(x) \dx = 1$ und
    $\supp(\eta_\varepsilon) = \overline{B_\varepsilon(0)}$.
\end{Bem}

\begin{Def}{\name{Friedrichs}glättung}
    Für $u \in L^1_\loc(\real^d)$ und $\varepsilon > 0$ heißt
    $u_\varepsilon := u \ast \eta_\varepsilon$ \begriff{\name{Friedrichs}glättung}.
\end{Def}

\begin{Lemma}{Glättungseigenschaft}
    Es gilt $u_\varepsilon \in \C^\infty(\real^d)$.
\end{Lemma}

\pagebreak

\subsection{%
    Dif"|fusionsgleichung/Wärmeleitungsgleichung%
}

\subsubsection{%
    Gleichung%
}

\begin{Def}{Dif"|fusionsgleichung/instat. Wärmeleitungsgleichung}\\
    Für $\Omega \subset \real^d$, $T \in (0, \infty]$ und $\Omega_T := \Omega \times (0, T)$
    heißt $\partial_t u - \Delta u = 0$ in $\Omega_T$
    \begriff{Dif"|fusionsgleichung} oder \begriff{instat. Wärmeleitungsgleichung}.
\end{Def}

\begin{Bem}\\
    Für $\Omega = \real^d$ betrachtet man das Cauchy-Problem (AWP) mit Anfangswerten
    $u(\cdot, 0) = u_0$ in $\Omega$ und
    für $\Omega \subsetneqq \real^d$ das ARWP
    $u(\cdot, 0) = u_0$ in $\Omega$ und
    $u(x, t) = g(x, t)$ für $(x, t) \in \partial\Omega \times (0, T)$.\\
    Ebenfalls möglich ist $\partial_t u - \Delta u = f$ in $\Omega_T$ (inhomogene Gleichung).
\end{Bem}

\linie

\begin{Satz}{Skalierungsinvarianz}\\
    Seien $\Omega := \real^d$, $T := \infty$ und $u \in \C^2(\Omega_T)$ eine klassische
    Lösung der Dif"|fusionsgleichung.\\
    Dann ist für $\lambda \in \real$ auch $u_\lambda$ eine klassische Lösung
    mit $u_\lambda(x, t) := u(\lambda x, \lambda^2 t)$.
\end{Satz}

\subsubsection{%
    Fundamentallösung/Faltungslösung der Dif"|fusionsgleichung%
}

\begin{Bem}
    Die Fundamentallösung soll rot.inv. und \begriff{selbstähnlich}
    ($u(x, t) = C(t, \lambda) u(\lambda x, \lambda^2 t)$) sein und die
    \begriff{Erhaltungseigenschaft} $\forall_{t > 0}\; \int_{\real^d} u(x, t) \dx = 1$ erfüllen.
    Dafür ist der Ansatz $u(x, t) := \gamma(t) v(\frac{\norm{x}^2}{t})$ mit $\gamma(t) > 0$
    geeignet (selbstähnlich mit $C(t, \lambda) := \frac{\gamma(t)}{\gamma(\lambda^2 t)}$).
    $\gamma$ ergibt sich aus
    $1 = \gamma(t) \int_{\real^d} v(\frac{\norm{x}^2}{t}) \dx = \gamma(t) t^{d/2} C_v$ mit
    $C_v := \int_{\real^d} v(\norm{x'}^2) \dx'$.
    Für $v$ benutzt man die PDE, also
    $\partial_t u(x, t) = \gamma'(t) v(s) - \gamma(t) v'(s) \frac{s}{t}$,
    $\partial_{x_i} u(x, t) = \gamma(t) v'(s) \frac{2x_i}{t}$,
    $\partial_{x_i}^2 u(x, t) = \gamma(t) \cdot (v'(s) \frac{2}{t} + v''(s) \frac{4x_i^2}{t^2})$
    und somit $0 = \partial_t u(x, t) - \Delta u(x, t)
    = \gamma(t) (-v'(s) \frac{s}{t} - v'(s) \frac{2d}{t} - v''(s) \frac{4s}{t}) + \gamma'(t) v(s)$
    mit $s := \frac{\norm{x}^2}{t}$.
    Durch Einsetzen von $\gamma(t)$ und $\gamma'(t) = -\frac{d}{2C_v t^{d/2+1}}$ erhält man
    $0 = \frac{d}{2} v(s) + (s + 2d) v'(s) + 4s v''(s)$.
    Diese ODE für $v$ löst man mit dem Ansatz $v(s) := be^{as}$ mit $a, b \in \real$.
    Man bekommt dann $0 = v(s) \cdot (s \cdot (4a + 1)a + (2a + \frac{1}{2})d)
    \iff a = -\frac{1}{4}$, also $v(s) = be^{-s/4}$.
    Es gilt daher\\
    $C_v = \int_{\real^d} be^{-\norm{x}^2/4} \dx = b (4\pi)^{d/2}$ sowie
    $\gamma(t) = \frac{1}{b(4\pi t)^{d/2}}$ und
    $u(x, t) = \frac{1}{(4\pi t)^{d/2}} e^{-\norm{x}^2/(4t)}$.
\end{Bem}

\linie

\begin{Def}{Fundamentallösung}
    Seien $\Omega := \real^d$ und $T := \infty$.
    Dann heißt die Funktion $\Phi \in \C^\infty(\Omega_T)$ mit
    $\Phi(x, t) := \frac{1}{(4\pi t)^{d/2}} e^{-\norm{x}^2/(4t)}$
    \begriff{Fundamentallösung} der Dif"|fusionsgleichung/\begriff{Wärmeleitungskern}.
\end{Def}

\begin{Bem}
    $\Phi$ ist eine klassische Lösung der Wärmeleitungsgleichung und erfüllt\\
    $\forall_{t > 0}\; \int_{\real^d} \Phi(x, t) \dx = 1$ (\begriff{Erhaltungseigenschaft}) sowie
    $\forall_{\beta \in \natural_0^{d+1}} \forall_{\delta > 0}\;
    \partial^\beta \Phi \in L^\infty(\Omega \times [\delta, \infty))$.\\
    Allerdings gilt $\lim_{t \to 0} \Phi(x, t) = 0$ für $x \not= 0$, aber
    $\lim_{t \to 0} \Phi(0, t) = \infty$, d.\,h. $\Phi \notin \C^0(\overline{\Omega_T})$.
    Insbesondere ist $\Phi$ keine klassische Lösung des AWPs
    (erfüllt Anfangswert $\delta_0$ im Distributionssinn).
    Eine klassische Lösung des AWPs erhält man mittels Faltung.
\end{Bem}

\begin{Satz}{Faltungslösung}\\
    Seien $\Omega := \real^d$, $T := \infty$,
    $u_0 \in L^\infty(\Omega)$ sowie $u\colon \Omega_T \to \real$ mit
    $u(\cdot, t) := \Phi(\cdot, t) \ast u_0$.
    Dann gilt
    \begin{enumerate}
        \item
        $u \in \C^\infty(\Omega_T)$,
        
        \item
        $u$ klassische Lösung der Wärmeleitungsgleichung und
        
        \item
        für $u_0 \in \C^0(\real^d)$, dass $\forall_{\overline{x} \in \Omega}\;
        \lim_{(x, t) \to (\overline{x}, 0)} u(x, t) = u_0(\overline{x})$.
    \end{enumerate}
\end{Satz}

\begin{Bem}
    Teil \emph{(1)} gilt z.\,B. auch, wenn $u_0$ unstetig ist.
    Dies nennt man den \begriff{glättenden/regula"-risierenden Ef"|fekt} der
    Dif"|fusionsgleichung.\\
    Wegen $u(x, t) = \int_{\real^d} \Phi(x - y, t) u_0(y) \dy$ und $\Phi(x, t) > 0$ für alle
    $(x, t) \in \Omega_T$ trägt jeder Punktwert $u_0(x)$ zu jedem späteren Wert $u(x', t)$
    für $t > 0$ bei,
    insbesondere auch, wenn $x'$ beliebig weit von $x$ entfernt und $t$ beliebig klein ist.
    Man nennt dies \begriff{unendliche Ausbreitungsgeschwindigkeit}.
\end{Bem}

\pagebreak

\subsubsection{%
    Eigenschaften der Lösung%
}

\begin{Satz}{$L^\infty$-Beschränktheit}
    Seien $\Omega := \real^d$ und $u$ die Faltungslösung für die Anfangswerte $u_0$.\\
    Dann gilt $\forall_{t > 0}\;
    \norm{u(\cdot, t)}_{L^\infty(\Omega)} \le \norm{u_0}_{L^\infty(\Omega)}$.
\end{Satz}

\linie

\begin{Satz}{Eindeutigkeit für ARWPs}
    Sei $\Omega \subset \real^d$ ein Lipschitz-Gebiet.\\
    Dann gibt es höchstens eine klassische Lösung des inhomogenen ARWPs\\
    $\partial_t u - \Delta u = f$ in $\Omega_T$,
    $u(\cdot, 0) = u_0$ in $\Omega$ und $u(x, t) = g(x, t)$ auf $\partial\Omega \times (0, T)$.
\end{Satz}

\begin{Bem}
    Die Aussage gilt ähnlich auch für Neumann-/Robin-RBen, aber sie sagt nichts über Existenz
    von Lösungen aus
    (z.\,B. mindestens Stetigkeit und Kompatibilität von $u_0$ und $g$ erforderlich).
\end{Bem}

\linie

\begin{Satz}{Maximumprinzip}
    Sei u eine klassische Lösung des ARWPs
    $\partial_t u - \Delta u = 0$ in $\Omega_T$,\\
    $u(\cdot, 0) = u_0$ in $\Omega$ und $u(x, t) = g(x, t)$ auf $\partial\Omega \times (0, T)$.\\
    Dann nimmt $u$ sein Maximum (und Minimum) auf dem
    \begriff{parabolischen Rand}\\
    $\Gamma := (\Omega \times \{0\}) \cup (\partial\Omega \times [0, T])$ an.
\end{Satz}

\subsubsection{%
    Konvergenz gegen die stationäre Lösung%
}

\begin{Satz}{\name{Poincaré}-Ungleichung}
    Sei $\Omega \subset \real^d$ ein Lipschitz-Gebiet.
    Dann gibt es eine kleinste
    \begriff{\name{Poincaré}-Konstante} $c_p = c_p(\Omega) > 0$ mit
    $\forall_{w \in \C^1_0(\Omega)}\;
    \int_\Omega w(x)^2 \dx \le c_p \int_\Omega \norm{\nabla w(x)}^2 \dx$\\
    (oder kurz $\norm{w}_{L^2(\Omega)}^2 \le c_p \norm{\nabla w}_{L^2(\Omega)}^2$).
\end{Satz}

\begin{Bem}
    Die Poincaré-Ungleichung gilt bereits für $w \in \C^1(\Omega)$ mit $w|_{\partial\Omega} = 0$.\\
    Hat $w$ keine Nullrandwerte, dann gilt die Poincaré-Ungleichung i.\,A. nicht mehr.
    Setzt man z.\,B. $w(x) :\equiv c$ mit $c \not= 0$,
    dann ist $\int_\Omega w(x)^2 \dx = c^2 |\Omega| > 0$, aber
    $\int_\Omega \norm{\nabla w(x)}^2 \dx = 0$.\\
    Ein Beweis für $\Omega = (0, 1)$ sieht wie folgt aus:
    Es gilt $w(x) = \int_0^x w'(\xi) \dxi$, weil $w(0) = 0$.
    Nach Cauchy-Schwarz folgt
    $|w(x)|^2 = |\int_0^x 1 \cdot w'(\xi) \dxi|^2
    \le (\int_0^x |1|^2 \dxi) \cdot (\int_0^x |w'(\xi)|^2 \dxi)$\\
    $\le x \cdot (\int_0^1 |w'(\xi)|^2 \dxi)$.
    Durch Integration folgt
    $\int_0^1 |w(x)|^2 \dx \le (\int_0^1 x \dx) \cdot (\int_0^1 |w'(\xi)|^2 \dxi)$\\
    $= \frac{1}{2} \int_0^1 |w'(\xi)|^2 \dxi$,
    also ist $c_p \le \frac{1}{2}$ für $\Omega = (0, 1)$.
    (Genauer gilt $c_p = \frac{1}{\pi^2}$.)
\end{Bem}

\linie

\begin{Satz}{Konvergenz gegen stationäre Lösung}\\
    Seien $\Omega \subset \real^d$ ein Lipschitz-Gebiet,
    $f$, $g$ zeitunabhängig,
    $u(x, t)$ klassische Lösung des inhomogenen ARWPs
    $\partial_t u - \Delta u = f$ in $\Omega_T$,
    $u(\cdot, 0) = u_0$ in $\Omega$ und $u(x, t) = g$ auf $\partial\Omega \times (0, T)$
    sowie\\
    $\overline{u}(x)$ klassische Lösung des \begriff{stat. \name{Poisson}-Problems}
    $-\Delta\overline{u} = f$ in $\Omega$ und $\overline{u} = g$ auf $\partial\Omega$.\\
    Dann konvergiert $u$ exp. gegen $\overline{u}$, genauer
    $\forall_{t \in (0, T)}\; \norm{u(\cdot, t) - \overline{u}}_{L^2(\Omega)}^2 \le
    e^{-2t/c_p} \norm{u_0 - \overline{u}}_{L^2(\Omega)}^2$\\
    mit $c_p = c_p(\Omega)$ der Poincaré-Konstanten von $\Omega$.
\end{Satz}

\linie

\begin{Bem}
    Man kann die Dif"|fusionsgleichung auch verallgemeinern.
    Ist $D > 0$ die \begriff{Dif"|fusions"-konstante}, dann betrachtet man
    $\partial_t u - D\Delta u = 0$.
    Die \begriff{allgemeine Fundamentallösung} ist dann
    $u(x, t) := \frac{1}{(4\pi Dt)^{d/2}} \exp(-\frac{\norm{x}^2}{4Dt})$.
    Die Aussage über die Konvergenz gegen die stationäre Lösung wird zu
    $\forall_{t > 0}\; \norm{u(\cdot, t) - \overline{u}}_{L^2(\Omega)}^2 \le
    e^{-2Dt/c_p} \norm{u_0 - \overline{u}}_{L^2(\Omega)}^2$.
    Ist also $D > 1$, dann ist die Fundamentallösung stärker glättend bzw.
    die Lösung fällt schneller gegen die stationäre Lösung ab.
\end{Bem}

\pagebreak

\subsection{%
    Wellengleichung%
}

\subsubsection{%
    Gleichung%
}

\begin{Def}{Wellengleichung}\\
    Für $\Omega := \real^d$, $T \in (0, \infty]$, $\Omega_T := \Omega \times (0, T)$,
    $c > 0$ und Anfangswerte $u_0 \in \C^2(\Omega), v_0 \in \C^1(\Omega)$ heißt
    das Problem, ein $u \in \C^2(\Omega_T) \cap \C^1(\overline{\Omega_T})$ zu finden mit
    $\partial_t^2 u - c^2 \Delta u = 0$ in $\Omega_T$,
    $u(\cdot, 0) = u_0$ in $\Omega$ und
    $\partial_t u(\cdot, 0) = v_0$ in $\Omega$,
    \begriff{\name{Cauchy}-Problem für die Wellengleichung}.
\end{Def}

\begin{Bem}
    Für $c = 1$ ist die Gleichung \emph{nicht} äquivalent zu $-\Delta_{(x,t)} u = 0$,
    weil das Vorzeichen von $\partial_t^2 u$ umgekehrt ist.
\end{Bem}

\subsubsection{%
    1D-Lösung für \texorpdfstring{$v_0 = 0$}{v₀ = 0} oder \texorpdfstring{$u_0 = 0$}{u₀ = 0}%
}

\begin{Bem}
    Im Folgenden wird eine Lösung für $d = 1$ konstruiert.
    Zunächst wird die PDE umgeschrieben in ein System 1. Ordnung.
    Dazu seien $w_1 := \partial_t u$ und $w_2 := \partial_x u$.\\
    Es gilt $\partial_t w_1 - c^2 \partial_x w_2 = 0$,
    wobei $w_1(\cdot, 0) = v_0$ und $w_2(\cdot, 0) = \partial_x u_0 = u_0'$ in $\Omega$.\\
    Mit $w := \smallpmatrix{w_1\\w_2}$, $A := \smallpmatrix{0&-c^2\\-1&0}$ und
    $w_0 := \smallpmatrix{v_0\\u_0'}$ ergibt sich
    $\partial_t w + A \partial_x w = 0$, $w(\cdot, 0) = w_0$ in $\Omega$,
    weil $\partial_x w_1 = \partial_t w_2$ (wegen $u \in \C^2(\Omega_T)$).\\
    $A$ ist diagonalisierbar mit $A = R\Lambda R^{-1}$ sowie
    $\Lambda := \smallpmatrix{-c&0\\0&c}$,
    $R := \smallpmatrix{c&-c\\1&1}$ und $R^{-1} = \frac{1}{2c} \smallpmatrix{1&c\\-1&c}$.\\
    Durch die Koordinatentransformation $z := R^{-1} w$ erhält man
    $\partial_t Rz + R\Lambda R^{-1} \partial_x Rz = 0$ und $z(\cdot, 0) = z_0$ in $\Omega$
    mit $z_0 = \smallpmatrix{z_{0,1}\\z_{0,2}} := R^{-1} w_0$.
    Multipliziert man von links mit $R^{-1}$, so bekommt man
    $\partial_t z + \Lambda \partial_x z = 0$ und $z(\cdot, 0) = z_0$ in $\Omega$.
    Ausgeschrieben erhält man also zwei entkoppelte Advektionsgleichungen
    $\partial_t z_1 - c \partial_x z_1 = 0$,
    $z_1(\cdot, 0) = z_{0,1}$ sowie
    $\partial_t z_2 + c \partial_x z_2 = 0$,
    $z_2(\cdot, 0) = z_{0,2}$.\\
    Mittels der Methode der Charakteristiken kann man eine explizite Lösung ermitteln als\\
    $z(x, t) = \smallpmatrix{z_{0,1} (x - (-c)t)\\z_{0,2} (x - ct)}
    = \smallpmatrix{z_{0,1} (x + ct)\\z_{0,2} (x - ct)}$,
    wobei $z_0 = R^{-1} w_0 =
    \frac{1}{2c} \smallpmatrix{w_{0,1} + cw_{0,2}\\-w_{0,1} + cw_{0,2}}$.\\
    Man erhält
    $z(x, t) = \frac{1}{2c} \smallpmatrix{w_{0,1} (x + ct) + cw_{0,2} (x + ct)\\
    -w_{0,1} (x - ct) + cw_{0,2} (x - ct)}$ bzw.
    $w(x, t) = \frac{1}{2c} \smallpmatrix{c&-c\\1&1}
    \smallpmatrix{v_0 (x + ct) + c u_0' (x + ct)\\-v_0 (x - ct) + c u_0' (x - ct)}$.
\end{Bem}

\linie

\begin{Bem}
    \begin{itemize}
        \item
        \textbf{Spezialfall: $v_0 = 0$}\\
        In diesem Fall ist
        $w(x, t) = \frac{1}{2} \smallpmatrix{cu_0'(x+ct) - cu_0'(x-ct)\\u_0'(x+ct) + u_0'(x-ct)}$,
        also $\partial_x u(x, t) = \frac{1}{2} (u_0'(x+ct) + u_0'(x-ct))$ und damit
        $u(x, t) = \frac{1}{2} (u_0(x+ct) + u_0(x-ct)) + K(t)$ mit geeignetem $K(t) \in \real$.
        Für $t = 0$ erhält man $u_0(x) = u(x, 0) = u_0(x) + K(0) \iff K(0) = 0$.
        Für $t > 0$ erhält man durch $\partial_t$ auf $u(x, t)$, dass
        $\frac{\d}{\dt} K(t) = \partial_t u(x, t) - \frac{c}{2} (u_0'(x+ct) - u_0'(x-ct))
        = \partial_t u(x, t) - w_1(x, t) \equiv 0$,
        d.\,h. $K(t) \equiv 0$.\\
        Somit ist $u(x, t) = \frac{1}{2} (u_0(x+ct) + u_0(x-ct))$ eine notwendige Bedingung
        für die Lösung, die auch hinreichend ist (Überprüfung durch Einsetzen in PDE).
        Damit ist eine eindeutige Lösung für $v_0 = 0$ gefunden.
        
        \item
        \textbf{Spezialfall: $u_0 = 0$}\\
        In diesem Fall ist
        $w(x, t) = \frac{1}{2c} \smallpmatrix{cv_0(x+ct) + cv_0(x-ct)\\v_0(x+ct) - v_0(x-ct)}$,
        also $\partial_x u(x, t) = \frac{1}{2c} (v_0(x+ct) - v_0(x-ct))$ und damit
        $u(x, t) = \frac{1}{2c} \left(\int_0^{x+ct} v_0(s) \ds - \int_0^{x-ct} v_0(s) \ds\right)
        + K(t)$ mit geeignetem $K(t) \in \real$,
        weil aus $g(x) := \int_0^{z(x)} v_0(s) \ds$ folgt $g'(x) = z'(x) v_0(z(x))$.\\
        Für $t = 0$ erhält man $0 = u_0(x) = u(x, 0) = 0 + K(0) \iff K(0) = 0$.
        Für $t > 0$ erhält man durch $\partial_t$ auf $u(x, t)$, dass
        $\frac{\d}{\dt} K(t) = \partial_t u(x, t) - \frac{1}{2c} (c v_0(x+ct) - (-c) v_0(x-ct))$\\
        $= \partial_t u(x, t) - \frac{1}{2} (v_0(x+ct) + v_0(x-ct))
        = \partial_t u(x, t) - w_1(x, t) \equiv 0$,
        d.\,h. $K(t) \equiv 0$.\\
        Somit ist $u(x, t) = \frac{1}{2c} \int_{x-ct}^{x+ct} v_0(s) \ds$ eine notwendige Bedingung
        für die Lösung, die ebenfalls wieder hinreichend ist.
        Damit ist eine eindeutige Lösung für $u_0 = 0$ gefunden.
    \end{itemize}
\end{Bem}

\pagebreak

\subsubsection{%
    \name{d'Alembert}sche Formel für 1D%
}

\begin{Satz}{Ex. + Eind., \name{d'Alembert}sche Formel für $d = 1$}
    Für $\Omega := \real$ ist die eindeutige klassische Lösung des AWPs gegeben durch
    $u(x, t) = \frac{1}{2} (u_0(x+ct) - u_0(x-ct)) + \frac{1}{2c} \int_{x-ct}^{x+ct} v_0(s) \ds$.
\end{Satz}

\begin{Bem}\\
    Die Notwendigkeit von $u_0 \in \C^2(\Omega)$ und $v_0 \in \C^1(\Omega)$ wird jetzt klar,
    denn dann gilt $u \in \C^2(\Omega)$.\\
    Die Wellengleichung hat keinen regularisierenden Ef"|fekt, da $u$ nicht glatter
    als die Anfangsdaten.\\
    Für $d > 1$ gibt es ebenfalls Lösungsformeln (die allerdings viel komplizierter sind).
\end{Bem}

\begin{Bem}
    \begriff{Stehende Wellen}, die man z.\,B. bei schwingenden Saiten beobachten kann,
    lassen sich mit der d'Alembertschen Formel erklären.
    Mit $u_0(x) := \sin(\omega t)$ für ein $\omega \not= 0$ und $v_0 :\equiv 0$
    erhält man $u(x, t) = \frac{1}{2} (\sin(\omega(x+ct)) + \sin(\omega(x-ct)))
    = \sin(\omega x) \cos(\omega ct)$ (mit dem Additionstheorem),
    d.\,h. eine Überlagerung zweier laufender Sinuswellen ergibt eine stehende Welle,
    denn für $\omega x \in \pi\integer$ ist $u(x, t) = 0$ für alle $t \ge 0$.
\end{Bem}

\subsubsection{%
    Eigenschaften der 1D-Lösung%
}

\begin{Satz}{$L^\infty$-Stabilität}
    Seien $\Omega := \real$, $u_0 \in \C^2(\Omega) \cap L^\infty(\Omega)$ und
    $v_0 \in \C^1(\Omega) \cap L^1(\Omega)$.\\
    Dann erfüllt die Lösung $u$ des AWPs
    $\forall_{t \ge 0}\; \norm{u(\cdot, t)}_{L^\infty(\Omega)} \le \norm{u_0}_{L^\infty(\Omega)} +
    \frac{1}{2c} \norm{v_0}_{L^1(\Omega)}$.
\end{Satz}

\begin{Bem}
    Es gilt kein Max.prinzip, denn trotz $u_0 = 0$ kann $u(\cdot, t) \not= 0$ gelten
    (wenn $v_0 \not= 0$).
\end{Bem}

\begin{Satz}{st. Abh. von Anfangsdaten}
    Seien $\Omega := \real$, $u, \overline{u}$ Lösungen des AWPs mit identischem $c$,
    aber unterschiedlichem $u_0, \overline{u_0} \in \C^2(\Omega) \cap L^\infty(\Omega)$ und
    $v_0, \overline{v_0} \in \C^1(\Omega) \cap L^1(\Omega)$.\\
    Dann gilt $\exists_{C > 0} \forall_{t \ge 0}\;
    \norm{u(\cdot, t) - \overline{u}(\cdot, t)}_{L^\infty(\Omega)}
    \le C \left(\norm{u_0 - \overline{u_0}}_{L^\infty(\Omega)} +
    \norm{v_0 - \overline{v_0}}_{L^1(\Omega)}\right)$.
\end{Satz}

\begin{Bem}
    Wie bei der Advektionsgleichung gibt es keine stetige Abh. bzgl. $c$ in der
    $L^\infty$-Norm.
\end{Bem}

\linie

\begin{Def}{Abhängigkeitskegel}
    Seien $\Omega := \real$ und $(x_0, t_0) \in \Omega_T$.\\
    Dann ist der \begriff{Abhängigkeitskegel} von $(x_0, t_0)$ definiert
    durch\\
    $C := \{(x, t) \in \Omega_T \;|\; t \in [0, t_0],\; |x - x_0| \le c(t_0 - t)\}$.
\end{Def}

\begin{Satz}{Abhängigkeitskegel}
    Seien $\Omega := \real$, $(x_0, t_0) \in \Omega_T$ und $C$ der Abhängigkeitskegel von
    $(x_0, t_0)$.
    Dann folgt aus $\forall_{x \in \Omega,\, |x - x_0| \le ct_0}\; u_0(x) = v_0(x) = 0$,
    dass $u|_C \equiv 0$.
\end{Satz}

\begin{Bem}
    Umgekehrt kann man sagen, dass der Anfangswert $u_0(x_0)$ im Punkt $x_0 \in \Omega$ die
    Lösungswerte $u(x, t)$ nur für $t \ge 0$ und $|x - x_0| \le ct$ beeinflusst.
    Information breitet sich also nur mit endlicher Geschwindigkeit $c$ aus
    (im Gegensatz zur Dif"|fusionsgleichung).
\end{Bem}

\subsubsection{%
    Eindeutigkeit für das inhomogene ARWP für \name{Lipschitz}-Gebiete%
}

\begin{Satz}{Eindeutigkeit}
    Seien $\Omega \subset \real^d$ ein Lipschitz-Gebiet, $\Omega_T := \Omega \times (0, T)$,
    $c > 0$, $f\colon \Omega_T \to \real$, $g\colon \partial\Omega \times (0, T) \to \real$,
    $u_0, v_0\colon \Omega \to \real$ und das ARWP
    $\partial_t^2 u - c^2 \Delta u = f$ in $\Omega_T$,
    $u(\cdot, 0) = u_0$ in $\Omega$,
    $\partial_t u(\cdot, 0) = v_0$ in $\Omega$ und
    $u(x, t) = g(x, t)$ auf $\partial\Omega \times (0, T)$ gegeben.\\
    Dann gibt es höchstens eine Lösung $u \in \C^2(\Omega_T) \cap \C^1(\overline{\Omega_T})$
    des ARWPs.
\end{Satz}

\begin{Bem}
    Ohne weitere Forderungen an die Daten (Regularität, Kompatibilität) kann man keine
    Existenzaussage beweisen.
    Die Anfangsdaten müssen sowohl für $u$ als auch für $\partial_t u$ vorgegeben werden,
    wogegen die Randdaten nur für eines von beiden vorgegeben werden dürfen,
    weil das ARWP sonst überbestimmt ist.
\end{Bem}

\pagebreak

\subsubsection{%
    Herleitung durch Linearisierung der \name{Euler}-Gleichungen%
}

\begin{Bem}
    Die Wellengleichung kann auch aus den \begriff{\name{Euler}-Gleichungen}
    (beschreiben Strömungen in reibungsfreien Fluiden) hergeleitet werden,
    die man z.\,B. in der Akustik verwendet.
    Nimmt man an, dass man Schallwellen modellieren will und die Luft ein isothermes Gas ist,
    sich also durch die Druckschwankungen nicht aufwärmt, so lauten die Euler-Gleichungen
    $\partial_t \varrho + \div_x(\varrho v) = 0$ (\begriff{Massenerhaltung}) und
    $\partial_t v + (v \cdot \nabla) v + \frac{1}{\varrho} \nabla p(\varrho) = 0$
    (\begriff{Impulserhaltung})
    mit den Unbekannten $\varrho\colon \Omega \to \real$ (Dichte) und
    $v\colon \Omega \to \real^d$ (Geschwindigkeit),
    wobei\\
    $(v \cdot \nabla) v := (\sum_{i=1}^d v_i \partial_{x_i} v_j)_{j=1}^d$.
    Der Druck $p(\varrho)$ wird meist als \begriff{Zustandsgleichung} problemabhängig
    vorgeschrieben (für ein ideales Gas mit adiabatischen NBen kann man z.\,B.
    $p(\varrho) := K \varrho^\gamma$ mit $K, \gamma > 0$ nehmen).
    
    Wenn man annimmt, dass die Dichte nur kleine Schwankungen um den Mittelwert
    $\varrho \in \real^+$ erfährt, also $\varrho = \varrho_0 (1 + g)$ mit "`kleinem"'
    $g, \nabla g, v, \div(v)$, so kann man die quadratischen Terme
    $g\div(v), v \nabla g, (v \cdot \nabla) v$ vernachlässigen.
    
    Eingesetzt in die Massenerhaltung bekommt man
    $\varrho_0 (\partial_t g + \div((1 + g) v)) = 0$\\
    $\iff \partial_t g + (1 + g) \div(v) + v \nabla g = 0
    \iff \partial_t g + \div(v) + \cancel{g \div(v)} + \cancel{v \nabla g} = 0
    \iff \partial_t g + \div(v) = 0$.
    
    Für die Impulserhaltung approximiert man
    $\frac{1}{\varrho} \nabla p(\varrho)
    = \frac{\nabla p(\varrho_0 (1 + g))}{\varrho_0 (1 + g)}
    \approx \frac{p'(\varrho_0) \nabla g}{\varrho_0}$
    und erhält so durch Einsetzen
    $\partial_t v + \cancel{(v \cdot \nabla) v} + \frac{p'(\varrho_0) \nabla g}{\varrho_0}
    = 0 \iff \partial_t v + c^2 \nabla g = 0$ mit $c^2 := \frac{p'(\varrho_0)}{\varrho_0}$.
    
    Wendet man nun $\partial_t$ auf die erste Gleichung und $\div$ auf die zweite an
    und zieht das zweite Ergebnis vom ersten ab, so bekommt man
    $\partial_t^2 g - c^2 \Delta g = 0$,
    also die Wellengleichung.
    
    Dies heißt auch \begriff{akustische Approximation der \name{Euler}-Gleichungen}
    und $c$ ist die Schallgeschwindigkeit.
\end{Bem}

\pagebreak

\subsection{%
    Klassifikation linearer PDEs zweiter Ordnung%
}

\begin{Def}{linearer Dif"|ferentialoperator 2. Ordnung}
    Seien $\Omega \subset \real^d$ of"|fen,\\
    $A = (a_{ij})_{i,j=1}^d \in \C^0(\Omega, \real^{d \times d})$,
    $b = (b_i)_{i=1}^d \in \C^0(\Omega, \real^d)$ und $c \in \C^0(\Omega)$.
    Dann heißt\\
    $\L\colon \C^2(\Omega) \to \C^0(\Omega)$ mit
    $(\L u)(x) := -\sum_{i,j=1}^d a_{ij}(x) \partial_{x_i} \partial_{x_j} u(x) +
    \sum_{i=1}^d b_i(x) \partial_{x_i} u(x) + c(x) u(x)$\\
    \begriff{linearer Dif"|ferentialoperator 2. Ordnung}.
\end{Def}

\begin{Bem}
    Mit dem Hadamard-Produkt $\circ$ (elementweise Matrizenmultiplikation) erhält man
    $\L u = -A \circ (\nabla \nabla^\tp u) + b \nabla u + cu$.
    Der erste Summand
    $-A \circ (\nabla \nabla^\tp u)$ heißt \begriff{Hauptteil} von $\L$.\\
    OBdA kann man $A$ symmetrisch wählen (sonst führt $\widetilde{A} := \frac{1}{2} (A + A^\tp)$
    zum selben $\L$).\\
    Mit $f \in \C^0(\Omega)$ erhält man eine PDE $\L u = f$ in $\Omega$.
\end{Bem}

\linie

\begin{Def}{Klassifikation von linearen PDEs 2. Ordnung}
    Sei $x \in \Omega$.
    Dann heißt $\L$
    \begin{itemize}
        \item
        \begriff{elliptisch in $x$},
        falls alle EWe von $A(x)$ positiv sind,
        
        \item
        \begriff{parabolisch in $x$},
        falls $(d - 1)$ EWe von $A(x)$ positiv sind und der übrige verschwindet,
        aber $\Rang(\smallpmatrix{A(x) & b(x)}) = d$, und
        
        \item
        \begriff{hyperbolisch in $x$},
        falls $(d - 1)$ EWe von $A(x)$ positiv sind und der übrige negativ ist.
    \end{itemize}
    $\L$ heißt \begriff{elliptisch}/\begriff{parabolisch}/\begriff{hyperbolisch},
    falls $\L$ die Eigenschaft in allen $x \in \Omega$ erfüllt.\\
    Die PDE $\L u = f$ heißt elliptisch/parabolisch/hyperbolisch,
    falls $\L$ diese Eigenschaft erfüllt.
\end{Def}

\begin{Bem}
    Die Begrif"|fe sind motiviert durch Quadriken,
    denn $\{z \in \real^d \;|\; z^\tp A(x) z = 1\}$ beschreibt unter obigen Bedingungen
    ein Ellipsoid, Paraboloid bzw. Hyperboloid.
\end{Bem}

\begin{Bsp}
    \begin{itemize}
        \item
        Die Laplace-/Poisson-Gleichung ist elliptisch, da aus $\L u := -\Delta u$ folgt,
        dass $A(x) :\equiv I_d$ und $b = c :\equiv 0$
        (das erklärt den Sinn des negativen Vorzeichens).
        
        \item
        Die Dif"|fusionsgleichung ist parabolisch, da aus $\L u := \partial_t u - \Delta_x u$
        folgt, dass\\
        $A(x, t) :\equiv \smallpmatrix{I_d&0\\0&0} \in \real^{(d+1)\times(d+1)}$,
        $b :\equiv e_{d+1} \in \real^{d+1}$ und $c :\equiv 0$.
        
        \item
        Die Wellengleichung ist hyperbolisch, da aus
        $\L u := \partial_t^2 u - \widetilde{c}^2 \Delta_x u$ folgt, dass\\
        $A(x, t) :\equiv \smallpmatrix{I_d&0\\0&-\widetilde{c}^2} \in \real^{(d+1)\times(d+1)}$ und
        $b = c :\equiv 0$.
        
        \item
        Die \begriff{\name{Tricomi}-Gleichung} $x_2 \partial_{x_1}^2 u + \partial_{x_2}^2 u = 0$
        in $\Omega := \real^2$ ist vom gemischten Typ, da aus $A(x) := \smallpmatrix{x_2&0\\0&1}$ 
        und $b = c :\equiv 0$ folgt, dass sie
        elliptisch in $(x_1, x_2) \in \real \times (0, \infty)$ und
        hyperbolisch in $(x_1, x_2) \in \real \times (-\infty, 0)$ ist.
    \end{itemize}
\end{Bsp}

\linie

\begin{Bem}
    Die Unterscheidung ist sinnvoll wg. unterschiedlicher Lösungseigenschaften.
    \begin{itemize}
        \item
        \emph{elliptische PDEs}:
        meist RBen vorgegeben,
        Lösungen meist sehr glatt ($\C^\infty$),
        erfüllen häufig ein Maximumprinzip
        
        \item
        \emph{parabolische PDEs}:
        ausgezeichnete Achse meist Zeit,
        Umschreiben als $\partial_t u + \widetilde{\L} u = \widetilde{f}$ mit
        $\widetilde{\L}$ elliptisch möglich,
        häufig ABen vorgegeben (ggf. RBen),
        regularisierender Ef"|fekt (Lösung glatter als Anfangsdaten),
        unendliche Ausbreitungsgeschwindigkeit
        
        \item
        \emph{hyperbolische PDEs}:
        ausgezeichnete Achse meist Zeit,
        Umschreiben als $\partial_t^2 u + \widetilde{\L} u = \widetilde{f}$ mit
        $\widetilde{\L}$ elliptisch möglich,
        beschreiben Schwingungsvorgänge,
        häufig ABen für $u$ und $\partial_t u$ vorgegeben (ggf. dazu noch RBen),
        endliche Ausbreitungsgeschwindigkeit
    \end{itemize}
\end{Bem}

\begin{Bem}
    Sei $\L$ elliptisch.
    Falls $\L$ rot.inv. ist, so gilt $\L u = -a \nabla u + cu$.
    $\L$ heißt \begriff{glm. ell.} mit \begriff{Ell.konst.} $\alpha$, falls
    $\exists_{\alpha > 0} \forall_{z \in \real^d} \forall_{x \in \Omega}\;
    z^\tp A(x) z \ge \alpha \norm{z}^2$
    (alle EWe von $A(\cdot)$ sind $\ge \alpha$).
    Maximum-/""Minimum-/Vergleichsprinzipien und Eind. von Lsg.en folgen wie bei
    der Poisson-Gleichung.
\end{Bem}

\pagebreak

\subsection{%
    \emph{Einschub}:
    Finite Volumen für skalare Erhaltungsgleichungen in 1D%
}

\begin{Bem}
    Im Folgenden betrachtet man für $\Omega := \real$, $\Omega_T := \real \times (0, T)$,
    $f \in \C^1(\Omega)$ (\begriff{Flussfunk"-tion})
    und $u_0 \in L^1_\loc(\Omega) \cap L^\infty(\Omega)$
    das Cauchy-Problem $\partial_t u + \partial_x f(u) = 0$ in $\Omega_T$ und
    $u(\cdot, 0) = u_0$.
    
    Gesucht ist ein num. Verfahren zur Lösung der PDE, das das Integral
    $\int_\Omega u(x, t) \dx$ für $t \in (0, T)$ erhält.
    Definiere das Gitter $x_j := j \Delta x$ und $t^n := n \Delta t$ für
    $j \in \integer \cup (\integer + \frac{1}{2})$ und $n \in \natural_0$.
    Man integriert nun über das Kontrollvolumen $V := [x_{j-1/2}, x_{j+1/2}] \times [t^n, t^{n+1}]$
    und wendet den Gauß-Integralsatz an:
    $0 = \int_V (\partial_t u + \partial_x f(u)) \dx
    = \int_V \div_{(t,x)}((u, f(u))^\tp) \dx
    = \int_{\partial V} (u, f(u))^\tp \cdot n \dsigma(t, x)$\\
    $= \int_{x_{j-1/2}}^{x_{j+1/2}} (u(x, t^{n+1}) - u(x, t^n)) \dx +
    \int_{t^n}^{t^{n+1}} (f(u(x_{j+1/2}, t)) - f(u(x_{j-1/2}, t))) \dt$.
    
    %Als Approximation nimmt man einen stückweise konstanten Wert $u_j^n$ für
    %$u$ in $V$ an und definiert
    Als Approximation nimmt man $u(x, t^n) \approx u_j^n$ konstant für
    $x \in [x_{j-1/2}, x_{j+1/2}]$ an und definiert
    $g_{j+1/2}^n := g(u_j^n, u_{j+1}^n) \approx
    \frac{1}{\Delta t} \int_{t^n}^{t^{n+1}} f(u(x_{j+1/2}, t)) \dt$
    für einen \begriff{num. Fluss} $g\colon \real^2 \to \real$.
    
    Damit erhält man das diskretisierte Problem
    $0 = \Delta x (u_j^{n+1} - u_j^n) + \Delta t (g_{j+1/2}^n - g_{j-1/2}^n)$ bzw.\\
    $u_j^{n+1} := u_j^n - \frac{\Delta t}{\Delta x} (g_{j+1/2}^n - g_{j-1/2}^n)$
    mit den Anfangswerten $u_j^0 := \frac{1}{\Delta x} \int_{x_{j-1/2}}^{x_{j+1/2}} u_0(x) \dx$\\
    (oder einfacher $u_j^0 := u_0(x_j)$).
    
    Die numerische Lösung ist dann stückweise konstant definiert als
    $u_h(x, t) := \sum_{j,n} u_j^n \cdot \chi_{V_{j,n}}(x, t)$.
\end{Bem}

\linie

\begin{Bem}
    Die Erhaltungseigenschaft des Integrals ist gegeben, weil\\
    $\sum_j u_j^{n+1} \Delta x
    = \sum_j u_j^n \Delta x - \cancel{\sum_j \Delta t (g_{j+1/2}^n - g_{j-1/2}^n)}
    = \sum_j u_j^n \Delta x
    = \dotsb
    = \sum_j u_j^0 \Delta x
    \approx \int_\Omega u_0(x) \dx$.
    
    Es gilt ein \begriff{lokales Maximumprinzip}:
    Ist $g$ Lipschitz-stetig mit Konstante $L$ und $\Delta t \le \frac{\Delta x}{2L}$,
    dann liegt $u_j^{n+1}$ in der konvexen Hülle von $u_{j-1}^n, u_j^n, u_{j+1}^n$.\\
    Daraus folgt direkt $L^\infty$-Stabilität,
    d.\,h. $\norm{u^{n+1}}_\infty \le \norm{u^n}_\infty \le \dotsb \le
    \norm{u_0}_{L^\infty(\Omega)}$.
\end{Bem}

\linie

\begin{Bem}
    Allgemein sollte ein geeigneter numerischer Fluss folgende Bedingungen erfüllen:
    \begin{itemize}
        \item
        \begriff{Konsistenz}:
        $g(u, u) = f(u)$
        
        \item
        \begriff{\name{Lipschitz}-Stetigkeit}:
        $g \in \C^{0,1}(\real^2)$
        
        \item
        \begriff{Monotonie}:
        $g(v, w)$ monoton wachsend in $v$ und fallend in $w$
    \end{itemize}
\end{Bem}

\begin{Bsp}
    Beispiele für numerische Flüsse umfassen:
    \begin{itemize}
        \item
        \begriff{\name{Lax}-\name{Friedrichs}-Fluss}:
        $g(u, v) := \frac{1}{2} (f(u) + f(v)) + \frac{1}{2\lambda} (u - v)$
        mit $\lambda := \frac{\Delta t}{\Delta x}$
        
        \item
        \begriff{\name{Engquist}-\name{Osher}-Fluss}:
        Für $f'(u) > 0$ sollte man Rückwärtsdif"|ferenzen (Downwind)
        und für $f'(u) < 0$ Vorwärtsdif"|ferenzen (Upwind) verwenden.
        Die Berechnung erfolgt mit\\
        $f^+(u) := f(0) + \int_0^u \max(f'(s), 0) \ds$ und
        $f^-(u) := \int_0^u \min(f'(s), 0) \ds$ (damit $f = f^+ + f^-$)
        durch $g(v, w) := f^+(v) + f^-(w)$.
    \end{itemize}
\end{Bsp}

\pagebreak
