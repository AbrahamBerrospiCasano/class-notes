\chapter{%
    Finite-Elemente-Methode%
}

\section{%
    \name{Galerkin}-Verfahren%
}

\subsection{%
    Diskrete Lösung und \name{Galerkin}-Projektion%
}

\begin{Bem}
    Die Idee des Galerkin-Verfahrens ist,
    dass man zur numerischen Lösung schwacher Formen von PDEs diese auf
    endlich-dimensionale Teilräume einschränkt.
\end{Bem}

\begin{Def}{diskrete Lösung}
    Seien $V$ ein Hilbertraum,
    $a(\cdot, \cdot)$ eine stetige, koerzive Bilinearform auf $V$,
    $\ell(\cdot) \in V'$,
    $\forall_{v \in V}\; a(u, v) = \ell(v)$ die schwache Form einer PDE und
    $V_h \le V$ ein endlich-dimensionaler Unterraum.
    Dann heißt $u_h \in V_h$ mit
    $\forall_{v \in V_h}\; a(u_h, v) = \ell(v)$
    \begriff{diskrete Lösung}.
\end{Def}

\begin{Satz}{Ex. + Eind. + Beschr.}
    Die diskrete Lösung $u_h \in V_h$ existiert,
    ist eindeutig und erfüllt $\norm{u_h} \le \frac{1}{\alpha} \norm{\ell}_{V'}$
    mit $\alpha$ der Koerzivitätskonstanten von $a$ auf $V$.
\end{Satz}

\linie

\begin{Bem}
    Das "`$h$"' in $V_h$ zeigt an, dass der Raum von $V_h$ durch einen
    Diskretisierungsparameter (z.\,B. Gitterweite) $h \in \real^+$ charakterisiert wird
    und Hoffnung besteht, dass $\lim_{h \to 0} u_h = u$ mit genügend schneller Konvergenz.
    Es treten folgende Fragen auf:
    \begin{itemize}
        \item
        Wie ist $V_h$ geschickt zu konstruieren?

        \item
        Existieren a-priori-Fehlerschranken $\norm{u - u_h} \le C(u) h^p$?

        \item
        Existieren a-posterori-Fehlerschranken $\norm{u - u_h} \le C(u_h) h^p$?

        \item
        Wie löst man numerisch das entsprechende LGS?
    \end{itemize}
\end{Bem}

\linie

\begin{Lemma}{\name{Galerkin}-Orthogonalität}
    Seien $u \in V$ die schwache Lösung der PDE und
    $u_h \in V_h$ die diskrete Lösung.
    Dann gilt $\forall_{v \in V_h}\; a(u - u_h, v) = 0$.
\end{Lemma}

\begin{Bem}
    Für $a(\cdot, \cdot)$ symmetrisch ist dies gerade die Orthogonalität des
    Projektionsfehlers der orth. Projektion von $u$ auf $V_h$ bzgl. des Energieskalarprodukts,
    denn für die orth. Projektion $P_a\colon V \to V_h$ mit
    $\forall_{v \in V} \forall_{w \in V_h}\; \sp{v - P_a v, w}_a = 0$
    folgt nach dem Lemma $P_a u = u_h$ für die schwache Lösung $u$.
    Damit ist für $a$ symmetrisch die diskrete Lösung $u_h$ genau das Bild der
    orth. Projektion der schwachen Lösung $u$ auf $V_h$ bzgl. $\sp{\cdot, \cdot}_a$
    und heißt daher \begriff{\name{Galerkin}-Projektion}.

    Für $a(\cdot, \cdot)$ nicht-symmetrisch ist die Galerkin-Projektion i.\,A.
    keine orth. Projektion bzgl. irgendeines Skalarprodukts,
    aber das Lemma gilt weiterhin und man spricht immer noch von
    Galerkin-Projektion/-Orthogonalität.
\end{Bem}

\subsection{%
    Eigenschaften der diskreten Lösung%
}

\begin{Lemma}{Reproduktion der schw. Lsg.}
    Sei $u \in V$ die schw. Lösung.
    Ist $u \in V_h$, dann $u_h = u$.
\end{Lemma}

\begin{Bem}
    Daher ist $V_h := \Span(u)$ ein optimaler, höchstens eindimensionaler Approximationsraum,
    der aber zur Berechnung so aufwendig ist wie $u$ selbst, also inpraktikabel.
\end{Bem}

\linie

\begin{Satz}{diskretes Problem als LGS}
    Sei $(\varphi_j)_{j=1}^n$ eine Basis von $V_h$.
    Definiere die \begriff{Steifigkeitsmatrix} $A_h = (a_{i,j})_{i,j=1}^n$
    und die \begriff{rechte Seite} $b_h = (b_i)_{i=1}^n$ durch
    $a_{i,j} := a(\varphi_j, \varphi_i)$ und $b_i := \ell(\varphi_i)$.\\
    Dann ist $A_h d = b$ eindeutig lösbar und es gilt $u_h = \sum_{j=1}^n d_j \varphi_j$.
\end{Satz}

\begin{Bem}
    $A_h$ ist symmetrisch für $a(\cdot, \cdot)$ symmetrisch
    (im Gegensatz zum \begriff{Kollokationsverf.}).\\
    $A_h$ ist positiv definit wg. $a(\cdot, \cdot)$ koerziv
    (für $d \not= 0$ gilt
    $d^\tp A_h d = \sum_{i,j=1}^n d_i d_j a(\varphi_j, \varphi_i)$\\
    $= a(\sum_{j=1}^n d_j \varphi_j, \sum_{i=1}^n d_i \varphi_i)
    = a(v, v) \ge \alpha \norm{v}^2 > 0$
    mit $v := \sum_{j=1}^n d_j \varphi_j \not= 0$).
\end{Bem}

\subsection{%
    Beispiele für Ansatzräume%
}

\begin{Bsp}
    Wählt man $V_h$ als Aufspann von Eigenfunktionen des Diff.operators,
    so erhält man eine optimale Basis bei unbekannter/variabler rechter Seite.

    Eigenfunktionen und -werte erhält man dabei aus der
    \begriff{schwachen Form des EW-Problems} des Diff.operators:
    $w \in V$ heißt \begriff{Eigenfunktion} zum \begriff{Eigenwert} $\lambda$, falls
    $\forall_{v \in V}\; a(w, v) = \lambda \sp{w, v}_{L^2(\Omega)}$.

    Unter gewissen Voraussetzungen ($a(\cdot, \cdot)$ symmetrisch und $\Omega$ zush.)
    kann man zeigen,
    \begin{itemize}
        \item
        dass alle EWe $\lambda_j \in (0, \infty)$ erfüllen (und insb. reell sind),

        \item
        dass es abzählbar unendlich viele, unbeschränkte Eigenwerte gibt,
        also $0 < \lambda_1 \le \lambda_2 \le \dotsb$ und
        $\lim_{j \to \infty} \lambda_j = \infty$, und

        \item
        dass es eine bzgl. $\sp{\cdot, \cdot}_{L^2(\Omega)}$ orthonormale Menge
        $\{w_j\}_{j\in\natural}$ von Eigenfunktionen zu den Eigenwerten $\lambda_j$ gibt.
    \end{itemize}

    Wählt man nun $n \in \natural$, $h := \frac{1}{n}$,
    $V_h := \Span(\varphi_1, \dotsc, \varphi_n)$,
    $\varphi_j := w_j$, so folgt für $A_h$, dass
    $a_{ij} = a(\varphi_j, \varphi_i) = \lambda_j \sp{w_j, w_i}_{L^2(\Omega)}
    = \lambda_j \delta_{ij}$, also ist $A_h$ diagonal und $d_h = \frac{b_j}{\lambda_j}$.

    Allerdings ist dieser Ansatz i.\,A. inpraktikabel, da $w_j$ und $\lambda_j$
    selten bekannt sind.
\end{Bsp}

\linie

\begin{Bsp}
    $V_h$ kann man auch als polynomiellen Ansatzraum wählen.
    Seien dazu $d := 1$, $\Omega := [0, 1]$ und
    $a(u, v) := \int_0^1 u'v' \dx$ auf $H^1_0(\Omega)$.
    Dann hat ein Polynom mit Nullrandwerten die Gestalt $p(x) = x(1-x)q(x)$ für $q \in \PP_m$
    mit $\PP_m$ den Polynomen vom Grad $\le m$.

    Wählt man nun $n \in \natural$, $h := \frac{1}{n}$,
    $\varphi_j(x) := x(1-x) \cdot x^{j-1} = x^j (1-x)$, so folgt für $A_h$, dass i.\,A.
    $a_{ij} = a(\varphi_j, \varphi_i) = \int_0^1 \varphi_j' \varphi_i' \dx \not= 0$,
    also ist $A_h$ i.\,A. dicht besetzt.
    Für große $n$ führt dies zu einem Speicherproblem,
    für mäßig große $n$ ist das Verfahren realisierbar (\begriff{Spektralverfahren}).
\end{Bsp}

\subsection{%
    \name{Céa}-Lemma%
}

\begin{Lemma}{\name{Céa}}
    Seien $a(\cdot, \cdot)$ eine stetige, koerzive Bilinearform auf $V$ mit
    Stetigkeitskonstante $\gamma$ und Koerzivitätskonstante $\alpha$
    und $\ell$ eine Bilinearform.\\
    Dann gilt $\norm{u - u_h} \le \frac{\gamma}{\alpha} \inf_{v \in V_h} \norm{u - v}$
    für $u \in V$ schw. Lsg. und $u_h \in V_h$ diskr. Lsg.
\end{Lemma}

\linie

\begin{Bem}
    Das Céa-Lemma erlaubt einen Zusammenhang zwischen dem Galerkin-Projek"-tionsfehler und
    der Bestapproximation, weil $\inf_{v \in V_h} \norm{u - v}$ der Bestapproximationsfehler
    der orth. Projektion $P\colon V \to V_h$ ist (unabhängig von $a$ und $\ell$).
    Weil der Galerkin-Projektionsfehler höchstens um einen konstanten Faktor schlechter als
    die Bestapproximation ist, spricht man von \begriff{Quasi-Optimalität} der Galerkin-Projektion.

    $V_h$ sollte man daher so wählen,
    dass alle möglichen $u \in V$ möglichst gut approximiert werden können
    (weil $\lim_{h \to 0} \inf_{v \in V_h} \norm{u - v} = 0 \implies
    \lim_{h \to 0} \norm{u - u_h} = 0$).

    Für $a(\cdot, \cdot)$ symmetrisch gilt das Céa-Lemma sogar mit Faktor
    $\sqrt{\frac{\gamma}{\alpha}}$.
    Es gilt dann nämlich Normäquivalenz zur Energienorm mit
    $\sqrt{\alpha} \norm{v} \le \norm{v}_a \le \sqrt{\gamma} \norm{v}$
    (wenn man $\norm{v}_a^2 = a(v, v)$ einsetzt und Stetigkeit/Koerzivität ausnutzt).
    Daraus erhält man für $v \in V_h$\\
    $\norm{u - u_h}_a^{\cancel{2}} = a(u - u_h, u - u_h) = a(u - u_h, u - v) =
    \sp{u - u_h, u - v}_a \le \cancel{\norm{u - u_h}_a} \norm{u - v}_a$,
    also $\norm{u - u_h}_a \le \norm{u - v}_a$ bzw.
    $\norm{u - u_h} \le \sqrt{\frac{\gamma}{\alpha}} \norm{u - v}$.
\end{Bem}

\pagebreak

\subsection{%
    Notwendigkeit der Koerzivität%
}

\begin{Bem}
    Die Koerzivität ist bei der Galerkin-Projektion wesentlich.
    Für $a(\cdot, \cdot)$ nicht-koerziv kann die Galerkin-Projektion aus einem
    regulären System in $V$ ein singuläres System in $V_h$ erzeugen.
\end{Bem}

\begin{Bsp}
    Setze $V := \real^2$,
    $a(u, v) := u^\tp \smallpmatrix{1&0\\0&-1} u$,
    $\ell(v) := \smallpmatrix{1&1} v$.
    Dann ist $a(\cdot, \cdot)$ nicht-koerziv (negativer EW),
    aber das System ist regulär, weil\\
    $\forall_{v \in V}\; a(u, v) = \ell(v) \iff
    \forall_{v \in V}\; u^\tp \smallpmatrix{1&0\\0&-1} v = \smallpmatrix{1&1} v
    \iff u^\tp \smallpmatrix{1&0\\0&-1} = \smallpmatrix{1&1}
    \iff u = \smallpmatrix{1\\-1}$.\\
    Wählt man nun $V_h := \Span(\varphi_1)$ mit $\varphi_1 := \smallpmatrix{1\\1}$,
    dann ist das diskrete System singulär, weil das LGS
    $A_h d = b_h$ mit $A_h = a_{1,1} = a(\varphi_1, \varphi_1) = 0$ und
    $b_h = \ell(\varphi_1) = 2$ nicht lösbar ist.
\end{Bsp}

\linie

\begin{Bem}
    Ein Ausweg kann es sein,
    getrennte Ansatz- und Testräume zu verwenden\\
    (\begriff{\name{Petrov}-\name{Galerkin}-Projektion}), d.\,h.
    seien $V_h, \widetilde{V}_h \le V$ endlich-dimensional,
    suche $u_h \in V_h$ mit $\forall_{v \in \widetilde{V}_h}\; a(u_h, v) = \ell(v)$.
    $\widetilde{V}_h$ sollte so gewählt werden, dass das diskrete System regulär ist.
\end{Bem}

\begin{Bsp}
    Im Beispiel von eben
    seien $\varphi_1 \in \real^2 \setminus \Span(\smallpmatrix{1\\1})$,
    $\widetilde{\varphi}_1 := \smallpmatrix{1&0\\0&-1} \varphi_1$,
    $V_h := \Span(\varphi_1)$ und
    $\widetilde{V}_h := \Span(\widetilde{\varphi}_1)$.
    Damit ist
    $A_h = a(\varphi_1, \widetilde{\varphi}_1)
    = \varphi_1^\tp \smallpmatrix{1&0\\0&-1} \widetilde{\varphi}_1
    = \varphi_1^\tp \cancel{\smallpmatrix{1&0\\0&-1} \smallpmatrix{1&0\\0&-1}} \varphi_1
    = \norm{\varphi_1}^2
    > 0$
    und $b_h := \ell(\widetilde{\varphi}_1)$,
    d.\,h. das diskrete System ist jetzt regulär.
\end{Bsp}

\section{%
    Implementierung der Finite-Elemente-Methode%
}

\subsection{%
    1D-Beispiel (\name{Poisson}-Gleichung)%
}

\begin{Bem}
    Üblicherweise wählt man Galerkin-Verfahren mit
    stückweise polynomiellen, globalen stetigen Ansatzfunktionen, die einen lokalen Träger
    besitzen.
\end{Bem}

\begin{Bsp}
    Für $d := 1$, $\Omega := (0, 1)$
    sei das äquidistante Gitter
    $x_i := ih$, $i = 0, \dotsc, n + 1$, mit $n \in \natural$ und $h := \frac{1}{n+1}$ gegeben.
    Wähle als Ansatzfunktionen die \begriff{Hütchenfunktionen}
    $\varphi_j$ für $j = 1, \dotsc, n$
    (d.\,h. stückweise linear mit $\varphi_j(x_i) = \delta_{i,j}$).
    Man spricht auch von der \begriff{nodalen Basis}.
    Es gilt $\supp\varphi_j = [x_{j-1}, x_{j+1}]$.
    Als Ansatzraum erhält man den Raum $V_h := \Span((\varphi_j)_{j=1}^n) \le H^1_0(\Omega)$
    der linearen Splines.

    Für das Poisson-Problem $-u'' = f$ in $\Omega$ und $u(0) = 0 = u(1)$ wählt man
    $a(u, v) := \int_\Omega u'v'\dx$ und $\ell(v) := \int_\Omega fv\dx$.
    Mit der Ableitung $\varphi_j'(x) = 1/h$ für $x \in (x_{j-1}, x_j)$ und
    $\varphi_j'(x) = -1/h$ für $x \in (x_j, x_{j+1})$ bekommt man
    $a_{j,j} = \int_{x_{j-1}}^{x_{j+1}} \frac{1}{h^2} \dx = \frac{2}{h}$,
    $a_{j+1,j} = \int_{x_j}^{x_{j+1}} \frac{1}{h} (-\frac{1}{h}) \dx = -\frac{1}{h} = a_{j-1,j}$
    und $a_{i,j} = 0$ für $|i - j| \ge 2$, weil dann
    $|\supp(\varphi_i) \cap \supp(\varphi_j)| = 0$.

    Somit ist $A_h = \frac{1}{h}
    \smallpmatrix{2&-1&&\\-1&2&-1&\\&\ddots&\ddots&\ddots&\\&&-1&2&-1\\&&&-1&2}$
    tridiagonal und dünn besetzt,
    weswegen es selbst für große $n$ kein Speicherproblem gibt.
    Wegen der guten Approximationsfähigkeit von Splines erhält man mit dem
    Céa-Lemma eine gute Approximation durch die Galerkin-Projektion.
\end{Bsp}

\pagebreak

\subsection{%
    Simplizes%
}

\begin{Def}{Simplex}
    Seien $a_0, \dotsc, a_s \in \real^d$ in \begriff{allgemeiner Lage},
    d.\,h. $a_1 - a_0, \dotsc, a_s - a_0 \in \real^d$ linear unabhängig.
    Dann heißt $T := \Conv(a_0, \dotsc, a_s)
    := \{\sum_{j=0}^s \lambda_j a_j \;|\; \lambda_j \ge 0,\; \sum_{j=0}^s \lambda_j = 1\}$
    (nicht-degeneriertes) \begriff{$s$-dim. Simplex} in $\real^d$ mit
    \begriff{Eckenmenge} $\E(T) := \{a_0, \dotsc, a_s\}$.
\end{Def}

\begin{Def}{Seitensimplex}
    Seien $r \in \{0, \dotsc, s\}$ und $\{a_0', \dotsc, a_r'\} \subset \{a_0, \dotsc, a_s\}$.
    Dann heißt\\
    $S := \Conv(a_0', \dotsc, a_r')$ \begriff{$r$-dim. Seitensimplex}
    von $T$ mit \begriff{Eckenmenge} $\E(S) := \{a_0', \dotsc, a_r'\}$.
\end{Def}

\begin{Def}{Einheitssimplex}
    Der Simplex mit Ecken $0, e_1, \dotsc, e_d \in \real^d$
    heißt \begriff{Einheitssimplex} oder \begriff{Refe"-renzelement} $\widehat{T}$ in $\real^d$.
\end{Def}

\begin{Bem}
    $T$ heißt \begriff{Strecke}, falls $s = 1$ und $d \ge 1$,
    \begriff{Dreieck}, falls $s = 2$ und $d \ge 2$, und
    \begriff{Tetraeder}, falls $s = 3$ und $d \ge 3$.
    $S$ ist Ecke von $T$, falls $r = 0$,
    und \begriff{Kante} von $T$, falls $r = 1$.
\end{Bem}

\linie

\begin{Lemma}{baryzentrische Koordinaten}
    Sei $T$ ein $s$-dim. Simplex in $\real^d$ und $x \in T$.
    Dann gibt es eind. bestimmte \begriff{baryzentrische Koord.en}
    $(\lambda_j)_{j=0}^s$ mit $x = \sum_{j=0}^s \lambda_j a_j$,
    $\lambda_j \ge 0$ und $\sum_{j=0}^s \lambda_j = 1$.
\end{Lemma}

\begin{Bem}
    Mit baryzentrischen Koordinaten kann man $x \in T$ testen
    (für $x \in \Span(a_j)_{j=0}^d$ gilt
    $x \in T \iff \forall_{j=0,\dotsc,s}\; \lambda_j \ge 0$).
    Außerdem lässt sich für $x \in T$ herausfinden, ob $x$ auf einem echten Seitensimplex
    von $T$ liegt
    ($|\{j \;|\; \lambda_j \not= 0\}| - 1$ ist die Dimension des Seitensimplex).
\end{Bem}

\linie

\begin{Def}{geometrische Maße}
    Für einen Simplex $T$ seien
    $h_T := \diam(T)$ der \begriff{Durchmesser} von $T$,
    $\varrho_T := 2 \cdot \sup\{R > 0 \;|\; \exists_{x_0 \in T}\; B_R(x_0) \subset T\}$
    der \begriff{Inkugeldurchmesser} und
    $\sigma_T := \frac{h_T}{\varrho_T}$.
\end{Def}

\begin{Bem}
    $\sigma_T$ ist ein Maß für die Degeneriertheit von $T$
    ($\sigma_T$ groß, falls $T$ einen sehr spitzen Winkel hat,
    und $\sigma_T$ klein, falls $T$ ähnliche Winkel besitzt)
    und ist invariant unter Translation und Skalierung.
\end{Bem}

\linie

\begin{Bem}
    Bei der Fehleranalyse und bei der Implementierung der FEM werden Operationen
    oft auf dem Referenzelement durchgeführt und dann auf
    beliebige Simplizies durch Transformation übertragen.
\end{Bem}

\begin{Satz}{Referenzabbildung}\\
    Seien $T \subset \real^d$ ein $d$-dim. Simplex mit Ecken $\{a_j\}_{j=0}^d$ und
    $\widehat{T}$ das Referenzelement.
    Dann gilt:
    \begin{enumerate}
        \item
        Es gibt genau eine af"|fine Abbildung (\begriff{Referenzabbildung})
        $F_T\colon \widehat{T} \to T$,
        $F_T(\widehat{x}) := B\widehat{x} + t$,
        mit $B \in \real^{d \times d}$ regulär und $t \in \real^d$,
        sodass $F_T(e_j) = a_j$ für $j = 0, \dotsc, d$.

        \item
        $\norm{B} \le \frac{h_T}{\varrho_{\widehat{T}}}$
        (mit $\norm{B} = \norm{B}_2 := \sup_{\widehat{x}\not=0}
        \frac{\norm{B\widehat{x}}}{\norm{\widehat{x}}}$) und

        \item
        $\norm{B^{-1}} \le \frac{h_{\widehat{T}}}{\varrho_T}$

        \item
        $|\det B| = \frac{|T|}{|\widehat{T}|}$ und
        $\exists_{c, C > 0}\; c\varrho_T^d \le |\det B| \le Ch_T^d$
        mit $c, C$ unabhängig von $T$ (abh. von $d$)
    \end{enumerate}
\end{Satz}

\pagebreak

\subsection{%
    Triangulierungen in \texorpdfstring{$d$}{d} Dimensionen%
}

\begin{Def}{Triangulierung}
    Seien $\Omega \subset \real^d$ of"|fen, beschränkt und polygonal berandet sowie
    $I$ eine endliche Indexmenge.
    Dann heißt $\T_h := \{T_i \;|\; i \in I\}$ \begriff{zulässige Triangulierung}
    von $\Omega$, falls
    \begin{itemize}
        \item
        $\forall_{i \in I}\; [\text{$T_i \subset \real^d$ ist $d$-dim. Simplex}]$,

        \item
        $\bigcup_{i \in I} T_i = \overline{\Omega}$
        (\emph{Überdeckung}),

        \item
        $\forall_{i \not= j}\; \interior(T_i) \cap \interior(T_j) \not= \emptyset$
        (\emph{keine Überlappung}) und

        \item
        für $i \not= j$ ist $S := T_i \cap T_j$
        leer oder $S$ ist Seitensimplex von $T_i$ und von $T_j$
        (\begriff{Konformität}).
    \end{itemize}
    In diesem Fall heißt
    $h := \max_{i \in I} h_{T_i}$ \begriff{globale Gitterweite/Feinheit} von $\T_h$,
    $\varrho := \min_{i \in I} \varrho_{T_i}$ \begriff{minima"-ler Inkugelradius} von $\T_h$ und
    $\E(\T_h) = \bigcup_{i \in I} \E(T_i)$ \begriff{Ecken-/Knotenmenge} von $\T_h$.
\end{Def}

\begin{Bem}
    Eine zulässige Triangulierung besitzt keine hängenden Knoten.\\
    Man kann die FEM auch für nicht-konforme Gitter definieren (aber technisch aufwändiger).\\
    Wenn $\Omega$ keinen polygonalen Rand besitzt, dann kann man mit
    \begriff{isoparametrischen Elementen} dem Rand approximieren
    (Zulassen von nicht-linearen Referenzabbildungen).\\
    Man kann die FEM auch für Vierecksgitter, allgemeine polygonale Triangulierungen
    oder Gitter gemischter Typen durchführen.
\end{Bem}

\subsection{%
    Polynome in baryzentrischen Koordinaten%
}

\begin{Def}{Polynome auf Simplex}
    Seien $T \subset \real^d$ ein Simplex und $k \in \natural_0$.\\
    Dann heißt $\PP_k(T) := \{p\colon T \to \real \;|\;
    p(x) = \sum_{|\beta| \le k,\; \beta \in \natural_0^d} a_\beta x^\beta,\; a_\beta \in \real\}$
    \begriff{Raum der polynomialen Funktionen} bis Grad $k$ auf $T$,
    wobei $x^\beta := x_1^{\beta_1} \dotsm x_d^{\beta_d}$.
\end{Def}

\begin{Def}{Polynome auf Triangulierung}
    Sei $\T_h$ eine zul. Triangulierung von $\Omega$ und $k \in \natural_0$.\\
    Dann heißt
    $\PP_k(\T_h) := \{p \in \C^0(\Omega) \;|\; \forall_{T \in \T_h}\; p|_T \in \PP_k(T)\}$
    \begriff{Raum der global stetigen, stückweise polynomialen Fkt.en} und
    $\PP_{k,0}(\T_h) := \{p \in \PP_k(\T_h) \;|\; p|_{\partial\Omega} \equiv 0\}$
    \begriff{Teilraum mit Nullrandwerten}.
\end{Def}

\linie

\begin{Lemma}{Polynome in baryzentrischen Koordinaten}
    Sei $T \subset \real^d$ ein $d$-dim. Simplex.
    Dann gilt:
    \begin{enumerate}
        \item
        Für alle $p \in \PP_k(T)$ gibt es ein $\overline{p} \in \PP_k(\real^{d+1})$
        in der Form $\overline{p}(\lambda) = \sum_{1 \le |\beta| \le k,\;
        \beta \in \natural_0^{d+1}} d_\beta \lambda^\beta$, sodass
        $\forall_{x \in T}\; p(x) = \overline{p}(\lambda(x))$ mit
        $\lambda(x)$ den baryzentr. Koord.en von $x$ bzgl. $T$.

        \item
        Für alle $\overline{p} \in \PP_k(\real^{d+1})$ gilt
        $\overline{p}(\lambda(x))|_T \in \PP_k(T)$.
    \end{enumerate}
\end{Lemma}

\pagebreak

\subsection{%
    Lineare Interpolation auf Triangulierungen%
}

\begin{Satz}{lineares Finite Element/\name{Courant}-Element}\\
    Seien $T \subset \real^d$ ein $d$-dim. Simplex mit Ecken $\{a_j\}_{j=0}^d$
    und $p_0, \dotsc, p_d \in \real$.\\
    Dann gibt es genau ein $p \in \PP_1(T)$ mit $\forall_{j=0,\dotsc,d}\; p(a_j) = p_j$.
\end{Satz}

\begin{Bem}
    Für die Numerik wählt man eine konkrete lokale Basis $\Phi := (\varphi_j)_{j=1}^d$ von
    $\PP_1(T)$ (z.\,B. nodale Basis zu den Ecken) und schreibt $p \in \PP_1(T)$
    als Linearkombination dieser Basis.\\
    Die $\varphi_j$ heißen auch \begriff{Formfaktoren} (\begriff{shape functions}).
\end{Bem}

\linie

\begin{Satz}{Ex. + Eind. der $\PP_1(\T_h)$-Intp.}\\
    Seien $\T_h$ eine zul. Triangulierung mit $n_\E := |\E(\T_h)|$,
    $\{v_j\}_{j=1}^{n_\E} := \E(\T_h)$ und $p_1, \dotsc, p_{n_\E} \in \real$.\\
    Dann gibt es genau ein $p \in \PP_1(\T_h)$ mit
    $\forall_{j=1,\dotsc,n_\E}\; p(v_j) = p_j$.
\end{Satz}

\linie

\begin{Bem}
    Die $n_\E$-fache Anwendung des Satzes auf $p_j = \delta_{i,j}$ für
    $i = 1, \dotsc, n_\E$ liefert die Lagrange-Basis für $\PP_1(\T_h)$.
\end{Bem}

\begin{Satz}{\name{Lagrange}-Basis für $k = 1$}
    Sei $\T_h$ eine zul. Triangulierung mit $\{v_j\}_{j=1}^{n_\E} := \E(\T_h)$.\\
    Dann gibt es $\varphi_1, \dotsc, \varphi_{n_\E} \in \PP_1(\T_h)$ mit
    $\forall_{i,j=1,\dotsc,n_\E}\; \varphi_i(v_j) = \delta_{i,j}$.\\
    $\Phi := (\varphi_i)_{i=1}^{n_\E}$ ist eine Basis von $\PP_1(\T_h)$
    und heißt \begriff{\name{Lagrange}-/nodale Basis} von $\PP_1(\T_h)$.
\end{Satz}

\begin{Bem}
    Man kann zeigen, dass
    $\PP_1(\T_h) \le H^1(\Omega)$ (siehe folgender Satz) und\\
    $\PP_{1,0}(\T_h) =
    \{p \in \PP_1(\T_h) \;|\; \forall_{v_j \in \E(\T_h) \cap \partial\Omega}\; p(v_j) = 0\}
    \le H^1_0(\Omega)$,
    d.\,h. man kann $V_h := \PP_{1,0}(\T_h)$ im Galerkin-Verfahren verwenden
    (\begriff{lineare FEM}).
    Freiheitsgrade sind nur Werte in inneren Knoten.
\end{Bem}

\linie

\begin{Bem}
    Der folgende Satz begründet im Fall $k = 1$ die Forderung der globalen Stetigkeit
    (dann ist nämlich $\PP_1(\T_h) \le H^1(\Omega)$).
\end{Bem}

\begin{Satz}{schwache Ableitung auf Seitensimplizes}\\
    Seien $\T_h$ eine zul. Triangulierung,
    $k \in \natural$ und
    $v\colon \Omega \to \real$ mit
    $\forall_{T \in \T_h}\; v|_{\interior(T)} \in \C^k(\interior(T))$.\\
    Dann gilt $v \in H^k(\Omega) \iff v \in \C^{k-1}(\overline{\Omega})$.
\end{Satz}

\pagebreak

\subsection{%
    Polynomiale Interpolation auf Triangulierungen%
}

\begin{Bem}
    Es folgt eine Erweiterung von $\PP_1(\T_h)$ auf höhere Polynomgrade.
\end{Bem}

\begin{Def}{\name{Lagrange}-Gitter}
    Sei $T$ ein $d$-dim. Simplex mit Ecken $\{a_j\}_{j=0}^d$.\\
    Dann ist das \begriff{\name{Lagrange}-Gitter} der Ordnung $k \in \natural$ von $T$
    definiert durch\\
    $G_k(T) := \{\sum_{j=0}^d \lambda_j a_j \;|\;
    \lambda_j \in \{\frac{i}{k} \;|\; i = 0, \dotsc, k\},\; \sum_{j=0}^d \lambda_j = 1\}$.
\end{Def}

\begin{Bem}
    Für $k = 1$ ist $G_1(T) = \E(T)$ und $|G_1(T)| = d+1$.\\
    Für $k \ge 1$ ist $G_k(T) \supset \E(T)$ mit $|G_k(T)| = \binom{d+k}{k}$.\\
    Für einen $(d-1)$-dim. Seitensimplex $S \subset T$ gilt
    $|G_k(T) \cap S| = |G_k(S)| = \binom{d-1+k}{k}$.\\
    Es gilt
    $\{\lambda \in \real^{d+1} \;|\; \lambda_j \in \{\frac{i}{k} \;|\; i = 0, \dotsc, k\},\;
    \sum_{j=0}^d \lambda_j = 1\} \cong
    \{\beta \in \natural_0^{d+1} \;|\; |\beta| = k\}$ via
    $\lambda := \frac{\beta}{k}$.
\end{Bem}

\linie

\begin{Lemma}{baryzentrische \name{Lagrange}-Polynome}\\
    Seien $k \in \natural$ und
    $p_\beta(\lambda) := \prod_{\ell=0}^d \prod_{j=0}^{\beta_\ell-1}
    \frac{\lambda_\ell - j/k}{\beta_\ell/k - j/k}$
    für $\beta \in \natural_0^{d+1}$, $|\beta| = k$, und $\lambda \in \real^{d+1}$.
    Dann gilt
    \begin{enumerate}
        \item
        $p_\beta(\lambda) \in \PP_k(\real^{d+1})$ und

        \item
        $\forall_{\overline{\beta} \in \natural_0^{d+1},\; |\overline{\beta}| \le k}\;
        p_\beta(\frac{\overline{\beta}}{k}) = \delta_{\beta,\overline{\beta}}$
        (mit $\delta_{\beta,\overline{\beta}} :=
        \prod_{i=0}^d \delta_{\beta_i,\overline{\beta_i}}$).
    \end{enumerate}
\end{Lemma}

\begin{Satz}{allg. simpl. \name{Lagrange}-Element}
    Seien $k \in \natural$,
    $T$ ein $d$-dim. Simplex mit Lagrange-Gitter
    $\{v_j\}_{j=1}^{n_k} := G_k(T)$ für
    $k \in \natural$ und $n_k := |G_k(T)|$ sowie $p_1, \dotsc, p_{n_k} \in \real$.\\
    Dann gibt es genau ein $p \in \PP_k(T)$ mit $\forall_{j=1,\dotsc,n_k}\; p(v_j) = p_j$.
\end{Satz}

\linie

\begin{Satz}{Ex. + Eind. der $\PP_k(\T_h)$-Interpolation}\\
    Seien $\T_h$ eine zul. Triangulierung,
    $k \in \natural$,
    $\{v_j\}_{j=1}^{m_k} := G_k(\T_h) := \bigcup_{T \in \T_h} G_k(T)$
    die \begriff{Vereinigung aller \name{Lagrange}-Gitter} mit
    $m_k := |G_k(\T_h)|$ und
    $p_1, \dotsc, p_{m_k} \in \real$.\\
    Dann gibt es genau ein $p \in \PP_k(\T_h)$ mit
    $\forall_{j=1,\dotsc,m_k}\; p(v_j) = p_j$.
\end{Satz}

\linie

\begin{Bem}
    Die $m_k$-fache Anwendung des Satzes auf $p_j = \delta_{i,j}$ für
    $i = 1, \dotsc, m_k$ liefert die Lagrange-Basis für $\PP_k(\T_h)$.
\end{Bem}

\begin{Satz}{\name{Lagrange}-Basis für $k \in \natural$}\\
    Seien $\T_h$ eine zul. Triangulierung, $k \in \natural$ und
    $\{v_j\}_{j=1}^{m_k} := G_k(\T_h)$.\\
    Dann existieren $\varphi_1, \dotsc, \varphi_{m_k} \in \PP_k(\T_h)$
    mit $\forall_{i,j=1,\dotsc,m_k}\; \varphi_i(v_j) = \delta_{i,j}$.\\
    $\Phi := \{\varphi_i\}_{i=1}^{m_k}$ ist eine Basis von $\PP_k(\T_h)$ und heißt
    \begriff{\name{Lagrange}-/nodale Basis der Ordnung $k$}.\\
    $\Phi_0 := \{\varphi_i \in \Phi \;|\; v_i \notin \partial\Omega\}$
    ist eine Basis von $\PP_{k,0}(\T_h)$ und heißt
    \begriff{\name{Lagrange}-/nodale Basis von $\PP_{k,0}(\T_h)$} zur Knotenmenge
    $G_{k,0}(\T_h) := G_k(\T_h) \setminus \partial\Omega$,
    wobei $m_{k,0} := \dim \PP_{k,0}(\T_h) = |\Phi_0|$.
\end{Satz}

\begin{Def}{\name{Lagrange}-FEM-Approximation}
    Seien $a(\cdot, \cdot)$ stetig, koerziv,
    $\ell(\cdot)$ stetig auf $H^1_0(\Omega)$ und
    $\T_h$ eine zul. Triangulierung mit inneren Lagrange-Knoten $G_{k,0}(\T_h)$
    und nodalen Basisfunktionen $\Phi_0 = \{\varphi_i\}_{i=1}^{m_{k,0}}$.
    Setze $V_h := \PP_{k,0}(\T_h) = \Span(\Phi_0)$.\\
    Dann heißt $u_h \in V_h$ \begriff{\name{Lagrange}-FEM-Approximation},
    falls $\forall_{v \in V_h}\; a(u_h, v) = \ell(v)$.
\end{Def}

\pagebreak

\subsection{%
    Quadraturen%
}

\begin{Bem}
    Für die Aufstellung des Galerkin-LGS (Assemblierung) werden Integrale der Form
    $\int_\Omega (A\nabla \varphi_i) \nabla \varphi_j \dx$,
    $\int_\Omega f\varphi_j \dx$ und $\int_{\partial\Omega} g_N \varphi_j \dsigma(x)$
    durch Quadratur berechnet.
    Es reicht dabei, die Quadratur nur auf Referenzelementen zu betrachten,
    die auf beliebige Simplizes transformiert und zu zusammengesetzten
    Quadraturen für Gebietsintegrale kombiniert werden können.
\end{Bem}

\begin{Def}{Quadratur}
    Seien $\widehat{T} \subset \real^d$ der Einheitssimplex,
    $\widehat{x}_i \in \widehat{T}$ und $w_i \in \real$ für $i = 1, \dotsc, \ell$.\\
    Dann heißt $\widetilde{I}(g) := \sum_{i=1}^\ell w_i g(\widehat{x}_i)$
    \begriff{Quadratur} für $g \in \C^0(\widehat{T})$.\\
    $\widetilde{I}$ heißt \begriff{exakt auf $\PP_k(\widehat{T})$} oder
    \begriff{von Ordnung $\ge k$}, falls $\forall_{g \in \PP_k(\widehat{T})}\;
    \widetilde{I}(g) = I(g) := \int_{\widehat{T}} g(\widehat{x})\d\widehat{x}$.
\end{Def}

\begin{Bsp}
    \begin{itemize}
        \item
        Für $d \in \natural$ erhält man mit dem \begriff{Schwerpunkt}
        $x_S := \frac{1}{d+1} (1, \dotsc, 1)^\tp \in \real^d$ von $\widehat{T}$
        die \begriff{Mittel"-punktsintegration} $\widetilde{I}(g) := |\widehat{T}| g(x_S)$
        (von Ordnung $0$).

        \item
        Für $d = 2$ erhält man mit den \begriff{Kantenmittelpunkten}
        $m_{i,j} := \frac{1}{2} (e_i + e_j)$ für $i, j = 0, 1, 2$, $i \not= j$,
        die Formel
        $\widetilde{I}(g) := \frac{1}{3} |\widehat{T}| \sum_{i,j=0,\; i<j}^2 g(m_{i,j})$
        (von Ordnung $2$).

        \item
        Für $d = 2$ erhält man die Formel\\
        $\widetilde{I}(g) := \frac{1}{60} |\widehat{T}|
        (3 \sum_{i=0}^2 g(e_i) + 8 \sum_{i,j=0,\; i<j}^2 g(m_{i,j}) + 27 g(x_S))$
        (von Ordnung $3$).
    \end{itemize}
\end{Bsp}

\linie

\begin{Bem}
    Für einen $d$-dim. Simplex $T \subset \real^d$ mit Referenzabb.
    $F_T\colon \widehat{T} \to T$, $F_T(\widehat{x}) = B_T \widehat{x} + t_T$, gilt
    $\int_T g(x)\dx = |\det B_T| \cdot \int_{\widehat{T}} g(F_T(\widehat{x})) \d\widehat{x}
    \approx |\det B_T| \cdot \widetilde{I}(g \circ F_T) =: \widetilde{I}_T(g)$.\\
    $\widetilde{I}_T$ ist exakt auf $\PP_k(T)$ genau dann,
    wenn $\widetilde{I}$ exakt auf $\PP_k(\widehat{T})$ ist.
\end{Bem}

\begin{Bem}
    Bei Dif"|ferentialausdrücken muss man die Transformation richtig durchführen.
    Sei $\varphi \in \C^1(T)$, dann ist
    $\widehat{\varphi} := \varphi \circ F_T \in \C^1(\widehat{T})$ und es gilt
    $\nabla_{\widehat{x}} \widehat{\varphi}(\widehat{x})
    = (\D\widehat{\varphi}(\widehat{x}))^\tp
    = (\D\varphi(x) \cdot \D F_T)^\tp$\\
    $= ((\nabla_x \varphi(x))^\tp \cdot B)^\tp
    = B^\tp \nabla_x \varphi(x)$ mit $x = F_T(\widehat{x})$.\\
    Damit gilt z.\,B. für die Steifigkeitsmatrix-Einträge\\
    $\int_T (\nabla_x \varphi_i)^\tp A (\nabla_x \varphi_j) \dx
    = \int_{\widehat{T}} (\nabla_{\widehat{x}} \widehat{\varphi}_{\widehat{i}})^\tp
    B^{-1} A B^{-\tp} (\nabla_{\widehat{x}} \widehat{\varphi}_{\widehat{j}})
    |\det B| \d\widehat{x}$\\
    für die nodale Basis $\{\widehat{\varphi}_{\widehat{i}}\}_{\widehat{i}=1}^{n_k}$ und
    geeignete $\widehat{i}, \widehat{j} \in \{1, \dotsc, n_k\}$.
\end{Bem}

\begin{Bem}
    Gebietsintegrale werden einfach approximiert durch zusammengesetzte Quadraturen, d.\,h.
    $\int_\Omega g(x)\dx
    = \int_{T \in \T_h} \int_T g(x)\dx
    \approx \sum_{T \in \T_h} \widetilde{I}_T(g)
    = \sum_{T \in \T_h} |\det B_T| \sum_{i=1}^\ell w_i g(F_T(\widehat{x}_i))$.
\end{Bem}

\begin{Bem}
    Die Ordnung der Quadratur sollte der (noch zu diskutierenden) FEM-Konvergenz"-ordnung
    angepasst sein.
    Zum einen sollte die Quadraturordnung hoch genug sein, damit die Konvergenz für $h \to 0$
    nicht beeinträchtigt wird.
    Zum anderen sollte sie aber auch nicht zu hoch sein, damit nicht ein Großteil der Rechenzeit
    für die Quadratur verwendet wird.
\end{Bem}

\pagebreak

\subsection{%
    Assemblierung%
}

\begin{Bem}
    Der Zusammenhang zwischen den lokalen und den globalen Freiheitsgraden wird durch eine
    sog. globale Indexabbildung realisiert.
    Seien dazu
    $\{\widehat{\varphi}_{\widehat{j}}\}_{\widehat{j}=1}^{n_k}$
    eine Basis von $\PP_k(\widehat{T})$ und
    $\{\varphi_i\}_{i=1}^{m_k}$ eine Basis von $\PP_k(\T_h)$.
    Dann heißt
    $\widehat{g}\colon \T_h \times \{1, \dotsc, n_k\} \to \{1, \dotsc, m_k\}$
    \begriff{globale Indexabbildung},
    falls $\forall_{T \in \T_h} \forall_{\widehat{j}=1,\dotsc,n_k}\;
    \widehat{\varphi}_{\widehat{j}} = \varphi_i \circ F_T$ für
    $i := \widehat{g}(T, \widehat{j})$.\\
    Mit der globalen Indexabb. ist die Kenntnis von $\{\varphi_i\}_{i=1}^{m_k}$
    nicht mehr nötig, es reicht, eine Basis auf dem Referenzelement zu definieren.
\end{Bem}

\linie

\begin{Bem}
    Die direkte Berechnung der Steifigkeitsmatrix ist i.\,A. teuer.
    Für die Poisson-Gleichung muss man
    $a_{i,j} = \int_\Omega (\nabla\varphi_i)^\tp (\nabla\varphi_j) \dx
    = \sum_{T \in \T_h} \int_T (\nabla\varphi_i)^\tp (\nabla\varphi_j) \dx$
    für $i,j = 1, \dotsc, m$ mit $m := m_{k,0}$ berechnen.
    Für $k = 1$ ist $|\T_h| = \O(m)$, d.\,h. der Gesamtaufwand für die Berechnung von $A_h$
    ist $\O(m^3)$ (inpraktikabel für $m$ groß).

    Stattdessen nutzt man die Lokalität und die globale Indexabbildung:\\
    Für $S_{i,j} := \supp(\varphi_i) \cap \supp(\varphi_j)$ gilt
    $a_{i,j} = \int_{S_{i,j}} (\nabla\varphi_i)^\tp (\nabla\varphi_j) \dx$\\
    $= \sum_{T \in \T_h,\; T \subset S_{i,j}} \int_T \nabla\varphi_i \nabla\varphi_j \dx
    = \sum_{T \in \T_h,\; T \subset S_{i,j}} \widehat{a}_{\widehat{i},\widehat{j}}$,
    $\widehat{a}_{\widehat{i},\widehat{j}}
    := \int_{\widehat{T}} |\det B_T| (\nabla\widehat{\varphi}_{\widehat{i}})^\tp
    B_T^{-1} B_T^{-\tp} (\nabla\widehat{\varphi}_{\widehat{j}}) \d\widehat{x}$,\\
    mit $\widehat{i}, \widehat{j} \in \{1, \dotsc, n_k\}$, sodass
    $i = \widehat{g}(T, \widehat{i})$ und $j = \widehat{g}(T, \widehat{j})$.

    Statt einer Schleife über $(i,j)$ kann man nun durch Addition der Beiträge
    der \begriff{lokalen Steifigkeits"-matrix}
    $A_{h,T} := (\widehat{a}_{\widehat{i},\widehat{j}})_{\widehat{i},\widehat{j}=1}^{n_k}$
    die globale Steifigkeitsmatrix $A_h$ assemblieren:
    \begin{enumerate}
        \item
        Setze $A_h := 0$.

        \item
        Wiederhole für alle $T \in \T_h$:
        \begin{enumerate}[label=\emph{(\roman*)}]
            \item
            Berechne $A_{h,T}$.

            \item
            Wiederhole für alle $\widehat{i}, \widehat{j} = 1, \dotsc, n_k$:
            Setze $(A_h)_{g(T,\widehat{i}),g(T,\widehat{j})} \;+\!\!:=
            (A_{h,T})_{\widehat{i},\widehat{j}}$.
        \end{enumerate}
    \end{enumerate}
    Die Gesamtkomplexität beträgt nun $\O(|\T_h| n_k^2) = \O(mn_k^2)$
    was wegen $n_k$ konstant und klein wesentlich besser als $\O(m^3)$ ist.

    Ähnlich ist die Assemblierung von $b_h$ möglich.
    Das geht sogar simultan mit $A_h$ (ohne zusätzliche Schleifen),
    d.\,h. ein einziger Gitterdurchlauf reicht zur Assemblierung des gesamten Systems aus.
\end{Bem}

\linie

\begin{Bem}
    $A_h$ ist dünnbesetzt,
    da $(A_h)_{i,j} = 0$ für $|\supp(\varphi_i) \cap \supp(\varphi_j)| = 0$.\\
    Ist $r$ die maximale Kantenzahl für einen Knoten,
    dann existieren für $k = 1$ höchstens $r + 1$ Nichtnull-Einträge pro Zeile und
    für $k > 1$ höchstens $r \cdot |G_k(T)|$.\\
    Dies muss man bei der Implementierung durch Sparse-Matrizen berücksichtigen
    (insb. sollte bei Verfeinerungen $r$ nicht unbegrenzt wachsen).
\end{Bem}

\begin{Bem}
    Für $k \ge d+1$ gibt es innere Lagrange-Knoten.
    Die entsprechenden Zeilen von $A_h$ haben nur Einträge für Knoten desselben Simplex,
    nicht aber seiner Nachbarn.
    Dies kann man ausnutzen, indem man z.\,B. durch Zeilenumformungen Einheitsvektoren
    in den jeweiligen Zeilen erzeugt und so nur Unbekannte auf Seitensimplizes übrig bleiben
    (vorteilhaft für $k$ groß).\\
    Man nennt dies \begriff{innere/statische Kondensation}.
\end{Bem}

\pagebreak

\subsection{%
    Verallgemeinerungen%
}

\begin{Def}{finites Element}
    Das Tripel $(T, \Phi, \N)$ heißt \begriff{finites Element}, falls
    \begin{itemize}
        \item
        $T \subset \real^d$ beschr. und abg. mit Lipschitz-Rand und $\interior(T) \not= \emptyset$
        (\begriff{Element-Geometrie}),

        \item
        $\Phi := \{\varphi_1, \dotsc, \varphi_k\}$ l.\,u. Fkt.en auf $T$
        (\begriff{Formfaktoren/-fkt.en}, \begriff{shape functions}) mit\\
        $\P := \Span(\Phi)$ dem zugehörigen \begriff{diskreten Fkt.enraum} und

        \item
        $\N := \{N_1, \dotsc, N_k\}$
        eine Basis von $\P'$ ist (\begriff{Menge der lokalen Freiheitsgrade}).
    \end{itemize}
    $\Phi$ heißt \begriff{nodale Basis} und die $N_i$ heißen \begriff{nodale Variablen},
    falls $\forall_{i,j=1,\dotsc,k}\; N_i(\varphi_j) = \delta_{i,j}$.
\end{Def}

\linie

\begin{Bem}
    Die Definition verallgemeinert Lagrange-Elemente.\\
    Sei $T$ ein Simplex mit Lagrange-Gitter
    $\{v_i\}_{i=1}^{n_{\widetilde{k}}} := G_{\widetilde{k}}(T)$,
    $k := n_{\widetilde{k}}$,
    $\Phi$ die nodale Basis von $\PP_{\widetilde{k}}(T)$ (d.\,h. $\varphi_i(v_j) = \delta_{i,j}$)
    und $N_i(p) := p(v_i)$ für $p \in \PP_{\widetilde{k}}(T)$,
    dann ist $(T, \Phi, \N)$ ein finites Element.
    Wegen $N_i(\varphi_j) = \varphi_j(v_i) = \delta_{i,j}$ sind die $N_i$ nodale Variablen.
\end{Bem}

\linie

\begin{Bem}
    Statt Simplizes kann man auch andere Geometrien verwenden.
    Auf Rechtecken und Würfeln verwendet man statt linearen Formfaktoren
    eher bi- bzw. trilineare.\\
    Allgemein sei für $d \in \natural$ der Funktionenraum
    $Q_1([0,1]^d) := \bigotimes_{i=1}^d \PP_1([0,1])$ aller $d$-variaten Polynome
    mit Koordinatengrad $\le 1$ definiert.\\
    Beispielsweise hat $p \in Q_1([0,1]^2)$ für $d = 2$
    die Gestalt $p(x_1, x_2) = a + bx_1 + cx_2 + dx_1 x_2$.
    Man setzt nun $\Phi := (x_1 x_2, (1-x_1)x_2, x_1(1-x_2), (1-x_1)(1-x_2))^\tp$
    und $\N_i(p) := p(a_i)$ für $p \in Q_1([0,1]^2)$ und $i = 1, \dotsc, 4$ mit
    $a_1, \dotsc, a_4 \in \{0, 1\}^2$ den Ecken von $[0, 1]^2$.\\
    Analog verfährt man für $d = 3$ (trilineare Elemente) oder
    für $\widetilde{k} = 2$ (bi-/triquadratische Elemente).
\end{Bem}

\linie

\begin{Bem}
    Statt Punktauswertungen kann man auch Ableitungswerte vorgeben.
    Das \begriff{kubische \name{Hermite}-Element} erhält man wie folgt:
    Sei $T \subset \real^2$ ein Dreieck mit Ecken $a_0, a_1, a_2$ und Schwerpunkt $x_S$.
    Dann ist durch Vorgabe von $p(a_i), \nabla p(a_i), p(x_S)$ für $i = 0, 1, 2$
    eindeutig ein interpolierendes Polynom $p \in \PP_3(T)$ definiert.
    Die lokalen Freiheitsgrade sind somit gegeben durch
    $\N := (N_1(p), \dotsc, N_{10}(p))$\\
    $:= (p(x_S), p(a_0), p(a_1), p(a_2), \partial_{x_1} p(a_0), \partial_{x_1} p(a_1),
    \partial_{x_1} p(a_2), \partial_{x_2} p(a_0), \partial_{x_2} p(a_1), \partial_{x_2} p(a_2))$.\\
    Dabei existiert eine eindeutige Basis $\Phi := (\varphi_1, \dotsc, \varphi_{10})$ von
    $\PP_3(T)$ mit $N_i(\varphi_j) = \delta_{i,j}$.\\
    Man kann zeigen, dass zusammengesetzte Funktionen aus kubischen Hermite-Elementen
    (auf zul. Triangulierungen) in $H^1(\Omega)$ sind, i.\,A. aber nicht in $H^2(\Omega)$.
\end{Bem}

\linie

\begin{Bem}
    Ein Element, welches durch Zusammensetzen $H^2(\Omega)$-Funktionen liefert, ist das sog.
    \begriff{\name{Agyris}-Element} auf Dreiecken.
    Dazu sei $T \subset \real^2$ ein Dreieck mit Ecken $a_0, a_1, a_2$ und
    Kantenmittelpunkten $m_0, m_1, m_2$.
    Dann ist durch Vorgabe von $p(a_i), \nabla p(a_i), \D^2 p(a_i), \nabla p(m_i) \cdot n_i$
    für $i = 0, 1, 2$ (mit $n_i \in \real^2$ dem Normalenvektor in $m_i$) eindeutig ein
    interpolierendes Polynom\\
    $p \in \PP_5(T)$ definiert
    (dabei enthält $\D^2 p(a_i)$ die drei Unbekannten
    $\partial_{x_1}^2 p(a_i), \partial_{x_2}^2 p(a_i), \partial_{x_1} \partial_{x_2} p(a_i)$).
\end{Bem}

\pagebreak

\section{%
    Approximationssätze und FEM-Fehlerabschätzung%
}

\begin{Bem}
    Zur Motivation sei $I_h\colon V \to V_h$ ein linearer (Interpolations-)Operator mit
    Approximationsgüte $\norm{u - I_h u} \le Ch^r$ mit $C > 0$ möglichst klein und
    $r > 0$ möglichst groß.
    Dann folgt mit dem Lemma von Céa, dass
    $\norm{u - u_h} \le \frac{\gamma}{\alpha} \inf_{v \in V_h} \norm{u - v}
    \le \frac{\gamma}{\alpha} \norm{u - I_h u} \le \frac{\gamma}{\alpha} Ch^r$,
    also die Konvergenz für $h \to 0$, die Konvergenzordnung $r$ und eine
    Fehlerschranke für die Galerkin-Projektion $u_h$.
\end{Bem}

\begin{Bem}
    Seien $\Omega \subset \real^d$ polygonal berandet, $V := H^m(\Omega)$,
    $V_h := \PP_k(\T_h)$ für eine zul. Triang. $\T_h$ von $\Omega$
    und $I_h$ der Lagrange-Interpolationsoperator.
    Damit $I_h\colon H^m(\Omega) \to \PP_k(\T_h)$ sinnvoll definiert ist,
    muss $H^m(\Omega)$ Punktauswertungen erlauben,
    d.\,h. jedes $v \in H^m(\Omega)$ muss einen stetigen Repräsentanten
    $\widetilde{v} \in \C^0(\Omega)$ besitzen mit $\norm{v - \widetilde{v}}_{H^m(\Omega)} = 0$,
    damit Punkauswertungen von $v$ als Punktauswertungen von $\widetilde{v}$ definiert werden
    können.\\
    Nach dem 2. Sobolevschen Einbettungssatz kann $m$ je nach $d$ aber immer so gewählt werden,
    dass $I_h$ wohldefiniert ist
    (z.\,B. für $d = 2$ reicht z.\,B. $m = 2$).
    Im Folgenden seien $d, m$ immer derart, dass $I_h$ wohldefiniert ist.
\end{Bem}

\subsection{%
    \name{Bramble}-\name{Hilbert}-Lemma%
}

\begin{Def}{gebrochene Normen}
    Sei $\T_h$ eine zul. Triang. von $\Omega$.
    Dann ist der \begriff{gitterabhängige Raum}
    $H^m(\T_h) := \{v\colon \Omega \to \real \;|\; \forall_{T \in \T_h}\; v|_T \in H^m(T)\}$
    zusammen mit der Seminorm\\
    $|v|_{H^m(\T_h)} := \sqrt{\sum_{T \in \T_h} |v|_{H^m(T)}^2}$
    und der Norm $\norm{v}_{H^m(\T_h)} := \sqrt{\sum_{T \in \T_h} \norm{v}_{H^m(T)}^2}$ definiert.
\end{Def}

\begin{Bem}
    Of"|fensichtlich gilt $H^m(\T_h) \supset H^m(\Omega)$ und
    $\forall_{v \in H^m(\Omega)}\; \norm{v}_{H^m(\T_h)} = \norm{v}_{H^m(\Omega)}$.
\end{Bem}

\begin{Satz}{\name{Rellich}scher Auswahlsatz}
    Seien $m \in \natural_0$ und $\Omega \subset \real^d$ polygonal berandet.\\
    Dann ist die Einbettung $H^{m+1}(\Omega) \to H^m(\Omega)$ kompakt.
\end{Satz}

\begin{Bem}
    Die Einheitskugel in $H^{m+1}(\Omega)$ ist kompakt bzgl. der
    $\norm{\cdot}_{H^m(\Omega)}$-Norm bzw.
    jede bzgl. $\norm{\cdot}_{H^{m+1}(\Omega)}$ beschränkte Folge enthält eine
    bzgl. $\norm{\cdot}_{H^m(\Omega)}$ konvergente Teilfolge.
\end{Bem}

\linie

\begin{Satz}{lokaler Interpolationsfehler}
    Seien $K \subset \real^p$ polygonal berandet und abgeschlossen
    sowie $k \ge 2$ und $\{x_i\}_{i=1}^{n_k} \subset K$, sodass
    die Polynominterpolation $I\colon H^k(K) \to \PP_{k-1}(K)$ wohldefiniert ist
    (insb. ist die Einbettung $H^k(K) \to \C^0(K)$ stetig und $n_k := \binom{k+1}{2}$).\\
    Dann gilt $\exists_{C = C(K, k) > 0}
    \forall_{v \in H^k(K)}\; \norm{v - Iv}_{H^k(K)} \le C |v|_{H^k(K)}$.
\end{Satz}

\begin{Bem}
    Die Voraussetzungen sind z.\,B. für $d = k = 2$, $K = T$ Dreieck
    und $\{x_1, x_2, x_3\}$ Ecken von $T$ erfüllt.
\end{Bem}

\linie

\begin{Lemma}{\name{Bramble}-\name{Hilbert}}
    Seien $K \subset \real^d$ polygonal berandet und abgeschlossen sowie $k \ge 2$ und
    $g \in (H^k(K))'$ mit $\forall_{p \in \PP_{k-1}(K)}\; g(p) = 0$.\\
    Dann gilt $\exists_{C = C(g, K, k) > 0}
    \forall_{v \in H^k(K)}\; |g(v)| \le C |v|_{H^k(K)}$.
\end{Lemma}

\pagebreak

\subsection{%
    Interpolationsabschätzung%
}

\begin{Satz}{Transformationsformel}\\
    Seien $T$ ein $d$-dim. Simplex mit Referenzabbildung $F_T(\widehat{x}) = B\widehat{x} + t$
    sowie $v \in H^m(T)$.\\
    Dann ist $\widehat{v} := v \circ F_T \in H^m(\widehat{T})$ und es gilt\\
    $\exists_{C = C(d, m) > 0} \forall_{v \in H^m(T)}\;
    |\widehat{v}|_{H^m(\widehat{T})} \le C \norm{B}^m |\det B|^{-1/2} |v|_{H^m(T)}$.
\end{Satz}

\linie

\begin{Satz}{Interpolationsabschätzung}\\
    Seien $k \ge 2$,
    $\T_h$ eine zul. Triangulierung von $\Omega \subset \real^d$,
    $h \le h_{\max}$ und
    $\sigma > 0$ mit $\forall_{T \in \T_h}\; \sigma_T \le \sigma$.
    Sei außerdem $I_h\colon H^k(\Omega) \to \PP_{k-1}(\T_h)$ die Lagrange-Interpolation.\\
    Dann gilt $\exists_{C = C(k, \Omega, d, h_{\max}, \sigma) > 0}
    \forall_{m=0,\dotsc,k} \forall_{u \in H^k(\Omega)}\;
    \norm{u - I_h u}_{H^m(\T_h)} \le Ch^{k-m} |u|_{H^k(\Omega)}$.
\end{Satz}

\begin{Bem}
    Für $k = 2$ und $I_h\colon H^2(\Omega) \to \PP_1(\T_h)$ erhält man\\
    $\norm{u - I_h u}_{H^1(\Omega)} \le Ch |u|_{H^2(\Omega)}$ und
    $\norm{u - I_h u}_{L^2(\Omega)} \le Ch^2 |u|_{H^2(\Omega)}$.\\
    Ähnliche Abschätzungen gelten auch für quadratische Gitter und bilineare Elemente.
\end{Bem}

\linie

\begin{Def}{nicht-entartet}
    Eine Folge $(\T_i)_{i\in\natural}$ von zul. Triang.en mit $\lim_{i \to \infty} h_i = 0$
    heißt \begriff{nicht-entartet}, falls
    $\exists_{\sigma>0} \forall_{i \in \natural} \forall_{T \in \T_i}\;
    \sigma_T = \frac{h_T}{\varrho_T} \le \sigma$.
\end{Def}

\begin{Def}{quasi-uniform}
    Eine Folge $(\T_i)_{i\in\natural}$ von zul. Triang.en mit $\lim_{i \to \infty} h_i = 0$
    heißt \begriff{quasi-uniform}, falls
    $\exists_{\sigma>0} \forall_{i \in \natural} \forall_{T \in \T_i}\;
    \frac{h_i}{\sigma_T} \le \sigma$.
\end{Def}

\begin{Bem}
    Nicht-entartete Gittersequenzen lassen auch lokale Verfeinerungen zu und die
    Innenwinkel sind nach unten beschränkt.
    Bei quasi-uniformen Gittersequenzen ist $h_T \le h \le Ch_T$.\\
    Quasi-uniforme Gittersequenzen sind nicht-entartet.
\end{Bem}

\linie

\begin{Satz}{Konvergenz der Interpolation}
    Seien $k \ge 2$,
    $\{\T_i\}$ eine nicht-entartete Gittersequenz,
    $h_{\max} := \max_{i \in \natural} h_i$ und
    $I_{h_i}\colon H^k(\Omega) \to \PP_{k-1}(\T_i)$ die Lagrange-Interpolation.\\
    Dann gilt $\forall_{m = 0, \dotsc, k - 1} \forall_{u \in H^m(\Omega)}\;
    \lim_{i \to \infty} \norm{u - I_{h_i} u}_{H^m(\T_i)} = 0$.
\end{Satz}

\linie

\begin{Bem}
    Die Interpolationsabschätzung sagt insb. aus, dass\\
    $\norm{u - I_h u}_{H^m(\T_h)} \le Ch^{k-m} \norm{u}_{H^k(\Omega)}$,
    d.\,h. der Interpolationsfehler wird für $m < k$ in schwächeren Normen gemessen als $u$,
    wobei man $h$-Potenzen gewinnt.\\
    Sog. \begriff{inverse Abschätzungen} leisten das Umgekehrte, indem man $h$-Potenzen opfert.
    Sie gelten aber nur für FE-Ansatzfunktionen, nicht für den ganzen Sobolev-Raum.
\end{Bem}

\begin{Satz}{inverse Abschätzung}\\
    Seien $k \in \natural$,
    $\T_h$ eine zul. Triangulierung von $\Omega \subset \real^d$ und
    $\sigma > 0$ mit $\forall_{T \in \T_h}\; \sigma_T \le \sigma$.\\
    Dann gilt $\exists_{C = C(k, \Omega, d, \sigma) > 0}
    \forall_{m=0,\dotsc,k} \forall_{v_h \in \PP_k(\T_h)}\;
    \norm{v_h}_{H^k(\T_h)} \le Ch^{m-k} \norm{v_h}_{H^m(\T_h)}$.
\end{Satz}

\begin{Bem}
    Für $d = 2$ und lineare Elemente ($k = 1$, $m = 0$) gilt also\\
    $\norm{v_h}_{H^1(\T_h)} \le Ch^{-1} \norm{v_h}_{L^2(\T_h)}$.
\end{Bem}

\pagebreak

\subsection{%
    FEM-a-priori-Abschätzungen%
}

\begin{Satz}{FEM-a-priori-Fehlerschranke in $H^1$}\\
    Seien $\Omega \subset \real^d$ of"|fen, beschränkt und polygonal berandet,
    $a(\cdot, \cdot)$ stetig, koerziv und $\ell(\cdot)$ stetig auf $H^1_0(\Omega)$,
    $u \in H^1_0(\Omega)$ die eindeutige schwache Lösung,
    $\T_h$ eine zulässige Triangulierung und $\sigma > 0$ mit
    $\forall_{T \in \T_h}\; \sigma_T \le \sigma$ sowie
    $V_h := \PP_{k,0}(\T_h)$ mit $k \in \natural$ und
    $u_h \in V_h$ der Lagrange-FEM-Lösung.\\
    Wenn es ein $s \in \natural$ gibt mit $h \in H^{s+1}(\Omega) \cap H^1_0(\Omega)$, dann gilt\\
    $\exists_{C = C(\Omega, d, \sigma, k) > 0}\;
    \norm{u - u_h}_{H^1(\Omega)} \le Ch^s \norm{u}_{H^{s+1}(\Omega)}$.
\end{Satz}

\begin{Satz}{Konvergenz der FEM}
    Sei $(\T_i)_{i \in \natural}$ eine nicht-entartete Gittersequenz
    ($\lim_{i \to \infty} h_i = 0$), sodass die Voraussetzungen des
    vorherigen Satzes für jedes $\T_i$ erfüllt seien.\\
    Dann gilt $\lim_{i \to \infty} \norm{u_{h_i} - u}_{H^1(\Omega)} = 0$.
\end{Satz}

\linie

\begin{Bem}
    Um $u \in H^{s+1}(\Omega)$ zu garantieren, wären zusätzlich
    $\C^{s+1}$-Regularität von $\partial\Omega$ und $f \in H^{s-1}(\Omega)$ für
    $\ell(v) = \int_\Omega fv \dx$ notwendig (Satz von Friedrichs).\\
    Für lineare FEM ($k = 1$) reicht $H^2$-Regularität ($s = 1$),
    denn dann folgt lineare Konvergenz in der $H^1$-Norm, weil
    $\norm{u - u_h}_{H^1(\Omega)} \le Ch\norm{u}_{H^2(\Omega)} \le Ch\norm{f}_{L^2(\Omega)}$.\\
    Für polygonal berandete Gebiete kann man eigentlich keine $H^3$-Regularität garantieren,
    weil kein $\C^3$-Rand vorhanden ist.
    Daher kann man nicht garantieren, dass quadratische oder kubische FEM eine bessere
    Konvergenzordnung besitzen.
    In der Praxis sind allerdings quadratische/kubische FEM gegenüber linearen FEM
    zu bevorzugen.
\end{Bem}

\subsection{%
    \name{Aubin}-\name{Nitsche}-Trick%
}

\begin{Bem}
    Wegen $\norm{\cdot}_{L^2(\Omega)} \le \norm{\cdot}_{H^1(\Omega)}$
    folgt aus der A-priori-$H^1$-Fehlerschranke trivialerweise eine $L^2$-Abschätzung
    mit derselben $h$-Potenz.
    Dies ist jedoch nicht optimal, mittels eines Dualitätsarguments
    (\begriff{\name{Aubin}-\name{Nitsche}-Trick}) gewinnt man eine $h$-Potenz.
\end{Bem}

\begin{Satz}{\name{Aubin}-\name{Nitsche}}
    Seien $H$ ein Hilbertraum mit Skalarprodukt $\sp{\cdot, \cdot}_H$ und Norm $\norm{\cdot}_H$
    und $V \le H$ ein Unterraum, der mit dem Skalarprodukt
    $\sp{\cdot, \cdot}_V$ und der Norm $\norm{\cdot}_V$ ein Hilbertraum ist,
    sodass die Einbettung $V \to H$ stetig ist
    (d.\,h. $\forall_{v \in V}\; \norm{v}_H \le C \norm{v}_V$).\\
    Sei außerdem die schwache Form $\forall_{v \in V}\; a(u, v) = \ell(v)$
    mit $a\colon V \times V \to \real$ einer stetigen, koerziven Bilinearform,
    wobei $u \in V$ die schwache Lösung und $u_h \in V_h \le V$
    die Galerkin-Projektion ist.\\
    Dann gilt $\norm{u - u_h}_H = \gamma \norm{u - u_h}_V \cdot
    \sup_{g \in H \setminus \{0\}} (\frac{1}{\norm{g}_H} \inf_{v_h \in V_h}
    \norm{\varphi_g - v_h}_V)$,
    wobei für $g \in H$ die \begriff{duale Lösung} $\varphi_g \in V$ definiert ist als
    die schwache Lösung von $\forall_{w \in V}\; a(w, \varphi_g) = \sp{g, w}_H$ ist.
\end{Satz}

\linie

\begin{Satz}{FEM-a-priori-Fehlerschranke in $L^2$}
    Unter den Vor.en der A-priori-$H^1$-Fehlerschranke gilt
    $\norm{u - u_h}_{L^2(\Omega)} \le Ch \norm{u - u_h}_{H^1(\Omega)}$,
    d.\,h. insb. im Fall von $H^{s+1}$-Regularität\\
    $\norm{u - u_h}_{L^2(\Omega)} \le Ch^{s+1} \norm{u}_{H^{s+1}(\Omega)}
    \le Ch^{s+1} \norm{f}_{H^{s-1}(\Omega)}$.
\end{Satz}

\begin{Bem}
    Für lineare FEM ($k = 1$, $s = 1$) folgt im Fall von $H^2$-Regularität\\
    $\norm{u - u_h}_{L^2(\Omega)} \le Ch^2 \norm{u}_{H^2(\Omega)}$.
\end{Bem}

\begin{Bem}
    Die bisherigen Abschätzungen können keine lokal großen Fehler ausschließen,
    was das Folgende macht.
\end{Bem}

\begin{Satz}{FEM-a-priori-Fehlerabschätzung in $L^\infty$}
    Unter den Vor.en der A-priori-$H^1$-Fehlerschran"-ke sowie
    $d = 2$ und $\T_h$ quasi-uniform gilt
    $\norm{u - u_h}_{L^\infty(\Omega)} \le Ch\norm{f}_{L^2(\Omega)}$.
\end{Satz}

\begin{Bem}
    Die Abschätzung ist nicht scharf, für $d = 2$ gilt z.\,B.\\
    $\norm{u - u_h}_{L^\infty(\Omega)} \le Ch^2 |\log h|^{3/2} \norm{D^2 u}_{L^\infty(\Omega)}$,
    für $d = 3$ verschwindet der $\log$-Term sogar.
\end{Bem}

\subsection{%
    A-posteriori-Schätzer und Gitteradaptivität%
}

\begin{Bem}
    Lagrange-Interpolationsoperatoren erfordern Punktauswertungen, d.\,h.
    mindestens $u \in H^2(\Omega)$.
    Sog. Clément-Operatoren bieten eine Appr.möglichkeit für $H^1(\Omega)$-Funktionen.
\end{Bem}

\begin{Def}{Patches}
    Sei $\T_h$ eine zul. Triangulierung von $\Omega$ mit
    $\{v_i\}_{i=1}^{m_\E} := \E(\T_h)$.\\
    Zu jedem $v_i$ definiere
    den \begriff{Patch aller angrenzenden Elemente}
    $w_i := \bigcup_{T \in \T_h,\; v_i \in T} T$ und für jedes $T \in \T_h$
    den \begriff{Patch der Nachbarn}
    $w_T := \bigcup_{v_i \in T} w_i = \bigcup_{T' \in \T_h,\; T' \cap T \not= \emptyset} T'$.
\end{Def}

\begin{Def}{\name{Clément}-Approximation}
    Für $V_h := \PP_1(\T_h)$ mit nodaler Basis $\{\varphi_i\}_{i=1}^{m_\E}$ sei der\\
    \begriff{\name{Clément}-Operator} $C_h\colon H^1(\Omega) \to \PP_1(\T_h)$ definiert
    durch $C_h v := \sum_{i=1}^{m_\E} (P_i v)(v_i) \varphi_i$ mit
    der orthogonalen $L^2$-Projektion $P_i\colon L^2(w_i) \to \PP_0(w_i)$ auf Konstanten.
\end{Def}

\begin{Satz}{Clément-Approximationsfehler}
    Seien $\T_h$ eine zul. Triang. und $\sigma > 0$ mit
    $\forall_{T \in \T_h}\; \sigma_T \le \sigma$.\\
    Dann gilt $\exists_{C = C(d, \sigma) > 0} \forall_{T \in \T_h}
    \forall_{\text{$S \subset \partial T$ Seitensimplex}} \forall_{v \in H^1(\Omega)}
    \forall_{m = 0, 1}$\\
    $\norm{v - C_h v}_{H^m(T)} \le Ch_T^{1-m} \norm{v}_{H^1(w_T)},\;
    \norm{v - C_h v}_{L^2(S)} \le Ch_T^{1/2} \norm{v}_{H^1(w_T)}$.
\end{Satz}

\linie

\begin{Bem}
    Die FEM-Approximation $u_h \in V_h$ erzeugt ein \begriff{Residuum},
    wenn man $u_h$ in die starke Form der PDE einsetzt.
    Daraus kann man A-posteriori-Fehlerschätzer definieren, die
    bis auf Konstanten obere und untere Schranken für den Fehler darstellen.
    Im Folgenden wird nur das Poisson-Problem betrachtet.
\end{Bem}

\begin{Def}{Residuum}
    Seien $\T_h$ eine zul. Triangulierung und
    $u_h \in V_h := \PP_{k,0}(\T_h) \le H^1_0(\Omega)$ die
    FEM-Lösung der Poisson-Gleichung.
    Dann ist das \begriff{elementbasierte Residuum} für $T \in \T_h$ definiert als
    $R_T = R_T(u_h) := \Delta u_h + f|_T$
    und das \begriff{kantenbasierte Residuum der Ableitungssprünge}
    ist für $S \in \S_0$  mit der \begriff{Menge der inneren Kanten}
    $\S_0 := \{\text{$S$ Seitensimplex} \;|\;
    \exists_{T \in \T_h}\; S \subset \partial T,\; S \not\subset \partial\Omega\}$
    definiert als
    $R_S = R_S(u_h) := [\frac{\partial u_h}{\partial n}] :=
    \frac{\partial u_h}{\partial n}|_{T_1} - \frac{\partial u_h}{\partial n}|_{T_2}$
    für $T_1, T_2 \in \T_h$ mit $T_1 \not= T_2$ und $S \subset T_1 \cap T_2$.
\end{Def}

\begin{Def}{residualer Fehlerschätzer}\\
    Für $T \in \T_h$ heißt
    $\eta_{T,R} := (h_T^2 \norm{R_T}_{L^2(T)}^2 +
    \frac{1}{2} \sum_{S \subset \partial T} h_S \norm{R_S}_{L^2(S)}^2)^{1/2}$
    \begriff{lokaler Fehlerschätzer} und\\
    $\eta_R := (\sum_{T \in \T_h} \eta_{T,R}^2)^{1/2}
    = (\sum_{T \in \T_h} h_T^2 \norm{R_T}_{L^2(T)}^2 +
    \sum_{S \in \S_0} h_S \norm{R_S}_{L^2(S)}^2)^{1/2}$
    \begriff{globaler Fehlersch.}
\end{Def}

\linie

\begin{Satz}{obere A-posterori-Fehlerschranke}
    Seien $\T_h$ eine zul. Triang. und $\sigma > 0$ mit
    $\forall_{T \in \T_h}\; \sigma_T \le \sigma$.\\
    Dann gilt $\exists_{C = C(\Omega, \sigma)}\;
    \norm{u - u_h}_{H^1(\Omega)} \le C\eta_R$.
\end{Satz}

\begin{Bem}
    Diese Schranke ist ein A-posteriori-Fehlerschätzer, d.\,h. erst nach Bestimmung der
    numerischen Lösung $u_h$ kann die Schranke (bis auf die Konstante) berechnet werden.\\
    Für $u_h \in \PP_1(\T_h)$ ist $\Delta u_h \equiv 0$, d.\,h.
    $R_T$ ist dann ohne Kenntnis von $u_h$ a priori berechenbar.\\
    Bei nicht-trivialem $f$ ist $\norm{R_T}_{L^2(T)}$ i.\,A. nicht exakt berechenbar.
    Daher wählt man in der Praxis eine Approximation $f_h \approx f$ und
    approximiert die Residuen durch
    $\widetilde{R}_T := \Delta u_h + f_h$,
    $\widetilde{\eta}_{T,R} := (h_T \norm{\widetilde{R}_T}_{L^2(T)}^2 +
    \frac{1}{2} \sum_{S \subset \partial T} h_S \norm{R_S}_{L^2(S)}^2)^{1/2}$ und
    $\widetilde{\eta}_R := (\sum_{T \in \T_h} \widetilde{\eta}_{T,R}^2)^{1/2}$.
    Durch die Dreiecksungleichung erhält man
    $\norm{\Delta u_h + f}_{L^2(T)}^2 \le
    \norm{\Delta u_h + f_h}_{L^2(T)} + \norm{f - f_h}_{L^2(T)}$
    und damit
    $\norm{u - u_h}_{H^1(\Omega)} \le C\widetilde{\eta}_R +
    C (\sum_{T \in \T_h} h_T^2 \norm{f - f_h}_{L^2(T)}^2)^{1/2}$.
\end{Bem}

\linie

\begin{Bem}
    $\eta_R$ ist bis auf eine Konstante eine obere Schranke des Fehlers und heißt daher
    \begriff{zuverlässiger Schätzer}.
    Man kann zeigen, dass $\eta_R$ und $\widetilde{\eta}_R$ auch untere Schranken des Fehlers
    sind, man spricht von einem \begriff{ef"|fizienten Schätzer}.
\end{Bem}

\begin{Satz}{untere A-posteriori-Fehlerschranke}
    Seien $\T_h$ eine zul. Triang., $\sigma > 0$ mit
    $\forall_{T \in \T_h}\; \sigma_T \le \sigma$.\\
    Definiere für einen Seitensimplex $S$ den
    \begriff{Patch der Kantennachbarn}
    $w_S := \bigcup_{T \in \T_h,\; S \subset \partial T} T$ und
    für $T \in \T_h$ analog $\widetilde{w}_T := \bigcup_{S \subset \partial T} w_S$.\\
    Dann gilt $\exists_{C = C(\Omega, \sigma) > 0}\;
    \widetilde{\eta}_{T,R} \le
    C(\norm{u - u_h}_{H^1(\widetilde{w}_T)}^2 +
    \sum_{T' \subset w_T} h_{T'}^2 \norm{f - f_h}_{L^2(T')}^2)^{1/2}$ und\\
    $\widetilde{\eta}_R \le C(\norm{u - u_h}_{H^1(\Omega)}^2 +
    \sum_{T \in \T_h} h_T^2 \norm{f - f_T}_{L^2(T)}^2)^{1/2}$.
\end{Satz}

\pagebreak
