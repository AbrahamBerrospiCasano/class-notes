\chapter{%
    Modellierung mit PDEs%
}

\section{%
    Grundlagen, Definitionen und Notationen%
}

\subsection{%
    Partielle Ableitungen%
}

\begin{Def}{Multiindex}
    $\beta = (\beta_1, \dotsc, \beta_d) \in \natural_0^d$ heißt \begriff{Multiindex} der
    \begriff{Ordnung} $k = |\beta| := \sum_{i=1}^d \beta_i$.
\end{Def}

\begin{Def}{partielle Ableitung}
    Seien $u\colon \real^d \to \real$ genügend oft dif"|ferenzierbar und $\beta$ ein Multiindex.\\
    Dann heißt $\partial^\beta u := (\frac{\partial}{\partial x_1})^{\beta_1} \dotsm
    (\frac{\partial}{\partial x_d})^{\beta_d} u$ \begriff{partielle Ableitung} von $u$
    zum Index $\beta$.
\end{Def}

\begin{Def}{Vektor aller part. Ableitungen}
    Sei $\BB_k := \{\beta \in \natural_0^d \;|\; |\beta| = k\}$ die Menge aller Multiindizes
    der Ordnung $k$.
    Dann heißt $\D^k u := (\partial^\beta u)_{\beta \in \BB_k}$ der
    \begriff{Vektor aller partiellen Ableitungen} der Ordnung $k$
    (mit beliebiger Reihenfolge).
\end{Def}

\begin{Def}{Räume stetig dif"|fb. Funktionen}
    Seien $k \in \natural_0$ sowie $\Omega \subset \real^d$ of"|fen und beschränkt.\\
    Dann ist $\C^k(\overline{\Omega}, \real^n)$ der
    \begriff{Raum aller $k$-mal stetig dif"|ferenzierbarer Funktionen},
    deren $k$-te Ableitungen stetig auf $\overline{\Omega}$ fortsetzbar sind.
    Für $n = 1$ schreibt man auch $\C^k(\overline{\Omega}) := \C^k(\overline{\Omega}, \real)$.\\
    Auf $\C^0(\overline{\Omega})$ definiert man die \begriff{Supremumsnorm}
    $\norm{u}_\infty := \sup_{x \in \overline{\Omega}} |u(x)|$ mit
    $u \in \C^0(\overline{\Omega})$.\\
    Auf $\C^k(\overline{\Omega})$ mit $k \ge 1$ definiert man die Norm
    $\norm{u}_{\C^k(\overline{\Omega})} := \sum_{|\beta| \le k} \norm{\partial^\beta u}_\infty$ mit
    $u \in \C^k(\overline{\Omega})$.
\end{Def}

\begin{Bem}
    \begin{itemize}
        \item
        $\C^k(\overline{\Omega})$ ist ein Banachraum.

        \item
        Für $u \in \C^k(\overline{\Omega})$ und $\ell \in \{0, \dotsc, k\}$ ist
        $\partial^\beta u \in \C^{k - |\beta|}(\overline{\Omega})$ und
        $\D^\ell u \in (\C^{k - \ell}(\overline{\Omega}))^{|\BB_\ell|}$.

        \item
        Die Reihenfolge der Einträge von $\D^1 u$ für $u \in \C^1(\overline{\Omega})$ wird
        vereinbart durch\\
        $\D^1 u := \nabla u =
        (\frac{\partial}{\partial x_1} u, \dotsc, \frac{\partial}{\partial x_d} u)^\tp$.

        \item
        Später werden auch Räume $\C^k(\Omega)$ für $\Omega$ of"|fen, unbeschränkt und
        $k = \infty$ erlaubt sein.
        Statt einer Norm kann man dann eine Metrik (\begriff{\name{Fréchet}-Metrik}) definieren.
        Bzgl. dieser ist $\C^k(\Omega)$ ebenfalls vollständig.
    \end{itemize}
\end{Bem}

\subsection{%
    \name{Hölder}räume%
}

\begin{Def}{\name{Hölder}räume}
    Seien $k \in \natural_0$, $\alpha \in [0, 1]$, $\Omega \subset \real^d$ of"|fen
    und beschränkt sowie $u \in \C^0(\overline{\Omega})$.\\
    Dann heißt $\hol_\alpha(u, \overline{\Omega}) := \sup_{x, y \in \overline{\Omega}, x \not= y}
    \frac{|u(x) - u(y)|}{\norm{x - y}^\alpha}$ \begriff{\name{Hölder}-Konstante} von $u$
    bzgl. $\alpha$ und\\
    $\C^{k,\alpha}(\overline{\Omega}) := \{u \in \C^k(\overline{\Omega}) \;|\;
    \forall_{|\beta| = k}\; \hol_\alpha(\partial^\beta u, \overline{\Omega}) < \infty\}$
    \begriff{\name{Hölder}raum}.\\
    $\C^{k,\alpha}(\overline{\Omega})$ enthält die \begriff{\name{hölder}stetigen} Funktionen
    (jeweils mit \begriff{Exponent} $\alpha$).\\
    Für $k = 0$ und $\alpha = 1$ spricht man von \begriff{\name{Lipschitz}-stetigen} Funktionen
    mit \begriff{\name{Lipschitz}-Kons"-tante}
    $L = \Lip(u, \overline{\Omega}) := \hol_1(u, \overline{\Omega})$.
\end{Def}

\begin{Satz}{\name{Hölder}räume vollständig}\\
    $\C^{k,\alpha}(\overline{\Omega})$ ist mit der Norm
    $\norm{u}_{\C^{k,\alpha}(\overline{\Omega})} :=
    \norm{u}_{\C^k(\overline{\Omega})} +
    \sum_{|\beta|=k} \hol_\alpha(\partial^\beta u, \overline{\Omega})$
    ein Banachraum.
\end{Satz}

\begin{Satz}{Schachtelung von \name{Hölder}räumen}\\
    Für $k \in \natural_0$ und $0 \le \widehat{\alpha} \le \alpha \le 1$ gilt
    $\C^k(\overline{\Omega}) \supset \C^{k,\widehat{\alpha}}(\overline{\Omega}) \supset
    \C^{k,\alpha}(\overline{\Omega}) \supset \C^{k+1}(\overline{\Omega})$.
\end{Satz}

\begin{Bsp}
    Seien $\Omega := (0, 1)$ und $u(x) := \sqrt{x}$.
    Dann ist $u \in \C^0(\overline{\Omega}) \setminus \C^1(\overline{\Omega})$.\\
    Es gilt $u \in \C^{0,\alpha}(\overline{\Omega}) \iff \alpha \le \frac{1}{2}$
    (d.\,h. $u$ ist insbesondere nicht Lipschitz-stetig).\\
    Die Richtung "`$\Rightarrow$"' gilt, weil\\
    $\hol_\alpha(u, \overline{\Omega})
    = \sup_{x \not= y} \frac{|\sqrt{x} - \sqrt{y}|}{|x - y|^\alpha}
    \ge \sup_{y \not= 0} \frac{|\sqrt{0} - \sqrt{y}|}{|0 - y|^\alpha} =
    \sup_{y \not = 0} y^{1/2-\alpha} = \infty$ für $\alpha > \frac{1}{2}$.
\end{Bsp}

\pagebreak

\subsection{%
    \texorpdfstring{$L^p$}{Lp}-Räume%
}

\begin{Def}{$\widetilde{L}^p$-Räume}
    Für $p \in [1, \infty)$ heißt
    $\widetilde{L}^p(\Omega) :=
    \{\text{$u\colon \Omega \to \real$ Lebesgue-messb.} \;|\; \int_\Omega |u|^p \dx < \infty\}$
    \begriff{$\widetilde{L}^p$-Raum}
    mit Seminorm $\norm{u}_p := (\int_\Omega |u|^p \dx)^{1/p}$ für
    $u \in \widetilde{L}^p(\Omega)$.\\
    Für $p = \infty$ ist $L^\infty(\Omega) :=
    \{\text{$u\colon \Omega \to \real$ Lebesgue-messb.} \;|\;
    \esssup_{x \in \Omega} |u(x)| < \infty\}$
    mit Norm $\norm{u}_\infty := \esssup_{x \in \Omega} |u(x)|$.
\end{Def}

\begin{Bem}
    \begin{itemize}
        \item
        $\norm{\cdot}_p$ ist i.\,A. keine Norm auf $\widetilde{L}^p(\Omega)$ für
        $p \in [1, \infty)$
        ($\exists_{u \in \widetilde{L}^p(\Omega)}$ mit $u \not= 0$, aber $\norm{u}_p = 0$).

        \item
        Für $u \in \C^0(\overline{\Omega})$ ist $u \in \widetilde{L}^\infty(\overline{\Omega})$ und
        beide Definitionen von $\norm{\cdot}_\infty$ stimmen überein.
    \end{itemize}
\end{Bem}

\linie

\begin{Def}{$L^p$-Räume}
    Definiere eine Äquivalenzrelation $\sim$ auf $\widetilde{L}^p(\Omega)$ durch
    $u \sim v$, falls\\
    $\exists_{\text{$N \subset \Omega$ Nullmenge}}
    \forall_{x \in \Omega \setminus N}\; u(x) = v(x)$.
    Dann heißt $L^p(\Omega) := \widetilde{L}^p(\Omega)/\!\sim$ \begriff{$L^p$-Raum}.\\
    $\norm{\cdot}_p$ ist auf $L^p(\Omega)$ erweiterbar (da konstant auf Äquivalenzklassen).
\end{Def}

\begin{Bem}
    \begin{itemize}
        \item
        $L^p(\Omega)$ ist ein Banachraum.

        \item
        Die Elemente von $L^p(\Omega)$ sind eigentlich Äquivalenzklassen von Funktionen.
        Trotzdem identifiziert man diese in der Praxis oft mit Repräsentanten und nennt
        $L^p(\Omega)$ einen "`Funktionenraum"'.
        Man sollte dabei immer bedenken, ob die definierten Operationen wohldefiniert sind
        (z.\,B. sei $T, S\colon L^1(\Omega) \to \real$,
        dann ist $T(u) := \int_\Omega u(x) \dx$ wohldefiniert,
        aber $S(u) := u(y)$ für ein festes $y \in \Omega$ nicht).

        \item
        $\innerproduct{u, v}_{L^2(\Omega)} := \int_\Omega uv \dx$ ist ein Skalarprodukt auf
        $L^2(\Omega)$ mit induzierter Norm\\
        $\norm{u}_{L^2(\Omega)} = \sqrt{\langle u, u \rangle_{L^2(\Omega)}}$,
        d.\,h. $L^2(\Omega)$ ist ein Hilbertraum.

        \item
        Ist $V$ ein normierter Raum, dann ist der Dualraum
        $V' := \{\varphi\colon V \to \real \;|\; \text{$\varphi$ linear, stetig}\}$
        mit der Norm
        $\norm{\varphi}_{V'} := \sup_{u \in V \setminus \{0\}} \frac{|\varphi(u)|}{\norm{u}_V}$
        ein Banachraum.

        \item
        Für $p, q \in (1, \infty)$ mit $\frac{1}{p} + \frac{1}{q} = 1$ ist
        $L^q(\Omega) \cong (L^p(\Omega))'$
        (z.\,B. $L^2(\Omega) \cong (L^2(\Omega))'$).
    \end{itemize}
\end{Bem}

\linie

\begin{Satz}{\name{Young}sche Ungleichung}
    Für $a, b \ge 0$ und $p, q \in (1, \infty)$ mit $\frac{1}{p} + \frac{1}{q} = 1$ gilt
    $ab \le \frac{1}{p} a^p + \frac{1}{q} b^q$.\\
    Ist zusätzlich $\varepsilon > 0$, so gilt
    $ab \le \frac{\varepsilon}{2} a^2 + \frac{1}{2\varepsilon} b^2$.
\end{Satz}

\begin{Bem}
    Die Young-Ungleichung wird zur Trennung von Produkten verwendet.
\end{Bem}

\begin{Satz}{\name{Hölder}-Ungleichung}\\
    Für $p, q \in [1, \infty]$ mit $\frac{1}{p} + \frac{1}{q} = 1$ sowie
    $u \in L^p(\Omega)$ und $v \in L^q(\Omega)$ gilt $\norm{uv}_1 \le \norm{u}_p \norm{v}_q$.
\end{Satz}

\begin{Bem}
    Insbesondere ist $uv \in L^1(\Omega)$ und für $p = q = 2$ folgt Cauchy-Schwarz
    für $L^2(\Omega)$.
\end{Bem}

\subsection{%
    Fundamentalsatz der Variationsrechnung%
}

\begin{Def}{lokal intb. Fkt.en}
    $L^1_\loc(\Omega) := \{\text{$u\colon \Omega \to \real$ L.-messbar} \;|\;
    \forall_{\text{$K \subset \Omega$ kpkt.}}\; \int_K |u|\dx < \infty\}$
    ist der \begriff{Raum aller lokal integrierbaren Funktionen}.
\end{Def}

\begin{Bsp}
    Für $u\colon \real \to \real$, $u(x) :\equiv 1$, gilt
    $u \in L^1_\loc(\real) \setminus L^1(\real)$.
\end{Bsp}

\linie

\begin{Def}{Fkt.en mit kpkt. Träger}
    Seien $\Omega \subset \real^d$ of"|fen (evtl. unbeschr.) und
    $m \in \natural_0 \cup \{\infty\}$.\\
    Dann ist
    $\C^m_0(\Omega) := \{u \in \C^m(\Omega) \;|\; \text{$\supp(u) \subset \Omega$ kpkt.}\}$
    der \begriff{Raum aller Fkt.en mit kpkt. Träger}.
\end{Def}

\begin{Satz}{Fundamentalsatz der Variationsrechnung}
    Seien $\Omega \subset \real^d$ of"|fen und $u \in L^1_\loc(\Omega)$.\\
    Dann gilt $\forall_{v \in \C^\infty_0(\Omega)}\; \int_\Omega uv \dx = 0$ genau dann, wenn
    $u = 0$ fast überall.
\end{Satz}

\pagebreak

\subsection{%
    Dif"|ferentialoperatoren%
}

\begin{Def}{Gradient}
    Für $u \in \C^1(\Omega)$ und $x \in \Omega$ ist
    $\grad u(x) := \nabla u(x) = (\partial_{x_1} u(x), \dotsc, \partial_{x_d} u(x))^\tp$
    der \begriff{Gradient} von $u$ (wobei $\partial_{x_i} := \frac{\partial}{\partial x_i}$).
\end{Def}

\begin{Def}{Divergenz}
    Für ein Vektorfeld $v = (v_i)_{i=1}^d \in \C^1(\Omega, \real^d)$ ist\\
    $\div v(x) := \nabla \cdot v(x) = \sum_{i=1}^d \partial_{x_i} v_i(x)$
    die \begriff{Divergenz} von $v$.
\end{Def}

\begin{Def}{Rotation}
    Für ein Vektorfeld $v \in \C^1(\Omega, \real^3)$ mit $\Omega \subset \real^3$ ist\\
    $\rot v(x) := \nabla \times v(x)$ die \begriff{Rotation} von $v$.
\end{Def}

\begin{Def}{\name{Laplace}-Operator}
    Für $u \in \C^2(\Omega)$ ist der \begriff{\name{Laplace}-Operator} definiert durch\\
    $\Delta u(x) := \nabla \cdot (\nabla u(x)) = \div(\grad(u)) =
    \sum_{i=1}^d \partial_{x_i}^2 u(x)$
    (wobei $\partial_{x_i}^2 := \frac{\partial^2}{\partial x_i^2}$).
\end{Def}

\subsection{%
    Satz von \name{Gauß}%
}

\begin{Def}{\name{Lipschitz}-Gebiet}
    Sei $\Omega \subset \real^d$ of"|fen und beschränkt.\\
    Dann heißt $\Omega$ \begriff{\name{Lipschitz}-Gebiet}, falls endlich viele
    of"|fene Mengen $U_1, \dotsc, U_n \subset \real^d$ existieren, sodass
    $\bigcup_{i=1}^n U_i \supset \partial\Omega$ gilt und
    sich $\partial\Omega \cap U_i$ in geeigneter Richtung als Graph einer Lipschitz-stetigen
    Funktion schreiben lässt, sodass $\Omega$ komplett auf einer Seite des Graphen liegt.
\end{Def}

\begin{Satz}{Satz von \name{Gauß} für \name{Lipschitz}-Gebiete}\\
    Seien $\Omega \subset \real^d$ ein L.-Gebiet und
    $v \in \C^1(\Omega, \real^d) \cap \C^0(\overline{\Omega}, \real^d)$ ein Vektorfeld mit
    $\div(v) \in L^1(\Omega)$.\\
    Dann gilt $\int_\Omega \div v \dx = \int_{\partial\Omega} v \cdot n \dsigma(x)$ mit
    $n$ der äußeren Einheitsnormalen an $\partial\Omega$.
\end{Satz}

\begin{Satz}{partielle Integration}
    Für $u \in \C^1(\overline{\Omega})$ und $v \in \C^1(\overline{\Omega}, \real^d)$ gilt\\
    $\int_\Omega \nabla u \cdot v \dx = -\int_\Omega u \div v \dx +
    \int_{\partial\Omega} uv \cdot n \dsigma(x)$.
\end{Satz}

\subsection{%
    Skalare PDEs%
}

\begin{Def}{skalare PDE}
    Seien $k \in \natural$ und
    $F\colon \real^{|\BB_k|} \times \real^{|\BB_{k-1}|} \times \dotsb
    \times \real \times \Omega \to \real$
    gegeben.\\
    Dann heißt $F(\D^k u, \D^{k-1} u, \dotsc, u, x) = 0$ für $x \in \Omega$
    \begriff{skalare PDE} der \begriff{Ordnung} $k$.
\end{Def}

\begin{Bem}
    Es gibt auch Systeme von PDEs, die hier nicht weiter betrachtet werden.
\end{Bem}

\begin{Def}{klassische Lösung}
    Sei eine skalare PDE gegeben.\\
    Eine Funktion $u \in \C^k(\Omega)$ heißt \begriff{klassische Lösung}, falls
    $\forall_{x \in \Omega}\; F(\D^k u, \D^{k-1} u, \dotsc, u, x) = 0$.
\end{Def}

\begin{Bem}
    Alle Notationen für $\Omega \subset \real^d$ mit $x = (x_1, \dotsc, x_d) \in \Omega$ werden
    auf \begriff{Ort-Zeit-Gebiete} $\Omega_T := \Omega \times (0, T) \subset \real^d \times \real$
    mit $T \in (0, \infty]$, $(x, t) \in \Omega_T$, $\partial_t := \frac{\partial}{\partial t}$
    und $\partial_t^2 := \frac{\partial^2}{\partial t^2}$ übertragen.\\
    In diesem Fall wird für $u \in \C^1(\Omega_T)$ festgelegt, dass
    $\nabla u := \nabla_x u = (\partial_{x_1} u, \dotsc, \partial_{x_d} u)$ und\\
    $\Delta u := \Delta_x u = \sum_{i=1}^k \partial_{x_i}^2 u$.
\end{Bem}

\pagebreak

\section{%
    Modellierung%
}

\begin{Bem}
    Unter \begriff{Modellierung} versteht man die Herleitung eines mathematischen Modells für einen
    realen Prozess.
    Es gibt verschiedene Modellierungsansätze, die zu PDEs führen,
    u.\,a. Erhaltungsprinzip, Variationsprinzip und Mikro-Makro-Skalenübergang.
\end{Bem}

\subsection{%
    Erhaltungsprinzip%
}

\begin{Bem}
    Das \begriff{Erhaltungsprinzip} wird wie folgt motiviert.
    Für eine Zustandsgröße (Masse, Impuls, Energie) gilt, dass die Änderung der Zustandsgröße
    in einem beliebigen Volumen $V$ nur durch Transport über den Rand $\partial V$ des Volumens
    geschehen kann (wenn keine Quellen und Senken vorhanden sind).
\end{Bem}

\linie

\begin{Bem}
    Im Folgenden wird die allgemeine Transport-Reaktionsgleichung hergeleitet.

    Seien $\Omega \subset \real^d$ und $\Omega_T := \Omega \times (0, T)$.
    Gesucht ist ein Modell für die Konzentration $u(x, t)$ eines Stof"|fes in $\Omega$
    (z.\,B. Tinte in Wasser, Ruß in Luft) unter den folgenden Annahmen:
    \begin{itemize}
        \item
        Der Stoff wird nur durch Transport im Raum verteilt.

        \item
        Der Stoff kann abgebaut oder erzeugt werden
        (Bsp. Tintenkiller oder Schornstein).
    \end{itemize}
    Dazu definiert man
    \begin{itemize}
        \item
        den Fluss $F(x, t) \in \real^d$ durch $x$ zur Zeit $t$ und

        \item
        den Konzentrationsgewinn/-verlust $G(x, t) \in \real$ in $x$ zur Zeit $t$.
    \end{itemize}
    Zur Herleitung der PDE stellt man eine Massenbilanz in einem Kontrollvolumen $V \subset \Omega$
    im Zeitintervall $[t, t + \Delta t]$ für $\Delta t > 0$ beliebig auf:
    Die Masse zur Zeit $t + \Delta t$ ist gleich der Masse zur Zeit $t$ minus dem Ausfluss aus $V$
    plus dem Konzentrationsgewinn durch Reaktion, d.\,h.
    $\int_V u(x, t + \Delta t) \dx = \int_V u(x, t) \dx -
    \int_t^{t + \Delta t} \int_{\partial V} F(x,s) \cdot n\dsigma(x) \ds +
    \int_t^{t + \Delta t} \int_V G(x, s) \dx \ds$\\
    $\iff \int_V \frac{u(x, t + \Delta t) - u(x, t)}{\Delta t} \dx =
    -\frac{1}{\Delta t} \int_t^{t + \Delta t} \int_{\partial V} F(x,s) \cdot n\dsigma(x) \ds +
    \frac{1}{\Delta t} \int_t^{t + \Delta t} \int_V G(x, s) \dx \ds$.\\
    Für $\Delta t \to 0$ erhält man
    $\int_V \partial_t u(x, t) \dx = -\int_{\partial V} F(x, t) \cdot n \dsigma(x) +
    \int_V G(x, t) \dx$ und nach dem Satz von Gauß somit
    $\int_V \partial_t u(x, t) \dx = -\int_V \div_x F(x, t) \dx + \int_V G(x, t) \dx$.
    Weil $V$ beliebig war, kann man $V$ auf einen Punkt $x$ "`zusammenziehen"' und bekommt die\\
    \begriff{Transport-Reaktionsgleichung}
    $\partial_t u + \div F = G$ in $\Omega_T$.
\end{Bem}

\linie

\begin{Bsp}
    Ohne Transport (d.\,h. $F :\equiv 0$), aber $u$-abhängiger Reaktion $G(t, x, u)$ bekommt man
    die in $x \in \Omega$ parametrisierte gewöhnliche DGL
    $\partial_t u(x, t) = G(x, t, u(x, t))$ für $t \in (0, T)$.
\end{Bsp}

\linie

\begin{Bsp}
    Sei $v \in \C^1(\Omega_T, \real^d)$ ein Geschwindigkeitsfeld.
    Mit $F(x, t) := v(x, t) \cdot u(x, t)$ und\\
    $G(x, t) := 0$ bekommt man die
    \begriff{Advektionsgleichung} $\partial_t u + \div(vu) = 0$
    (lineare PDE).
\end{Bsp}

\linie

\begin{Bsp}
    Wenn man Autos in einer Einbahnstraße modellieren will,
    dann setzt man $d = 1$.
    $\Omega := \real$ entspricht der Straße
    und $u(x, t) \in \real$ der Fahrzeugdichte (Anzahl pro Strecke).\\
    Eine $u$-abhängige Geschwindigkeit ist realistisch (z.\,B. $v(u) = v_{\max}(1 - u)$).\\
    Mit $F(u) := v(u) \cdot u$ (d.\,h. $F(x, t) = v(u(x, t)) \cdot u(x, t)$)
    und $G(x, t) := 0$ erhält man die \begriff{Konvektionsgleichung}
    $\partial_t u + \partial_x F(u) = 0$ in $\Omega_T$.
\end{Bsp}

\linie
\pagebreak

\begin{Bsp}
    Sei $a \in \C^1(\Omega)$ (\begriff{Dif"|fusionskoef"|fizient}).
    Will man Transport durch Dif"|fusion modellieren, so benutzt man
    $G :\equiv 0$ und $F(x, t) := -a(x) \nabla u(x, t)$ (\begriff{\name{Fick}sches Gesetz}).
    Die Motivation ist, dass starke Gradienten ausgeglichen werden.
    Damit erhält man die \begriff{allg. Dif"|fusionsgleichung}
    $\partial_t u - \div(a \nabla u) = 0$.

    Ist $a \in \C^1(\Omega, \real^{d \times d})$ matrix-/tensorwertig, so heißt $a$
    \begriff{Dif"|fusionstensor} (sinnvoll, wenn die Dif"|fusion wie in Faserstrukturen
    richtungsabhängig unterschiedlich verläuft).
    Für $a(x) \equiv 1$ konstant ergibt sich die \begriff{Dif"|fusionsgleichung}
    $\partial_t u - \Delta u = 0$.

    Ist $u(x, t)$ eine Temperatur, so heißt diese Gleichung
    \begriff{instationäre Wärmeleitungsgleichung},
    $F$ \begriff{Wärmefluss} und $a$ \begriff{Wärmeleitkoef"|fizient}.
\end{Bsp}

\linie

\begin{Bsp}
    Falls die Lösung $u(x, t)$ der instationären Wärmeleitungsgleichung in einen stationären
    Zustand übergeht, d.\,h. $\overline{u} \in \C^2(\overline{\Omega})$ existiert mit
    $u(\cdot, t) \to \overline{u}$ gleichmäßig, so erfüllt $\overline{u}$ die
    \begriff{\name{Laplace}-Gleichung} $-\Delta \overline{u} = 0$ in $\Omega$.

    Falls in der Wärmeleitungsgleichung ein Quellterm $G(x, t) := f(x)$ enthalten ist
    (Ofen, Kühlschrank), so führt dies asymptotisch zur
    \begriff{\name{Poisson}-Gleichung} $-\Delta \overline{u} = f$ in $\Omega$.
\end{Bsp}

\linie

\begin{Bem}
    Ohne weitere Bedingungen sind Lösungen von PDEs i.\,A. nicht eindeutig.
    Für die Transport-Reaktionsgleichung fordert man häufig:
    \begin{itemize}
        \item
        \emph{Anfangsbedingungen}:
        $u_0(\cdot, 0) = u_0$ für ein gegebenes $u_0\colon \Omega \to \real$\\
        (wie bei gewöhnlichen DGLs, da sonst Lsg. mehrdeutig)

        \item
        \emph{Randbedingungen für $\Omega \subsetneqq \real^d$}:
        \begin{itemize}
            \item
            \begriff{\name{Dirichlet}-Randbedingungen}:
            $u(x, t) = g(x, t)$ auf $\partial\Omega \times (0, T)$\\
            (z.\,B. bei Wärmeleitung Kühlung/Heizung durch vorgeg. Temp. am Rand)

            \item
            \begriff{\name{Neumann}-Randbedingungen}:
            $F(x, t, u) \cdot n(x) = g(x, t)$ auf $\partial\Omega \times (0, T)$\\
            (Vorgeben des Flusses, bei Wärmeleitung isolierende, No-Flow-RBen
            $-(a \nabla u) \cdot n = 0$)

            \item
            \begriff{weitere Mischformen}:
            auf Teilen des Randes Dirichlet-, auf Teilen Neumann-RBen

            \item
            \begriff{\name{Robin}sche Randbedingungen}:
            $F(x, t, u) \cdot n(x) = g_0(x, t) + g_1(x, t) \cdot u$

            \item
            \begriff{Inflow-Randbedingungen}:
            RBen auf Einflussrand bei reiner Konvektion (ohne Dif"|fusion)
        \end{itemize}
    \end{itemize}
\end{Bem}

\pagebreak

\subsection{%
    Variationsprinzip%
}

\begin{Bem}
    Die Motivation des \begriff{Variationsprinzips} ist z.\,B. bei der Energieminimierung, dass
    ein physikalisches System immer in den Zustand minimaler Energie strebt.
\end{Bem}

\begin{Def}{Variationsproblem}
    Seien $\Omega \subset \real^d$ of"|fen und beschränkt,
    $\F \subset \C^1(\Omega)$ die \begriff{Menge zulässiger Funktionen} und
    $L(p, z, x) \in \C^2(\real^d \times \real \times \Omega)$ die
    \begriff{\name{Lagrange}-Funktion}.
    Das Problem, $u \in \F$ mit $\forall_{w \in \F}\; I(u) \le I(w)$ zu finden,
    wobei $I(w) := \int_\Omega L(\nabla w(x), w(x), x) \dx$, heißt \begriff{Variationsproblem}.\\
    $u$ heißt in diesem Fall \begriff{Minimierer} des Variationsproblems.
\end{Def}

\begin{Bem}
    $L$ soll zweifach stetig diffb. sein, weil das für die Euler-Lagrange-Gleichungen benötigt
    wird.
\end{Bem}

\linie

\begin{Bsp}
    Seien $\Omega \subset \real^d$ of"|fen und beschränkt,
    $A := (a_{ij})_{i,j=1}^d \in \C^2(\overline{\Omega}, \real^{d \times d})$
    symmetrisch, $c, f \in \C^2(\overline{\Omega})$ und $\F := \C^1(\Omega)$.
    Man wählt die quadratische Lagrange-Funktion\\
    $L(p, z, x) := \frac{1}{2} p^\tp A(x) p + \frac{1}{2} c(x) z^2 - z f(x)$.
    Das zugehörige Funktional\\
    $I(w) = \int_\Omega (\frac{1}{2} \nabla w^\tp A \nabla w + \frac{1}{2} cw^2 - wf) \dx$
    heißt \begriff{\name{Dirichlet}-Funktional}.
    $I(w)$ ist dabei endlich.\\
    Wenn $A(\cdot)$ positiv definit und $c$ positiv (d.\,h. $\forall_{x \in \Omega}\; c(x) > 0$)
    ist,
    dann ist $I$ \begriff{strikt konvex}, also
    $\forall_{w, w' \in \F,\; w \not= w'} \forall_{\lambda \in (0, 1)}\;
    \lambda I(w) + (1 - \lambda) I(w') > I(\lambda w + (1-\lambda) w')$.
    Nach einem Satz weiter unten folgt damit die Existenz und Eindeutigkeit des Minimierers.
\end{Bsp}

\linie

\begin{Bsp}
    Das \begriff{\name{Hamilton}-Prinzip} besagt, dass im räumlich-zeitlichen Mittel
    die Dif"|ferenz zwischen
    kinetischer und potentieller Energie extremal wird.
    Im Folgenden wird dies für die Bewegung einer eingespannten elastischen Membran angewendet
    (z.\,B. Seifenhaut in Drahtring, Trommel).

    Seien $\Omega \subset \real^d$ of"|fen und beschränkt und $\Omega_T := \Omega \times (0, T)$.
    Gegeben ist die feste Randhöhe $g(x)$ der Membran in $x \in \partial\Omega$,
    die Anfangshöhe $u_0(x)$ der Membran in $x \in \Omega$ und
    die vertikale Anfangsgeschwindigkeit $v_0(x)$ in $x \in \Omega$.
    Gesucht ist die Höhe $u(x, t)$ (und die Geschwindigkeit $\partial_t u(x, t)$)
    der Membran für $(x, t) \in \Omega_T$.

    Die kinetische Energie zur Zeit $t$ beträgt
    $E_\kin(t) := \int_\Omega \frac{1}{2} (\partial_t u)^2 \dx$ und
    die potentielle Energie zur Zeit $t$ beträgt
    $E_\pot(t) := \int_\Omega \frac{c^2}{2} \norm{\nabla_x u}^2 \dx$ mit $c > 0$.

    Die Menge der zulässigen Funktionen sei\\
    $\F := \{w \in \C^1(\overline{\Omega_T}) \;|\; w(\cdot, 0) = u_0,\;
    \partial_t w(\cdot, 0) = v_0,\;
    \forall_{t \in (0, T)}\; w(\cdot, t)|_{\partial\Omega} = g\}$.\\
    $\F$ ist als af"|fin-linearer Unterraum von $\C^1(\overline{\Omega_T})$ konvex
    (wählt man $\widehat{w} \in \F$ beliebig, dann ist
    $\F_0 := \F - \widehat{w} = \{w \in \C^1(\overline{\Omega_T}) \;|\;
    w(\cdot, 0) = 0,\; \partial_t w(\cdot, 0) = 0,\;
    \forall_{t \in (0, T)}\; w(\cdot, t)|_{\partial\Omega} = 0\}$
    linearer UR).

    Die Lagrange-Funktion ist nach dem Hamilton-Prinzip zu wählen als\\
    $L(p, z, x) := \frac{c^2}{2} \sum_{i=1}^d |p_i|^2 - \frac{1}{2} |p_{d+1}|^2$,
    denn damit erhält man das Variationsfunktional\\
    $I(w) = \int_0^T \int_\Omega
    \left(\frac{c^2}{2} \norm{\nabla_x w}^2 - \frac{1}{2} (\partial_t w)^2\right) \dx\dt
    = \int_0^T (E_\pot(t) - E_\kin(t)) \dt$.
\end{Bsp}

\linie
\pagebreak

\begin{Bsp}
    Es soll die Trennung von zwei Phasen in $\Omega$ modelliert werden (Wasser/Öl).
    Im Gleichgewicht sind beide Phasen so getrennt,
    dass die in der Trennfläche gesp. Energie minimal wird.

    Sei $\Omega \subset \real^d$ of"|fen und beschränkt.
    Gesucht ist eine \begriff{Phasenfeld-Variable} $u\colon \Omega \to \real$ mit\\
    $u(x) = -1$, falls sich im Punkt $x$ nur Phase $1$ befindet,
    $u(x) = 1$, falls sich im Punkt $x$ nur Phase $2$ befindet, und
    $u(x) \in (-1, 1)$, falls im Punkt $x$ beide Phasen anteilig vorhanden sind.

    $\partial\Omega$ sei undurchlässig, d.\,h. das Phasenverhältnis ist konstant,
    also
    $\exists_{\alpha \in [-1, 1]}\; \alpha = \frac{1}{|\Omega|} \int_\Omega u(x) \dx$.

    Es sind verschiedene Trennungen für $\Omega = [0, 1]^2$ und $\alpha = 0$ denkbar, z.\,B.
    \begin{itemize}
        \item
        keine Trennung (kontinuierlicher Verlauf von $u = 1$ zu $u = -1$),

        \item
        Trennung, aber Trennfläche groß ("`wilde"' Trennfläche, Blasen), oder

        \item
        Trennung, aber scharfe Kanten (nicht dif"|ferenzierbar).
    \end{itemize}
    Das Ziel ist ein \begriff{Energiefunktional}, dessen Minimum der vollständigen Trennung
    entspricht.

    Wählt man eine \begriff{Double-Well-Funktion} $W(z)$, die große Abweichungen von $-1$ und $1$
    bestraft, z.\,B. $W(z) := (z^2-1)^2$, und setzt $L(p, z, x) := W(z)$,
    so erhält man das Variationsfunktional $I(w) = \int_\Omega W(w(x)) \dx$.
    Allerdings ergeben sich folgende Probleme:
    \begin{itemize}
        \item
        \emph{mathematisch}:
        Für $\alpha \in (-1, 1)$ existieren keine $\C^1$-Minimierer, denn in $L^1$ ist jedes\\
        $u \in L^1(\Omega)$ mit $\frac{1}{|\Omega|} \int_\Omega u(x) \dx = \alpha$ und
        $\Bild(u) \subset \{-1, 1\}$ ein Minimierer mit $I(u) = 0$.\\
        Das Variationsproblem ist also über $\C^1$ schlecht gestellt,
        weil diese $u$ sehr unregulär sind.

        \item
        \emph{physikalisch}:
        Die Lösungen sind unnatürlich, z.\,B. sind beliebig viele Vorzeichenwechsel möglich.
        Außerdem wird die Trennfläche nicht berücksichtigt.
    \end{itemize}

    Eine Verbesserung kann eine \begriff{Regularisierung} sein, indem ein zusätzlicher Summand
    eine kleine Norm der Lösung erzwingt, z.\,B. $L(p, z, x) := W(z) + \frac{1}{2} \norm{p}^2$.
    Damit bekommt man das \begriff{\name{van-der-Waals}-Funktional}
    $I(w) = \int_\Omega \left(W(w(x)) + \frac{1}{2} \norm{\nabla w(x)}^2\right) \dx$.

    Man erhält durch Minimierung tatsächlich eine Trennung mit einer
    \begriff{dif"|fusiven Grenzschicht} (im Gegensatz zu scharfen Phasengrenzen) und die Lösung
    ist dif"|ferenzierbar.
\end{Bsp}

\linie

\begin{Bem}
    Variationsfunktionale können also
    physikalische Energieterme,
    künstliche Regularisierungsterme ($\norm{\nabla u}^2$, $\norm{u}^2$, \dots) und
    Zielwert-Funktionale umfassen.
\end{Bem}

\linie

\begin{Bem}
    Im Folgenden werden PDEs aus Variationsproblemen hergeleitet.
\end{Bem}

\begin{Satz}{Variationsprinzip, notwendige Bedingung}\\
    Seien $\F$ ein af"|fin-linearer Raum, $I(\cdot)$ stetig diffb. und $u \in \F$ ein Minimierer
    von $I(\cdot)$.\\
    Dann gilt $\frac{\d}{\d\varepsilon} I(u + \varepsilon v)|_{\varepsilon=0} = 0$
    für alle zulässigen Variationen $v \in \F - u$.
\end{Satz}

\begin{Satz}{hinreichende Bedingung}
    Seien $\F$ konvex sowie $I(\cdot)$ stetig diffb. und konvex.
    Dann gilt:
    \begin{enumerate}
        \item
        Jede Funktion $u \in \F$ mit
        $\forall_{v \in \F - u}\;
        \frac{\d}{\d\varepsilon} I(u + \varepsilon v)|_{\varepsilon=0} = 0$ ist ein Minimierer.

        \item
        Die Menge der Minimierer ist konvex in $\F$.

        \item
        Ist $I(\cdot)$ strikt konvex, so ist der Minimierer (falls existent) eindeutig.
    \end{enumerate}
\end{Satz}

\begin{Satz}{\name{Euler}-\name{Lagrange}-Gleichung}\\
    Seien $\Omega \subset \real^d$ ein Lipschitz-Gebiet,
    $\F \subset \C^1(\overline{\Omega})$ ein af"|fin-linearer Unterraum mit
    $\F + \C_0^\infty(\Omega) \subset \F$,
    $u \in \F \cap \C^2(\Omega)$ ein Minimierer von $I(\cdot)$ und
    $L(p, z, x)$ genügend glatt.\\
    Dann erfüllt $u$ die PDE
    $-\sum_{i=1}^d \partial_{x_i} ((\partial_{p_i} L) (\nabla u, u, x)) +
    \partial_z L(\nabla u, u, x) = 0$
    für $x \in \Omega$.\\
    Die PDE $-\div_x(\nabla_p L(\nabla u, u, x)) + \partial_z L(\nabla u, u, x) = 0$ heißt \begriff{\name{Euler}-\name{Lagrange}-Gleichung}.
\end{Satz}

\linie

\begin{Bsp}
    Betrachtet man die Lagrange-Funktion
    $L(p, z, x) := \frac{1}{2} p^\tp A(x) p + \frac{1}{2} c(x) z^2 - zf(x)$
    des Dirichlet-Funktionals, so erhält man wegen $\nabla_p L(p, z, x) = Ap$ und
    $\partial_z L(p, z, x) = cz - f$ die Euler-Lagrange-Gleichung
    $-\div(A\nabla u) + cu - f = 0$ für $x \in \Omega$.
    Spezialfälle sind:
    \begin{itemize}
        \item
        $A :\equiv I_d$, $c, f :\equiv 0$
        $\implies$ $-\Delta u = 0$ (Laplace-Gleichung)

        \item
        $d := 1$, $a_{11} > 0$, $c > 0$
        $\implies$ $-a_{11} u'' + cu = f$ (\begriff{\name{Sturm}-\name{Liouville}-Problem})
    \end{itemize}
\end{Bsp}

\linie

\begin{Bsp}
    Mit der Lagrange-Funktion
    $L(p, z, x) := \frac{c^2}{2} \sum_{i=1}^d |p_i|^2 - \frac{1}{2} |p_{d+1}|^2$ aus dem
    Hamilton-Prinzip erhält man $\partial_{p_i} L(p, z, x) = c^2 p_i$ für $i = 1, \dotsc, d$,
    $\partial_{p_{d+1}} L(p, z, x) = -p_{d+1}$ und\\
    $\partial_z L(p, z, x) = 0$.
    Die Euler-Lagrange-Gleichung lautet also
    $\partial_t^2 u - c^2 \Delta u = 0$ für $x \in \Omega$\\
    (\begriff{Wellengleichung}).
\end{Bsp}

\linie

\begin{Bsp}
    Die PDE für den Trennungsprozess von eben erhält man wie folgt.
    Mit $\F := \C^1(\Omega)$ ohne die Nebenbedingung
    $\frac{1}{|\Omega|} \int_\Omega u(x)dx = \alpha$
    gilt $\F + \C_0^\infty(\Omega) \subset \F$.
    In diesem Fall gilt mit der Lagrange-Funktion $L(p, z, x) := W(z) + \frac{1}{2} \norm{p}^2$,
    dass $\nabla_p L(p, z, x) = p$ und $\partial_z L(p, z, x) = W'(z)$.\\
    Man erhält also die Euler-Lagrange-Gleichung
    $-\Delta u + W'(u) = 0$ für $x \in \Omega$\\
    (\begriff{stationäre \name{Allen}-\name{Cahn}-Gleichung}).
\end{Bsp}

\subsection{%
    Mikro-Makro-Skalenübergang%
}

\begin{Bem}
    PDEs können aus stochastischen Überlegungen und einem Mikro-Makro-Skalen"-übergang resultieren.
    Im nächsten Beispiel erhält man aus einem Mikroskalenmodell (Partikel) ein
    Makroskalenmodell (Kontinuum).
\end{Bem}

\linie

\begin{Bsp}
    Im Folgenden soll die Brownsche Bewegung von Partikeln in einem Fluid modelliert werden.
    Man geht davon aus, dass der Weg eines Partikels sehr irregulär und die Bewegung
    unterschiedlicher Partikel unabhängig ist.

    Für ein eindimensionales Modell seien $\Omega_T := \real \times (0, \infty)$,
    $h > 0$ die Ortsschrittweite,
    $x_m := mh$ Gitterpunkte für $m \in \integer$,
    $k := \alpha h^2$ die Zeitschrittweite für ein $\alpha > 0$,
    $t_n := nk$ diskrete Zeitpunkte für $n \in \natural_0$ und
    $\T_h := \{(x_m, t_n) \;|\; m \in \integer,\; n \in \natural_0\}$ das Raum-Zeit-Gitter.

    Es soll ein einzelnes Partikel modelliert werden.
    Anfangs (zu $t = t_0$) befindet es sich in $x_0$.
    Danach gilt:
    Wenn es sich zur Zeit $t = t_n$ in $x_m$ befindet, dann ist es einen Zeitschritt später
    (zu $t = t_{n+1}$) entweder in $x_{m-1}$ oder in $x_{m+1}$
    (jeweils mit $50$-prozentiger Wahrscheinlichkeit).

    Sei $p_h(x_m, t_n)$ die Wahrscheinlichkeit, dass sich das Partikel zu $t = t_n$ in $x_m$
    befindet (es gilt $\sum_{m \in \integer} p_h(x_m, t_n) = 1$).
    Für $t = t_0$ gilt $p_h(x_m, t_0) = \delta_{m,0}$.
    Danach gilt\\
    $p_h(x_m, t_{n+1}) = \frac{1}{2} (p_h(x_{m-1}, t_n) + p_h(x_{m+1}, t_n))$, also
    $\frac{p_h(x_m, t_{n+1}) - p_h(x_m, t_n)}{k}$\\
    $= \frac{1}{2} \frac{p_h(x_{m-1}, t_n) - 2p_h(x_m, t_n) + p_h(x_{m+1}, t_n)}{k}
    = \frac{1}{2\alpha} \cdot \frac{1}{h}
    \left(\frac{p_h(x_{m+1}, t_n) - p_h(x_m, t_n)}{h} -
    \frac{p_h(x_m, t_n) - p_h(x_{m-1}, t_n)}{h}\right)$
    aufgrund $k = \alpha h^2$,
    man erhält also einen 2. zentralen Dif"|ferenzenquotienten.

    Sei $p_h$ geeignet auf $\Omega_T$ fortgesetzt
    (z.\,B. stückweise konstant/linear).
    Falls\\
    $p := \lim_{h \to 0} p_h \in \C^2(\Omega_T)$ existiert, dann ist es plausibel
    anzunehmen, dass\\
    $\partial_t p(x, t) = \frac{1}{2\alpha} \partial_x^2 p(x, t)$ gilt,
    d.\,h. $p$ erfüllt die Dif"|fusionsgleichung.

    Falls $\int_\real p(x, 0)\dx = 1$ gilt, dann gilt wegen der Erhaltungseigenschaft der
    Dif"|fusionsgleichung auch $\int_\real p(x, t)\dx = 1$ für $t \in (0, \infty)$, d.\,h.
    $p(\cdot, t)$ ist eine Wahrscheinlichkeitsdichte.
\end{Bsp}

\pagebreak
