\chapter{%
    B-Splines%
}

\section{%
    Rekursionsformel%
}

\textbf{Knotenfolge}:
Eine \begriff{Knotenfolge}
\begin{align*}
    \xi\colon \dotsb \le \xi_{-1} \le \xi_0 \le \xi_1 \le \dotsb
\end{align*}
ist eine endliche, unendliche oder bi-unendliche nicht-fallende Folge von reellen Zahlen.
Sie induziert eine Partition von $[\inf_k \xi_k, \sup_k \xi_k)$ in \begriff{Knotenintervalle}
$[\xi_\ell, \xi_{\ell+1})$.
Die \begriff{Vielfachheit} $\#\xi_k$ eines Knotens ist die maximale Zahl an Wiederholungen von
$\xi_k$ in der Folge $\xi$.
Man spricht analog zu Nullstellen von Funktionen von einfachen und doppelten Knoten usw.

\linie

\textbf{Rekursionsformel}:
Für eine Knotenfolge $\xi$ sind die \begriff{B-Splines} $b_{k,\xi}^n$ vom Grad $n$ definiert
durch die Rekursion
\begin{align*}
    b_{k,\xi}^n := \gamma_{k,\xi}^n b_{k,\xi}^{n-1} +
    (1 - \gamma_{k+1,\xi}^n) b_{k+1,\xi}^{n-1},\quad
    \gamma_{k,\xi}^n(x) := \frac{x - \xi_k}{\xi_{k+n} - \xi_k},
\end{align*}
beginnend mit den charakteristischen Funktionen
\begin{align*}
    b_{k,\xi}^0(x) :=
    \begin{cases}1 & \text{für } \xi_k \le x < \xi_{k+1}\\0 & \text{sonst}\end{cases}
\end{align*}
der Knotenintervalle $[\xi_k, \xi_{k+1})$, wobei Terme mit $0$ als Nenner weggelassen werden.

Jeder B-Spline $b_{k,\xi}^n$ ist durch seine Knoten $\xi_k, \dotsc, \xi_{k+n+1}$ eindeutig
bestimmt und verschwindet außerhalb des Intervalls $[\xi_k, \xi_{k+n+1})$.
Außerdem ist $b_{k,\xi}^n$ auf jedem nicht-leeren Knotenintervall $[\xi_\ell, \xi_{\ell+1})$ mit
$\ell = k, \dotsc, k + n$ ein nicht-negatives Polynom vom Grad $\le n$.\\
Falls der Grad und die Knotenfolge bei der Diskussion eines bestimmten Themas fixiert sind,
schreibt man $b_k = b_{k,\xi}^n$, um eine übermäßige Nutzung von tief- und hochgestellten Indizes
zu vermeiden.

\linie

\textbf{\name{Bernstein}-Polynome als B-Splines}:
Die B-Splines enthalten die Bernstein-Polynome als Spezialfall.
Der B-Spline $b_{k,\xi}^n$ mit den Knoten
\begin{align*}
    0 = \xi_k = \dotsb = \xi_n < \xi_{n+1} = \dotsb = \xi_{k+n+1} = 1
\end{align*}
entspricht einem Bernstein-Polynom:
\begin{align*}
    b_{k,\xi}^n(x) = b_k^n(x) = \binom{n}{k} (1 - x)^{n - k} x^k,\quad 0 \le x < 1.
\end{align*}

\linie

\textbf{erstes und letztes B-Spline-Segment}:
Für $\xi_k < \xi_{k+1}$ ist bloß der erste Summand in der Rekursion für $b_{k,\xi}^n$
auf dem Intervall $[\xi_k, \xi_{k+1})$ ungleich Null.
Daher ist auf diesem Intervall der B-Spline ein Produkt der Faktoren $\gamma_{k,\xi}^m$:
\begin{align*}
    b_{k,\xi}^n(x) = \frac{(x - \xi_k)^n}{(\xi_{k+1} - \xi_k) \dotsm (\xi_{k+n} - \xi_k)},\quad
    \xi_k \le x < \xi_{k+1}.
\end{align*}
Eine analoge Formel gilt für das am weitesten rechts liegende Intervall $[\xi_{k+n}, \xi_{k+n+1})$
des Trägers von $b_{k,\xi}^n$.
Insbesondere gilt für $\xi_{k+1} = \dotsb = \xi_{k+n}$, dass der B-Spline aus zwei Monomen
zusammengesetzt ist, die am mittleren Knoten $1$ ergeben.

\linie

\textbf{stetige Abhängigkeit vom Knotenvektor}:
Wenn $x$ im Inneren von einem der Knotenintervalle des B-Splines $b_{k,\xi}^n$ liegt und
\begin{align*}
    \eta_\ell \to \xi_\ell,\quad \ell = k, \dotsc, k + n + 1,
\end{align*}
dann gilt
\begin{align*}
    \lim_{\eta \to \xi} b_{k,\eta}^n(x) = b_{k,\xi}^n(x).
\end{align*}

\section{%
    Ableitung eines B-Splines%
}

\textbf{Ableitung eines B-Splines}:
Die Ableitung eines B-Splines vom Grad $n$ mit den Knoten $\xi_k, \dotsc,$\\
$\xi_{k+n+1}$ ist die gewichtete Dif"|ferenz zweier B-Splines vom Grad $n - 1$.
Auf jedem Knotenintervall $[\xi_\ell, \xi_{\ell+1})$ gilt
\begin{align*}
    (b_{k,\xi}^n)' = \alpha_{k,\xi}^n b_{k,\xi}^{n-1} - \alpha_{k+1,\xi}^n b_{k+1,\xi}^{n-1},\quad
    \alpha_{k,\xi}^n := \frac{n}{\xi_{k+n} - \xi_k},
\end{align*}
wobei Terme mit $0$ als Nenner weggelassen werden.

Aus der Rekursion folgt, dass $b_{k,\xi}$ an einem Knoten $\xi_\ell$
$(n - m)$-mal stetig dif"|ferenzierbar ist, falls $\xi_\ell$ unter $\xi_k, \dotsc, \xi_{k+n+1}$
die Vielfachheit $m \le n$ besitzt.
Insbesondere ist $b_{k,\xi}^n$ stetig auf $\real$, falls keiner der Knoten die Vielfachheit
$n + 1$ hat.

\linie

\textbf{Nullstellenordnungen bei B-Splines}:
Mithilfe der Ableitungsformel kann man auch das genaue Verhalten eines B-Splines an den Endpunkten
seines Trägers bestimmen.
Wenn $\xi_k$ die Vielfachheit $m$ unter den Knoten von $b_{k,\xi}^n$ besitzt, dann ist der
linke Endpunkt $\xi_k$ des B-Spline-Trägers eine Nullstelle der Ordnung $n + 1 - m$.

Dies folgt per Induktion über dem Grad:
Für den Induktionsschritt von $n - 1$ nach $n$ bemerkt man, dass die zwei B-Splines von Grad
$n - 1$ im Ausdruck von $(b_{k,\xi}^n)'$ jeweils Nullstellen von Ordnung $n - m$ und $n - (m - 1)$
besitzen.
Daher hat $(b_{k,\xi}^n)'$ in $\xi_k$ eine Nullstelle der Ordnung $n - m$.
Durch Integration erhöht sich die Ordnung um $1$.

\section{%
    Darstellung von Polynomen durch B-Splines%
}

\textbf{\name{Marsden}-Identität}:
Für eine bi-unendliche Knotenfolge $\xi$ mit $\lim_{k \to \pm\infty} \xi_k = \pm\infty$
kann jedes Polynom vom Grad $\le n$ durch eine Linearkombination von B-Splines dargestellt werden.
Genauer gilt für beliebige $y \in \real$
\begin{align*}
    (x - y)^n = \sum_{k \in \integer} \psi_{k,\xi}^n(y) b_{k,\xi}(x),\quad x \in \real,
\end{align*}
mit $\psi_{k,\xi}^n(y) := (\xi_{k+1} - y) \dotsm (\xi_{k+n} - y)$.

Durch Dif"|ferentiation der Identität nach $y$ und Auswertung in $y = 0$ erhält man explizite
Darstellungen für die Monome $x^m$.
Es gilt z.\,B.
\begin{align*}
    1 = \sum_k b_{k,\xi}^n(x),\quad
    x = \sum_k \xi_k^n b_{k,\xi}^n(x)
\end{align*}
mit $\xi_k^n := (\xi_{k+1} + \dotsb + \xi_{k+n})/n$ den \begriff{Knotenmitteln}.

\linie
\pagebreak

\textbf{Einschränkung des Parameters}:
Die Marsden-Identität ist zwar für bi-unendliche Knotenfolgen formuliert, die sich über die ganze
reelle Achse erstrecken.
Of"|fensichtlich kann man aber auch endliche Knotenfolgen betrachten.
Wenn man $x$ auf ein Knotenintervall $D_\ell = [\xi_\ell, \xi_{\ell+1})$ einschränkt,
sind nur B-Splines mit Träger, die $D_\ell$ überlappen, relevant:
\begin{align*}
    (x - y)^n = \sum_{k=\ell-n}^\ell \psi_{k,\xi}^n(y) b_{k,\xi}^n(x),\quad
    \xi_\ell \le x < \xi_{\ell+1},
\end{align*}
falls $\xi$ die involvierten Knoten $\xi_{\ell-n}, \dotsc, \xi_{\ell+n+1}$ enthält.

\linie

\textbf{Beispiel (Darstellung der Standard-Parabel durch B-Splines)}:
Die Darstellung des Monoms $x^2$ erhält man durch Koef"|fizientenvergleich in der Marsden-Identität
bei $(-y)^{n-2}$ und Division durch $\binom{n}{2}$ auf beiden Seiten:
\begin{align*}
    x^2 = \sum_k c_k b_k(x),\quad
    c_k = \frac{2}{n(n - 1)} \sum_{1 \le i < j \le n} \xi_{k+i}\xi_{k+j},
\end{align*}
da die Summe gleich dem Koef"|fizienten von $(-y)^{n-2}$ in einer Entwicklung von $\psi_{k,\xi}^n$
ist.
Mit
\begin{align*}
    \xi_k^n := \frac{1}{n} \sum_{\ell=1}^n \xi_{k+\ell},\quad
    \sigma_k^2 := \frac{1}{n - 1} \sum_{\ell=1}^n (\xi_{k+\ell} - \xi_k^n)^2
\end{align*}
den Knotenmitteln und der \begriff{geometrischen Varianz} von $n$ aufeinanderfolgenden Knoten kann
man $c_k$ kompakter schrieben, denn es gilt
\begin{align*}
    c_k = (\xi_k^n)^2 - \sigma_k^2/n.
\end{align*}

\section{%
    Splines%
}

\textbf{Spline}:
Ein \begriff{Spline} $p$ vom Grad $\le n$ mit Knotenfolge $\xi$ auf einem Parameterintervall $D$
ist eine Linearkombination der B-Splines $b_k = b_{k,\xi}^n$:
\begin{align*}
    p(x) = \sum_{k \sim D} c_k b_k(x),\quad
    x \in D.
\end{align*}
Der Summenindex läuft über alle relevanten B-Splines, d.\,h. die B-Splines, die an Punkten $x$ in
$D$ nicht verschwinden.
Die Koef"|fizienten $c_k$ sind eindeutig bestimmt, d.\,h. die B-Splines formen eine Basis des
\begriff{Spline-Raums} $S_\xi^n(D)$.

Ein \begriff{Standard-Spline-Raum} $S_\xi^n$ hat eine endliche Knotenfolge
$\xi_0, \dotsc, \xi_{m+n}$ mit Vielfachheiten $\le n$, das Parameterintervall $D = [\xi_n, \xi_m]$
und relevante B-Splines $b_0, \dotsc, b_{m-1}$.
Der Raum besteht aus allen stetigen Funktionen, die
\begin{itemize}
    \item
    auf den abgeschlossenen Knotenintervallen $[\xi_\ell, \xi_{\ell+1}]$ in $D$
    Polynome vom Grad $\le n$ und

    \item
    an Knoten im Inneren von $D$ mit Vielfachheit $\mu$
    mindestens $(n - \mu)$-mal stetig dif"|ferenzierbar sind.
\end{itemize}
$S_\xi^n$ hat die $\real$-Dimension $m$.

\linie
\pagebreak

\textbf{Beispiel (Knotenfolgen bei kubischen Splines)}:
In Anwendungen sind kubische Splines wichtig.
Zwei häufig benutzte Knotenfolgen sind:
\begin{itemize}
    \item
    \emph{einfache Knoten}:
    Wenn $\xi_0 < \dotsb < \xi_{m+3}$ gilt,
    dann besteht der Standard-Spline-Raum $S_\xi^3$ aus allen zweifach stetig dif"|ferenzierbaren
    Funktionen, die Polynome vom Grad $\le 3$ auf jedem abgeschlossenen Knotenintervall
    $[\xi_\ell, \xi_{\ell+1}]$ in $D = [\xi_3, \xi_m]$ sind.

    \item
    \emph{doppelte Knoten}:
    Wenn $\xi_0 = \xi_1 < \xi_2 = \xi_3 < \dotsb < \xi_m = \xi_{m+1} < \xi_{m+2} = \xi_{m+3}$ gilt,
    dann können die zweiten Ableitungen der kubischen Splines Sprünge besitzen.
    In diesem Fall sind die kubischen Splines auf jedem abgeschlossenen Knotenintervall
    $[\xi_{2k-1}, \xi_{2k}]$ eindeutig bestimmt durch die Werte und Ableitungen an den Knoten.
    Diese Daten stellen bei der Bestimmung eines Splines in $S_\xi^3$
    eine Alternative zur B-Spline-Basis dar.
\end{itemize}

\linie

\textbf{Beispiel (äußere Knoten beim Standard-Spline-Raum)}:
Für eine Knotenfolge\\
$\xi_0, \dotsc, \xi_{n+m}$ benötigt die B-Spline-Basis des
Standard-Spline-Raums $S_\xi^n$ $n$ äußere Knoten auf jeder Seite des Parameterintervalls
$D = [\xi_n, \xi_m]$.
Diese Knoten $\xi_k$ mit $k < n$ oder $k > m$ sind für die Definition von $S_\xi^n$ bezüglich
der abschnittsweisen polynomialen Struktur irrelevant.
Jedoch beeinflussen sie die B-Spline-Basis.
Es gibt zwei Standardfälle:
\begin{itemize}
    \item
    \emph{einfache äußere Knoten}:
    Mit $\Delta\xi_\ell := \xi_{\ell+1} - \xi_\ell$ definiert man
    $\xi_{n-\ell} := \xi_n - \ell \Delta\xi_n$ und $\xi_{m+\ell} := \xi_m + \ell \Delta\xi_{m-1}$
    für $\ell = 1, \dotsc, n$.
    Diese Wahl der Knoten erhält den Abstand zwischen Knoten auf dem ersten und letzten
    Knotenintervall in $D$ und maximiert die Glattheit der B-Splines.

    \item
    \emph{mehrfache äußere Knoten}:
    Man definiert  $\xi_1 := \dotsb := \xi_n$ und $\xi_m := \dotsb := \xi_{m+n-1}$
    und wählt nur $\xi_0$ und $\xi_{m+n}$ außerhalb des Intervalls $D$.
    Wegen der maximal möglichen Vielfachheit sind nur die B-Splines $b_0$ und $b_{m-1}$
    nicht Null auf den Intervallendpunkten (diese sind dort gleich Eins).
    Daher gilt für einen Spline $p = \sum_{k=0}^{m-1} c_k b_k$, dass
    $p(\xi_n) = c_0$ und $p(\xi_m) = c_{m-1}$.
    Der Nachteil der Vielfachheit $n$ ist, dass die Ableitungen der B-Splines nicht länger
    stetig sind.
    Die Konvention der Stetigkeit von rechts führt daher zu einem asymmetrischen Verhalten
    auf den Intervallendpunkten von $D$.
\end{itemize}

\linie

\textbf{uniforme B-Splines}:
Der \begriff{uniforme B-Spline} $b^n$ hat die Knoten $0, 1, \dotsc, n + 1$.
In diesem Spezialfall vereinfachen sich die Rekursionen für Auswertung und Dif"|ferentiation:
\begin{align*}
    n b^n(x) &= xb^{n-1}(x) + (n + 1 - x)b^{n-1} (x - 1),\\
    \frac{d}{dx} b^n(x) &= b^{n-1}(x) - b^{n-1}(x - 1).
\end{align*}
Die zweite Gleichung kann als Bildung eines Mittelwerts geschrieben werden:
\begin{align*}
    b^n(x) = \int_0^1 b^{n-1}(x - y)\dy.
\end{align*}
Die B-Splines für eine beliebige uniforme Knotenfolge
\begin{align*}
    \xi = h\integer\colon \dotsc, -h, 0, h, \dotsc,
\end{align*}
sind skalierte Translate von $b^n$, d.\,h. $b_{k,h}^n(x) := b^n(x/h - k)$, $k \in \integer$.

\linie
\pagebreak

\textbf{Beispiel (Rekursion für die \name{Taylor}-Koef"|fizienten von uniformen B-Splines)}:\\
Die Rekursion für die Auswertung kann auch als Rekursion für die Taylor-Koef"|fizienten der
polynomialen Abschnitte eines uniformen B-Splines formuliert werden.
Für $x \in [k, k + 1)$ definiert man
\begin{align*}
    p_k^n(y) := \sum_{\ell=0}^n a_{k,\ell}^n y^\ell = b^n(x),\quad
    x - k = y \in [0, 1).
\end{align*}
Damit kann die Rekursion umgeschrieben werden zu
\begin{align*}
    np_k^n(y) = (k + y)p_k^{n-1}(y) + (n + 1 - k - y) p_{k-1}^{n-1}(y),\quad
    k = 0, \dotsc, n,
\end{align*}
mit $p_{-1}^{n-1} = p_n^{n-1} = 0$.
Die entsprechende Identität für die Koef"|fizienten ist
\begin{align*}
    n a_{k,\ell}^n = k a_{k,\ell}^{n-1} + a_{k,\ell-1}^{n-1} + (n + 1 - k) a_{k-1,\ell}^{n-1} -
    a_{k-1,\ell-1}^{n-1}
\end{align*}
mit $a_{k,-1}^{n-1} = a_{k,n}^{n-1} = 0$.
Die ersten paar Koef"|fizientenvektoren lauten
\begin{align*}
    a_0^1 &= \left(0, 1\right), &
    a_1^1 &= \left(1, -1\right),\\
    a_0^2 &= \left(0, 0, \frac{1}{2}\right), &
    a_1^2 &= \left(\frac{1}{2}, 1, -1\right), &
    a_2^2 &= \left(\frac{1}{2}, -1, \frac{1}{2}\right),\\
    a_0^3 &= \left(0, 0, 0, \frac{1}{6}\right), &
    a_1^3 &= \left(\frac{1}{6}, \frac{1}{2}, \frac{1}{2}, -\frac{1}{2}\right), &
    a_2^3 &= \left(\frac{2}{3}, 0, -1, \frac{1}{2}\right), &
    a_3^3 &= \left(\frac{1}{6}, -\frac{1}{2}, \frac{1}{2}, -\frac{1}{6}\right).
\end{align*}
Mithilfe der tabulierten Taylorkoef"|fizienten kann man uniforme B-Splines schneller als mit
der Rekursion auswerten.

\section{%
    Auswertung und Dif"|ferentiation%
}

\textbf{Auswertung eines Splines}:
Ein Spline $p = \sum_k c_k b_k$ vom Grad $\le n$ mit Knotenfolge $\xi$ kann in
$x \in [\xi_\ell, \xi_{\ell+1})$ ausgewertet werden, indem man
Konvexkombinationen der Koef"|fizienten der B-Splines bildet, die in $x$ nicht verschwinden.
Mit
\begin{align*}
    p_k^0 := c_k,\quad
    k = \ell - n, \dotsc, \ell,
\end{align*}
startend berechnet man sukzessive für $i = 0, \dotsc, n - 1$
\begin{align*}
    p_k^{i+1} := \gamma_{k,\xi}^{n-i} p_k^i + (1 - \gamma_{k,\xi}^{n-i}) p_{k-1}^i,\quad
    k = \ell - n + i + 1, \dotsc, \ell,
\end{align*}
mit
\begin{align*}
    \gamma_{k,\xi}^{n-i} := \frac{x - \xi_k}{\xi_{k+n-i} - \xi_k}
\end{align*}
und erhält $p(x)$ als letzten Wert $p_\ell^n$.

Das entstehende Dreiecksschema vereinfacht sich etwas, wenn $x = \xi_\ell$ gilt.
In diesem Fall gilt $p(x) = p_{\ell-\mu}^{n-\mu}$, d.\,h.
es sind nur $n - \mu$ Schritte für $c_{\ell-n}, \dotsc, c_{\ell-\mu}$ notwendig,
wenn $\xi_\ell$ Vielfachheit $\mu$ besitzt.

\linie
\pagebreak

\textbf{Dif"|ferentiation eines Splines}:
Die Ableitung eines Splines ist ein Spline mit derselben Knotenfolge.
Genauer gilt für jedes $x$ in einem of"|fenen Knotenintervall $(\xi_\ell, \xi_{\ell+1})$
mit $n + 1$ relevanten B-Splines $b_{k,\xi}^n$
\begin{align*}
    \frac{d}{dx} \left(\sum_{k=\ell-n}^\ell c_k b_{k,\xi}^n(x)\right)
    = \sum_{k=\ell-n+1}^\ell \alpha_{k,\xi}^n \nabla c_k b_{k,\xi}^{n-1}(x),\quad
    \alpha_{k,\xi}^n := \frac{n}{\xi_{k+n} - \xi_k}
\end{align*}
mit $\nabla$ dem Rückwärts-Dif"|ferenz-Operator (d.\,h. $\nabla c_k := c_k - c_{k-1}$).
Die Identität bleibt auch an den Endpunkten $\xi_\ell$ und $\xi_{\ell+1}$ des Knotenintervalls
gültig, falls die Knoten Vielfachheit $< n$ haben.

Für einen Spline $p = \sum_{k=0}^{m-1} c_k b_k$ in einem Standard-Spline-Raum $S_\xi^n$ mit
Knotenfolge\\
$\xi_0, \dotsc, \xi_{m+n}$ und Vielfachheiten $< n$ gilt
mit $d_k := \alpha_{k,\xi} \nabla c_k$
\begin{align*}
    p' = \sum_{k=1}^{m-1} d_k b_{k,\xi'}^{n-1} \in S_{\xi'}^{n-1},
\end{align*}
wobei $\xi'$ aus $\xi$ durch Weglassen des ersten und des letzten Knotens entsteht.
Dies ist mit der Dif"|ferenzenbildung $\nabla$ konsistent, weil die Anzahl der möglichen Indizes
um Eins reduziert wird.

\section{%
    Periodische Splines%
}

\textbf{periodische Splines}:
Ein Spline $p = \sum_{k \in \integer} c_k b_k \in S_\xi^n(\real)$
mit einer bi-unendlichen Knotenfolge $\xi$ ist $T$-periodisch genau dann,
wenn die Knoten $\xi_k$ und die Koef"|fizienten $c_k$ die Periodizitätsbedingungen
\begin{align*}
    \xi_{k+M} = \xi_k + T,\quad
    c_{k+M} = c_k,\quad
    k \in \integer,
\end{align*}
erfüllen (für ein $M \in \natural$).

Die periodischen Splines bilden einen Unterraum $S_{\eta,T}^n$ von $S_\xi^n(\real)$
der Dimension $M$, wobei $\eta$ eine beliebige Teilfolge von $M$ aufeinanderfolgenden Knoten aus
$\xi$ ist.

\pagebreak
