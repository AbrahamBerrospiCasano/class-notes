\chapter{%
    Spline-Kurven%
}

\section{%
    Kontrollpolygon%
}

\textbf{Spline-Kurve}:
Seien
\begin{align*}
    \tau\colon \tau_0 \le \dotsb \le \tau_n < \tau_{n+1} \le \dotsb \le \tau_{m-1} < \tau_m \le
    \dotsb \le \tau_{m+n}
\end{align*}
eine Knotenfolge mit Vielfachheiten $\le n$ und $D = [\tau_n, \tau_m]$ das
Standard-Parameterintervall.
Eine \begriff{Spline-Kurve} vom Grad $\le n$ in $\real^d$ hat eine Parametrisierung
\begin{align*}
    t \mapsto (p_1(t), \dotsb, p_d(t)) = \sum_{k=0}^{m-1} c_k b_k(t),\quad
    t \in D,
\end{align*}
wobei die Komponenten $p_\nu$ sich im Standard-Spline-Raum $S_\tau^n$ befinden.

Die Koef"|fizienten $c_k = (c_{k,1}, \dotsc, c_{k,d})$ können in einer $(m \times d)$-Matrix
zusammengefasst werden.
Sie heißen \begriff{Kontrollpunkte} und bilden das \begriff{Kontrollpolygon} $c$ von $p$.

\linie

\textbf{Beispiel (gebräuchliche Knotenfolgen)}:
Es gibt zwei häufige Möglichkeiten für Knotenfolgen, wie Knoten außerhalb von $D$ platziert werden:
\begin{itemize}
    \item
    $\tau_n$ und $\tau_m$ haben Vielfachheit $1$ und
    $\tau_0, \dotsc, \tau_{n+1}$ bzw. $\tau_{m-1}, \dotsc, \tau_{m+n}$ sind äquidistant.
    In diesem Fall hat die Kurve zwar maximale Glattheit, aber sie interpoliert nicht die
    Endpunkte.

    \item
    $\tau_1 = \dotsb = \tau_n$ und $\tau_m = \dotsb = \tau_{m+n-1}$ haben
    die maximale Vielfachheit $n$ sowie $\tau_0 < \tau_1$ und $\tau_{m+n-1} < \tau_{m+n}$.
    In diesem Fall ist zwar Endpunktinterpolation vorhanden, aber die Kurve ist i.\,A.
    am Endpunkt nicht dif"|ferenzierbar (nur einseitig).
\end{itemize}

\linie

\textbf{geschlossene Spline-Kurve}:
Eine \begriff{geschlossene Spline-Kurve} vom Grad $\le n$ in $\real^d$ hat eine Parametrisierung
\begin{align*}
    t \mapsto (p_1(t), \dotsc, p_d(t)) = \sum_{k \in \integer} c_k b_k(t),\quad
    t \in \real,
\end{align*}
wobei die Komponenten $p_\nu$ stetige $T$-periodische Splines sind, d.\,h.
$p_\nu \in S_{\tau,T}^n$ mit\\
$\tau = (\tau_0, \dotsc, \tau_{M-1})$.
Die B-Splines $b_k$ entsprechen der periodisch erweiterten Knotenfolge\\
$(\dotsc, \tau - T, \tau, \tau + T, \dotsc)$.

Wegen der Periodizitätsbedingungen ist $p$ bestimmt durch $M$ aufeinanderfolgende Kontrollpunkte
\begin{align*}
    C = \begin{pmatrix}c_0\\\vdots\\c_{M-1}\end{pmatrix},
\end{align*}
die das \begriff{geschlossene Kontrollpolygon} $c$ von $p$ bestimmen.

\linie

\textbf{Beispiel (nicht-periodische Darstellung)}:
Eine geschlossene Spline-Kurve kann auch nicht-periodisch parametrisiert werden.
Dazu fügt man links von $\tau = (\tau_0, \dotsc, \tau_{M-1})$ die $n$ Knoten
$\tau_{M-n} - T, \dotsc, \tau_{m-1} - T$ und rechts von $\tau$ die $n + 1$ Knoten
$\tau_0 + T, \dotsc, \tau_n + T$ an.
Mit den entsprechenden Kontrollpunkten $c_{M-n}, \dotsc, c_{M-1}, c_0, \dotsc, c_{M-1}$
erhält man eine Parametrisierung mit Komponenten im Standard-Spline-Raum.

\linie
\pagebreak

\textbf{rationale Parametrisierungen (NURBS)}:
Eine \begriff{nicht-uniforme rationale B-Spline-Parametrisie"-rung (NURBS)} $r = p/q$ ist der
Quotient einer Spline-Parametrisierung $t \mapsto p(t)$ mit gewichteten Kontrollpunkten
\begin{align*}
    c_k w_k \in \real^d,\quad w_k > 0
\end{align*}
und einer Spline-Funktion $t \mapsto q(t)$ mit Koef"|fizienten $w_k \in \real$.

Die Gewichte $w_k$ ermöglichen zusätzliche Gestaltungsfreiheit.
Das Erhöhen eines Gewichts "`zieht"' die Kurve zum entsprechenden Kontrollpunkt.
Man kann $r$ mit einer Spline-Kurve in homogenenen Koordinaten identifizieren,
parametrisiert durch
\begin{align*}
    t \mapsto (p(t), q(t)) = \sum_k (c_k w_k, w_k) b_k(t) \in \real^{d+1}.
\end{align*}
Diese Interpretation ist vor allem für die Implementierung von Algorithmen nützlich,
z.\,B. Knoteneinfügung, Auswertung und Dif"|ferentiation.

\linie

\textbf{Beispiel (Kreis als NURBS-Parametrisierung)}:
Kegelschnitte können durch quadratische NURBS dargestellt werden.
Beispielsweise hat der Viertelkreis im 1. Quadranten die Kontrollpunkte und Gewichte
\begin{align*}
    (c_k, w_k)\colon\quad
    (1, 0, 1), (1, 1, 1/\sqrt{2}), (0, 1, 1).
\end{align*}
Entsprechend kann ein ganzer Kreis durch eine geschlossene rationale Spline-Kurve $r$ in
Bézier-Form mit doppelten Knoten
\begin{align*}
    t_0 = 0, 0, 1, 1, 2, 2, 3, 3 = \tau_7,\quad
    \tau_{k+8} = \tau_k + 4
\end{align*}
dargestellt werden, wenn man die Kontrollpunkte und Gewichte analog zum Viertelkreis wählt.

Man kann zeigen, dass der Kreis durch eine geschlossene, rationale, quadratische
Spline-Kurve mit einfachen, uniformen Knoten nicht dargestellt werden kann.

\section{%
    Eigenschaften von Spline-Kurven%
}

\textbf{Eigenschaften von Spline-Kurven}:
Die Form einer Spline-Kurve, parametrisiert durch\\
$p = \sum_{k=0}^{m-1} c_k b_k$, $p_\nu \in S_\tau^n$,
wird qualitativ durch ihr Kontrollpolygon $c$ modelliert.
Genauer gilt:
\begin{itemize}
    \item
    $p(t)$ liegt in der konvexen Hülle von $c_{\ell-n}, \dotsc, c_\ell$, falls
    $t \in [\tau_\ell, \tau_{\ell+1})$.
\end{itemize}
Zusätzlich gilt, falls beide Endpunkte des Standard-Parameterintervalls $D = [\tau_n, \tau_m]$
Knoten mit Vielfachheit $n$ sind, dass
\begin{itemize}
    \item
    $p(\tau_n) = c_0$ und $p(\tau_m) = c_{m-1}$ sowie

    \item
    $p'(\tau_n^+) = \alpha_{1,\tau}^n (c_1 - c_0)$ und
    $p'(\tau_m^-) = \alpha_{m-1,\tau}^n (c_{m-1} - c_{m-2})$
\end{itemize}
mit $\alpha_{k,\tau}^n := n/(\tau_{k+n} - \tau_k)$.
Die letzten beiden Eigenschaften werden auch \begriff{Endpunktinterpolation} bezeichnet,
da das Kontrollpolygon tangential zur Spline-Kurve ist, was sehr nützlich für Design-Zwecke ist.

Die Parametrisierung einer Spline-Kurve ist stetig, die Ableitung kann jedoch Sprünge enthalten.
Daher werden in der Formel für $p'$ die hochgestellten Indizes $+$ und $-$ verwendet,
um rechts- bzw. linksseitige Ableitung zu bezeichnen.

\linie
\pagebreak

\textbf{Abstand zum Kontrollpolygon}:
Seien eine Spline-Kurve gegeben, die durch $p = \sum_{k=0}^{m-1} c_k b_k$, $p_\nu \in S_\tau^n$,
mit $n > 1$ parametrisiert wird,
und $c$ eine stückweise lineare Parametrisierung des Kontrollpolygons,
die die $c_k$ an den Knotenmitteln $\tau_k^n = (\tau_{k+1} + \dotsb + \tau_{k+n})/n$
interpoliert.

Dann kann der Abstand von $p$ zum Kontrollpolygon durch zweite gewichtete Dif"|ferenzen
der Kontrollpunkte abgeschätzt werden.
Genauer gilt für $t \in [\tau_\ell, \tau_{\ell+1})$
\begin{align*}
    \norm{p(t) - c(t)}_\infty \le \frac{1}{2n} \max_{\ell-n \le k \le \ell} \sigma_k^2
    \max_{\ell-n+2 \le k \le \ell} \norm{\nabla_\tau^2 c_k}_\infty,
\end{align*}
wobei
\begin{align*}
    \sigma_k^2 := \frac{1}{n - 1} \sum_{i=1}^n (\tau_{k+i} - \tau_k^n)^2
\end{align*}
und $\nabla_\tau^2 c_k$ den Kontrollpunkten der zweiten Ableitung $p''$, in expliziter Form
\begin{align*}
    \nabla_\tau^2 c_k := \frac{n - 1}{\tau_{k+n-1} - \tau_k}
    \left(\frac{c_k - c_{k-1}}{\tau_k^n - \tau_{k-1}^n} -
    \frac{c_{k-1} - c_{k-2}}{\tau_{k-1}^n - \tau_{k-2}^n}\right).
\end{align*}
Keiner der Nenner verschwindet, da die Dif"|ferenzen mindestens so groß sind wie
$\tau_{\ell+1} - \tau_\ell$.

Die lokale Abschätzung impliziert eine globale Abschätzung, indem man auf der rechten Seite
das Maximum über alle für das Parameterintervall $D = [\tau_n, \tau_m]$ der Spline-Kurve
relevanten $k$ nimmt.
In diesem Fall wird $\nabla_\tau^2 c_k$ auf Null gesetzt, wenn $\tau_k = \dotsb = \tau_{k+n-1}$.

Die Distanzabschätzung ist scharf, d.\,h. es gibt Fälle, in denen in der Ungleichung Gleichheit
gilt.
Dadurch ist die Abstandsabschätzung in den meisten Fällen (besonders bei höherem Grad)
besser möglich als mit der konvexen Hülle.
Die $\sigma_k^2$ stellen eine Art "`geometrische Varianz"' dar,
nämlich bis auf einen Faktor die Abstandsquadratsumme der relevanten Knoten zu ihrem Knotenmittel.
Das kann man sich auch im Fall $\tau_{k+1} = \dotsb = \tau_{k+n} = \tau_k^n$ verdeutlichen.
Dann gilt $\sigma_k^2 = 0$ und die rechte Seite der Abschätzung wird $0$,
weil die Spline-Kurve in diesem Fall den Punkt $c_k$ interpoliert.

\linie

\textbf{Beispiel (äquidistante Knoten)}:
Für uniforme Knoten $\tau_k = kh$ und ungeraden Grad $n = 2m + 1$
gilt $\tau_k^n = k + m + 1$ und $\sigma_k^2 =  \frac{1}{n - 1} \sum_{i=-m}^m (ih)^2$.
Dadurch erhält man für das Produkt der ersten beiden Faktoren in der Abstandsabschätzung
$\frac{1}{2n} \frac{2h^2}{n - 1} \sum_{i=1}^m i^2 = \frac{n + 1}{24} h^2$.
Daher erhält man wegen $\nabla_\tau^2 c_k = h^{-2} \Delta^2 c_{k-2}$ mit
$\Delta^2 c_{k-2} = c_k - 2c_{k-1} + c_{k-2}$
als Abschätzung für äquidistante Knoten
\begin{align*}
    \norm{p(t) - c(t)}_\infty \le \frac{n + 1}{24} \max_{\ell-n \le k \le \ell-2}
    \norm{\Delta^2 c_k}_\infty
\end{align*}
für $\ell h \le t < (\ell + 1)h$.

Es stellt sich durch analoge Berechnung heraus, dass die Formel auch für geraden Grad $n = 2m$
gilt.
Die Tatsache, dass in diesem Fall die Knotenmittel nicht mit den Knoten zusammenfallen, aber
stattdessen Mittelpunkte der Knotenintervalle sind, macht keinen Unterschied.

\pagebreak

\section{%
    Verfeinerung%
}

\textbf{Knoten einfügen}:
Sei $p = \sum_{k=0}^{m-1} c_k b_k$, $p_\nu \in S_\tau^n$, die Parametrisierung einer Spline-Kurve.
Wenn ein neuer Knoten $s$ im Parameterintervall $D$ mit $s \in [\tau_\ell, \tau_{\ell+1})$
eingefügt wird, dann werden die Kontrollpunkte $\widetilde{c}_k$ von $p$ bzgl. der
verfeinerten Knotenfolge
\begin{align*}
    \widetilde{\tau}\colon \dotsc, \widetilde{\tau}_\ell := \tau_\ell,
    \widetilde{\tau}_{\ell+1} := s, \widetilde{\tau}_{\ell+2} := \tau_{\ell+1}, \dotsc
\end{align*}
wie folgt berechnet.

Auf den Segmenten $[c_{k-1}, c_k]$ mit $\tau_k < s < \tau_{k+n}$ werden neue Kontrollpunkte
erzeugt:
\begin{align*}
    \widetilde{c}_k := \gamma_{k,\tau}^n c_k + (1 - \gamma_{k,\tau}^n) c_{k-1},\quad
    \gamma_{k,\tau}^n := \frac{s - \tau_k}{\tau_{k+n} - \tau_k}.
\end{align*}
Die anderen Strecken des Kontrollpolygons bleiben unverändert, d.\,h.
\begin{align*}
    \widetilde{c}_k := c_k \text{ für } \tau_{k+n} \le s,\quad
    \widetilde{c}_k := c_{k-1} \text{ für } s \le \tau_k.
\end{align*}
Neue Kontrollpunkte $\widetilde{c}_k$ teilen das Segment $[c_{k-1}, c_k]$ im selben Verhältnis
$\gamma_{k,\tau}^n : (1 - \gamma_{k,\tau}^n)$
wie der Parameter $s$ das Intervall $[\tau_k, \tau_{k+n}]$,
welches der Schnitt der Träger der entsprechenden B-Splines darstellt.

Wenn $s$ mit einem Knoten zusammenfällt, also $s = \tau_\ell$ gilt,
dann müssen weniger Kontrollpunkte berechnet werden.
Genauer müssen nur $n + 1 - j$ Konvexkombinationen gebildet werden,
wenn $s$ die Vielfachheit $j$ in $\widetilde{\tau}$ hat.

Mehrere neue Knoten können durch Wiederholung der Prozedur eingefügt werden.
Insbesondere kann man durch Erhöhen der Vielfachheit eines Knotens zu $n$ erreichen,
dass die Spline-Kurve einen Kontrollpunkt interpoliert:
\begin{align*}
    \tau_{\ell-n} < \tau_{\ell-n+1} = \dotsb = \tau_\ell < \tau_{\ell+1}
    \quad\Rightarrow\quad p(\tau_\ell) = c_{\ell-n}.
\end{align*}
Daher kann das Schema zur Auswertung von Splines als $n$-fache Knoteneinfügung betrachtet werden.

\linie

\textbf{uniforme Subdivision}:
Sei
\begin{align*}
    p(t) = \sum_{k \sim D} c_k b^n(t/h - k),\quad t \in D,
\end{align*}
die Parametrisierung einer Spline-Kurve vom Grad $\le n$ mit uniformen Knoten $\tau_k = hk$.
Wenn an allen Knotenintervall-Mittelpunkten gleichzeitig neue Knoten eingefügt werden sollen,
dann können die Kontrollpunkte $\widetilde{c}_k$ der verfeinerten Knotenfolge
$\widetilde{\tau}\colon \widetilde{\tau}_k = kh/2$ wie folgt berechnet werden:
\begin{enumerate}
    \item
    Die relevanten Kontrollpunkte für das Parameterintervall $D$ werden verdoppelt:
    \begin{align*}
        \widetilde{c}_{2k} := \widetilde{c}_{2k+1} = c_k.
    \end{align*}

    \item
    Gleichzeitige Mittelwertbildung von benachbarten Kontrollpunkten:
    \begin{align*}
        \widetilde{c}_k \leftarrow (\widetilde{c}_k + \widetilde{c}_{k-1})/2.
    \end{align*}
    Dieser Schritt wird $n$ Mal durchgeführt.
\end{enumerate}
Die explizite Darstellung der neuen Kontrollpunkte ist
\begin{align*}
    \widetilde{c}_k = \sum_i s_{k-2i} c_i,\quad
    s_j := 2^{-n} \binom{n + 1}{j},
\end{align*}
wobei $s_j := 0$ für $j < 0$ oder $j > n + 1$ nach der Konvention für Binomialkoef"|fizienten.

\linie
\pagebreak

\textbf{Variationsverringerung}:
Die \begriff{Variation} einer Spline-Kurve, die durch $p = \sum_{k=0}^{m-1} c_k b_k$
parametrisiert wird, bzgl. einer Hyperebene $H$ ist nicht größer als die Variation
ihres Kontrollpolygons $c$:
\begin{align*}
    V(p, H) \le V(c, H),
\end{align*}
wobei $V$ die maximale Anzahl von Paaren von aufeinanderfolgenden Punkten auf gegenüberliegenden
Seiten $H$ bezeichnet.

Insbesondere liegt die ganze Spline-Kurve auf einer Seite von $H$, wenn das
ganze Kontrollpolygon auf einer Seite von $H$ liegt.

\section{%
    Algorithmen%
}

\textbf{Auswertung und Dif"|ferentiation}:
Ein Punkt $p(s) = \sum_{k=0}^{m-1} c_k b_k(s)$ einer Spline-Kurve mit Knotenfolge $\tau$ kann
durch wiederholtes Einsetzen von $s$ als neuen Knoten bis Vielfachheit $n$ berrechnet werden:
\begin{align*}
    \widetilde{\tau}_\ell < \widetilde{\tau}_{\ell+1} = \dotsb = \widetilde{\tau}_{\ell+n} = s <
    \widetilde{\tau}_{\ell+n+1} \quad\Rightarrow\quad p(s) = \widetilde{c}_\ell,
\end{align*}
wobei $\widetilde{\tau}_\ell$ und $\widetilde{c}_k$ die modifizierten Knoten bzw. Kontrollpunkte
bezeichnen.

Das verfeinerte Kontrollpolygon $\widetilde{c}$ ist tangential zur Spline-Kurve:
\begin{align*}
    p'(s^-) = \frac{n (\widetilde{c}_\ell - \widetilde{c}_{\ell-1})}
    {s - \widetilde{\tau}_\ell},\quad
    p'(s^+) = \frac{n (\widetilde{c}_{\ell+1} - \widetilde{c}_\ell)}
    {\widetilde{\tau}_{\ell+n+1} - s},
\end{align*}
wobei die einseitigen Ableitungen zusammenfallen, wenn $s$ nicht ein Knoten mit Vielfachheit $n$
in der ursprünglichen Knotenfolge $\tau$ ist (d.\,h. wenn mindestens ein Knoten eingefügt wird).
In diesem Fall ist
\begin{align*}
    p'(s) = \frac{n}{\widetilde{\tau}_{\ell+n+1} - \widetilde{\tau}_\ell}
    (\widetilde{c}_{\ell+1} - \widetilde{c}_{\ell-1})
\end{align*}
eine alternative Formel für den Tangentenvektor.

\linie

\textbf{\name{Bézier}-Form}:
Die \begriff{\name{Bézier}-Form} einer Spline-Kurve, die durch
$p = \sum_{k=0}^{m-1} c_k b_k$ mit B-Splines vom Grad $n$ parametrisiert wird,
erhält man durch Erhöhung der Vielfachheit jedes Knotens $\tau_k$ im Parameterintervall
$D = [\tau_n, \tau_m]$ auf $n$.
Dann gilt für $t$ in einem nicht-leeren Parameterintervall
$[\widetilde{\tau}_\ell, \widetilde{\tau}_{\ell+1}] \subset D$
der verfeinerten Knotenfolge $\widetilde{\tau}$, dass
\begin{align*}
    p(t) = \sum_{k=0}^n \widetilde{c}_{\ell-n+k} b_k^n(s),\quad
    s := \frac{t - \widetilde{\tau}_\ell}
    {\widetilde{\tau}_{\ell+1} - \widetilde{\tau}_\ell} \in [0, 1],
\end{align*}
wobei $b_k^n$ die Bernstein-Polynome und $\widetilde{c}_k$ die Kontrollpunkte bzgl.
$\widetilde{\tau}$ sind.
Daher haben die Spline-Segmente bis auf eine lineare Reparametrisierung
(welche die Form der Kurve nicht beeinflusst) Bézier-Form.

In Bézier-Form liegt jeder $(n + 1)$te Kontrollpunkt auf der Kurve und trennt die Bézier-Segmente.
Daher kann man nach Umwandlung in Bézier-Form polynomiale Algorithmen simultan auf den
verschiedenen Knotenintervallen durchführen.

\linie
\pagebreak

\textbf{Beispiel (\name{Bézier}-Form bei äquidistanten Knoten)}:
Für Spline-Kurven $p = \sum_k c_k b_k$ mit äquidistanten Knoten $\tau_k = kh$ gibt es
für die Umwandlung in Bézier-Form eine schöne geometrische Interpretation.
Zunächst werden die Kontrollpunkte mit den Tupeln beschriftet, die die Indizes der inneren Knoten
der entsprechenden B-Splines enthalten (z.\,B. hat für $n = 4$ der Kontrollpunkt
$c_3$ die Beschriftung $(4, 5, 6, 7)$, weil $b_3$ den Träger $[3, 8]h$ hat).
Dann werden die Paare von Punkten verbunden, deren Beschriftungen $(a, b, c, d)$ und
$(b, c, d, e)$ $n - 1$ Indizes gemeinsam haben.
Auf der entstehenden Verbindungsstrecke werden $e - a - 1$ Punkte mit gleichem Abstand platziert,
die mit $(a + 1, b, c, d), \dotsc, (e - 1, b, c, d)$ beschriftet werden.
Dabei müssen ggf. die Indizes jeweils aufsteigend neu geordnet werden.
Dieser Prozess wird solange wiederholt, bis alle möglichen Verbindungen erstellt wurden.
Am Ende bestehen die Bézier-Segmente aus allen Punkten mit Beschriftungen, die höchstens
zwei unterschiedliche Indizes haben.
Insbesondere haben die Bézier-Endpunkte Beschriftungen mit nur einem Index der Vielfachheit $n$.

\section{%
    Interpolation%
}

\textbf{Interpolation}:
Punkte $p_k$ und Tangentenvektoren $d_k$ (wenn nötig) können durch eine
Spline-Kurve an Parameterwerten $t_k$ interpoliert werden, indem man eine der
Interpolationsmethoden für Splinefunktionen benutzt.
Die univariaten Methoden können für jede Komponente getrennt angewandt werden,
um die Komponenten der Parametrisierung $p = \sum_k c_k b_k$ zu erhalten.
Standard-Methoden sind die kubische Hermite-Interpolation und die kubische Spline-Interpolation
mit Not-a-Knot, natürlichen oder eingespannten Randbedingungen.

Wenn nur Knoten gegeben sind, können Knoten $\tau_j$, Parameterwerte $t_k$ und
Tangentenvektoren $d_k$ (wenn nötig) durch die verfügbare Information bestimmt werden.
Einfache Möglichkeiten sind
\begin{itemize}
    \item
    $t_k - t_{k-1} = \norm{p_k - p_{k-1}}_2$,

    \item
    $t_k = \tau_{k+\ell}$, wobei der Shift $\ell$ von der Benennung der Knoten abhängt, und

    \item
    $d_k = (p_{k+1} - p_{k-1}) / (t_{k+1} - t_{k-1})$.
\end{itemize}
Genauere Approximationen der Ableitung verwenden lokale quadratische Interpolation.
Die entstehenen Formeln können insbesondere an den Endpunkten des Parameterintervalls benutzt
werden, wo einseitige Approximationen benötigt werden.

\linie
\pagebreak

\textbf{Beispiel (Interpolation mit natürlicher Spline-Kurve)}:
Um Punkte $p_0, \dotsc, p_M$ durch eine natürliche Spline-Kurve zu interpolieren,
führt man zunächst eine Standardwahl der Parameterwerte durch.
Man wählt $t_k$, sodass
\begin{align*}
    t_{k+1} - t_k = \norm{p_{k+1} - p_k}_2,\quad
    k = 0, \dotsc, M - 1.
\end{align*}
Die Parameter $t_k$ fallen mit den Knoten $\tau_3, \dotsc, \tau_m$, $m = M + 3$,
im Parameterintervall $D = [t_0, t_M]$ zusammen.
Außerhalb von $D$ wählt man äquidistante Knoten, die den Abstand des ersten bzw. letzten
Knotenintervalls erhalten.
Nun kann man das univariate Schema mit den Randbedingungen $p''(t_0) = p''(t_M) = 0$ anwenden.

Eine genauere Approximation erhält man entweder durch eingespannte oder
Not-a-Knot-Rand\-bedingungen.
Im ersten Fall sind Ableitungen an den Endpunkten vorgegeben:
\begin{align*}
    p'(t_0) = d_0,\quad p'(t_M) = d_M.
\end{align*}
Die Not-a-Knot-Randbedingungen bedeuten, dass die Knoten $\tau_4 = t_1$ und die
$\tau_{m-1} = t_{M-1}$ entfernt werden.
Daher interpoliert man nun mit den B-Splines $\widetilde{b}_0, \dotsc, \widetilde{b}_M$,
die zur reduzierten Knotenfolge
\begin{align*}
    \widetilde{\tau}\colon
    \tau_0 < \dotsb < \tau_3 = t_0 < \tau_5 < \tau_6 < \dotsb < \tau_{m-3} < \tau_{m-2} <
    t_M = \tau_m < \dotsb < \dotsb \tau_{m+3}.
\end{align*}
Daher stimmt die Dimension den Standard-Spline-Raums $S_{\widetilde{\tau}}^3$ mit der
Anzahl an Interpolationsbedingungen überein und das Interpolationsproblem ist nach
den Schoenberg-Whitney-Bedingungen korrekt gestellt.

\pagebreak
