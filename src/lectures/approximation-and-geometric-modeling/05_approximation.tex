\section{%
    Approximation%
}

\subsection{%
    \name{Schoenberg}-Schema%
}

\textbf{\name{Schoenberg}-Schema}:
Das \begriff{\name{Schoenberg}-Schema} benutzt Funktionswerte an den Knotenmitteln
$\xi_k^n = (\xi_{k+1} + \dotsb + \xi_{k+n})/n$ als Koef"|fizienten einer Spline-Approximation
an eine glatte Funktion $f$:
\begin{align*}
    f \mapsto Qf := \sum_{k=0}^{m-1} f(\xi_k^n) b_k \in S_\xi^n.
\end{align*}
Die Methode hat die Fehlerordnung $2$, d.\,h. für
$x \in [\xi_\ell, \xi_{\ell+1}] \subset D = [\xi_n, \xi_m]$
\begin{align*}
    |f(x) - (Qf)(x)| \le \frac{1}{2} \norm{f''}_{\infty,D_x} h(x)^2
\end{align*}
mit $D_x := [\xi_{\ell-n}^n, \xi_\ell^n]$.
$\norm{f''}_{\infty,D_x}$ bezeichnet dabei die Maximumsnorm von $f''$ auf $D_x$,\\
$h(x) := \max\{\xi_\ell^n - x, x - \xi_{\ell-n}^n\}$ und $k \sim x$, falls
der B-Spline $b_k$ relevant für $x$ ist ($b_k(x) \not= 0$).

Die Schoenberg-Operator erhält Positivität, Monotonie und Konvexität.
Das heißt
\begin{align*}
    f^{(k)} \ge 0 \quad\Rightarrow\quad (Qf)^{(k)} \ge 0
\end{align*}
für $k \le 2$, wenn beide Ableitungen stetig sind.
Für eine äquidistante Knotenfolge bleiben die Vorzeichen
aller Ableitungen bis Ordnung $n$ erhalten.

\subsection{%
    Quasi-Interpolation%
}

\textbf{Quasi-Interpolation}:
Ein lineares Spline-Approximations-Schema
\begin{align*}
    f \mapsto Qf := \sum_{k \sim D} (Q_k f) b_k \in S_\xi^n(D)
\end{align*}
für stetige Funktionen heißt \begriff{Quasi-Interpolant}, falls
\begin{enumerate}
    \item
    die $Q_k$ lokal beschränkte Funktionale sind, d.\,h.
    \begin{align*}
        |Q_k f| \le \norm{Q} \norm{f}_{\infty, [\xi_k, \xi_{k+n+1})}
    \end{align*}
    mit $\norm{f}_{\infty, U} := \sup_{x \in U} |f(x)|$, und

    \item
    $Q$ Polynome vom Grad $\le n$ exakt abbildet, d.\,h. $Qp = p$ auf $D$.\\
    Äquivalent dazu ist für $y \in \real$
    \begin{align*}
        Q_k p = \psi_k(y),\quad
        p(x) := (x - y)^n,
    \end{align*}
    mit $\psi_k(y) := (\xi_{k+1} - y) \dotsm (\xi_{k+n} - y)$.
\end{enumerate}

\linie
\pagebreak

\textbf{Beispiel (Quasi-Interpolant für $h\integer$)}:
Im Folgenden wird ein Quasi-Interpolant für $\xi = h\integer$ konstruiert.
Eine natürliche Wahl der Funktionale ist
\begin{align*}
    Q_k f := \sum_{\nu=0}^n w_\nu f((k + 1/2 + \nu)h),
\end{align*}
also eine gewichtete Summe von Funktionswerten an den Mittelpunkten der Knotenintervalle.
Ein solches Schema ist besonders ef"|fizient, da benachbarte Funktionale viele Funktionswerte
gemeinsam haben.
Die Koef"|fizienten $w_\nu$ bestimmen sich aus der Bedingung für die Reproduktion von Polynomen:
\begin{align*}
    \sum_{\nu=0}^n w_\nu ((k + 1/2 + \nu)h - y)^n = \prod_{\alpha=1}^n ((k + \alpha)h - y).
\end{align*}
Man muss die Polynome nicht ausmultiplizieren, um das Gleichungssystem zu lösen.
Stattdessen nutzt man mit polynomialer Interpolation die Tatsache aus, das zwei Polynome vom
Grad $\le n$ gleich sind, wenn sie in $n + 1$ verschiedenen Stützstellen gleich sind.
Wählt man die Stützstellen $y = (k + 1/2 + \mu)h$, so erhält man nach Kürzen von $h^n$ das LGS
\begin{align*}
    \sum_{\nu=0}^n w_\nu (\nu - \mu)^n = \prod_{\alpha=1}^n (\alpha - 1/2 - \mu),\quad
    \mu = 0, \dotsc, n.
\end{align*}
Die Koef"|fizienten, die $Q_k$ definieren, hängen also weder von $k$ noch von $h$ ab und
so sind die Funktionale gleichmäßig beschränkt durch $\norm{Q} := \sum_{\nu=0}^n |w_\nu|$,
was die andere Bedingung für einen Quasi-Interpolanten erfüllt.

\linie

\textbf{Standard-Projektor}:
Ein Quasi-Interpolant $f \mapsto Qf = \sum_k (Q_k f) b_k \in S_\xi^n(D)$,
bei dem jedes lineare Funktional $Q_k$ nur von Werten von $f$ in einem einzigen Knotenintervall in
$D$ abhängt, ist eine Projektion, d.\,h.
\begin{align*}
    \forall_{p \in S_\xi^n(D)}\; Qp = p.
\end{align*}
Solche Quasi-Interpolanten heißen \begriff{Standard-Projektoren}, falls die Normen
der linearen Funktionale durch eine Konstante $\norm{Q}$ begrenzt sind, die nur vom Grad $n$
abhängt.
Standard-Projektoren existieren, falls alle B-Splines das größte Knotenintervall ihres
Trägers in $D$ haben.

\linie

\textbf{Beispiel (Standard-Projektor für quadratische Splines)}:
Wählt man für quadratische Splines $[\xi_\ell, \xi_{\ell+1})$ als das mittlere Intervall
des Trägers von $b_k$, d.\,h.
\begin{align*}
    Q_k f := w_0 f(\xi_{k+1}) + w_1 f(\eta_k) + w_2 f(\xi_{k+2}),\quad
    \eta_k := (\xi_{k+1} + \xi_{k+2})/2,
\end{align*}
so erhält man durch die Marsden-Identität das LGS
\begin{align*}
    \begin{array}{rcrcrcc}
        0w_0 & + & w_1 & + & 4w_2 & = & 0\\
        w_0 & + & 0w_1 & + & w_2 & = & -1\\
        4w_0 & + & w_1 & + & 0w_2 & = & 0.
    \end{array}
\end{align*}
Die Lösung $w_0 = w_2 = -1/2$ und $w_1 = 2$ ist unabhängig von der Knotenfolge, insbesondere gilt
\begin{align*}
    |Q_k f| \le \left(\frac{1}{2} + 2 + \frac{1}{2}\right) \cdot
    \max_{x \in [\xi_{k+1}, \xi_{k+2}]} |f(x)|,
\end{align*}
also $\norm{Q} = 3$.

\pagebreak

\subsection{%
    Genauigkeit der Quasi-Interpolation%
}

\textbf{Genauigkeit der Quasi-Interpolation}:\\
Der Fehler eines Quasi-Interpolanten
$f \mapsto Qf = \sum_{k \sim D} (Q_k f) b_k \in S_\xi^n(D)$
erfüllt
\begin{align*}
    |f(x) - (Qf)(x)| \le \frac{\norm{Q}}{(n + 1)!} \norm{f^{(n+1)}}_{\infty,D_x} h(x)^{n+1},\quad
    x \in D,
\end{align*}
wobei $D_x$ die Vereinigung der Träger aller relevanten B-Splines $b_k$ mit $k \sim x$ und\\
$h(x) := \max_{y \in D_x} |y - x|$ ist.

Wenn das \begriff{lokale Gitterverhältnis} beschränkt ist, d.\,h. wenn die Quotienten der Längen
von benachbarten Knotenintervallen $\le r_\xi$ sind, dann kann der Fehler der Ableitungen auf den
Knotenintervallen $[\xi_\ell, \xi_{\ell+1})$ abgeschätzt werden durch
\begin{align*}
    |f^{(j)}(x) - (Qf)^{(j)}(x)| \le \text{const}(n, r_\xi) \norm{Q}
    \norm{f^{(n+1)}}_{\infty,D_x} h(x)^{n+1-j}
\end{align*}
für alle $j \le n$.

Durch Wahl von $Q$ als Standard-Projektor folgt insbesondere, dass Splines glatte Funktionen mit
optimaler Fehlerordnung approximieren.

\subsection{%
    Stabilität%
}

\textbf{Stabilität}:
Die Größe eines Splines $p = \sum_{k \sim D} c_k b_k \in S_\xi^n(D)$
lässt sich durch die Größe der Koef"|fizienten abschätzen, d.\,h.
\begin{align*}
    c(n) \sup_k |c_k| \le \max_{x \in D} |p(x)| \le \sup_k |c_k|.
\end{align*}
Die Konstante $c(n)$ hängt vom Grad $n$ ab.
Sie hängt nicht von der Knotenfolge $\xi$ ab, wenn $D$ das größte Intervall des Trägers von jedem
B-Spline enthält.

\linie

\textbf{Beispiel (Gegenbeispiel für die Beschränkung)}:
Die Beschränkung für die äußeren Knoten ist tatsächlich notwendig.
Dafür betrachtet man den Standard-Spline-Raum $S_\xi^2$ mit
\begin{align*}
    \xi\colon\quad
    \xi_0 = \xi_1 = -h,\;
    \xi_2 = 0,\;
    \xi_3 = 1,\;
    \xi_4 = \xi_5 = 2.
\end{align*}
Dann gilt für $x \in D = [0, 1]$, dass $b_{0,\xi}^2(x) = \frac{(x - 1)^2}{1 + h}$.
Deswegen gilt für $p = c_0 b_{0,\xi}^2$ mit $c_0 = 1$, dass
$\norm{p}_{\infty,D} = \frac{1}{1 + h} \to 0$ für $h \to \infty$,
währenddessen $\max_k |c_k| = 1$ für jedes $h$.

\pagebreak

\subsection{%
    Interpolation%
}

\textbf{\name{Schoenberg}-\name{Whitney}-Bedingungen}:
Die Koef"|fizienten eines Splines $p = \sum_{k=0}^{m-1} c_k b_k$
vom Grad $\le n$ mit Knotenfolge $\xi$,
der Daten $f_i$ an einer Folge von Punkten $x_0 < \dotsb < x_{m-1}$ interpoliert,
sind bestimmt durch das lineare Gleichungssystem
\begin{align*}
    Ac = f,\quad
    a_{i,k} := b_k(x_i).
\end{align*}
Eine eindeutige Lösung $c$ existiert für beliebige Daten $f$ genau dann, wenn
\begin{align*}
    b_k(x_k) > 0,\quad
    k = 0, \dotsc, m - 1.
\end{align*}
Wenn alle B-Splines stetig sind, bedeutet die Bedingung, dass
\begin{align*}
    \xi_k < x_k < \xi_{k+n+1},\quad
    k = 0, \dotsc, m - 1.
\end{align*}
Wegen der kleinen Träger der B-Splines ist die Koef"|fizientenmatrix eine Bandmatrix.
Jede Zeile der Matrix $A$ hat höchstens $n + 1$ Einträge, die nicht Null sind.

\linie

\textbf{Beispiel (Not-a-Knot-Bedingung)}:
Interpolation mit kubischen Splines an einfachen Knoten im Parameterintervall $D = [\xi_3, \xi_m]$
des Standard-Spline-Raums $S_\xi^3$ kommt in der Praxis öfters vor.
Allerdings gibt es nur $m - 2$ Interpolationspunkte in $D$, und wegen $\dim S_\xi^3 = m$
werden zwei zusätzliche Bedingungen benötigt.
Eine Möglichkeit ist die \begriff{Not-a-Knot-Bedingung}, die verlangt,
dass die dritte Ableitung des interpolierenden Splines am ersten und am letzten inneren Knoten
stetig ist.
Dazu nutzt das Interpolationsschema die B-Splines $\widetilde{b}_k$ bzgl. einer
reduzierten Knotenfolge $\widetilde{\xi}$, die aus $\xi$ durch Löschen von $\xi_4$ und $\xi_{m-1}$
hervorgeht.
Die Not-a-Knot-Bedingung ist daher im Spline-Raum mit eingebaut und die
Interpolationsmatrix hat die Größe $(m - 2) \times (m - 2)$.
Mit dieser Formulierung ist das entstehende LGS eindeutig lösbar, was unmittelbar
aus den Schoenberg-Whitney-Bedingungen
\begin{align*}
    \widetilde{\xi}_k < x_k = \xi_{k+3} < \widetilde{\xi}_{k+4},\quad
    k = 0, \dotsc, m - 3,
\end{align*}
folgt, die of"|fensichtlich erfüllt sind.

\linie
\pagebreak

\textbf{Beispiel (uniforme Splines)}:
Für uniforme Knoten $\xi_k = kh$ sind die Mittelpunkte der Träger der B-Splines
$b_k = b^n(\cdot/h - k)$ eine natürliche Wahl für die Interpolationspunkte:
\begin{align*}
    x_i := ih + (n + 1)h/2.
\end{align*}
Für ungeraden bzw. geraden Grad fallen diese Punkte mit den Knoten bzw. den Mittelpunkten der
Knotenintervalle zusammen.
Die entsprechenden Nicht-Null-Einträge
\begin{align*}
    a_{i,k} = b^n(x_i/h - k) = b^n(i - k + (n + 1)/2)
\end{align*}
der Interpolationsmatrix $A$ sind im Folgenden aufgeführt.
\begin{align*}
    \begin{array}{c|ccc}
        n & a_{k,k} & a_{k \pm 1,k} & a_{k \pm 2,k} \\\hline
        2 & 3/4 & 1/8 &\\
        3 & 2/3 & 1/6 &\\
        4 & 115/192 & 19/96 & 1/384\\
        5 & 11/20 & 13/60 & 1/120
    \end{array}
\end{align*}
Für Splines auf der reellen Achse ist $A$ eine \begriff{\name{Toeplitz}-Matrix},
d.\,h. $a_{i,k}$ hängt nur von der Dif"|ferenz $i - k$ ab.
Analog gilt für periodische Splines mit Periode $T = Mh$, dass $a_{i,k} = \alpha_{i - k \mod M}$.

Für einen Standard-Spline-Raum $S_\xi^n$ mit Parameterintervall $D = [nh, mh]$ müssen ein
paar Veränderungen vorgenommen werden, weil $2 \lfloor n/2 \rfloor$ der Interpolationspunkte
auf den Mittelpunkten der Träger der B-Splines außerhalb von $D$ liegen.
Wenn der Grad ungerade bzw. gerade ist, können diese Punkte auf den ersten und auf den letzten
$\lfloor n/2 \rfloor$ Mittelpunkten der Knotenintervalle bzw. Knoten in $D$ platziert werden.
Durch diese Modifikation verändern sich die ersten und letzten Zeilen der Interpolationsmatrix
$A$.
Zum Beispiel ist für $n = 3$
\begin{align*}
    A = \frac{1}{48} \begin{pmatrix}
        8 & 32 & 8 & & &\\
        1 & 23 & 23 & 1 & &\\
        & 8 & 32 & 8 & &\\
        & & 8 & 32 & 8 &\\
        & & & \ddots & \ddots & \ddots
    \end{pmatrix}.
\end{align*}
Die Einträge hängen nicht von der Gitterweite $h$ ab.
Das gilt auch im Allgemeinen und daher kann man zur Erstellung der
Interpolationsmatrizen tabulierte Werte benutzen.

\linie

\textbf{Fehler der Spline-Interpolation}:
Der Fehler eines Spline-Interpolanten $p = \sum_{k=0}^{m-1} c_k b_k \in S_\xi^n$ für eine glatte
Funktion $f$ kann durch
\begin{align*}
    |f(x) - p(x)| \le c\!\left(n, \norm{A^{-1}}_\infty\right)
    \norm{f^{(n+1)}}_{\infty,D} h^{n+1},\quad
    x \in D,
\end{align*}
abgeschätzt werden, wobei die Konstante vom Grad und der Maximumsnorm der Inversen der
Interpolationsmatrix $A$ abhängt mit $a_{i,k} = b_k(x_i)$.
$h$ ist die maximale Länge der Knotenintervalle und es wird angenommen, dass das
Standard-Parameterintervall $D = [\xi_n, \xi_m]$ alle Interpolationspunkte und auch für jeden
B-Spline das größte Intervall des Trägers enthält.

\pagebreak

\subsection{%
    Glättung%
}

\textbf{natürlicher Spline-Interpolant}:
Der \begriff{natürliche Spline-Interpolant} der Daten $(x_i, f_i)$,\\
$x_0 < \dotsb < x_M$ ist ein kubischer Spline $p$ mit einfachen Knoten an $x_\ell$,
der die Randbedingungen
\begin{align*}
    p''(x_0) = p''(x_M) = 0
\end{align*}
erfüllt.

Unter allen zweifach stetig dif"|ferenzierbaren Interpolanten minimiert $p$ das Integral
\begin{align*}
    \int_a^b |p''(x)|^2 \dx,
\end{align*}
das als Maß für die Stärke der Oszillationen von $p$ angesehen werden kann.

Alternativ sind die Randbedingungen
\begin{align*}
    p'(x_0) = d_0,\quad
    p'(x_M) = d_M
\end{align*}
möglich.
Der resultierende \begriff{eingespannte natürliche Spline} minimiert dann ebenfalls obiges
Integral.

\linie

\textbf{Glättungsspline}:
Der \begriff{Glättungsspline} $p_\sigma$ für die Daten $(x_i, f_i)$,
$x_0 < \dotsb < x_M$ und die Gewichte $w_i > 0$ ist der eindeutige kubische Spline mit
einfachen Knoten an $x_i$, der
\begin{align*}
    E(p, \sigma) := (1 - \sigma) \sum_{i=0}^M w_i |f_i - p(x_i)|^2 +
    \sigma \int_{x_0}^{x_M} |p''|^2
\end{align*}
unter allen zweifach stetig dif"|ferenzierbaren Funktionen $p$ minimiert.

Der Parameter $\sigma \in (0, 1)$ beeinflusst die Signifikanz der Daten und der Glättung.
Für $\sigma \to 0$ konvergiert $p_\sigma$ gegen den natürlichen kubischen Spline-Interpolanten,
währenddessen für $\sigma \to 1$ $p_\sigma$ gegen die Regressionsgerade
(kleinste Quadrate) konvergiert.

\pagebreak
