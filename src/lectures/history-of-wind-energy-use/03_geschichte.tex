\section{%
    Geschichte%
}

\textbf{frühe Windnutzung}:
Mensch nutzt den Wind schon seit mehr als 4000 Jahren,
die Ägypter trieben mit ihm Segelboote an

\textbf{chinesische Segel-Karussell-Windmühle}:
sog. \begriff{Vertikalläufer},
diese haben gegenüber horizontalen Achse den Vorteil, dass sie von der Windrichtung unabhängig
sind,
andererseits laufen die Blätter auf einer Hälfte des Umfangs stets gegen den Wind,
es gibt zwar die Möglichkeit, den Wind auf einer Seite abzusperren,
andererseits strömt die Luft dann eher um die Anlage herum
(z.\,B. persische Windmühle bei Neh ab 1271),
außerdem waren die chinesischen Windmühöem Langsamläufer mit großen Widerstandsflächen,
daher schlechte Leistung

\textbf{Windwagen, Ventomobil}:
gegen den Wind fahren ist möglich

\textbf{Windanlagen im Mittelalter}:
Savonius-Vorläufer des Mittelalters, 1567 -- 1617

\textbf{Holläder-Windmühlen}:
Wasserpumpen mit Windkraft,
Horizontalachse treibt Wasserpumpe an (z.\,B. archimedische Spirale),
Bockwindmühlen sind auf Drehscheiben, damit man sie in den Wind drehen kann,
Holländer verwendeten Holzzahnräder, die besser austauschbar war als Gusseisen
(wenn ein Zahn ausbricht, einfach neuen Zahn einkeilen, bei Gusseisen musste das ganze
Rad ausgetauscht werden)

\textbf{Seitenradantrieb}:
automatische Nachführung des Hauptrotos

\textbf{Segelwindmühlen auf Mykonos}:
Sturmsicherung durch einrollbare Segel,
Hochebene von Lassithi, Kreta:
Wasserpumpen für Wasser in trockenen Zeiten

\linie

\textbf{archimedische Spirale}:
Wassertransport um bis zu \SI{2}{\meter},
mehr schafft die Anlage wegen großer Reibung nicht,
Langsamläufer,
Vielblattturbine,
Blätter aus Blech
(Western Mill)

\textbf{Unterschied USA -- D}:
in Deutschland gibt es zuerst viele Papierstudien, bevor gebaut wird,
in den USA ist es umgekehrt: "`einfach mal bauen"'

\textbf{Gittermasten}:
billigste und schnellste Bauweise,
wegen vielen sich wiederholenden Elementen für die Massenfertigung geeignet

\textbf{Hermann \name{Honnef}}:
Vorschlag von drei Rotoren pro Mast,
inef"|fizient,
am besten ist ein Rotor pro Mast,
außerdem Vorschlag von Riesen-Doppelrotoren, die sich mit Magneten gegeneinander drehen und
so einen Generator betreiben,
Durchmesser von \SI{180}{\meter}, Leistung $3 \cdot \SI{20}{\mega\watt}$

\textbf{Offshore-Windenergienutzung}:
Honnef hat 1932 schwimmende Plattformen mit je zwei Rotoren vorschlagen,
automatische Windnachführung, da mit Seilen ein einem fest im mehr verankerten Pfahl befestigt,
Offshore-Windenergienutzung ist wesentlich ef"|fektiver in der Nordsee als in der Ostsee,
da in der Ostsee der Wind durch das Land abgeschwächt wurde

\textbf{Windanlage auf dem Feldberg}:
1955, später abgebaut, "`Inselanlage"'

\textbf{Allgaier WE-10}:
auch in Südafrika aufgebaut

\textbf{Windcharger}:
Jacobs Windcharger zur Stromversorgung von Farmen

\linie
\pagebreak

\textbf{Yalta, Krim}:
Zweiflügler mit Windnachführung durch einen Wagen am Boden auf Schienen,
Experiment: Klappen zur Vergrößerung des Auftriebs (wie bei Flugzeugen),
hat nichts gebracht, da die Klappen hier im Gegensatz zum Flugzeug rotieren,
außerdem Problem der dauerhaften, langfristigen Befestigung

\textbf{Putnam, USA}:
1937 -- 1945 stand dort eine Windkraftanlage mit \SI{1250}{\kilo\watt} Leistung und
\SI{52}{\meter} Durchmesser (Smith, Zweiflügler),
im Sturm hat es eines Nachts einen Flügel von \SI{8}{\tonne} weggerissen,
\SI{1}{\kilo\meter} weit geflogen

\textbf{Enfield-Andreau-Anlage}:
"`aerodynamisches"' Getriebe,
da Schlitze in der Flügeln, wo Luft aus dem Inneren der Flügel bei der Rotation austritt,
in Schlitzen unten am Boden kann die Luft in den Turm nachströmen, dort befindet sich im
Turm ein zweiter Generator,
war in Marokko in Betrieb, aber nicht sehr ef"|fizient (zu hoher Widerstand im Turm)

\pagebreak
