\chapter{%
    Der Wind%
}

\section{%
    Übersicht über die regenerativen Energiequellen%
}

\enlargethispage{10mm}

\textbf{Übersicht regenerative Energiequellen}:
regenerative Energiequellen lassen sich nach der Ursache einteilen,
zum Isotopenzerfall im Inneren der Erde gehört die Geothermie,
zur Strahlung aus dem Weltraum gehören die meisten Energieformen wie
Wind, Sonne, Wellen usw.,
zur Planetenbewegung gehört die Energie aus Gezeiten,
andererseits lassen sich die Energiequellen in Energieformen
(thermische/elektrische/chemische) einteilen,
natürlich kann Windenergie auch in Wärmeenergie umgewandelt werden,
es gab sogar einen Anlagentyp, der genau das gemacht hat,
aber das ist mit Verlusten behaftet,
daher sollte gewonnene Energie möglichst spät in die gewünschte Energieform umgewandelt werden

\section{%
    Was ist Wind?%
}

\textbf{Wind}:
\begriff{Wind} ist jede Luftbewegung relativ zum Erdboden,
auf der Erde gibt es sowohl globale Austauschvorgänge großer Luftmassen als auch
kleingliedrige, räumliche, turbulente Wirbelfelder,
die beiden Bewegungstypen liegen übereinander und sind durch eine große Grenzschicht
voneinander getrennt,
atmosphärische Luft ist Gemisch aus Gasen und Wasser in allen drei Aggregatzuständen,
Wärmeaufnahme/-abgabe verursacht oder verhindert Luftbewegung

\textbf{globale Luftmassenbewegungen}:
jede Halbkugel der Erde unterteilt sich in fünf Zonen,
polare Ostwinde,
Westwinde,
Rossbreiten,
Nordost-/Südostpassat,
Kalmenzone

\textbf{antitriptische Winde}:
Land-/Seewinde und Berg-/Talwinde sind \begriff{antitriptische Winde} (durch Reibung entstehend),
tagsüber heizt sich der Boden schneller auf als das Meer,
dadurch steigt die Luft über dem Boden auf und vom Meer strömt Luft nach (\begriff{Seewind}),
weiter oben schließt sich der Kreislauf durch umgekehrte Bewegung,
nachts ist es andersherum (Meer kühlt langsamer ab als der Boden), daher \begriff{Landwind},
\begriff{Berg-/Talwinde} entstehen auf ähnliche Weise

\section{%
    Windmessung%
}

\textbf{\name{Beaufort}-Skala}:
wurde 1806 erfunden, erst Anfang des 20. Jh. mit 12 Stufen eingeführt

\textbf{Schalenkreuz-Anemometer}:
primitiver Windmesser mit drei Schalen
und misst den Weg, den der Wind zurückgelegt hat,
für die Geschwindigkeit muss man noch durch eine Zeitspanne teilen,
moderne Windmesser messen nicht nur den Wind,
sondern klassifizieren ihn auch automatisch in einem Histogramm mit dem Anteil der Zeit
über die Windgeschwindigkeit

\textbf{Diagramme}:
in der globalen Windverteilung (Jahresmittel) ist der Wind an den Küsten und auf Bergen stärker
(über \SI{5}{\meter/\second} im Jahresmittel),
im Inneren von Kontinenten kaum Wind,
\begriff{Isoventen} sind Linien gleicher Windgeschwindigkeit,
verschiedene Diagramme möglich
(Jahresgang, Tagesgang, Windprofile über der Höhe)

\textbf{Näherungsformel zur Berechnung der Windgeschwindigkeit in der Höhe}:\\
$\overline{v_H} = \overline{v_{10}} \cdot \left(\frac{H}{10}\right)^a$
in \si{\meter/\second} mit $\overline{v_{10}}$ der Windgeschwindigkeit in \SI{10}{\meter} Höhe
in \si{\meter/\second} und $H$ der Höhe in \si{\meter},
$a = \num{0.16}$ über dem Meer,
$a = \num{0.28}$ über dem Dorf,
$a = \num{0.4}$ über der Stadt

\textbf{Windenergieanlagen}:
optimale Anlage für Baden-Württemberg hat eine
Höhe von \SI{140}{\meter} und Leistung eine von \SI{4}{\mega\watt},
bei höheren Anlagen lohnt sich die größere Leistung nicht wegen der höheren Kosten für den
Bau und die Instandhaltung,
bei doppelter Windstärke erbringen Windanlagen die achtfache Leistung

\pagebreak

\section{%
    Theorie des Windes%
}

\textbf{Wind-Theorie}:
Wind lässt sich durch Lage- und Bewegungsdefinition von Luftteilchen erklären (DGLs),
aber aufgrund der schieren Anzahl der Luftteilchen sehr kompliziert

\textbf{Schnelllaufzahl}:
dimensionsloses Verhältnis $\lambda = \frac{u}{v}$ von
Umfangsgeschwindigkeit $u$ des Rotors zur Windgeschwindigkeit $v$,
ist von der Drehzahl unabhängig,
da z.\,B. große Rotoren langsamer drehen müssen, um eine bestimmte Umfangsgeschwindigkeit zu
erreichen

\linie

\textbf{Leistungsbeiwert}:
auch \begriff{Wirkungsgrad}, $c_P$ ist das Verhältnis aus genutzter Leistung zur nutzbaren Leistung
des Winds,
bei modernen Anlagen \num{0.45} bis \num{0.5}

\textbf{\name{Betz}-Limit}:
1919 als \begriff{\name{Betz}sches Gesetz} formuliert,
gibt den maximalen Leistungsbeiwert $c_P$ an, den eine Windkraftanlage erreichen kann,
Herleitung: Windanlage als "`Black Box"', bei der der Wind mit der Geschwindigkeit $v_{\text{FFL}}$
eintritt und aus der der Wind mit der Geschwindigkeit $v_{\text{DFW}}$ austritt, bei einem
Verhältnis von $\frac{v_{\text{DNW}}}{v_{\text{FFL}}} = \frac{1}{3}$ ist der maximal mögliche
Leistungsbeiwert $c_P$ am größten, er beträgt dann $c_P = \frac{16}{27} \approx \SI{60}{\percent}$

\textbf{Leistung einer freifahrenden Turbine}:
$P = c_P \cdot \frac{\rho}{2} \cdot v_{\text{FFL}}^3 \cdot A_\Phi$
mit $c_P$ dem Leistungsbeiwert,
$\rho$ der Luftdichte,
$v_{\text{FFL}}$ der Geschwindigkeit des einströmenden Winds und
$A_\Phi$ der Rotorfläche,
Einfluss der Luftdichte wichtig für Anlagen in größerer Höche
(kann sich von \SI{1.2}{\kilogram/\meter\cubed} auf \SI{0.7}{\kilogram/\meter\cubed} absenken)
für höhere Leistung sind $c_P, \rho, v_{\text{FFL}}$ kaum beeinflussbar, d.\,h.
man muss die Rotorfläche $A_\Phi = \frac{\pi d^2}{4}$ vergrößern,
für die doppelte Leistung muss man $d$ nur um den Faktor $\sqrt{2}$ vergrößern
(d.\,h. um ca. \SI{41}{\percent} größerer Durchmesser)

\textbf{Kennlinie}:
auch $c_P$-Kurve,
gibt den Leistungsbeiwert $c_P$ über die Schnelllaufzahl $\lambda$ an,
Maximum ist der Auslegungspunkt,
bei schwankendem Wind schwankt immer auch die Leistung der Anlage,
sollte man vor dem Kauf einer Anlage genau anschauen

\section{%
    Rotorblätter%
}

\textbf{Anzahl der Blätter}:
mehr Blätter führen zwar zu höherem $c_P$,
aber bei Schnellläufer ($\lambda$ zwischen $5$ und $8$) ist der Einfluss so gering,
dass die Blattzahl unwichtig ist, Einfluss der Gleitzahl (Aerodynamik) ist viel wichtiger,
bei Langsamläufer ($\lambda$ zwischen $1$ und $3$) ist es umgekehrt

\textbf{Form der Rotorblätter}:
wählt man "`Bretter"' als Rotorblätter, die innen gleich breit wie außen sind,
dann ergibt sich ein Verlust von \SI{8.1}{\percent},
aber innen müssen die Blätter breiter sein, da dort die Umlaufgeschwindigkeit geringer ist,
wenn man das berücksichtigt, kommt man auf Verluste von \SI{1.5}{\percent} oder \SI{0.2}{\percent}

\textbf{Windmühlen}:
haben vier Flügel, zum einen, weil sie Langsamläufer sind (da sind mehr Blätter besser),
zum anderen, weil das einfacher zu bauen war (durchgängige Baumstämme),
schlechter $c_P$-Wert von anfangs \num{0.2}, da Blätter nur ebene Platten,
durch die Verwendung von gewölbten Flügeln erreicht man Werte zwischen \num{0.3} und \num{0.35}

\textbf{Flügelform StGW}:
NACA-Profile, \SI{30}{\percent} dick

\textbf{Belastung von Flügeln}:
starke Durchbiegung bei Rotorblättern (mehrere Meter)

\pagebreak

\section{%
    Herstellung von Rotorblättern%
}

\textbf{moderne Leichtbau-Fasern}:
Kevlar, Kohle, Glas

\textbf{Torusflansch}: fasergerechte Krafteinleitung

\textbf{verschiedene Bauarten für kleine Anlagen bis \SI{10}{\meter} Durchmesser}:\\
Segeltuch-Tragfläche,
Holz-Tragfläche,
Blech-Bauweise,
Honigwaben-Struktur,
Schaum usw.

\textbf{Growian-Rotorblatt}:
Probleme, weil sich unterschiedliche Materialen (Glasfaser und Stahl) nicht vertragen haben,
haben zu Rissen in der Struktur geführt

\textbf{moderne Blattherstellung}:
manuelles Hineinlegen von Faser-Tapes (in Harz getränkte Fasern) in eine Negativ-Form,
zwei Teile mit verschiedenen anderen Teilen ($\Omega$-Holm) zusammensetzen, sonst hohl

\textbf{Loch durch Fasern}:
durch Fasern dürfen nicht einfach Löcher gebohrt werden,
dies ist keine fasergerechte Bauweise und macht das System instabil,
z.\,B. gab es schon Unfälle aufgrund Luftverwirbelungen hinter startenden Flugzeugen,
stattdessen Fasern um das Loch herum leiten

\pagebreak
