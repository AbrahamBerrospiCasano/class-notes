\chapter{%
    Vorwort%
}

\textbf{Ulrich \name{Hütter} (1910 -- 1990)}:
Pionier der Windenergie-Nutzung,
auch als "`Windpapst"' bezeichnet,
lehrte von 1944 bis 1980 an der TH/TU Stuttgart (später Universität Stuttgart),
entwickelte mit der WE10 eine Windkraftanlage
(\SI{10}{\kilo\watt} Leistung, \SI{10}{\meter} Durchmesser),
200 Stück wurden ab 1949 von der Firma Allgaier Werke hergestellt,
1957 dann die StGW-34 mit einer Leistung von \SI{100}{\kilo\watt}
und \SI{34}{\meter} Durchmesser
das Urmodell aller modernen Windkraftanlagen mit freifahrenden Turbinen

\textbf{Windenergie aktuell}:
in Deutschland gibt es momentan ca. \num{22000} Windenergieanlagen mit einer Gesamtleistung von
ca. \SI{30000}{\mega\watt} oder ungefähr \SI{1.3}{\mega\watt} pro Anlage,
macht \SI{9}{\percent} der Stromversorgung in Deutschland aus

\textbf{Energiemix in Deutschland}:
der betrachtete Energiemix enthält alle Formen von Energie, also z.\,B. auch Benzin usw.,
Öl, Gas und Kohle (Braun- und Steinkohle) ergeben zusammen \SI{78.2}{\percent},
die regenerativen Energien zusammen \SI{9.4}{\percent}

\textbf{moderne Windenergieanlagen}:
2004 wurde die E-112-Anlage entwickelt mit einer Leistung von \SI{4.5}{\mega\watt} und
\SI{112}{\meter} Durchmesser (also ca. \SI{10000}{\square\meter} vom Rotor durchstrichene Fläche)

\pagebreak
