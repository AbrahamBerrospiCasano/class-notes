\section{%
    Planare Unterteilungen%
}

Planare Unterteilungen finden viele Anwendungen bei Geoinformationssystemen (GIS).
Dabei wird ein Teil der Ebene $\real^2$ in Zellen unterteilt
(Postleitzahlen, Telefonvorwahlen, Landkreise usw.).
Typische Operationen auf Unterteilungen sind zum einen die Lokalisierung
(in welcher Zelle befindet sich ein gegebener Punkt),
zum anderen die Überlagerung zweier Unterteilungen
(z.\,B. Gebiete gleicher Vorwahl und gleicher Postleitzahl).
Für das letzte Problem bestimmt man im Wesentlichen die Schnittpunkte der Zellenränder
und berechnet so die Zellen der verfeinerten Überlagerung.

\subsection{%
    Schnitt von Strecken%
}

\textbf{Problem}:
Gegeben sei eine Menge $S$ von Strecken,
jede angegeben durch ihre beiden Endpunkte in $\real^2$.
Gesucht ist die Menge aller Paare sich schneidender Strecken.

Für $|S| = n$ und $k$ der Anzahl der Schnittpunkte
ist das Ziel, einen Algorithmus mit Laufzeit $\O(n \log n + k)$ zu finden.
$\O(n \log n)$ ist nicht möglich, da $k \in \omega(n \log n)$ sein kann
(z.\,B. lauter horizontale und lauter vertikale Linien führen zu $k \in \Theta(n^2)$).

\subsubsection{%
    Sweep-Line-Algorithmus%
}

Beim \begriff{Sweep-Line-Paradigma} lässt man im Prinzip eine vertikale Gerade von links nach
rechts über die Ebene laufen, wobei die Invariante gelten soll, dass
alles links der Gerade schon berechnet wurde (hier die Schnittpunkte links der Sweep-Line)
und alles rechts der Gerade noch erkundet werden muss.

Im Folgenden geht man davon aus, dass es keine vertikalen Strecken gibt
(oBdA durch Rotation möglich).

\textbf{Sweep-Line-Datenstrukturen}:
Der Sweep-Line-Algorithmus verwaltet 2 Datenstrukturen.
\begin{itemize}
    \item
    \begriff{$X$-Struktur} (Min-Heap über $x$-Koordinaten):
    enthält alle Strecken-Endpunkte rechts der SL und die Schnittpunkte rechts der SL,
    die zu Strecken gehören, die die SL momentan schneiden und auf ihr benachbart sind

    \item
    \begriff{$Y$-Struktur} (2-4-Baum):
    enthält alle Strecken, die die SL momentan schneiden, geordnet gemäß der $y$-Koordinate
    des Schnittpunkts zwischen Segment und SL
\end{itemize}

\textbf{Sweep-Line-Algorithmus}:
Der \begriff{Sweep-Line-Algorithmus (SL-Algorithmus)} arbeitet wie folgt.
\begin{enumerate}
    \item
    Füge alle Endpunkte gemäß ihrer $x$-Koordinate in die $X$-Struktur ein.

    \item
    Die $Y$-Struktur enthält anfangs keine Elemente.

    \item
    Wiederhole Folgendes, während die $X$-Struktur nicht-leer ist:
    \begin{enumerate}
        \item
        Bestimme den Punkt $p$ aus der $X$-Struktur mit der kleinsten $x$-Koordinate und
        entferne ihn aus der $X$-Struktur.

        \item
        Rufe $\code{process}(p)$ auf.
    \end{enumerate}
\end{enumerate}

\linie
\pagebreak

\textbf{Prozedur $\code{process}(p)$}:
\begin{enumerate}
    \item
    Ist $p$ der linke Endpunkt einer Strecke $s$, dann füge $s$ in die $Y$-Struktur ein,
    teste auf Schnittpunkte rechts der SL und füge sie ggf. in die $X$-Struktur ein.

    \item
    Ist $p$ der rechte Endpunkt einer Strecke $s$, dann entferne $s$ aus der $Y$-Struktur,
    teste ggf.
    die beiden neu benachbarten Strecken in der $Y$-Struktur auf einen Schnittpunkt rechts der SL
    und füge ihn ggf. in die $X$-Struktur ein.

    \item
    Ist $p$ der Schnittpunkt zweier Strecken $s_1$ und $s_2$, dann tausche $s_1$ und $s_2$ in
    der $Y$-Struktur, teste zwei neue Nachbarschaften auf Schnittpunkte rechts der SL und
    füge sie ggf. in die $X$-Struktur ein.
\end{enumerate}

\linie

\textbf{Korrektheit}:
Der Algorithmus arbeitet korrekt.

\begin{Beweis}
    Zwei Strecken müssen auf der SL benachbart sein,
    bevor die SL ihren Schnittpunkt erreicht.
    Allerdings wird jedes Mal, wenn zwei Strecken benachbart werden können, ein Test auf
    Schnittpunkte durchgeführt, d.\,h. alle Schnittpunkte werden erfasst.
\end{Beweis}

\linie

\textbf{Problem}:
Ein Problem ist, dann Schnittpunkte mehrfach in die $X$-Struktur eingefügt werden können.
Man stelle sich zwei große, sich schneidende Strecken $s_1, s_2$
und sehr viele kurze Strecken zwischen den beiden linken Endpunkten von $s_1, s_2$ und
ihrem Schnittpunkt $S$ vor, wobei sich die kurzen Strecken vertikal nicht überlagern.
In diesem Fall wird jedes Mal, wenn die SL auf einen rechten Endpunkt einer der kurzen Strecken
stößt, der Schnittpunkt $S$ in die $X$-Struktur aufgenommen, d.\,h. $\Theta(n)$-mal.

\textbf{Lösung}:
Man kann zwar zeigen, dass das nicht sonderlich problematisch ist.
Eine einfache Lösung besteht aber darin, dass Schnittpunkte wieder entfernt werden,
wenn zwei Strecken auf der SL auf einmal nicht mehr benachbart sind
(was man ohnehin eigentlich tun muss, weil obiger Algorithmus die
oben definierte Eigenschaft der $X$-Struktur nicht erfüllt).

\linie

\textbf{Zeitbedarf}:
$\O((n + k) \log n)$

\begin{Beweis}
    In die $X$-Struktur werden insgesamt $2n + k$ viele Punkte aufgenommen, nämlich genau alle
    Strecken-Endpunkte und alle Schnittpunkte.
    Bei jedem "`Ereignis"' wird $\code{process}(p)$ aufgerufen.
    Implementiert man die $X$-Struktur als Min-Heap und die $Y$-Struktur als 2-4-Baum,
    dann kostet jeder Aufruf $\O(\log n)$ Zeit,
    denn $X$- und $Y$-Struktur enthalten jeweils stets $\O(n)$ Elemente.
    Dies ergibt eine Laufzeit von $\O((2n + k) \log n) = \O((n + k) \log n)$.
\end{Beweis}

Ist $k \in \Theta(n^2)$, dann ist der Sweep-Line-Algorithmus sogar schlechter als die naive
Methode, alle möglichen Paare zu überprüfen (was $\O(n^2)$ Zeit braucht).
Daher folgt ein besserer Algorithmus zur Bestimmung von Schnittpunkten.

\pagebreak

\subsubsection{%
    RIC-Algorithmus%
}

Seien wieder alle Strecken nicht-senkrecht und keine drei Strecken schneiden sich in einem Punkt.

\textbf{Trapezierung}:
Eine \begriff{Trapezierung} einer Menge $S$ von Strecken in $\real^2$ wird gebildet durch die Menge
selbst sowie alle vertikalen Strahlen, die von Endpunkten und Schnittpunkten aus nach oben oder
unten laufen, bis sie auf eine Strecke tref"|fen (oder bis $\pm\infty$).

\textbf{Pseudoecke}:
Eine \begriff{Pseudoecke} ist der Endpunkt eines Strahls,
der in diesem Punkt auf eine Strecke trif"|ft.

\textbf{Komplexität}:
Für $|S| = n$ und $k$ der Anzahl der Schnittpunkte gibt es $\le 2(2n + k) = 4n + 2k$ Pseudoecken
(von jedem Endpunkt und Schnittpunkt gehen zwei Strahlen aus).
Betrachtet man die Trapezierung als Graph mit den Endpunkten, Schnittpunkten und
Pseudoecken als Ecken, so hat dieser als planarer Graph $\O(n + k)$ Kanten und Facetten.
Damit hat die Trapezierung den Platzbedarf $\O(n + k)$.

\linie

Wenn eine Trapezierung von $S$ berechnet wird, werden die Schnittpunkte automatisch mitberechnet.
Der Sweep-Line-Algorithmus könnte auch die Trapezierung berechnen, allerdings in
$\O((n + k)\log n)$ Zeit.

\textbf{Berechnung einer Trapezierung}:
\begin{enumerate}
    \item
    Berechne die Trapezierung der Endpunkte
    (also nur die vertikalen Strahlen der Endpunkte).

    \item
    Füge die Strecken zufällig nacheinander in die Trapezierung ein
    (evtl. müssen vertikale Strahlen gekürzt werden oder bei Schnittpunkten
    entstehen neue).
\end{enumerate}
Der erste Schritt kostet die Zeit $\O(n \log n)$ (Zeit zum Sortieren der Endpunkte).
Beim zweiten Schritt fügt man ein Segment ein, indem man in einem Endpunkt startet und dann
das Trapez berechnet, in das die Strecke als "`nächstes"' eintreten wird.
Ist $\tau$ das aktuelle Trapez, so kostet die Bestimmung des nächsten Trapezes
$\deg(\tau)$-viele Schritte, wobei $\deg(\tau)$ die Anzahl der Endpunkte, Schnittpunkte und
Pseudoecken ist, die auf dem Rand von $\tau$ liegen.

\textbf{Probleme}:
\begin{itemize}
    \item
    Es kann Trapeze $\tau$ geben mit $\deg(\tau) = \Theta(n)$.

    \item
    Es kann sein, dass bei jeder Einfügung einer Strecke viele Trapeze durchquert werden müssen.
    Wählt man z.\,B. lauter horizontale, übereinander liegende Strecken, die symmetrisch um einen
    Punkt nach oben hin immer kleiner werden, und fügt diese Strecken von unten nach oben ein,
    dann werden bei der $i$-ten Einfügung $2(n - i)$ Trapeze durchlaufen, d.\,h.
    insgesamt $\Omega(n^2)$
    (dabei haben diese Strecken nicht einmal einen Schnittpunkt).
\end{itemize}

Die Probleme werden allerdings durch die Randomisierung beseitigt.

\linie
\pagebreak

\textbf{erwartete Laufzeit}:
$\O(n \log n + k)$

\begin{Beweis}
    Sei $s \in S$ eine Strecke, die in die Trapezierung $\T(R')$ der bisherigen Strecken
    $R'$ eingefügt werden soll.
    Die Kosten für die Einfügung von $s$ in $\T(R')$ sind
    $\sum_{\tau \in \T(R'),\; s \cap \tau \not= \emptyset} \deg(\tau)$.
    Diese Kosten können in Abhängigkeit der Trapezierung $\T(R)$, nachdem $s$ eingefügt wurde,
    mit $R := R' \cup \{s\}$ ausgedrückt werden:
    $\sum_{\tau \in \T(R),\; s \text{ begrenzt } \tau} \deg(\tau)$
    (Rückwärtsanalyse).

    Ist nur die Trapezierung $\T(R)$ gegeben, so hätte jede Strecke $s \in R$ mit gleicher
    Wahrscheinlichkeit als letzte eingefügt werden können.
    Mit $r := |R|$ sind die erw. Kosten für das Einfügen des $r$-ten Segments gleich
    $\frac{1}{r} \sum_{s \in R} \sum_{\tau \in \T(R),\; s \text{ begrenzt } \tau} \deg(\tau)
    = \frac{1}{r} \sum_{\tau \in \T(R)} \deg(\tau) \sum_{s \in R,\; s \text{ begrenzt } \tau} 1$.
    Weil es höchstens zwei Segmente $s \in R$ gibt, die ein Trapez begrenzen können, ist
    die innere Summe höchstens $2$.
    Damit sind die erw. Kosten für das $r$-te Segment
    $\le \frac{2}{r} \sum_{\tau \in \T(R)} \deg(\tau)$\\
    $= \frac{2}{r} \cdot \O(n + k_R)$ mit $k_R$ der Anzahl an Schnittpunkten in $R$,
    denn die Trapezierung $\T(R)$ hat als planarer Graph $\O(n + k_R)$ viele Ecken, Schnittpunkte
    und Pseudoecken
    (bei der Summerierung werden gleiche Ecken, Schnittpunkte oder Pseudoecken nur $\O(1)$-mal
    mehrfach gezählt).

    Es gilt $\EE[k_R] = \frac{r(r-1)}{n(n-1)} k$
    (ein Schnittpunkt ist in $R$ genau dann, wenn beide Strecken in $R$ sind).
    Damit erhält man $\frac{2}{r} \cdot \O(n + k_R)
    = \O(\frac{n}{r} + k \cdot \frac{r - 1}{n(n-1)})$
    und die erwarteten Gesamtkosten betragen
    $\sum_{r=1}^n \O(\frac{n}{r} + k \cdot \frac{r - 1}{n(n-1)})
    = \O(n \log n + k \cdot \sum_{r=1}^n \frac{r}{n^2}) = \O(n \log n + k)$.
\end{Beweis}

\subsection{%
    Lokalisierung in planaren Unterteilungen%
}

\textbf{Problem}:
Gegeben sei eine planare Unterteilung $\P$ mit $n$ Ecken
(als planare Einbettung mit geraden Linien eines planaren Graphs).
Gesucht ist für einen Punkt $q \in \real^2$ die Facette $f$ von $\P$ mit $q \in f$.

Eine naive Lösung kostet $\O(n)$ Zeit
(ob $q$ in einem Polygon $f$ ist, kann in $\O(1)$ Zeit geprüft werden, indem ein
Strahl in eine beliebige Richtung konstruiert und die Anzahl der Schnittpunkte mit dem Polygon
gezählt wird).

Das Ziel ist die Lokalisierung in $\O(\log n)$ Zeit mit $\O(n)$ Speicher.

Im Folgenden ist ein Polygon immer ein einfaches Polygon ohne Löcher
(keine Selbstüberschneidungen).

\subsubsection{%
    Triangulierung eines Polygons%
}

\textbf{Triangulierung eines Polygons}:
Aus einer Trapezierung eines Polygons mit $n$ Ecken kann man in $\O(n)$ Zeit
eine Triangulierung des Polygons berechnen.
\begin{enumerate}
    \item
    Konstruiere die Trapezierung der Kanten.

    \item
    Für jedes Trapez, das Polygonecken auf gegenüberliegenden vertikalen Strahlen besitzt,
    verbinde diese Ecken, falls sie noch nicht verbunden sind.
    Das zerlegt das Polygon in \begriff{$x$-monotone Polygone}
    (Polygone, deren Kanten in zwei Teile aufgeteilt werden können,
    sodass die $x$-Koordinaten jedes Teils monoton steigen oder fallen),
    deren eine Seite eine einzelne Kante ist
    (\begriff{$x$-Kammpolygon}).

    \item
    Trianguliere jedes $x$-monotone Polygon.
\end{enumerate}
Damit können Polygone mit $n$ Ecken in $\O(n \log n)$ Zeit trianguliert werden.

\pagebreak

\subsubsection{%
    \name{Kirkpatrick}-Hierarchie%
}

\textbf{Annahmen}:
\begin{itemize}
    \item
    Durch obigen Algorithmus kann man annehmen, dass jede Facette von $\P$ ein Dreieck ist.

    \item
    Man kann außerdem annehmen, dass die \begriff{äußere Fläche} (d.\,h. die Umrandung von $\P$)
    ebenfalls ein Dreieck ist.
    Falls das nicht der Fall ist, fügt man einfach drei Ecken um $\P$ weit entfernt von den
    anderen Ecken hinzu und trianguliert neu.
\end{itemize}
Der resultierende Graph ist immer noch planar, d.\,h. es gibt $\O(n)$ Kanten und $\O(n)$ Facetten
($n$ Anzahl der Ecken).

\linie

\textbf{unabhängige Menge}:
Sei $G = (V, E)$ ein ungerichteter Graph.\\
Eine Teilmenge $I \subset V$ heißt \begriff{unabhängig} (oder \begriff{stabil}), falls
$\forall_{x, y \in I}\; \{x, y\} \notin E$.

\textbf{Lemma}:
In einem planaren Graph $G = (V, E)$ existiert eine unabhängige Menge $I \subset V$ mit
$|I| \ge \frac{|V|}{24}$ und $\forall_{x \in I}\; \deg(x) \le 11$.

\begin{Beweis}
    Seien $n := |V|$, $m := |E|$ und $k$ die Anzahl der Facetten von $G$.
    Für eine gegebene Kantenzahl $m$ bekommt man die maximale Facettenzahl $k$,
    wenn der Graph nur aus Dreiecken besteht und voll trianguliert ist
    (bei größeren $n$-Ecken verschwendet man Kanten).
    Nutzt man je zwei Kanten zur Bildung eines Dreiecks,
    so bekommt man $3k \le 2m$.
    Aus dem eulerschen Polyedersatz $n - m + k = 2$ erhält man
    $m = n + k - 2 \le n + \frac{2m}{3} - 2$, d.\,h. $m \le 3n - 6$.

    Damit gilt $\sum_{x \in V} \deg(x) = 2m \le 6n - 12$, denn jede Kante ist genau zu zwei Knoten
    inzident.

    Es gibt daher $\ge \frac{n}{2}$ Knoten, die Grad $\le 11$ haben:
    Sei $S := \{x \in V \;|\; \deg(x) \le 11\}$, dann wäre andernfalls $|S| < \frac{n}{2}$, also
    $|V \setminus S| > n - \frac{n}{2} = \frac{n}{2}$.
    Daraus folgt
    $6n - 12 \ge \sum_{x \in V} \deg(x)$\\
    $\ge \sum_{x \in V \setminus S} \deg(x)
    \ge \sum_{x \in V \setminus S} 12 = 6n$, ein Widerspruch.

    Indem man nacheinander Ecken aus $S$ wählt, wobei man darauf achtet, keine Ecke zu wählen,
    der zu einer der bereits gewählten Ecken benachbart ist,
    erhält man eine unabhängige Menge $I$ mit $I \subset S$.
    Man wählt dabei mindestens jede zwölfte Ecke, da zu einer Ecke höchstens $11$ Ecken benachbart
    sind.
    Daher gilt $|I| \ge \frac{|S|}{12} \ge \frac{n}{24}$.
\end{Beweis}

\linie

\textbf{Konstruktion der \name{Kirkpatrick}-Hierarchie}:
Die \begriff{\name{Kirkpatrick}-Hierarchie} ist eine Folge $T_0, \dotsc, T_h$
von Triangulierungen mit $h \in \natural_0$ und wird wie folgt konstruiert.
\begin{enumerate}
    \item
    Sei $T_0 := T$ mit $T$ der ursprünglichen Triangulierung von $\P$.

    \item
    $T_{i+1}$ erhält man aus $T_i$, indem man gemäß dem Lemma eine unabhängige Menge
    $I$ von Knoten aus $T_i$ findet mit
    $|I| \ge \frac{n_i}{24}$ und $\forall_{x \in I}\; \deg(x) \le 11$
    (mit $n_i$ der Knotenzahl von $T_i$).
    Anschließend entferne die Knoten von $I$ aus $T_i$ und trianguliere die
    so entstandenen "`Löcher"' neu, um $T_{i+1}$ zu erhalten.

    \item
    Höre auf, sobald $n_i$ eine konstante Größe erreicht hat.
\end{enumerate}

\linie\
\pagebreak


\textbf{Beantwortung einer Anfrage $q \in \real^2$}:
\begin{enumerate}
    \item
    Lokalisiere $q$ in $T_h$.

    \item
    Bestimmung der Position in $T_i$ aus der Position in $T_{i+1}$, indem die $\O(1)$
    (hier höchstens elf) Dreiecke untersucht werden, die zu dem Loch gehören, in dem sich
    $q$ in $T_{i+1}$ befindet.
\end{enumerate}

\textbf{Zeitaufwand für Abfrage}:
$\O(\log n)$

\begin{Beweis}
    Es gilt $h = \O(\log n)$, da $n_i \le n (\frac{23}{24})^i$.
    Hört man auf, wenn $n_i < c$ gilt, dann hört man spätestens im $j$-ten Schritt auf mit
    $n (\frac{23}{24})^j < c$, d.\,h.
    $h \le j < \log_{23/24} \frac{c}{n}
    = \frac{\log(c/n)}{\log(23/24)}$\\
    $= \frac{1}{\log(23/24)} (\log c - \log n) = \O(\log n)$.

    Jeder Lokalisierungsschritt in obigem Algorithmus kostet $\O(1)$ Zeit,
    da immer nur $\O(1)$ Dreiecke untersucht werden müssen.
    Damit ist der Gesamt-Zeitaufwand $\O(\log h)$.
\end{Beweis}

\textbf{Platzaufwand}:
$\O(n)$

\begin{Beweis}
    Aus $n_i \le n (\frac{23}{24})^i$ folgt für den Platzbedarf
    $\sum_{i=0}^{h} n_i
    \le n \frac{1 - (23/24)^{h+1}}{1 - 23/24}
    = \O(n (\frac{23}{24})^h)$.
    Wegen $(\frac{23}{24})^h \le 1$
    erhält man einen Platzbedarf von $\O(n (\frac{23}{24})^h) = \O(n)$.
\end{Beweis}

\subsubsection{%
    Anwendung von Polygontriangulierung: Sichtbarkeitsprobleme%
}

\textbf{Museumswächterproblem}:
Gegeben sei ein Museum als ein einfaches Polygon mit $n$ Ecken.
Gesucht ist die minimale Anzahl von omnidirektionalen Wächtern, sodass das Museum vollständig
überwacht wird.

\textbf{Lemma}:
Jedes einfache Polygon mit $n$ Ecken kann mit
$\left\lfloor \frac{n}{3} \right\rfloor$ Wächtern überwacht werden.

\begin{Beweis}
    Zuerst konstruiere man eine Triangulierung des Polygons.
    Dann ist der \begriff{duale Graph}
    (Dreiecke als Knoten, Kanten zwischen benachbarten Dreiecken) ein Baum,
    da er zusammenhängend und kreisfrei ist
    (gäbe es einen Kreis, hätte das Polygon ein Loch).

    Anschließend berechne eine $3$-Färbung der Ecken der Triangulierung,
    d.\,h. jede Ecke bekommt eine Farbe $\text{r}, \text{g}, \text{b}$ zugewiesen,
    wobei zwei benachbarte Ecken jeweils verschiedene Farben haben müssen.
    Das geht wie folgt:
    \begin{enumerate}
        \item
        Wähle ein beliebiges Dreieck und färbe die Ecken in den drei Farben.

        \item
        Wähle ein Dreieck, von dem zwei Ecken bereits gefärbt sind, und färbe die dritte Ecke
        entsprechend.

        \item
        Wiederhole \emph{(2)}, bis alle Dreiecke gefärbt sind.
    \end{enumerate}
    Das ist möglich, weil der duale Graph ein Baum ist:
    Gäbe es zwei benachbarte Ecken mit derselben Farbe, dann gäbe es einen Kreis im Baum,
    ein Widerspruch.

    Schließlich platziert man Wächter an den Ecken mit der Farbe, die am seltensten vorkommt.
    Das sind $\le \left\lfloor \frac{n}{3} \right\rfloor$-viele und die Wächter überwachen
    jedes Dreieck, damit auch das gesamte Polygon.
\end{Beweis}

\textbf{Lemma}:
Es gibt einfache Polygone mit $n$ Ecken,
die $\left\lfloor \frac{n}{3} \right\rfloor$ Wächter benötigen.

\begin{Beweis}
    Stellt man sich einen Kamm vor, der einen schmalen Gang und $k$ lange Zacken besitzt,
    so benötigt jede Zacke einen eigenen Wächter.
    Das zugehörige Polygon hat $n = 3k$ Ecken, für $n \equiv_3 1$ bzw. $n \equiv_3 2$
    fügt man noch 1 bzw. 2 Ecken hinzu, ohne die Struktur zu verändern.
\end{Beweis}

\pagebreak

\subsubsection{%
    \emph{Zusatz}: Polygontriangulierung in erwartet
    \texorpdfstring{$\O(n\log^\ast n)$}{O(n log* n)}%
}

Im Folgenden wird gezeigt, wie man ein einfaches Polygon mit $n$ Ecken in Zeit
$\O(n \log^\ast n)$ trapezieren kann.
Daraus folgt dann automatisch, dass ein solches Polygon in Zeit $\O(n \log^\ast n)$
trianguliert werden kann.

\textbf{Idee}:
Man kann nicht einfach so wie vorher trapezieren, da dies die Endpunkte nebenbei sortiert und
Sortierung $\Omega(n \log n)$ Zeit kostet.
Die Idee ist nun, die Trapezierung zu berechnen und nebenbei eine Datenstruktur zur
Punktlokalisierung (für die Endpunkte der noch einzufügenden Strecken)
in der aktuellen Trapezierung aufzubauen.

\textbf{Datenstruktur}:
Die Datenstruktur zur Punktlokalisierung ist in jedem Schritt ein gerichteter Graph ohne Zykeln.
Sie sieht baumähnlich aus und besteht aus Knoten der folgenden Arten:
\begin{itemize}
    \item
    \emph{Trapeze}:
    stellen die Blätter dar (Knoten mit Ausgangsgrad $0$)

    \item
    \emph{$X$-Knoten}:
    enthalten einen Punkt,
    stellen eine Unterteilung in zwei Hälften
    (mit geraden Rändern) dar, gemäß der $x$-Koordinate des Knotenpunkts

    \item
    \emph{$Y$-Knoten}:
    enthalten eine Strecke
    stellen eine Unterteilung in zwei Hälften
    (mit schiefem Rand) dar, gemäß der Knotenstrecke
\end{itemize}
Der Graph ist i.\,A. kein Baum, da es Knoten mit Eingangsgrad $> 1$ geben kann.

\linie

\textbf{Lemma}:
Seien $\tau_i, Q_i$ die Trapezierungen bzw. Suchstrukturen, die durch die Einfügung von
$s_i$ in $\tau_{i-1}, Q_{i-1}$ entstehen.
Ist $q$ in $\tau_{i-1}, Q_{i-1}$ lokalisiert worden,
dann ist $\O(\frac{1}{i})$ der erwartete Aufwand,
$q$ in $\tau_i, Q_i$ zu lokalisieren.

\begin{Beweis}
    Es gibt zwei verschiedene Fälle.
    \begin{enumerate}
        \item
        Ist das Trapez, das $q$ enthält, identisch in $\tau_i$ und $\tau_{i-1}$,
        dann ist der Aufwand, $q$ in $\tau_i$ zu lokalisieren, gleich $0$.

        \item
        Ansonsten ist der Aufwand $\O(1)$.
        Dieser Fall tritt mit Wahrscheinlichkeit $\O(\frac{1}{i})$ ein, da $s_i$
        einer der Grenzen des Trapezes in $\tau_i$ definieren muss.
    \end{enumerate}
    Daraus ergibt sich ein erwarteter Aufwand von
    $\O(\frac{1}{i}) \cdot \O(1) + \O(1 - \frac{1}{i}) \cdot 0 = \O(\frac{1}{i})$.
\end{Beweis}

\textbf{Korollar}:
Ein Punkt $q$ kann in $\tau_i$ in erwartet $\O(\log i)$ lokalisiert werden
(von der Wurzel aus).

\begin{Beweis}
    Zunächst lokalisiert man $q$ in $\tau_0$,
    anschließend induktiv in $\tau_j$ (ausgehend von $\tau_{j-1}$).
    Das kostet nach dem Lemma insgesamt erwartet $\sum_{j=1}^i \O(\frac{1}{j}) = \O(\log i)$.
\end{Beweis}

\textbf{Theorem}:
Seien $S$ eine Menge von $n$ sich nicht schneidenden Strecken
(außer ggf. in den Endpunkten) und $\T(S), Q(S)$ die
zu $S$ gehörige Trapezierung bzw. Suchstruktur.
Dann können $\T(S), Q(S)$ in erwartet $\O(n \log n)$ konstruiert werden.
$Q(S)$ hat dabei die Größe $\O(n)$.
Die erwartete Zeit zur Lokalisierung von $q \in \real^2$ in $\T(S), Q(S)$ beträgt $\O(\log n)$.

\begin{Beweis}
    Der Aufwand zur Konstruktion von $\T(S), Q(s)$ ist der Aufwand für die Aktualisierung der
    Trapezierung und der Suchstruktur und der Aufwand zur Punktlokalisierung.
    Weil der erste Aufwand bei Einfügung eines Segments erwartet $\O(1)$ beträgt und
    die Punktlokalisierung in $\O(\log i)$ durchgeführt werden kann,
    ist der Gesamtaufwand $\sum_{i=1}^n \O(\log i) = \O(n \log n)$.
\end{Beweis}

\linie
\pagebreak

\textbf{Algorithmus zur Polygontriangulierung in erwartet $\O(n \log^\ast n)$}:
\begin{enumerate}
    \item
    Bringe die Strecken in eine zufällige Reihenfolge.

    \item
    Füge die ersten $\frac{n}{\log n}$ Strecken ein.

    \item
    Verfolge das gesamte Polygon durch $\T_{n/\log n}$ und bestimme dabei die Positionen
    aller noch nicht eingefügten Strecken.

    \item
    Füge die nächsten $\frac{n}{\log \log n}$ Strecken ein usw.

    \item
    Wiederhole, bis $\log \dotsb \log n$ konstant ist
    (d.\,h. bis Lokalisierung in konstanter Zeit möglich ist).
\end{enumerate}

\linie

\textbf{Lemma}:
Sei $R \subset S$ eine zufällige Teilmenge der Strecken mit $r := |R|$.
Dann ist die erwartete Anzahl an Schnitten zwischen Segmenten aus $S \setminus R$ und
Vertikalen von $\T(R)$ höchstens $4(n-r)$.

\begin{Beweis}
    Für $T \subset S$ und $s \in T$ sei $\deg(s, \T(T))$
    gleich der Anzahl der Vertikalen, die in $\T(T)$ an $s$ anstoßen.
    Es gilt $\sum_{s \in T} \deg(s, \T(T)) \le 4|T|$, weil
    jedes Segment zwei Endpunkte hat und von jedem Endpunkt zwei Vertikalen ausgehen
    (die nicht zwangsläufig in $\T(T)$ anstoßen müssen).
    Für $R \subset S$ und $s \notin R$ ist die Anzahl der von $s$ geschnittenen
    Vertikalen von $\T(R)$ gleich $\deg(s, \T(R \cup \{s\}))$.
    Damit ist die erwartete Anzahl an Schnitten zwischen Segmenten aus $S \setminus R$ und
    Vertikalen von $\T(R)$ gleich
    $\frac{1}{\binom{n}{r}} \sum_{R \subset S,\; |R| = r}
    \sum_{s \in S \setminus R} \deg(s, \T(R \cup \{s\}))$
    (die erste Summe berücksichtigt die Wahl einer zufälligen Teilmenge $R \subset S$
    mit $|R| = r$).
    Man kann dies umschreiben zu
    $\frac{1}{\binom{n}{r}} \sum_{R' \subset S,\; |R'| = r+1} \sum_{s \in R'} \deg(s, \T(R'))
    \le \frac{1}{\binom{n}{r}} \sum_{R' \subset S;\; |R'| = r+1} 4|R'|$\\
    $= \frac{1}{\binom{n}{r}} \cdot \binom{n}{r+1} 4(r+1)
    = \frac{r!(n-r)!}{(r+1)!(n-r-1)!} \cdot 4(r+1)
    = 4(n-r)$.
\end{Beweis}

\linie

\textbf{Zeitbedarf}:
$\O(n \log^\ast n)$

\begin{Beweis}
    Schritt \emph{(2)} kostet erwartet $n \log n \cdot \O(\log n) = \O(n)$ Zeit.
    Nach dem Lemma von eben kostet Schritt \emph{(3)} erwartet $\O(n)$.
    Für $i \ge \frac{n}{\log n}$ ist danach die Lokalisierung in $\T_i$ in erwartet
    $\O(\sum_{j=n/\log n}^i \frac{1}{j})
    = \O(\log i - \log(\frac{n}{\log n}))
    = \O(\log(\log n))$ möglich.
\end{Beweis}

\pagebreak
