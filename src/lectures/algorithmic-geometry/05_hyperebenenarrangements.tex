\section{%
    Hyperebenenarrangements und Dualität%
}

\subsection{%
    Hyperebenenarrangements%
}

\textbf{Hyperebene}:
Eine \begriff{Hyperebene} $h \subset \real^d$ ist eine Teilmenge von $\real^d$ der Form\\
$h = \{x \in \real^d \;|\; \sp{x, a} = c\}$ für ein $a \in \real^d$ und ein $c \in \real$.

\textbf{Äquivalenzrelation $\sim_\H$ auf $\real^d$}:
Eine Hyperebene $h = \{x \in \real^d \;|\; \sp{x, a_h} = c_h\}$ induziert die Partition
$\real^d = h \dcup \{x \in \real^d \;|\; \sp{x, a_h} > c_h\} \dcup
\{x \in \real^d \;|\; \sp{x, a_h} < c_h\}$ von $\real^d$.\\
Definiere $\sigma_h\colon \real^d \to \{-1, 0, +1\}$ mit
$\sigma_h(x) := \sgn(\sp{x, a_h} - c_h)$ (Lage von $x$ bzgl. $h$).\\
Für eine Familie $\H := \{h_1, \dotsc, h_n\}$ von Hyperebenen sei
$\sigma_\H\colon \real^d \to \{-1, 0, +1\}^n$ mit\\
$\sigma_\H(x) := (\sigma_{h_1}(x), \dotsc, \sigma_{h_n}(x))$.\\
Dann ist $\sim_\H$ eine Äquivalenzrelation auf $\real^d$, wobei $x \sim_\H y$ gelte,
falls $\sigma_\H(x) = \sigma_\H(y)$.

\textbf{Hyperebenenarrangement}:
Sei $\H = \{h_1, \dotsc, h_n\}$ eine Familie von Hyperebenen.\\
Dann heißt die Menge $\real^d/{\sim_\H} \subset \P(\real^d)$ aller Äquivalenzklassen
bzgl. $\sim_\H$\\
\begriff{Hyperebenenarrangement (HE-Arrangement oder HEA)}.

Jede Äquivalenzklasse ist als Schnitt von (konvexen) Halbräumen konvex.

\textbf{Fragen}:
\begin{itemize}
    \item
    Wie kann man HE-Arrangements berechnen?
    
    \item
    Was ist die Komplexität eines HE-Arrangements?
\end{itemize}

\linie

Man nimmt an, dass die Hyperebenen in allgemeiner Lage liegen,
d.\,h. der Schnitt von $k$ Hyperebenen ist stets $(d - k)$-dimensional
($k = 1, \dotsc, d + 1$).

\textbf{Komplexität eines HE-Arrangements in $\real^2$}:
\begin{itemize}
    \item
    \emph{$\binom{n}{2}$ Ecken}:
    Jedes Paar zweier Geraden schneidet sich.
    
    \item
    \emph{$n^2$ Kanten}:
    Jede Gerade wird durch die anderen $n - 1$ Geraden in $n$ Stücke geteilt.
    
    \item
    \emph{$1 + \frac{n(n+1)}{2}$ Zellen}:
    Induktiv teilt eine neue $n$-te Gerade $n$ Zellen in jeweils zwei Hälften auf.
    Eine andere Zählung ist, dass jeder Knoten der unterste Punkt genau einer Zelle ist
    und kein Knoten der unterste Punkt zweier Zellen ist.
    Es gibt also $\binom{n}{2}$ (Anzahl der Punkte) nach unten beschränkte Zellen,
    es fehlen noch die $n + 1$ nach unten unbeschränkten Zellen.
    Damit erhält man $\binom{n}{2} + n + 1 = 1 + \frac{n(n+1)}{2}$ Zellen.
\end{itemize}

\textbf{Komplexität eines HE-Arrangements in $\real^3$}:
\begin{itemize}
    \item
    \emph{$\binom{n}{3}$ Ecken}: Jedes Tripel dreier Ebenen schneidet sich.
    
    \item
    \emph{$\binom{n}{2} (n - 1)$ Kanten}:
    Die Geraden, auf denen die Kanten liegen, korrespondieren zu allen möglichen Ebenenpaaren
    (jedes Ebenenpaar schneidet sich in einer Gerade und diese Gerade liegt auf keinen
    anderen zwei Ebenen).
    Daher gibt es $\binom{n}{2}$ solcher Geraden, von denen jede durch die übrigen $n - 2$
    Ebenen in $(n - 1)$ Kanten unterteilt wird.\\
    Eine andere Zählung zählt die Kanten auf einer bestimmten Ebene $E$.
    %Projiziert man die anderen Ebenen auf $E$, so erhält man $(n - 1)$ Geraden.
    %Nach der $\real^2$-Komplexität gibt es $(n-1)^2$ Kanten auf $E$.
    %Lässt man $E$ durch alle Ebenen laufen, so zählt man jede Kante doppelt, d.\,h.
    %man bekommt $(n-1)^2 \frac{n}{2} = \binom{n}{2} (n - 1)$ Kanten.
    
    \item
    \emph{$n (\binom{n-1}{2} + n)$ Facetten}:
    Auf einer bestimmten Ebene $E$ gibt es nach dem $\real^2$-Fall genau $\binom{n-1}{2} + n$
    viele Zellen für $(n - 1)$ Geraden.
    Gehe für $E$ durch alle $n$ Ebenen durch.
    
    \item
    \emph{$\binom{n}{3} + \binom{n}{2} + n + 1$ Zellen}:
    Es gibt $\binom{n}{3}$ unterste Punkte.
    Die nach unten in der $z$-Richtung unbeschränkten Zellen entsprechen genau den
    $\real^2$-Zellen, die entstehen, wenn man die Ebenen
    von unten in der $z$-Richtung "`anschaut"'.
    Von diesen Zellen gibt es nach dem $\real^2$-Fall genau $\binom{n}{2} + n + 1$ Stück.
\end{itemize}

\pagebreak

\subsection{%
    Inkrementelle Konstruktion und Zonensatz%
}

Die Berechnung von HE-Arrangements in $\real^2$ kann auf zwei bereits bekannte Arten erfolgen:
\begin{itemize}
    \item
    \emph{naiver Sweepline-Algorithmus}:
    $\O(n^2 \log n)$ Zeit ($\O((n+k) \log n)$ und $k = \Theta(n^2)$ Ereignisse/Schnittpunkte
    müssen verarbeitet werden)
    
    \item
    \emph{RIC-Algorithmus zur Bestimmung von Strecken-Schnittpunkten}:
    erwartet $\O(n^2)$ Zeit
\end{itemize}
Im Folgenden wird ein Algorithmus gezeigt, mit dem man HE-Arrangements deterministisch in
$\O(n^2)$ Zeit bestimmen kann.
Dazu verwendet man einen inkrementellen Ansatz.
Wegen der "`großzügigeren"' Schranke von $\O(n^2)$ benötigt man keine Randomisierung.

$\Omega(n^2)$ Zeit wird auf jeden Fall benötigt, weil die Ausgabegröße $\Theta(n^2)$ ist.

\linie

\textbf{inkrementelle Konstruktion von HEAs}:
Die \begriff{inkrementelle Konstruktion von HEAs} in $\real^2$ verläuft für $n$ Geraden
$h_1, \dotsc, h_n$ in $\real^2$ wie folgt.
Definiere $\A_i$ als das HE-Arrangement der Geraden $h_1, \dotsc, h_i$.
\begin{enumerate}
    \item
    Konstruiere das leere Arrangement $\A_0$.
    
    \item
    Für $i = 1, \dotsc, n$ konstruiere $\A_i$ aus $\A_{i-1}$ wie folgt:
    \begin{enumerate}
        \item
        Finde die Zelle ganz links, durch die $h_i$ geht
        (geht in $\O(i)$ Zeit, wenn die $h_i$ nach ihrer Steigung sortiert sind).
        
        \item
        Bestimme, wo $h_i$ die Zelle verlässt bzw. welche neue Zelle von $h_i$ betreten wird.
        
        \item
        Wiederhole, bis die neue Zelle nach rechts unbeschränkt ist.
    \end{enumerate}
\end{enumerate}

Die Kosten, um von einer Zelle $c$ in die nächste zu kommen, sind $\O(\deg(c))$ mit
$\deg(c)$ der Anzahl von Kanten oder Ecken von $c$.
Damit sind die Gesamtkosten für die Einfügung von $h_i$ gleich
$\O(\sum_{c \in \A_{i-1},\; C \cap h_i \not= \emptyset} \deg(c))$.
Es ist allerdings nicht direkt klar, ob das in $\O(i)$ ist.
Man kann sich z.\,B. Arrangements vorstellen, in der eine einzelne Zelle bereits durch $\O(i)$
Kanten begrenzt wird.

\linie

\textbf{Satz (Zonensatz)}:
Sei $\zone(h, \A) := \{c \in \A \;|\; c \cap h \not= \emptyset\}$
für ein Arrangement $\A$ von $n$ Geraden und eine zusätzliche Gerade $h$.
Definiere $z(h, \A) := \sum_{c \in \zone(h, \A)} \deg(c)$ sowie\\
$z(n) := \max\{z(h, \A) \;|\; \text{$\A$ Arrangement von $n$ Geraden, $h$ zusätzliche Gerade}\}$.\\
Dann gilt $z(n) \le 6n$.

\begin{Beweis}
    Seien $\A$ ein beliebiges Arrangement von $n$ Geraden und $h$ eine zusätzliche Gerade.
    Gezählt werden nun die Kanten der Zellen in $\A$, die $h$ schneidet.
    Ist die Anzahl nach oben beschränkt durch $6n$, so gilt $z(n) \le 6n$.
    
    Es werden zunächst nur die Linksadjazenzen gezählt.
    Betrachte die Geraden $h_1, \dotsc, h_n$ von $\A$ aufsteigend geordnet nach der $x$-Koordinate
    ihres Schnittpunkts mit $h$.
    Die erste Gerade $h_1$ erzeugt eine Linksadjazenz.
    Jede weitere Gerade $h_i$ teilt die Zelle, die sich bis dahin am weitesten rechts befindet,
    in zwei Zellen und erzeugt höchstens $3$ Linksadjazenzen.
    Damit gibt es $\le 3n$ Linksadjazenzen.
    Für die Rechtsadjazenzen geht die Argumentation analog, so dass es
    $\le 6n$ Adjazenzen gibt.
\end{Beweis}

\linie

\textbf{Zeitbedarf}:
$\O(n^2)$

\begin{Beweis}
    Nach dem Zonensatz gilt
    $\O(\sum_{c \in \A_{i-1},\; C \cap h_i \not= \emptyset} \deg(c)) = \O(i)$.\\
    Dadurch erhält man $\sum_{i=1}^n \O(i) = \O(n^2)$ als Gesamtlaufzeit.
\end{Beweis}

\pagebreak

\subsection{%
    Dualität und Anwendungen%
}

\subsubsection{%
    Dualität%
}

\textbf{(nicht-vertikale) Gerade}:
Im Folgenden sind \begriff{(nicht-vertikale) Geraden} in der $x$-$y$-Ebene definiert durch
$(y = kx - d) := \{(x, y) \in \real^2 \;|\; y = kx - d\}$ mit $k, d \in \real$.

\textbf{Seiten einer Gerade}:\\
Für eine Gerade $\ell := (y = kx - d)$ sei
$\ell^{\pm} := \{(x, y) \in \real^2 \;|\; y \gtreqless kx - d\}$.\\
($\ell^+$ ist die Menge aller Punkte über $\ell$, einschließlich $\ell$ selbst).

\textbf{Dualitätstransformation}:
Die \begriff{Dualitätstransformation} $\D$ bildet\\
Punkte $(p_x, p_y) \in \real^2$ auf Geraden $\D(p_x, p_y) := (y = p_x x - p_y)$ ab und\\
Geraden $y = kx - d$ auf Punkte $\D(y = kx - d) := (k, d)$.

\textbf{Lemma (Dualität)}:
Für einen Punkt $p \in \real^2$ und eine Gerade $\ell$ gilt
\begin{enumerate}
    \item
    $p \in \ell \iff \D(\ell) \in \D(p)$,
    
    \item
    $p \in \ell^+ \iff \D(\ell) \in \D(p)^+$ und
    
    \item
    $p \in \ell^- \iff \D(\ell) \in \D(p)^-$.
\end{enumerate}

\begin{Beweis}
    Seien $(p_x, p_y) := p$ und $(y = kx - d) := \ell$, d.\,h.
    $\D(\ell) = (k, d)$ und $\D(p) = (y = p_x x - p_y)$,
    es gilt also $p \in \ell \iff p_y = kp_x - d \iff d = p_x k - p_y \iff \D(\ell) \in \D(p)$.
    Für \emph{(2)} und \emph{(3)} gilt analog
    $p \in \ell^\pm \iff p_y \gtreqless kp_x - d \iff
    d \gtreqless p_x k - p_y \iff \D(\ell) \in \D(p)^\pm$.
\end{Beweis}

\subsubsection{%
    Erkennung von Kollinearität von Punkten%
}

\textbf{Erkennung von Kollinearität von Punkten}:\\
Gegeben sind $n$ Punkte in $\real^2$.
Gefragt ist, ob drei der Punkte kollinear sind\\
(d.\,h. hier, ob die Punkte auf einer nicht-vertikalen Geraden liegen).

Seien $p_1, p_2, p_3 \in \real^2$ drei Punkte mit $\ell_i := \D(p_i)$.
Dann gilt:\\
$p_1, p_2, p_3 \text{ kollinear} \iff
\exists_{\ell \text{ Gerade}}\; p_1, p_2, p_3 \in \ell \iff
\exists_{p \in \real^2}\; p \in \ell_1 \cap \ell_2 \cap \ell_3$,
nämlich $p := \D(\ell)$.
Drei der $n$ Punkte sind also kollinear genau dann, wenn sich drei der dualen Geraden in einem
Punkt schneiden.
Dies kann während der Konstruktion des entsprechenden HE-Arrangements festgestellt werden
(wenn die nächste Gerade eine Zelle genau auf einem Randknoten verlässt),
d.\,h. in Zeit $\O(n^2)$.

\pagebreak

\subsubsection{%
    Bestimmung des flächenkleinsten Dreiecks%
}

\textbf{flächenkleinstes Dreieck}:
Gegeben ist $P \subset \real^2$ mit $n := |P|$.\\
Gesucht ist $p, q, r \in P$ mit $A(\triangle pqr)$ (Fläche von $\triangle pqr$) minimal
(sowie $|\{p, q, r\}| = 3$).

\textbf{naiv}:
Teste alle $\binom{n}{3}$ Tupel in $\O(n^3)$ Zeit.

\linie

\textbf{besser}:
Ein Dreieck $\triangle pqr$ ist durch eine Strecke $qr$ und einen Punkt $p$ eindeutig bestimmt.
Im dualen Raum entspricht dies einem Punkt $\D(qr)$ und einer Gerade $\D(p)$
(wenn man $qr$ mit der Gerade durch $qr$ identifiziert).

\textbf{Satz}:
Seien $p, q, r \in P$ mit $\triangle pqr$ flächenminimal.\\
Dann gibt es eine Zelle $c$ im dualen Arrangement zu $P$, sodass
$\D(qr)$ und $\D(p)$ auf $\partial c$ liegen.

\begin{Beweis}
    Sei $(p_x, p_y) \in \real^2$ ein Punkt und $y = kx - d$ eine Gerade.
    Dann ist der vertikale Abstand von $(p_x, p_y)$ zu $y = kx - d$ ist $D := p_y - kp_x + d$.
    Der vertikale Abstand von $\D(y = kx - d) = (k, d)$ zu $\D(p_x, p_y) = (y = p_x x - p_y)$
    gleich $d - p_x k + p_y = D$, d.\,h. $\D$ lässt vertikale Abstände invariant.
    
    $\Delta pqr$ ist flächenminimal genau dann, wenn $p$ den kleinsten Abstand aller Punkte zu
    $qr$ besitzt.
    Dies gilt genau dann, wenn $p$ den kleinsten vertikalen (d.\,h. vertikal gemessenen) Abstand
    zu $qr$ besitzt.
    Dies gilt genau dann, wenn $\D(qr)$ den kleinsten vertikalen Abstand zu $\D(p)$ besitzt.
    
    Angenommen, $\D(p)$ und $\D(qr)$ würden an verschiedenen Zellen anliegen.
    Dann gäbe es einen Punkt $p' \in P$, sodass die Gerade $\D(p')$ zwischen $\D(p)$ und $\D(qr)$
    liegt, d.\,h. $\D(qr)$ hätte zu $\D(p')$ einen kleineren vertikalen Abstand,
    also $A(\triangle p'qr) < A(\triangle pqr)$, ein Widerspruch zu $\triangle pqr$ flächenminimal.
\end{Beweis}

\textbf{Algorithmus}:
Es müssen nur alle Kanten im dualen Arrangement jeweils
zusammen mit den Ecken der beiden Zellen, die an der Kante anliegen, inspiziert werden.
Dies geht in $\O(n^2)$ Zeit während der Konstruktion des HE-Arrangements:
Wenn die neue Gerade $g$ eine Zelle $c$ betritt, dann inspiziere $g$ zusammen mit allen Ecken von
$c$.
Außerdem erzeugt $g$ mit $c$ zwei neue Schnittpunkte, diese müssen zusammen mit allen Kanten
von $c$ inspiziert werden.
Nach dem Zonensatz wird für $g$ die Zeit $\O(n)$ benötigt
(Anzahl Kanten/Punkte der Zellen, durch die $g$ geht),
d.\,h. insgesamt $\O(n^2)$ Zeit.

\linie

Es ist unbekannt, ob es einen Algorithmus gibt, der das Problem in $o(n^2)$ löst.
Man geht aber davon aus, dass dies nicht der Fall ist, weil das Problem 3SUM-schwer ist.
Man glaubt, dass $\Omega(n^2)$ die untere Schranke für 3SUM ist.

\textbf{3SUM}:
Gegeben sei $S \subset \integer$ mit $n := |S|$.
Gefragt ist, ob $a, b, c \in S$ existieren mit $a + b + c = 0$.

\textbf{3SUM $\le$ "`flächenkleinstes Dreieck"'}:
Sei $S \subset \integer$ eine Instanz von 3SUM.
Dann gilt für $a, b, c \in S$, dass $a + b + c = 0$ genau dann, wenn
$(a, a^3), (b, b^3), (c, c^3) \in \real^2$ kollinear sind.

Das Problem des flächenkleinsten Dreiecks ist eine Verallgemeinerung der Kollinearität
von Punkten
(drei Punkte sind kollinear $\iff$ das flächenkleinste Dreieck besitzt Fläche $0$).

\pagebreak

\subsubsection{%
    Polarität: Dualität von Halbraumschnitten und konvexen Hüllen%
}

\textbf{Polytop}:
Ein \begriff{Polytop} ist die konvexe Hülle einer
endlichen Punktmenge $S \subset \real^d$, d.\,h.\\
$\CH(S) := \{\sum_{p \in S} \lambda_p p \;|\; \lambda_p \ge 0,\; \sum_{p \in S} \lambda_p = 1\}$.

\textbf{Polyeder}:
Ein \begriff{Polyeder} ist der Schnitt einer endlichen Menge von abg. Halbräumen in $\real^d$.

\textbf{Fall $\real^3$}:
In $\real^3$ ist der Rand eines Polytops beschrieben durch seinen planeren Oberflächengraph
(bestehend aus $v$ Ecken, $e$ Kanten und $f$ Facetten).
Nach dem Eulerschen Polyedersatz gilt $v - e + f = 2$.
Ein Polytop/Polyeder heißt \begriff{simpliziell}, falls jede Facette ein Dreieck ist,
und \begriff{simpel}, falls jeder Knoten Grad 3 hat.

\linie

\textbf{duale Hyperebene/dualer Halbraum}:
Sei $p \in \real^d$ ein Punkt.\\
Dann ist $\D(p) := \{x \in \real^d \;|\; \sp{x, p} = 1\}$ die \begriff{duale Hyperebene} und\\
$\H(p) := \{x \in \real^d \;|\; \sp{x, p} \le 1\}$ der \begriff{duale Halbraum}.

\textbf{dualer Halbraumschnitt}:
Sei $P$ ein Polytop in $\real^d$ mit $0 \in \interior(P)$.\\
Dann ist $P^\ast := \bigcap_{p \in P} \H(p) = \bigcap_{\text{$p$ Ecke von $P$}} \H(p)$
der \begriff{duale Halbraumschnitt}.

Die Voraussetzung $0 \in \interior(P)$ benötigt man, damit der Halbraumschnitt beschränkt ist.

\linie

\textbf{Bijektion}:
Sei $S \subset \real^d$ eine Punktmenge und $P := \CH(S)$ mit $0 \in \interior(P)$.\\
Dann gibt es eine Bijektion zwischen
der Menge aller Facetten von $P$ und
der Menge aller Facetten von $P^\ast$ (jeweils von den Dimensionen $0, \dotsc, d - 1$),
sodass $k$-dimensionale Facetten von $P$ auf $(d - k - 1)$-dimensionale Facetten von $P^\ast$
abgebildet werden ($k = 0, \dotsc, d - 1$).

Genauer gilt:
Falls eine $(k + 1)$-elementige Teilmenge $F \subset S$ eine $k$-dimensionale Facette
von $P$ aufspannt,
dann bildet der Schnitt der $(k + 1)$-vielen dualen Halbräume eine $(d - k - 1)$-dimensionale
Facette des Halbraumschnitts $P^\ast$.

Letzteres zeigt man, indem man beweist, dass das $(d - k - 1)$-dimensionale Schnittobjekt in
allen dualen Halbräumen liegt.

\linie

\textbf{zusammenhängende Probleme}:
\begin{enumerate}
    \item
    Gegeben ist $S \subset \real^d$ mit $|S| = n$.
    Berechne $\CH(S)$.
    
    \item
    Gegeben ist eine Menge $\H$ von $n$ Halbräumen in $\real^d$.
    Berechne $\bigcap_{h \in \H} h$.
\end{enumerate}

\textbf{Reduktion von \emph{(1)} auf \emph{(2)}}:
Um \emph{(1)} mithilfe von \emph{(2)} zu lösen, geht man wie folgt vor.
\begin{enumerate}
    \item
    Verschiebe zunächst $S$, sodass oBdA $0 \in \interior(\CH(S))$ gilt
    (z.\,B. um $-\frac{1}{n} \sum_{p \in S} p$).
    
    \item
    Wende die Dualitätsabbildung an, d.\,h. berechne $\H(p)$ für $p \in S$.
    
    \item
    Berechne $\bigcap_{p \in S} \H(p)$.
    
    \item
    Wende die inverse Dualitätsabbildung an und mache evtl. die Verschiebung rückgängig.
\end{enumerate}

\textbf{Umkehrung}:
Die Umkehrung ist nicht so einfach, weil nicht jede beliebige Halbraumschnitt-Instanz zu einem
CH-Problem dual ist (leer, unbeschränkt und $0$ nicht im Schnitt möglich).
Damit $0$ im Schnitt ist, muss man einen Punkt im Halbraumschnitt kennen, was i.\,A.
nicht der Fall ist.

Weil i.\,A. die Komplexität zur Beschreibung der konvexen Hülle von $S$ gleich
$\Theta(n^{\lfloor d/2 \rfloor})$ ist und daher auch die Komplexität von Halbraumschnitten
exponentiell wächst, ist man oft nicht in einer vollständigen Beschreibung des Halbraumschnitts
interessiert, sondern nur in einem Extrempunkt des Polyeders.
Diesen erhält man mit linearer Programmierung.

\pagebreak
