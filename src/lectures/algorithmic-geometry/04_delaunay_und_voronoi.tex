\chapter{%
    \name{Delaunay}-Triangulierungen und \name{Voronoi}-Diagramme%
}

\section{%
    \name{Delaunay}-Triangulierungen%
}

\textbf{Motivation}:
Gegeben sei eine sehr dünne Raute, in der die Diagonalen stark unterschiedliche Längen besitzen.
Angenommen, an jeder der vier Ecken sei eine Höhe gegeben.
Gesucht ist eine Triangulierung der Raute, sodass man dem Mittelpunkt (Schnittpunkt der Diagonalen)
z.\,B. durch lineare Interpolation eine Höhe zuweisen kann.
Nimmt man an, dass sich die Höhe in einer kleinen Entfernung auch nur wenig ändert,
dann erscheint die Triangulierung mit der kurzen Diagonalen natürlicher als die mit der langen,
denn bei der langen Diagonalen berechnet sich die Höhe des Mittelpunkts aus zwei sehr
weit voneinander entfernten Eckpunkten.

Auf"|fällig ist, dass der Innenwinkel bei "`natürlicheren"' Triangulierung mit der kurzen
Diagonalen doppelt so groß ist wie bei der anderen Triangulierung.
Um auf kanonische Weise eine "`beste"' Triangulierung für eine gegebene Punktmenge zu definieren,
ist ein sinnvolles Ziel, die Triangulierung zu finden, die den Vektor lexikografisch maximiert,
der aufsteigend sortiert alle Innenwinkel der Dreiecke der Triangulierung enthält.
%(d.\,h. $(\alpha_1, \alpha_2, \dotsc) > (\beta_1, \beta_2, \dotsc)$, falls
%$(\alpha_1 > \beta_1) \lor (\alpha_1 = \beta_1 \land \alpha_2 > \beta_2) \lor \dotsb$).

Die Delaunay-Triangulierung ist die in diesem Sinne beste Triangulierung.

\linie

\textbf{Umkreis}:
Sei $T$ ein nicht-entartetes Dreieck in $\real^2$.\\
Dann heißt das Innere des Kreises durch die Eckpunkte von $T$
\begriff{Umkreis} $\cc(T)$ von $T$.

\textbf{\name{Delaunay}-Triangulierung}:
Sei $P \subset \real^2$ eine endliche Punktmenge.
Eine Triangulierung $\T$ von $P$ heißt \begriff{\name{Delaunay}-Triangulierung} $\DT(P)$,
falls $\forall_{T \in \T}\; \cc(T) \cap P = \emptyset$.

\linie

\textbf{Fragen}:
\begin{itemize}
    \item
    Existiert eine eindeutige Delaunay-Triangulierung für jede Punktmenge $P \subset \real^2$?

    \item
    Falls ja, wie berechnet man die Delaunay-Triangulierung?

    \item
    Warum maximiert die Delaunay-Triangulierung den kleinsten Innenwinkel?
\end{itemize}

\section{%
    Lifting-Abbildung%
}

\textbf{Lifting-Abbildung}:
$'\colon \real^2 \rightarrow \real^3,
(p_x, p_y) =: p \mapsto p' := (p_x, p_y, p_x^2 + p_y^2)$
heißt \begriff{Lifting-Abbildung}.

\linie

\begin{floatingfigure}[r]{65mm}
    \footnotesize$
    \begin{vmatrix}1 & a'_x & a'_y & a'_z\\1 & b'_x & b'_y & b'_z\\
    1 & c'_x & c'_y & c'_z\\1 & p'_x & p'_y & p'_z\end{vmatrix} =
    \begin{vmatrix}1 & a_x & a_y & a_x^2 + a_y^2\\1 & b_x & b_y & b_x^2 + b_y^2\\
    1 & c_x & c_y & c_x^2 + c_y^2\\1 & p_x & p_y & p_x^2 + p_y^2\end{vmatrix}$
\end{floatingfigure}

\textbf{Lemma (Lokalisierung im Umkreis)}:\\
Seien $a, b, c \in \real^2$ drei nicht-kollineare Punkte, $p \in \real^2$.
Dann gilt $p \in \cc(\triangle abc)$ genau dann, wenn $p'$ unterhalb der Ebene durch $a', b', c'$
liegt.\\
Auf welcher Seite $p'$ bzgl. der Ebene durch $a', b', c'$ liegt,\\
kann mit dem Vorzeichen der Determinanten rechts bestimmt werden.

\linie

\textbf{Lemma (Zusammenhang DT -- CH)}:
%Drei paarweise verschiedene Punkte $p, q, r \in P$ bilden ein Dreieck der Delaunay-Triangulierung
%genau dann, wenn $p', q', r' \in \real^3$ eine Facette der konvexen Hülle von
%$P' := \{p' \;|\; p \in P\}$ bilden.
Es gilt $\triangle pqr \in \DT(P) \iff \triangle p'q'r' \in \partial(\CH(P'))$
(wobei $\partial(\CH(P'))$ die oberste Facette nicht enthalten soll).

\begin{Beweis}
    Für $\triangle pqr \in \DT(P)$ und $s \in P \setminus \{p, q, r\}$ beliebig gilt
    $s \notin \cc(\triangle pqr)$, d.\,h. nach obigem Lemma
    liegt $s$ oberhalb der Ebene durch $p', q', r'$.
    Damit liegen alle Punkte in $P \setminus \{p, q, r\}$ auf einer Seite und somit
    $\triangle p'q'r' \in \partial(\CH(P'))$.
    Die Umkehrung geht analog.
\end{Beweis}

Damit kann man $\DT(P)$ berechnen, indem man zunächst $\CH(P')$ in $\real^3$ berechnet
und anschließend alle Kanten auf die $x_1$-$x_2$-Ebene projiziert.
Insbesondere existiert $\DT(P)$, ist eindeutig
und kann in erwartet $\O(n \log n)$ Zeit konstruiert werden (RIC-Algorithmus für CH).

\section{%
    Lokale und globale \name{Delaunay}-Bedingung%
}

Im Folgenden ist $\T$ eine Triangulierung der Punktmenge $P \subset \real^2$.

\textbf{\name{Delaunay}-Dreieck}:
Ein Dreieck $T \in \T$ heißt \begriff{\name{Delaunay}-Dreieck}, falls $\cc(T) \cap P = \emptyset$.
%(\begriff{$T$ erfüllt die Delaunay-Bedingung}).

\textbf{\name{Delaunay}-Kante}:
Eine Kante $e = pq$ in $\T$ heißt \begriff{\name{Delaunay}-Kante}, falls es einen Kreis $C$ gibt
mit $p, q \in \partial C$ und $\interior(C) \cap P = \emptyset$.
%(\begriff{$e = pq$ erfüllt die globale Delaunay-Bedingung}).

\textbf{lokale \name{Delaunay}-Kante}:
Eine Kante $e = pq$ in $\T$ heißt \begriff{lokale \name{Delaunay}-Kante}, falls
\begin{itemize}
    \item
    $e$ Kante nur eines einzigen Dreiecks ist
    (d.\,h. $e$ liegt auf dem Rand von $\CH(P)$) oder

    \item
    $e$ Kante zweier Dreiecke $\triangle psq, \triangle pqr \in \T$ ist sowie
    $r \notin \cc(\triangle psq)$ und $s \notin \cc(\triangle pqr)$.
\end{itemize}

\linie

\textbf{Lemma (\name{Delaunay}-Lemma)}:
Folgendes ist äquivalent.
\begin{enumerate}
    \item
    Jedes Dreieck in $\T$ ist ein Delaunay-Dreieck (d.\,h. $\T = \DT(P)$).

    \item
    Jede Kante in $\T$ ist eine Delaunay-Kante.

    \item
    Jede Kante in $\T$ ist eine lokale Delaunay-Kante.
\end{enumerate}

\begin{Beweis}
    "`\emph{(1)} $\implies$ \emph{(2)}"':
    Sei $e = pq$ eine Kante in $\T$.
    Wähle ein Dreieck $T \in \T$, sodass $e$ eine Seite von $T$ ist.
    Dann gilt für $C := \cc(T)$, dass $p, q \in \partial C$ und $\interior(C) \cap P = \emptyset$.

    "`\emph{(2)} $\implies$ \emph{(3)}"':
    Sei $e = pq$ eine Kante zweier Dreiecke $\triangle psq, \triangle pqr \in \T$
    und $C$ ein Kreis mit $p, q \in \partial C$ und $\interior(C) \cap P = \emptyset$.
    Man kann sich klar machen, dass $C$ vollständig in der Vereinigung
    $\cc(\triangle psq) \cup \cc(\triangle pqr)$ liegt, weil sonst $s \in C$ oder $r \in C$ gilt.
    Außerdem gilt $\cc(\triangle psq) \cap \cc(\triangle pqr) \subset C$.
    Damit kann $C$ so innerhalb von $\cc(\triangle psq) \cup \cc(\triangle pqr)$ "`verschoben"'
    werden, sodass zusätzlich zu $p, q \in \partial C$ auch noch $r \in \partial C$ gilt
    (ohne dass $s \in C$), d.\,h. dann gilt $C = \cc(\triangle pqr)$ und $s \notin C$.
    Analog geht das mit $C = \cc(\triangle psq)$ und $r \notin C$.

    "`\emph{(3)} $\implies$ \emph{(1)}"':
    Angenommen, alle Kanten sind lokale Delaunay-Kanten, aber es gibt ein Dreieck $\triangle pqr$
    und ein Punkt $s \in P$ mit $s \in \cc(\triangle pqr)$.
    $s$ sei oBdA in dem Kreissegment, das durch die Kante $pr$ begrenzt wird.
    Betrachte die Strecke zwischen $s$ und irgendeinem Punkt auf $pr$ und
    alle Dreiecke zwischen $pr$ und $s$ auf dieser Strecke.
    Man wird im Folgenden argumentieren, dass $s$ in den Umkreisen aller dieser Dreiecke liegt.
    Insbesondere liegt $s$ dann auch im Umkreis des "`vorletzten"' Dreiecks $\triangle tuv$
    (teilt mit einem Dreieck mit $s$ als Ecke eine gemeinsame Kante $vu$),
    d.\,h. $vu$ ist keine lokale Delaunay-Kante, ein Widerspruch.

    Dazu "`verformt"' man den Umkreis von $\triangle pqr$, sodass $p$ und $r$ immer noch auf dem
    Kreis liegen, aber statt $q$ nun der Eckpunkt $t$ des nächsten Dreiecks auf dem Kreis liegt
    (man verschiebt den Mittelpunkt des Kreises auf der Mittelsenkrechten von $pr$ solange
    Richtung $s$, bis $t$ auf dem Kreis liegt).
    Weil der Mittelpunkt Richtung $s$ verschoben wird, wird der Kreis in Richtung von $s$ nur
    größer, d.\,h. $s$ liegt auch im Umkreis von $\cc(\triangle prt)$.
    Induktiv erhält man damit, dass $s$ in den Umkreisen aller Dreiecke zwischen $s$ und $pr$
    liegt.
\end{Beweis}

\linie

\textbf{Anwendung}:
Das Delaunay-Lemma erlaubt es, in $\O(n)$ Zeit zu überprüfen, ob eine gegebene Triangulierung
eine Delaunay-Triangulierung ist, indem alle
$\O(n)$ Kanten auf die lokale Delaunay-Bedingung überprüft werden
(was jeweils in $\O(1)$ Zeit geht,
im Gegensatz dazu, die Dreiecke auf die Delaunay-Bedingung zu prüfen).
Außerdem motiviert das Lemma den Delaunay-Flip-Algorithmus, der vom Typ "`lokale Suche"' ist
(versuche in jedem Schritt, lokal besser zu werden, um irgendwann die "`beste"' Lösung zu
erreichen).

\pagebreak

\section{%
    \name{Delaunay}-Flip-Algorithmus%
}

\textbf{\name{Delaunay}-Flip-Algorithmus}:
Der \begriff{\name{Delaunay}-Flip-Algorithmus} berechnet die Delaunay-Triangulierung $\DT(P)$
für eine Punktmenge $P \subset \real^2$ mit $n := |P|$ wie folgt.
\begin{enumerate}
    \item
    Berechne eine beliebige Triangulierung von $P$.

    \item
    Wiederhole, solange es eine Kante $e = pr$ gibt, die keine lokale Delaunay-Kante ist:
    \begin{enumerate}
        \item
        "`Flippe"' $e = pr$,
        d.\,h. liegt die Kante an die Dreiecke $\triangle pqr$ und $\triangle prs$ an,
        dann ersetze $pr$ durch $qs$.
    \end{enumerate}
\end{enumerate}

\linie

\textbf{Lemma (Korrektheit)}:
Sei $e = pr$ eine Kante in $\T$ mit zwei anliegenden Dreiecken $\triangle pqr$ und $\triangle prs$.
Ist $e$ keine lokale Delaunay-Kante, dann kann sie geflippt werden
(d.\,h. das Viereck $pqrs$ ist konvex)
und die neu erstellte Kante $qs$ ist eine lokale Delaunay-Kante.

\begin{Beweis}
    Sei oBdA $s \in \cc(\triangle pqr)$.
    Das Viereck $pqrs$ ist konvex, weil alle Punkte $p, q, r, s$ im Umkreis $\cc(\triangle pqr)$
    enthalten sind, d.\,h. alle Innenwinkel sind kleiner als $\pi$
    (das würde nicht gehen, wenn $e$ eine lokale Delaunay-Kante wäre).
    Damit liegt die Diagonale $qs$ vollständig im Viereck und die Kante $e = pr$ kann geflippt
    werden.

    Die neue Kante $qs$ ist eine lokale Delaunay-Kante, weil $\cc(\triangle pqr)$ zu
    $\cc(\triangle qrs)$ deformiert werden kann, indem $q, r$ auf dem Rand gehalten werden,
    während der Kreis schrumpft.
    Dadurch fällt $p$ automatisch aus dem Kreis heraus, d.\,h. $p \notin \cc(\triangle qrs)$.
    Analog zeigt man $r \notin \cc(\triangle pqs)$.
\end{Beweis}

\linie

\textbf{Lemma (Flip vergrößert min. Winkel)}:
Durch einen Flip einer Kante, die keine lokale Delaunay-Kante ist,
vergrößert sich der minimale Innenwinkel der beiden Dreiecke.

\textbf{Lemma (Flip-Algorithmus terminiert)}:
Der Delaunay-Flip-Algorithmus terminiert.

\begin{Beweis}
    Ordne jeder Triangulierung den aufsteigend sortierten Vektor der Innenwinkel aller Dreiecke zu.
    Nach dem Lemma von eben führt ein Flip zu einem lexikografisch größeren Vektor.
    Weil jede Triangulierung von $P$ gleich viele Dreiecke (und Kanten) besitzt,
    gibt es nur endlich viele Triangulierungen von $P$, d.\,h.
    der Flip-Algorithmus terminiert spätestens, wenn der lexikografisch größte Vektor
    erreicht ist.
\end{Beweis}

\textbf{Lemma ($\DT(P)$ maximiert Innenwinkel)}:
$\DT(P)$ maximiert den aufstetigend sortierten Vektor der Innenwinkel aller
Dreiecke unter allen möglichen Triangulierungen von $P$.

\begin{Beweis}
    Angenommen, es gibt eine Triangulierung $\T$, die zwar den Innenwinkel-Vektor maximiert,
    aber nicht die Delaunay-Triangulierung ist.
    Dann gibt es eine Kante, die keine lokale Delaunay-Kante ist, d.\,h.
    diese Kante kann geflippt werden.
    Nach obigem Lemma vergrößert sich dabei der minimale Innenwinkel der beteiligten Dreiecke,
    d.\,h. die neue Triangulierung $\T'$ wäre lexikografisch größer als $\T$, ein Widerspruch.
\end{Beweis}

\textbf{Lemma (Spezialfall)}:
Sei $e = pr$ eine Kante, die an die Dreiecke $\triangle pqr$ und $\triangle prs$ anliegt.\\
Sind $p, q, r, s$ \begriff{kozirkulär} (d.\,h. $s \in \partial(\cc(\triangle pqr))$),
dann ändert ein Flip den minimalen Innenwinkel nicht.

\begin{Beweis}
    Das Lemma folgt aus dem Peripheriewinkel-Satz:
    Ist ein Kreis mit einer Sehne $ab$ gegeben und wählt man einen dritten Punkt $c$ auf dem Kreis,
    dann ist der Winkel $\sphericalangle acb$ unabhängig von der Wahl von $c$.
    Daraus folgt, dass es vier Winkel $\alpha, \beta, \gamma, \delta$ gibt,
    sodass $\alpha, \beta, \gamma, \delta, \alpha + \delta, \beta + \gamma$
    die Innenwinkel vor dem Flip sind und
    $\alpha, \beta, \gamma, \delta, \alpha + \beta, \gamma + \delta$
    die Innenwinkel nach dem Flip.
    Wegen $\alpha + \delta > \alpha$ usw.
    muss der kleinste Innenwinkel vor und nach dem Flip in $\{\alpha, \beta, \gamma, \delta\}$
    enthalten sein, d.\,h. er ändert sich nicht.
\end{Beweis}

\pagebreak

\section{%
    Ef"|fiziente Implementierung des Flip-Algorithmus%
}

\textbf{ef"|fiziente Implementierung des Flip-Algorithmus}:
\begin{enumerate}
    \item
    Berechne eine beliebige Triangulierung z.\,B. mittels eines Sweepline-Algorithmus in Zeit
    $\O(n \log n)$
    (oder konstruiere zunächst aus der Punktmenge wie im Graham-Scan-Algorith"-mus
    für konvexe Hüllen ein Polygon und trianguliere dieses dann, beides geht in Zeit
    $\O(n \log n)$).

    \item
    Erstelle einen Stack und füge alle Kanten der Triangulierung hinzu.

    \item
    Solange es eine Kante $e$ im Stack gibt, wiederhole:
    \begin{itemize}
        \item
        Ist $e$ eine lokale Delaunay-Kante, dann entferne $e$ vom Stack.

        \item
        Ist $e$ keine lokale Delaunay-Kante, dann entferne $e$ vom Stack,
        ersetze $e$ durch die geflippte Kante $e'$ in der Triangulierung und
        füge die äußeren Kanten $e_1, \dotsc, e_4$ des Vierecks zum Stack hinzu,
        das durch die beiden zu $e$ adjazenten Dreiecke gebildet wird.
    \end{itemize}
\end{enumerate}

\textbf{Zeitbedarf}: $\O(\text{\#Flips} + n\log n)$

\linie

\textbf{Lemma}:
Die Anzahl der Flips ist $\O(n^2)$.

\begin{Beweis}
    Betrachte die Abbildung $h\colon \CH(P) \rightarrow \real_0^+$,
    die jedem Punkt $p$ in der konvexen
    Hülle der Punktmenge $P$ die "`Höhe"' $h(p)$ von $\CH(P')$ über $p$ zuweist\\
    (d.\,h. $h(p) := \min\{z \ge 0 \;|\; (p_x, p_y, z) \in \CH(P')\}$).\\
    Dann ist $h(p)$ punktweise für alle $p \in \CH(P)$ während des Flip-Algorithmus monoton
    fallend, denn bei Flips werden "`Dächer"'
    (Vierecke in $\real^3$ mit hoher Diagonale)
    zu "`Tälern"' (Vierecke in $\real^3$ mit niedriger Diagonale)
    aufgrund des Lifting-Lemmas bzgl. Umkreisen.

    Eine bereits geflippte Kante kann also niemals wieder auf dem Stack auftauchen
    und wieder die lokale Delaunay-Eigenschaft verletzen,
    d.\,h. es wurden insgesamt höchstens $\binom{n}{2}$ Kanten dem Stack hinzugefügt,
    die geflippt werden müssen.
\end{Beweis}

\textbf{Lemma}:
Im Worst-Case ist die Anzahl der Flips $\Omega(n^2)$.

\linie

Die Laufzeit des Flip-Algorithmus ist damit $\O(n^2)$
(scharfe Schranke, d.\,h. im Worst-Case ist die Laufzeit $\Theta(n^2)$).
Daraus folgt insbesondere, dass sich zwei Triangulierungen in $\O(n^2)$ Zeit ineinander
überführen lassen (über die Delaunay-Triangulie"-rung).

\pagebreak

\section{%
    RIC-Algorithmus%
}

Man nimmt an, dass die Punkte in einem großen Dreieck liegen.
Im $i$-ten Schritt ist $\DT_i$
die Delaunay-Triangulierung der Punkte $p_1, \dotsc, p_i$.
Der Algorithmus terminiert, da der Flip-Algorithmus terminiert.

\textbf{RIC-Algorithmus für DT}:
Der \begriff{RIC-Algorithmus} berechnet die Delaunay-Triangulierung $\DT(P)$
für eine Punktmenge $P \subset \real^2$ mit $n := |P|$ wie folgt.
\begin{enumerate}
    \item
    Permutiere die Punktmenge $P$ zufällig zu $p_1, \dotsc, p_n$.

    \item
    Konstruiere die Triangulierung $\DT_1$ durch Verbindung von $p_1$ mit den Ecken des
    großen Dreiecks.

    \item
    Wiederhole für $i = 1, \dotsc, n - 1$:
    \begin{enumerate}
        \item
        Lokalisiere $p_{i+1}$ in $\DT_i$, d.\,h. finde $T \in \DT_i$ mit $p_{i+1} \in T$.

        \item
        Verbinde $p_{i+1}$ mit den Ecken von $T$.

        \item
        Wende den Delaunay-Flip-Algorithmus auf die entstehende Triangulierung an,
        um $\DT_{i+1}$ zu erhalten.
    \end{enumerate}
\end{enumerate}

\linie

\textbf{Lemma (lokale Delaunay-Eigenschaft direkt nach Einfügung)}:
Die drei neuen Kanten zwischen $p_{i+1}$ und den Ecken von $\DT_i$
sind direkt nach der Einfügung lokale Delaunay-Kanten.
Die einzigen Kanten, die zunächst evtl. die lokale Delaunay-Eigenschaft verloren haben,
sind die Kanten von $T$.

\begin{Beweis}
    Die zu den neuen Kanten gehörigen Vierecke sind nicht konvex
    und daher lokale Delaunay-Kanten (sonst könnte man sie evtl. flippen, was nur geht, wenn
    die Vierecke konvex sind).
    Die einzigen Vierecke direkt nach Einfügung von $p_{i+1}$, die nicht schon in $\DT_i$ waren,
    sind die, die zu den Kanten von $T$ gehören,
    d.\,h. nur diese Kanten können die lokale Delaunay-Eigenschaft verloren haben.
\end{Beweis}

\textbf{Lemma (zerstörte und neue Dreiecke)}:
Die Dreiecke in $\DT_i \setminus \DT_{i+1}$
(die Dreiecke, die im $i$-ten Schritt zerstört wurden)
sind genau die, die $p_{i+1}$ im Umkreis enthalten.\\
Die Dreiecke in $\DT_{i+1} \setminus \DT_i$
(die Dreiecke, die im $i$-ten Schritt erstellt wurden)
sind genau die, die $p_{i+1}$ als eine Ecke besitzen.

\begin{Beweis}
    Wird $p_{i+1}'$ in die konvexe Hülle von $\{p_1', \dotsc, p_i'\}$ eingefügt,
    dann ändert sich diese nur adjazent zu $p_{i+1}'$
    (jede neue Facette hat $p_{i+1}'$ als Eckpunkt),
    d.\,h. mit jedem Flip entsteht eine Kante adjazent zu $p_{i+1}$.
    Damit besitzen neue Dreiecke $p_{i+1}$ als eine Ecke.
    Umgekehrt sind Dreiecke mit Ecke $p_{i+1}$ neue Dreiecke, weil $p_{i+1}$ vorher nicht in der
    Punktmenge war.

    Analog argumentiert man, dass genau die Facetten von $\CH(\{p_1', \dotsc, p_i'\})$
    zerstört werden, bei denen $p_{i+1}'$ unterhalb der Ebene durch die jeweilige Facette liegt.
    Damit werden im $i$-ten Schritt genau die Dreiecke zerstört, in deren Umkreis $p_{i+1}$
    liegt.
\end{Beweis}

\textbf{Lemma}:
Die erwarteten Flipkosten für das Einfügen von $p_{i+1}$ sind $\O(1)$.

\begin{Beweis}
    Die Anzahl der Flips ist $\O(\text{$\deg(p_{i+1})$ nach fertiger Einfügung})$
    nach dem Lemma von eben.
    Der erwartete Grad eines zufälligen Knotens in einem planeren Graph mit $i + 1$ Knoten ist
    kleiner als $6$ (lässt sich aus Euler folgern).
    Damit sind die Flipkosten erwartet $\O(1)$.
\end{Beweis}

\linie
\pagebreak

Um die Konstruktion der Delaunay-Triangulierung mittels des RIC-Algorithmus zu verschnellern,
wird während des Algorithmus eine Suchstruktur aufgebaut, mit der ef"|fizient $p_{i+1}$ in
$\DT_i$ lokalisiert werden kann.

\textbf{Suchstruktur für RIC-Algorithmus}:
Der RIC-Algorithmus verwaltet einen \begriff{gerichteten, azyklischen Graphen (DAG)}
und erweitert diesen in jedem Schritt.
Knoten des Graphen entsprechen dabei Dreiecken wie folgt:
\begin{itemize}
    \item
    Senken (Sinks, Knoten mit Ausgangsgrad $0$) entsprechen aktuell existierenden Dreiecken.

    \item
    Innere Knoten entsprechen Dreiecken, die schon zerstört wurden.
\end{itemize}
Während des RIC-Algorithmus wird der DAG wie folgt geändert:
\begin{itemize}
    \item
    \begriff{Split}
    (Verbindung von $p_{i+1}$ mit den Ecken von $T \in \DT_i$ mit $p_{i+1} \in T$):\\
    Erstelle drei neue Kindknoten unter dem Knoten, der zu $T_i$ gehört.

    \item
    \begriff{Flip}
    (Flip einer Kante während des Flip-Algorithmus):\\
    Bezeichnen $1, 2$ die alten Dreiecke und $3, 4$ die neuen,
    dann erstelle zwei Kindknoten und verbinde beide mit den beiden Knoten, die zu $1, 2$ gehören.
\end{itemize}

Zur Punktlokalisierung geht man wie bei der Kirkpatrick-Hierarchie "`von oben nach unten"' vor,
d.\,h. für den aktuellen Knoten entscheidet man, in welchem Kindknoten sich $p_{i+1}$ befindet.

\linie

Die benötigte Zeit im $i$-ten Schritt ist die Summe des Aufwands der Lokalisierung von $p_{i+1}$
in $\DT_i$ und des Flip-Algorithmus.
Letztere benötigt erwartet nur $\O(1)$ Zeit.
Weil jeder DAG-Knoten $\le 3$ und damit $\O(1)$ viele Kinder besitzt, ist der Aufwand einer
Punktlokalisierung linear in der Länge des Suchpfads, d.\,h. linear in der Anzahl an
Dreiecken, die irgendwann
(auch während des Flip-Algorithmus) mal existierten und $p_{i+1}$ enthalten.

\textbf{Lemma}:
Die Kosten der Lokalisierung von $p_{i+1}$ in $\DT_i$ ist höchstens linear in der Anzahl an
(verschiedenen) Delaunay-Dreiecken, die irgendwann mal existierten und $p_{i+1}$ im Umkreis
enthalten.

\textbf{Satz (Gesamt-Laufzeit der Punktlokalisierungen)}:\\
Die erwartete Gesamt-Laufzeit der Punktlokalisierungen ist $\O(n \log n)$.

\begin{Beweis}
    Für jedes Delaunay-Dreieck $T$ in einer der $\DT_i$ sei
    $k(T) := |\cc(T) \cap P|$ die Anzahl der Punkte in seinem Umkreis.
    Nach dem Lemma ist der gesamte  Punktlokali"-sierungs-Aufwand beschränkt durch
    $\O(\sum_{T \in \bigcup_{i=1}^n \DT_i} k(T))$.

    Zunächst analysiert man $\EE[\sum_{T \in \DT_i \setminus \DT_{i-1}} k(T)]$,
    d.\,h. man betrachtet nur die Delaunay-Drei"-ecke, die im $(i-1)$-ten Schritt entstanden sind.
    Die Dreiecke in $\DT_i \setminus \DT_{i-1}$ sind genau die Dreiecke
    in $\DT_i$, die $p_i$ als eine Ecke besitzen (siehe Lemma oben).
    Weil $p_i$ eine
    zufällige Ecke von $\DT_i$ ist und jedes Dreieck in $\DT_i$ drei Ecken besitzt, gilt\\
    $\EE[\sum_{T \in \DT_i \setminus \DT_{i-1}} k(T)]
    = \EE[\sum_{T \in \DT_i,\; \text{$p_i$ Ecke von $T$}} k(T)]
    = \frac{3}{i} \EE[\sum_{T \in \DT_i} k(T)]$.

    Auf der anderen Seite sind die Dreiecke in $\DT_i \setminus \DT_{i+1}$ genau die Dreiecke
    in $\DT_i$, die $p_{i+1}$ in ihrem Umkreis haben.
    Weil $p_{i+1}$ ein zufälliger Punkt von $P \setminus \{p_1, \dotsc, p_i\}$ ist, gilt\\
    $\EE[|\DT_i \setminus \DT_{i+1}|] = \frac{1}{n-i} \EE[\sum_{T \in \DT_i} k(T)]$.
    Allerdings ist die Zahl der im $i$-ten Schritt zerstörten Dreiecke genau zwei weniger als
    die Anzahl der erstellten Dreiecke, Letzteres ist erwartet $\O(1)$ (siehe oben).
    Damit gilt
    $\EE[\sum_{T \in \DT_i} k(T)] = (n-i) \cdot \EE[|\DT_i \setminus \DT_{i+1}|] = \O(n - i)$.

    Man erhält also $\EE[\sum_{T \in \DT_i \setminus \DT_{i-1}} k(T)] = \O(\frac{n - i}{i})$ und
    somit als Gesamt-Laufzeit der Punktlokalisierungen
    $\sum_{i=2}^n \EE[\sum_{T \in \DT_i \setminus \DT_{i-1}} k(T)] =
    \sum_{i=2}^n \O(\frac{n-i}{i}) = \O(n \log n)$.
\end{Beweis}

\linie

\textbf{Zeitbedarf des RIC-Algorithmus}:
erwartet $\O(n \log n)$

\pagebreak

\section{%
    Divide-and-Conquer-Algorithmus%
}

\textbf{Divide-and-Conquer-Algorithmus}:
Der \begriff{Divide-and-Conquer-Algorithmus} berechnet die\\
Delaunay-Triangulierung $\DT(P)$ für eine Punktmenge $P \subset \real^2$ mit $n := |P|$ wie folgt.
\begin{enumerate}
    \item
    Enthält $P$ nur konstant viele Punkte, dann berechne die Delaunay-Triangulierung direkt
    und gebe diese zurück.

    \item
    \emph{Divide-Schritt}:
    Sonst berechne den $x$-Median der Punkte und
    teile anhand diesem $P$ in zwei Hälften $P_L$ und $P_R$,
    d.\,h. $|P_L| \approx |P_R|$
    (z.\,B. anhand des $x$-Medians in zwei Hälften).

    \item
    Berechne rekursiv $\DT(P_L)$ und $\DT(P_R)$.

    \item
    \emph{Conquer-Schritt}:
    Vereine $\DT(P_L)$ und $\DT(P_R)$ durch Löschen und Hinzufügen einiger Kanten
    zu $\DT(P)$.
\end{enumerate}

Das Problem dabei ist, dass man Schritt \emph{(4)} irgendwie so durchführen muss,
dass dabei nur Kosten von $\O(n)$ entstehen, wenn der Algorithmus die
Gesamtlaufzeit $\O(n \log n)$ besitzen soll.

\linie

\textbf{Beobachtungen}:
\begin{itemize}
    \item
    $\DT(P)$ enthält die obere und die untere Tangente an $\CH(P_L)$ und $\CH(P_R)$.

    \item
    Alle neuen Delaunay-Kanten und -Dreiecke haben mindestens einen Punkt in $P_L$ und
    einen in $P_R$.
\end{itemize}

\textbf{eine Möglichkeit für den Conquer-Schritt}:
Trianguliere den Bereich zwischen $\DT(P_L)$ und $\DT(P_R)$ beliebig
und führe anschließend den Flip-Algorithmus durch.
Allerdings ist nicht klar, ob und warum der Flip-Algorithmus nur $\O(n)$ Kanten flippen muss.

\linie

\textbf{besser}:
Nutze die Tatsache, dass alle neuen Kanten und Dreiecke die Teilungsgerade schneiden.
Finde die neuen Kanten und Dreiecke in der Reihenfolge, in der sie die Teilungsgerade schneiden,
wie folgt:
\begin{enumerate}
    \item
    Nimm an, dass keine zwei Punkte dieselbe $x$-Koordinate haben
    (sonst Rotation).

    \item
    Berechne die unterste neue Kante als die untere Kante der beiden Kanten auf dem Rand von
    $\CH(P)$, die die Teilungsgerade schneiden.
    Definiere einen Kreis durch die beiden Endpunkte $a \in P_L$ und $b \in P_R$ der Kante,
    der seinen Mittelpunkt sehr weit unten hat.

    \item
    Verschiebe den Mittelpunkt des letzten berechneten Kreises
    so lange nach oben auf der Mittelsenkrechten von $ab$,
    bis ein neuer Punkt $c$ aus $P$ auf dem Rand des Kreises liegt,
    wobei $a, b$ immer auf dem Rand des Kreises liegen sollen.

    \item
    $\triangle abc$ bildet ein Delaunay-Dreieck,
    verbinde also $c$ mit $a$ und mit $b$ und lösche Kanten, die $ac$ oder $ab$ schneiden.

    \item
    Für $c \in P_L$ setze $a \leftarrow c$ und für $c \in P_R$ setze $b \leftarrow c$.

    \item
    Wiederhole Schritte \emph{(3)} bis \emph{(5)} solange, bis $ab$ die obere Tangente ist
    (obere Kante der beiden Kanten auf dem Rand von $\CH(P)$, die die Teilungsgerade schneiden).
\end{enumerate}

Das Problem ist, dass dieser Algorithmus so nicht implementiert werden kann,
weil im Programm Kreise nicht kontinuierlich verschoben werden können.

\linie
\pagebreak

\textbf{Lemma}:
Sei $c_L$ der erste Punkt in $P_L$, der vom Kreis durch $a \in P_L$ und $b \in P_R$ getrof"|fen
wird.
Seien $b, n_1, n_2, \dotsc$ die Nachbarn von $a$ im Gegenuhrzeigersinn, startend mit $b$.\\
Ist $i$ der kleinste Index mit $b \notin \cc(\triangle a n_i n_{i+1})$,
dann ist $an_j$ für $j = 1, \dotsc, i - 1$ keine Delaunay-Kante und es gilt $c_L = n_i$.

\begin{Beweis}
    Für alle $j < i$ gilt $b \in \cc(\triangle a n_j n_{j+1})$ nach Voraussetzung.
    $n_{j+1}$ und $b$ liegen wegen der Nummerierung im Gegenuhrzeigersinn auf verschiedenen Seiten
    von $an_j$.
    Aus diesen beiden Aussagen folgt, dass jeder Kreis durch $a$ und $n_j$
    einen der beiden Punkte $b$ und $n_{j+1}$ enthält
    (geht der Kreis durch einen dritten Punkt $q$,
    so liegt $b$ im Kreis, wenn $q$ auf der Seite von $b$, aber außerhalb von
    $\cc(\triangle a n_j n_{j+1})$ liegt,
    oder wenn $q$ auf der Seite von $n_{j+1}$, aber innerhalb von $\cc(\triangle a n_j n_{j+1})$
    liegt).
    Somit ist $an_j$ keine Delaunay-Kante.

    Außerdem folgt, dass der Kreis mit dem sich verschiebenden Mittelpunkt auf der
    Mittelsenkrechten von $ab$ zuerst $n_{j+1}$ trifft und dann $n_j$,
    d.\,h. der Kreis trifft $n_i$ vor $n_j$ für $j < i$.
    Die Aussage $c_L = n_i$ ist damit äquivalent dazu, dass $n_i$ vor $n_j$ für $j > i$
    getrof"|fen wird.

    Sei also $j > i$ beliebig.
    Angenommen, $n_j$ wird vor $n_i$ getrof"|fen.
    Dann gilt $n_j \in \cc(\triangle abn_i)$
    (der Umkreis $\cc(\triangle abn_i)$ ist nach oben hin größer worden im Vergleich zum
    Zeitpunkt, als $n_j$ getrof"|fen wurde, und $n_j$ liegt über $ab$),
    aber $n_j$ liegt links von $an_i$
    ($j > i$ und Nummerierung im Gegenuhrzeigersinn).
    Dieser Teil des Kreises ist in $\cc(\triangle an_i n_{i+1})$ enthalten,
    ein Widerspruch, denn dieser Umkreis enthält keine Punkte aus $P_L$.
\end{Beweis}

\linie

\textbf{Implementierung}:
\begin{enumerate}
    \item
    Der Algorithmus betrachtet die Dreiecke $\triangle a n_j n_{j+1}$ für steigendes $j$ und
    führt Umkreis-Tests für diese Dreiecke und $b$ durch.
    Für $j = i_L$ stoppt die Suche.
    Das Ergebnis ist ein Punkt $c_L$, den der Kreis durch $a$ und $b$ zuerst von denen in $P_L$
    trifft, und der zugehörige Index $i_L$.
    Die Kanten $an_j$ für $j < i_L$ können entfernt werden, weil sie nach obigem Lemma keine
    Delaunay-Kanten sind.

    \item
    Dann führt der Algorithmus die Prozedur analog für $P_R$ durch, um $c_R$ und $i_R$ zu
    erhalten und bestimmte Kanten aus $\DT(P_R)$ zu entfernen.

    \item
    Schließlich wird ein weiterer Umkreis-Test durchgeführt, um zu bestimmen, ob $c_L$ oder $c_R$
    zuerst getrof"|fen wird.
    Wenn $c_L$ zuerst getrof"|fen wird, wird die Kante $bc_L$ hinzugefügt,
    sonst $ac_R$.

    \item
    In jedem Fall macht der Algorithmus mit der hinzugefügten Kante $bc_L$ oder $ac_R$
    anstelle von $ab$ weiter.
\end{enumerate}

\linie

\textbf{Zeitbedarf}:
$\O(n \log n)$

\begin{Beweis}
    Die Laufzeit des Conquer-Schritts ist linear in der Gesamtzahl $i_L + i_R$
    an "`Suchschritten"' auf beiden Seiten.
    Weil für jeden Suchschritt eine andere Kante in $\DT(P_L)$ oder $\DT(P_R)$ betrachtet wird,
    ist diese Zahl höchstens gleich der Anzahl $\O(n)$ an Kanten in beiden
    Delaunay-Triangulierungen,
    d.\,h. der Conquer-Schritt kostet $\O(n)$ Zeit.

    Bezeichnet $T(n)$ die Zeit, die der Divide-and-Conquer-Algorithmus zur Berechnung der
    Delau"-nay-Triangulierung von $n$ Punkten benötigt, so gilt damit
    $T(n) = 2T(\frac{n}{2}) + \O(n)$ (der Median kann direkt in $\O(n)$ berechnet werden,
    alternativ vorsortiert man die Punkte nach ihrer $x$-Koordinate und halbiert dann in jedem
    Divide-Schritt die Punktmenge in der Mitte).\\
    Mit dem Master-Theorem ergibt sich $T(n) = \O(n \log n)$.
\end{Beweis}

\pagebreak

\section{%
    \name{Voronoi}-Diagramme%
}

\textbf{Postamt-Problem}:
Gegeben sind $n$ Punkte (Postämter) in $P \subset \real^2$ und ein Punkt $q \in \real^2$.\\
Gesucht ist der Punkt $p \in P$ mit $\norm{p - q}$ minimal.

\linie

\textbf{\name{Voronoi}-Diagramm}:
Das \begriff{\name{Voronoi}-Diagramm} ist die Partition der Ebene
in Knoten, Kanten und $n$ Zellen,
sodass alle Punkte einer Zelle genau einem bestimmten Postamt am nächsten sind.
(Kanten sind dabei Schnitte der Abschlüsse benachbarter Zellen
und Knoten sind Schnitte der Abschlüsse benachbarter Kanten.)

\textbf{Lemma (\name{Voronoi}-Zellen)}:
Die Zellen sind konvex, polygonal berandet (aber evtl. unbegrenzt),
zusammenhängend und enthalten das zu ihnen assoziierte Postamt.

\begin{Beweis}
    Jede Zelle ist ein Schnitt von bestimmten of"|fenen Halbebenen.
    Diese Halbebenen sind konvex, damit ist der Schnitt ebenfalls konvex
    (und polygonal berandet).
    Als konvexe Mengen sind die Zellen zusammenhängend.
    Das zu einer Zelle assoziierte Postamt ist sich selbst am nächsten, weswegen
    es in der Zelle enthalten ist.
\end{Beweis}

\textbf{ef"|fiziente Lösung des Postamt-Problems mit \name{Voronoi}-Diagrammen}:
\begin{enumerate}
    \item
    Konstruiere das Voronoi-Diagramm von $P$.

    \item
    Trianguliere alle Zellen.

    \item
    Konstruiere Suchstruktur über der entstehenden Triangulierung (Kirkpatrick-Hierarchie).
\end{enumerate}

\textbf{Zeitbedarf für Abfrage}:
$\O(\log n)$

\linie

\textbf{Dualität zwischen \name{Delaunay}-Triangulierung und \name{Voronoi}-Diagramm}:
\begin{itemize}
    \item
    \emph{\name{Voronoi}-Knoten}
    sind charakterisiert durch drei Postämter, die dem Knoten am nächsten liegen
    und alle gleich weit vom Knoten entfernt sind,
    und entsprechen Kreisen durch diese drei Postämter, die keine Postämter enthalten,
    d.\,h. Delaunay-Dreiecken.\\
    Ein Voronoi-Knoten ist der Umkreis-Mittelpunkt des zugehörigen Delaunay-Dreiecks.

    \item
    \emph{\name{Voronoi}-Kanten}
    sind charakterisiert dadurch, dass für alle Punkte auf der Kante genau zwei bestimmte
    Postämter am nächsten sind,
    und entsprechen Kreisen durch diese zwei Postämter, die keine Postämter enthalten,
    d.\,h. Delaunay-Kanten.\\
    Eine Voronoi-Kante ist ein Teil der Mittelsenkrechten der zugehörigen Delaunay-Kante.
\end{itemize}
Die unbegrenzten Voronoi-Zellen und -Kanten entsprechen den Punkten und Delaunay-Kanten
auf dem Rand von $\CH(P)$.

\linie

\textbf{in drei Dimensionen}:
In $\real^3$ verbraucht das Voronoi-Diagramm $\O(n^2)$ Platz, d.\,h. Punktlokalisierung ist
ef"|fizient nicht möglich.
Man geht daher zu approximativer Lokalisierung über.

\linie

\textbf{Anwendungen von \name{Delaunay}-Triangulierung und \name{Voronoi}-Diagramm}:
\begin{itemize}
    \item
    Berechnung von \begriff{Minimum Spanning Trees (MSTs)} einer Punktmenge $P \subset \real^2$
    (vollständiger Graph mit $n$ Knoten, wobei die Kantengewichte gleich den euklidischen
    Abständen sind),
    weil man zeigen kann, dass die MST-Kanten eine Teilmenge der Delaunay-Triangulierung bilden
    (d.\,h. nur $\O(n)$ Konstruktionsaufwand)

    \item
    \begriff{Meshing} von Gebieten:
    oft mit Delaunay-Dreiecken, außerdem wird oft mit Umkreis-Mittelpunkten
    verfeinert

    \item
    \begriff{Kurvenrekonstruktion}:
    gegeben sind $n$ Punkte (Abtastung/Sampling)
    einer glatten, geschlossenen Kurve, aber ohne Reihenfolge,
    gesucht ist eine Approximation der Kurve,
    alle Ansätze arbeiten mit Delaunay-Triangulierungen oder Voronoi-Diagrammen
\end{itemize}

\pagebreak
\ifthenelse{\equal{\standalonedoc}{true}}{\addtocontents{toc}{\protect\newpage}}{}
