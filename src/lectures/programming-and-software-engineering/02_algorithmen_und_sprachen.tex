\section{%
    Algorithmen und Sprachen%
}

\subsection{%
    Darstellung von Algorithmen%
}

\begin{Def}{Algorithmus}
    ein Verfahren, das prinzipiell von einer mechanisch arbeitenden
    Maschine durchgeführt werden kann; \\
    exakt beschriebenes Verfahren inkl. genauer Festlegung von
    Eingabe/Ausgabe/Zwischen\-spei\-cherung von Daten usw., das Verfahren
    muss so genau ausformuliert sein, dass jeder ohne Rücksprache mit dem
    Autor den Algorithmus nachvollziehen kann
\end{Def}

\begin{Def}{Pseudosprache}
    \begin{itemize}
        \item \begriff{Algorithmus} erhält einen Bezeichner und ist Folge von
        Anweisungen
        
        \item \begriff{Variablen} werden in einem \pseudocode{declare}-Teil vor
        dem Algorithmus deklariert
        
        \item \begriff{elementare Anweisungen}: \pseudocode{skip}, \quad
        \pseudocode{x := a}, \quad
        \pseudocode{read(x)}, \quad \pseudocode{write(x)}, \quad
        \pseudocode{halt}, \quad
        \pseudocode{exit}, \quad
        \pseudocode{Alg(a, b, c)}
        
        \item \begriff{Ausdrücke}: arithmetische, logische oder
        Zeichenausdrücke
        
        \item \begriff{Hintereinanderausführung}/Trennung von Anweisungen
        mittels \pseudocode{;}
        
        \item \begriff{Fallunterscheidung}:
        \pseudocode{if foo then A [else B] fi}
        
        \item \begriff{Schleifen}: \pseudocode{while foo do A od}, \quad
        \pseudocode{repeat A until foo}, \\
        \pseudocode{for i := a [by x] to b do C od} \quad
        (\pseudocode{i}, \pseudocode{a}, \pseudocode{b}, \pseudocode{x}
        dürfen nicht verändert werden)
        
        \item \begriff{Kommentare} beginnen mit \pseudocode{{-}{-}}, ein
        Algorithmus hat die Form
        
\begin{lstlisting}[language=pseudosprache]
program <Name> is
declare
    <Deklarationen>;
begin
    <Anweisungen>
end
\end{lstlisting}
    \end{itemize}
\end{Def}

\begin{Def}{Ablaufprotokoll}
    Tabelle, die Spalten für Schrittnummer, ausgeführte Anweisung sowie
    alle Variablen nach Ausführung der Anweisung enthält (Ein-/Ausgabe werden
    gesondert notiert)
\end{Def}

\subsection{%
    Charakteristika von Algorithmen%
}

\begin{Def}{Eigenschaften}
    Algorithmus ist Vorschrift, die die Reihenfolge von Handlungen auf Daten
    beschreibt. Es muss gelten:
    \begin{itemize}
        \item Daten sind "`disket"' aufgebaut (mit endlich vielen
        digitalen Zeichen darstellbar)
        
        \item Operationen sind "`diskret"' aufgebaut
        
        \item Vorschrift ist eine endliche Folge von Operationen/wird
        schrittweise abgearbeitet
        
        \item eine Operation ist als Startoperation ausgezeichnet
        
        \item für jede Operation ist direkt nach der Ausführung bekannt,
        welches die möglichen (endlich vielen) Folgeoperationen sind oder ob
        der Algorithmus terminiert
        
        \item Eingabe ist eine Folge von Daten (auch unendlich oder leer)
        
        \item die bis zu jedem Zeitpunkt bearbeitete Menge an Daten
        und durchgeführten Operationen ist endlich
    \end{itemize}
\end{Def}

\begin{Def}{Determinismus}
    Ein Verfahren, bei dem nach Abarbeitung jeder Operation feststeht, welche
    Operation als nächste ausgeführt wird, heißt \begriff{deterministisch}. \\
    Sonst (falls mehrere Operationen alternativ zugelassen sind) heißt es
    \begriff{nicht-deterministisch}.
\end{Def}

\begin{Def}{Terminierung}
    Ein Algorithmus/Programm \begriff{terminiert für eine Eingabe $u$},
    wenn der Algorithmus bei Eingabe von $u$ nach endlich vielen Schritten
    anhält. \\
    Ein Algorithmus \begriff{terminiert stets}, wenn er für alle möglichen
    Eingaben terminiert.
\end{Def}

\subsection{%
    Unentscheidbare Probleme%
}

\begin{Satz}{Unlösbarkeit des Halteproblems}
    Es gibt keinen Algorithmus, der zu jedem beliebigen Algorithmus und jeder
    beliebigen Eingabe feststellen kann, ob dieser für diese Eingabe terminiert
    oder nicht (\begriff{Halteproblem}).
    
    Gäbe es nämlich einen solchen Algorithmus $H$, so könnte man auch einen
    Algorithmus $J'$ konstruieren, der einen Algorithmus $A$ übergeben bekommt,
    mit:
    Wenn $A$ bei Eingabe von $A$ selbst terminiert, so gehe in eine
    Endlosschleife, andernfalls verlasse den Algorithmus.
    
    Was passiert beim Aufruf von $J'$ mit $J'$? Würde $J'$ terminieren, so
    würde $J'$ in eine Endlosschleife gehen, also nicht terminieren.
    Würde $J'$  nicht terminieren, so würde $J'$ den Algorithmus verlassen,
    also terminieren.
    Widerspruch!
\end{Satz}

\subsection{%
    Grundlegende Datenbereiche%
}

\begin{Def}{Elementare Datentypen}
    Dazu gehören Boolean ($\mathbb{B}$), Natural ($\mathbb{N}_0$),
    Integer ($\mathbb{Z}$), Real ($\mathbb{R}$) und Character ($\mathbb{A}$).
\end{Def}

\begin{Def}{Darstellungen}
    \begin{itemize}
        \item natürliche Zahlen: verschiedene \begriff{Stellenwertsysteme}
        möglich \\
        (Dezimal-/Binär-/Oktal-/Hexadezimalsystem usw.)
        
        \item ganze Zahlen: \begriff{Zweierkomplementdarstellung}
        (erstes Bit Vorzeichen)
        
        \item rationale/reelle Zahlen: \begriff{Festkommadarstellung}
        (die letzten $x$~Bit sind Nachkommastellen),
        \begriff{Gleitkommadarstellung}
        ($z = m \cdot b^e$, $m$ Mantisse, $e$ Exponent bzgl. Basis $b$),
        Rundungsfehler
    \end{itemize}
\end{Def}

\subsection{%
    Realisierte Abbildung%
}

\begin{Def}{Realisierte Abbildung} $f_\pi: E \rightarrow A$ von der
    Eingabemenge $E$ in die Ausgabemenge $A$, \\
    $f_\pi(e) =
    \begin{cases}
        a & \text{falls } \pi \text{ bei Eingabe von } e
        \text{ mit der Ausgabe } a \text{ terminiert} \\
        \bot & \text{falls } \pi \text{ bei Eingabe von } e
        \text{ nicht terminiert}
    \end{cases}$
\end{Def}

\begin{Def}{Menge der berechenbaren Funktionen}\\
    $\mathcal{P}_{E,A} = \{f: E \rightarrow A \;|\; \text{es gibt ein
    Programm } \pi \text{ mit } f_\pi = f\}$
\end{Def}

\begin{Def}{freies Monoid über $\boldsymbol{M}$}
    $M^\ast = \{a_1 a_2 \ldots a_n \;|\;
    n \in \mathbb{N}_0,\; a_i \in M\}$ (\begriff{Menge der Wörter}), \\
    $\varepsilon$ leeres Wort (Wort der Länge $n=0$)
\end{Def}

\begin{Def}{Menge der von Programmen berechenbaren Funktionen}\\
    $\mathcal{P} = \{f \;|\; \text{es gibt ein Programm } \pi \text{ mit }
    f = f_\pi: D^\ast \rightarrow D^\ast\}$, wobei
    $D$ die Menge aller darstellbaren Boolean-, Zeichen-, Ganzzahl- und
    Gleitkommazahl-Werte ist.
\end{Def}

\subsection{%
    (Künstliche) Sprachen%
}

\begin{Def}{Alphabet}
    Eine endliche, linear geordnete Menge $A = \{a_1, \ldots, a_n\}$ \quad
    ($a_1 < \cdots < a_n$) heißt \begriff{(endliches) Alphabet}.
\end{Def}

\begin{Def}{Sprache}
    Jede Menge von Zeichenfolgen $L$ über $A$ heißt
    \begriff{Sprache über dem Alphabet $A$}, d.~h. $L$ ist Sprache über $A$
    genau dann, wenn $L \subseteq A^\ast$.
\end{Def}

\begin{Def}{Wort}
    Ein Element $w \in L \subseteq A^\ast$ einer Sprache heißt \begriff{Wort}.
\end{Def}

\begin{Def}{Operationen mit Sprachen}
    Vereinigung $L_1 \cup L_2$, Durchschnitt $L_1 \cap L_2$, \\
    Komplement $A^\ast \setminus L$,
    Konkatenation $L_1 \circ L_2 = \{uv \;|\;u \in L_1, v \in L_2\}$,
    Iteration $L^\ast = \bigcup_{i \in \mathbb{N}_0} L^i$, \\
    $L^+ = \bigcup_{i \in \mathbb{N}} L^i$ ($L^i = L \circ \cdots \circ L$),
    Ergebnis ist wieder eine Sprache
\end{Def}

\subsection{%
    Grammatiken%
}

\begin{Def}{Kontextfreie Grammatik}
    Viertupel $G = (V, \Sigma, P, S)$ mit
    \begin{itemize}
        \item $V$ nicht-leere endliche Menge
        (\begriff{Nichtterminalzeichen}),
        
        \item $\Sigma$ nicht-leere endliche Menge mit
        $V \cap \Sigma = \emptyset$ (\begriff{Terminalzeichen}),
        
        \item $P \subset V \times (V \cup \Sigma)^\ast$ endliche Menge
        (\begriff{Regeln} oder \begriff{Produktionen}),
        
        \item $S \in V$ (\begriff{Startsymbol}).
    \end{itemize}
    
    Sei z.~B. $G_1 = (V_1, \Sigma_1, P_1, S_1)$ mit $V_1 = \{S_1\}$ und
    $\Sigma_1 = \{0, 1\}$. Man schreibt statt \\
    $P_1 = \{(S_1,\; 1), (S_1,\; S_{1}0), (S_1,\; S_{1}1)\}$ normalerweise
    $P_1 = \{S_1 \rightarrow 1,\; S_1 \rightarrow S_{1}0,\;
    S_1 \rightarrow S_{1}1\}$.
\end{Def}

\begin{Def}{Ableitungen}
    Sei $G = (V, \Sigma, P, S)$ (kontextfreie) Grammatik. Auf
    $(V \cup \Sigma)^\ast$ werden definiert:
    \begin{itemize}
        \item $u \Rightarrow v \quad\Leftrightarrow\quad u = xAy, v = xwy$ mit
        $x, y \in (V \cup \Sigma)^\ast$ und $(A, w) \in P$ \\
        ($v$ ist aus $u$ \begriff{in einem Schritt ableitbar})
        
        \item $u \Rightarrow^\ast v \quad\Leftrightarrow\quad$
        $u = v$ oder
        $u = z_0 \Rightarrow z_1 \Rightarrow \cdots \Rightarrow z_{k-1}
        \Rightarrow z_k = v$, $z_i \in (V \cup \Sigma)^\ast$, $k \ge 1$ \\
        ($v$ ist aus $u$ \begriff{ableitbar}),
        "`$\Rightarrow^\ast$"' ist der reflexive und transitive Abschluss
        von "`$\Rightarrow$"'
    \end{itemize}
\end{Def}

\begin{Def}{Erzeugte Sprache}
    Die von einer (kontextfreien) Grammatik $G = (V, \Sigma, P, S)$ \\
    \begriff{erzeugte Sprache} ist die Menge
    $L(G) = \{w \in \Sigma^\ast \;|\; S \Rightarrow^\ast w\}$.
    Eine Sprache $L \subseteq \Sigma^\ast$ heißt
    \begriff{kontextfreie Sprache}, falls es eine kontextfreie Grammatik $G$
    mit $L(G) = L$ gibt.
\end{Def}

\begin{Def}{Bäume}
    Man kann alle möglichen Ableitungen aus dem Startsymbol einer kontextfreien
    Grammatik als Baum darstellen.
    Dieser besteht aus \begriff{Wurzel}, \begriff{Knoten}, \begriff{Blätter}
    und \begriff{Kanten}.
    Jede Ableitung entspricht einem Pfad in dem Baum, alle Blätter bilden
    die erzeugte Sprache $L(G)$. \\
    Es kann auch die Ableitung eines bestimmten Worts als sog.
    \begriff{Ableitungsbaum} dargestellt werden, das dem Baum zugehörige Wort
    heißt dann das \begriff{abgeleitete Wort} des Ableitungsbaums. \\
    Ableitungen eines bestimmten Worts mit gleichem Ableitungsbaum werden als
    gleich angesehen.
    Sie unterscheiden sich nur in der Reihenfolge, in der die Regeln auf die
    Nichtterminalzeichen angewendet werden.
\end{Def}

\begin{Def}{Eindeutigkeit}
    Sei $w \in L(G)$ ein Wort der erzeugten Sprache.
    $w$ heißt \begriff{eindeutig}, wenn es genau einen Ableitungsbaum gibt,
    dessen abgeleitetes Wort $w$ ist.
    Sonst heißt $w$ \begriff{mehrdeutig}. \\
    $G$ heißt \begriff{eindeutig}, wenn alle Wörter $w \in L(G)$ eindeutig
    sind.
    Sonst heißt $G$ \begriff{mehrdeutig}.
\end{Def}

\begin{Def}{(Kontextsensitive) Grammatik}
    $G = (V, \Sigma, P, S)$ heißt \begriff{(Chomsky-)Grammatik}, wenn \\
    $P \subset V^+ \times (V \cup \Sigma)^\ast$ endliche Menge ist
    (ansonsten wie bei kontextfreier Grammatik). \\
    Ableitungsrelationen und erzeugte Sprache sind analog wie bei
    kontextfreien Grammatiken definiert.
    Allerdings kann man einzelne Ableitungen nun nicht mehr als Baum
    darstellen, man muss dazu zu einer Netzstruktur greifen.
\end{Def}

\begin{Def}{\name{Backus}-\name{Naur}-Form}
    Eine BNF ist ein Viertupel $(V, \Sigma, P, S)$ mit
    \begin{itemize}
        \item $V$ nicht-leere endliche Menge der Form
        $<\!\!Zeichenkette\!\!>$ (\begriff{Nichtterminalzeichen}),
        
        \item $\Sigma$ nicht-leere endliche Menge mit
        $V \cap \Sigma = \emptyset$ und $| \notin \Sigma$
        (\begriff{Terminalzeichen}),
        
        \item $P \subset V \{::=\}
        (V \cup \Sigma)^\ast (\{|\} (V \cup \Sigma)^\ast)^\ast$
        endliche Menge (\begriff{Regeln} oder \begriff{Produktionen}),
        
        \item $S \in V$ (\begriff{Startsymbol}).
    \end{itemize}
\end{Def}

\subsection{%
    Syntaxdiagramme%
}

\begin{Def}{Syntaxdiagramm}
    Jede BNF kann als sog. \begriff{Syntaxdiagramm} grafisch dargestellt
    werden. Dazu zeichnet man für jedes Nichtterminalzeichen ein Pfeildiagramm
    mit in Rechtecke eingerahmte Nichtterminal- und in Kreise eingerahmte
    Terminalzeichen.
    
    Man kann ein Wort der BNF erzeugen, indem man das Diagramm des
    Startsymbols in Pfeilrichtung durchläuft. Trifft man auf ein
    Nichtterminalzeichen, so wird sein Diagramm an dieser Stelle
    "`eingeklebt"'. Trifft man auf ein Terminalzeichen, so fügt man es an die
    (anfangs leere) Ausgabe hinzu.
    
    Alle möglichen, sich so ergebenden Ausgaben bilden die erzeugte Sprache.
\end{Def}

\subsection{%
    Sprachen zur Beschreibung von Sprachen%
}

\begin{Def}{Metasprache}
    Eine \begriff{Metasprache} ist eine Sprache, mit der man andere Sprachen
    beschreiben kann.
    
    Es gibt die Ebenen \begriff{Syntax} (korrekter Aufbau der Wörter),
    \begriff{Semantik} (Bedeutungszuordnung zu jedem Wort/Satz) und
    \begriff{Pragmatik} (Beziehungen zwischen der Sprache und den
    Anforderungen).
    
    Fast alle natürliche Sprachen (Deutsch, Englisch usw.) besitzen wie viele
    formale Sprachen die Eigenschaft, sich selbst beschreiben zu können.
    Bspw. kann man mit EBNF den Aufbau einer EBNF beschreiben.
    Allerdings kann nicht alles in EBNF beschrieben werden
    (Terminalzeichen müssen paarweise verschieden sein usw.).
\end{Def}

\pagebreak
