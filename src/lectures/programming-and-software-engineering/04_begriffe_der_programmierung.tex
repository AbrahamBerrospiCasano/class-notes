\section{%
    Begriffe der Programmierung%
}

\subsection{%
    Blöcke, Ausnahmen, Überladen%
}

\begin{Def}{Block}
    Ein Block ist eine sich geschlossene, durch \adacode{begin}/\adacode{end}
    geklammerte Folge von Anweisungen mit Deklarations- bzw. Ausnahmeteil
    am Anfang bzw. Ende. \\
    Die Bezeichnungen des Deklarationsteils können nur innerhalb dieses
    Blocks und seiner Unterblöcke verwendet werden.
    Nach Verlassen des Blocks sind die Bezeichner und die entsprechenden
    Objekte undefiniert/unbekannt. \\
    Ein Bezeichner, der explizit im Deklarationsteil eines Blocks oder implizit
    als Laufvariable/Marke/Bezeichnung eingeführt wird, heißt \begriff{lokal}
    zu diesem Block.
    Bezeichner, die in Oberblöcken deklariert wurden, heißen \begriff{global}
    in den Unterblöcken. \\
    Ein in einem Oberblock deklarierter Bezeichner wird durch Neudeklaration
    in einem Unterblock "`ausgeblendet"' und kann nicht mehr angesprochen
    werden.
\end{Def}

\begin{Def}{Bezeichner}
    Ein Bezeichner $X$ bezieht sich stets auf die Deklaration von $X$,
    die sich im Deklarationsteil des innersten Blocks befindet.
\end{Def}

\begin{Def}{Lebensdauer/Sichtbarkeit}
    Die \begriff{Lebensdauer} eines Bezeichners (und des zugehörigen Objekts)
    ist der Block, in dem der Bezeichner deklariert wurde.
    Der Bezeichner/das Objekt lebt genau ab der Stelle seiner Deklaration,
    bis zu der Stelle, an dem der Block verlassen wird.
    Wird der Block später neu betreten, so wird ein neues Objekt erzeugt. \\
    Der \begriff{Gültigkeits-/Sichtbarkeitsbereich} eines Bezeichners/Objekts
    ist der Teil der Lebensdauer, in dem unmittelbar über den Bezeichner
    auf das Objekt zugegriffen werden kann.
    Ein Objekt kann unsichtbar und dann wieder sichtbar werden.
\end{Def}

\begin{Def}{Speicher}
    Blöcke und die Variablen werden auf dem Stack gespeichert und verwaltet.
\end{Def}

\begin{Def}{Vorteile von Blöcken}
    getrennte Entwicklung/Optimierung, besseres Verständnis; \\
    Hilfsvariablen und Zwischenrechnung verschwinden nach Abarbeiten; \\
    Vermeidung von Namenskonflikten bei größeren Programmeinheiten; \\
    Einfluss auf den Stack und eigene gezielte Verwaltung des Speicherplatzes.
\end{Def}

\begin{Def}{Überladen}
    \begriff{Überladen} ist die Mehrfachdeklaration eines Bezeichners
    (d.~h. einem Bezeichner sind mehrere verschiedene Objekte zugeordnet).
    An jeder Stelle des Programms muss aber aus dem Kontext eindeutig
    hervorgehen, welche Bedeutung gemeint ist. \\
    In \Ada{} ist Überladen für bestimmte Bezeichner (Literale in
    Aufzählungstypen, Funktionen, Operatoren, Unterprogramme) zulässig,
    andere (Datenobjekte, implizite Bezeichner) dürfen nicht überladen
    werden. \\
    Bei Unterprogrammen ist Überladen zulässig, falls sich die Deklarationen
    in der Reihenfolge der Parametertypen, in mindestens einem Parametertyp
    oder im Ergebnistyp unterscheiden.
\end{Def}

\begin{Def}{Ausnahmen}
    In \Ada{} kann am Ende jedes Blocks eine Ausnahmebehandlung definiert
    werden.
    Dazu deklariert man im Deklarationsteil (z.~B. des Packages) mittels \\
    \adacode{Fehlername : exception;} die Ausnahme.
    In einem Unterprogramm wird diese mittels \\
    \adacode{raise Fehlername;} geworfen.
    Fehler kann man in einem übergeordneten Unterprogramm
    durch \adacode{exception when Fehlername => Put ("1");
    when others => Put ("2");} kontrolliert abfangen.
    Im Falle eines Fehlers wird dabei nach der Ausnahmebehandlung gesucht
    (notfalls wird zum übergeordneten Block gewechselt).
    Dabei werden u.~U. auch Prozeduren/ Funktionen verlassen und der
    entsprechende Platz auf dem Stack freigegeben. \\
    In \Ada{} sind vier Standardfehler \adacode{Constraint\_Error},
    \adacode{Program\_Error}, \adacode{Strorage\_Error}, \adacode{Tasking\_Error}
    vorhanden.
\end{Def}

\subsection{%
    Prozeduren und Funktionen%
}

\begin{Def}{Unterprogramm}
    Eine Folge von Deklarationen und Anweisungen kann zur einer Programmeinheit
    (\begriff{Prozedur}/\begriff{Unterprogramm}) mit Namen und formalen
    Parametern zusammengefasst werden.
    Eine Prozedur besteht aus Spezifikation (Name, Parameter) und Rumpf
    (Deklarationsteil, Anweisungen).
    \begriff{Seiteneffekte}, die durch Verwendung globaler Variablen auftreten
    können, sind zu vermeiden.
\end{Def}

\begin{Def}{Rekursion}
    \begriff{Rekursion} ist die Verwendung eines Unterprogramms
    in seinem eigenen Rumpf.
\end{Def}

\begin{Def}{Parameterübergabe (Pseudocode)}
    \begin{itemize}
        \item \begriff{Call-By-Value}: Die mit \pseudocode{value} versehenen
        formalen Parameter werden als lokale Variablen aufgefasst, denen beim
        Funktionsaufruf die Werte der aktuellen Parameter zugewiesen werden.
        Sie dürfen neue Werte erhalten, diese werden jedoch am Ende der
        Prozedur nicht wieder zurückgeschrieben.
        
        \item \begriff{Call-By-Reference}: Die mit \pseudocode{access} versehenen
        formalen Parameter sind Zeiger auf die aktuellen Parameter.
        
        \item \begriff{Call-By-Name}: Die mit \pseudocode{name} versehenen
        formalen Parameter werden beim Funktionsaufruf textuell durch die
        aktuellen Parameter ersetzt.
    \end{itemize}
\end{Def}

\begin{Def}{Kopierregel} \\
    \begin{tabular}{p{7.0cm}p{9.1cm}}
        \begin{minipage}[c]{7.0cm}
            \begin{tabular}{p{0.2cm}l}
                \multicolumn{2}{l}{$\mathtt{\mathbf{declare}\; X_1 : Typ_1;\; \ldots;\; X_n : Typ_n;}$} \\
                \multicolumn{2}{l}{$\mathtt{\mathbf{begin}}$} \\
                & $\mathtt{X_1 := \alpha_1;\;\; \ldots;\;\; X_n := \alpha_n;}$ \\
                & $\mathtt{modifizierterPRUMPF}$ \\
                \multicolumn{2}{l}{$\mathtt{\mathbf{end};}$}
            \end{tabular}
        \end{minipage}
        &
        \begin{minipage}[c]{9.1cm}
            Gegeben sei eine Prozedur\\
            $\mathtt{\mathbf{procedure}\; }$
            $\mathtt{P(X_1: p"u_1\; T_1;\; \ldots;\; X_n: p"u_n\; T_n)}$\\
            $\mathtt{\mathbf{is}\; PRUMPF;}$, wobei
            $\mathtt{p"u_i} \in \{\mathtt{\mathbf{value}},
            \mathtt{\mathbf{access}}, \mathtt{\mathbf{name}}\}$ die
            Parameterübergabe angibt.
            Der Prozeduraufruf $\mathtt{P(\alpha_1, \ldots, \alpha_n)}$ mit den
            aktuellen Parametern $\mathtt{\alpha_1}, \mathtt{\ldots},
            \mathtt{\alpha_n}$ wird dann durch
            nebenstehenden Block ersetzt.
        \end{minipage}
    \end{tabular}
    
    Dabei sei $\mathtt{Typ_i = T_i}$ für $\mathtt{p"u_i = \mathbf{value}}$,
    $\mathtt{Typ_i = \mathbf{access}\; T_i}$ für
    $\mathtt{p"u_i = \mathbf{access}}$ und \\
    $\mathtt{X_i : Typ_i;}$ sowie $\mathtt{X_i := \alpha_i;}$ entfallen für
    $\mathtt{p"u_i = \mathbf{name}}$. \\
    $\mathtt{modifizierterPRUMPF}$ ist ein Block, der folgendermaßen aus
    $\mathtt{PRUMPF}$
    entsteht:
    \begin{enumerate}
        \item Jeder formale Parameter
        $\mathtt{X_i}$ mit $\mathtt{p"u_i = \mathbf{access}}$ wird durch
        $\mathtt{\mathbf{deref}\; X_i}$ ersetzt.
        
        \item Jeder formale Parameter und jeder lokale Name in
        $\mathtt{PRUMPF}$, der gleich einem Namen ist, der in irgendeinem
        aktuellen Parameter $\mathtt{\alpha_i}$ mit
        $\mathtt{p"u_i = \mathbf{name}}$ vorkommt, wird durchgehend mit einem
        neuen Namen bezeichnet.
        
        \item Alle $\mathtt{X_i}$ mit $\mathtt{p"u_i = \mathbf{name}}$ werden
        durch $\mathtt{\alpha_i}$ textuell ersetzt.
        
        \item (Globale Variablen dürfen nicht "`lokaler"' werden.)
    \end{enumerate}
    
    Dann wird dieser Block ausgeführt.
    Nach der Ausführung wird er wieder durch den Prozeduraufruf
    $P(\alpha_1, \ldots, \alpha_n)$ ersetzt und das Programm setzt
    mit der folgenden Anweisung fort. \\
    Die obige Kopie des Prozedurrumpfs heißt
    \begriff{Inkarnation}/\begriff{konkrete Ausprägung} der Prozedur.
\end{Def}

\begin{Def}{Nur Call-By-Value}
    Manche Sprachen erlauben nur Call-By-Value als Übergabeart (z.~B. C).
    Jedoch kann man dann einen Pointer als Parameter übergeben, sodass
    man die referenzierte Variable abändern kann.
\end{Def}

\begin{Def}{Parameterübergabe in \Ada{}}
    In \Ada{} gibt es Parameter vom Typ \adacode{in} (formaler Parameter wird
    wie eine Konstante behandelt, darf nicht verändert werden), \adacode{out}
    (wird wie eine Variable behandelt, aktueller Parameter muss eine Variable
    sein, zugewiesene Werte werden erst bei Beendigung der Prozedur dem
    aktuellen Parameter zugewiesen) und \adacode{in out}
    (wie \adacode{out}, aber dem formalen Parameter
    wird wie bei \adacode{in} anfangs der Wert des aktuellen Parameters
    zugewiesen). \\
    In Funktionen sind nur \adacode{in}-Parameter erlaubt
    (Standard, wenn nicht angegeben).
\end{Def}

\subsection{%
    Moduln%
}

\begin{Def}{Eigenschaften von Moduln}
    \emph{in sich abgeschlossene Einheit} mit klar definierter Aufgabe;
    genau definierte \emph{Schnittstelle} nach außen (nur die dort
    genannten Eigenschaften sind nach außen hin sichtbar);
    die \emph{interne Arbeitsweise}/\emph{Implementation} ist außen nicht
    bekannt (zwei Sichten: Außenansicht und Innensicht, die nach außen hin
    versteckt bleibt);
    \emph{überschaubar}, gut zu testen, einfach zu warten;
    in \emph{Bibliotheken} aufbewahrbar und leichte Einbaubarkeit in
    beliebige Programmsysteme
\end{Def}

\begin{Def}{Schematischer Aufbau eines Moduls}
\begin{lstlisting}[language=pseudosprache]
module <Name des Moduls> is
[with ...; use ...]   --  Angaben, welche anderen Einheiten verwendet werden und in welcher Weise
specification ...     --  Angabe der nach aussen sichtbaren Datentypen, Konstanten, Variablen und
                      --  "Methoden" (also Funktionen, Operatoren usw.)
[implementation ...]  --  weitere (nach aussen nicht sichtbare) Deklarationen sowie Programme
                      --  zur Implementierung der Methoden und Typen
[begin ... end]       --  Anweisungen zur Initialisierung, einschliesslich der Ausnahmebehandlungen
end module [<Name des Moduls>]
\end{lstlisting}
    \vspace{-10pt}
    Moduln sind die programmiersprachliche Realisierung von Datentypen.
    Beispielsweise kann ein "`Stack für Zeichen"' als Modul umgesetzt werden.
\end{Def}

\begin{Def}{Moduln in \Ada{} ("`Pakete"')}
    Das Schlüsselwort in \Ada{} lautet \adacode{package}, man spricht von
    \begriff{Paketen}.
    Spezifikations-/Implementierungsteil werden voneinander
    getrennt und lauten \\
    \adacode{package <Paketname> is <einfache Deklarationen> end <Paketname>;}
    bzw. \\
    \adacode{package body <Paketname> is <Deklarationen> end <Paketname>;}. \\
    Der Implementierungsteil kann entfallen, falls die Spezifikation nur
    aus Datentyp- und Konstantendeklarationen besteht. \\
    Die Deklaration \begriff{privater Typen} erfolgt durch
    \adacode{type <Typname> is private;}, die Struktur des Typs wird
    am Ende der Spezifikation nach dem Schlüsselwort \adacode{private}
    angegeben und so vor dem Benutzer versteckt.
    In \Ada{} sind mit einem Datentype (auch privat) stets die Operationen
    \adacode{=}, \adacode{/=} und \adacode{:=} verbunden.
    Sollen diese Operationen nicht für die Benutzer des Moduls zugelassen
    werden, so muss man den Typ als \adacode{limited private} deklarieren.
\end{Def}

\subsection{%
    Polymorphie%
}

\begin{Def}{Allgemein}
    \begriff{Polymorphie} (griechisch: \emph{Vielgestaltigkeit}) ist ein
    Grundprinzip der Informatik, das sich durch folgende Maßnahmen äußert:
    Möglichst lange den konkreten Datentyp von Variablen offen lassen
    (z.~B. unspezifizierte Feldgrenzen), möglichst lange konkrete
    Realisierung offen lassen (z.~B. Spezifikations-/Implementierungsteil
    trennen) und Parametrisierung von Paketen und Unterprogrammen für den
    Einsatz in möglichst vielen Programmen (z.~B. Generizität).
\end{Def}

\begin{Def}{in der Programmierung}
    In der Programmierung spricht man von \begriff{Polymorphie}, falls
    Bezeichner mehrfach verwendet werden (Überladen), falls
    Variablen je nach aktueller Umgebung Elemente verschiedener Datentypen
    bezeichnen, falls Parametrisierung mit Typen erfolgt (also falls Typen als
    Parameter für Prozeduren/Typen verwendet werden) und falls
    \begriff{Generizität} bei Unterprogrammen/Moduln verwendet wird.
\end{Def}

\begin{Def}{Generizität}
\begin{lstlisting}[language=ada]
--  Spezifikation                           --  Implementierung
generic type Element is private;            procedure Tausch (A, B : in out Element) is ...
procedure Tausch (A, B : in out Element);   begin ... end Tausch;

--  Verwendung
procedure IntTausch is new Tausch (Integer);
X, Y : Integer;
...
IntTausch (X, Y);
\end{lstlisting}
    In \Ada{} wird der variabel gehaltene Bereich mit \adacode{generic}
    eingeleitet.
    Im Beispiel ist \adacode{Element} ein formaler Parameter, der bei der
    Instanziierung durch \adacode{is new} textuell durch den aktuellen
    Parameter (hier \adacode{Integer}) ersetzt wird.
    Ein \adacode{generic}-Parameter darf in \Ada{} nicht bereits im
    \adacode{generic}-Bereich verwendet werden.
    Das Problem wird durch generische Pakete gelöst.
\end{Def}

\subsection{%
    Vererbung%
}

\begin{Def}{Ableitungen von Datentypen}
    Ist ein Datentyp bereits deklariert, so kann man durch Hinzufügen weiterer
    Komponenten aus ihm weitere Datentypen ableiten
    (\begriff{Spezialisierung}). \\
    Liegen mehrere Datentypen vor, die gewisse Komponenten gemeinsam haben, so
    kann man diese Gemeinsamkeiten als eigenen Datentyp herausziehen
    (\begriff{Generalisierung}).
\end{Def}

\begin{Def}{Spezialisierung in \Ada{}}
    Mittels \adacode{type abc is tagged record ... end record;} kann man einen
    Record erstellen, der erweitert werden soll.
    Bei der Erweiterung mit einem \begriff{Unterda\-tentyp} muss der
    \begriff{Obertyp} per
    \adacode{type xyz is new abc with record ... end record;} angegeben werden.
    Man spricht beim Vorgang, Eigenschaften an andere Einheiten
    weiterzureichen, von \begriff{Vererbung}.
    Die Obertypen heißen \begriff{Eltern}, die Untertypen \begriff{Kinder}.
    Man sagt, \adacode{xyz} ist ein aus \adacode{abc}
    \begriff{abgeleiteter Typ}.
    (Die Eigenschaft \adacode{tagged} vererbt sich automatisch an die
    Untertypen, d.~h. diese können wiederum ohne Zusätze weiter abgeleitet
    werden.)
\end{Def}

\begin{Def}{Generalisierung in \Ada{}}
    Gemeinsame Komponenten kann man in einen Obertyp herausziehen.
    Man deklariert einen solchen Obertyp mit \\
    \adacode{type abc is abstract tagged record ... end record;},
    Untertypen lassen sich dann wie oben erstellen.
    Der Unterschied ist, dass sich abstrakte Datentypen (wie hier
    \adacode{abc}) im Gegensatz zu den Untertypen nicht als Variable oder
    formaler Parameter deklarieren lassen.
\end{Def}

\begin{Def}{Umdefinitionen}
    Bei der Vererbung von Typen kann man vererbte Komponenten neu definieren.
    Die vererbten Komponenten sind dann wegen der Sichtbarkeitsregel
    automatisch ausgeblendet (\begriff{overridden}).
\end{Def}

\begin{Def}{Mehrfachvererbung}
    Es gibt Sprachen (wie \Ada{}), in denen ein Datentyp höchsten einen
    direkten Obertyp besitzen kann (\begriff{Einfach-Vererbung}).
    Können Eigenschaften mehrerer Obertypen an einen Datentyp weitergereicht
    werden, spricht man von \begriff{Mehrfach-Vererbung}.
\end{Def}

\subsection{%
    Objekte%
}

\begin{Def}{Objekte}
    \begriff{Objekte} sind in sich geschlossene Einheiten, die wie Moduln
    aufgebaut sind.
    Es gibt ein Schema (\begriff{Klasse}), das aus \begriff{Attributen} und
    \begriff{Methoden} besteht.
    Aus diesem kann ein konkretes Objekt (eine \begriff{Instanz}) erzeugt
    werden.
    Objekte können einen individuellen Zustand besitzen und miteinander
    kommunizieren.
    Durch Vererbung können sie ihre Eigenschaften an neue Objekte/Klassen
    weiterreichen.
\end{Def}

\begin{Def}{Prinzipien der Objektorientierung}
    es gibt nur Objekte (eindeutig über Namen identifizierbar);
    handeln in eigener Verantwortung;
    Klassen werden in Bibliotheken aufbewahrt und stehen allen
    Programmen zur Verfügung\dots
\end{Def}

\pagebreak
