\chapter{%
    Komplexität von Algorithmen und Programmen%
}

\section{%
    Aufwandsfunktionen%
}

\begin{Def}{Zeitaufwand}
    Die \begriff{Zeitkomplexität} behandelt die Zeitspanne, innerhalb derer
    die Ergebnisse ausgegeben werden.
    Diese "`Rechendauer"' ist abhängig von der Eingabe:
    Man definiert in der Regel den Zeitaufwand als Funktion
    $t_\pi: \mathbb{N}_0 \rightarrow \mathbb{N}_0$ mit $t_\pi(n) =$ maximale
    Zeit/Anzahl der Schritte, die das Programm $\pi$ für irgendeine Eingabe der
    Länge $n$ bis zum Anhalten benötigt.
    Meist verlangt man, dass $\pi$ stets terminiert, damit $t_\pi$ überall
    definiert ist.
    Die \begriff{uniforme Zeitkomplexität} geht davon aus, dass für jeden
    Befehl (Zuweisung, Ausdruck usw.) gleich viel Zeit in Anspruch genommen
    wird (in der Realität ist dies oft nicht so).
\end{Def}

\begin{Def}{Größenordnung}
    Bei der Zeitkomplexität interessiert man sich meist nur für eine
    Abschätzung, bei der multiplikative und additive Konstanten ignoriert
    werden.
    Zur Beschreibung der \begriff{Größenordnung} einer
    reellwertigen Funktion werden die \textsc{Landau}-Symbole verwendet.
\end{Def}

\begin{Def}{\textsc{Landau}-Symbole}
    Sei $f: \mathbb{R}^+ \rightarrow \mathbb{R}^+$.
    Dann sind die \textsc{Landau}-Symbole wie folgt definiert:
    \begin{itemize}
        \item $\mathcal{O}(f) =
        \{g: \mathbb{R}^+ \rightarrow \mathbb{R}^+ \;|\;
        \exists_{C > 0} \exists_{n_0 \in \mathbb{N}}
        \forall_{n \ge n_0}\; g(n) \le C \cdot f(n)\}$

        \item $o(f) =
        \{g: \mathbb{R}^+ \rightarrow \mathbb{R}^+ \;|\;
        \forall_{\varepsilon > 0} \exists_{n_0 \in \mathbb{N}}
        \forall_{n \ge n_0}\; g(n) \le \varepsilon \cdot f(n)\}$

        \item $\Omega(f) =
        \{g: \mathbb{R}^+ \rightarrow \mathbb{R}^+ \;|\;
        \exists_{C > 0} \exists_{n_0 \in \mathbb{N}}
        \forall_{n \ge n_0}\; f(n) \le C \cdot g(n)\}$

        \item $\omega(f) =
        \{g: \mathbb{R}^+ \rightarrow \mathbb{R}^+ \;|\;
        \forall_{\varepsilon > 0} \exists_{n_0 \in \mathbb{N}}
        \forall_{n \ge n_0}\; f(n) \le \varepsilon \cdot g(n)\}$

        \item $\Theta(f) =
        \{g: \mathbb{R}^+ \rightarrow \mathbb{R}^+ \;|\;
        \exists_{C_1, C_2 > 0} \exists_{n_0 \in \mathbb{N}}
        \forall_{n \ge n_0}\; C_1 \cdot f(n) \le g(n) \le C_2 \cdot f(n)\}$
    \end{itemize}
    Es gilt $g \in \Omega(f) \;\Leftrightarrow\; f \in \mathcal{O}(g)$
    sowie $g \in \omega(f) \;\Leftrightarrow\; f \in o(g)$. \\
    Außerdem ist $\Theta(f) = \mathcal{O}(f) \cap \Omega(f)$.
\end{Def}

\begin{Def}{Komplexitätsklassen}
    In der Praxis betrachtet man meistens nur die Funktionsklassen
    $\mathcal{O}(1)$ (konstante Fkt.),
    $\mathcal{O}(\log n)$ (höchstens logarithmisch wachsende Fkt.),
    $\mathcal{O}(\sqrt[k]{n})$ (höchstens mit einer $k$-ten Wurzel wachsende
    Fkt.),
    $\mathcal{O}(n)$ (lineare Fkt.),
    $\mathcal{O}(n \cdot \log n)$ ("`ein wenig"' stärker als linear wachsende
    Fkt.),
    $\mathcal{O}(n^2)$ (höchstens quadratisch wachsende Fkt.),
    $\mathcal{O}(n^k)$ (höchstens polynomiell vom Grad $k$ wachsende Fkt.) und
    $\mathcal{O}(2^n)$ (höchstens exponentiell zur Basis 2 wachsende Fkt.).
\end{Def}

\section{%
    Registermaschinen und andere Rechenmodelle%
}

\begin{Def}{Registermaschine}
    Eine \begriff{Registermaschine} ist wie ein kleiner Mikroprozessor
    aufgebaut und besteht aus
    \begin{itemize}
        \item einer \begriff{Zentraleinheit} mit sechs \begriff{Registern}
        (\assembler{X}, \assembler{Y} und \assembler{Z} für
        arithmetische/logische Operationen,
        Adressregister \assembler{A}, Befehlsregister \assembler{B}
        und Flagregister \assembler{F}),

        \item einem endlichen \begriff{Programmspeicher}, in dem nacheinander
        die Befehle
        des abzuarbeitenden (endlichen) Programms stehen

        \item sowie einem unendlich langen \begriff{Rechenspeicher} mit
        durchnummerierten Speicherzellen, die jeweils eine beliebig große ganze
        Zahl aufnehmen können.
    \end{itemize}
\end{Def}

\begin{Def}{Programmieren mit der Registermaschine}
    \emph{Jedes} \Ada{}-Programm lässt sich ein Registermaschinenprogramm
    übersetzen.
    Dieser Prozess lässt sich automatisieren (\begriff{Compiler}).
    Alle Kontrollstukturen lassen sich mit dem bedingten Sprung
    \assembler{jump b} realisieren (vgl. \adacode{goto} und Marken in \Ada{}).
\end{Def}

\pagebreak

\begin{Def}{Befehle der Registermaschine} \\
    \begin{tabular}{p{11.5cm}p{4.5cm}}
        \begin{minipage}[c]{11.5cm}
    \begin{tabular}{lp{3cm}|ll|lll}
\assembler{load V,c} & \pseudocode{V := c} &
\assembler{copy V,V'} & \pseudocode{V := V'} \\
\assembler{read V} & \pseudocode{V := R<A>} &
\assembler{write V} & \pseudocode{R<A> := V} \\
\assembler{add} & \pseudocode{X := Y + Z} &
\assembler{sub} & \pseudocode{X := Y - Z} \\
\assembler{succ} & \pseudocode{X := X + 1} &
\assembler{shift} & \pseudocode{X := X div 2} \\ \hline
\assembler{comp (v)} &
\multicolumn{3}{l|}{\pseudocode{if XvY then F := 1 else F := 0 fi}} \\
\assembler{jump b} &
\multicolumn{3}{l|}{\pseudocode{if F=1 then B := b else B := B + 1 fi}} \\ \hline
\assembler{stop} & Anhalten
    \end{tabular}
        \end{minipage}
        &
        \begin{minipage}[c]{4.5cm}
            $V, V'$ Register, \\
            $c \in \mathbb{Z}$, \quad
            $b \in \mathbb{N}_0$, \\
            $Rk$ $k$-te Speicherzelle, \\
            $v \in \{>, \ge, <, \le, =, \not=\}$, \\
            außer bei \assembler{jump} wird nach jedem Befehl \assembler{B}
            um eins erhöht.
        \end{minipage}
    \end{tabular}

    Diese Befehle finden sich bei allen Mikroprozessoren (bei diesen kommen
    Befehle hinzu). \\
    Die Befehle eines Programms werden bei 0 beginnend durchnummeriert.
\end{Def}

\begin{Def}{Kellerautomat}
    \begriff{Keller/Pushdown-Automaten} besitzen ein Eingabe- und ein
    Ausgabeband, sowie einen Keller, auf den Daten abgelegt werden können.
    Lässt man das Kellerband weg, so erhält man einen endlichen Automaten
    (also eine Maschine, die eine Eingabe von links nach rechts liest,
    synchron dazu ein Ausgabeband beschreibt und nur endlich viel Information
    in ihrer Zustandsmenge speichern kann).
\end{Def}

\begin{Def}{Endlicher Automat}
    $A = (Q, \Sigma, \Omega, \delta, Q_0, F)$ heißt
    \begriff{endlicher Automat}, falls
    \begin{itemize}
        \item $Q$ nicht-leere endliche Menge (\begriff{Zustandsmenge}),
        \item $\Sigma$ nicht-leere endliche Menge (\begriff{Eingabealphabet}),
        \item $\Omega$ nicht-leere endliche Menge (\begriff{Ausgabealphabet}),
        \item $\delta \subseteq Q \times \Sigma \times Q \times \Omega^\ast$
        endliche Menge (\begriff{Überführungsrelation}),
        \item $Q_0 \subseteq Q$ (\begriff{Menge der Anfangszustände}),
        \item $F \subseteq Q$ (\begriff{Menge der Endzustände}).
    \end{itemize}
    $A$ heißt \begriff{deterministisch}, falls es zu jedem Paar
    $(q,a) \in Q \times \Sigma$ höchstens ein Paar
    $(q',w) \in Q \times \Omega^\ast$ gibt, sodass
    $(q,a,q',w) \in \delta$ ist.
    $A$ heißt \begriff{endlicher Akzeptor}, falls $\Omega$ entfällt,
    $\delta \subset Q \times \Sigma \times Q$ ist und es genau einen
    Anfangszustand $Q_0 \in Q$ gibt.
\end{Def}

\begin{Def}{Grafische Darstellung}
    Die Zustände werden durch Kreise dargestellt, die Übergänge durch Kanten
    (Pfeile), an die Eingabe/Ausgabe getrennt durch "`/"' geschrieben werden.
    Beim Akzeptor entfällt die Ausgabe, dann steht nur die Eingabe über dem
    Pfeil.
    Anfangszustände erhalten einen Pfeil ("`aus dem Nichts"') und
    Endzustände werden doppelt umkringelt.
\end{Def}

\begin{Def}{Interpretation als Automat}
    $L(A) = \{(u,v) \in \Sigma^\ast \times \Omega^\ast \;|\;$
    es gibt eine Folge von Übergängen, die einen Anfangszustand aus
    $Q_0$ bei der Eingabe von $u$ in einen Endzustand aus $F$
    überführen und hierbei die Ausgabe $v$ erzeugen $\}$ heißt die
    von $A$ \begriff{realisierte Relation}. \\
    Falls $A$ deterministisch ist, wird $L(A)$ zu einer (partiellen)
    Abbildung $\operatorname{Res}_A: \Sigma^\ast \rightarrow \Omega^\ast$.
    Diese heißt dann die von $A$ \begriff{realisierte Abbildung} oder die
    \begriff{Resultatsfunktion} von $A$.
\end{Def}

\begin{Def}{Interpretation als Akzeptor}
    $L(A) = \{u \in \Sigma^\ast \;|\;$
    es gibt eine Folge von Übergängen, die den Anfangszustand
    $Q_0$ bei der Eingabe von $u$ in einen Endzustand aus $F$
    überführen $\}$ heißt die von $A$ \begriff{erkannte Sprache}.
\end{Def}

\pagebreak
