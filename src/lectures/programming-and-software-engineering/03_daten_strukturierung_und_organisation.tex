\section{%
    Daten, ihre Strukturierung und Organisation%
}

\subsection{%
    Programmaufbau%
}

\begin{lstlisting}[language=ebnf,
emph={procedure,function,is,begin,end,not,overriding},
emphstyle=\underbar]
subprogram_declaration ::= [overriding_indicator] subprogram_specification ;
subprogram_specification ::= procedure_specification | function_specification
procedure_specification ::= procedure defining_program_unit_name parameter_profile
function_specification ::= function defining_designator parameter_and_result_profile

subprogram_body ::= [overriding_indicator] subprogram_specification is declarative_part
  begin handled_sequence_of_statements end [designator] ;
overriding_indicator ::= [not] overriding
\end{lstlisting}

\subsection{%
    Lexikalische Einheiten%
}

\begin{Def}{Zeichensatz}
    Früher wurden \Ada{}-Programme in Latin-1 (ISO 8859-1) geschrieben, der den
    ASCII-Code enthält.
    Ab \Ada{ 2005} wird ISO/IEC 10646:2003 verwendet, der
    äquivalent zu Unicode ist.
\end{Def}

\begin{Def}{Lexikalische Einheit}
    Ein Programm ist eine Folge von Zeichen aus dem Zeichensatz. \\
    Die Zeichen bilden eine Folge von sog. \begriff{lexikalischen Einheiten}.
    Zu diesen gehören:
    \begin{itemize}
        \item \begriff{Bezeichner} zur Identifizierung von Programmobjekten
        
        \item \begriff{Literale} zur Bezeichnung von festen Werten
        (Zahlen, Zeichen, Strings)
        
        \item \begriff{Begrenzer} (delimiter), die eine spezielle Bedeutung
        besitzen und gleichzeitig lexikalische Einheiten voneinander trennen
        (z.~B. \adacode{\&}, \adacode{'}, \adacode{+}, \adacode{(},
        \adacode{)}, \adacode{:=}, \adacode{>=} usw.)
        
        \item \begriff{Wortsymbole} (reservierte Wörter der Sprache) mit
        spezieller Bedeutung (\adacode{for}, \adacode{if}, \linebreak
        \adacode{procedure}, \dots),
        dürfen nicht als Bezeichner verwendet werden
        
        \item \begriff{Trennzeichen} (separator) zur Trennung von lexikalischen
        Einheiten aus Lesbarkeitsgründen (Zwischenraum, Whitespace)
        
        \item \begriff{Kommentare} dienen der Erläuterung und Lesbarkeit
        (in \Ada{} Zeichenfolgen ab \adacode{{-}{-}})
    \end{itemize}
\end{Def}

\addtocounter{subsection}{1}

\subsection{%
    Zeigertypen%
}

\begin{Def}{Zeiger}
    Zeiger auf Variablen enthalten nicht einen bestimmten Wert, sondern die
    Speicheradresse der Variable im Arbeitsspeicher.
\end{Def}

\begin{Def}{Interne/externe Namen}
    Jede Variable hat einen internen (Speicheradresse) und einen externen
    Namen (symbolische Bezeichnung für die Speicheradresse). \\
    In \Ada{} dürfen "`normale"' Zeigervariablen nur auf Variablen zeigen, die
    ausschließlich über Zeiger ansprechbar sind (sich also im Heap befinden).
\end{Def}

\begin{Def}{Speicherbereiche}
    Es gibt drei große Speicherbereiche: \begriff{Programmspeicher}
    (Programmcode), \begriff{Stack}/Keller (Speicher für deklarierte Variablen,
    Blockstruktur) und \begriff{Heap}/Haufen (Variablen ohne externen Namen).
    
    Eine Zeigervariable wird bei Deklaration im Stack angelegt. Mittels
    \adacode{new} wird dann eine Variable im Heap allokiert, bei Wertzuweisung
    erhält der Zeiger die Adresse des erzeugten Objekts.
\end{Def}

\begin{Def}{Dynamische Datenstrukturen in \Ada{}}
    \begriff{Dynamische Datenstrukturen} sind durch Zeiger
    "`zusammengehaltene"' Daten.
    Mit Zeiger können sich ständig verändernde Daten gut beschrieben werden.
    In \Ada{} geht dies z.~B. so:
\begin{lstlisting}[language=ada]
type Zelle;                               --  Vorwaertsdeklaration
type Ptr_Zelle is access Zelle;           --  Zeigertyp
type Zelle is record                      --  Datentyp
   Inhalt    : Character;
   Naechster : Ptr_Zelle;
end record;
...
Anker : Ptr_Zelle;                        --  Zeigervariable
Test  : constant Ptr_Zelle := new Zelle;  --  konstante Zeigervariable
...
Anker := new Zelle'('a', null);           --  Allokation eines neues Objekts, Nullpointer
Put (Anker.Inhalt);
--  oder Anker.all.Inhalt (automatische Dereferenzierung bei Records!)
\end{lstlisting}
\end{Def}

\begin{Def}{Wertzuweisung bei Zeigern}
    Bei einer Zuweisung zwischen Zeigern müssen linke und rechte Seite den
    gleichen Datentyp (Zeigertyp) besitzen.
    Prinzipiell muss in \Ada{} jede Dereferenzierung mittels \adacode{.all}
    angegeben werden.
    Bei Records darf dies jedoch weggelassen werden.
\end{Def}

\begin{Def}{Gleichheit bei Zeigern}
    Zwei Zeiger sind gleich, wenn sie auf dieselbe Variable zeigen.
\end{Def}

\subsection{%
    Listen%
}

\begin{Def}{Lineare Liste}
    In einer \begriff{linearen Liste} sind die Elemente linear angeordnet, es
    gibt keine Verzweigungen. Lineare Listen können als \begriff{einfach}
    (jedes Element zeigt auf seinen Nachfolger) oder \begriff{doppelt
    verkettete Liste} (Zeiger für den Vorgänger) realisiert werden.
    Bei einer \begriff{zyklischen Liste} zeigt das letzte Element nicht auf
    \adacode{null}, sondern auf das erste Element der Liste.
    
    Operationen der linearen Liste:
    \adacode{Empty} (gibt leere Liste zurück, Nullpointer),\\
    \adacode{Isempty(A)} (überprüft, ob \adacode{A} leer ist),
    \adacode{First(A)} (gibt das erste Element der Liste zurück),
    \adacode{In\_Front(E, A)} (fügt am Anfang ein Element hinzu),
    \adacode{Append(E, A)} (fügt am Ende ein Element hinzu),
    \adacode{Delete(E, A)} (löscht alle Elemente mit dem Inhalt aus der Liste).
\end{Def}

\begin{Def}{Stack (Stapel)}
    Ein \begriff{Stack} ist eine lineare Liste mit genau den Operationen
    \adacode{Empty(A)} (leert eine Liste durch Setzen auf Nullpointer),
    \adacode{Isempty(A)} (überprüft, ob \adacode{A} leer ist),
    \adacode{Top(A)} (gibt das letzte Element zurück),
    \adacode{Push(A, E)} (fügt ein Element ans Ende an),
    \adacode{Pop(A)} (löscht das letzte Element der Liste). \\
    Ein Stack ist eine Liste, die nach dem \begriff{LIFO-Prinzip}
    (last in first out) arbeitet.
\end{Def}

\begin{Def}{Queue (Schlange)}
    Eine \begriff{Queue} ist eine lineare Liste mit genau den Operationen
    \adacode{Empty(A)} (leert eine Liste durch Setzen auf Nullpointer),
    \adacode{Isempty(A)} (überprüft, ob \adacode{A} leer ist),
    \adacode{First(A)} (gibt das erste Element zurück),
    \adacode{Enter(A, E)} (fügt ein Element ans Ende an),
    \adacode{Remove(A)} (löscht das erste Element der Liste). \\
    Eine Queue ist eine Liste, die nach dem \begriff{FIFO-Prinzip}
    (first in first out) arbeitet.
\end{Def}

\begin{Def}{Graphen (Geflechte)}
    Ein Graph ist ein beliebig vernetztes Gebilde (also i.~A. keine lineare
    Liste). Dieses besteht aus \begriff{Knoten} (Elemente des Datentyps) und
    \begriff{Kanten} (Verweise zwischen ihnen).
\end{Def}

\begin{Def}{Zeiger auf Stackvariablen}
    Dies ist in \Ada{} möglich, falls der Zeigertyp mit \adacode{all} und die
    referenzierte Variable mit \adacode{aliased} (Warnung für den Programmierer
    und Compiler optimiert nicht)
    deklariert wurde:
    
\begin{lstlisting}[language=ada]
type Ptr_Integer is access all Integer;
I      : aliased Integer;
Zeiger : Ptr_Integer := I'Access;
\end{lstlisting}
    
    \begriff{Dangling pointers} sind Zeiger, deren referenzierte Variable
    irgendwann nicht mehr existiert. Daher sollte die Lebensdauer des
    referenzierten Objekts mindestens so groß sein wie die des Zeigers.
\end{Def}

\subsection{%
    Referenzkonzept%
}

\begin{Def}{Interne Namen}
    Jedes Objekt erhält in der Programmierung einen Namen, auch wenn dies nicht
    im Programmtext geschieht (z.~B. implizite Typdeklaration bei
    Variablendeklaration).
    Im Programm ist es wichtig, dass alle Namen unterschieden werden können,
    da man sonst die Objekte nicht auseinander halten kann.
\end{Def}

\begin{Def}{Konstanten im Speicher}
    Bisher wurden Konstanten als Inhalte der Variablen aufgefasst.
    Diese Vorstellung kann modifizieren, indem man die normalen Variablen als
    "`Zeiger auf Konstanten"' ansieht.
    Man nimmt dabei an, dass sich alle Konstanten im Speicher befinden und die
    Variablen nur noch Adressen enthalten, wo sich die Konstanten befinden.
    Eine Wertzuweisung bewirkt dann nur noch eine Änderung des Zeigers.
\end{Def}

\begin{Def}{Referenzkonzept}
    Jedes Objekt erhält eine \begriff{Referenzstufe}.
    Konstanten erhalten hierbei die Referenzstufe $0$, Variablen $1$ usw.
    Allgemein erhält ein Objekt, das auf Objekte der Referenzstufe $k$
    verweist, die Referenzstufe $k + 1$.
\end{Def}

\subsection{%
    Bäume%
}

\begin{Def}{Gerichteter Graph, Weg}
    $G = (V, E)$ heißt \begriff{gerichteter Graph}/\begriff{Digraph}, falls
    $V$ eine nicht-leere endliche Menge ist (\begriff{Knoten}) und
    $E \subseteq V \times V$ (\begriff{Kanten}). \\
    Eine Folge von Knoten $(u_1, \ldots, u_r)$ (mit $r \in \mathbb{N}$) heißt
    \begriff{(gerichteter) Weg} im Graphen $G$, falls $(u_i, u_{i+1}) \in E$
    für $i = 1, \ldots, r - 1$ gilt.
    $r - 1$ ist die \begriff{Länge des Weges}.
\end{Def}

\begin{Def}{Wurzel, Baum}
    Ein Knoten $w$ heißt \begriff{Wurzel} eines Graphen $G$, falls es von $w$
    zu jedem Knoten des Graphen einen Weg gibt und $w$ keinen direkten
    Vorgänger besitzt (s.~u.). \\
    Ein gerichteter Graph heißt \begriff{Baum}, falls er eine Wurzel $w$
    besitzt und jeder Knoten außer der Wurzel genau einen direkten Vorgänger
    hat, d.~h. für $x \in V$, $x \not= w$ gibt es genau ein $y \in V$ mit
    $(y, x) \in E$.
    $y$ heißt \begriff{Vater}/\begriff{direkter Vorgänger}.
    $x$ ist dann das \begriff{Kind}/\begriff{direkter Nachfolger}. \\
    Ein Knoten ohne direkten Nachfolger heißt \begriff{Blatt}. \\
    Ein Graph heißt \begriff{Wald}, wenn er sich als disjunkte Vereinigung
    von Bäumen schreiben lässt. \\
    Ein Baum mit $n$ Knoten besitzt $n - 1$ Kanten.
    Knoten mit gleichem Vater heißen \begriff{Geschwister}.
    Knoten, die auf dem Weg von der Wurzel $w$ zu einem Knoten $v$
    liegen, heißen \begriff{Vorgänger} von $v$.
    Die Länge des längsten Wegs von der Wurzel $w$ zu einem Blatt ist die
    \begriff{Tiefe}/\begriff{Höhe} des Baums. \\
    Jedem Knoten ist ebenfalls eindeutig ein \begriff{Level}/\begriff{Niveau}
    zugeordnet: $w$ hat den Level $0$, die direkten Nachfolger eines Knoten
    mit Level $k$ haben den Level $k + 1$.
\end{Def}

\begin{Def}{Rekursive Definition für Bäume}
    Die leere Menge ist ein Baum. Wenn $w$ ein Knoten und $U$ eine endliche
    Menge von Bäumen sind, dann ist auch $w(U)$ ein Baum. \\
    $w$ heißt \begriff{Wurzel} von $w(U)$, die Elemente von $U$ heißen
    \begriff{Unterbäume}. Sind die Unterbäume geordnet, so spricht man von
    einem \begriff{geordneten Baum}.
\end{Def}

\begin{Def}{Binäre Bäume}
    Die leere Menge ist ein \begriff{binärer Baum}.
    Wenn $w$ ein Knoten und $B_L$ sowie
    $B_R$ binäre Bäume sind, dann ist auch $w(B_L, B_R)$ ein binärer Baum. \\
    $B_L$/$B_R$ heißen linker/rechter Unterbaum des Knotens $w$.
    Ein binärer Baum ist ein geordneter Baum, in dem jeder Knoten
    genau zwei (evtl. leere) Unterbäume hat.
\end{Def}

\begin{Def}{Suchbaum}
    Ein binärer Baum mit Knoten, die Werte eines geordneten Datentyps
    beinhalten, heißt \begriff{Suchbaum}, falls für jeden Knoten $u$ gilt:
    Alle Inhalte von Knoten im linken Unterbaum von $u$ sind echt kleiner als
    der Inhalt von $u$ und alle Inhalte von Knoten im rechten Unterbaum von $u$
    sind größer/gleich dem Inhalt von $u$.
\end{Def}

\pagebreak

\begin{Def}{Durchlauf von binären Bäumen}

    \begin{tabular}{lll}
        \begriff{Inorder} & \begriff{Preorder} & \begriff{Postorder} \\
\begin{lstlisting}[language=ada]
procedure Inorder (b : Ref_BinBaum)
is begin
   if b /= null then
      Inorder (b.L);
      --  hier Knoten b bearbeiten
      Inorder (b.R);
   end if;
end Inorder;
\end{lstlisting} &
\begin{lstlisting}[language=ada]
--  hier Knoten b bearbeiten
Preorder (b.L);
Preorder (b.R);
\end{lstlisting} &
\begin{lstlisting}[language=ada]
Postorder (b.L);
Postorder (b.R);
--  hier Knoten b bearbeiten
\end{lstlisting}
    \end{tabular}
    
    Ist $n$ die Zahl der Knoten, so erfolgt der Durchlauf in $3n$ Schritten. \\
    Im ungünstigsten Fall benötigt man $n$ Speicherplätze, im günstigsten
    proportional zu $\log n$.
\end{Def}

\subsection{%
    Relationen und Graphen%
}

\begin{Def}{Relation}
    Seien $M, M_1 \ldots, M_n$ Mengen.
    Eine Teilmenge $R \subseteq M_1 \times \cdots \times M_n$ heißt
    \begriff{$n$-stellige Korrespondenz}.
    Eine Teilmenge $R \subseteq M^n$ heißt
    \begriff{$n$-stellige Relation über $M$}.
    Eine Teilmenge $R \subseteq M^2$ heißt
    \begriff{(binäre) Relation über $M$}.
    Für $(x,y) \in R$ schreibt man $xRy$.
\end{Def}

\begin{Def}{Eigenschaften}
    Sei $R \subseteq M \times M$ eine Relation.
    $R$ heißt \begriff{reflexiv}, falls $\forall_{x \in M}\; xRx$.
    $R$ heißt \begriff{irreflexiv}, falls $\forall_{x \in M}\; \lnot(xRx)$.
    $R$ heißt \begriff{symmetrisch}, falls
    $\forall_{x, y \in M}\; (xRy \Leftrightarrow yRx)$.
    $R$ heißt \begriff{antisymmetrisch}, falls
    $\forall_{x, y \in M}\; (xRy \land yRx \Rightarrow x = y)$.
    $R$ heißt \begriff{transitiv}, falls \\
    $\forall_{x, y, z \in M}\; (xRy \land yRz \Rightarrow xRz)$.
    $R$ heißt \begriff{alternativ}, falls
    $\forall_{x, y \in M}\; (xRy \lor yRx)$.
\end{Def}

\begin{Def}{Relationsarten}
    Eine reflexive, symmetrische und transitive Relation heißt
    \begriff{Äquivalenzrela\-tion}.
    Eine reflexive, antisymmetrische und transitive Relation heißt
    \begriff{Ordnung}.
    Eine irreflexive, antisymmetrische und transitive Relation heißt
    \begriff{echte Ordnung}.
    Eine Ordnung heißt \begriff{totale/line\-are Ordnung}, falls sie alternativ
    ist.
\end{Def}

\begin{Def}{Gerichteter Graph}
    $G = (V, E)$ heißt \begriff{gerichteter Graph}/\begriff{Digraph}, falls
    $V$ eine nicht-leere endliche Menge ist (\begriff{Knoten}) und
    $E \subseteq V \times V$ (\begriff{Kanten}).
\end{Def}

\begin{Def}{Ungerichteter Graph}
    $G = (V, E)$ heißt \begriff{ungerichteter Graph}, falls
    $V$ eine nicht-leere endliche Menge ist (\begriff{Knoten}) und
    $E \subseteq \big\{\{x, y\} \;|\; x, y \in V, x \not= v\big\} \cup
    \big\{\{x\} \;|\; x \in V\big\}$ (\begriff{Kanten}).
\end{Def}

\begin{Def}{Umwandeln von Graphen}
    Ist $G = (V, E)$ ein ungerichteter Graph, so ist der gerichtete
    Graph $G_{ger} = (V, E_{ger})$ mit
    $E_{ger} = \big\{(x, y), (y, x) \;|\; \{x, y\} \in E\big\} \cup
    \big\{(x, x) \;|\; \{x\} \in E\big\}$
    die \begriff{gerichtete Version} des Graphen $G$. \\
    Ist $G = (V, E)$ ein gerichteter Graph, so ist der ungerichtete
    Graph $G_{ung} = (V, E_{ung})$ mit
    $E_{ung} = \big\{\{x, y\} \;|\; (x, y) \in E \lor (y, x) \in E\big\} \cup
    \big\{\{x\} \;|\; (x, x) \in E\big\}$
    die \begriff{ungerichtete Version} des Graphen $G$. \qquad
    Ein gerichteter Graph $H$ heißt
    \begriff{Orientierung}/\begriff{Ausrichtung} des ungerichteten Graphen $G$,
    falls $G$ die ungerichtete Version von $H$ ist.
\end{Def}

\begin{Def}{Teilgraph}
    Ist $G = (V, E)$ ein Graph, so heißt ein Graph
    $G' = (V', E')$ \begriff{Teilgraph} von $G$, falls $V' \subseteq V$ und
    $E' \subseteq E$. \qquad
    $G' = (V', E')$ heißt \begriff{der von $V'$ induzierte Teilgraph}, falls
    im \\
    ungerichteten Fall
    $E' = \big\{\{x, y\} \in E \;|\; x, y \in V'\big\}$ bzw.
    im gerichteten Fall \\
    $E' = \big\{(x, y) \in E \;|\; x, y \in V'\big\}$.
\end{Def}

\begin{Def}{Nachbarn}
    Jede Kante $\{x, y\}$ bzw. $(x, y)$ heißt \begriff{inzident} zu ihren
    Knoten $x$ und $y$.
    Zwei Knoten $x, y$ mit $\{x, y\} \in E$ bzw. $(x, y) \in E$
    (oder $(y, x) \in E$) heißen \begriff{adjazent}/\begriff{benachbart}. \\
    Die Menge $N(x) = \big\{y \in V \;|\; \{x, y\} \in E\big\}$ bzw.
    $N(x) = \big\{y \in V \;|\; (x, y) \in E \lor (y, x) \in E\big\}$
    heißt die Menge der (direkten) \begriff{Nachbarn} von $x$.
    Ist $G$ gerichtet, so heißt
    $S(x) = \big\{y \in V \;|\; (x, y) \in E\big\}$ bzw.
    $P(x) = \big\{y \in V \;|\; (y, x) \in E\big\}$ die Menge der
    (direkten) \begriff{Nachfolger} bzw. \begriff{Vorgänger} von $x$.
    Eine Kante $\{x\}$ bzw. $(x, x)$ heißt \begriff{Schlinge}.
\end{Def}

\begin{Def}{Grad}
    Ist $G$ ungerichtet, so heißt
    $d(x) = |N(x) \setminus \{x\}|$
    ($+2$ für $\{x\} \in E$) der \begriff{(Knoten-)Grad} von $x$.
    Der maximale Knotengrad heißt \begriff{Grad} $d(G)$ des Graphen $G$. \\
    Ist $G$ gerichtet, so heißen
    $d^{\boldsymbol{+}}(x) = |S(x)|$ \begriff{Ausgangs-} und
    $d^{\boldsymbol{-}}(x) = |P(x)|$ \begriff{Eingangsgrad} von $x$.
    $d(x) = d^{\boldsymbol{+}}(x) + d^{\boldsymbol{-}}(x)$ heißt
    \begriff{(Knoten-)Grad} von $x$. \\
    Ein Graph heißt \begriff{geordnet}, falls für jeden Knoten $x$
    die Menge der Nachbarn $N(x)$ (ungerichteter Fall) bzw.
    die Menge der Nachfolger $S(x)$ (gerichteter Fall)
    linear geordnet ist.
\end{Def}

\begin{Def}{Adjazenzmatrix}
    Sei $G = (V, E)$ mit $V = \{x_1, \ldots, x_n\}$ ein Graph.
    Die \begriff{Adjazenzmatrix} $A = (a_{ij})$ ist definiert durch
    $a_{ij} = 1$ (ggf. Kantengewicht) falls $\{x_i, x_j\} \in E$ (ungerichtet)
    bzw. $(x_i, x_j) \in E$ (gerichtet) und $a_{ij} = 0$ sonst
    ($1 \le i,j \le n$). \\
    Die \begriff{erweiterte Adjazenzmatrix}
    $A' = (a_{ij}')$ ist
    $a_{ij}' = a_{ij}$ für $i \not= j$ und $a_{ij} = 1$ für $i = j$. \\
    ($A^k$ gibt die Anzahl der verschiedenen Wege der Länge $k$ zwischen zwei
    Knoten an.)
\end{Def}

\begin{Def}{Adjazenzliste}
    Man erstellt eine Liste der Knoten (mit Inhalt und ID-Nummer).
    Jeder Knoten enthält wieder eine Liste der inzidenten Kanten,
    deren Einträge das Gewicht und einen Verweis auf den Endknoten enthalten.
\end{Def}

\begin{Def}{Inzidenzliste}
    Man erstellt jeweils eine (lineare) Liste der Knoten und eine Liste der
    Kanten. \\
    Die Einträge der Kanten entalten zwei Verweise auf die zugehörigen Knoten.
\end{Def}

\begin{Def}{Graphendurchlauf}
    Ein \begriff{Graphendurchlauf} lässt sich durch Adjazenzlisten realisieren.
    Man kann alle Knoten nacheinander durchgehen und bei jedem Knoten
    die entsprechenden Kanten ablaufen, um den Algorithmus mit dem Zielknoten
    rekursiv aufzurufen.
\end{Def}

\begin{Def}{Ungerichtete Wege und Pfade}
    Eine Folge von Knoten $(u_1, \ldots, u_r)$ ($r \ge 1$) heißt \begriff{Weg}
    in $G$, falls $\{u_i, u_{i+1}\} \in E$ für $i = 1, \ldots, r-1$.
    $r-1$ heißt die \begriff{Länge des Weges}.
    $u$ und $v$ heißen \begriff{verbunden}, falls es einen Weg von $u$ nach
    $v$ gibt.
    Der zugehörige \begriff{Pfad} des Wegs ist
    $(\{u_1, u_2\}, \ldots, \{u_{r-1}, u_r\})$.
\end{Def}

\begin{Def}{Zusammenhang für ungerichtete Graphen}
    Ein Weg heißt \begriff{doppelpunktfrei}/\begriff{einfach}, falls
    $u_i \not= u_j$ für $i \not= j$.
    Ein Weg heißt \begriff{geschlossen}, falls $u_r = u_1$.
    Ein Weg heißt \begriff{Kreis}/\begriff{Zyklus}, falls
    $r \ge 4$, der Weg geschlossen ist und $(u_1, \ldots, u_{r-1})$ einfach.
    Ein Graph heißt \begriff{zyklenfrei}/\begriff{azyklisch}, falls er keine
    Zyklen besitzt. \\
    Ein Graph heißt \begriff{zusammenhängend}, falls jeder Knoten mit jedem
    Knoten verbunden ist. \\
    $Z(u) = \{v \in V \;|\; u, v \text{ sind verbunden}\}$ heißt
    \begriff{Zusammenhangskomponente} des Knotens $u$.
\end{Def}

\begin{Def}{Gerichtete Wege und Pfade}
    Eine Folge von Knoten $(u_1, \ldots, u_r)$ ($r \ge 1$) heißt\\
    \begriff{(gerichteter) Weg} in $G$, falls $(u_i, u_{i+1}) \in E$ für
    $i = 1, \ldots, r-1$.
    $r-1$ heißt die \begriff{Länge des Weges}.
    $u$ und $v$ heißen \begriff{verbunden}, falls es einen Weg von $u$ nach
    $v$ \emph{und} von $v$ nach $u$ gibt.
    Der zugehörige \begriff{Pfad} des Wegs ist
    $((u_1, u_2), \ldots, (u_{r-1}, u_r))$.
\end{Def}

\begin{Def}{Zusammenhang für gerichtete Graphen}
    Ein Weg heißt \begriff{doppelpunktfrei}/\begriff{einfach}, falls
    $u_i \not= u_j$ für $i \not= j$.
    Ein Weg heißt \begriff{geschlossen}, falls $u_r = u_1$.
    Ein Weg heißt \begriff{Kreis}/\begriff{Zyklus}, falls
    $r \ge 4$, der Weg geschlossen ist und $(u_1, \ldots, u_{r-1})$ einfach.
    Ein Graph heißt \begriff{zyklenfrei}/\begriff{azyklisch}, falls er keine
    Zyklen besitzt. \\
    Ein Graph heißt \begriff{stark zusammenhängend}, falls jeder Knoten mit
    jedem Knoten verbunden ist.
    $Z(u) = \{v \in V \;|\; u, v \text{ sind verbunden}\}$ heißt
    \begriff{starke Zusammenhangskomponente} des Knotens $u$.
    $SwZ(u) = Z(u) \text{ in } G_{ung}$ heißt
    \begriff{schwache Zusammenhangskomponente}.
\end{Def}

\begin{Def}{Transitive Hülle}
    Zu einem gerichteten bzw. ungerichteten Graphen $G = (V, E)$ heißt der
    gerichtete bzw. ungerichtete Graph $G_{tH} = (V, E_{tH})$ mit \\
    $E_{tH} = \big\{(x, y) \;|\; \text{es gibt einen Weg von } x
    \text{ nach } y\big\}$ bzw. \\
    $E_{tH} = \big\{\{x, y\} \;|\; \text{es gibt einen Weg von } x
    \text{ nach } y\big\}$
    die \begriff{transitive Hülle} des Graphen $G$. \\
    Ein \begriff{vollständiger Graph} ist ein Graph, in dem zwei verschiedene
    Knoten durch nur eine Kante miteinander verbunden sind.
    Ein Graph, bei dem $n$ Knoten einen Ring bilden, heißt \begriff{Kreis}.
\end{Def}
