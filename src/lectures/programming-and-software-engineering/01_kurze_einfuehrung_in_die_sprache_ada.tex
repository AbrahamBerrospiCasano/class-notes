\chapter{%
    Kurze Einführung in die Sprache \Ada{}%
}

\section{%
    Grundsätzliches%
}

\begin{Def}{Techniken des imperativen Programmierens}
    Verfahren wird durch Folge von \begriff{Anwei\-sungen} beschrieben;
    Daten werden in strukturierten Behältern (\begriff{Variablen}) abgelegt,
    zu jedem Behälter gibt ein Datentyp an, was hineingelegt werden darf;
    \begriff{Zuweisung} als grundlegende Anweisung;
    Festlegung der Bearbeitungsreihenfolge durch \begriff{Kontrollstrukturen};
    Zusammenfassung von Teilverfahren zu \begriff{Prozeduren}.
\end{Def}

\begin{Def}{Variablen und Zuweisung}
    Variable als Behälter, zu jedem Behälter gehört ein Kontrollmechanismus,
    der den Datentyp des zugewiesenen Werts prüft; \\
    elementare Datentypen: Integer, Natural, Float, Char sowie Boolean.
\end{Def}

\section{%
    Diszipliniertes Vorgehen%
}

\begin{Def}{Vorgehen beim "`Programmieren im Kleinen"'}
    Ideen umgangssprachlich aufschreiben;
    präzise Formulierung;
    Ermitteln von Eigenschaften;
    Entwicklung eines Algorithmus;
    Implementierung in ein Programm;
    Testen \& Messen;
    kritisches Überdenken.
\end{Def}

\section{%
    Programmaufbau in \Ada{}%
}

\begin{Def}{Programmaufbau} \\
    \begin{tabular}{p{6.0cm}p{9.8cm}}
\vspace{-30pt}
\begin{lstlisting}[language=ada]
<with/use-Bereich>
procedure <Name> is
   <Deklarationsteil>
begin
   <Anweisungsteil>
end <Name>;
\end{lstlisting}
\vspace{-25pt}
        &
        \begin{minipage}[c]{9.8cm}
    \adacode{<abc>} ist ein \begriff{Platzhalter}, \begriff{Kommentare} durch
    \adacode{--} bis zum Zeilenende, \begriff{Name} besteht aus mindestens
    einem \begriff{Bezeichner} (Folge von Buchstaben, Ziffern und
    Unterstrichen, die mit einem Buchstaben beginnt), keine Unterscheidung der
    Groß-/Kleinschreibung, Schlüsselwörter dürfen nicht als Name verwendet
    werden (Datentypen aber schon)
        \end{minipage}
    \end{tabular}
\end{Def}

\begin{Def}{Deklarationsteil}
    enthält \begriff{Variablen-}
    (\adacode{<Liste von Variablen> : <Typ>;}) sowie\\
    \begriff{Konstantendeklarationen}
    (\adacode{<Name> : constant <Typ> := <Ausdruck>;}),
    bei \adacode{Integer} und \adacode{Float} darf der Datentyp weggelassen
    werden),
    zu den wichtigsten skalaren Datentypen gehören \adacode{Boolean},
    \adacode{Integer}, \adacode{Float} sowie \adacode{Character}
\end{Def}

\begin{Def}{Anweisungsteil}
    endliche Folge von elementaren und strukturierten Anweisungen, wie \\
    \begriff{Zuweisung} (\adacode{<Variable> := <Ausdruck>;}),
    \begriff{No-Op} (\adacode{null;}) und
    \begriff{Ein-/Ausgabe} \\
    (\adacode{Get (<Variable>);}, \adacode{Put (<Variable>);})
\end{Def}

\pagebreak

\section{%
    Funktionen und Prozeduren%
}

\begin{Def}{Funktionsdeklaration} \\
    \begin{tabular}{p{7.6cm}p{8.3cm}}
\vspace{-30pt}
\begin{lstlisting}[language=ada]
function <Name> (<Parameter>)
  return <Datentyp> is
   <Deklarationsteil>
begin
   <Anweisungsteil>
end <Name>;
\end{lstlisting}
\vspace{-25pt}
        &
        \begin{minipage}[c]{8.3cm}
    in der Funktion muss man auf eine elementare Anweisung der Form
    \adacode{return <Ausdruck>;} stoßen,
    die Liste der \begriff{formalen Parameter} enthält
    die Parameter mit Name und Datentyp getrennt durch Semikolon
        \end{minipage}
    \end{tabular}
\end{Def}

\begin{Def}{Prozedurdeklaration} \\
    \begin{tabular}{p{8.0cm}p{8.0cm}}
\vspace{-30pt}
\begin{lstlisting}[language=ada]
procedure <Name> (<Parameter>) is
   <Deklarationsteil>
begin
   <Anweisungsteil>
end <Name>;
\end{lstlisting}
\vspace{-25pt}
        &
        \begin{minipage}[c]{8.0cm}
    die Prozedur kann vorzeitig über \adacode{return;} verlassen werden,
    wird als Anweisung verwendet, wie bei Funktion heißt der Teil bis zum
    \adacode{is} \begriff{Kopf}, der Rest \begriff{Rumpf}
        \end{minipage}
    \end{tabular}
\end{Def}

\begin{Def}{Rekursion}
    \begriff{Rekursion} ist die (in)direkte Verwendung einer
    Funktion in ihrem Rumpf.%
\end{Def}

\begin{Def}{Operatoren}
    \begriff{Operatoren} sind spezielle Funktionen (in der Regel ein- oder
    zweistellig) und besitzen statt eines Namens ein Operatorsymbol,
    in \Ada{} können Operatoren wie Funktionen deklariert werden
    (z.~B. \adacode{function "+"} usw.)
\end{Def}

\pagebreak

\section{%
    Skalare Datentypen%
}

\begin{Def}{Datentyp}
    Eine Menge zusammen mit der auf ihr definierten Operationen heißt
    \begriff{Datentyp}, es gibt \begriff{skalare} (Datentypen, die man nicht
    auf andere zurückführt, wie vordefinierte und Aufzählungs-Datentypen) und
    \begriff{zusammengesetzte Datentypen}
\end{Def}

\begin{Def}{Aufzählungstyp}
    \adacode{type <Name des Datentyps> is (<Liste der Elemente>);} \\
    (der Aufzählungstyp ist geordnet, d.~h. die Reihenfolge ordnet die Elemente
    an) \\
    Operationen (\begriff{Attribute}): nullstellig (alle Elemente von
    \adacode{T}, \adacode{T'First}, \adacode{T'Last}, \adacode{T'(Sa)}), \\
    einstellig (\adacode{T'Pred (X)}, \adacode{T'Succ (X)},
    \adacode{T'Pos (X)}, \adacode{T'Val (I)}) sowie \\
    zweistellig (\adacode{=}, \adacode{/=}, \adacode{<}, \adacode{<=},
    \adacode{>}, \adacode{>=}, \adacode{T'Min (X, Y)}, \adacode{T'Max (X, Y)})
\end{Def}

\begin{Def}{Standard-Datentypen}
    \adacode{Boolean},
    \adacode{Character},
    \adacode{Integer} (\adacode{Natural}, \adacode{Positive}),
    \adacode{Float}
\end{Def}

\begin{Def}{\adacode{Boolean}}
    Wertebereich \adacode{False}/\adacode{True},
    Operationen \adacode{not}, \adacode{and}, \adacode{or}, \adacode{xor},
    \adacode{=}, \adacode{and then},\\
    \adacode{or else},
    Klammerung bei logischen Ausdrücken wird empfohlen
    (teilweise verpflichtend!)
\end{Def}

\begin{Def}{\adacode{Character}}
    wird als Aufzählungstyp mit den Zeichen aus ISO-8859-1 (Latin-1)
    als Wertebereich aufgefasst, Zeichen werden in Apostrophe eingeschlossen
    (Apostroph durch zwei Apostrophe),
    Operationen: wie bei allen Aufzählungstypen
\end{Def}

\begin{Def}{\adacode{Integer}}
    wird als Aufzählungstyp von \adacode{Integer'First} bis
    \adacode{Integer'Last} (rechnerabhängig) aufgefasst,
    Operationen: alle Konstanten, \adacode{+}, \adacode{-}, \adacode{abs}
    (alle einstellig), \adacode{+}, \adacode{-}, \adacode{*}, \adacode{mod}
    (nicht-negativ), \adacode{/} (ganzzahlige Division, Abschneiden der
    Dezimalen), \adacode{rem} (auch negativ), \adacode{**},
    \adacode{=}, \adacode{/=}, \adacode{<}, \adacode{<=}, \adacode{>},
    \adacode{>=}, \adacode{Min}, \adacode{Max}, \adacode{Pred}, \adacode{Succ},
    \adacode{Pos}, \adacode{Val}
    (\emph{Ausnahme}: \adacode{Integer'Pos (I) = I})
\end{Def}

\begin{Def}{\adacode{Float}}
    eigene Floattypen festlegen
    durch Angabe von Ober-/Untergrenze \\
    (\adacode{type <Datentypname> is delta d range unten .. oben;})
    oder durch Angabe \\
    der Dezimalziffern
    (\adacode{type <Datentypname> is delta p digits a;}, hier ist \adacode{p}
    eine Zehnerpotenz),
    der eingebaute Typ \adacode{Float} besitzt eine Genauigkeit von mindestens
    6 Dezimalziffern (eigene Genauigkeit durch
    \adacode{type <Datentypname> is digits a;}),
    Operationen: \adacode{+}, \adacode{-}, \adacode{abs} (alle einstellig),
    \adacode{+}, \adacode{-}, \adacode{*}, \adacode{/}, \adacode{**}
    (rechts darf nur eine ganze Zahl stehen), \adacode{=}, \adacode{/=},
    \adacode{<}, \adacode{<=}, \adacode{>}, \adacode{>=}
    (Vermeidung von \adacode{=}, \adacode{/=} aufgrund Rundungsfehlern),
    ausnahmsweise können ganze mit reellen Zahlen multipliziert
    und reelle Zahlen durch ganze dividiert werden
\end{Def}

\begin{Def}{Typumwandlung}
    \Ada{} ist streng typisiert, d.~h. es können meist nur zwei Operanden
    gleichen Typs verrechnet werden.
    Abhilfe schaffen die Funktionen \adacode{Float (I);} sowie\\
    \adacode{Integer (X);} (\emph{rundet zur nächsten ganzen Zahl!}).
\end{Def}

\begin{Def}{Initialisierung}
    Variablen können bei der Deklaration durch \\
    \adacode{<Liste von Bezeichnern> : <Datentyp> := <Ausdruck>;}
    initialisiert werden. \\
    Dies gilt auch für Aufzählungstypen, Arrays und Records.
\end{Def}

\begin{Def}{Ein-/Ausgabe}
    erfolgt über die Paket \adacode{Ada.Text\_IO},
    \adacode{Ada.Integer\_Text\_IO}, und \\
    \adacode{Ada.Float\_Text\_IO}
    (über \adacode{with} und ggf. \adacode{use} einbinden)
\end{Def}

\pagebreak

\section{%
    Felder%
}

\begin{Def}{Unterbereiche}
    In \Ada{} werden Unterbereiche durch \adacode{subtype} mittels \\
    \adacode{subtype <Name> is <Datentyp> range <unten> .. <oben>;}
    deklariert. \\
    \adacode{range <unten> .. <oben>} kann weggelassen werden, in diesem Fall
    besitzt der Unterbereich die gleiche Wertemenge wie der Datentyp.
    Alle Operationen des Datentyps werden vom Unterbereich übernommen.
    Ein Variable vom Typ eines Unterbereichs wird als Variable des Basistyps
    aufgefasst, erst bei der Zuweisung wird die Einschränkung geprüft. \\
    Vorderfinierte Unterbereiche sind \adacode{Natural} und
    \adacode{Positive}. \\
    Unterbreiche kann man auch für \adacode{Float} einführen, sofern die
    Grenzen reellwertig sind.
\end{Def}

\begin{Def}{Array}
    Arraytypen können durch \\
    \adacode{type <Feldname> is array (<Indexdatentyp>) of <Datentyp>;}
    deklariert werden. \\
    Mehrdimensionale Arrays können durch Angabe von zwei kommagetrennten
    Indexdatentypen deklariert werden (z. B.
    \adacode{type Matrix is array (1 .. 10, 1 .. 50) of Float;}). \\
    Bei einem \begriff{statischen Feld} sind alle Feldgrenzen zur
    Übersetzungszeit bekannt, sonst heißt das Feld \begriff{dynamisch}
    (in \Ada{} können die Feldgrenzen aber nicht verändert werden).
\end{Def}

\begin{Def}{Unspezifizierte Feldgrenzen}
    bei Typdeklaration ist es erlaubt, die Feldgrenzen wegzulassen und
    stattdessen \adacode{<Datentyp> range <>} zu schreiben \\
    (z.~B. \adacode{type Text is array (Natural range <>) of Character;}). \\
    Die Feldgrenzen müssen jedoch bei der Variablendeklaration angegeben
    werden.
    Operationen: \adacode{T'Range (I)} (gibt den Felddatentyp der
    \adacode{I}-ten Dimension an, bei eindimensionalen Arrays kann
    \adacode{(1)} weggelassen werden), \adacode{T'Length (I)} (Anzahl der
    Elemente), \adacode{T'First (I)}, \adacode{T'Last (I)} (erstes/letztes
    Element des Indexdatentyps)
\end{Def}

\section{%
    BNF und EBNF%
}

\begin{Def}{BNF (\name{Backus}-\name{Naur}-Form)}
    Eine BNF ist ein Viertupel $(V, \Sigma, P, S)$ mit
    \begin{itemize}
        \item $V$ nicht-leere endliche Menge der Form \ebnfcode{<Zeichenkette>}
        (\begriff{Nichtterminalzeichen}),

        \item $\Sigma$ nicht-leere endliche Menge mit
        $V \cap \Sigma = \emptyset$ und $| \notin \Sigma$
        (\begriff{Terminalzeichen}),

        \item $P$ endliche Menge (\begriff{Regeln} oder
        \begriff{Produktionen}), zu jedem Nichtterminalzeichen $A \in V$ gibt
        es genau eine Regel aus $P$ mit der Form
        \ebnfcode{A ::= u1 | ... | uk}, wobei jedes \ebnfcode{ui} eine
        Folge von Zeichen aus $V \cup \Sigma$ ist
        (die leere Zeichenfolge $\varepsilon$ ist als ein \ebnfcode{ui}
        zugelassen),

        \item $S \in V$ (\begriff{Startsymbol}).
    \end{itemize}
\end{Def}

\begin{Def}{EBNF (erweiterte BNF)}
    die BNF wird um folgende Möglichkeiten erweitert:
    \begin{itemize}
        \item \begriff{Verwenden von Schlüsselwörtern}, die in Apostrophe
        eingeschlossen werden (z.~B. \\
        \ebnfcode{<Operatoren> ::= 'and' | 'or'}), die dann als
        Terminalzeichen aufgefasst werden

        \item \begriff{Einführen von eckigen Klammern}, deren Inhalt auch
        ausgelassen werden kann (z.~B. \\
        \ebnfcode{<Ziffernfolge> ::= <Ziffer> [<Ziffernfolge>]})

        \item \begriff{Einführen von geschweiften Klammern}, deren Inhalt
        beliebig oft wiederholt und ausgelassen werden kann (z.~B.
        \ebnfcode{<Ziffernfolge> ::= <Ziffer> \{<Ziffer>\}})
    \end{itemize}
\end{Def}

\pagebreak

\section{%
    Kontrollstrukturen%
}

\begin{tabular}{p{8.2cm}p{7.9cm}}
\begin{Def}{Fallunterscheidung}
\begin{lstlisting}[language=ada]
if Ausdruck1 then <Anweisungsfolge>
elsif Ausdruck2 then <Anweisungsfolge>
elsif Ausdruck3 then <Anweisungsfolge>
else <Anweisungsfolge>
end if;
\end{lstlisting}
\end{Def}
\vspace{-20pt}
&
\begin{Def}{Auswahlanweisung}
\begin{lstlisting}[language=ada]
case Tag is
   when Mo | Di => <Anweisungsfolge>
   when Mi .. Fr => <Anweisungsfolge>
   when others => <Anweisungsfolge>
end case;
\end{lstlisting}
\end{Def}
\vspace{-20pt}
\end{tabular}

Die Auswahlmöglichkeiten bei \adacode{case} müssen disjunkt und vollständig
sein!
Von \Ada{} werden alle Möglichkeiten ausgewertet, die
\adacode{case}-Anweisung wird nur dann ausgeführt, wenn genau einmal der Wert
\adacode{True} vorkam.

\begin{tabular}{p{4.1cm}p{4.1cm}p{7.5cm}}
\begin{Def}{\adacode{while}-Schleife}
\begin{lstlisting}[language=ada]
while B loop
    <Anweisungsfolge>
end loop;
\end{lstlisting}
\end{Def}
&
\begin{Def}{\adacode{for}-Schleife}
\begin{lstlisting}[language=ada]
for I in 1 .. 10 loop
    <Anweisungsfolge>
end loop;
\end{lstlisting}
\end{Def}
&
    Mit \adacode{reverse} wird der Bereich der Laufvariablen rückwärts
    durchlaufen.
    Mit dem Befehl \adacode{exit when B;} wird die aktuelle Schleife verlassen,
    falls \adacode{B} zutrifft (\adacode{when B} kann weggelassen werden).
\end{tabular}

\adacode{loop <Anweisungsfolge> end loop;} erzeugt eine Endlosschleife.
Wird eine Schleife mit einem Namen versehen (z.~B.
\adacode{Schleife1: while B loop}), so kann sie mit \\
\adacode{exit Schleife1 when B;} verlassen werden
(\adacode{when B} kann weggelassen werden). \\
Mittels \adacode{exit} dürfen allerdings keine Blöcke verlassen werden.

\begin{Def}{Block}
\begin{lstlisting}[language=ebnf,
emph={declare,begin,end},
emphstyle=\underbar]
<Block> ::= [<Blockname>:] [declare <Deklarationsteil>]
  begin
     <volle_Anweisungsfolge>
  end [<Blockname>];
\end{lstlisting}
    Blöcke liefern die zentrale Grundstruktur von Programmen in \Ada{}.
    Mittels \adacode{declare}-Blöcken können dynamische Felder erstellt werden.
    Alle Variablen des Blocks unterliegen \begriff{Lebensdauer} und
    \begriff{Sichtbarkeit}
    (nicht sichtbare Objekte können mittels
    \adacode{<Blockname>.<Bezeichner>}
    sichtbar gemacht werden).
\end{Def}

\begin{Def}{Sprunganweisung}
    Mittels \adacode{goto abc;} springt man an die Anweisung des Programms,
    die mit \adacode{abc} markiert ist (z.~B. \adacode{<<abc>> null;}).
    Mit \adacode{goto} darf man nicht in eine struktuierte Anweisung oder
    aus einem Block heraus springen.
\end{Def}

\pagebreak

\section{%
    Records%
}

\begin{Def}{Records} \\
\begin{tabular}{@{}p{5.7cm}p{10.8cm}@{}}
\vspace{-10pt}
\begin{lstlisting}[language=ada]
type Datum is record
   Jahr : Integer := 2010;
   Monat : Monatsname;
   Tag : Integer range 1 .. 31;
end record;
\end{lstlisting}
\vspace{-10pt}
&
\vspace{-10pt}
\begin{lstlisting}[language=ada]
--  Tag, Monat und Jahr heissen Selektoren
WM_Beginn : Datum := (2010, Juni, 11);
NJ : Datum := Datum'(2011, Tag => 1, Monat => Januar);
--  Datum'/Verbundaggregat kann man weglassen
\end{lstlisting}
\vspace{-10pt}
\end{tabular}

    Einen leeren Verbund kann man mit
    \adacode{type Leerer\_Verbund is null record;} deklarieren.
    Mit \begriff{Verbundaggregaten} kann man einen ganzen Record auf einmal
    befüllen (entweder Liste von Werten, Liste von Werten mit Angabe der
    Selektoren oder Liste der Werte, danach Liste mit Angabe der Selektoren).
    \adacode{others} am Ende befüllt die restlichen Komponenten. \\
    Vorbesetzungen der Komponenten sind durch
    \begriff{Verbund-Initialisierungen} möglich. \\
    Ist der Datentyp nicht eindeutig, so ist \begriff{Typ-Qualifizierung}
    (\adacode{Datum'}) erforderlich.
\end{Def}

\begin{Def}{Variante Records} \\
\begin{tabular}{p{8.0cm}p{8.0cm}}
\vspace{-10pt}
\begin{lstlisting}[language=ada]
type SL is (D, EU, sonst);
type Student (Herkunft : SL) is record
   Name : Unbounded_String;
   Matrikelnummer : Positive;
   case Herkunft is
      when D => Ort : Unbounded_String;
      when EU | sonst =>
              Land : Unbounded_String;
   end case;
end record;
\end{lstlisting}
\vspace{-10pt}
&
\vspace{-10pt}
\begin{lstlisting}[language=ada]
type Kategorie is (PKW, Bus, Karren);
type Fahrzeug (Art : Kategorie := PKW)
        is record
   Laenge, Breite, Hoehe : Float;
   case Art is
      when Bus => record Sitzpl : 8 .. 60;
        Stehpl : 0 .. 80; end record;
      when PKW => Airbags : Positive;
      when Karren => null record;
   end case;
end record;
\end{lstlisting}
\vspace{-10pt}
\end{tabular}

    \begriff{Variante Records} bekommen bei ihrer Initialisierung eine sog.
    \begriff{Diskriminante} mitgegeben, von der abhängt, welche Komponenten
    der Record enthält.
    Man spricht von einem \begriff{varianten Anteil} des Records.
    Die Diskriminante darf eine Vorbesetzung enthalten
    (\adacode{Herkunft : SL := D}).
    Die Auflistung im varianten Teil muss vollständig und disjunkt sein,
    ein Selektor darf in einem gesamten Record höchstens einmal vorkommen.
    Diskriminanten können auch die Größe von Records beeinflussen
    (z.~B. übergibt man einen Integer-Wert als Diskriminante, dieser bestimmt
    die Größe eines Arrays).
\end{Def}
