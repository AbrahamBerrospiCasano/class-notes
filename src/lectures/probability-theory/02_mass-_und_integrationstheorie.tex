\section{%
    Maß- und Integrationstheorie%
}

\subsection{%
    Die erweiterte Zahlengerade\texorpdfstring{ $\extreal$}{}%
}

\begin{Def}{erweiterte Zahlengerade}
    Die \begriff{erweiterte Zahlengerade} ist definiert als
    $\extreal := \real \cup \{\pm \infty\}$.
    Die Operationen $+$ und $\cdot$ werden auf $\extreal$ erweitert durch
    $\infty + \infty := \infty$, $(-\infty) + (-\infty) := -\infty$,
    $-\infty < a < +\infty$ für alle $a \in \real$ usw.
    Für $c \in \real$ sei $c \cdot \infty := \infty$ für $c > 0$,
    $c \cdot \infty := -\infty$ für $c < 0$ und $c \cdot \infty := 0$ für $c = 0$.
    Damit ist das Produkt $a \cdot b$ für alle $a, b \in \extreal$ definiert.\\
    Ausdrücke wie $\infty - \infty$ werden undefiniert gelassen.
\end{Def}

\begin{Def}{Intervalle und Umgebungen}
    Intervalle $[-\infty, a)$, $[-\infty, a]$, $(a, \infty]$ und $[a, \infty]$ sind definiert
    durch $[-\infty, a) := \{-\infty\} \cup (-\infty, a)$ usw. mit $a \in \extreal$.
    Für $\varepsilon > 0$ sind $\varepsilon$-Umgebungen von $\pm \infty$ definiert durch
    $U_\varepsilon(\infty) := (\frac{1}{\varepsilon}, \infty]$ und
    $U_\varepsilon(-\infty) := [-\infty, -\frac{1}{\varepsilon})$.
    Dadurch sind auch of"|fene und abgeschlossene Teilmengen von $\extreal$ definiert.
\end{Def}

\begin{Bem}
    Eine Teilmenge von $\real$ ist of"|fen in $\extreal$ genau dann,
    wenn sie of"|fen in $\real$ ist.
    Mit "`abgeschlossen"' statt "`of"|fen"' stimmt die Aussage nicht mehr:
    Es gibt Teilmengen von $\real$, die zwar abgeschlossen in $\real$, aber nicht in $\extreal$
    abgeschlossen sind (z.\,B. $M = [0, \infty)$).
\end{Bem}

\subsection{%
    Die \name{Borel}-\texorpdfstring{$\sigma$}{σ}-Algebra%
}

\begin{Def}{erzeugte $\sigma$-Algebra}
    Seien $\Omega \not= \emptyset$ eine nicht-leere Menge und $\E \in \P(\Omega)$ ein
    System von Teilmengen.
    Dann gibt es eine kleinste $\sigma$-Algebra $\sigma(\E)$, die $\E$ enthält.
    Sie heißt \begriff{die von $\E$ erzeugte $\sigma$-Algebra}.
\end{Def}

\begin{Bsp}
    Sei $\Omega \not= \emptyset$ eine nicht-leere Menge.
    \begin{enumerate}
        \item
        Für $A \subset \Omega$ ist $\sigma(\{A\}) = \{\emptyset, A, \Omega \setminus A, \Omega\}$.
        
        \item
        Für eine Partition $(A_n)_{n \in \natural}$ von $\Omega$
        (d.\,h. $A_n \subset \Omega$ paarweise disjunkt und
        $\bigcup_{n \in \natural} A_n = \Omega$) gilt
        $\sigma(\{A_1, A_2, \dotsc\}) = \left\{\bigcup_{k \in K} A_k \;|\;
        K \subset \natural\right\}$.
    \end{enumerate}
\end{Bsp}

\begin{Satz}{Abschluss der $\sigma$-Algebra}
    Seien $\Omega \not= \emptyset$ eine nicht-leere Menge, $\E \in \P(\Omega)$ ein
    System von Teilmengen und $A, A_1, A_2, \dotsc \in \E$.
    Dann ist $\Omega \setminus A, \bigcup_{n=1}^\infty A_n,
    \bigcap_{n=1}^\infty A_n \in \sigma(\E)$.
\end{Satz}

\linie

\begin{Def}{\name{Borel}-$\sigma$-Algebra auf $\real$}
    Sei $\E(\real) := \{(a, b] \;|\; -\infty < a \le b < \infty\}$.\\
    Dann heißt $\B(\real) := \sigma(\E(\real))$ die
    \begriff{\name{Borel}-$\sigma$-Algebra auf $\real$}.\\
    Die Elemente von $\B(\real)$ heißen \begriff{\name{Borel}-Mengen}.
\end{Def}

\begin{Bsp}
    Beispiele für Borel-Mengen von $\real$ sind
    \begin{enumerate}
        \item
        die of"|fenen und abgeschlossenen Intervalle
        $(a, b) = \bigcup_{n=1}^\infty (a, b - \frac{1}{n}]$ und
        $[a, b] = \bigcap_{n=1}^\infty (a - \frac{1}{n}, b + \frac{1}{n})$,
        
        \item
        die Elementarereignisse $\{a\} = [a, a]$ (d.\,h. jede abzählbare Teilmenge von $\real$),
        
        \item
        alle of"|fenen und abgeschlossenen Teilmengen von $\real$
        (jede of"|fene Menge ist eine höchstens abzählbare Vereinigung of"|fener Intervalle) und
        
        \item
        alle höchstens abzählbaren Vereinigungen oder Schnitte von of"|fenen und/oder
        abgeschlossenen Teilmengen von $\real$
        (z.\,B. das \begriff{\name{Cantor}sche Diskontinuum}).
    \end{enumerate}
\end{Bsp}

\linie
\pagebreak

\begin{Bem}
    $\B(\real)$ hat auch andere Erzeugendensysteme:
    \begin{enumerate}
        \item
        $\{[a, b) \;|\; -\infty < a \le b < \infty\}$
        
        \item
        $\{(a, b) \;|\; -\infty < a \le b < \infty\}$
        
        \item
        $\{[a, b] \;|\; -\infty < a \le b < \infty\}$
        
        \item
        $\{(-\infty, b] \;|\; b \in \real\}$
        
        \item
        $\{O \subset \real \;|\; O \text{ of"|fen}\}$
        
        \item
        $\{A \subset \real \;|\; A \text{ abgeschlossen}\}$
    \end{enumerate}
\end{Bem}

\linie

\begin{Def}{\name{Borel}-$\sigma$-Algebra auf $\extreal$}
    Sei $\E(\extreal) := \{(a, b] \;|\; -\infty \le a \le b \le \infty\}$.\\
    Dann heißt $\B(\extreal) := \sigma(\E(\extreal))$ die
    \begriff{\name{Borel}-$\sigma$-Algebra auf $\extreal$}
\end{Def}

\begin{Def}{\name{Borel}-$\sigma$-Algebra auf $\real^n$}
    Sei $\E(\real^n) :=
    \{I_1 \times \dotsb \times I_n \;|\; I_1, \dotsc, I_n \in \E(\real)\}$.\\
    Dann heißt $\B(\real^n) := \sigma(\E(\real^n))$ die
    \begriff{\name{Borel}-$\sigma$-Algebra auf $\real^n$}
\end{Def}

\begin{Def}{Spur-$\sigma$-Algebra}
    Sei $(\Omega, \A)$ ein Messraum und $M \subset \Omega$ eine nicht-leere Teilmenge.\\
    Dann ist $\A_M := \{A \cap M \;|\; A \in \A\}$ eine $\sigma$-Algebra auf $M$,
    die sog. \begriff{Spur-$\sigma$-Algebra}.\\
    Ist $M$ eine Teilmenge von $\real$, $\extreal$ oder $\real^n$,
    dann heißt die Spur-$\sigma$-Algebra $\B(\real)_M$, $\B(\extreal)_M$ oder $\B(\real^n)_M$
    die \begriff{Borel-$\sigma$-Algebra $\B(M)$ auf $M$}.
\end{Def}

\subsection{%
    Fortsetzung von Maßen%
}

\begin{Def}{Halbring}
    Sei $\Omega \not= \emptyset$.
    Dann heißt $\H \subset \P(\Omega)$ \begriff{Halbring} über $\Omega$, falls
    \begin{enumerate}
        \item
        $\emptyset \in \H$
        
        \item
        $A, B \in \H \;\Rightarrow\; A \cap B \in \H$
        
        \item
        Für alle $A, B \in \H$ mit $A \subset B$ existieren $C_1, \dotsc, C_n \in \H$
        paarweise disjunkt mit\\
        $B \setminus A = \bigcup_{k=1}^n C_k$.
    \end{enumerate}
\end{Def}

\begin{Bsp}
    \begin{enumerate}
        \item
        Jede $\sigma$-Algebra über $\Omega$ ist ein Halbring.
        
        \item
        Für $\Omega \not= \emptyset$ ist $\H := \{A \subset \Omega \;|\; |A| \le 1\} =
        \{\emptyset\} \cup \{\{x\} \;|\; x \in \Omega\}$ ein Halbring über $\Omega$.
        
        \item
        $\E(\real^n)$ ist ein Halbring über $\real^n$.
    \end{enumerate}
\end{Bsp}

\begin{Satz}{Vereinigung von zwei Mengen im Halbring}
    Sei $\H$ ein Halbring über $\Omega$ und $A, B \in \H$.\\
    Dann gibt es paarweise disjunkte Mengen $C_1, \dotsc, C_n \in \H$ mit
    $A \cup B = \bigcup_{k=1}^n C_k$.
\end{Satz}

\linie

\begin{Def}{Prämaß}
    Sei $\H$ ein Halbring über $\Omega \not= \emptyset$.\\
    Dann heißt eine Abbildung $\mu_0\colon \H \rightarrow [0, \infty]$ \begriff{Prämaß}
    auf $\H$, falls
    \begin{enumerate}
        \item
        $\mu_0(\emptyset) = 0$ (\begriff{Nulltreue})
        
        \item
        $\mu_0(\bigcup_{n=1}^\infty A_n) = \sum_{n=1}^\infty \mu_0(A_n)$ für
        $A_n \in \A$ paarweise disjunkt mit $\bigcup_{n=1}^\infty A_n \in \H$\\
        (\begriff{$\sigma$-Additivität})
    \end{enumerate}
    Falls es zusätzlich Mengen $(A_n)_{n \in \natural}$ mit $\mu_0(A_n) < \infty$ für alle
    $n \in \natural$ und $\bigcup_{n=1}^\infty A_n = \Omega$ gibt, so heißt $\mu_0$
    \begriff{$\sigma$-endlich}.
\end{Def}

\linie
\pagebreak

\begin{Bsp}
    \begin{enumerate}
        \item
        Jedes Maß ist auch ein Prämaß.
        
        \item
        Für $\Omega \not= \emptyset$ und den Halbring $\H = \{A \subset \Omega \;|\; |A| \le 1\}$
        ist $\mu_0\colon \H \rightarrow [0, \infty]$ mit $A \mapsto 0$ für $A = \emptyset$ und
        $A \mapsto 1$ sonst ein Prämaß.
        Es ist $\sigma$-endlich genau dann, wenn $\Omega$ höchstens abzählbar ist.
        
        \item
        $\E(\real^n) = \{(a_1, b_1] \times \dotsb \times (a_n, b_n] \;|\; a_i \le b_i\}$
        ist ein Halbring.
        Die Abbildung $\lambda_0^n\colon \E(\real^n) \rightarrow [0, \infty]$ mit
        $(a_1, b_1] \times \dotsb \times (a_n, b_n] \mapsto \prod_{i=1}^n (b_i - a_i)$
        ist ein $\sigma$-endliches Prämaß und heißt \begriff{\name{Lebesgue}-Prämaß}.
    \end{enumerate}
\end{Bsp}

\linie

\begin{Satz}{Fortsetzungssatz von \upshape\,\!\name{Carathéodory}}\\
    Seien $\H$ ein Halbring über $\Omega \not= \emptyset$ und
    $\mu_0\colon \H \rightarrow [0, \infty]$ ein $\sigma$-endliches Prämaß.\\
    Dann gibt es genau ein Maß $\mu\colon \sigma(\H) \rightarrow [0, \infty]$ mit
    $\mu|_\H = \mu_0$.\\
    Außerdem gilt für $A \in \sigma(\H)$ beliebig
    $\mu(A) = \inf\{\sum_{n=1}^\infty \mu_0(B_n) \;|\;
    B_n \in \H,\; A \subset \bigcup_{n=1}^\infty B_n\}$.
\end{Satz}

\begin{Kor}
    Seien $\H$ ein Halbring über $\Omega \not= \emptyset$ und
    $\mu, \nu\colon \sigma(\H) \rightarrow [0, \infty]$ zwei $\sigma$-endliche Maße
    mit $\mu|_\H = \nu|_\H$.
    Dann gilt $\mu = \nu$.
\end{Kor}

\begin{Bsp}
    Seien wieder $\Omega \not= \emptyset$, $\H = \{A \subset \Omega \;|\; |A| \le 1\}$ und
    $\mu_0\colon \H \rightarrow [0, \infty]$ wie oben.
    Wenn $\Omega$ höchstens abzählbar ist, dann gibt es genau ein Maß auf
    $\sigma(\H) = P(\Omega)$ mit $\mu|_\H = \mu_0$.\\
    Weil das Zählmaß auch diese Eigenschaft hat, muss $\mu$ nach dem Fortsetzungssatz von
    Carathéodory gleich dem Zählmaß sein
    (d.\,h. $\mu(A) = |A| \in \natural_0 \cup \{\infty\}$).
\end{Bsp}

\linie

\begin{Def}{\name{Lebesgue}-Maß}
    Das Lebesgue-Prämaß $\lambda_0^n$ lässt sich nach Carathéodory eindeutig zu einem Maß
    $\lambda^n\colon \B(\real^n) \rightarrow [0, \infty]$ fortsetzen.
    $\lambda^n$ heißt \begriff{\name{Lebesgue}-Maß}.\\
    Es gilt $\lambda^n((a_1, b_1] \times \dotsb \times (a_n, b_n]) = \prod_{i=1}^n (b_i - a_i)$
    bzw. insbesondere gilt für\\
    $\lambda := \lambda^1\colon \B(\real) \rightarrow [0, \infty]$,
    dass $\lambda((a, b]) = b - a$.
\end{Def}

\begin{Bsp}
    Für $\Omega \in \B(\real^n)$ mit $0 < \lambda^n(\Omega) < \infty$ definiert
    $P\colon \B(\Omega) \rightarrow [0, 1]$ mit $A \mapsto \frac{\lambda^n(A)}{\lambda^n(\Omega)}$
    ein W-Maß auf $(\Omega, \B(\Omega))$.
    $P$ heißt \begriff{kontinuierliche Gleichverteilung}.
\end{Bsp}

\begin{Satz}{Aussagen über das \name{Lebesgue}-Maß}
    \begin{enumerate}
        \item
        Für $A \subset \real^n$ höchstens abzählbar gilt $\lambda^n(A) = 0$,
        d.\,h. $A$ ist eine \begriff{Nullmenge}.
        
        \item
        Es gilt $\lambda^n([0, 1]^n) = 1$.
        
        \item
        Für $O \subset \real^n$ of"|fen mit $O \not= \emptyset$ gilt $\lambda^n(O) > 0$.
        
        \item
        Sei $A \in \B(\real^n)$ mit $\lambda^n(A) < \infty$.
        Dann gilt $\lambda^n(A) = \sup\{\lambda^n(K) \;|\; K \subset A \text{ kompakt}\}$\\
        (\begriff{Regularität von innen}).
        
        \item
        Sei $A \in \B(\real^n)$.
        Dann gilt $\lambda^n(A) = \inf\{\lambda^n(O) \;|\; O \supset A \text{ of"|fen}\}$\\
        (\begriff{Regularität von außen}).
        
        \item
        Für jede \begriff{Isometrie} $f\colon \real^n \rightarrow \real^n$
        (d.\,h. $f(x) = Lx + b$ mit $L \in \real^{n \times n}$ orthogonal und $b \in \real^n$)
        und alle Mengen $A \in \B(\real^n)$ gilt
        $f(A) \in \B(\real^n)$ sowie $\lambda^n(f(A)) = \lambda^n(A)$\\
        (\begriff{Bewegungsinvarianz des \name{Lebesgue}-Maßes}).
    \end{enumerate}
\end{Satz}

\begin{Bem}
    Es gibt kein bewegungsinvariantes Maß $\mu\colon \P(\real) \rightarrow [0, \infty]$ mit
    $\mu([0, 1]) = 1$\\
    (\begriff{Unlösbarkeit des Maßproblems}).
    Mithilfe des Auswahlaxioms kann man zeigen, dass es Mengen
    (sog. \begriff{\name{Vitali}-Mengen}) gibt, die nicht messbar sind.
\end{Bem}

\pagebreak

\subsection{%
    Konstruktion von Wahrscheinlichkeitsmaßen auf \texorpdfstring{$\real$}{ℝ}%
}

\begin{Def}{Verteilungsfunktion}
    Eine \begriff{Verteilungsfunktion} auf $\real$ ist eine Funktion
    $F\colon \real \rightarrow \real$ mit
    \begin{enumerate}
        \item
        $F$ monoton wachsend und rechtsstetig (d.\,h. $\lim_{y \to x+0} F(y) = F(y)$)
        
        \item
        $\lim_{x \to -\infty} F(x) = 0$ und $\lim_{x \to +\infty} F(x) = 1$
    \end{enumerate}
\end{Def}

\begin{Bsp}
    Die Funktion $F\colon \real \rightarrow \real$ mit
    $F(x) = 0$ für $x < 0$, $F(x) = 1$ für $x > 1$ und
    $F(x) = x$ für $x \in [0, 1]$ ist eine Verteilungsfunktion.
\end{Bsp}

\begin{Satz}{von Verteilungsfunktion zu W-Maß}
    Sei $F\colon \real \rightarrow \real$.\\
    Dann existiert genau ein W-Maß $P$ auf $(\real, \B(\real))$ mit $P((-\infty, x]) = F(x)$
    für alle $x \in \real$.
\end{Satz}

\begin{Bsp}
    Die Funktion $F$ von oben erzeugt ein W-Maß $P$ mit $P((-\infty, x]) = F(x)$.
    Es gilt $P(A) = \lambda(A \cap [0, 1])$.
\end{Bsp}

\linie

\begin{Def}{Dichte}
    Eine \begriff{Dichte} ist eine nicht-negative, integrierbare Funktion
    $f\colon \real \rightarrow [0, \infty)$ mit $\int_{-\infty}^{+\infty} f(u)\du = 1$.
    Ein W-Maß $P$ auf $\real$ \begriff{besitzt die Dichte}
    $f\colon \real \rightarrow [0, \infty)$,
    falls $f$ eine Dichte ist und $P((-\infty, x]) = \int_{-\infty}^x f(u)\du$ gilt
    (das ist äquivalent zu $P([a, b]) = \int_a^b f(u)\du$).
\end{Def}

\begin{Bem}
    In den meisten praktischen Anwendungen ist $f$ stückweise stetig, sodass\\
    $f$ Riemann-integrierbar ist.
    Später wird ein weiterer Integrationsbegriff eingeführt\\
    (das Lebesgue-Integral), sodass man die Integrierbarkeit auf diesen Begriff erweitern kann.
\end{Bem}

\subsection{%
    Beispiele für Wahrscheinlichkeitsmaße mit Dichte%
}

\begin{Def}{Gleichverteilung}
    Für $a < b$ ist $f(x) := \frac{1}{b - a} \cdot \1_{[a,b]}(x)$ eine Dichte.
    Sie erzeugt ein W-Maß $\U([a, b])$,
    die \begriff{Gleichverteilung auf $[a, b]$}.
    Die Verteilungsfunktion ist $F(x) = \int_{-\infty}^x \frac{1}{b - a} \1_{[a, b]}(u) \du$,
    d.\,h. $F(x) = 0$ für $x < a$, $F(x) = \frac{x - a}{b - a}$ für $a \le x \le b$
    und $F(x) = 1$ für $x > b$.
    Für die Gleichverteilung gilt $\U([a, b])(A) = \lambda(A \cap [a, b]) \cdot \frac{1}{b - a}$.
\end{Def}

\linie

\begin{Def}{Exponentialverteilung}
    Für $\lambda > 0$ ist $f(x) := \lambda \cdot e^{-\lambda x} \cdot \1_{[0, \infty)}(x)$
    eine Dichte, denn $\int_{-\infty}^x f(u)\du = \int_0^x \lambda e^{-\lambda u} \du =
    [-e^{-\lambda u}]_0^x = 1 - e^{-\lambda x} \xrightarrow{x \to +\infty} 1$ für $x \ge 0$.
    Die zugehörige Verteilungsfunktion ist $F(x) = 1 - e^{-\lambda x}$.
    Die Dichte erzeugt ein W-Maß $\Exp(\lambda)$,
    die \begriff{Exponentialver\-teilung zum Parameter $\lambda$}.
\end{Def}

\begin{Bem}
    Die Exponentialverteilung ist das kontinuierliche Pendant zur geometrischen Verteilung im
    diskreten Fall.
    Zum Beispiel kann durch die Exponentialverteilung atomarer Zerfall durch Radioaktivität
    modelliert werden.
    Die Exponentialverteilung ist wie die geometrische Verteilung \begriff{gedächtnislos}, d.\,h.
    $P(\{x > s + t\} \;|\; \{x > t\}) = P(\{x > s\})$.
\end{Bem}

\linie

\begin{Def}{Normalverteilung}
    Für $\mu, \sigma \in \real$ mit $\sigma \not= 0$ ist
    $\varphi_{\mu,\sigma^2}(x) := \frac{1}{\sqrt{2\pi\sigma^2}}
    \exp\left(-\frac{(x - \mu)^2}{2\sigma^2}\right)$ eine Dichte,
    denn $\int_{-\infty}^{+\infty} \varphi_{\mu,\sigma^2}(u)\du =
    \frac{1}{\sqrt{2\pi}} \int_{-\infty}^{+\infty} exp\left(-\frac{x^2}{2}\right)\dx =
    \frac{1}{\sqrt{2\pi}} \sqrt{2\pi} = 1$ für $\mu = 0$ und $\sigma = 1$
    (sonst führt man eine Koordinatensubstitution durch).
    Die Dichte $\varphi_{\mu,\sigma^2}$ definiert ein W-Maß $\N(\mu, \sigma^2)$,
    die \begriff{Normalverteilung mit Erwartungswert $\mu$ und Varianz $\sigma^2$}.
    $\Phi_{\mu,\sigma^2}(x) = \frac{1}{\sqrt{2\pi\sigma^2}}
    \int_{-\infty}^x \exp\left(-\frac{(u - \mu)^2}{2\sigma^2}\right) \du$
    ist die Verteilungsfunktion zu $\N(\mu, \sigma^2)$.
\end{Def}

\begin{Satz}{Aussagen zu Normalverteilung}
    \begin{enumerate}
        \item
        $\Psi_{\mu,\sigma^2}(x) = \Psi_{0,1}\left(\frac{x - \mu}{\sigma}\right)$
        
        \item
        $\Psi_{\mu,\sigma^2}(2\mu - x) = 1 - \psi_{\mu,\sigma^2}(x)$
    \end{enumerate}
\end{Satz}

\pagebreak

\subsection{%
    Messbare Abbildungen%
}

\begin{Def}{messbare Abbildung}
    Seien $(\Omega, \A)$ und $(\Omega', \A')$ zwei Messräume.
    Dann heißt eine Abbildung $f\colon \Omega \rightarrow \Omega'$ \begriff{messbar},
    falls $f^{-1}(A') \in \A$ für alle $A' \in \A'$.
    Die Menge $\M(\Omega, \Omega')$ sei die Menge der messbaren Abbildungen von $\Omega$ nach
    $\Omega'$.
    Die Menge $\M(\Omega)$ sei definiert als $\M(\Omega, \extreal)$.
\end{Def}

\begin{Satz}{Erzeugendensystem überprüfen}
    Seien $(\Omega, \A)$ und $(\Omega', \A')$ zwei Messräume,
    $\E'$ ein Erzeugendensystem für $\A'$ und $f\colon \Omega \rightarrow \Omega'$.
    Dann ist $f$ messbar genau dann, wenn $f^{-1}(A') \in \A$ für alle $A' \in \E'$.
\end{Satz}

\begin{Bsp}
    \begin{enumerate}
        \item
        Ist $(\Omega, \P(\Omega))$ diskret, so ist jede Abbildung
        $f\colon \Omega \rightarrow \Omega'$ messbar.
        
        \item
        Für $f\colon \real \rightarrow \real$ stetig mit der Borel-$\sigma$-Algebra und
        $g\colon \real^m \rightarrow \real^n$ stetig, so sind $f$ und $g$ messbar.
        Für $h\colon \real \rightarrow \real$ monoton ist $h$ ebenfalls messbar.
    \end{enumerate}
\end{Bsp}

\linie

\begin{Satz}{messbare Funktionen}
    \begin{enumerate}
        \item
        Für $f\colon (\Omega, \A) \rightarrow (\Omega', \A')$ messbar und
        $g\colon (\Omega, \A') \rightarrow (\Omega'', \A'')$ messbar ist
        $g \circ f$ auch messbar.
        
        \item
        Für $X = (X_1, \dotsc, X_n)\colon \Omega \rightarrow \real^n$ mit
        $X(\omega) = (X_1(\omega), \dotsc, X_n(\omega))$ und $X_k\colon \Omega \rightarrow \real$
        gilt, dass $X$ messbar ist genau dann, wenn $X_1, \dotsc, X_n$ messbar sind.
    \end{enumerate}
\end{Satz}

\begin{Satz}{messbare Funktionen}\\
    Seien $X_k\colon \Omega \rightarrow \extreal$ messbare Funktionen für $k \in \natural$.
    Dann sind ebenfalls messbar:
    \begin{enumerate}
        \item
        $c_1 X_1 + \dotsb + c_n X_n$ für $n \in \natural$ und $c_1, \dotsc, c_n \in \real$
        
        \item
        $X_1 \cdot \dotsm \cdot X_n$ für $n \in \natural$
        
        \item
        $\sup_{k \in \natural} X_k$
        
        \item
        $\inf_{k \in \natural} X_k$
        
        \item
        $\limsup_{k \to \infty} X_k$
        
        \item
        $\liminf_{k \to \infty} X_k$
        
        \item
        $\lim_{k \to \infty} X_k$ (wenn $X_k$ punktweise konvergiert)
    \end{enumerate}
\end{Satz}

\linie

\begin{Def}{Bildmaß}
    Seien $(\Omega, \A)$ und $(\Omega', \A')$ zwei Messräume und
    $f\colon \Omega \rightarrow \Omega'$ eine messbare Abbildung.
    Ist $\mu$ ein Maß auf $(\Omega, \A)$, so ist $\mu_f\colon \A' \rightarrow [0, \infty]$,
    $\mu_f(A') := \mu(f^{-1}(A'))$ ein Maß auf $(\Omega', \A')$.
    $\mu_f$ heißt \begriff{Bildmaß} von $\mu$ unter $f$.\\
    $\mu_f$ ist ein W-Maß genau dann, wenn $\mu$ ein W-Maß ist.
\end{Def}

\begin{Def}{$\mu$-maßerhaltend}
    Sei $(\Omega, \A, \mu)$ ein Maßraum.
    Eine messbare Abbildung $T\colon \Omega \rightarrow \Omega$ heißt \begriff{$\mu$-maßerhaltend},
    falls $\mu_T = \mu$ mit $\mu_T$ dem Bildmaß von $\mu$ unter $T$ gilt.
\end{Def}

\pagebreak

\subsection{%
    Zufallsvariablen und ihre Verteilungen%
}

\begin{Def}{Zufallsvariable}
    Seien $(\Omega, \A, P)$ ein W-Raum und $(E, \A')$ ein Messraum.
    Dann heißt eine messbare Abbildung $X\colon \Omega \rightarrow E$
    \begriff{Zufallsvariable} auf $\Omega$ mit Werten in $E$.
\end{Def}

\begin{Def}{Verteilung}
    Sei $X\colon \Omega \rightarrow E$ eine Zufallsvariable.
    Dann heißt $P_X\colon \A' \rightarrow [0, 1]$, $P_X(B) := P(X \in B) = P(X^{-1}(B))$
    die \begriff{Verteilung} von $X$.
\end{Def}

\begin{Def}{Verteilungsfunktion}
    Sei $X\colon \Omega \rightarrow \real$ eine reelle Zufallsvariable.
    Dann heißt $F_X\colon \real \rightarrow [0, 1]$, $F_X(x) := P(X \le x) = P_X((-\infty, x])$
    die \begriff{Verteilungsfunktion} von $X$.
\end{Def}

\begin{Bsp}
    Beim "`gebrochenen Stab"' mit $\Omega = [0, L]$ kann man die Zufallsvariable betrachten,
    die jedem Ergebnis die Länge der kürzeren Bruchstücks zuordnet, also
    $X\colon \Omega \rightarrow \real$ mit $X(\omega) = \min\{\omega, L - \omega\}$.
    Für die Verteilungsfunktion $F_X$ gilt $F_X(x) = P(X \le x) = P([0, x] \cup [L - x, L]) = 2x$
    für $x \in [0, \frac{L}{2}]$,
    $F_X(x) = P(\emptyset) = 0$ für $x < 0$ und $F_X(x) = P(\Omega) = 1$ für $x > \frac{L}{2}$.
    Also ist $P_X$ die Gleichverteilung auf $[0, \frac{L}{2}]$.
\end{Bsp}

\linie

\begin{Satz}{Aussagen über Verteilungsfunktionen}\\
    Sei $X\colon \Omega \rightarrow \real$ eine reelle Zufallsvariable mit Verteilungsfunktion
    $F_X$.
    Dann gilt:
    \begin{enumerate}
        \item
        $F_X$ ist monoton wachsend und rechtsseitig stetig.
        
        \item
        $\lim_{x \to -\infty} F_X(x) = 0$ und
        $\lim_{x \to +\infty} F_X(x) = 1$
        
        \item
        Für alle $x \in \real$ gilt $F_X(x) - \lim_{y \to x-0} F_X(y) = P(X = x)$,
        wobei $F_X(x) = \lim_{y \to x+0} F_X(y)$ aufgrund der rechtsseitigen Stetigkeit.\\
        Somit ist $F_X$ stetig genau dann, wenn für alle $x \in \real$ $P(X = x) = 0$ gilt.
    \end{enumerate}
\end{Satz}

\begin{Bem}
    Nach (a) und (b) stimmt also obige Definition der Verteilungsfunktion mit der
    Definition im diskreten Fall überein.
\end{Bem}

\begin{Def}{diskret verteilt/mit Dichte verteilt}
    Eine reelle Zufallsvariable $X\colon \Omega \rightarrow \real$ heißt
    \begriff{diskret verteilt}, falls $P_X$ diskret verteilt ist.
    $X$ heißt \begriff{mit Dichte verteilt} oder \begriff{absolutstetig verteilt},
    falls $P_X$ eine Dichte besitzt.
\end{Def}

\pagebreak

\subsection{%
    Das \name{Lebesgue}-Integral%
}

\begin{Bem}
    Im Folgenden sei $(\Omega, \A, \mu)$ ein Maßraum.
\end{Bem}

\begin{Def}{\name{Lebesgue}-Integral für Indikatorfunktionen}
    Seien $A \in \A$ und $\1_A\colon \Omega \rightarrow \extreal$ mit\\
    $\1_A(x) = 1$ für $x \in A$ und $\1_A(x) = 0$ für $x \notin A$
    die zugehörige Indikatorfunktion.\\
    Dann ist $\1_A \in \M(\Omega)$ messbar und
    $\int_\Omega \1_A d\mu := \mu(A)$ das \begriff{\name{Lebesgue}-Integral von $\1_A$}.
\end{Def}

\begin{Def}{Elementarfunktion}
    Seien $c_k \in \real \setminus \{0\}$ und $A_k \in \A$ paarweise disjunkt für $k \in \natural$.
    Dann heißt $\varphi = \sum_{k=1}^n c_k \cdot \1_{A_k}$ \begriff{Elementarfunktion} über
    $\Omega$.
    $\E(\Omega)$ sei der Raum der Elementarfunktionen über $\Omega$
    (ein $\real$-Vektorraum).
    $\E_+(\Omega)$ sei der Raum aller nicht-negativen Elementarfunktionen.
\end{Def}

\begin{Def}{\name{Lebesgue}-Integral für Elementarfunktionen}\\
    Sei $\varphi = \sum_{k=1}^n c_k \cdot \1_{A_k}$ eine Elementarfunktion.\\
    Dann ist $\int_\Omega \varphi d\mu := \sum_{k=1}^n c_k \cdot \int_{\Omega} \1_{A_k} d\mu =
    \sum_{k=1}^n c_k \cdot \mu(A_k)$ das \begriff{\name{Lebesgue}-Integral von $\varphi$}.
\end{Def}

\begin{Bem}
    Man kann zeigen, dass dieses Integral wohldefiniert ist, d.\,h. der Wert des Integrals
    hängt nicht von der Zerlegung in die $A_k$ ab.
    Mit dieser Definition ist das Lebesgue-Integral für Elementarfunktionen linear
    (d.\,h. $\int_\Omega (a\varphi + b\psi) d\mu = a \int_\Omega \varphi d\mu +
    b \int_\Omega \psi d\mu$ für $\varphi, \psi \in \E(\Omega)$ mit $a, b \in \real$)
    und monoton (d.\,h. $\int_\Omega \varphi d\mu \le \int_\Omega \psi d\mu$ für
    $\varphi, \psi \in \E(\Omega)$ mit $\varphi \le \psi$).
\end{Bem}

\begin{Def}{\name{Lebesgue}-Integral für nicht-negative Funktionen}
    Sei $f \in \M(\Omega)$ mit $f \ge 0$.
    Dann ist $\int_\Omega f d\mu := \sup\left\{\left.\int_\Omega \varphi d\mu \;\right|\;
    \varphi \in \E_+(\Omega),\; \varphi \le f\right\} \in [0, \infty]$
    das \begriff{\name{Lebesgue}-Integral von $f \ge 0$}.\\
    $f$ heißt \begriff{positiv \name{Lebesgue}-integrierbar} ($f \in \L^1_+(\mu)$), falls
    $\int_\Omega f d\mu < \infty$.
\end{Def}

\begin{Bem}
    Damit gilt eine gewisse "`Linearität"'
    (nämlich $\int_\Omega (af + bg) d\mu = a \int_\Omega f d\mu + b \int_\Omega g d\mu$
    für $f, g \in \L^1_+(\mu)$ und $a, b \ge 0$)
    und Monotonie (genauer $f \in \L^1_+(\mu)$ und $\int_\Omega f d\mu \le \int_\Omega g d\mu$
    für $f \in \M(\Omega)$ und $g \in \L^1_+(\mu)$ mit $0 \le f \le g$).
\end{Bem}

\linie

\begin{Def}{punktweise Konvergenz von unten}
    Seien $f_n, f\colon \Omega \rightarrow \extreal$ Funktionen.
    $(f_n)_{n \in \natural}$ \begriff{konvergiert punktweise von unten gegen $f$}
    ($f_n \nearrow f$), falls $\forall_{x \in \Omega}\; f_1(x) \le f_2(x) \le \dotsb \le f(x)$
    und\\
    $\lim_{n \to \infty} f_n(x) = f(x)$.
\end{Def}

\begin{Satz}{Existenz einer Folge von Treppenfunktionen und Grenzwertsatz}\\
    Sei $f \in \M(\Omega)$ mit $f \ge 0$.
    Dann gilt:
    \begin{enumerate}
        \item
        Es gibt eine Folge $(\varphi_n)_{n \in \natural}$ mit $\varphi_n \in \E_+(\Omega)$ und
        $\varphi_n \nearrow f$.
        
        \item
        Ist $(\varphi_n)_{n \in \natural}$ eine Folge mit $\varphi_n \in \E_+(\Omega)$ und
        $\varphi_n \nearrow f$, dann gilt\\
        $\int_\Omega f d\mu = \lim_{n \to \infty} \left(\int_\Omega \varphi_n d\mu\right)$.
    \end{enumerate}
\end{Satz}

\linie

\begin{Def}{Nullmenge, $\mu$-fast überall}
    \begin{enumerate}
        \item
        Eine Menge $N \in \A$ heißt \begriff{Nullmenge}, falls $\mu(N) = 0$.
        
        \item
        Eine Aussage $A(x)$ mit $x \in \Omega$ gilt \begriff{$\mu$-fast überall}
        (für \begriff{$\mu$-fast alle} $x \in \Omega$), falls es eine Nullmenge $N \in \A$ gibt
        mit $\forall_{x \in \Omega \setminus N}\; \lnot A(x)$.
        
        \item
        Für zwei Funktionen $f, g\colon \Omega \rightarrow \real$ mit
        $f(x) = g(x)$ für $\mu$-fast alle $x \in \Omega$ schreibt man
        $f \underset{\mu}{=} g$
        (analog sind auch $f \underset{\mu}{\le} g$, $f \underset{\mu}{<} g$,
        $f \underset{\mu}{\nearrow} g$ usw. definiert).
    \end{enumerate}
\end{Def}

\begin{Satz}{\name{Lebesgue}-Integral invariant auf Nullmenge}
    Seien $f \in \L^1_+(\mu)$ und $g \in \M(\Omega)$ mit $g \ge 0$,
    sodass $N := \{x \in \Omega \;|\; f(x) \not= g(x)\} \in \A$ eine Nullmenge ist.\\
    Dann gilt $g \in \L^1_+(\mu)$ und $\int_\Omega g d\mu = \int_\Omega f d\mu$.
\end{Satz}

\linie
\pagebreak

\begin{Def}{positiver/negativer Anteil}
    Sei $f\colon \Omega \rightarrow \extreal$ eine Funktion.\\
    Dann heißt $f_+ := \max(f, 0)$ \begriff{positiver Anteil} und
    $f_- := \max(-f, 0)$ \begriff{negativer Anteil} von $f$.
\end{Def}

\begin{Bem}
    Es gilt $f = f_+ - f_-$ und $|f| = f_+ + f_-$.
\end{Bem}

\begin{Def}{\name{Lebesgue}-Integral}
    Sei $f \in \M(\Omega)$.\\
    $f$ heißt \begriff{\name{Lebesgue}-integrierbar} ($f \in \L^1(\mu)$), falls
    $f_+ \in \L^1_+(\mu)$ und $f_- \in L^1_+(\mu)$.\\
    In diesem Fall ist $\int_\Omega f d\mu := \int_\Omega f_+ d\mu - \int_\Omega f_- d\mu$ das
    \begriff{\name{Lebesgue}-Integral von $f$}.
\end{Def}

\begin{Satz}{Aussagen über das \name{Lebesgue}-Integral}
    \begin{enumerate}
        \item
        $\L^1(\mu)$ ist ein $\real$-Vektorraum und es gilt
        $\int_\Omega (af + bg) d\mu = a \int_\Omega f d\mu + b \int_\Omega g d\mu$
        für alle $f, g \in \L^1(\mu)$ und $a, b \in \real$
        (\begriff{Linearität}).
        
        \item
        Für $f, g \in \L^1(\mu)$ mit $f \le g$ gilt
        $\int_\Omega f d\mu \le \int_\Omega g d\mu$
        (\begriff{Monotonie}).
        
        \item
        Für $f \in \L^1(\mu)$ gilt $|f| \in \L^1(\mu)$ und
        $\left|\int_\Omega f d\mu\right| \le \int_\Omega |f| d\mu$.
        
        \item
        Für $h \in \M(\Omega)$ und $f \in \L^1(\mu)$ mit $|h| \le f$ gilt
        $h \in \L^1(\mu)$.
        
        \item
        Für $f \in \L^1(\mu)$ ist $A := \{x \in \Omega \;|\; |f(x)| = \infty\}$ eine Nullmenge.
    \end{enumerate}
\end{Satz}

\begin{Satz}{allgemeiner Transformationssatz}
    Seien $(\Omega, \A, \mu)$ ein Maßraum, $(\Omega', \A')$ ein Messraum,
    $T \in \M(\Omega, \Omega')$, $\mu_T$ das Bildmaß von $T$ auf $\Omega'$ und
    $f \in \M(\Omega')$.\\
    Dann gilt $f \in \L^1(\mu_T) \iff f \circ T \in \L^1(\mu)$.
    In diesem Fall gilt $\int_{\Omega'} f d\mu_T = \int_\Omega (f \circ T) d\mu$.
\end{Satz}

\subsection{%
    Grenzwertsätze für das \name{Lebesgue}-Integral%
}

\begin{Satz}{Satz von \name{Beppo}-\name{Levi} zur monotonen Konvergenz}\\
    Sei $(f_n)_{n \in \natural}$ eine Folge mit $f_n \in \M(\Omega)$ und
    $0 \le f_1 \le f_2 \le \dotsb$.\\
    Dann gibt es ein $f \in \M(\Omega)$ mit $f \ge 0$ und $f_n \xrightarrow{(\cdot)} f$
    und es gilt $\int_\Omega f d\mu = \lim_{n \to \infty} \left(\int_\Omega f_n d\mu\right)$.
\end{Satz}

\begin{Satz}{Satz von \name{Lebesgue} zur majorisierten Konvergenz}\\
    Seien $(f_n)_{n \in \natural}$ eine Folge mit $f_n \in \M(\Omega)$,
    $f \in \M(\Omega)$ mit $f_n \xrightarrow{(\cdot)} f$ und
    $h \in \L^1(\mu)$ mit $|f_n| \le h$.\\
    Dann gilt $f \in \L^1(\mu)$ und
    $\int_\Omega f d\mu = \lim_{n \to \infty} \left(\int_\Omega f_n d\mu\right)$.
\end{Satz}

\begin{Lemma}{Lemma von \name{Fatou}}
    Seien $(f_n)_{n \in \natural}$ eine Folge mit $f_n \in \M(\Omega)$ und $f_n \ge 0$.\\
    Dann gilt $\int_\Omega (\liminf_{n \to \infty} f_n) d\mu \le
    \liminf_{n \to \infty} \left(\int_\Omega f_n d\mu\right)$.
\end{Lemma}

\pagebreak

\subsection{%
    Integration in \texorpdfstring{$\real$}{ℝ} und \texorpdfstring{$\real^n$}{ℝⁿ}%
}

\begin{Satz}{\name{Riemann}-integrierbar $\Rightarrow$ \name{Lebesgue}integrierbar}\\
    Sei $f\colon [a, b] \rightarrow \real$ messbar und Riemann-integrierbar.\\
    Dann ist $f \in \L^1(\lambda)$ und $\int_{[a, b]} f d\lambda = \int_a^b f(x)\dx$.
\end{Satz}

\begin{Satz}{uneigentliche \name{Riemann}-Integrierbarkeit}
    Seien $I \subset \real$ ein Intervall und $f\colon I \rightarrow \real$ messbar,
    sodass $f|_K$ für jede kompaktes Intervall $K \subset I$ Riemann-integrierbar ist.\\
    Dann gilt $f \in \L^1(\lambda) \iff
    |f| \text{ ist über } I \text{ uneigentlich Riemann-integrierbar}$.\\
    In diesem Fall gilt $\int_I f d\lambda = \int_a^b f(x)\dx$.
\end{Satz}

\begin{Satz}{Integration bzgl. W-Maßen mit Dichte}\\
    Seien $(\real, \B(\real), P)$ ein W-Raum,
    $P$ ein W-Maß mit Dichtefunktion $f\colon \real \rightarrow \real$ und $g \in \M(\real)$.\\
    Dann gilt $g \in \L^1(P) \iff g \cdot f \in \L^1(\lambda)$.\\
    In diesem Fall gilt $\int_\real g dP = \int_\real g \cdot f d\lambda$
    (es gilt außerdem $\int_{[a, b]} g dP = \int_{[a, b]} g \cdot f d\lambda$).
\end{Satz}

\begin{Satz}{Satz von \upshape\name{Fubini}}\\
    Seien $(\Omega_1, \A_1, \mu_1)$ und $(\Omega_2, \A_2, \mu_2)$ zwei $\sigma$-endliche Maßräume
    und $f \in \M(\Omega_1 \times \Omega_2)$.\\
    Zusätzlich gilt mindestens einer der beiden folgenden Fälle:
    \begin{enumerate}[label=(\roman*)]
        \item
        $f \ge 0$
        
        \item
        $f \in \L^1(\mu_1 \otimes \mu_2)$
    \end{enumerate}
    Dann gilt
    $\int_{\Omega_1 \times \Omega_2} f d(\mu_1 \otimes \mu_2)
    = \int_{\Omega_2} \left(\int_{\Omega_1} f(\omega_1, \omega_2)
    d\mu_1(\omega_1)\right) d\mu_2(\omega_2)$\\
    $= \int_{\Omega_1} \left(\int_{\Omega_2} f(\omega_1, \omega_2)
    d\mu_2(\omega_2)\right) d\mu_1(\omega_1)$.
\end{Satz}

\subsection{%
    Integration auf diskreten Maßräumen%
}

\begin{Bem}
    Sei $(\Omega, \P(\Omega), \mu)$ ein diskreter Maßraum, d.\,h.
    $\Omega = \{\omega_1, \omega_2, \dotsc\}$ ist abzählbar.
    Jede Funktion $f\colon \Omega \rightarrow \extreal$ ist automatisch messbar.
    Was ist das Lebesgue-Integral $\int_\Omega f d\mu$ einer solchen Funktion?
    
    $f$ lässt sich als Reihe $f = \sum_{k=1}^\infty f(\omega_k) \cdot \1_{\{\omega_k\}}$
    darstellen.
    Für $f \ge 0$ gilt für die Folge $(\varphi_n)_{n \in \natural}$ mit
    $\varphi_n := \sum_{k=1}^n f(\omega_k) \cdot \1_{\{\omega_k\}}$,
    dass $\varphi_n \in \E_+(\Omega)$ und $\varphi_n \nearrow f$.
    Es gilt $\int_\Omega \varphi_n d\mu = \sum_{k=1}^n f(\omega_k) \cdot \mu(\{\omega_k\})$.
    Somit ist $f \in \L^1_+(\omega) \iff
    \int_\Omega f d\mu = \sum_{k=1}^\infty f(\omega_k) \cdot \mu(\{\omega_k\}) < \infty$.
    
    Für eine beliebige Funktion $f$ gilt wegen der Messbarkeit
    $f \in \L^1(\mu) \iff |f| \in \L^1(\mu)$
    $\iff \sum_{k=1}^\infty |f(\omega_k)| \cdot \mu(\{\omega_k\}) < \infty \iff
    \sum_{k=1}^\infty f(\omega_k) \cdot \mu(\{\omega_k\}) \text{ konvergiert absolut}$.\\
    In diesem Fall gilt
    $\int_\Omega f d\mu = \sum_{k=1}^\infty f(\omega_k) \cdot \mu(\{\omega_k\})$.
    
    Mit diesen Beziehungen kann man die Theorie der diskreten W-Räume als Spezialfall der Theorie
    der kontinuierlichen W-Räume sehen.
    
    Wählt man speziell $\Omega = \natural$ und $\mu = \sigma$ das Zählmaß auf $\natural$,
    so sind Funktionen $a\colon \Omega \rightarrow \real$ eigentlich reelle Zahlenfolgen
    $(a_n)_{n \in \natural}$ mit $a_n = a(n)$.
    Nach eben Gesagtem ist $a$ bzgl. des Zählmaßes Lebesgue-integrierbar genau dann, wenn
    die Reihe $\int_\natural a \d\sigma = \sum_{k=1}^\infty a_k$ absolut konvergiert.
    Daher ist die Theorie der absolut konvergenten Reihen in der Theorie des
    Lebesgue-Integrals enthalten.
\end{Bem}

\pagebreak

\subsection{%
    Erwartungswerte von Zufallsvariablen%
}

\begin{Def}{Erwartungswert}\\
    Seien $(\Omega, \A, P)$ ein W-Raum und $X\colon \Omega \rightarrow \real$ eine reelle
    Zufallsvariable mit $X \in \L^1(P)$.\\
    Dann heißt $\EW(X) := \int_\Omega X dP$ \begriff{Erwartungswert (EW)} von $X$.
\end{Def}

\begin{Satz}{Rechenregeln für den Erwartungswert}
    Seien $X, Y \in \L^1(P)$ zwei reelle Zufallsvariablen.\\
    Dann gilt:
    \begin{enumerate}
        \item
        $X + Y \in \L^1(P)$ und $\EW(X + Y) = \EW(X) + \EW(Y)$
        
        \item
        Für $\alpha \in \real$ gilt $\alpha X \in \L^1(P)$ und
        $\EW(\alpha \cdot X) = \alpha \cdot \EW(X)$.
        
        \item
        Für $A \in \A$ ist $\1_A \in \L^1(P)$ eine reelle Zufallsvariable und
        $\EW(\1_A) = P(A)$.
        
        \item
        Aus $X \le Y$ folgt $\EW(X) \le \EW(Y)$.
        
        \item
        $|X| \in \L^1(P)$ ist eine reelle Zufallsvariable und $|\EW(X)| \le \EW(|X|)$.
    \end{enumerate}
\end{Satz}

\begin{Satz}{Transformationssatz für Erwartungswerte}
    Seien $(\Omega, \A, P)$ ein W-Raum, $X\colon \Omega \rightarrow \real$ eine reelle
    Zufallsvariable und $P_X$ ihre Verteilung.\\
    Dann ist $X \in \L^1(P) \iff \id_\real \in \L^1(P_X)$.
    In diesem Fall gilt
    $\EW(X) = \int_\Omega X dP = \int_\Omega (\id_\real \circ X) dP =
    \int_\real \id_\real dP_X$.\\
    Hat $P_X$ die Dichtefunktion $f$, dann gilt außerdem
    $\EW(X) = \int_\real x \cdot f(x) d\lambda$.
\end{Satz}

\begin{Satz}{Erwartungswert von elementaren Verteilungen}
    Sei $X\colon \Omega \rightarrow \real$ eine reelle Zufallsvariable mit $X \in \L^1(P)$
    und $P_X$ ihre Verteilung.
    Dann gilt:
    \begin{enumerate}
        \item
        Ist $P_X = \U([a, b])$ (d.\,h. \begriff{$X$ ist gleichverteilt}),
        dann gilt $\EW(X) = \frac{a + b}{2}$.
        
        \item
        Ist $P_X = \Exp(\lambda)$ mit $\lambda > 0$ (d.\,h. \begriff{$X$ ist exponentialverteilt}),
        dann gilt $\EW(X) = \frac{1}{\lambda}$.
        
        \item
        Ist $P_X = \N(\mu, \sigma^2)$ mit $\mu, \sigma \in \real$ und $\sigma^2 > 0$
        (d.\,h. \begriff{$X$ ist normalverteilt}),
        dann gilt $\EW(X) = \mu$.
    \end{enumerate}
\end{Satz}

\linie

\begin{Def}{(stochastisch) unabhängig}
    Seien $(\Omega, \A, P)$ ein W-Raum, $(E_i, \A_i)$ Messräume und\\
    $X_i\colon \Omega \rightarrow E_i$ Zufallsvariablen für $i \in I$.
    Dann heißt die Folge $(X_i)_{i \in I}$ \begriff{(stochastisch) unabhängig}, falls
    für jede Wahl von $B_i \in \A_i'$ ($i \in I$)
    die Ereignisse $\{X_i \in B_i\} = X_i^{-1}$ stochastisch unabhängig sind.
\end{Def}

\begin{Lemma}{Kriterium für Unabhängigkeit von Zufallsvariablen}
    Seien $X_1, X_2\colon \Omega \rightarrow \real$ zwei reelle Zufallsvariablen,
    $X := (X_1, X_2)\colon \Omega \rightarrow \real^2$ und
    $P_{X_1}, P_{X_2}, P_X$ die Verteilungen dieser Zufallsvariablen.\\
    Dann sind $X_1$ und $X_2$ unabhängig genau dann, wenn $P_X = P_{X_1} \otimes P_{X_2}$.
\end{Lemma}

\begin{Satz}{EW von Produkt von unabhängigen ZV ist Produkt der EW}
    Seien $(\Omega, \A, P)$ ein W-Raum und $X_1, \dotsc, X_n\colon \Omega \rightarrow \real$
    reelle Zufallsvariablen mit $X_1, \dotsc, X_n \in \L^1(P)$, die unabhängig sind.
    Dann ist auch $X_1 \dotsm X_n \in \L^1(P)$ und
    $\EW(X_1 \dotsm X_n) = \EW(X_1) \dotsm \EW(X_n)$.
\end{Satz}

\pagebreak

\subsection{%
    \texorpdfstring{$k$}{k}-te Momente, Varianz und Streuung von Zufallsvariablen%
}

\begin{Def}{$p$-fach \name{Lebesgue}-integrierbar}
    Seien $(\Omega, \A, \mu)$ ein Maßraum, $f \in \M(\Omega)$ und $p > 0$.\\
    Dann heißt $f$ \begriff{$p$-fach \name{Lebesgue}-integrierbar} ($f \in \L^p(\mu)$),
    falls $|f| \in \L^1(\mu)$.\\
    In diesem Fall definiert man $\norm{f}_p := (\int_\Omega |f|^p d\mu)^{1/p}$.
\end{Def}

\begin{Bem}
    $\L^p(\mu)$ ist ein reeller Vektorraum.\\
    $\norm{\cdot}_p$ ist im Allgemeinen keine Norm auf $\L^p(\mu)$, sondern nur eine Halbnorm.
    Es gilt also $\norm{f}_p \ge 0$, $\norm{c f}_p = |c| \norm{f}_p$ (Homogenität) und
    $\norm{f + g}_p = \norm{f}_p + \norm{g}_p$ (Dreiecksungleichung, in diesem Fall als
    \begriff{\name{Minkowski}-Ungleichung} bekannt), aber aus
    $\norm{f}_p = 0$ folgt nicht unbedingt $f = 0$
    (sondern nur $f \underset{\mu}{=} 0$).
    In der Tat ist $\norm{\cdot}_p$ eine Norm genau dann, wenn $\emptyset$ die einzige
    $\mu$-Nullmenge in $\A$ ist.
\end{Bem}

\begin{Satz}{\name{Hölder}sche Ungleichung}
    Seien $p, q, r > 0$ mit $\frac{1}{p} + \frac{1}{q} = \frac{1}{r}$
    sowie $f \in \L^p(\mu)$ und $g \in \L^q(\mu)$.
    Dann ist $f \cdot g \in \L^r(\mu)$ und es gilt
    $\norm{f \cdot g}_r \le \norm{f}_p \cdot \norm{g}_q$.
\end{Satz}

\begin{Kor}
    Seien $(\Omega, \A, P)$ ein W-Raum und $0 < p < q < \infty$.\\
    Dann gilt $\L^q(P) \subset \L^p(P)$
    (es gilt sogar $\norm{f}_p \le p \cdot \norm{f}_q$ für $f \in \L^q(P)$).
\end{Kor}

\linie

\begin{Def}{(zentriertes) $k$-tes Moment}
    Seien $(\Omega, \A, P)$ ein W-Raum, $X\colon \Omega \rightarrow \extreal$ eine Zufallsvariable
    und $k \in \natural$.
    Für $X \in \L^k(P)$ oder $X^k \ge 0$ heißt
    $\EW(X^k)$ \begriff{$k$-tes Moment} und
    $\EW((X - \EW(X))^k)$ \begriff{zentriertes $k$-tes Moment} von $X$.
\end{Def}

\begin{Def}{Varianz}\\
    Das zentrierte 2. Moment $\Var(X) := \EW((X - \EW(X))^2) = \EW(X^2) - \EW(X)^2$
    heißt \begriff{Varianz} von $X$.
\end{Def}

\begin{Def}{Standardabweichung}\\
    Die Wurzel $\sigma_X := \sqrt{\Var(X)}$ der Varianz heißt \begriff{Standardabweichung} von $X$.
\end{Def}

\linie

\begin{Satz}{Transformationssatz für Momente}
    Seien $(\Omega, \A, P)$ ein W-Raum, $X\colon \Omega \rightarrow \real$ eine reelle
    Zufallsvariable, $P_X$ ihre Verteilung und $k \in \natural$.\\
    Dann ist $X^k \in \L^1(P) \iff x^k \in \L^1(P_X)$ mit $x^k\colon \real \rightarrow \real$,
    $x \mapsto x^k$.\\
    In diesem Fall gilt $\EW(X^k) = \int_\real x^k dP_X$ und
    $\EW((X - E(X))^k) = \int_\real (x - \EW(X))^k dP_X$.
\end{Satz}

\begin{Satz}{Varianz von elementaren Verteilungen}
    Sei $X\colon \Omega \rightarrow \real$ eine reelle Zufallsvariable mit $X \in \L^1(P)$
    und $P_X$ ihre Verteilung.
    Dann gilt:
    \begin{enumerate}
        \item
        Ist $P_X = \U([a, b])$ (d.\,h. \begriff{$X$ ist gleichverteilt}),
        dann gilt $\Var(X) = \frac{1}{12} (b - a)^2$.
        
        \item
        Ist $P_X = \Exp(\lambda)$ mit $\lambda > 0$ (d.\,h. \begriff{$X$ ist exponentialverteilt}),
        dann gilt $\Var(X) = \frac{1}{\lambda^2}$.
        
        \item
        Ist $P_X = \N(\mu, \sigma^2)$ mit $\mu, \sigma \in \real$ und $\sigma^2 > 0$
        (d.\,h. \begriff{$X$ ist normalverteilt}),
        dann gilt $\Var(X) = \sigma^2$.
    \end{enumerate}
\end{Satz}

\begin{Satz}{Rechenregeln für die Varianz}
    Seien $(\Omega, \A, P)$ ein W-Raum und $X, X_1, \dotsc, X_n\colon \Omega \rightarrow \real$
    reelle Zufallsvariablen mit $\Var(X), \Var(X_i) < \infty$ für $i = 1, \dotsc, n$.
    Dann gilt:
    \begin{enumerate}
        \item
        Für $\alpha, c \in \real$ ist $\Var(\alpha \cdot X) = \alpha^2 \cdot \Var(X)$ und
        $\Var(X + c) = \Var(X)$.
        
        \item
        Sind die Zufallsvariablen $X_1, \dotsc, X_n$ unabhängig, so gilt\\
        $\Var(X_1 + \dotsb + X_n) = \Var(X_1) + \dotsb + \Var(X_n)$.
        
        \item
        Für $\Var(X) = 0$ gilt $X \underset{P}{=} \EW(X)$.
    \end{enumerate}
\end{Satz}

\linie
\pagebreak

\begin{Bem}
    Aussage (b) gilt auch schon, wenn die Bildung der Erwartungswerte irgendwelcher der beteiligten
    Zufallsvariablen mit der Multiplikation verträglich ist.
    Daher kann man diese Aussage verallgemeinern.
\end{Bem}

\begin{Def}{unkorreliert}
    Seien $(\Omega, \A, P)$ ein W-Raum und $(X_i)_{i \in I}$ eine Familie von
    reellen Zufallsvariablen mit $X_i \in \L^1(P)$ mit $I \not= \emptyset$.
    Dann heißt die Familie $(X_i)_{i \in I}$ \begriff{unkorreliert}, falls für jede
    endliche Teilmenge $K \subset I$ mit $K \not= \emptyset$ gilt, dass
    $\EW(\prod_{i \in K} X_i) = \prod_{i \in K} \EW(X_i)$.
\end{Def}

\begin{Bem}
    Nach einem vorherigen Satz ist jede unabhängige Familie von Zufallsvariablen auch
    unkorelliert.
    Daher ist der folgende Satz eine Verallgemeinerung der Aussage (b) von eben.
\end{Bem}

\begin{Satz}{Satz von \upshape\,\!\name{Bienaymé}}
    Seien $(\Omega, \A, P)$ ein W-Raum und $X_1, \dotsc, X_n\colon \Omega \rightarrow \real$
    unkorrelierte, reelle Zufallsvariablen mit $\Var(X_k) < \infty$ für $k = 1, \dotsc, n$.
    Dann gilt $\Var(X_1 + \dotsb + X_n) = \Var(X_1) + \dotsb + \Var(X_n)$.
\end{Satz}

\linie

\begin{Bem}
    Für eine reelle Zufallsvariable $X \in \L^1(P)$ gilt\\
    $P(|X| \ge t = P_X((-\infty, -t] \cup [t, \infty)) \xrightarrow{t \to \infty} 0$
    wegen der Stetigkeit von oben.
    Die folgende Ungliehcung ergibt eine Abschätzung der Konvergenzgeschwindigkeit in Abhängigkeit
    vom Grad der Integrierbarkeit von $X$.
\end{Bem}

\begin{Satz}{\name{Markov}sche Ungleichung}
    Seien $(\Omega, \A, P)$ ein W-Raum, $X\colon \Omega \rightarrow \extreal$ eine Zufallsvariable
    und $q > 0$.
    Dann gilt:
    \begin{enumerate}
        \item
        Für $X \in \L^q(P)$ gilt $P(|X| \ge t) \le \frac{\EW(|X|^q)}{t^q}$ für jedes $t > 0$\\
        (\begriff{\name{Markov}sche Ungleichung}, für $q = 2$
        \begriff{\name{Tschebyscheff}-Ungleichung}).
        
        \item
        Wenn es ein $c > 0$ gibt mit $P(|X| \ge t) \le \frac{c}{t^q}$ für jedes $t > 0$,
        dann gilt $X \in \L^{q - \varepsilon}(P)$ für jedes $\varepsilon \in (0, q)$.
    \end{enumerate}
\end{Satz}

\begin{Kor}\\
    Seien $(\Omega, \A, P)$ ein W-Raum, $X\colon \Omega \rightarrow \extreal$ eine Zufallsvariable
    mit $X \in \L^2(P)$ und $\varepsilon > 0$.\\
    Dann gilt $P(|X - \EW(X)| \ge \varepsilon) \le \frac{\Var(X)}{\varepsilon^2}$.
\end{Kor}

\pagebreak
