\section{%
    Grundbegrif"|fe der Wahrscheinlichkeitstheorie%
}

\subsection{%
    Zufallsexperimente%
}

\begin{Bem}
    Im Folgenden soll ein mathematisches Fundament aufgebaut werden, sodass
    Zufallexperimente in der Realität durch Modellbildung so abstrahiert werden können,
    dass sie in ein mathematisches Modell (einen Wahrscheinlichkeitsraum) übersetzt werden.
    Durch Theoreme der Wahrscheinlichkeitstheorie sind (Vor-)Aussagen über das Modell möglich,
    die dann durch Interpretation auf die Realität, also das Zufallsexperiment übertragen werden
    können.
\end{Bem}

\linie

\begin{Bem}
    Für die Definition eines Wahrscheinlichkeitsraums (Modell eines Zufallsexperiments)
    sind drei Elemente notwendig:
    ein Ergebnisraum $\Omega$,
    ein Ereignisraum $\A$ und
    ein Wahrscheinlichkeitsmaß $P$.
\end{Bem}

\begin{Def}{Ergebnisraum}
    Ein \begriff{Ergebnisraum} ist eine Menge $\Omega \not= \emptyset$.
    Seine Elemente $\omega \in \Omega$ heißen \begriff{Ergebnisse/Realisierungen}.
    Eine Stichprobe ist ein $n$-Tupel $(\omega_1, \dotsc, \omega_n) \in \Omega^n$.
\end{Def}

\begin{Bem}
    Nach Durchführung eines Zufallsexperiments soll genau ein Ergebnis $\omega \in \Omega$
    feststehen.
    Ein Ereignis ist nun eine Aussage, die anhand eines Ergebnisses eines Zufallsexperiments
    eindeutig entschieden werden kann.
    Daher kann man Ereignisse als Teilmengen $\Omega$ interpretieren.
    Dabei sollen für $A, B \subset \Omega$ Ereignisse verschiedene Interpretationen möglich sein:
    
    \begin{tabular}{p{60mm}p{100mm}}
        \toprule
        \emph{Ereignis} &
        \emph{Interpretation}\\
        
        \midrule
        
        $A^C = \Omega \setminus A$ &
        $A$ tritt nicht ein \\
        
        $A \cup B$ &
        $A$ oder $B$ tritt ein \\
        
        $A \cap B$ &
        $A$ und $B$ treten ein \\
        
        $A \setminus B$ &
        $A$, aber nicht $B$ tritt ein \\
        
        $A \vartriangle B = (A \cup B) \setminus (A \cap B)$ &
        entweder $A$ oder $B$ tritt ein \\
        
        $\bigcup_{i \in I} A_i$ &
        mindestens eins der $A_i$ tritt ein \\
        
        $\bigcap_{i \in I} A_i$ &
        alle $A_i$ treten ein \\
        
        $\limsup_{n \in \natural} A_n :=
        \bigcap_{n=1}^\infty \left(\bigcup_{k=n}^\infty A_k\right)$ &
        unendlich viele der $A_n$ treten ein \\
        
        $\liminf_{n \in \natural} A_n :=
        \bigcup_{n=1}^\infty \left(\bigcap_{k=n}^\infty A_k\right)$ &
        alle bis auf endlich viele der $A_n$ treten ein \\
        
        $A \subset B$ &
        $A$ impliziert $B$ \\
        
        $A \cap B = \emptyset$ &
        $A$ und $B$ schließen einander aus \\
        
        \bottomrule
    \end{tabular}
    
    Man fordert daher als Ereignismenge eine Teilmenge der Potenzmenge von $\Omega$ mit bestimmten
    Abschlusseigenschaften.
\end{Bem}

\begin{Def}{$\sigma$-Algebra}
    Sei $\Omega \not= \emptyset$.
    Dann heißt $\A \subset \P(\Omega)$ \begriff{$\sigma$-Algebra} über $\Omega$, falls
    \begin{enumerate}
        \item
        $\emptyset \in \A$
        
        \item
        $A \in \A \;\Rightarrow\; \Omega \setminus A \in \A$
        
        \item
        $A_n \in \A$ für $n \in \natural$ $\Rightarrow\;$ $\bigcup_{n=1}^\infty A_n \in \A$
    \end{enumerate}
    In diesem Fall heißt $(\Omega, \A)$ \begriff{Messraum} und $A \in \A$ heißt \begriff{messbar}.
\end{Def}

\begin{Satz}{Eigenschaften von Messräumen}
    Sei $(\Omega, \A)$ ein Messraum.
    Dann gilt:
    \begin{enumerate}
        \item
        $\Omega \in \A$
        
        \item
        $\bigcap_{n=1}^\infty A_n, \limsup_{n \in \natural} A_n,
        \liminf_{n \in \natural} A_n \in \A$
        für $A_n \in \A$ und $n \in \natural$
        
        \item
        $A \cup B, A \cap B, A \setminus B, A \vartriangle B \in \A$ für $A, B \in \A$
    \end{enumerate}
\end{Satz}

\linie
\pagebreak

\begin{Def}{Ereignisraum}
    Ein \begriff{Ereignisraum} über $\Omega \not= \emptyset$ ist eine $\sigma$-Algebra
    $\A \subset \P(\Omega)$ über $\Omega$.\\
    $\emptyset$ heißt \begriff{unmögliches Ereignis} und
    $\Omega$ heißt \begriff{sicheres Ereignis}.\\
    Für $\omega \in \Omega$ heißt $\{\omega\}$ \begriff{Elementarereignis}.
\end{Def}

\begin{Bsp}
    Es gibt manchmal mehrere Möglichkeiten, ein Zufallsexperiment zu modellieren.
    \begin{enumerate}
        \item
        Wurf eines Würfels und "`gerade Augenzahl"':
        $\Omega = \{1, \dotsc, 6\}$, $A = \{2, 4, 6\}$
        
        \item
        Wurf zweier Würfel und "`Pasch"':\\
        $\Omega_1 = \{(k, \ell) \;|\; k, \ell \in \{1, \dotsc, 6\}\}$
        für Beachtung der Reihenfolge, $A_1 = \{(1, 1), \dotsc, (6, 6)\}$,\\
        $\Omega_2 = \{\{k, \ell\} \;|\; k, \ell \in \{1, \dotsc, 6\}\}$
        ohne Beachtung der Reihenfolge, $A_2 = \{\{1\}, \dotsc, \{6\}\}$,\\
        $\Omega_3 = \{2, \dotsc, 12\}$ mit Summe der Augenzahlen,
        hier ist das Ereignis nicht modellierbar
        
        \item
        Mischen eines Kartenblatts mit 52 Karten:
        $\Omega = \{\text{mögliche Anordnungen}\}$,\\
        $|\Omega| = 52! \approx 8 \cdot 10^{67}$
        
        \item
        unendlicher Münzwurf mit Reihenfolge:
        $\Omega = \{(\omega_k)_{k \in \natural} \;|\; \omega_k \in \{0, 1\}\}$
        (0 = Kopf, 1 = Zahl),\\
        z.\,B. "`der fünfte Wurf ist Zahl"` durch
        $A = \{(\omega_k)_{k \in \natural} \;|\; \omega_5 = 1\}$\\
        oder "`es wird unendlich oft Zahl geworfen"' durch
        $B = \{(\omega_k)_{k \in \natural} \;|\;
        \forall_{n \in \natural} \exists_{k \ge n}\; \omega_k = 1\}$
        
        \item
        Brechen eines Stabs der Länge $L$ an einer zufälligen Stelle:
        $\Omega = [0, L]$ überabzählbar,\\
        z.\,B. "`Bruch ist ist linken Dritten"' durch $A = [0, \frac{1}{3} L]$\\
        oder "`Bruch ist genau in der Mitte"' durch $B = \{\frac{1}{2} L\}$
        
        \item
        Schadenshöhe bei einem Autounfall:
        $\Omega = [0, \infty)$
        
        \item
        zufällige Bewegung eines Teilchens in einer Flüssigkeit (\begriff{random walk}):\\
        $\Omega = \{\omega \in [0, \infty) \rightarrow \real^3 \;|\; \omega \text{ stetig}\}$
    \end{enumerate}
\end{Bsp}

\subsection{%
    Wahrscheinlichkeitsmaße%
}

\begin{Def}{Maß}
    Sei $(\Omega, \A)$ ein Messraum.\\
    Dann heißt eine Abbildung $\mu\colon \A \rightarrow [0, \infty]$ \begriff{Maß}
    auf $(\Omega, \A)$, falls
    \begin{enumerate}
        \item
        $\mu(\emptyset) = 0$ (\begriff{Nulltreue})
        
        \item
        $\mu(\bigcup_{n=1}^\infty A_n) = \sum_{n=1}^\infty \mu(A_n)$ für
        $A_n \in \A$ paarweise disjunkt (\begriff{$\sigma$-Additivität})
    \end{enumerate}
    Falls zusätzlich $\mu(\Omega) < \infty$ gilt, dann heißt $\mu$ \begriff{endlich}.\\
    Falls sogar $\mu(\Omega) = 1$ gilt, dann heißt $\mu = P$ \begriff{Wahrscheinlichkeitsmaß}
    (\begriff{W-Maß}).
\end{Def}

\begin{Def}{Wahrscheinlichkeitsraum}\\
    Ein \begriff{Wahrscheinlichkeitsraum} (\begriff{W-Raum}) ist ein Tripel $(\Omega, \A, P)$ mit
    \begin{enumerate}
        \item
        $\Omega \not= \emptyset$ (Ergebnisraum)
        
        \item
        $\A$ eine $\sigma$-Algebra über $\Omega$ ($\sigma$-Algebra der messbaren Ereignisse)
        
        \item
        $P$ ein Wahrscheinlichkeitsmaß auf $(\Omega, \A)$
    \end{enumerate}
\end{Def}

\linie
\pagebreak

\begin{Satz}{Eigenschaften von W-Räumen}\\
    Seien $(\Omega, \A, P)$ ein W-Raum und $A, B, A_k \in \A$ für $k \in \natural$. Dann gilt:
    \begin{enumerate}
        \item
        $P(B \setminus A) = P(B) - P(A \cap B)$
        
        \item
        $A \subset B \;\Rightarrow\; P(A) \le P(B)$
        (\begriff{Monotonie})
        
        \item
        $0 \le P(A) \le 1$, $P(\Omega \setminus A) = 1 - P(A)$
        
        \item
        $P(A \cup B) = P(A) + P(B) - P(A \cap B) \le P(A) + P(B)$
        
        \item
        $P(\bigcup_{k=1}^n A_k) = \sum_{k=1}^n (-1)^{k+1} \sum_{1 \le i_1 < \dotsb < i_k \le n}
        P(A_{i_1} \cap \dotsb \cap A_{i_k}) \le \sum_{k=1}^n P(A_k)$\\
        (\begriff{\name{Poincaré}-\name{Sylvester}-Formel},
        \begriff{Formel des Ein- und Ausschließens})
        
        \item
        $A_1 \subset A_2 \subset \dotsb \;\Rightarrow\;
        P(\bigcup_{k=1}^\infty A_k) = \lim_{n \to \infty} P(A_n)$
        (\begriff{Stetigkeit von unten})
        
        \item
        $A_1 \supset A_2 \supset \dotsb \;\Rightarrow\;
        P(\bigcap_{k=1}^\infty A_k) = \lim_{n \to \infty} P(A_n)$
        (\begriff{Stetigkeit von oben})
        
        \item
        $P(\bigcup_{k=1}^\infty A_k) \le \sum_{k=1}^\infty P(A_k)$
        (\begriff{$\sigma$-Subadditivität})
        
        \item
        $\sum_{k=1}^\infty P(A_k) < \infty \;\Rightarrow\; P(\limsup_{k \to \infty} A_k) = 0$
        (\begriff{Satz von \name{Borel}-\name{Cantelli}, 1. Teil})
        
        \item
        $\lim_{k \to \infty} P(A_k) = 0,\; \sum_{k=1}^\infty P(A_k^c \cap A_{k+1}) < \infty
        \;\Rightarrow\; P(\limsup_{k \to \infty} A_k) = 0$\\
        (\begriff{\name{Barndorff-Nielsen}s Verschärfung des
        Satzes von \name{Borel}-\name{Cantelli}})
    \end{enumerate}
\end{Satz}

\begin{Bsp}
    \begin{enumerate}
        \item
        Für $\Omega \not= \emptyset$ endlich ist $(\Omega, \P(\Omega), P)$ mit
        $P\colon \P(\Omega) \rightarrow [0, 1]$, $A \mapsto \frac{|A|}{n}$ ein W-Raum.\\
        $P$ heißt \begriff{Gleichverteilung}.
        
        \item
        Für $\Omega \not= \emptyset$ beliebig ist $(\Omega, \P(\Omega), \mu)$
        mit dem \begriff{Zählmaß}
        $\mu\colon \P(\Omega) \rightarrow [0, \infty]$, $A \mapsto |A|$ für $A$ endlich und
        $A \mapsto \infty$ für $A$ unendlich ein W-Raum.
        
        \item
        Für $\Omega \not= \emptyset$ beliebig und festes $\omega \in \Omega$ ist
        $(\Omega, \P(\Omega), P_{\{\omega\}})$ mit dem \begriff{\name{Dirac}maß}\\
        $P_{\{\omega\}}\colon \P(\Omega) \rightarrow [0, 1]$,
        $A \mapsto 1$ für $\omega \in A$ und $A \mapsto 0$ für $\omega \notin A$ ein W-Raum.
    \end{enumerate}
\end{Bsp}

\subsection{%
    Diskrete Wahrscheinlichkeitsräume%
}

\begin{Def}{diskreter W-Raum}\\
    Ein W-Raum $(\Omega, \A, P)$ heißt \begriff{diskret}, falls $\Omega$ höchstens abzählbar ist
    und $\A = \P(\Omega)$ gilt.
\end{Def}

\begin{Def}{Zähldichte}
    Wenn $(\Omega, \P(\Omega), P)$ ein diskreter W-Raum ist, dann gilt für jedes Ereignis\\
    $A \subset \Omega$, dass $P(A) = P(\bigcup_{\omega \in A} \{\omega\}) =
    \sum_{\omega \in A} P(\{\omega\}) = \sum_{\omega \in A} p_\omega$
    für $p_\omega := P(\{\omega\})$.\\
    Für die $p_\omega$ gilt $p_\omega \in [0, 1]$ und
    $\sum_{\omega \in \Omega} p_\omega = P(\Omega) = 1$.
    Die Folge $(p_\omega)_{\omega \in \Omega}$ heißt \begriff{Zähldichte}.
\end{Def}

\begin{Satz}{Konstruktion von diskreten W-Räumen}
    \begin{enumerate}
        \item
        Sei $\Omega \not= \emptyset$ höchstens abzählbar und $(p_\omega)_{\omega \in \Omega}$
        eine Folge von Zahlen in $[0, 1]$ mit\\
        $\sum_{\omega \in \Omega} p_\omega = 1$ (d.\,h. eine Zähldichte).
        Dann ist $(\Omega, \P(\Omega), P)$ mit $P\colon \P(\Omega) \rightarrow [0, 1]$,\\
        $A \mapsto \sum_{\omega \in A} p_\omega$ ein diskreter W-Raum mit
        $P(\{\omega\}) = p_\omega$ für $\omega \in \Omega$.
        
        \item
        Ist $(\Omega, \P(\Omega), P)$ ein diskreter W-Raum, so ist $(p_\omega)_{\omega \in \Omega}$
        mit $p_\omega := P(\{\omega\})$ eine Zähldichte
        auf $\Omega$ und $P$ entsteht aus $(p_\omega)_{\omega \in \Omega}$ durch die Konstruktion
        in (a).
    \end{enumerate}
\end{Satz}

\linie
\pagebreak

\begin{Def}{diskrete Gleichverteilung}
    Seien $N \in \natural$, $\Omega := \{1, \dotsc, N\}$ und $p_k := \frac{1}{N}$
    für $k \in \Omega$.
    Dann ist $(p_k)_{k \in \Omega}$ eine Zähldichte und heißt \begriff{Gleichverteilung} oder
    \begriff{\name{Laplace}-Verteilung}.
    Dadurch ist ein diskreter W-Raum $(\Omega, \P(\Omega), P)$, ein sog.
    \begriff{\name{Laplace}scher W-Raum} gegeben.\\
    Für ein Ereignis $A \subset \Omega$ gilt
    $P(A) = \sum_{\omega \in A} p_\omega = \frac{|A|}{|\Omega|}$.
\end{Def}

\begin{Bsp}
    \begin{enumerate}
        \item
        Wurf eines fairen, sechsseitigen Würfels:
        $\Omega = \{1, \dotsc, 6\}$, Laplace-W-Raum $(\Omega, \P(\Omega), P)$.\\
        Für das Ereignis $A :=$ "`Augenzahl gerade"' $= \{2, 4, 6\}$ gilt
        $P(A) = \frac{|A|}{|\Omega|} = \frac{3}{6} = \frac{1}{2}$.
        
        \item
        $n$-maliger Wurf einer fairen Münze:\\
        $\Omega = \{0, 1\}^n = \{(\omega_1, \dotsc, \omega_n) \;|\; \omega_k \in \{0, 1\}\}$,
        Laplace-W-Raum $(\Omega, \P(\Omega), P)$.\\
        Für $n = 3$ und das Ereignis $A$, dass mindestens einmal Zahl auftritt, gilt
        $\Omega \setminus A = \{0, 0, 0\}$, d.\,h.
        $P(A) = P(\Omega \setminus (\Omega \setminus A)) = 1 - P(\Omega \setminus A) =
        1 - \frac{|\Omega \setminus A|}{|\Omega|} = 1 - \frac{1}{8} = \frac{7}{8}$.
        
        \item
        \begriff{Ziegenproblem}:
        In einer Spielshow gibt es drei Tore.
        Hinter genau einem befindet sich ein Auto, hinter den anderen beiden sind Nieten.
        Nach Auswahl eines Tores durch einen Kandidaten öffnet der Showmaster ein anderes Tor,
        hinter dem sich eine Niete befindet, und fragt den Kandidaten, ob er das Tor wechseln
        möchte.
        Wie soll der Kandidat sich entscheiden, wenn er das Auto gewinnen will?\\
        Angenommen, der Kandidat wählt Tor 1 und es ist unbekannt, hinter welchem das Auto steht.
        Es liegt Gleichverteilung vor und es gibt drei Möglichkeiten:
        \begin{enumerate}[label=\arabic*.]
            \item
            Das Auto ist hinter Tor 1.
            Der Showmaster öffnet also Tor 2 oder Tor 3, aber in beiden Fällen sollte der Kandidat
            nicht wechseln.
            
            \item
            Das Auto ist hinter Tor 2.
            Der Showmaster öffnet also Tor 3, in diesem Fall sollte der Kandidat wechseln.
            
            \item
            Das Auto ist hinter Tor 3.
            Der Showmaster öffnet also Tor 2,
            in diesem Fall sollte der Kandidat ebenfalls wechseln.
        \end{enumerate}
        Mit einer Wahrscheinlichkeit von $p = \frac{2}{3}$ ist es also ratsam, das Tor zu wechseln.
    \end{enumerate}
\end{Bsp}

\linie

\begin{Def}{\name{Bernoulli}-Verteilung}
    Seien $\Omega := \{0, 1\}$, $p \in [0, 1]$, $p_0 := 1 - p$ und $p_1 := p$.
    Dann ist $(p_k)_{k \in \Omega}$ eine Zähldichte und
    heißt \begriff{\name{Bernoulli}-Verteilung}.
    Sie definiert ein diskretes W-Maß.
\end{Def}

\begin{Bsp}
    \begin{enumerate}
        \item
        Wurf einer unfairen Münze: $\Omega = \{0, 1\}$,
        Wahrscheinlichkeit für Zahl (1) sei $p \in [0, 1]$
        
        \item
        $n$-maliger Wurf einer unfairen Münze: $\Omega = \{0, 1\}^n =
        \{\omega = (\omega_1, \dotsc, \omega_n) \;|\; \omega_k \in \{0, 1\}\}$.\\
        Ein Ergebnis $\omega$ hat die Wahrscheinlichkeit
        $p_\omega = \prod_{k=1}^n$
        \matrixsize{$\begin{cases}p & \omega_k = 1\\1 - p & \omega_k = 0\end{cases}$},
        da jeder Wurf unabhängig von allen anderen ist.
        Definiert man $k(\omega) = \sum_{k=1}^n \omega_k$ als die Anzahl von Zahl in $\omega$,
        so gilt $p_\omega = p^{k(\omega)} (1 - p)^{n - k(\omega)}$.
        Diese Zähldichte definiert ein diskretes W-Maß auf $\Omega$.
        Wie groß ist die Wahrscheinlichkeit, genau $k$-mal Zahl zu werfen?
        Dazu sei\\
        $A_k := \text{"`es wird genau } k\text{-mal Zahl geworfen"'} = 
        \{\omega \in \Omega \;|\; k(\omega) = k\}$.
        Für die Wahrscheinlichkeit gilt aufgrund der Diskretheit
        $P(A_k) = \sum_{p_\omega \in A_k} p_\omega = p^k (1 - p)^{n-k} \cdot |A_k|$.\\
        Um $|A_k|$ zu bestimmen, verteilt man in das $n$-Tupel $\omega = (0, \dotsc, 0)$ $k$
        Einsen und zählt die verschiedenen möglichen Resultate.
        Allerdings darf es auf die Reihenfolge nicht ankommen
        (ob man zuerst $\omega_1 = 1$ und dann $\omega_2 = 1$ setzt oder andersherum muss
        egal sein).
        Daher muss man noch durch die Anzahl der verschiedenen Verteilungen mit gleichen Resultaten
        dividieren.
        Dies entspricht genau den $k!$ Permutationen der Einsen, daher ist
        $|A_k| = \frac{n (n - 1) \dotsm (n - k + 1)}{k (k - 1) \dotsm 1} =
        \frac{n!}{(n - k)! k!} = \binom{n}{k}$ ein \begriff{Binomialkoef"|fizient}.\\
        Damit gilt $P(A_k) = \binom{n}{k} p^k (1 - p)^{n-k}$.
    \end{enumerate}
\end{Bsp}

\linie
\pagebreak

\begin{Def}{Binomialverteilung}
    Seien $n \in \natural$ und $p \in [0, 1]$.
    Auf $\Omega := \natural_0$ ist $p_k := \binom{n}{k} p^k (1 - p)^{n-k}$ für $0 \le k \le n$
    und $p_k := 0$ sonst mit $k \in \Omega$ eine Zähldichte.
    Das zugehörige W-Maß heißt \begriff{Binomialverteilung} $B(n, p)$.
    Man definiert $B(n, p, k) := p_k$.
\end{Def}

\begin{Def}{\name{Poisson}-Verteilung}
    Sei $\lambda > 0$.
    Auf $\Omega := \natural_0$ ist $p_k := \frac{\lambda^k}{k!} e^{-\lambda}$ eine Zähldichte
    und heißt \begriff{\name{Poisson}-Verteilung} zum Parameter $\lambda$.
    Man definiert $\Pois(\lambda, k) := p_k$.
    Das zugehörige W-Maß heißt \begriff{\name{Poisson}-Maß} $\Pois(\lambda)$.
\end{Def}

\begin{Satz}{\name{Poisson}-Verteilung als Grenzwert der Binomialverteilung}\\
    Für $\lambda > 0$ und $k \in \natural_0$ gilt
    $\lim_{n \to \infty} B(n, \frac{\lambda}{n}, k) = \Pois(\lambda, k)$.
\end{Satz}

\begin{Bsp}
    Beim Roulette gibt es 37 mögliche Zahlen, die alle gleich wahrscheinlich sind.
    Die Gewinnwahrscheinlichkeit pro Spiel beträgt also $p = \frac{1}{37}$.\\
    Wie groß ist die Wahrscheinlichkeit für genau $k$ Gewinne bei 37 Spielen?\\
    Die Binomialverteilung ergibt $B(37, \frac{1}{37}, k) = \binom{37}{k} p^k (1 - p)^{37-k} =
    \frac{37!}{(37 - k)! k!} \cdot \frac{36^{37-k}}{37^{37}}$.\\
    Weil sich das nicht so leicht ausrechnen lässt, nutzt man die ungefähre Gleichheit zu\\
    $\Pois(1, k) = \frac{1}{e \cdot k!}$ aus.
    Die Abweichung beträgt nur $0{,}005$.
\end{Bsp}

\subsection{%
    Kombinatorik%
}

\begin{Satz}{Produktregel}
    Wenn eine Auswahl von Objekten in $n$ Schritten getroffen werden soll, dabei die Reihenfolge
    wichtig ist und für die $k$-te Auswahl $\alpha_k$ Möglichkeiten zur Verfügung stehen,
    so gibt es für die Gesamtauswahl $\alpha_1  \dotsm \alpha_n$ verschiedene
    Möglichkeiten.
\end{Satz}

\begin{Satz}{Summenregel}
    Wenn ein einzelnes Objekt in der Weise ausgewählt werden soll, dass zunächst eine Wahl unter
    $n$ Sorten von Objekten getrof"|fen und dann ein Objekt der gewählten Sorte ausgewählt wird,
    so gibt es, falls es jeweils $\alpha_k$ verschiedene Objekte der $k$-ten Sorte gibt,
    insgesamt $\alpha_1 + \dotsb + \alpha_n$ Möglichkeiten, ein Objekt auszuwählen.
\end{Satz}

\linie

\begin{Def}{Permutation}
    Gegeben seien $n$ Objekte $a_1, \dotsc, a_n$ (nicht notwendigerweise verschieden).\\
    Dann heißt ein $n$-Tupel $(a_{i_1}, \dotsc, a_{i_n})$ mit $i_j \in \{1, \dotsc, n\}$ und
    $i_j \not= i_k$ für $j \not= k$ \begriff{Permutation} der gegebenen Objekte.
\end{Def}

\begin{Bsp}
    Es gibt $6$ Permutationen der Zahlen $1, 2, 3$, nämlich
    $(1, 2, 3)$, $(1, 3, 2)$, $(2, 1, 3)$, $(2, 3, 1)$, $(3, 1, 2)$ und $(3, 2, 1)$.\\
    Dagegen gibt es nur die $3$ Permutationen $(1, 2, 2)$, $(2, 1, 2)$ und $(2, 2, 1)$
    der Zahlen $1, 2, 2$.
\end{Bsp}

\begin{Satz}{Anzahl an Permutationen}
    \begin{enumerate}
        \item
        Es gibt $n!$ viele Permutationen von $n$ verschiedenen Objekten.
        
        \item
        Es gibt $\frac{n!}{n_1! \dotsm n_p!}$ viele Permutationen von $n$ Objekten,
        unter denen es $p \le n$ verschiedene Objekte gibt,
        falls das $i$-te Objekt insgesamt $n_i$-mal unter den gegebenen Objekten vorkommt
        ($i = 1, \dotsc, p$).
    \end{enumerate}
\end{Satz}

\begin{Bsp}
    Die $32$ Karten eines Skatblatts lassen sich in $32! \approx 2{,}6 \cdot 10^{35}$
    verschiedene Anordnungen bringen.
    Wenn es nur auf die $8$ verschiedenen Symbole ankommt, die jeweils in $4$ Farben vorkommen,
    so gibt es $\frac{32!}{(4!)^8} \approx 2{,}4 \cdot 10^{24}$ Möglichkeiten.
    Bei der gleichen Frage für die Farben gibt es $\frac{32!}{(8!)^4} \approx 1{,}0 \cdot 10^{16}$
    Möglichkeiten.
\end{Bsp}

\linie
\pagebreak

\begin{Bem}
    Bei wahrscheinlichkeitstheoretischen Fragen im Zusammenhang mit Kombinatorik müssen häufig
    aus $n$ Objekten $k$-viele ausgewählt werden.
    Solche Sachverhalte werden durch \begriff{Urnenmodelle} veranschaulicht.
    Dabei stellt man sich vor, dass man eine Urne $S$ gegeben hat, in der sich $n$ Kugeln befinden.
    Aus dieser werden $k$ Kugeln zufällig gezogen (\begriff{Stichprobe}).\\
    Es gibt vier Möglichkeiten der Ziehung:
    \begin{enumerate}
        \item
        \begriff{geordnete Stichprobe ohne Zurücklegen}:\\
        Stichprobe ist $k$-Tupel $(\omega_1, \dotsc, \omega_k)$ mit
        $\omega_1, \dotsc, \omega_k \in S$ paarweise verschieden
        
        \item
        \begriff{geordnete Stichprobe mit Zurücklegen}:\\
        Stichprobe ist $k$-Tupel $(\omega_1, \dotsc, \omega_k)$ mit
        $\omega_1, \dotsc, \omega_k \in S$
        
        \item
        \begriff{ungeordnete Stichprobe ohne Zurücklegen}:\\
        Stichprobe ist $k$-elementige Teilmenge $\{\omega_1, \dotsc, \omega_k\}$ von $S$
        
        \item
        \begriff{ungeordnete Stichprobe mit Zurücklegen}:\\
        Stichprobe ist "`Sammlung"' $[\omega_1, \dotsc, \omega_k]$ von $k$ Elementen aus $S$,
        die mehrfach vorkommen können, hinsichtlich ihrer Reihenfolge aber nicht unterschieden
        werden (\begriff{Multimenge})
    \end{enumerate}
\end{Bem}

\begin{Satz}{Anzahl der Möglichkeiten beim Urnenmodell}
    Sei $S$ eine $n$-elementige Menge und\\
    $k \in \natural_0$.
    Dann gilt für die Anzahl an Stichproben bei Ziehung von $k$ Elementen aus $S$:
    \begin{enumerate}
        \item
        \begriff{geordnete Stichprobe ohne Zurücklegen}:
        $\frac{n!}{(n - k)!} = n \cdot (n - 1) \cdot \dotsm \cdot (n - k + 1)$
        
        \item
        \begriff{geordnete Stichprobe mit Zurücklegen}:
        $n^k$
        
        \item
        \begriff{ungeordnete Stichprobe ohne Zurücklegen}:
        $\binom{n}{k} = \frac{n!}{(n - k)! \cdot k!} =
        \frac{n \cdot (n - 1) \cdot \dotsm \cdot (n - k + 1)}{k!}$
        
        \item
        \begriff{ungeordnete Stichprobe mit Zurücklegen}:
        $\binom{n + k - 1}{k}$
    \end{enumerate}
\end{Satz}

\begin{Bsp}
    \begin{enumerate}
        \item
        In einem Turnier mit $20$ Teilnehmern sollen die ersten drei Podiumspläte zufällig
        ermittelt werden.
        Das entspricht einer geordneten Stichprobe ohne Zurücklegen mit $n = 20$ und $k = 3$.
        Es gibt daher $20 \cdot 19 \cdot 18 = 6840$ Möglichkeiten.
        
        \item
        Die Verteilung von Geburtstagen von $k$ Schülern einer Klasse ist äquivalent zu
        einer geordneten Stichprobe mit Zurücklegen mit $n = 365$.
        Hier gibt es $365^k$ Möglichkeiten.
        
        \item
        Beim Lotto (6 aus 49) ist das Ergebnis eine ungeordnete Stichprobe ohne Zurücklegen
        mit $n = 49$ und $k = 6$.
        Es gibt $\binom{49}{6} = 13983816$ Möglichkeiten.\\
        Um die Wahrscheinlichkeit für genau $\ell$ Richtige zu berechnen, zählt man die
        Anzahl der Möglichkeiten, aus 49 Zahlen $\ell$ Richtige und $6 - \ell$ Falsche zu ziehen.
        Es gibt genau $\binom{6}{\ell} \binom{43}{6 - \ell}$ solche Möglichkeiten.
        Somit erhält man für die gesuchte Wahrscheinlichkeit\\
        $p_\ell = \binom{6}{\ell} \binom{43}{6 - \ell} \left/\binom{49}{6}\right.$.
        Es gilt $p_0 \approx 43{,}6\%$, $p_1 \approx 41{,}5\%$, $p_2 \approx 13{,}2\%$,
        \dots, $p_6 \approx 0{,}0000072\%$.
    \end{enumerate}
\end{Bsp}

\linie

\begin{Def}{hypergeometrische Verteilung}
    Seien $n, k \in \natural$ und $m \in \natural_0$ mit $m, k \le n$.
    Auf $\Omega := \natural_0$ ist
    $p_\ell := \binom{m}{l}\binom{n - m}{k - \ell} \left/\binom{n}{k}\right.$ für
    $\ell \in \{\max\{0, k - (n - m)\}, \dotsc, \min\{m, k\}\}$ und
    $p_\ell := 0$ sonst mit $\ell \in \Omega$ eine Zähldichte.
    Das zugehörige W-Maß heißt \begriff{hypergeometrische Verteilung} $H(n, m, k)$
    mit Parametern $n$, $m$ und $k$.
    Man definiert $H(n, m, k, \ell) := p_\ell$.\\
    $H(n, m, k, \ell)$ gibt die Wahrscheinlichkeit an, dass bei einer ungeordneten Ziehung
    von $k$ Kugeln ohne Zurücklegen aus einer Urne mit $m$ schwarzen Kugeln und $n - m$
    weißen Kugeln genau $\ell$ schwarze Kugeln gezogen werden.
\end{Def}

\begin{Satz}{Grenzwert der hypergeometrischen Verteilung}\\
    Sei $n_0(n) \in \{0, \dotsc, n\}$ eine Folge mit
    $\lim_{n \to \infty} \frac{n_0(n)}{n} = p \in (0, 1)$.\\
    Dann gilt $\lim_{n \to \infty} H(n, n_0(n), k, l) = \binom{k}{l} p^l (1 - p)^{k-l}$
    für $k \in \natural$ und $l \in \{0, \dotsc, k\}$.
\end{Satz}

\linie

\begin{Bsp}
    Die Wahrscheinlichkeit, dass in einer Klasse mit $k$ Schülern mindestens zwei am selben Tag
    Geburtstag haben, lässt sich mit der Wahrscheinlichkeit des Komplementärereignisses berechnen,
    d.\,h. alle Schüler haben an unterschiedlichen Tagen Geburtstag.
    Dies entspricht einer geordneten Ziehung von $k$ Kugeln aus einer Urne mit 365 Kugeln
    ohne Zurücklegen.
    Dementsprechend gibt es $\frac{365!}{(365 - k)!}$ Möglichkeiten.
    Insgesamt gibt es $365^k$ verschiedene Geburtstagsverteilungen, d.\,h. die gesuchte
    Wahrscheinlichkeit, dass alle an unterschiedlichen Tagen Geburtstag haben, beträgt
    $1 - \frac{365!}{(365 - k)! 365^k}$.
    Schon für $k = 23$ ist das ungefähr $50{,}7\%$.
    Das erscheint intuitiv erstaunlich, was als \begriff{Geburtstagsparadoxon} bekannt ist.\\
    Die auf"|tretenden Fakultäten lassen sich mit der \begriff{\name{Stirling}-Formel} gut
    abschätzen:\\
    $n! \approx \sqrt{2\pi n} (n/e)^n e^{n/12}$.
\end{Bsp}

\subsection{%
    Bedingte Wahrscheinlichkeiten%
}

\begin{Bsp}
    Einer Gruppe von $66$ Menschen wird befragt, ob sie Sport machen und
    welches Geschlecht sie haben.
    Es stellt sich heraus, dass von den Männern $12$ Sport machen und $18$ nicht;
    bei den Frauen machen $16$ Sport und $20$ nicht.
    Nun wird zufällig eine Person ausgewählt.
    Dann ist die Wahrscheinlichkeit, dass diese Person weiblich ist, gleich
    $\frac{38}{66} = \frac{19}{33}$.\\
    Angenommen, man weiß schon, dass die Person Sport treibt.
    Wie hoch ist dann die Wahrscheinlichkeit, dass die Person eine Frau ist?
    Of"|fensichtlich muss man nur noch die Befragten anschauen, die Sport machen.
    Von diesen sind $12$ männlich und $16$ weiblich, d.\,h. die Wahrscheinlichkeit ist jetzt
    gleich $\frac{|\text{Frau} \cap \text{Sport}|}{|\text{Sport}|} = \frac{18}{30} = \frac{3}{5}$.
    Man kann das zu $\frac{|\text{Frau} \cap \text{Sport}|/|\Omega|}{|\text{Sport}|/|\Omega|} =
    \frac{P(\text{Frau} \cap \text{Sport})}{P(\text{Sport})}$ umschreiben.
    Genau diese Darstellung verwendet man nun zur Verallgemeinerung von so einer
    bedingten Wahrscheinlichkeit auf allgemeine Wahrscheinlichkeitsräume.
\end{Bsp}

\linie

\begin{Def}{bedingte Wahrscheinlichkeit}
    Seien $(\Omega, \A, P)$ ein W-Raum und $A, B \in \A$ mit $P(A) > 0$.\\
    Dann heißt $P(B \,|\, A) := \frac{P(B \cap A)}{P(A)}$ die \begriff{bedingte Wahrscheinlichkeit}
    von $B$ unter dem Vorwissen von $A$.
\end{Def}

\begin{Satz}{Aussagen über bedingte Wahrscheinlichkeit}\\
    Seien $(\Omega, \A, P)$ ein W-Raum und $A \in \A$.
    Dann gilt:
    \begin{enumerate}
        \item
        $P(\cdot \,|\, A)\colon \A \rightarrow [0, 1]$, $B \mapsto P(B \,|\, A)$ ist ein W-Maß
        auf $\Omega$ mit $P(A \,|\, A) = 1$.
        
        \item
        Für $B \in \A$ mit $P(B) > 0$ gilt
        $P(A \,|\, B) = P(B \,|\, A) \cdot \frac{P(A)}{P(B)}$
        (\begriff{erste Formel von \name{Bayes}}).
        
        \item
        Für $B_1, \dotsc, B_m \in \A$ mit $P(B_1 \cap \dotsb \cap B_m) > 0$ gilt\\
        $P(B_1 \cap \dotsb \cap B_m) = P(B_1) \cdot P(B_2 \,|\, B_1) \cdot
        P(B_3 \,|\, (B_1 \cap B_2)) \cdot \dotsm \cdot
        P(B_m \,|\, (B_1 \cap \dotsb B_{m-1}))$.
    \end{enumerate}
\end{Satz}

\begin{Satz}{bedingte Wahrscheinlichkeit mit unendlich vielen Ereignissen}\\
    Seien $(\Omega, \A, P)$ ein W-Raum und $A_i, B \in \A$ mit $P(A_i) > 0$ für $i \in I$
    mit $I$ höchstens abzählbar.
    Die $A_i$ sollen eine Zerlegung von $\Omega$ bilden
    (d.\,h. $A_i$ paarweise disjunkt, $\bigcup_{i \in I} A_i = \Omega$).
    Dann gilt:
    \begin{enumerate}
        \item
        $P(B) = \sum_{i \in I} P(B \,|\, A_i) \cdot P(A_i)$
        (\begriff{Formel von der totalen Wahrscheinlichkeit})
        
        \item
        $P(A_j \,|\, B) = \frac{P(B \,|\, A_j) \cdot P(A_j)}
        {\sum_{i \in I} P(B \,|\, A_i) \cdot P(A_i)}$ für $j \in I$
        (\begriff{zweite Formel von \name{Bayes}})
    \end{enumerate}
\end{Satz}

\linie
\pagebreak

\begin{Bsp}
    \begin{enumerate}
        \item
        Ein Ehepaar hat zwei Kinder.
        Man weiß, dass mindestens eines davon männlich ist.\\
        Wie hoch ist die Wahrscheinlichkeit, dass das Ehepaar sogar zwei Söhne hat?\\
        Mit $A := \text{"`mindestens ein Sohn"'}$ und $B := \text{"`zwei Söhne"'}$ gilt
        mithilfe der ersten Formel von Bayes
        $P(B \,|\, A) = P(A \,|\, B) \cdot \frac{P(B)}{P(A)} = 1 \cdot \frac{1/4}{3/4} =
        \frac{1}{3}$, wieder ein unintuitives Ergebnis.
        
        \item
        Ein Test auf eine Krankheit hat eine Zuverlässigkeit von $99{,}9\%$.
        Die Krankheit tritt für eine einzelne Person mit einer Wahrscheinlichkeit von
        $0{,}01\%$ auf.
        Nun ist der Test bei einer bestimmten Person positiv ausgefallen.
        Wie hoch ist die Wahrscheinlichkeit, dass die Person tatsächlich erkrankt ist?\\
        Mit $A := \text{"`Test positiv"'}$ und $B := \text{"`Person krank"'}$ ist zunächst
        mit der Formel von der totalen Wahrscheinlichkeit
        $P(A) = P(A \,|\, B) \cdot P(B) + P(A \,|\, B^c) \cdot P(B^c)$\\
        $= (1 - 10^{-3}) \cdot 10^{-4} + 10^{-3} \cdot (1 - 10^{-4}) = \frac{10998}{10^7}$.
        Damit kann man nun\\
        $P(B \,|\, A) = P(A \,|\, B) \cdot \frac{P(B)}{P(A)}
        = (1 - 10^{-3}) \cdot 10^{-4} \cdot \frac{10^7}{10998} =
        \frac{999}{10998} \approx 9{,}1\%$ errechnen.
        Diese geringe Wahrscheinlichkeit lässt sich damit erklären, dass die Krankheit im Vergleich
        zur Zuverlässigkeit des Tests zu selten auftritt.
        Ohne weitere Tests oder andere Anhaltspunkte lässt sich also nicht pauschal sagen, dass
        die Person krank ist.
        
        \item
        Auch das Ziegenproblem lässt sich bedingten Wahrscheinlichkeiten erklären:\\
        Angenommen, der Kandidat wählt Tor 1.
        Seien $A_k := \text{"`Auto hinter Tor } k \text{"'}$ und\\
        $B := \text{"`Showmaster öf"|fnet Tor } 2 \text{"'}$.
        Wenn der Showmaster nun tatsächlich Tor 2 öf"|fnet, wie groß ist die Wahrscheinlichkeit,
        dass das Auto hinter Tor $k$ ist?
        Es gilt $P(A_k) = \frac{1}{3}$ sowie $P(B \,|\, A_1) = \frac{1}{2}$,
        $P(B \,|\, A_2) = 0$ und $P(B \,|\, A_3) = 1$.
        Damit gilt mit der zweiten Formel von Bayes
        $P(A_k \,|\, B) = \frac{P(A_k) \cdot P(B \,|\, A_k)}
        {\sum_{i=1}^3 P(A_i) \cdot P(B \,|\, A_i)} = \frac{1/3 \cdot P(B \,|\, A_k)}{1/2} =
        \frac{2}{3} \cdot P(B \,|\, A_k)$,
        d.\,h. $P(A_1 \,|\, B) = \frac{1}{3}$, $P(A_2 \,|\, B) = 0$ und
        $P(A_3 \,|\, B) = \frac{2}{3}$, der Kandidat sollte also wechseln.
    \end{enumerate}
\end{Bsp}

\subsection{%
    Unabhängigkeit von Ereignissen%
}

\begin{Bem}
    Man möchte einen Unabhängigkeitsbegriff für Ereignisse definieren.
    Für zwei unabhängige Ereignisse $A$ und $B$ soll $P(B) = P(B \,|\, A)$ gelten,
    d.\,h. die Wahrscheinlichkeit von $B$ soll sich mit dem Vorwissen von $A$ nicht ändern
    (und umgekehrt).
    Wenn man die Definition von $P(B \,|\, A)$ ausschreibt, kommt man auf
    $P(A \cap B) = P(A) \cdot P(B)$, eine Definition, die auch für $P(A) = 0$ oder $P(B) = 0$
    verwendet werden kann.
\end{Bem}

\begin{Def}{(stochastisch) unabhängig}
    Seien $(\Omega, \A, P)$ ein W-Raum und $A, B \in \A$.\\
    $A$ und $B$ heißen \begriff{(stochastisch) unabhängig}, falls $P(A \cap B) = P(A) \cdot P(B)$.
\end{Def}

\begin{Bsp}
    Man wirft einen Würfel zweimal unter Beachtung der Reihenfolge,
    $\Omega = \{1, \dotsc, 6\}^2$.
    Seien $A := \text{"`erster Wurf gerade"'}$, $B := \text{"`zweiter Wurf gerade"'}$ und
    $C := \text{"`Summe gerade"'}$.\\
    Dann ist $P(A) = P(B) = P(C) = \frac{1}{2}$ und
    $P(A \cap B) = P(A \cap C) = P(B \cap C) = \frac{9}{36} = \frac{1}{4}$.\\
    Somit sind $A$ und $B$, $A$ und $C$ bzw. $B$ und $C$ stochastisch unabhängig.
\end{Bsp}

\linie
\pagebreak

\begin{Bem}
    Im Folgenden soll ein W-Raum $(\Omega, \A, P)$ konstruiert werden,
    der die Durchführung von zwei Experimenten, dargestellt von zwei W-Räumen
    $(\Omega_1, \A_1, P_1)$ und $(\Omega_2, \A_2, P_2)$, hintereinander darstellt.
    Es soll also $\Omega = \Omega_1 \times \Omega_2$ sein.
    Für den Ereignisraum $\A$ soll gelten, dass alle Ereignisse $A_1 \times A_2$ mit
    $A_1 \in \A_1$ und $A_2 \in \A_2$ messbar sein sollen.
    Allerdings reichen diese "`Rechtecke"' noch nicht, denn die Vereinigung von zwei disjunkten
    Rechtecken ist i.\,A. nicht wieder ein Rechteck.
    Somit wählt man $\A$ als die kleinste $\sigma$-Algebra, die alle $A_1 \times A_2$ enthält.
    Für das Wahrscheinlichkeitsmaß $P$ soll $P(A_1 \times A_2) = P_1(A_1) \times P_2(A_2)$ für alle
    $A_1 \in \A_1$ und $A_2 \in \A_2$ gelten, weil die beiden Experimente sich nicht
    beeinflussen sollen.
\end{Bem}

\begin{Satz}{Existenz und Eindeutigkeit des Produktraums}
    Seien $(\Omega_1, \A_1, P_1)$ und $(\Omega_2, \A_2, P_2)$ zwei W-Räume.
    Dann gibt es eine kleinste $\sigma$-Algebra $\A_1 \otimes \A_2$ auf $\Omega_1 \times \Omega_2$
    mit $A_1 \times A_2 \in \A_1 \otimes \A_2$ für alle $A_1 \in \A_1$ und $A_2 \in \A_2$
    und es gibt genau ein W-Maß $P_1 \otimes P_2$ auf dem Messraum
    $(\Omega_1 \times \Omega_2, \A_1 \otimes \A_2)$ mit
    $(P_1 \otimes P_2)(A_1 \times A_2) = P_1(A_1) \cdot P_2(A_2)$
    für alle $A_1 \in \A_1$ und $A_2 \in \A_2$.
\end{Satz}

\begin{Def}{Produktraum}
    Seien $(\Omega_1, \A_1, P_1)$ und $(\Omega_2, \A_2, P_2)$ zwei W-Räume.\\
    Dann heißt $P_1 \otimes P_2$ \begriff{Produktmaß} und
    $(\Omega_1 \times \Omega_2, \A_1 \otimes \A_2, P_1 \otimes P_2)$ heißt \begriff{Produktraum}
    der beiden W-Räume $(\Omega_1, \A_1, P_1)$ und $(\Omega_2, \A_2, P_2)$.
\end{Def}

\linie

\begin{Def}{(stochastisch) unabhängig für beliebige Familien}
    Seien $(\Omega, \A, P)$ ein W-Raum und $A_i \in \A$ für $i \in I$.
    Die $(A_i)_{i \in I}$ heißen \begriff{(stochastisch) unabhängig},
    falls $P(\bigcap_{i \in K}) = \prod_{i \in K} P(A_i)$
    für alle $K \subset I$ endlich.
\end{Def}

\begin{Bem}\\
    $(A_i)_{i \in I}$ sind unabhängig genau dann, wenn die Komplemente
    $(\Omega \setminus A_i)_{i \in I}$ unabhängig sind.
\end{Bem}

\begin{Satz}{Satz von \name{Borel}-\name{Cantelli}, 2. Teil}\\
    Seien $(\Omega, \A, P)$ ein W-Raum und $A_k \in \A$ für $k \in \natural$.
    Gilt $\sum_{k=1}^\infty P(A_n) = \infty$ und sind die $(A_k)_{k \in \natural}$ stochastisch
    unabhängig, dann gilt $P(\limsup_{k \to \infty} A_k) = 1$.
\end{Satz}

\begin{Bem}
    Somit gilt für $(A_k)_{k \in \natural}$ stoch. unabhängig,
    dass entweder $P(\limsup_{k \to \infty} A_k) = 0$ oder $P(\limsup_{k \to \infty} A_k) = 1$
    (\begriff{Null-Eins-Gesetz für stochastisch unabhängige Ereignisse}).
\end{Bem}

\subsection{%
    Zufallsvariablen in diskreten Wahrscheinlichkeitsräumen%
}

\begin{Def}{Zufallsvariable}
    Seien $(\Omega, \P(\Omega), P)$ ein diskreter W-Raum und $E$ eine Menge.\\
    Dann heißt eine Abbildung $X\colon \Omega \rightarrow E$ eine
    \begriff{Zufallsvariable (ZV) mit Werten in $E$}.
\end{Def}

\begin{Bem}
    Seien $(\Omega, \P(\Omega), P)$ ein diskreter W-Raum und $X\colon \Omega \rightarrow E$
    eine Zufallsvariable.
    Dann ist $(p_x)_{x \in \widetilde{E}}$ mit
    $p_x := P(\{\omega \in \Omega \;|\; X(\omega) = x\})$
    eine Zähldichte auf dem Bild $\widetilde{E} := X(\Omega)$,
    da $p_x \ge 0$ für alle $x \in \widetilde{E}$ und $\sum_{x \in \widetilde{E}} p_x = 1$.
    Somit ist $\widetilde{P}_X\colon \P(\widetilde{E}) \rightarrow [0, 1]$,
    $B \mapsto \sum_{x \in B} p_x$ ein diskretes W-Maß auf $(\widetilde{E}, P(\widetilde{E}))$.
    Man kann dieses W-Maß auf $E$ fortsetzen, indem man $p_x := 0$ für
    $x \in E \setminus \widetilde{E}$ setzt.
    Dann erhält man ein W-Maß auf $(E, \P(E))$ durch
    $P_X\colon \P(E) \rightarrow [0, 1]$, $B \mapsto \widetilde{P}_X(B \cap \widetilde{E})
    = \sum_{x \in B \cap \widetilde{E}} p_x = P(\{\omega \in \Omega \;|\; X(\omega) \in B\})$.
    Man beachte, dass die Summe zwar überabzählbar viele Glieder enthält, aber nur
    höchstens abzählbar viele davon können ungleich Null sein.
\end{Bem}

\begin{Def}{Notation}
    Man definiert für $B \subset E$ und $x \in B$ die Schreibweisen\\
    $\{X \in B\} := \{\omega \in \Omega \;|\; X(\Omega) \in B\}$ und
    $\{X = x\} := \{\omega \in \Omega \;|\; X(\Omega) = x\}$.
    Die Mengenklammern können weggelassen werden.
    (Für $E \subset \real$ kann man auch "`$\le$"' und "`$<$"' statt "`$=$"' verwenden.)
\end{Def}

\begin{Def}{Verteilung einer Zufallsvariablen}
    Seien $(\Omega, \P(\Omega), P)$ ein diskreter W-Raum und\\
    $X\colon \Omega \rightarrow E$ eine Zufallsvariable.
    Dann heißt das W-Maß $P_X\colon \P(E) \rightarrow [0, 1]$, $B \mapsto P(X \in B)$
    die \begriff{Verteilung von $X$ unter $P$}.
\end{Def}

\linie
\pagebreak

\begin{Bsp}
    \begin{enumerate}
        \item
        Wurf von zwei fairen Würfeln mit Reihenfolge:
        $\Omega = \{1, \dotsc, 6\}^2$ mit\\
        $X\colon \Omega \rightarrow \natural$, $(\omega_1, \omega_2) \mapsto \omega_1 + \omega_2$
        der Summe der Augenzahlen.\\
        Um $P(X = n)$ für $n \in \{2, \dotsc, 12\}$ zu berechnen, ermittelt man wegen\\
        $P(X = n) = P(\{\omega \in \Omega \;|\; X(\omega) = n\}) =
        \frac{|\{\omega \in \Omega \;|\; X(\omega) = n\}|}{|\Omega|}$ die Mengen
        $\{X = 2\} = \{(1, 1)\}$, $\{X = 3\} = \{(1, 2), (2, 1)\}$ usw.\\
        Somit ist $P(X = 2) = P_X(\{2\}) = \frac{1}{36}$,
        $P(X = 3) = P_X(\{3\}) = \frac{2}{36}$ etc.
        
        \item
        Angenommen, ein Kongress mit 500 Teilnehmern wird eröf"|fnet.
        Zur Überraschung der Gäste sollen diejenigen ein Geschenk bekommen, die am Tag der
        Eröf"|fnung Geburtstag haben.
        Wie viele Geschenke sollen gekauft werden, damit zu einer vorgegebenen Wahrscheinlichkeit
        (z.\,B. $99\%$) die Anzahl der Geschenke für die Gäste ausreichen?\\
        Sei $\Omega = \{1, \dotsc, 365\}^{500}$, wobei
        $\omega = (\omega_1, \dotsc, \omega_{500}) \in \Omega$ bedeutet, dass Gast Nummer $k$
        an Tag $\omega_k$ Geburtstag hat ($k = 1, \dotsc, 500$).
        Außerdem sei $h \in \{1, \dotsc, 365\}$ der Tag der Kongresseröf"|fnung.
        Definiere nun $X_k\colon \Omega \rightarrow \natural_0$,
        $\omega \mapsto 1$ für $\omega_k = h$ und $\omega \mapsto 0$ sonst.
        Damit ist $X\colon \Omega \rightarrow \natural_0$ mit
        $\omega \mapsto \sum_{k=1}^{500} X_k(\omega)$ die Anzahl der Teilnehmer in $\omega$,
        die am Tag $h$ der Kongresseröf"|fnung Geburtstag haben.
        Für die Anzahl der Blumensträuße $n \in \natural_0$ ist nun $P(X \le n)$ gesucht,
        d.\,h. die Wahrscheinlichkeit, dass weniger als $n$ Leute am Tag $h$ Geburtstag haben.\\
        Es gilt $P_X(\{n\}) = P(X = n) = \frac{|\{X = n\}|}{|\Omega|} =
        \frac{1}{365^{500}} \cdot \binom{500}{n} \cdot 364^{500-n}$
        (Binomialverteilung: Ziehen von $500$ Kugeln aus einer Urne mit $365$ Kugeln mit
        Zurücklegen ohne Beachtung der Reihenfolge).
        Damit ist $P(X \le n) = P_X(\{0, \dotsc, n\}) = \sum_{k=0}^n P_X(\{k\})
        = \frac{1}{365^{500}} \cdot \sum_{k=0}^n 364^{500-k} \cdot \binom{500}{k}$.
        Es gilt zum Beispiel $P(X \le 4) \approx 98{,}7\%$ und
        $P(X \le 5) \approx 99{,}7\%$.
    \end{enumerate}
\end{Bsp}

\linie

\begin{Def}{Verteilungsfunktion}
    Seien $(\Omega, \P(\Omega), P)$ ein diskreter W-Raum und\\
    $X\colon \Omega \rightarrow E$ eine Zufallsvariable.
    Dann heißt $F_X\colon \real \rightarrow [0, 1]$,\\
    $F_X(x) := P(X \le x) = P_X((-\infty, x])$ die
    \begriff{Verteilungsfunktion von $X$ unter $P$}.
\end{Def}

\begin{Satz}{Eigenschaften der Verteilungsfunktion}
    Seien $(\Omega, \P(\Omega), P)$ ein diskreter W-Raum und\\
    $X\colon \Omega \rightarrow E$ eine Zufallsvariable.
    Dann gilt:
    \begin{enumerate}
        \item
        $F_X$ ist monoton wachsend und rechtsseitig stetig.
        
        \item
        $\lim_{x \to -\infty} F_X(x) = 0$,
        $\lim_{x \to +\infty} F_X(x) = 1$
        
        \item
        Es gilt $F_X(x) - \lim_{y \to x - 0} F_X(y) = P(X = x)$,
        d.\,h. $P$ kann aus $P_X$ rekonstruiert werden und
        $F_X$ hat höchstens abzählbar viele Sprungstellen.
        
        \item
        $F_X(x) = \sum_{y \le x} P(X = y)$
    \end{enumerate}
\end{Satz}

\linie

\begin{Def}{geometrische Verteilung}
    Sei $p \in (0, 1]$. Auf $\Omega := \natural$ ist
    $p_k := p \cdot (1 - p)^{k-1}$ eine Zähldichte.
    Das zugehörige W-Maß heißt \begriff{geometrische Verteilung} $G(p)$ mit Parameter $p$.
    Man definiert $G(p, k) := p_k$.\\
    $G(p, k)$ gibt die Wahrscheinlichkeit an, dass bei unabhängigen Würfen auf eine Dartscheibe
    mit $p = \frac{1}{4}$ im $k$-ten Wurf erstmals das rechte obere Viertel trifft.
\end{Def}

\begin{Satz}{Zufallsvariable geometrisch verteilt $\iff$ gedächtnislos}\\
    Eine Zufallsvariable $X\colon \Omega \rightarrow \natural$ auf einem diskreten W-Raum
    $(\Omega, \P(\Omega), P)$ ist geometrisch verteilt genau dann, wenn sie \begriff{gedächtnislos}
    ist, d.\,h. $P(\{X > n + k\} \;|\; \{X > n\}) = P(X > k)$ für alle $n, k \in \natural$.
\end{Satz}

\pagebreak

\subsection{%
    Erwartungswert in diskreten Wahrscheinlichkeitsräumen%
}

\begin{Def}{Erwartungswert}
    Seien $(\Omega, \P(\Omega), P)$ ein diskreter W-Raum und
    $X\colon \Omega \rightarrow E$ eine Zufallsvariable mit
    $\sum_{\omega \in \Omega} |X(\omega)| \cdot P(\{\omega\}) < \infty$
    (d.\,h. \begriff{$X$ hat endlichen Erwartungswert}).\\
    Dann heißt $\EW(X) := \sum_{\omega \in \Omega} X(\omega) \cdot P(\{\omega\})$
    \begriff{Erwartungswert (EW)} von $X$.\\
    (Für $X \ge 0$ sei auch $\EW(X) = \infty$ zugelassen.)
\end{Def}

\begin{Bsp}
    \begin{enumerate}
        \item
        Wurf eines fairen Würfels:
        Sei $X\colon \Omega \rightarrow \real$ die Zufallsvariable, die jedem $\omega \in \Omega$
        die Augenzahl zuordnet.
        Dann gilt $\EW(X) = 1 \cdot \frac{1}{6} + \dotsb + 6 \cdot \frac{1}{6} = \frac{7}{2}$.
        
        \item
        Wurf zweier Würfel:
        Sei $X\colon \Omega \rightarrow \real$,
        $(\omega_1, \omega_2) \mapsto \omega_1 + \omega_2$ die Zufallsvariable, die jedem Ergebnis
        die Summe der Augenzahlen zuordnet.\\
        Dann gilt $\EW(X) = X((1, 1)) \cdot P(\{(1, 1)\}) +
        X((1, 2)) \cdot P(\{(1, 2)\}) + \dotsb = 7$.\\
        Dies lässt sich allerdings einfacher bestimmen, wenn man die Summanden mit gleichem
        Wert der Zufallsvariablen $X$ zusammenfasst.
        Mit dieser Methode ist\\
        $\EW(X) = \sum_{n=2}^{12} n \cdot P(X = n) =
        2 \cdot \frac{1}{36} + 2 \cdot \frac{2}{36} + \dotsb + 12 \cdot \frac{1}{36} = 7$.
    \end{enumerate}
\end{Bsp}

\begin{Satz}{diskreter Transformationssatz}
    Seien $(\Omega, \P(\Omega), P)$ ein diskreter W-Raum und
    $X\colon \Omega \rightarrow E$ eine Zufallsvariable mit Verteilung $P_X$.
    Dann sind äquivalent:
    \begin{enumerate}
        \item
        $X$ hat endlichen Erwartungswert.
        
        \item
        $\sum_{x \in X(\Omega)} |x| \cdot P_X(\{x\}) < \infty$
    \end{enumerate}
    In diesem Fall gilt $\EW(X) = \sum_{x \in X(\Omega)} x \cdot P_X(\{x\})$.
\end{Satz}

\linie

\begin{Satz}{Rechenregeln für den Erwartungswert}\\
    Seien $X, Y\colon \Omega \rightarrow \real$ zwei reelle Zufallsvariablen mit endlichen
    Erwartungswerten.
    Dann gilt:
    \begin{enumerate}
        \item
        $X + Y$ hat endlichen Erwartungswert $\EW(X + Y) = \EW(X) + \EW(Y)$.
        
        \item
        Für $\alpha \in \real$ hat
        $\alpha \cdot X$ endlichen Erwartungswert $\EW(\alpha \cdot X) = \alpha \cdot \EW(X)$.
        
        \item
        Für $A \in \P(\Omega)$ hat die \begriff{Indikatorfunktion} von $A$
        $\1_A\colon \Omega \rightarrow \real$ mit $\omega \mapsto 1$ für $\omega \in A$ und
        $\omega \mapsto 0$ für $\omega \notin A$ endlichen Erwartungswert $\EW(\1_A) = P(A)$.
        
        \item
        Aus $X \le Y$ folgt $\EW(X) \le \EW(Y)$.
        
        \item
        $|X|$ hat endlichen Erwartungswert mit $|\EW(X)| \le \EW(|X|)$.
    \end{enumerate}
\end{Satz}

\begin{Bsp}
    Wenn man den Erwartungswert etwas umschreibt, lässt er sich oft leichter berechnen.
    \begin{enumerate}
        \item
        Wurf zweier Würfel:
        Hier kann man die Zufallsvariable "`Summe der Augenzahlen"' als Summe $X = X_1 + X_2$
        der beiden Zufallsvariablen $X_k\colon \Omega \rightarrow \real$,
        $(\omega_1, \omega_2) \mapsto \omega_k$ für $k = 1, 2$ schreiben.
        Somit gilt $\EW(X) = \EW(X_1) + \EW(X_2) = \frac{7}{2} + \frac{7}{2} = 7$.
        
        \item
        Kongresseröf"|fnung:
        Es gilt $X = X_1 + \dotsb + X_{500}$, wobei
        $X_k\colon \Omega \rightarrow \{0, 1\}$ mit $\omega \mapsto 1$ für $\omega_k = h$
        und $\omega \mapsto 0$ sonst (also $X_k = \1_{\{\omega_k = h\}}$).
        Den Erwartungswert $\EW(X_k)$ kann man einfach ausrechnen,
        da $\EW(X_k) = P(\{\omega_k = h\}) = \frac{1}{365}$.
        Damit gilt $\EW(X) = \frac{500}{365} = \frac{100}{73}$.
    \end{enumerate}
\end{Bsp}

\linie
\pagebreak

\begin{Satz}{Erwartungswert von elementaren Verteilungen}
    Seien $(\Omega, \P(\Omega), P)$ ein diskreter\\
    W-Raum und $X\colon \Omega \rightarrow E$ eine Zufallsvariable mit Verteilung $P_X$.
    Dann gilt:
    \begin{enumerate}
        \item
        Für $X \equiv c$ hat $X$ endlichen Erwartungswert $\EW(X) = c$.
        
        \item
        Ist $n := |X(\Omega)| < \infty$ und $P_X$ die Gleichverteilung auf X($\Omega)$
        (\begriff{$X$ ist diskret gleichverteilt}), dann hat $X$ endlichen Erwartungswert
        $\EW(X) = \frac{1}{n} \cdot \sum_{x \in X(\Omega)} x$.
        
        \item
        Ist $P_X$ die Binomialverteilung auf $X(\Omega) = \natural_0$
        zu den Parametern $n$ und $p$, dann hat $X$ endlichen Erwartungswert $\EW(X) = np$.
        
        \item
        Ist $P_X$ die Poissonverteilung auf $X(\Omega) = \natural_0$ zum Parameter $\lambda > 0$,
        dann hat $X$ endlichen Erwartungswert $\EW(X) = \lambda$.
        
        \item
        Ist $P_X$ die hypergeometrische Verteilung auf $X(\Omega) = \natural_0$ zu den Parametern
        $k$, $n$ und $s$, dann hat $X$ endlichen Erwartungswert $\EW(X) = \frac{ks}{n}$.
    \end{enumerate}
\end{Satz}

\linie

\begin{Bem}
    Im Allgemeinen gilt $\EW(X \cdot Y) \not= \EW(X) \cdot \EW(Y)$ für zwei reelle Zufallsvariablen
    $X$ und $Y$.
    Es gilt nämlich\\
    $\EW(X \cdot Y) =
    \sum_{\omega \in \Omega} X(\omega) \cdot Y(\omega) \cdot P(\{\omega\}) =
    \sum_{x \in X(\Omega)} \sum_{y \in Y(\Omega)} x \cdot y \cdot P(\{X = x\} \cap \{Y = y\})$,\\
    aber $\EW(X) \cdot \EW(Y) = \left(\sum_{x \in X(\Omega)} x \cdot P(X = x)\right) \cdot
    \left(\sum_{y \in Y(\Omega)} y \cdot P(Y = y)\right)$\\
    $= \sum_{x \in X(\Omega)} \sum_{y \in Y(\Omega)} x \cdot y \cdot P(X = x) \cdot P(Y = y)$.\\
    Wenn also $P(\{X = x\} \cap \{Y = y\}) = P(X = x) \cdot P(Y = y)$ für alle $x \in X(\Omega)$
    und $y \in Y(\Omega)$ gelten würde, dann würde $\EW(X \cdot Y) = \EW(X) \cdot \EW(Y)$ gelten
    (das ist i.\,A. aber nicht der Fall).
    Dies ist äquivalent zur stochastischen Unabhängigkeit von $\{X = x\}$ und $\{Y = y\}$.
    Man erweitert diese Bedingung noch etwas, um zu einer Definition von stochastischer
    Unabhängigkeit für Zufallsvariablen zu kommen.
\end{Bem}

\begin{Def}{(stochastisch) unabhängig für ZV}\\
    Seien $(\Omega, \P(\Omega), P)$ ein diskreter W-Raum und
    $X_i\colon \Omega \rightarrow E_i$ Zufallsvariablen für $i \in I$ ($I \not= \emptyset$).\\
    Die $(X_i)_{i \in I}$ heißen \begriff{(stochastisch) unabhängig},
    falls für jede Wahl von Teilmengen
    $B_i \subset E_i$ für $i \in I$ die Familie von Ereignissen
    $(\{X_i \in B_i\})_{i \in I}$ stochastisch unabhängig ist.
\end{Def}

\begin{Satz}{Produktregel für unabhängige ZV}
    Seien $X, Y\colon \Omega \rightarrow \real$ reelle Zufallsvariablen mit endlichen
    Erwartungswerten.
    Wenn $X$ und $Y$ unabhängig sind, dann hat $X \cdot Y$ endlichen Erwartungswert
    $\EW(X \cdot Y) = \EW(X) \cdot \EW(Y)$.
\end{Satz}

\begin{Bem}
    Der Satz lässt sich auf eine beliebige endliche Zahl von Zufallsvariablen verallgemeinern.
\end{Bem}

\pagebreak

\subsection{%
    Varianz in diskreten Wahrscheinlichkeitsräumen%
}

\begin{Bem}
    Im Folgenden soll eine Größe für die "`Schwankungsbreite"' einer reellen Zufallsvariable
    eingeführt werden.
    Dafür betrachtet man den Fehler $X - \EW(X)$.
    Um den absoluten Fehler zu berücksichtigen und die Rechnung nicht unnötig zu verkomplizieren,
    verwendet man das Quadrat $(X - \EW(X))^2$.
    Dies ist wieder eine Zufallsvariable, von der man unter gewissen Umständen den Erwartungswert
    berechnen kann, d.\,h. der "`durchschnittliche"' Fehler, den die Zufallsvariable $X$ gegenüber
    dem Erwartungswert $\EW(X)$ macht.
\end{Bem}

\linie

\begin{Lemma}{Varianz}
    Seien $(\Omega, \P(\Omega), P)$ ein diskreter W-Raum und
    $X\colon \Omega \rightarrow \real$ eine reelle ZV.\\
    Wenn $X^2$ einen endlichen Erwartungswert besitzt, dann auch $X$ und
    $(X - \EW(X))^2$.
    In diesem Fall gilt $\EW((X - \EW(X))^2) = \EW(X^2) - \EW(X)^2$.
\end{Lemma}

\begin{Def}{Varianz}
    Seien $(\Omega, \P(\Omega), P)$ ein diskreter W-Raum und
    $X\colon \Omega \rightarrow \real$ eine reelle ZV mit $\EW(X^2) < \infty$.
    Dann heißt $\Var(X) := \EW((X - \EW(X))^2) = \EW(X^2) - \EW(X)^2$
    die \begriff{Varianz} von $X$.
\end{Def}

\begin{Satz}{Aussagen über Varianz}
    Seien $(\Omega, \P(\Omega), P)$ ein diskreter W-Raum und\\
    $X, X_1, \dotsc, X_n\colon \Omega \rightarrow \real$ reelle ZV mit
    $\EW(X^2), \EW(X_i^2) < \infty$ für $i = 1, \dotsc, n$.
    Dann gilt:
    \begin{enumerate}
        \item
        $\Var(\alpha X) = \alpha^2 \Var(X)$ und
        $\Var(X + c) = \Var(X)$ für $\alpha, c \in \real$
        
        \item
        $\Var(X_1 + \dotsb + X_n) = \Var(X_1) + \dotsb + \Var(X_n)$,
        falls $X_1, \dotsc, X_n$ unabhängig sind\\
        (\begriff{Satz von \name{Bienaymé}})
        
        \item
        $\Var(X) = 0 \;\Rightarrow\; P(X = \EW(X)) = 1$
    \end{enumerate}
\end{Satz}

\linie

\begin{Satz}{Varianz von elementaren Verteilungen}
    Seien $(\Omega, \P(\Omega), P)$ ein diskreter W-Raum und
    $X\colon \Omega \rightarrow \real$ eine reelle Zufallsvariable mit Verteilung $P_X$.
    Dann gilt:
    \begin{enumerate}
        \item
        Ist $X$ konstant, so gilt $\Var(X) = 0$.
        
        \item
        Ist $n := |X(\Omega)| < \infty$ und $P_X$ die Gleichverteilung auf $X(\Omega)$, dann gilt\\
        $\Var(X) = \frac{1}{n} \cdot \sum_{x \in X(\Omega)} (x - \EW(X))^2$.
        
        \item
        Ist $P_X$ die Binomialverteilung auf $X(\Omega) = \natural_0$
        zu den Parametern $n$ und $p$, dann gilt $\Var(X) = np(1 - p)$.
        
        \item
        Ist $P_X$ die Poissonverteilung auf $X(\Omega) = \natural_0$ zum Parameter $\lambda > 0$,
        dann gilt $\Var(X) = \lambda$.
        
        \item
        Ist $P_X$ die hypergeometrische Verteilung auf $X(\Omega) = \natural_0$ zu den Parametern
        $k$, $n$ und $s$, dann gilt
        $\Var(X) = \frac{ks}{n} \cdot (1 - \frac{s}{n}) \cdot \frac{n - k}{n - 1}$.
    \end{enumerate}
\end{Satz}

\linie

\begin{Def}{standarisiert}
    Seien $(\Omega, \P(\Omega), P)$ ein diskreter W-Raum und
    $X\colon \Omega \rightarrow \real$ eine reelle Zufallsvariable mit $\EW(X^2) < \infty$.
    Dann heißt $X$ standarisiert, falls $\EW(X) = 0$ und $\Var(X) = 1$.
\end{Def}

\begin{Def}{Standarisierung}
    Sei $X\colon \Omega \rightarrow \real$ eine reelle Zufallsvariable mit $\EW(X^2) < \infty$
    und $\Var(X) \not= 0$.
    Dann ist $X^\ast := \frac{X - \EW(X)}{\sqrt{\Var(X)}}$ eine standarisierte Zufallsvariable,
    die sog. \begriff{Standarisierung} von $X$.
\end{Def}

\pagebreak
