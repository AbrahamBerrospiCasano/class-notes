\section{%
    Einführung in OpenGL%
}

\subsection{%
    Was ist OpenGL?%
}

\textbf{Wie zeigt man Geometrie auf dem Bildschirm an?}
\begin{itemize}
    \item
    \begriff{Raytracing}:
    Bilden von imaginären Strahlen von Kamera zu Szenerie, die die "`Bildfläche"' irgendwo
    schneiden, was am Endpunkt des Strahls ist, bestimmt den Farbwert
    
    \item
    \begriff{Rasterisierung}:
    Die Anwendung verwaltet Punkte, sog. \begriff{Vertices}.
    Durch Geometrie-Verar"-beitung erhält man \begriff{Primitive} (oft Dreiecke).
    Die Rasterisierung unterteilt die Primitive in \begriff{Fragmente}.
    Durch Anwendung von Operationen auf jedem Fragment erhält man \begriff{Pixel},
    die dann in den Frame-Buffer geschrieben werden.
\end{itemize}

\linie

\textbf{OpenGL}:
\begriff{OpenGL} ist eine plattform- und hardwareunabhängige 3D-Darstellungs-API.
Der OpenGL-Standard spezifiziert ca. 200 Befehle zur Definition von Geometrien und
Ausführen von typischen Operationen für interaktive 3D-Grafik.
Es fehlen allerdings Ereignisse, Fenster, Menüs usw.
Die klassische OpenGL-Versionen 2.x unterscheiden sich von den modernen OpenGL-Versionen
(ab 3.0), die programmierbare Stufen unterstützen.
Aktuell ist OpenGL 4.4.

\textbf{Funktionen}:
OpenGL unterstützt \begriff{Zustände}, die solange fest sind, bis sie verändert werden.
Außerdem wird ein Client-Server-Konzept unterstützt (wobei heutzutage Client und Server immer
auf demselben Rechner sind).
Bestimmte GPU-Funktionen werden durch \begriff{Erweiterungen} aktiviert.
Mit der Shading-Sprache \begriff{GLSL} (C-ähnlich, wird allerdings erst zur Laufzeit kompiliert)
können eigene Shader programmiert werden.

\textbf{OpenGL-Primitive}:
Alle geometrischen Primitive werden durch Vertices in homogenen Koordinaten gegeben.
Falls die homogene Koordinate $w$ fehlt, dann wird $w := 1$ gesetzt.\\
Zu den eindimensionalen Primitiven gehören
\code{GL\_POINTS} (Punktmenge),
\code{GL\_LINES} (Strecken AB, CD usw..),
\code{GL\_LINE\_LOOP} (geschlossenes Polygon ohne Inneres) und
\code{GL\_LINE\_STRIP} (of"|fenes Polygon ohne Inneres).
Zu den zweidimensionalen Primitiven zählt man
\code{GL\_TRIANGLES} (Dreiecke ABC, DEF usw.),
\code{GL\_TRIANGLE\_STRIP} (Dreiecke ABC, BCD usw.),
\code{GL\_TRIANGLE\_FAN} (Dreiecke ABC, ACD usw.),
\code{GL\_QUADS} (Vierecke ABCD, EFGH usw.),
\code{GL\_QUAD\_STRIP} (Vierecke ABCD, CDEF usw.) und
\code{GL\_POLYGON} (Polygon der beteiligten Vertices).
Die letzten drei Primitiven sind in OpenGL 3 entfernt worden, dafür gibt es jetzt
\code{GL\_PATCHES}.

\subsection{%
    Grafikpipeline und Vertex-Transformation%
}

\textbf{Grafikpipeline}:
Die komplette Grafikpipeline von OpenGL 4.4 ist sehr komplex.
Für eine grundlegende Funktionalität benötigt man prinzipiell einen
\begriff{Vertex-Shader} und einen \begriff{Frag"-ment-Shader}.
Dabei speichert die CPU die Vertices im sog. \begriff{Vertex Buffer Object (VBO)}.
Mit dem Vertex-Shader wird die Geometrie verarbeitet, d.\,h. es erfolgt die Umwandlung in
Display-Koordinaten.
Das Ergebnis wird rasterisiert und mittels des Fragment-Shaders gefärbt.
Die Ausgabe wird im Framebuffer gespeichert, der schließlich angezeigt wird.

\textbf{Vertex-Transformation}:
Jedes Objekt hat seine eigenen \begriff{Objekt-Koordinaten} $p_\text{obj}$.
Mit $M_\text{model}$ werden diese Koordinaten zu den \begriff{Welt-Koordinaten} $p_\text{world}$
vereinigt.
Mit einer weiteren Transformation $M_\text{view}$ erhält man die \begriff{Kamera-Koordinaten}
$p_\text{cam}$.
Daraus folgen mit $M_\text{proj}$ die \begriff{Clip-Koor"-dinaten} $p_\text{clip}$
und mit einer projektiven Division die \begriff{normierten Gerätekoordinaten} $p_\text{ndc}$.
Durch Einschränkung des sichtbaren Bereichs folgen schließlich die \begriff{Fenster-Koordinaten}
$p_\text{win}$.

In Formeln gilt $p_\text{clip} = M_\text{proj} M_\text{view} M_\text{model} p_\text{obj}$ und
$p_\text{ndc} = \smallpmatrix{\frac{x_\text{clip}}{w_\text{clip}} &
\frac{y_\text{clip}}{w_\text{clip}} & \frac{z_\text{clip}}{w_\text{clip}}} \to p_\text{win}$.

\pagebreak

\subsection{%
    OpenGL Utility Toolkit (GLUT)%
}

\textbf{OpenGL Utility Toolkit (GLUT)}:
Das \begriff{OpenGL Utility Toolkit (GLUT)} ist eine Befehlssammlung, die sich um das
"`Drumherum"' kümmert wie Fenstererstellung und Maus-/Tastatur-Ereignisse.

\textbf{Struktur einer GLUT-Anwendung}:
Mit \code{glutInit} muss GLUT zunächst initialisiert werden.
Durch \code{glutCreateWindow} kann man Fenster erstellt.
Nach der Initialisierung von OpenGL kann man die Callback-Funktionen registieren,
die z.\,B. aufgerufen werden, wenn
gezeichnet (\code{glutDisplayFunc}),
die Fenstergröße geändert (\code{glutResizeFunc}),
Animationen abgespielt (\code{glutIdleFunc}) oder
Tastatur- (\code{glutKeyboardFunc}, \code{glutSpecialFunc}) und
Maus-Ereignisse verarbeitet (\code{glutMouseFunc}, \code{glutMotionFunc}) werden sollen.
Schließlich ruft man \code{glutMainLoop} auf, sodass man die Kontrolle der Endlosschleife von
GLUT übergibt.

\subsection{%
    OGL4Core%
}

\textbf{OGL4Core}:
\begriff{OGl4Core} ist ein von der Universität Stuttgart entwickeltes OpenGL-Framework,
das die immer wieder nötigen Standard-Programmierungsschritte auf ein Minimum reduzieren soll.
OGl4Core ist plattformunabhängig und durch ein Plugin-System benötigt man keine weiteren
Bibliotheken
(für jede OpenGL-Anwendung schreibt man ein Plugin, das dann im vorkompilierten Hauptprogramm
dynamisch eingebunden wird).
Veränderliche Parameter der Plugins (Schnelligkeit einer Animation etc.)
können über eine grafische Benutzeroberfläche (GUI) wie bei einem HUD verändert werden.
OGL4Core benötigt mindestens OpenGL 2.x und basiert auf C++11.

\textbf{Erstellung eines Plugins}:
Ein neues Plugin muss durch Ableitung von der Oberklasse \code{Render"-Plugin} erstellt werden.
Konstruktor, Destruktor und die Methoden \code{Init}, \code{Activate}, \code{Deactivate} und
\code{Render} sind zwingend notwendig.
Optionale Methoden zur Maus-/Tastaturkontrolle und Erstellung von Animationen können ebenfalls
verwendet werden.
Durch Deklaration spezieller Felder können veränderliche Parameter aller möglichen Typen
angelegt werden, die dann über die GUI während der Ausführung des Programms eingestellt werden.

\pagebreak
