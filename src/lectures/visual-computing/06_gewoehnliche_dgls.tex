\chapter{%
    Gewöhnliche Dif"|ferentialgleichungen%
}

\section{%
    ODEs erster Ordnung%
}

\textbf{ODE 1. Ordnung}:
Sei $f\colon D \to \real$ stetig auf $D \subset \real^2$ of"|fen.\\
Dann heißt $y'(x) = f(x, y(x))$ oder $y' = f(x, y)$ \begriff{ODE 1. Ordnung}.
$f$ heißt \begriff{rechte Seite} der ODE,
$y$ \begriff{abhängige} und $x$ \begriff{unabhängige Variable}.
Eine \begriff{Lösung} der ODE auf einem Intervall $I \subset \real$ ist eine
stetig diffb. Funktion $u\colon I \to \real$ mit
$\forall_{x \in I}\; (x, u(x)) \in D$, $u'(x) = f(x, u(x))$.\\
Eine DGL $y'(x) = f(x, y(x))$ mit der Bedingung $y(x_0) = y_0$ heißt
\begriff{Anfangswertproblem (AWP)}.\\
Eine DGL kann durch ein \begriff{2D-Richtungsfeld} in $x$-$y$-Koordinaten dargestellt werden,
indem in diskreten Punkten $(x, y) \in D$ Pfeile mit Steigung $\tan \varphi = y'(x)$
gezeichnet werden (meistens mit Länge $1$).
Jede Lösung verläuft tangential zum Richtungsfeld.

\linie

\textbf{homogene lineare ODE 1. Ordnung}:
Das AWP einer \begriff{homogenen linearen ODE 1. Ordnung} ist gegeben durch
$\forall_{x \in I}\; y'(x) = a(x) y(x)$ mit $y(x_0) = y_0$,
wobei $I \subset \real$, $a \in \C^0(I)$,
$x_0 \in I$, $y_0 \in \real$ und $y \in \C^1(I)$.
Die Lösung ist gleich $y(x) = y_0 e^{A(x)}$ mit $A(x) := \int_{x_0}^x a(t) \dt$.

\textbf{inhomogene lineare ODE 1. Ordnung}:
Das AWP einer \begriff{inhomogenen linearen ODE 1. Ordnung} ist gegeben durch
$\forall_{x \in I}\; y'(x) = a(x) y(x) + b(x)$ mit $y(x_0) = y_0$
mit $b \in \C^0(I)$.\\
Die Lösung ist gleich $y(x) = (y_0 + \int_{x_0}^x e^{-A(s)} b(s) \ds) e^{A(x)}$
mit $A(x) := \int_{x_0}^x a(t) \dt$.

\section{%
    Phasenbilder autonomer Systeme%
}

Im Folgenden betrachtet man Systeme zweier autonomer ODEs 1. Ordnung, also\\
$x'(t) = f_1(x(t), y(t))$ und $y'(t) = f_2(x(t), y(t))$ mit
$f_1, f_2 \in \C^1(\Omega)$ und $\Omega \subset \real^2$.\\
Jede Lösung $\vec{\varphi}(t, \vec*{\eta}{0})$ ist entweder injektiv, periodisch oder konstant.

\textbf{Trajektorie/Orbit}:
Jede Lösung $t \mapsto \vec{\varphi}(t, \vec*{\eta}{0})$ des Systems heißt \begriff{Trajektorie}.\\
Die Spur $\vec{\varphi}(I, \vec*{\eta}{0})$ heißt \begriff{Orbit}.

\textbf{kritischer Punkt}:\\
Punkte $\vec{\eta} \in \Omega$ mit $f_1(\vec{\eta}) = f_2(\vec{\eta}) = 0$ heißen
\begriff{kritisch}/\begriff{stationär}/\begriff{GG-Punkte}.

\textbf{Phasenraum}:
Der \begriff{Phasenraum} $\Omega$ ist die disjunkte Vereinigung aller Orbits,
jeder Punkt liegt auf genau einem Orbit.
Zwei Orbits sind entweder disjunkt oder gleich.

\textbf{Phasenportrait}:\\
Ein \begriff{Phasenportrait} zeigt die kritischen Punkte und ein paar typische Orbits
als Pfeile.

\linie

\textbf{Linearisierung in kritischen Punkten}:
Sei $\vec*{\eta}{0}$ ein kritischer Punkt, also $\vec{f}(\vec*{\eta}{0}) = \vec{0}$.\\
Mit der Taylor-Entwicklung gilt $\vec{f}(\vec*{\eta}{0} + \vec{h}) = A\vec{h} + \O(|\vec{h}|^2)$
mit $A := \D\vec{f}(\vec*{\eta}{0})$.
Ist die Lösung $\vec{\varphi}(t)$ nahe bei $\vec*{\eta}{0}$, so ist
$\vec{\psi}(t) := \vec{\varphi}(t) - \vec*{\eta}{0}$ "`klein"' und\\
$\vec{\psi}'(t) = \vec{\varphi}'(t) = \vec{f}(\vec{\varphi}(t))
= \vec{f}(\vec*{\eta}{0} + \vec{\psi}(t)) = A\vec{\psi}(t) + \O(|\vec{\psi}(t)|^2)$,
also $\vec{\psi}'(t) \approx A\vec{\psi}(t)$.\\
Um kritische Punkte herum kann man also das Verhalten des Systems durch
die lineare ODE $\vec{\psi}'(t) = A\vec{\psi}(t)$ mit
$\vec{\varphi}(t) \approx \vec{\psi}(t) + \vec*{\eta}{0}$ approximieren.

\textbf{Satz von \name{Hartman}-\name{Grobman}}:
Seien die Realteile der Eigenwerte von $\D\vec{f}(\vec*{\eta}{0})$ für den kritischen Punkt
$\vec*{\eta}{0}$ ungleich Null (\begriff{hyperbolischer kritischer Punkt}).\\
Dann ist das Phasenportrait des linearisierten Systems "`ähnlich"' dem des originalen Systems.

\pagebreak

\section{%
    Klassifikation von kritischen Punkten in 2D%
}

Gegeben sei ein autonomes lineares System $\vec{y}' = A\vec{y}$ mit
$A := \smallpmatrix{a_{11}&a_{12}\\a_{21}&a_{22}} \in \real^{2 \times 2} \setminus \{0\}$.\\
Ist $S \in \real^{2 \times 2}$ inv.bar, so ist das System äquivalent zu
$\vec{x}' = B\vec{x}$ mit $B := S^{-1} AS$ (mit $\vec{y} = S\vec{x}$).
$S$ kann stets so gewählt werden, dass $B$ einer der Matrizen\\
$B_1 := \smallpmatrix{\lambda_1&0\\0&\lambda_2}$,
$B_2 := \smallpmatrix{\lambda&1\\0&\lambda}$ und
$B_3 := \smallpmatrix{-\varrho&-\omega\\\omega&-\varrho}$
gleicht mit $\lambda, \lambda_1, \lambda_2, \varrho, \omega \in \real$.

\textbf{Fall 1: zwei reelle Eigenwerte}\\
Ist $B = B_1$, dann ist die Lösung gegeben durch $x_1(t) = \xi_1 e^{\lambda_1 t}$ und
$x_2(t) = \xi_2 e^{\lambda_2 t}$ mit $x_1(0) = \xi_1$ und $x_2(0) = \xi_2$.
\begin{itemize}
    \item
    Für $\lambda_2 < \lambda_1 < 0$ oder $0 < \lambda_1 < \lambda_2$
    erhält man einen \begriff{Zweitangentenknoten/echten Knoten ((proper) node)}.
    Dabei gilt $x_2 = cx_1^k$ mit
    $c := \xi_2 \xi_1^{-k}$ für $\xi_1 \not= 0$
    und $k := \frac{\lambda_2}{\lambda_1} > 1$.
    Die Orbits sind Parabelstücke.

    \item
    Für $\lambda_1 = \lambda_2$ erhält man einen
    \begriff{Sternknoten (singular/star node)}.\\
    Dabei gilt $x_2 = cx_1$ mit $c := \xi_2 \xi_1^{-k}$ für $\xi_1 \not= 0$.
    Die Orbits sind Geradenstücke.

    \item
    Für $\lambda_1 \not= 0$ und $\lambda_2 = 0$ liegen die kritischen Knoten alle auf
    einer Geraden ($x_1 = 0$) und die Orbits sind zu dieser Gerade orthogonale Geraden
    (parallel zur $x_1$-Achse).

    \item
    Für $\lambda_2 < 0 < \lambda_1$ erhält man einen \begriff{Sattelpunkt (saddle)}.
    Dabei gilt $x_2 = \pm c|x_1|^{-k}$ mit
    $c := \pm\xi_2 |\xi_1|^k$ für $\xi_1 \not= 0$
    und $k := -\frac{\lambda_2}{\lambda_1} > 0$.
    Die Orbits sind Hyperbelstücke.
\end{itemize}

\textbf{Fall 2: nur ein reeller Eigenwert}\\
Ist $B = B_2$, dann ist die Lösung gleich $x_1(t) = (\xi_1 + \xi_2 t) e^{\lambda t}$ und
$x_2(t) = \xi_2 e^{\lambda t}$ mit $x_1(0) = \xi_1$\\
und $x_2(0) = \xi_2$.
Für $\lambda \not= 0$ erhält man dann einen
\begriff{Eintangentenknoten/unechten Knoten (degenerate/improper node)}.
Die Orbits laufen spiralförmig auf den kritischen Punkt zu bzw. von ihm weg,
wobei sie auf der Richtung des Eigenvektors genau auf ihn zu bzw. weg laufen.

\textbf{Fall 3: zwei komplexe Eigenwerte}\\
Ist $B = B_3$, dann hat $B$ die Eigenwerte $\lambda_{1,2} := -\varrho \pm \iu\omega$
mit $\omega > 0$.
Mit der Substitution $z = x_1 + \iu x_2$
(damit $z' = \lambda_1 z \implies z(t) = z_0 e^{\lambda_1 t}$) erhält man dann die Lösung\\
$x_1(t) = r_0 e^{-\varrho t} \cos(\omega t + \varphi_0)$,
$x_2(t) = r_0 e^{-\varrho t} \sin(\omega t + \varphi_0)$\\
mit $x_1(0) = r_0 \cos(\varphi_0)$ und
$x_2(0) = r_0 \sin(\varphi_0)$ (wobei $z_0 = r_0 e^{\iu\varphi_0}$).
\begin{itemize}
    \item
    Für $\varrho = 0$ erhält man ein \begriff{Zentrum (center)},
    Orbits = Kreise um den krit. Pkt.

    \item
    Für $\varrho \not= 0$ erhält man einen \begriff{Spiralknoten (focus)}.
    Orbits = Spiralen um den krit. Pkt.
\end{itemize}

In allen Fällen gilt, dass die Orbits auf den kritischen Punkt zu laufen,
wenn der Realteil des Eigenwerts negativ ist, und von ihm weg laufen,
wenn der Realteil positiv ist.

\linie

\textbf{Zusammenfassung}:
Für die Eigenwerte gilt
$\lambda_{1,2} = \frac{\tr(A)}{2} \pm \frac{1}{2} \sqrt{\tr(A)^2 - 4\det(A)}$
(char. Gleichung $\lambda^2 - \tr(A) \lambda + \det(A) = 0$).
Von den Vorzeichen von Spur, Determinante und Diskriminate $\tr(A)^2 - 4\det(A)$ lässt sich der
Typ des kritischen Punkts bestimmen:
\begin{itemize}
    \item
    $\det(A) < 0$: \emph{Sattelpunkt}

    \item
    $\det(A) > 0$:
    \begin{itemize}
        \item
        $\tr(A) = 0$: \emph{Zentrum}

        \item
        $\tr(A)^2 - 4\det(A) = 0$: \emph{Sternknoten} oder \emph{unechter Knoten}
        (stabil $\iff \tr(A) < 0$)

        \item
        $\tr(A)^2 - 4\det(A) < 0$: \emph{Spiralknoten} (stabil $\iff \tr(A) < 0$)

        \item
        $\tr(A)^2 - 4\det(A) > 0$: \emph{echter Knoten} (stabil $\iff \tr(A) < 0$)
    \end{itemize}
\end{itemize}

\pagebreak

\section{%
    Grenzzykel und Separatrizen%
}

Gegeben sei nun das zweidimensionale autonome System $\vec{x}' = \vec{f}(\vec{x})$.

\textbf{Grenzzyklus}:
Ein \begriff{Grenzzyklus} ist eine isolierte periodische Lösung.

\textbf{\name{Bendixon}-Kriterium}:
Seien $D \subset \real^2$ ein einfach zush. Gebiet (d.\,h. keine Löcher) und\\
$x' = f_1(x, y)$, $y' = f_2(x, y)$ mit $f_1, f_2 \in \C^1(D)$.\\
Wenn $\div \vec{f} = \partial_x f_1 + \partial_y f_2$ nicht identisch Null ist und
keinen VZ-Wechsel hat, dann gibt es keine geschlossenen Orbits des Systems, die vollständig
in $D$ liegen.

\linie

\textbf{Fluss}:
Ein \begriff{Fluss} ist eine Abbildung $\vec{\phi} \in \C^1(\real^2, \real^2)$ mit
$\forall_{\vec{x} \in \real^2}\; \vec{\phi}(\vec{x}, 0) = \vec{x}$ und\\
$\forall_{\vec{x} \in \real^2} \forall_{s, t \in \real}\;
\vec{\phi}(\vec{\phi}(\vec{x}, t), s) = \vec{\phi}(\vec{x}, t+s)$.

\textbf{Grenzpunkt}:
Seien $\vec{x} \in \real^2$ und $\vec{\phi}$ der vom System $\vec{x}' = \vec{f}(\vec{x})$
erzeugte Fluss.\\
Ein Punkt $\vec*{x}{0} \in \real^2$ heißt \begriff{$\omega$-Grenzpunkt} von $\vec{x}$
für das System, falls es eine Folge $(t_i)_{i \in \natural}$
mit $t_i \to \infty$ gibt, sodass $\vec{\phi}(t_i, \vec{x}) \to \vec*{x}{0}$.\\
\begriff{$\alpha$-Grenzpunkte} sind analog mit $t_i \to -\infty$ definiert.

\textbf{Grenzmenge}:
Die Menge $\omega(\vec{x})$ aller $\omega$-Grenzpunkte von $\vec{x}$ heißt
\begriff{$\omega$-Grenzmenge}.\\
Die \begriff{$\alpha$-Grenzmenge} ist analog definiert.

\linie

\textbf{stabiler Grenzzyklus}:
Ein Grenzzyklus $\Gamma$ heißt \begriff{stabil}
(oder \begriff{$\omega$-Grenzzyklus}), falls $\Gamma$ die\\
$\omega$-Grenzmenge aller Lösungen in einer Umgebung von $\Gamma$ ist.

\textbf{instabiler Grenzzyklus}:
Ein Grenzzyklus $\Gamma$ heißt \begriff{instabil}
(oder \begriff{$\alpha$-Grenzzyklus}), falls $\Gamma$ die\\
$\alpha$-Grenzmenge aller Lösungen in einer Umgebung von $\Gamma$ ist.

\textbf{semistabiler Grenzzyklus}:
Ein Grenzzyklus heißt \begriff{semistabil}, falls er auf der einen Seite stabil und auf der
anderen instabil ist.

\linie

\textbf{homokliner Orbit}:
Seien $\vec*{x}{0}$ ein kritischer Punkt von $\vec{x}' = \vec{f}(\vec{x})$
und $\gamma$ ein Orbit des Systems.\\
$\gamma$ heißt \begriff{homoklin}, falls $\omega(\gamma) = \{\vec*{x}{0}\} = \alpha(\gamma)$.

\textbf{heterokliner Orbit}:
Seien $\vec*{x}{0} \not= \vec*{y}{0}$ zwei kritische Punkte von $\vec{x}' = \vec{f}(\vec{x})$
und $\gamma$ ein Orbit des Systems.
$\gamma$ heißt \begriff{heteroklin}, falls $\omega(\gamma) = \{\vec*{x}{0}\}$
und $\alpha(\gamma) = \{\vec*{y}{0}\}$.

\textbf{Separatrix}:
Eine \begriff{Separatrix} ist ein Orbit, der den Phasenraum in zwei Bereiche
qualitativ unterschiedlichen Verhaltens teilt.

\pagebreak

\section{%
    Pfadlinien, Stromlinien und Streichlinien%
}

Gegeben sei ein zeitabhängiges 2D-Vektorfeld $\vec{v}(\vec{x}, t)$ auf $D \subset \real^2$
(z.\,B. ein Fluss).

\textbf{Pfadlinie}:
Eine \begriff{Pfadlinie} ist eine Trajektorie eines masselosen Teilchens im Fluss.\\
Man erhält sie durch Lösung von $\vec{x}' = \vec{v}(\vec{x}, t)$ für $t > 0$ und
$\vec{x}(0) = \vec*{x}{0}$.

\textbf{Stromlinie}:
Eine \begriff{Stromlinie} ist eine Kurve, die überall tangential zum
Vektorfeld $\vec{v}(\cdot, t_s)$ für ein festes $t_s$ ist.
Man erhält sie durch Lösung von $\frac{\d}{\ds}\vec{x} = \vec{v}(\vec{x}, t_s)$ für $s > 0$ und
$\vec{x}(0) = \vec*{x}{0}$.

\textbf{Streichlinie}:
Eine \begriff{Streichlinie} ist eine Kurve, die entsteht, wenn man ständig zu Zeitpunkten
$t' \in [0, t]$ Partikel in einem bestimmten Punkt $(x_0, y_0)$ starten lässt
und dann schaut, wo sich die Partikel zum Zeitpunkt $t$ befinden.
Die Streichlinie ist nun die Kurve, die diese $t$-Aufenthaltsorte verbindet.
Zur Berechnung von Streichlinien verfährt man wie folgt:
\begin{enumerate}
    \item
    Berechne zunächst die Pfadlinie $(x(t, c_1, c_2), y(t, c_1, c_2))$
    im Anfangspunkt $(c_1, c_2)$ für $t = 0$.

    \item
    Setze $x_0 := x(t', c_1, c_2)$ und $y_0 := y(t', c_1, c_2)$.
    Diese Gleichungen beschreiben die Anfangspositionen $(c_1, c_2)$ zu $t = 0$,
    von denen das Partikel zum Zeitpunkt $t' \in [0, t]$ durch $(x_0, y_0)$ gewandert ist.

    \item
    Löse nach $c_1$ und $c_2$ auf und setze diese Ausdrücke in $x(t, c_1, c_2)$ und
    $y(t, c_1, c_2)$ ein.

    \item
    Eliminiere $t'$, um die Streichlinien-Parametrisierungen in Abhängigkeit von $t$
    zu einer Kurve der Art $y = y(x)$ umzuformen.
\end{enumerate}

\section{%
    Numerische Lösung%
}

Gegeben seien das AWP $y'(x) = f(x, y(x))$ in $I := [a, b]$ und $y(a) = y_0$
und eine Diskretisierung $x_j := a + jh$ von $I$ für $j = 0, \dotsc, N$ und $h := \frac{b-a}{N}$
mit $N \in \natural$.\\
Gesucht sind Approximationen $u_j \approx y(x_j)$.

\textbf{explizites \name{Euler}-Verfahren}:
Mit Taylor-Enwicklung gilt $y(x_{j+1}) = y(x_j) + y'(x_j) h + \O(h^2)$.
Für kleines $h$ erhält man daher Approximationen
$u_0 := y_0$ und $u_{j+1} := u_j + h f(x_j, u_j)$ für $j = 0, \dotsc, N-1$
(\begriff{explizites \name{Euler}-Verfahren}).
Das Verfahren hat Fehlerordnung $\O(h^2)$.

\linie

\textbf{\name{Runge}-\name{Kutta}-Verfahren}:\\
Beim \begriff{\name{Runge}-\name{Kutta}-Verfahren} ist ebenfalls $u_0 := y_0$.
$u_{j+1}$ errechnet sich aus $u_j$ durch
\begin{enumerate}
    \item
    $k_1 := f(x_j, u_j)$,

    \item
    $k_2 := f(x_j + h/2, u_j + hk_1/2)$,

    \item
    $k_3 := f(x_j + h/2, u_j + hk_2/2)$,

    \item
    $k_4 := f(x_j + h, u_j + hk_3)$ und

    \item
    $u_{j+1} := u_j + h/6 \cdot (k_1 + 2k_2 + 2k_3 + k_4)$.
\end{enumerate}
Das Verfahren hat Fehlerordnung $\O(h^4)$.

\linie

\textbf{\name{Störmer}-\name{Verlet}-Verfahren}:
Das \begriff{\name{Störmer}-\name{Verlet}-Verfahren} ist geeignet,
um newtonsche Bewegungsgleichungen $m\vec{a} = m\vec{x}'' = \vec{F}$ zu lösen.
Zunächst wandelt man mit $\vec{v} = \vec{x}'$ die ODE in ein 2D-System um.
Anschließend berechnet man für jedes Partikel $k$ zum Zeitschritt $n$ zunächst
$\vec*{a}{k}^n := \vec*{F}{k}^n(\vec*{x}{k}^n)/m_k$,
$\vec*{v}{k}^{n+1/2} := \vec*{v}{k}^n + \vec*{a}{k}^n \Delta t/2$ und dann
$\vec*{x}{k}^{n+1} := \vec*{x}{k}^n + \vec*{v}{k}^{n+1/2} \Delta t$ sowie
$\vec*{a}{k}^{n+1} := \vec*{F}{k}^n(\vec*{x}{k}^{n+1})/m_k$ und
$\vec*{v}{k}^{n+1} := \vec*{v}{k}^{n+1/2} + \vec*{a}{k}^{n+1} \Delta t/2$.
Das Verfahren hat Fehlerordnung $\O(h^2)$ (ist aber sehr stabil).

\pagebreak

\section{%
    Anwendungen%
}

\textbf{einfache Partikelsimulation}:
Die newtonsche Gravitation einer großen Masse $M$ auf eine kleine Masse $m$
im Punkt $\vec{r}$ wird durch die Gleichung
$m\vec{r}'' = -\frac{GMm}{r^2} \frac{\vec{r}}{|\vec{r}|}$ beschrieben (mit Konstante $G$).
Mit kartesischen Koordinaten $\vec{r} = (x, y)^\tp$ erhält man
$\smallpmatrix{x''\\y''} = -\frac{GM}{(x^2 + y^2)^{3/2}} \smallpmatrix{x\\y}$.
Mit $v_x := x'$ und $v_y := y'$ wandelt man diese ODE-System in das System
$\smallpmatrix{x'\\y'\\v_x'\\v_y'}
= \smallpmatrix{v_x\\v_y\\-\frac{GM}{(x^2 + y^2)^{3/2}} x\\-\frac{GM}{(x^2 + y^2)^{3/2}} y}$
um.
Dieses System kann mit dem Störmer-Verlet-Verfahren gelöst werden.

\linie

\textbf{elektrostatische Umwandlung in Rasterbilder}:\\
Gegeben sei ein Graustufenbild $f\colon \Omega \to [0, 1]$.
Gesucht wird eine Methode, die mithilfe der Elektrostatik $f$ in ein Rasterbild $g$
(nur einzelne überschneidungsfreie schwarze Kreise auf weißer Fläche) umwandelt.
Dazu nimmt man an, dass die Rasterpunkte Partikel gleicher Ladung (z.\,B. Elektronen) sind,
die mithilfe der Coulomb-Kraft $\vec{F} \propto \frac{q_1 q_2}{r^2} \frac{\vec{r}}{|\vec{r}|}$
wechselwirken.
Wegen der gegenseitigen Abstoßung ergibt sich asymptotisch eine gleichmäßige Verteilung
auf $\Omega$.
Daher erzeugt man durch die Grauwerte $f$ eine Verteilung fester positiver Ladungen,
die die Elektronen anziehen.

\linie

\textbf{bilineare Interpolation}:
Ein \begriff{regelmäßiges zweidim. Gitter} ist gegeben durch
$x_i := x_0 + i \Delta x$, $y_j := y_0 + j \Delta y$ für
$i = 0, \dotsc, N_x$ und $j = 0, \dotsc, N_y$.
Eine \begriff{Zelle} $(i, j)$ ist das Rechteck mit den Ecken
$(i, j)$, $(i+1, j)$, $(i+1, j+1)$ und $(i, j+1)$.
Ein Punkt $(x_p, y_p) \in \real^2$ befindet sich in der Zelle
$i = \lfloor \frac{x_p - x_0}{\Delta x} \rfloor$,
$j = \lfloor \frac{y_p - y_0}{\Delta y} \rfloor$.
Um eine Annäherung an den Wert einer Funktion $f(x, y)$ in einem Punkt
$(x, y) \in [x_i, x_{i+1}] \times [y_j, y_{j+1}]$ aus den Werten
$f_{i,j}, f_{i+1,j}, f_{i+1,j+1}, f_{i,j+1}$ an den Zellecken zu erhalten,
benutzt man \begriff{bilineare Interpolation}:\\
$f(x, y) = (1-\alpha)(1-\beta) f_{i,j} + \alpha(1-\beta) f_{i+1,j} +
(1-\alpha)\beta f_{i,j+1} + \alpha\beta f_{i+1,j+1}$, wobei\\
$\alpha := \frac{x - x_i}{\Delta x}, \beta := \frac{y - y_j}{\Delta y} \in [0, 1]$
die \begriff{lokalen Koordinaten} sind.

\linie

\textbf{LIC}:
Sei $\Omega := \{x_0, x_0 + \Delta r, \dotsc, x_0 + N_x \Delta r\} \times
\{y_0, y_0 + \Delta r, \dotsc, y_0 + N_y \Delta r\}$
ein reguläres Gitter mit Gitterweite $\Delta r > 0$ sowie
$\widetilde{\Omega} := [x_0, x_0 + N_x \Delta r] \times [y_0, y_0 + N_y \Delta r]$.
Gesucht ist eine in $\widetilde{\Omega}$ dichte Darstellung eines Stromlinien-Vektorfelds
$\vec{v}\colon \Omega \to \real^2$.
Seien dazu
\begin{itemize}
    \item
    $\vec{\varphi}(s, \vec{\eta})$ die Stromlinie des Vektorfelds
    mit Anfangswert $\vec{\eta} \in \widetilde{\Omega}$ für $s = 0$,

    \item
    $k\colon [-L, L] \to \real$ eine Funktion mit $\int_{-L}^L k(s) \ds = 1$
    (\begriff{Faltungskern}) und

    \item
    $T\colon \Omega \to [0, 1]$ eine \begriff{Rauschtextur}.
\end{itemize}
Dann ist die Intensität $I$ des \begriff{LIC-Bilds} in $(x_i, y_j) \in \Omega$ gegeben durch die
Faltung\\
$I(x_i, y_j) = \int_{-L}^L k(s) T(\vec{\varphi}(s, (x_i, y_j)^\tp)) \ds$
(LIC steht für \begriff{Line Integral Convolution}).

\textbf{OLIC}:
LIC hat bei Verwendung von $k(s) :\equiv \frac{1}{2L}$ den Nachteil, dass die Richtungsinformation
verloren geht.
Durch Verwendung eines asymmetrischen Faltungskerns und Spot Noise bleibt die Richtungsinformation
erhalten, man spricht von \begriff{OLIC (Oriented LIC)}.

\pagebreak

\section{%
    Numerische Bestimmung von kritischen Punkten und Separatrizen%
}

\textbf{Bestimmung von kritischen Punkten}:
Sei ein Vektorfeld $\vec{f}\colon \Omega \to \real^2$ mit $\Omega, \widetilde{\Omega}$ wie eben
gegeben.
Kritische Punkte der zugehörigen ODE sind genau die Nullstellen von $\vec{f}$.
Allerdings ist $\vec{f} = (u, v)^\tp$ nur diskret gegeben.
Daher markiert man zunächst die Gitterpunkte mit $(+,+)$, $(+,-)$, $(-,+)$, $(-,-)$ je nach
Vorzeichen von $u$ und $v$.
Anschließend bestimmt man die Zellen, bei denen sich in den Eckpunkten das Vorzeichen in
beiden Komponenten $u, v$ jeweils ändert.
Anschließend gibt es verschiedene Möglichkeiten:
\begin{itemize}
    \item
    \emph{Isogeraden}:
    Finde mittels Interpolation Nullstellen von $u$ und $v$ auf den Kanten der Zelle,
    verbinde die Nullstellen und schneide die beiden Geraden, die zu $u$ und zu $v$ gehören.

    \item
    \emph{Isolinien aus Interpolation}:
    Besser ist es, $u$ und $v$ jeweils bilinear zu interpolieren und
    Nullstellen der Interpolierenden zu bestimmen.
    Setze also\\
    $u(x, y) = (1-\alpha)(1-\beta) u_{i,j} + \alpha(1-\beta) u_{i+1,j} +
    (1-\alpha)\beta u_{i,j+1} + \alpha\beta u_{i+1,j+1}$ sowie\\
    $v(x, y) = (1-\alpha)(1-\beta) v_{i,j} + \alpha(1-\beta) v_{i+1,j} +
    (1-\alpha)\beta v_{i,j+1} + \alpha\beta v_{i+1,j+1}$ und
    bestimme $\alpha, \beta \in [0, 1]$, sodass $u(x, y) = 0 = v(x, y)$
    (nicht-lineares Gleichungssystem).

    \item
    \emph{Subdivision}:
    Noch besser ist eine Subdivision der Zelle in vier Teilzellen,
    Überprüfung, in welcher der Teilzellen die Nullstelle liegt, und Wiederholung,
    bis die Zellgröße eine bestimmte Grenze unterschritten hat.
    Dieses Verfahren ist numerisch robuster und liefert bessere Ergebnisse.
\end{itemize}

\linie

\textbf{Bestimmung des Typs eines kritischen Punkts}:
Um den Typ eines kritischen Punkts $(x, y)^\tp$ zu bestimmen, muss man die Eigenwerte der
Jacobi-Matrix $J(x, y)$ von $\vec{f}$ im Punkt $(x, y)^\tp$ berechnen.
Zur approximativen Berechnung der Jacobi-Matrix gibt es zwei Möglichkeiten:
\begin{itemize}
    \item
    \emph{Interpolation der Jacobi-Matrix}:
    Berechne zunächst die Jacobi-Matrizen $J_{k,\ell}$ in den Eckpunkten $(k, \ell)$
    der Gitterzelle $(i, j)$, die $(x, y)^\tp$ enthält, also\\
    $J_{k,\ell} := \smallpmatrix{(u_{k+1,\ell} - u_{k-1,\ell}) / (2\Delta x) &
    (u_{k,\ell+1} - u_{k,\ell-1}) / (2\Delta y) \\ (v_{k+1,\ell} - v_{k-1,\ell}) / (2\Delta x) &
    (v_{k,\ell+1} - v_{k,\ell-1}) / (2\Delta y)}$.
    Interpoliere dann die vier Jacobi-Matrizen bilinear, d.\,h.
    $J(x, y) = (1-\alpha)(1-\beta) J_{i,j} + \alpha(1-\beta) J_{i+1,j} +
    (1-\alpha)\beta J_{i,j+1} + \alpha\beta J_{i+1,j+1}$.

    \item
    \emph{Jacobi-Matrix des Interpolanten}:
    Interpoliere\\
    $u(x, y) = (1-\alpha)(1-\beta) u_{i,j} + \alpha(1-\beta) u_{i+1,j} +
    (1-\alpha)\beta u_{i,j+1} + \alpha\beta u_{i+1,j+1}$ sowie\\
    $v(x, y) = (1-\alpha)(1-\beta) v_{i,j} + \alpha(1-\beta) v_{i+1,j} +
    (1-\alpha)\beta v_{i,j+1} + \alpha\beta v_{i+1,j+1}$ und berechne die Jacobi-Matrix
    $J(x, y) = \smallpmatrix{\partial_x u & \partial_y u \\ \partial_x v & \partial_y v}$
    des Interpolanten.
\end{itemize}

\linie

\textbf{Bestimmung der Separatrizen}:
Für jeden Sattelpunkt $\vec{\eta}$ mit Eigenwerten $\lambda_i$ und Eigenvektoren $\vec*{e}{i}$
setze $\vec*{p}{i}^{\pm} := \vec{\eta} \pm \varepsilon \vec*{e}{i}$,
wobei $\varepsilon > 0$ so klein ist, dass sich alle $\vec*{p}{i}^\pm$ immer noch in derselben
Zelle wie $\vec{\eta}$ befinden.
Berechne nun den positiven Halborbit mit Anfangswert $\vec*{p}{i}^{\pm}$, wenn $\lambda_i > 0$,
und den negativen Halborbit, wenn $\lambda_i < 0$.

\pagebreak
