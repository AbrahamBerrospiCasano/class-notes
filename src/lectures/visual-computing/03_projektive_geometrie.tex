\chapter{%
    Projektive Geometrie%
}

\section{%
    Klassfikation von Projektionen%
}

\textbf{planare geometrische Projektionen}:\\
\begriff{Planare geometrische Projektionen} lassen sich unterteilen in:
\begin{itemize}
    \item
    \begriff{parallele Projektionen} (Projektionsstrahlen sind parallel):
    \begin{itemize}
        \item
        \begriff{Orthogonalprojektionen} (Projektionsrichtung normal zur Zeichenebene):
        \begin{itemize}
            \item
            \begriff{Normalprojektionen} (Projektionsrichtung achsenparallel):\\
            Aufsicht, Grundriss usw.

            \item
            \begriff{axonometrische Projektionen}
            (Projektionsrichtung nicht achsenparallel):\\
            \begriff{isometrische/dimetrische/trimetrische Axonometrie}\\
            (Skalierung aller Achsen/von zwei Achsen gleich/aller Achsen unterschiedlich)
        \end{itemize}

        \item
        \begriff{Schrägrisse} (Projektionsrichtung nicht normal zur Zeichenebene):\\
        \begriff{Kabinettriss}, \begriff{Kavalierriss}
    \end{itemize}

    \item
    \begriff{perspektivische Projektionen} (Projektionsstrahlen sind nicht parallel):\\
    \begriff{Einpunkt-/Zweipunkt-/Dreipunkt-Perspektive}
\end{itemize}

\section{%
    Projektive Räume%
}

\textbf{projektiver Raum}:
Sei $\vec{V}$ ein Vektorraum.
Dann heißt die Menge $\PP(\vec{V})$ aller eindimensionalen Unterräume von $\vec{V}$
\begriff{projektiver Raum} mit \begriff{zugrunde liegendem Vektorraum} $\vec{V}$.

\textbf{Dimension}:
Ist $\dim \vv{V} < \infty$, dann ist $\dim \PP(\vec{V}) := \dim \vv{V} - 1$ die
\begriff{Dimension} von $\PP(\vec{V})$.

\textbf{projektiver Raum über $\real$}:\\
$\PP^n(\real) := \PP(\vv{\real}^{n+1})$ heißt
\begriff{$n$-dimensionaler projektiver Raum über $\real$}.

\textbf{kanonische Abbildung}:
$\pi\colon \vec{V} \setminus \{\vec{0}\} \to \PP(\vec{V})$, $\pi(\vec{v}) = \real\vec{v}$
heißt \begriff{kanonische Abbildung}.\\
Für zwei Vektoren $\vec{v}, \vec{w} \in \vec{V} \setminus \{\vec{0}\}$ gilt
$\pi(\vec{v}) = \pi(\vec{w}) \iff$ $\vec{v}, \vec{w}$ sind linear abhängig.

\textbf{projektiver Unterraum}:
Eine Teilmenge $\XX \subset \PP(\vec{V})$ heißt \begriff{projektiver Unterraum},
falls es einen Unterraum $\vec{U} \le \vec{V}$ gibt mit $\XX = \PP(\vec{U})$.

\section{%
    Homogene Koordinaten%
}

\textbf{homogene Koordinaten}:
Sei $\vec{v} = (v_1, \dotsc, v_n, v_{n+1}) \in \vec{\real}^{n+1} \setminus \{\vec{0}\}$.\\
Dann heißt $(v_1 : \dotsc : v_n : v_{n+1}) := \real\vec{v} \in \PP^n(\real)$
\begriff{homogene Koordinaten} von $\vec{v}$.

Homogene Koordinaten sind nicht eindeutig.
Sie werden eindeutig, wenn man sie normiert,
überlicherweise in der letzten Komponente.

\textbf{Fernhyperebene}:
$\H := \{(x_1 : \dotsc : x_n : 0) \;|\; x_1, \dotsc, x_n \in \real\} \subset \PP^n(\real)$
heißt \begriff{Fernhyperebene}.

\textbf{af"|finer Teil}:
$\PP^n(\real) \setminus \H$ heißt \begriff{af"|finer Teil des projektiven Raums $\PP^n(\real)$}.

\textbf{Einbettung von $\real^n$ in $\PP^n(\real)$}:\\
Die Abbildung
$\vec{\real}^n \to \PP^n(\real) \setminus \H$,
$(x_1, \dotsc, x_n) \mapsto (x_1 : \dotsc : x_n : 1)$
ist eine Bijektion\\
mit Umkehrabbildung
$\PP^n(\real) \setminus \H \to \vec{\real}^n$,
$(y_1 : \dotsc : y_n : y_{n+1}) \mapsto (\frac{y_1}{y_{n+1}}, \dotsc, \frac{y_n}{y_{n+1}})$.

\pagebreak

\section{%
    Projektive Geraden in der projektiven Ebene%
}

\textbf{projektive Gerade}:
$\PP^1(\real)$ ist eine \begriff{projektive Gerade}.\\
Der af"|fine Teil ist $\H = \{(x_1 : x_2) \in \PP^1(\real) \;|\; x_2 \not= 0\}$
und $(1 : 0)$ ist der Fernpunkt.

\textbf{projektive Ebene}:
$\PP^2(\real)$ ist eine \begriff{projektive Ebene}.\\
Der af"|fine Teil ist $\H = \{(x_1 : x_2 : x_3) \in \PP^2(\real) \;|\; x_3 \not= 0\}$,
wogegen die Fernpunkte durch $\{(x_1 : x_2 : 0) \in \PP^2(\real) \;|\;
(x_1, x_2) \in \vec{\real}^2\}$ gegeben sind.

\linie

\textbf{projektive Geraden in der projektiven Ebene}:
Eine Gerade in der af"|finen Ebene $\vec{\real}^2$ kann implizit durch
$n_1x_1 + n_2x_2 + n_3 = 0$ dargestellt werden.
Indem man $x_1 = \xi_1/\xi_3$ und $x_2 = \xi_2/\xi_3$ mit $\xi_3 \not= 0$ setzt,
erhält man die Gleichung $n_1 \xi_1 + n_2 \xi_2 + n_3 \xi_3 = 0 \iff \vec{n} \cdot \vec{\xi} = 0$.
Die Gerade ist also gegeben durch Projektion von
$\vec{U} = \{\vec{\xi} \in \real^3 \;|\; \vec{n} \cdot \vec{\xi} = 0\}$
auf die af"|fine Ebene $\xi_3 = 1$.
$\vec{U}$ ist ein zweidimensionaler Unterraum von $\vec{\real}^3$, d.\,h.
$\PP(\vec{U})$ ist ein projektiver Unterraum der Dimension $1$
der projektiven Ebene $\PP^2(\real)$, also eine projektive Gerade.

\linie

\textbf{projektive Gerade durch zwei Punkte}:
Seien $p, q \in \PP^2(\real)$.
Gesucht ist eine projektive Gerade $L := \{\xi \in \real^3 \;|\; \vec{n} \cdot \vec{\xi} = 0\}$
mit $\vec{p}, \vec{q} \in L \iff \vec{n} \cdot \vec{p} = 0 = \vec{n} \cdot \vec{q}$.\\
Der gesuchte Vektor $\vec{n}$ ist also durch $\vec{n} := \vec{p} \times \vec{q}$ gegeben.

\textbf{Schnitt zwischen zwei projektiven Geraden}:
Seien $L_1, L_2 \subset \PP^2(\real)$ zwei projektive Geraden
$L_1\colon \vec{n} \cdot \vec{\xi} = 0$ und $L_2\colon \vec{m} \cdot \vec{\xi} = 0$ mit
$\vec{n}, \vec{m} \in \vec{\real}^3$.
Gesucht ist ein projektiver Punkt $p = \real\vec{p} \in \PP^2(\real)$ mit $\vec{p} \in L_1 \cap L_2
\iff \vec{n} \cdot \vec{p} = 0 = \vec{m} \cdot \vec{p}$.\\
Der gesuchte Punkt $p$ ist also durch $\vec{p} := \vec{n} \times \vec{m}$ gegeben.

\section{%
    Projektive Basen%
}

\textbf{projektiv unabhängig}:
$r+1$ Punkte $p_1, \dotsc, p_{r+1} \in \PP^n(\real)$ heißen \begriff{projektiv unabhängig}, falls
$\vec*{p}{1}, \dotsc, \vec*{p}{r+1} \in \vec{\real}^{n+1} \setminus \{\vec{0}\}$ linear unabhängig
sind, wobei $p_i = \real\vec*{p}{i}$.

\textbf{projektive Basis}:
$n+2$ Punkte $p_1, \dotsc, p_{n+2} \in \PP^n(\real)$ heißen
\begriff{projektive Basis/projektives Koordina"-tensystem}
von $\PP^n(\real)$, falls je $n+1$ Punkte projektiv unabhängig sind.

\textbf{Beispiel}:
Die vier Punkte $(1:0:0), (0:1:0), (0:0:1), (1:1:1) \in \PP^2(\real)$ bilden die
\begriff{projektive Standardbasis} von $\PP^2(\real)$.

\pagebreak

\section{%
    Projektive Abbildungen%
}

\textbf{projektive Abbildung}:
Eine Abbildung $F\colon \PP(\vec{V}) \to \PP(\vec{W})$ zwischen zwei projektiven Räumen heißt
\begriff{projektive Abbildung/Projektivität}, falls es eine injektive lineare Abbildung
$f\colon \vec{V} \to \vec{W}$ gibt mit
$\forall_{v \in \vec{V} \setminus \{\vec{0}\}}\; F(\real\vec{v}) = \real f(\vec{v})$.

\textbf{Lemma}:
Zwei injektive lineare Abbildungen $f_1, f_2\colon \vec{V} \to \vec{W}$ definieren dieselbe
projektive Abbildung genau dann, wenn $\exists_{\lambda\not=0}\; f_2 = \lambda f_1$.

\textbf{Satz (eindeutige projektive Abbildung für projektive Basen)}:\\
Seien $\dim \vec{V} = n + 1 = \dim \vec{W}$ sowie $p_1, \dotsc, p_{n+2} \in \PP(\vec{V})$ und
$q_1, \dotsc, q_{n+2} \in \PP(\vec{W})$ projektive Basen von $\PP(\vec{V})$ bzw. $\PP(\vec{W})$.\\
Dann gibt es genau eine projektive Abbildung $F\colon \PP(\vec{V}) \to \PP(\vec{W})$ mit
$\forall_{i = 1, \dotsc, n+2}\; F(p_i) = q_i$.

\linie

\textbf{projektive Abb. als af"|fine Abb.}:
Sei $F\colon \PP^n(\real) \to \PP^n(\real)$ eine projektive Abbildung.\\
Dann heißt $F$ \begriff{af"|fine Abbildung}, falls $F(\H) = \H$,
wobei $\H$ die Fernhyperebene von $\PP^n(\real)$ ist\\
(oder äquivalent $p \in \H \iff F(p) \in \H$).

\textbf{Bedingung für af"|fine Abb.}:
Sei $F\colon \PP^n(\real) \to \PP^n(\real)$ eine projektive Abbildung mit zugehöriger
linearer Abbildung $f\colon \vec{\real}^{n+1} \to \vec{\real}^{n+1}$.
Dann gibt es eine invertierbare Matrix $A \in \GL_{n+1}(\real)$ mit
$f(\vec{v}) = A\vec{v}$ für alle $\vec{v} \in \vec{\real}^{n+1}$.
Ist $p = (x_1 : \dotsc : x_{n+1}) \in \PP^n(\real)$ ein projektiver Punkt,
dann ist $F(p) = q := (y_1 : \dotsc : y_{n+1})$ mit
$y_i := (A\vec{x})_i = a_{i,1} x_1 + \dotsb + a_{i,n+1} x_{n+1}$.
Wenn $F$ af"|fin ist, dann muss
$0 \overset{!}{=} y_{n+1} = a_{n+1,1} x_1 + \dotsb + a_{n+1,n} x_n + a_{n+1,n+1} 0$ für alle
$p = (x_1 : \dotsc : x_n : 0) \in \H$ gelten.
Das geht aber nur für $a_{n+1,1} = \dotsb = a_{n+1,n} = 0$.
Damit gilt dann $F(\H) \subset \H$.\\
($p \notin \H \implies F(p) \notin \H$ gilt deshalb, weil
$a_{n+1,n+1} \not= 0$ aufgrund $A$ invertierbar gelten muss und daher
$y_{n+1} = a_{n+1,n+1} x_{n+1} \not= 0$ für
$p = (x_1 : \dotsc : x_n : x_{n+1}) \notin \H \iff x_{n+1} \not= 0$.)\\
Somit ist $F$ af"|fin genau dann, wenn $a_{n+1,1} = \dotsb = a_{n+1,n} = 0$.

\linie

\textbf{Beispiele}:
\begin{itemize}
    \item
    \emph{af"|fine Verschiebung in $\PP^3(\real)$}:
    Ist $\vec{t} := (t_x, t_y, t_z)^\tp$ der Verschiebungsvektor, so definieren
    $T := \smallpmatrix{1&0&0&t_x\\0&1&0&t_y\\0&0&1&t_z\\0&0&0&1}$ bzw.
    $\widetilde{T} := \smallpmatrix{t_w&0&0&t_x\\0&t_w&0&t_y\\0&0&t_w&t_z\\0&0&0&t_w}$
    eine Verschiebung in homogenen Koordinaten.\\
    Will man einen projektiven Punkt $p = (p_x:p_y:p_z:1)$ um $t = (t_x:t_y:t_z:1)$ verschieben,
    so erhält man $Tp = (p_x + t_x, p_y + t_y, p_z + t_z, 1)^\tp$.\\
    Will man einen projektiven Punkt $p = (p_x:p_y:p_z:p_w)$ um $t = (t_x:t_y:t_z:t_w)$
    verschieben, so erhält man $\widetilde{T}p
    = (t_w p_x + t_x p_w, t_w p_y + t_y p_w, t_w p_z + t_z p_w, t_w p_w)^\tp$ mit\\
    $\real(\widetilde{T}p) = (\frac{p_x}{p_w} + \frac{t_x}{t_w} :
    \frac{p_y}{p_w} + \frac{t_y}{t_w} : \frac{p_z}{p_w} + \frac{t_z}{t_w} : 1)$.

    \item
    \emph{af"|fine Drehungen in $\PP^3(\real$)}:
    Af"|fine Drehungen mit Winkel $\alpha$ werden definiert durch\\
    $R_x(\alpha) := \smallpmatrix{1&0&0&0\\0&\cos\varphi&-\sin\varphi&0\\
    0&\sin\varphi&\cos\varphi&0\\0&0&0&1}$,
    $R_y(\alpha) := \smallpmatrix{\cos\varphi&0&\sin\varphi&0\\
    0&1&0&0\\-\sin\varphi&0&\cos\varphi&0\\0&0&0&1}$,
    $R_z(\alpha) := \smallpmatrix{\cos\varphi&-\sin\varphi&0&0\\
    \sin\varphi&\cos\varphi&0&0\\0&0&1&0\\0&0&0&1}$.
\end{itemize}

\pagebreak

\section{%
    Parallele Orthogonalprojektionen%
}

\subsection{%
    Normalprojektion%
}

\textbf{Normalprojektion}:
Bei der \begriff{Normalprojektion} ist die Projektionsebene achsenparallel.
Zum Beispiel ist die Normalprojektion eines Punkts $p = (x:y:z:1)$ auf die Ebene $z = z_0$
gegeben durch
$\smallpmatrix{1&0&0&0\\0&1&0&0\\0&0&0&z_0\\0&0&0&1} \smallpmatrix{x\\y\\z\\1} =
\smallpmatrix{x\\y\\z_0\\1}$.

\subsection{%
    Axonometrische Projektion%
}

\textbf{axonometrische Projektion}:
Bei \begriff{axonometrischen Projektionen} ist die Projektionsrichtung $\vec{p}$ orthogonal
zur Zeichenebene (die durch den Ursprung geht).
Sie lassen sich aufspalten in eine Koordinatentransformation und eine anschließende
Normalprojektion auf die Ebene $z' = 0$,
wobei die Koordinatentransformation die Zeichenebene auf die Ebene $z' = 0$ abbildet.

\textbf{Herleitung der Projektionsmatrix}:
Seien $\vec{p} \in \real^3$ die Projektionsrichtung (Normale der Zeichenebene) mit $|\vec{p}| = 1$,
$\vec{u} \in \real^3$ ein beliebiger Vektor mit $|\vec{u}| = 1$, der sog. \begriff{Hochvektor},
und $\vec*{e}{x}, \vec*{e}{y}, \vec*{e}{z} \in \vec{\real}^3$ die Standardbasis von
$\vec{\real}^3$.
Dann ist $\vec*{e}{x}', \vec*{e}{y}', \vec*{e}{z}' \in \vec{\real}^3$
mit $\vec*{e}{z}' := -\vec{p}$,
$\vec*{e}{x}' := \frac{\vec{u} \times \vec*{e}{z}'}{|\vec{u} \times \vec*{e}{z}'|}$ und
$\vec*{e}{y}' := \vec*{e}{z}' \times \vec*{e}{x}'$
ein rechtshändiges Koordinatensystem, dessen $x'$-$y'$-Ebene in der Projektionsebene liegt.
Die Koordinatentransformation erhält man durch
$S := \smallpmatrix{\vec*{e}{x}' & \vec*{e}{y}' & \vec*{e}{z}'}$,
wobei sich die Koordinaten mit $S^{-1} = S^\tp$ transformieren ($S$ ist orthogonal),
in homogenen Koordinaten also $A := \smallpmatrix{S^{-1} & \vec{0} \\ \vec{0}^\tp & 1}$.
Mit der Projektion auf die Ebene $z' = 0$ bekommt man
$P := P_z A = \smallpmatrix{1&0&0&0\\0&1&0&0\\0&0&0&0\\0&0&0&1} \cdot A
= \smallpmatrix{e_{x1}'&e_{x2}'&e_{x3}'&0\\
e_{y1}'&e_{y2}'&e_{y3}'&0\\0&0&0&0\\0&0&0&1}$.

\vspace{3mm}
\linie

\textbf{Beispiel}:
Mit $\vec{p} := \frac{1}{\sqrt{3}} (-1, -1, -1)^\tp$ und $\vec{u} := (0, 0, 1)^\tp$
bekommt man die Projektionsmatrix
$P = \smallpmatrix{-\frac{1}{\sqrt{2}}&\frac{1}{\sqrt{2}}&0&0\\
-\frac{1}{\sqrt{6}}&-\frac{1}{\sqrt{6}}&\frac{2}{\sqrt{6}}&0\\0&0&0&0\\0&0&0&1}$.
Man kann berechnen, dass zwischen den Bildpunkten $\widetilde{x_i} = Px_i$ von
$x_1 := (1:0:0:1)$, $x_2 := (0:1:0:1)$ und $x_3 := (0:0:1:1)$ und der $x'$-Achse der
Projektionsebene die Winkel \SI{150}{\degree} bzw. \SI{30}{\degree} liegen.
Außerdem bleibt das Seitenverhältnis erhalten, es handelt sich um eine Isometrie.

\textbf{andere parallele Projektionen}:
Man zählt außerdem noch dimetrische (nur zwei Achsen werden nicht skaliert)
und trimetrische (nur eine Achse wird nicht skaliert) Projektionen zu den parallelen
Orthogonalprojektionen.
Ist die Projektionsrichtung nicht mehr normal zur Projektionsebene, so spricht man von
Schrägrissen (dazu gehören Kavlierriss und Kabinettriss).

\pagebreak

\section{%
    Perspektivische Projektionen%
}

\subsection{%
    Einpunkt-Perspektive%
}

\textbf{Herleitung der Einpunkt-Perspektive in $\PP^2(\real)$}:
Seien $p, o \in \PP^2(\real)$ zwei af"|fine Punkte mit $p = (x:y:1)$ und $o := (-x_0:0:1)$,
wobei $x_0 > 0$.
Der Beobachter befindet sich im \begriff{Augpunkt} $o$ und die Projektionsgerade sei die $y$-Achse.
Um $p$ auf die $y$-Achse zu projizieren, benutzt man den Strahlensatz und erhält
$y_0 = \frac{x_0}{x + x_0} y$ als $y$-Achsenabschnitt, d.\,h.\\
$(x:y:1) \mapsto (0:y_0:1) = (0:\frac{x_0}{x + x_0} y:1) = (0:y:1+\frac{x}{x_0})$.
In Matrixschreibweise erhält man\\
$\smallpmatrix{0&0&0\\0&1&0\\\frac{1}{x_0}&0&1} \smallpmatrix{x\\y\\1} =
\smallpmatrix{0\\y\\1+\frac{x}{x_0}}$.
Definiert man $P_x := \smallpmatrix{0&0&0\\0&1&0\\0&0&1}$ und
$T_1 := \smallpmatrix{1&0&0\\0&1&0\\\frac{1}{x_0}&0&1}$,
soll lässt sich die Matrix aufteilen in $P_x T_1$ mit
$P_x$ der Projektion auf die $y$-Achse und
$T_1$ der \begriff{perspektivischen Transformation der Einpunkt-Perspektive}.
Es gilt $T_1 p = (x : y : \frac{x}{x_0} + w)$ mit $p := (x:y:w)$.

\linie

\textbf{Eigenschaften der perspektivischen Transformation $T_1$}:
\begin{itemize}
    \item
    \emph{Grenze $x = -x_0$ des Sichtfelds wird auf $\H$ abgebildet}:
    Für Punkte $(-x_0:y:1)$ auf der Grenze $x = -x_0$ des Sichtfelds $x > -x_0$ gilt
    $T_1 (-x_0:y:1) = (-x_0:y:0)$.

    \item
    \emph{Punkte auf Projektionsgerade sind Fixpunkte}:
    Für Punkte $(0:y:1)$ auf der Projektionsgeraden (hier die $y$-Achse) gilt
    $T_1 (0:y:1) = (0:y:1)$, d.\,h. die Projektionsgerade ist eine Fixpunktgerade.

    \item
    \emph{Parallelen zur Projektionsgeraden bleiben parallel}:
    Alle Parallelen zur Projektionsgeraden schneiden sich im Fernpunkt $(0:1:0)$,
    der ein Fixpunkt ist ($T_1 (0:1:0) = (0:1:0)$).
    Damit bleiben Parallelen zur Projektionsgeraden nach der Projektion auch parallel.

    \item
    \emph{Bild von Geraden durch $o$}:
    Sei $g_1$ die projektive Gerade durch $o$ und $p := (0:y_p:1)$.
    Dann geht das Bild $T_1 g_1$ der Geraden durch $o' := (1:0:0)$ und $p$ ($p$ ist ein Fixpunkt).
    Man erhält $\vec{n} = \vec{o}' \times \vec{p} = (0, -1, y_p)^\tp$
    (Formel für die Gerade durch zwei Punkte).
    Somit ist die Gerade $T_1 g_1$ gegeben durch $-\xi_2 + y_p \xi_3 = 0 \iff y = y_p$.
    Das Bild von Geraden durch $o$ ist also nach der perspektivischen Transformation orthogonal
    zur Projektionsgeraden.

    \item
    \emph{die Bilder von Parallelen schneiden sich in einem af"|finen Punkt}:\\
    Sei wieder $p := (0:y_p:1)$.
    Dann schneiden sich alle Geraden, die parallel zu der durch $o$ und $p$ sind,
    im Fernpunkt $s := (x_0:y_p:0)$.
    Damit müssen sich die Bilder dieser Geraden alle im Punkt $s' := T_1 s = (x_0:y_p:1)$ schneiden.
    Unabhängig von $y_p$ liegt dieser gemeinsame Schnittpunkt (\begriff{Fluchtpunkt})
    auf der Geraden $x = x_0$, der sog. \begriff{Fluchtgeraden}.
    Alle Geraden enden in diesem Punkt.
    Wählt man $y_0 = 0$ (also die Parallelen zur $x$-Achse), dann erhält man den
    \begriff{Hauptfluchtpunkt} $(x_0:0:1)$.
\end{itemize}

\pagebreak

\subsection{%
    Zweipunkt-Perspektive%
}

\textbf{Zweipunkt-Perspektive in $\PP^2(\real)$}:
Die \begriff{perspektivische Transformation der Zweipunkt-Perspek"-tive} ist gegeben durch
$T_2 p := \smallpmatrix{1&0&0\\0&1&0\\\frac{1}{x_0}&\frac{1}{y_0}&1} \smallpmatrix{x\\y\\w}
= \smallpmatrix{x\\y\\\frac{x}{x_0} + \frac{y}{y_0} + w}$ mit $x_0, y_0 > 0$.

\vspace{2mm}
\linie

\textbf{Grenze des Sichtfelds}:
Punkte der af"|finen Gerade $\frac{x}{x_0} + \frac{y}{y_0} = -1$ durch $o_1 := (-x_0:0:1)$ und
$o_2 := (0:-y_0:1)$ werden auf Fernpunkte abgebildet, d.\,h. $(x:y:1) \mapsto (x:y:0)$.
Diese Gerade definiert daher die \begriff{Grenze des Sichtfelds}.

\textbf{Projektionsgerade}:
Die Gerade $\frac{x}{x_0} + \frac{y}{y_0} = 0$ ist eine Fixpunktgerade, die
\begriff{Projektionsgerade}.

% \textbf{Herleitung des Augpunkts $o$}:
% Der Augpunkt $o = (o_x:o_y:1)$ liegt auf der \begriff{optischen Achse} $g$,
% die auf sich selbst abgebildet werden und orthogonal zur Projektionsgeraden sein sollte.
% Angenommen, $g$ geht durch $o$ und einen Punkt $q$ auf der Projektionsgeraden, d.\,h.\\
% $q = (q_x : -\frac{y_0}{x_0} q_x : 1)$.
% Um $q_x$ zu bestimmen, betrachtet man die Bildgerade von $g$, die durch
% $\vec{n}' := (o_x, o_y, 0)^\tp \times (q_x, -\frac{y_0}{x_0} q_x, 1)^\tp
% = (o_y, -o_x, -q_x(\frac{y_0}{x_0} o_x + o_y))^\tp$ definiert ist.
% Weil sie invariant unter $T_2$ sein soll, muss $o$ darauf liegen,
% also $0 = \vec{n}' \cdot (o_x, o_y, 1)^\tp = -q_x(\frac{y_0}{x_0} o_x + o_y)$.
% Weil $o$ nicht auf der Projektionsgeraden liegt, gilt $\frac{y_0}{x_0} o_x + o_y \not= 0$,
% also $q_x = 0$ und es ist $q = (0:0:1)$.\\
% Mit der zweiten Bedingung ($g$ orthogonal zur Projektionsgeraden)
% folgert man, dass $o$ außerdem auf der Geraden $y = m_h x$ durch $q$ und senkrecht zur
% Projektionsgerade $y = m_g x$ mit $m_g := -\frac{y_0}{x_0}$ sein sollte.
% Es folgt $m_h = -\frac{1}{m_g} = \frac{x_0}{y_0}$.
% Durch Schnitt von $y = m_g x$ mit $y = m_h x$ erhält man schließlich
% $o = \left(-\frac{x_0 y_0^2}{x_0^2 + y_0^2} : -\frac{x_0^2 y_0}{x_0^2 + y_0^2} : 1\right)$
% und $T_2 o = \left(-\frac{x_0 y_0^2}{x_0^2 + y_0^2} : -\frac{x_0^2 y_0}{x_0^2 + y_0^2} : 0\right)
% = (y_0 : x_0 : 0)$.

\textbf{Augpunkt}:
$o = \left(-\frac{x_0 y_0^2}{x_0^2 + y_0^2} : -\frac{x_0^2 y_0}{x_0^2 + y_0^2} : 1\right)$
mit $T_2 o = (y_0 : x_0 : 0)$

\linie

\textbf{Bild von Geraden durch $o$}:
Geraden durch $o$ werden auf Geraden abgebildet, die zur Projektionsgeraden senkrecht sind.

\textbf{Bild von Parallelen zu Koordinatenachsen}:
Parallelen zu den Koordinatenachsen werden auf Geraden abgebildet, die in einem der
zwei \begriff{Hauptfluchtpunkte} $(x_0:0:1)$, $(0:y_0:1)$ enden.
Die Gerade $\frac{x}{x_0} + \frac{y}{y_0} = 1$,
die die beiden Hauptfluchtpunkte verbindet, heißt \begriff{Fluchtgerade}:
Die Bilder aller anderen Geraden enden in einem Punkt auf der Fluchtgeraden.
Außerdem werden Fernpunkte $(s:t:0)$ auf af"|fine Punkte der Fluchtgeraden abgebildet.

\textbf{Zweipunkt-Perspektive durch Drehung der Einpunkt-Perspektive}:\\
Es gilt $T_2 = R(\mu) T_1 R(-\mu)$ mit
$R(\mu) = \smallpmatrix{\cos\mu&-\sin\mu&0\\\sin\mu&\cos\mu&0\\0&0&1}$
(Eintrag von $T_1$ links unten ist $\frac{1}{d}$).\\
Sind $x_0, y_0$ bekannt, so ist $\mu = \arctan \frac{x_0}{y_0}$ und
$d = \frac{x_0y_0}{\sqrt{x_0^2+y_0^2}}$.

\subsection{%
    Dreipunkt-Perspektive%
}

\textbf{Dreipunkt-Perspektive in $\PP^3(\real)$}:
Die \begriff{perspektivische Transformation der Dreipunkt-Per"-spektive} ist gegeben durch
$T_3 p := \smallpmatrix{1&0&0&0\\0&1&0&0\\0&0&1&0\\\frac{1}{x_0}&\frac{1}{y_0}&\frac{1}{z_0}&1}
\smallpmatrix{x\\y\\z\\w}
= \smallpmatrix{x\\y\\z\\\frac{x}{x_0} + \frac{y}{y_0} + \frac{z}{z_0} + w}$ mit
$x_0, y_0, z_0 > 0$.

\textbf{Eigenschaften}:
Die Grenzebene $\frac{x}{x_0} + \frac{y}{y_0} + \frac{z}{z_0} = -1$ des Sichtfelds wird
auf $\H$ abgebildet.
Die Projektionsebene $\frac{x}{x_0} + \frac{y}{y_0} + \frac{z}{z_0} = 0$ ist eine Fixpunktebene.
Der Augpunkt $o$ liegt auf der Grenzebene und alle Geraden durch $o$ werden auf af"|fine Geraden
abgebildet, die senkrecht zur Projektionsebene liegen.
Daraus folgt $o = \left(-\frac{x_0 y_0^2 z_0^2}{x_0^2 + y_0^2 + z_0^2} :
-\frac{x_0^2 y_0 z_0^2}{x_0^2 + y_0^2 + z_0^2} : -\frac{x_0^2 y_0^2 z_0}{x_0^2 + y_0^2 + z_0^2} :
1\right)$.
Fernpunkte $(s:t:u:0)$ auf $\H$ werden auf $(s:t:u:\frac{s}{x_0}+\frac{t}{y_0}+\frac{u}{z_0})$
abgebildet, d.\,h. landen auf der
\begriff{Fluchtebene} $\frac{x}{x_0} + \frac{y}{y_0} + \frac{z}{z_0} = 1$.
Die Hauptfluchtpunkte sind $(x_0:0:0:1)$, $(0:y_0:0:1)$ und $(0:0:z_0:1)$.

\pagebreak

\subsection{%
    Rekonstruktion von perspektivischen Transformationen%
}

\textbf{Rekonstruktion}:
Bei dem Problem der \begriff{Rekonstruktion von perspektivischen Transformationen}
in $\PP^3(\real)$ sind fünf Paare $(P_i, P_i')$ von Punkten in $\PP^3(\real)$ gegeben.
Gesucht ist die perspektivische Transformation $T$ mit $TP_i = P_i'$ für $i = 1, \dotsc, 5$.
Die Rekonstruktion verläuft wie folgt:
\begin{enumerate}
    \item
    Wähle die projektive Standardbasis $p_1 := (1:0:0:0)$, $p_2 := (0:1:0:0)$,\\
    $p_3 := (0:0:1:0)$, $p_4 := (0:0:0:1)$, $p_5 := (1:1:1:1)$ von $\PP^3(\real)$.

    \item
    Bestimme homogene Matrizen $A$ und $B$, sodass $Ap_i = P_i$ und $Bp_i = P_i'$ für
    $i = 1, \dotsc, 4$.
    Weil sie homogen sind, sind sie nur bis auf skalare Faktoren $\lambda$ bzw. $\lambda'$
    eindeutig bestimmt
    (jede Spalte mit einem eigenem Faktor $\lambda_i$ bzw. $\lambda_i'$ durchmultiplizieren).

    \item
    Indem $Ap_5 = P_5$ und $Bp_5 = P_5'$ eingesetzt wird, kann man $\lambda_1, \dotsc, \lambda_4$
    bzw. $\lambda_1', \dotsc, \lambda_4'$ in Abhängigkeit von $\lambda_5$ bzw. $\lambda_5'$
    darstellen (nach Lösung von jeweils einem LGS).

    \item
    Aus $P_i' = TP_i$ folgt $Bp_i = TAp_i$, was äquivalent zu $B = TA$ ist.
    Nun lässt sich $T$ (wiederum nur bis auf einen skalaren Faktor)
    durch $T = BA^{-1}$ bestimmen.
\end{enumerate}

\section{%
    OpenGL-Vertex-Pipeline%
}

\begin{floatingfigure}[r]{54mm}
    \vspace{-7mm}
    \displaymathother
    \begin{align*}
        \xymatrix@R=10mm@C=20mm{
            *=<47mm,8mm>[F]{\text{Objektkoord. $p_\text{obj}$}}\ar[d]^{M_\text{model}}\\
            *=<47mm,8mm>[F]{\text{Weltkoord. $p_\text{world}$}}\ar[d]^{M_\text{view}}\\
            *=<47mm,8mm>[F]{\text{Kamerakoord. $p_\text{cam}$}}\ar[d]^{M_\text{proj}}\\
            *=<47mm,8mm>[F]{\text{Clipkoord. $p_\text{clip}$}}\ar[d]^{\text{proj. Divison}}\\
            *=<47mm,8mm>[F]{\text{norm. Gerätekoord. $p_\text{ndc}$}}\ar[d]^{\text{Sichtbereich}}\\
            *=<47mm,8mm>[F]{\text{Fensterkoord. $p_\text{win}$}}
        }
    \end{align*}
    \displaymathnormal
\end{floatingfigure}
\textbf{OpenGL-Vertex-Pipeline}:
Rechts dargestellt ist die \begriff{OpenGL-""Vertex-Pipeline}, also die verschiedenen Wechsel von
Koordinatensystemen, die OpenGL durchführt, um letztendlich zu bestimmen, wo auf dem Bildschirm
was angezeigt wird.

\textbf{Objektkoordinaten}:
Jedes Objekt "`lebt"' in seinem eigenen Koordinatensystem und hat eigene
\begriff{Objektkoordinaten} $p_\text{obj}$.

\textbf{Weltkoordinaten}:
Wenn man alle darzustellenden Objekte in ein Koordinatensystem legt, erhält man die
\begriff{Weltkoordinaten} $p_\text{world}$.
Von den Objekt- zu den Weltkoordinaten gelangt man durch eine Transformation $M_\text{model}$,
die Rotationen, Scherungen, Skalierungen und Translationen enthalten kann.

\textbf{Kamerakoordinaten}:
Die Welt aus der Sicht der Kamera wird mithilfe der  \begriff{Kamerakoordinaten} $p_{\text{cam}}$
ausgedrückt.
Die Kamera befindet sich dann im Ursprung und schaut in negative $z$-Richtung,
wobei der Hochvektor der positiven $y$-Richtung entspricht.
Von den Welt- zu den Kamerakoordinaten gelangt man durch die Transformation
$M_\text{view} :=
\smallpmatrix{\vec*{e}{x}'^\tp&-c_x\\\vec*{e}{y}'^\tp&-c_y\\\vec*{e}{z}'^\tp&-c_z\\\vec{0}^\tp&1}$,
wobei\\
$\vec*{e}{x}', \vec*{e}{y}', \vec*{e}{z}'$ die Basisvektoren des Kamerasystems sind und
$\vec{c}$ die Kameraposition ist (jeweils bzgl. Weltkoordinaten).

\linie
\pagebreak

\textbf{Clipkoordinaten}:
Die \begriff{Clipkoordinaten} $p_\text{clip}$ geben an, was die Kamera sehen kann und was nicht.
Dieser Bereich heißt \begriff{Sichtvolumen}.
Es gibt zwei Arten von Sichtvolumen: orthografisches Sichtvolumen und perspektivischer Sichtstumpf.
Zur Vereinheitlichung transformiert man am Ende den sichtbaren Bereich auf den Würfel
$[-1, 1]^3$, das sog. \begriff{kanonische Sichtvolumen}.
Nahe Objekte sind dabei bei $z = -1$ und ferne Objekte sind bei $z = 1$.

\textbf{orthografisches Sichtvolumen}:
Beim \begriff{orthografischen Sichtvolumen} seien Werte $l < r$, $b < t$ und $0 < n < f$
vorgegeben.
Das Sichtvolumen umfasst dann den Quader $[l, r] \times [b, t] \times [-f, -n]$\\
(die Kamera befindet sich dabei im Ursprung und blickt in negative $z$-Richtung).
Zur Transformation in das kanonische Sichtvolumen benutzt man die Transformation\\
$M_\text{proj}^\text{orth}
:= \smallpmatrix{\frac{2}{r-l}&0&0&0\\0&\frac{2}{t-b}&0&0\\0&0&\frac{-2}{f-n}&0\\0&0&0&1} \cdot
\smallpmatrix{1&0&0&-\frac{r+l}{2}\\0&1&0&-\frac{t+b}{2}\\0&0&1&\frac{f+n}{2}\\0&0&0&1}
= \smallpmatrix{\frac{2}{r-l}&0&0&-\frac{r+l}{r-l}\\0&\frac{2}{t-b}&0&-\frac{t+b}{t-b}\\
0&0&\frac{-2}{f-n}&-\frac{f+n}{f-n}\\0&0&0&1}$\\
(Komposition von Translation des Mittelpunkts in $\vec{0}$ und anschließender Skalierung).

\textbf{Sichtstumpf}:
Beim \begriff{Sichtstumpf} legt man fest, dass das Sichtvolumen einem Pyramidenstumpf entspricht.
Ist wieder $l < r$, $b < t$ und $0 < n < f$, dann hat die Pyramide ihre Grundfläche in der Ebene
$z = -f$ und ihre Spitze im Ursprung (Kameraposition).
Sie wird in der Ebene $z = -n$ abgeschnitten, die Deckfläche ist gegeben durch
$[l, r] \times [b, t] \times \{-n\}$.
Die Transformation
$M_\text{proj}^\text{persp}
:= nI \cdot M_\text{proj}^\text{orth} \cdot T_z(-n) \cdot S_z(f/n) \cdot P_z(-n) \cdot T_z(n)$
transformiert den Sichtstumpf in das Volumen $[-n, n]^3$,
d.\,h. $\frac{1}{n} M_\text{proj}^\text{persp}$ transformiert den Sichtstumpf in das
kanonische Sichtvolumen ($nI$ ist nur kosmetisch).
Dabei sind
$T_z(n) := \smallpmatrix{1&0&0&0\\0&1&0&0\\0&0&1&n\\0&0&0&1}$,
$S_z(n) := \smallpmatrix{1&0&0&0\\0&1&0&0\\0&0&s&0\\0&0&0&1}$,
$P_z(-n) := \smallpmatrix{1&0&0&0\\0&1&0&0\\0&0&1&0\\0&0&-\frac{1}{n}&1}$
jeweils Translation um $n$ in $z$-Richtung,
Skalierung um $n$ in $z$-Richtung und
perspektivische Transformation auf die Ebene $z = 0$.
Insgesamt ergibt sich\\
$M_\text{proj}^\text{persp}
= \smallpmatrix{\frac{2n}{r-l}&0&\frac{r+l}{r-l}&0\\0&\frac{2n}{t-b}&\frac{t+b}{t-b}&0\\
0&0&-\frac{f+n}{f-n}&-\frac{2fn}{f-n}\\0&0&-1&0}
= \smallpmatrix{\frac{\cot(\text{fovY}/2)}{\text{aspect}}&0&0&0\\
0&\cot(\text{fovY}/2)&0&0\\0&0&-\frac{f+n}{f-n}&-\frac{2fn}{f-n}\\0&0&-1&0}$.\\
Die letzte Formel gilt für den Fall, dass $l = -r$, $b = -t$, $r = \text{aspect} \cdot t$
und $t = n \cdot \tan(\text{fovY}/2)$ gelten.
Dabei ist $\text{aspect}$ das Seitenverhältnis (Breite durch Höhe) und
$\text{fovY}$ der vertikale Öffnungswinkel der Kamera.

\linie

\textbf{normierte Gerätekoordinaten}:
Die \begriff{normierten Gerätekoordinaten} $p_\text{ndc}$ erhält aus den Clipkoordinaten
durch komponentenweise Division durch die homogene Koordinate.

\textbf{Fensterkoordinaten}:
Die \begriff{Fensterkoordinaten} $p_\text{win}$ geben die Position bzgl. des Fensters an,
das auf dem Bildschirm angezeigt wird.
Der Ursprung befindet sich unten links und es wird nach rechts bzw. oben positiv gezählt.
Wenn sich die linke untere Ecke des Fensters sich im Punkt $(o_x, o_y)$ befindet und die
Fenstergröße $(\text{width}, \text{height})$ ist, so erhält man die Fenster- aus den
NDC-Koordinaten durch
$\smallpmatrix{x_\text{win}\\y_\text{win}\\z_\text{win}}
= \smallpmatrix{\frac{\text{width}}{2} (x_\text{ndc} + 1) + o_x\\
\frac{\text{height}}{2} (y_\text{ndc} + 1) + o_y\\
\frac{f-n}{2} (z_\text{ndc} + 1) + n}$,
wobei $(x_\text{ndc}, y_\text{ndc}, z_\text{ndc}) \in [-1, 1]^3$.

\vspace{3mm}
\linie

\textbf{Tiefengenauigkeit}:
Weil beim Übergang von normierten Gerätekoordinaten zu Clipkoordinaten durch die homogene
Komponente geteilt wird, sollte die Dif"|ferenz zwischen $n$ und $f$ nicht zu groß gewählt werden
(für $f = 1000$ ist $n = 500$ besser als $n = 10$),
sonst wirken sich Rechenungenauigkeiten bei entfernten Objekten zu stark aus.

\textbf{Clipping}:
In 2D ist Clipping mit einem Rechteck einfach durchführbar.
In 3D geht das ebenfalls noch einfach bei der orthografischen Projektion.
Bei der perspektivischen Projektion gibt es Probleme, wenn man erst nach der Projektion clippt,
weil sonst Objekte, die hinter der Kamera liegen, nach vorne auf die Sichtfläche gelangen können.

\pagebreak
