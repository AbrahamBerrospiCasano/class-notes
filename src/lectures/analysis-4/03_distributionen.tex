\chapter{%
    Distributionen%
}

\section{%
    Der Raum der Testfunktionen \texorpdfstring{$\D$}{D}%
}

Im Folgenden wird die Menge $\C_0^\infty(\real^d, \complex)$
aller glatten Funktionen mit kompaktem Träger betrachtet.
Eine Funktion $\varphi\colon \real^d \rightarrow \complex$ ist in
$\C_0^\infty(\real^d, \complex)$ genau dann, wenn
sie beliebig oft (stetig) partiell dif"|ferenzierbar ist und
es ein $R = R(\varphi) < \infty$ gibt mit $\varphi(x) \equiv 0$ für alle
$|x| \ge R$.

\textbf{Konvergenz auf $\D$}:
Für eine Folge von Funktionen $\{\varphi_n\}_{n \in \natural}$ und
$\varphi$ in $\C_0^\infty(\real^d, \complex)$
\emph{konvergiert $\varphi_n$ auf $\D$ gegen $\varphi$}
($\varphi_n \xrightarrow{\D} \varphi$), falls
$\exists_{R < \infty} \forall_{n \in \natural} \forall_{|x| \ge R}\;
\varphi_n(x) \equiv 0$ und
$\forall_{\alpha \in \natural_0^d}\;
\partial^\alpha \varphi_n \xrightarrow[\text{glm.}]{\real^d}
\partial^\alpha \varphi$.

\textbf{Raum der Testfunktionen $\D = \D(\real^d)$}:
Der \emph{Raum der Testfunktionen}
$\D = \D(\real^d)$ ist der topologische Vektorraum gebildet durch die Menge
$\C_0^\infty(\real^d, \complex)$ und obiger Konvergenz.

Der Raum ist nicht metrisierbar, d.\,h. es gibt keine Metrik, die obigen
Konvergenzbegriff induziert.

\linie

\textbf{Träger}:
Für $\varphi \in \D(\real^d)$ ist
$\supp \varphi := \overline{\{x \in \real^d \;|\; \varphi(x) \not= 0\}}$
der \emph{Träger} von $\varphi$.

\textbf{Lemma}:
Es gilt $\supp \partial^\alpha \varphi \subset \supp \varphi$ für alle
$\varphi \in \D(\real^d)$ und $\alpha \in \natural_0^d$.

\textbf{Raum der Testfunktionen $\D(G)$}:
Sei $G \subset \real^d$ of"|fen.\\
Dann ist $\D(G) := \{\varphi \in \D(\real^d) \;|\; \supp \varphi \subset G\}$
der \emph{Raum der Testfunktionen auf $G$}.

\linie

\textbf{Lemma ("`Vollständigkeit"' von $\D$)}:\\
Für $n \in \natural$ seien $\varphi_n \in \D$ gegeben mit
$\exists_{R < \infty} \forall_{n \in \natural}\;
\supp \varphi_n \subset U_R(0)$.\\
Außerdem gelte für alle $\alpha \in \natural_0^d$, dass
$\partial^\alpha \varphi_n
\xrightarrow[\text{glm.}]{\real^d} \psi_\alpha$ mit
$\psi_\alpha$ stetig.\\
Dann gibt es ein $\varphi \in \D$ mit $\varphi_n \xrightarrow{\D} \varphi$.

\linie

\textbf{stetige Abbildung auf $\D$}:
Sei $T\colon \D \rightarrow \D$ eine Abbildung.\\
$T$ heißt stetig, falls aus $\varphi_n \xrightarrow{\D} \varphi$ stets
$T \varphi_n \xrightarrow{\D} T\varphi$ folgt.

\emph{Beispiel}:
Für $T = \partial^\beta$ sei
$T \varphi_n = \partial^\beta \varphi_n =: \psi_n^{(\beta)}$ und
$\varphi_n \xrightarrow{\D} \varphi$.
Aufgrund des Träger-Lemmas gilt
$\supp \psi_n^{(\beta)} = \supp \partial^\beta \varphi_n \subset
\supp \varphi_n \subset \overline{U_R(0)}$,
d.\,h. die erste Bedingung für
$\psi_n^{(\beta)} \xrightarrow{\D} \psi^{(\beta)} := \partial^\beta \varphi$
ist überprüft.
Außerdem gilt $\partial^\alpha \psi_n^{(\beta)} =
\partial^\alpha (\partial^\beta \varphi_n) =
\partial^{\alpha + \beta} \varphi_n \xrightarrow{\text{glm.}}
\partial^{\alpha + \beta} \varphi = \partial^\alpha \psi^{(\beta)}$,
also ist auch die zweite Bedingung erfüllt und es gilt
$\psi_n^{(\beta)} \xrightarrow{\D} \psi^{(\beta)}$.
Daher ist $T = \partial^\beta$ eine stetige Abbildung.

\emph{Beispiel}:
Für $\alpha \in \C^\infty(\real^d, \complex)$ ist
$T\colon \D \rightarrow \D$, $T\varphi = \alpha \cdot \varphi$ stetig
(Leibnizregel).

\emph{Beispiel}:
Für eine $d \times d$-Matrix $A \in \real^{d \times d}$ mit $\det A \not= 0$
ist $A\colon \real^d \rightarrow \real^d$ eine bijektive Abbildung.
Dies definiert eine stetige Abbildung $T\colon \D \rightarrow \D$,
$(T\varphi)(x) = \varphi(Ax + b)$.

\pagebreak

\section{%
    Distributionen über \texorpdfstring{$\D$}{D}%
}

\textbf{Distributionen über $\D$}:\\
Eine \emph{Distribution} $f$ über $\D$ ist ein lineares stetiges Funktional
$f\colon \D \rightarrow \complex$.\\
Man schreibt $\varphi \mapsto f[\varphi] = (f, \varphi)$ für $\varphi \in \D$.

Es müssen also zwei Bedingungen erfüllt werden:
Zum einen muss für $\varphi, \psi \in \D$ und $\lambda, \mu \in \complex$
gelten, dass
$f[\lambda \varphi + \mu \psi] = \lambda f[\varphi] + \mu f[\psi]$,
und es muss aus $\varphi_n \xrightarrow{\D} \varphi$ stets
$f[\varphi_n] \to f[\varphi]$ folgen.\\
Ist Linearität gezeigt, genügt es, die Stetigkeit für $\varphi \equiv 0$
zu überprüfen.

Zwei Distributionen sind gleich, falls sie auf allen Testfunktionen angewendet
gleich sind, d.\,h. $\forall_{\varphi \in \D}\; (f, \varphi) = (g, \varphi)$.

\textbf{Konvergenz auf $\D'$}:
Für eine Folge von Distributionen $\{f_n\}_{n \in \natural}$ und
eine Distribution $f$ \emph{konvergiert $f_n$ auf $\D'$ gegen $f$}
($f_n \xrightarrow{\D'} f$), falls
$\forall_{\varphi \in \D}\; (f_n, \varphi) \to (f, \varphi)$.

\textbf{Raum der Distributionen $\D'$}:
Der \emph{Raum der Distributionen $\D'$} ist der topologische Vektorraum
gebildet durch die Menge der Distributionen über $\D$ und obiger Konvergenz.

Da $\D'$ ein Vektorraum ist, gilt
$(\alpha f + \beta g)[\varphi] := \alpha f[\varphi] + \beta g[\varphi]$
für $f, g \in \D'$, $\alpha, \beta \in \complex$ und $\varphi \in \D$\\
(stetig, da $f, g$ stetig sind).

Für $f_n \xrightarrow{\D'} f$, $g_n \xrightarrow{\D'} g$ und
$\alpha, \beta \in \complex$ gilt
$\alpha f_n + \beta g_n \xrightarrow{\D'} \alpha f + \beta g$.

\linie

\textbf{Lemma ("`Vollständigkeit"' von $\D'$)}:
Für $n \in \natural$ seien $f_n \in \D'$ gegeben mit\\
$\forall_{\varphi \in \D} \exists_{\ell_\varphi \in \complex}\;
\ell_\varphi = \lim_{n \to \infty} (f_n, \varphi)$.\\
Dann gibt es ein $\ell \in \D'$ mit $\ell_\varphi = \ell[\varphi]$ für
alle $\varphi \in \D$,\\
d.\,h. es gilt $\ell \in \D'$ und $f_n \xrightarrow{\D'} \ell$ mit
$(\ell, \varphi) := \ell_\varphi$.

\linie

\textbf{Gleichheit von Distributionen auf $G$}:
Seien $f, g \in \D'$ und $G \subset \real^d$ of"|fen.\\
Dann sei $f|_G \equiv 0$, falls
$\forall_{\varphi \in \D,\; \supp \varphi \subset G}\; (f, \varphi) = 0$.\\
Außerdem sei $f|_G \equiv g|_G$, falls $(f - g)|_G \equiv 0$.

\textbf{Satz}:
Seien $f \in \D'$ und $G \subset \real^d$ of"|fen mit
$\forall_{x \in G} \exists_{V_x \subset G \text{ of"|fen},\; x \in V_x}\;
f|_{V_x} \equiv 0$
($f$ ist lokal $0$).\\
Dann gilt $f|_G \equiv 0$
($f$ ist global $0$).

\textbf{Träger einer Distribution}:
Seien $f \in \D'$ und
$O_f := \bigcup_{G \text{ of"|fen},\; f|_G \equiv 0} G$.\\
Dann ist $\supp f := \real^d \setminus O_f$ der \emph{Träger} von $f$
(abgeschlossen, da $O_f$ of"|fen).

Nach dem Satz gilt $f|_{O_f} \equiv 0$.\\
Es ist $x \in O_f$ genau dann, wenn es ein
$V_x \subset \real^d$ of"|fen gibt mit $x \in V_x$ und $f|_{V_x} \equiv 0$.\\
Daher ist $x \in \supp f$ genau dann, wenn es kein
$V_x \subset \real^d$ of"|fen gibt mit $x \in V_x$ und $f|_{V_x} \equiv 0$.

\pagebreak

\section{%
    Reguläre und singuläre Distributionen%
}

\textbf{Raum der lokal integrierbaren Funktionen
$L^1_\loc = L^1_\loc(\real^d)$}:\\
Der \emph{Raum der lokal integrierbaren Funktionen $L^1_\loc(\real^d)$} ist
der Raum aller Funktionen\\
$f\colon \real^d \rightarrow \complex$ mit
$f|_K \in L^1(K, dx)$ für alle $K \subset \real^d$ kompakt.

\textbf{von $f$ erzeugte reguläre Distribution}:
Sei $f \in L^1_\loc$.
Dann ist die \emph{von $f$ erzeugte reguläre Distribution} $\ell_f$ definiert
durch $\ell_f(\varphi) = f[\varphi] := \int_{\real^d} f(x) \varphi(x) \dx$
für alle $\varphi \in \D$.

Das Integral ist wohldefiniert, da $\varphi$ als Testfunktion
kompakt getragen ist.\\
$\ell_f$ ist in der Tat eine Distribution:
$\ell_f$ ist of"|fensichtlich linear, da das Integral ebenfalls linear ist.
$\ell_f$ ist außerdem in $\varphi \equiv 0$ stetig, denn:
Seien $\varphi_n \in \D$ mit $\varphi_n \xrightarrow{\D} 0$,
dann gilt $\supp \varphi_n \subset \overline{U_R(0)}$ und
$\varphi_n \xrightarrow{\text{glm.}} 0$.
Daraus folgt $|\ell_f(\varphi_n)| =
\left|\int_{\real^d} f(x) \varphi_n(x) \dx\right| \le
\int_{|x| \le R} |f(x)| |\varphi_n(x)| \dx$\\
$\le \sup_{|x| \le R} |\varphi_n(x)| \cdot \int_{|x| \le R} |f(x)| \dx$.
Der erste Faktor geht gegen $0$, da $\varphi_n \xrightarrow{\text{glm.}} 0$.
Der zweite Faktor ist endlich, da $f$ lokal integrierbar ist.
Somit gilt $\ell_f(\varphi_n) \to 0$.
Also ist $\ell_f \in \D'$.

\linie

\textbf{reguläre und singuläre Distributionen}:\\
Eine Distribution $h \in \D'$ heißt \emph{regulär}, falls
es ein $f \in L^1_\loc$ gibt mit $\ell_f = h$.\\
Andernfalls heißt $h$ \emph{singulär}.

Für $f, g \in L^1_\loc$ mit $f \not= g$ in $L^1_\loc$ gibt es ein
$\varphi \in \D$ mit $(f, \varphi) \not= (g, \varphi)$, d.\,h. es gilt
$\ell_f \not= \ell_g$.

\emph{Beispiel}:
Die Distribution $\delta \in \D'$ mit
$\delta[\varphi] = (\delta, \varphi) := \varphi(0)$ ist die sogenannte\\
\textbf{Delta-Distribution}.
Sie ist eine singuläre Distribution.

\emph{Beispiel}:
Für eine Mannigfaltigkeit $S \subset \real^d$ mit Volumenform ist
$\delta_S[\delta] := \int_S \varphi(x) dS$ eine\\
Distribution.

\emph{Beispiel}:
Für $d = 1$ ist $f(x) = \frac{1}{x}$ keine lokal-integrierbare Funktion
(nicht integrierbar auf jedem Intervall, das die $0$ enthält).
Man versucht dieses Problem zu umgehen, in dem man
$(\frac{1}{x \pm \i \cdot 0}, \varphi) :=
\lim_{\varepsilon \to 0} \left(\int_\real
\frac{\varphi(x)}{x \pm \i \varepsilon} \dx\right)$ definiert.
Für $\varepsilon > 0$ fest ist der Ausdruck in Klammern gleich
$(f_\varepsilon, \varphi)$ mit einer regulären Distribution $f_\varepsilon$.

\emph{Beispiel}:
$(P\frac{1}{x}, \varphi) :=
\vp\int_{-\infty}^{+\infty} \frac{\varphi(x)}{x} \dx =
\lim_{\varepsilon \to 0 + 0} \left(\int_{-\infty}^{-\varepsilon} +
\int_\varepsilon^{+\infty}\right) \frac{\varphi(x)}{x} \dx$
ist eine Distribution.

\pagebreak

\section{%
    Koordinatentransformation%
}

Seien $\pi\colon \real^d \rightarrow \real^d$ mit $x = \pi y = Ay + b$
und $A \in \real^{d \times d}$, $\det A \not= 0$ und
$b \in \real^d$.

\emph{Motivation}:
Für $f \in L^1_\loc(\real^d)$ gilt $(f \circ \pi, \varphi) =
\int_{\real^d} (f \circ \pi)(y) \varphi(y) \dy =
\int_{\real^d} f(x) \varphi(\pi^{-1} x) \left|\frac{Dy}{Dx}\right| \dx$\\
$= \frac{1}{|\det A|} \int_{\real^d} f(x) (\varphi \circ \pi^{-1})(x) \dx =
\frac{1}{|\det A|} (f, \varphi \circ \pi^{-1})$.
Dies verwendet man als Definition.

\textbf{Koordinatentransformation einer Distribution}:
Sei $f \in \D'$.\\
Dann ist die Distribution $f \circ \pi$ definiert durch
$(f \circ \pi, \varphi) := \frac{1}{|\det A|} (f, \varphi \circ \pi^{-1})$
für alle $\varphi \in \D$.

\textbf{Korrektheit}:
Für $\varphi \in \D$ ist $\varphi \circ \pi^{-1} \in \D$, d.\,h.
$(f, \varphi \circ \pi^{-1})$ ist wohldefiniert.\\
$f \circ \pi$ ist linear (klar).
Für $\varphi_n \xrightarrow{\D} 0$ gilt aufgrund
$\cdot \circ \pi^{-1}\colon \D \rightarrow \D$ stetig, dass
$\varphi_n \circ \pi^{-1} \xrightarrow{\D} 0$,
also gilt für $f \in \D'$, dass $(f, \varphi_n \circ \pi^{-1}) \to 0$
(da $f$ stetig ist).

\emph{Beispiel}:
Für $A = E$ gleich der Einheitsmatrix gilt $\det A = 1$ und
$x = \pi y = y + b$, also $y = \pi^{-1} x = x - b$.
Somit ist $(f \circ \pi, \varphi) = (f, \varphi(x - b))$\\
(im regulären Fall wäre das z.\,B. $\int_{\real^d} f(x) \varphi(x - b) \dx$).\\
Insbesondere gilt für die Delta-Distribution $\delta$, dass
$(\delta \circ \pi, \varphi) = \varphi(-b)$.

\emph{Beispiel}:
Für $A = c \cdot E$ mit $c > 0$ und $b = 0$ gilt\\
$(\delta \circ \pi, \varphi) =
\frac{1}{|\det A|} (\delta, \varphi \circ \pi^{-1}) =
\frac{1}{|\det A|} (\varphi \circ \pi^{-1})(0) =
c^{-d} \varphi(0)$.

\linie

\emph{Motivation}:
Für $f \in L^1_\loc(\real^d)$ und $\alpha \in \C^\infty(\real^d, \complex)$
ist $\alpha f \in L^1_\loc(\real^d)$ und es gilt\\
$(\alpha f, \varphi) = \int_{\real^d} \alpha(x) f(x) \varphi(x) \dx =
(f, \alpha \varphi)$, da $\alpha \varphi \in \D$.

\textbf{Multiplikation einer Distribution mit einer glatten Funktion}:
Seien $f \in \D'$ und $\alpha \in \C^\infty$.\\
Dann ist die Distribution $\alpha f$ definiert durch
$(\alpha f, \varphi) := (f, \alpha \varphi)$ für alle $\varphi \in \D$.

\textbf{Korrektheit}:
Für $\varphi \in \D$ ist $\alpha \varphi \in \D$, d.\,h.
$(f, \alpha \varphi)$ ist wohldefiniert.\\
$\alpha f$ ist linear (klar).
Für $\varphi_n \xrightarrow{\D} 0$ gilt $\alpha \varphi_n \xrightarrow{\D} 0$,
also $(f, \alpha \varphi_n) \to 0$
(da die Multiplikation $T_\alpha\colon \D \rightarrow \D$,
$\varphi \mapsto \alpha \varphi$ eine stetige Abbildung ist).

\emph{Beispiel}:
Für $\alpha \in \C^\infty(\real^d, \complex)$ und der Delta-Distribution
$\delta \in \D'$ gilt $(\alpha \cdot \delta, \varphi) =
(\delta, \alpha \cdot \varphi) = \alpha(0) \varphi(0)$.
Für $d = 1$ ist z.\,B. $x \cdot \delta = 0$ die Nulldistribution
($\alpha(x) = x$).

\emph{Beispiel}:
Für $d = 1$ soll $x \cdot P \frac{1}{x}$ betrachtet werden.
Es ist\\
$(x \cdot P \frac{1}{x}, \varphi) = (P \frac{1}{x}, x \cdot \varphi) =
\vp\int_{-\infty}^{+\infty} \frac{1}{x} \cdot x \varphi(x) \dx =
\int_{-\infty}^{+\infty} 1 \cdot \varphi(x) = (1, \varphi)$, also
$x \cdot P \frac{1}{x} = 1$ in $\D'$.

\emph{Vorsicht}:
Es gibt keine assoziative und kommutative Multiplikation auf den
Distributionen, denn sonst wäre
$0 = 0 \cdot P \frac{1}{x} = (x \cdot \delta) \cdot P \frac{1}{x} =
\delta \cdot (x \cdot P \frac{1}{x}) = \delta \cdot 1 = \delta$.

\pagebreak

\section{%
    Dif"|ferentiation von Distributionen%
}

\emph{Motivation}:
Für $f \in \C^1(\real^d, \complex)$ gilt
$f'_{x_j} \in \C(\real^d, \complex)$ für $j = 1, \dotsc, d$.
Betrachtet man die erzeugte reguläre Distribution, so ergibt sich
(da $\varphi$ kompakt getragen ist)\\
$(f'_{x_j}, \varphi) = \int_{\real^d} f'_{x_j}(x) \varphi(x) \dx =
\int_{\real^{d-1}} \left(\left.f(x) \varphi(x)\right|_{x_j=-\infty}^{+\infty} -
\int_{-\infty}^{+\infty} f(x) \varphi'_{x_j}(x) \dx_j\right) \dx' =
-(f, \varphi'_{x_j})$, wobei bei $x'$ die $j$-te Komponente $x_j$ fehlt.
Allgemeiner ist $(\partial^\alpha f, \varphi) =
(-1)^{|\alpha|} (f, \partial^\alpha \varphi)$.

\textbf{Ableitung einer Distribution}:
Seien $f \in \D'$ und $\alpha \in \natural_0^d$.\\
Dann ist die Distribution $\partial^\alpha f$ definiert durch
$(\partial^\alpha f, \varphi) := (-1)^{|\alpha|} (f, \partial^\alpha \varphi)$
für alle $\varphi \in \D$.

\textbf{Korrektheit}:
Für $\varphi \in \D$ ist $\partial^\alpha \varphi \in \D$, d.\,h.
$(f, \partial^\alpha \varphi)$ ist wohldefiniert.\\
$\partial^\alpha f$ ist linear (klar).
Für $\varphi_n \xrightarrow{\D} 0$ gilt
$\partial^\alpha \varphi \xrightarrow{\D} 0$, also
$(f, \partial^\alpha \varphi_n) \to 0$, da
$\partial^\alpha\colon \D \rightarrow \D$ stetig ist.

\emph{Beispiel}:
Die Ableitung der Delta-Distribution $\delta \in \D$ für $d = 1$ ist\\
$(\delta', \varphi) = (-1) \cdot (\delta, \varphi') = -\varphi'(0)$.
Man kann sich eine analoge Formel für $\partial^\alpha \delta$ überlegen.
Dabei gilt, dass $\supp \partial^\alpha \delta = \{0\}$.
In der Tat ist jede Distribution mit nur einem Punkt als Träger eine
Linearkombination von der Delta-Distribution und ihren Ableitungen.

\linie

\textbf{Rechenregeln}:
Für $f, g \in \D'$ gilt
$\partial^\alpha (f + g) = \partial^\alpha f + \partial^\alpha g$.\\
Es gilt $\partial^\alpha (\partial^\beta f) = \partial^{\alpha + \beta} f$ und
$\partial^\alpha (cf) = c (\partial^\alpha f)$ für $c \in \complex$.

\textbf{Produktregel}:
Für $f \in \D'$ und $\alpha \in \C^\infty$ gilt
$\frac{\partial}{\partial x_j} (\alpha f) =
\left(\frac{\partial \alpha}{\partial x_j}\right) f +
\alpha \left(\frac{\partial f}{\partial x_j}\right)$.

\textbf{Träger von Ableitungen}:
Es ist $\supp f = \real^d \setminus O_f$ mit
$O_f = \bigcup_{G \text{ of"|fen},\; f|_G \equiv 0} G$.\\
Dabei bedeutet $f|_G \equiv 0$, dass
$\forall_{\varphi \in \D(G)}\; (f, \varphi) = 0$.
Daraus folgt $(f, \partial^\alpha \varphi) = 0$ für alle $\varphi \in \D(G)$
und daher $\partial^\alpha f|_G \equiv 0$.
Also gilt $O_f \subset O_{\partial^\alpha f}$ bzw.
$\supp \partial^\alpha f \subset \supp f$.

\textbf{Satz}:
Die Abbildung $\partial^\alpha\colon \D' \rightarrow \D'$
ist linear und stetig.

\textbf{Folgerung}:
Für $f_n \in L^1_\loc$ mit $f_n \xrightarrow{L^1_\loc} f$ gilt
$(f_n, \varphi) = \int_{\real^d} f_n(x) \varphi(x) \dx \to
\int_{\real^d} f(x) \varphi(x) \dx$, d.\,h.
$f_n \xrightarrow{\D'} f$ und
$\partial^\alpha f_n \xrightarrow{\D'} \partial^\alpha f$.

\linie

\textbf{Reihen von Distributionen}:
Seien $f_k \in \D'$ für $k \in \natural$ und
$S_n := \sum_{k=1}^n f_k \in \D'$ für $n \in \natural$.\\
Dann ist $\sum_{k=1}^\infty f_k \overset{\D'}{:=} S$, falls
$S_n \xrightarrow{\D'} S$.

\textbf{Folgerung}:
In diesem Fall gilt auch
$\partial^\alpha S = \sum_{k=1}^\infty \partial^\alpha f_k$.

\textbf{Satz}:
Seien $c_k \in \complex$ für $k \in \integer$ mit
$|c_k| \le a |k|^m + b$ für ein $m \in \natural$ und $a, b > 0$.\\
Dann konvergiert $S = \sum_{k=-\infty}^{+\infty} c_k e^{\i kx} =
\lim_{N \to \infty} \sum_{k=-N}^N c_k e^{\i kx}$ in $\D'$.

\section{%
    Stammfunktion einer Distribution%
}

\textbf{Stammfunktion einer Distribution}:
Sei $f \in \D'$.\\
Dann heißt eine Distribution $F = f^{-1} \in \D'$
\emph{Stammfunktion} von $f$, falls $F' = f$, d.\,h.\\
$(F, \varphi') = -(f, \varphi)$ für alle $\varphi \in \D$.

Beachte:
$(F, \varphi') = -(f, \varphi)$ ist nur auf $\psi = \varphi'$ mit
$\varphi \in \D$ gegeben.\\
Nicht alle $\widetilde{\varphi} \in \D$ sind Ableitungen von Stammfunktionen.

\textbf{Satz}:
Für jede Distribution $f \in \D'$ existiert eine Stammfunktion $F \in \D'$.
Diese ist bis auf eine additive Konstante eindeutig.

\textbf{Folgerung}:
Falls $f \in \D'$ mit $f' = 0$ gilt, so ist $f \equiv \const$.

\pagebreak

\section{%
    Wichtige Beispiele%
}

\emph{Beispiel}:
\textbf{Ableitung von regulären Distributionen mit Sprungstellen}\\
Sei $f\colon \real \rightarrow \real$ (oder $\complex$) eine Funktion
mit Sprungstelle $x_0 \in \real$, d.\,h.
$f|_{\left]-\infty, x_0\right[} \in \C^1$ und
$f|_{\left]x_0, +\infty\right[} \in \C^1$.
Dabei sei $[f]_{x_0} := f(x_0 + 0) - f(x_0 - 0)$ die Höhe des Sprungs und\\
$\{f'\}(x) := f'(x)$ für $x \not= x_0$ die klassische Ableitung.
Es gilt $f \in L^1_\loc(\real)$, d.\,h. $f \in \D'$ ist eine reguläre
Distribution.
Was ist nun die distributionelle Ableitung $f'$?\\
Für $\varphi \in \D(\real)$ gilt
$(f', \varphi) = -(f, \varphi') = -\int_\real f(x)\varphi'(x)\dx$\\
$= -\int_{-\infty}^{x_0} f(x)\varphi'(x)\dx -
\int_{x_0}^{+\infty} f(x)\varphi'(x)\dx$\\
$= -f(x)\varphi(x)|_{-\infty}^{x_0 - 0} - f(x)\varphi(x)|_{x_0 + 0}^{+\infty} +
\left(\int_{-\infty}^{x_0} + \int_{x_0}^{+\infty}\right)
\{f'\}(x)\varphi(x)\dx$\\
$= -f(x_0 - 0)\varphi(x_0 - 0) + f(x_0 + 0)\varphi(x_0 + 0) +
(\{f'\}, \varphi) =
[f]_{x_0} \varphi(x_0) + (\{f'\}, \varphi)$\\
$= ([f]_{x_0} \delta(x - x_0) + \{f'\}, \varphi)$,
d.\,h. es gilt $f' = [f]_{x_0} \delta(x - x_0) + \{f'\}$.\\
Im Spezialfall $f(x) = \theta(x) :=$ \matrixsize{$\begin{cases}
0 & x \le 0 \\
1 & x > 0
\end{cases}$} (\emph{\name{Heaviside}-Funktion})
gilt $\theta' = \delta$.

\linie

\emph{Beispiel}:
\textbf{Distributionen mit Träger in einem Punkt}\\
Gesucht ist $u \in \D'(\real)$ mit $x^m u = 0$ für ein $m \in \natural$.\\
Man sieht schnell, dass dafür notwendigerweise $\supp u = \{0\}$ gelten muss.\\
Eine Lösung ist eine Linearkombination $u = \sum_{k=0}^{m-1} c_k \delta^{(k)}$
von Ableitungen der Delta-Distr.\\
Wie die Probe
$(x^m c_k \delta^{(k)}, \varphi) = c_k (\delta^{(k)}, x^m \varphi) =
c_k (-1)^k (\delta, \frac{d^k}{dx^k} (x^m \varphi)) =
c_k (-1)^k \frac{d^k}{dx^k} (x^m \varphi)|_{x=0} = 0$ für
$\varphi \in \D(\real)$ zeigt, ist dies tatsächlich eine Lösung.
Man kann zeigen, dass das sogar die allgemeine Lösung ist
(d.\,h. jede Lösung ist von dieser Form).

\linie

\emph{Beispiel}:
\textbf{Lösung von ODE}\\
Sei $L = \frac{d^m}{dt^m} + a_1(t) \frac{d^{m-1}}{dt^{m-1}} + \dotsb +
a_{m-1}(t) \frac{d}{dt} + a_m(t)$ ein Dif"|ferentialausdruck mit
$a_j \in \C^\infty(\real)$ und $|a_j| \le C$.
Man betrachtet das Cauchy-Problem $Lz(t) = 0$ für $t > 0$ mit\\
$z(0) = z'(0) = \dotsb = z^{(m-2)(}(0) = 0$ und $z^{m-1}(0) = 1$.\\
Mit der Heaviside-Funktion kann die Lösung erweitert werden zu
$\varepsilon(t) = \theta(t) z(t)$ für $t \in \real$.\\
Dabei gilt $\varepsilon^{(k)}(0) = 0$ für $k = 0, \dotsc, m - 2$ und
$\varepsilon^{(m-1)}(0 - 0) = 0$ sowie $\varepsilon^{(m-1)}(0 + 0) = 1$.\\
Somit ist $\varepsilon^{(m)}(t) = \theta(t) z^{(m)}(t) + \delta(t)$
nach obiger Formel.\\
Wegen $\varepsilon^{(k)}(t) = \theta(t) z^{(k)}(t)$ für $k = 0, \dotsc, m - 2$
gilt $L\varepsilon(t) = \theta(t) Lz(t) + \delta(t) = \delta(t)$
($Lz(t) = 0$).\\
Somit löst $\varepsilon(t) = \theta(t) z(t)$ die Gleichung
$L\varepsilon(t) = \delta(t)$.
Man spricht von einer \emph{Fundamental"|lösung}.

\linie

\emph{Beispiel}:
\textbf{Fundamental"|lösung für $\Delta$ und $d = 2$}\\
Es soll verifiziert werden, dass $\varepsilon_2(x) = \frac{1}{2\pi} \ln |x|$
eine Fundamental"|lösung für den Laplace-Operator
$\Delta = \frac{\partial^2}{\partial x_1^2} +
\frac{\partial^2}{\partial x_2^2}$ in zwei Dimensionen ist.
Dabei seien $x = (x_1, x_2) \in \real^2$ kartesische Koordinaten und
$|x| = \sqrt{x_1^2 + x_2^2}$.\\
Zunächst gilt $\ln |x| \in L^1_\loc(\real^2)$ und mit
$\chi_\varepsilon(x) :=$ \matrixsize{$\begin{cases}
1 & |x| \ge \varepsilon \\
0 & |x| < \varepsilon\end{cases}$} ist
$\chi_\varepsilon(x)\ln |x| \xrightarrow{L^1_\loc} \ln |x|$ für
$\varepsilon \to 0$.
Daraus folgt $\chi_\varepsilon(x) \ln |x| \xrightarrow{\D'} \ln |x|$ und
$\Delta (\chi_\varepsilon(x) \ln |x|) \xrightarrow{\D'} \Delta \ln |x|$
für $\varepsilon \to 0$, da $\Delta\colon \D' \rightarrow \D'$ stetig ist.\\
Man geht nun zu Polarkoordinaten $(r, \theta)$ über,
der entsprechend transformierte Ausdruck für den Laplace-Operator ist
$\Delta = \frac{\partial^2}{\partial x_1^2} +
\frac{\partial^2}{\partial x_2^2} =
\frac{1}{r} \frac{\partial}{\partial r} r \frac{\partial}{\partial r} +
\frac{1}{r^2} \frac{\partial^2}{\partial \theta^2}$.\\
Daraus folgt dann $(\Delta (\chi_\varepsilon(x)\ln |x|), \varphi) =
(\chi_\varepsilon(x)\ln |x|, \Delta \varphi) =
\int_0^{2\pi} \int_\varepsilon^R
\ln r \cdot (\Delta \varphi(r, \theta)) \cdot r\dr\dtheta$\\
$= \int_0^{2\pi} \int_\varepsilon^R
\ln r \cdot \left(\frac{1}{r} \frac{\partial}{\partial r} r
\frac{\partial}{\partial r} \varphi +
\frac{1}{r^2} \frac{\partial^2}{\partial \theta^2} \varphi\right) \cdot
r\dr\dtheta$.\\
Dabei ist $\int_0^{2\pi} \int_\varepsilon^R
\ln r \cdot
\left(\frac{1}{r^2} \frac{\partial^2}{\partial \theta^2} \varphi\right) \cdot
r\dr\dtheta =
\int_\varepsilon^R \frac{\ln r}{r} \cdot
\left(\int_0^{2\pi} \frac{\partial^2 \varphi}{\partial \theta^2}d\theta\right)
\dr = 0$,
da $\varphi$ in $\theta$ $2\pi$-periodisch ist
(der Ausdruck in Klammern ist $0$).

\pagebreak

Also ist
$\int_0^{2\pi} \int_\varepsilon^R
\ln r \cdot \left(\frac{1}{r} \frac{\partial}{\partial r} r
\frac{\partial}{\partial r} \varphi +
\frac{1}{r^2} \frac{\partial^2}{\partial \theta^2} \varphi\right) \cdot
r\dr\dtheta =
\int_0^{2\pi} \int_\varepsilon^R
\ln r \cdot \left(\frac{\partial}{\partial r} r
\frac{\partial}{\partial r} \varphi\right) \dr\dtheta$\\
$= \int_0^{2\pi} \left(\left.\ln r \cdot r
\frac{\partial}{\partial r} \varphi\right|_\varepsilon^R -
\int_\varepsilon^R \left(\frac{1}{r} \cdot
r \frac{\partial}{\partial r} \varphi\right)d\theta\right)\dr =
o(\varepsilon) - \int_0^{2\pi}
\left(\int_\varepsilon^R \frac{\partial}{\partial r}
\varphi \dr\right)d\theta$\\
$= o(\varepsilon) - \int_0^{2\pi}
\left(\varphi(R, \theta) - \varphi(\varepsilon, \theta)\right) d\theta$.
Dabei verschwindet $\varphi(R, \theta)$ (kompakter Träger) und
$\varphi(\varepsilon, \theta)$ geht für $\varepsilon \to 0$ gleichmäßig gegen
$\varphi(0)$.
Somit gilt
$\lim_{\varepsilon \to 0}
\left(\Delta \chi_\varepsilon(x)\ln |x|, \varphi\right) =
0 + \int_0^{2\pi} \varphi(0)d\theta$\\
$= 2\pi\varphi(0) = (2\pi\delta, \varphi)$,
d.\,h. $\Delta\ln|x| = 2\pi\delta$ für $d = 2$
und $\Delta \varepsilon_2 = \delta$.

\section{%
    Tensorprodukt von Distributionen%
}

\textbf{Tensorprodukt von Funktionen}:
Seien $f \in L^1_\loc(\real^n_x)$ und $g \in L^1_\loc(\real^m_y)$.\\
Dann ist das \emph{Tensorprodukt}
$f \otimes g \in L^1_\loc(\real^{n+m}_{(x,y)})$ gegeben durch
$(f \otimes g)(x, y) := f(x) \cdot g(y)$.

Ist $\varphi(\cdot, \cdot) \in \D(\real^{n+m}_{(x,y)})$
eine Testfunktion, so sind auch
$\varphi(x_0, \cdot) \in \D(\real^m_y)$ und
$\varphi(\cdot, y_0) \in \D(\real^n_x)$ Tesfunktionen und es gilt
$(f \otimes g, \varphi) = \int_{\real^{n+m}} f(x)g(y)\varphi(x, y)d^nxd^my$\\
$= \int_{\real^n} f(x) \cdot
\left(\int_{\real^m} g(y)\varphi(x, y)d^my\right)d^nx$.
Den Ausdruck in Klammern kann man als Testfunktion $\psi \in \D(\real^n_x)$
auf"|fassen.
Daher gilt $(f \otimes g, \varphi) = (f, \psi)$ mit
$\psi(x) = (g(y), \varphi(x, y))$.

\textbf{Tensorprodukt von Distributionen}:
Seien $f \in \D'(\real^n_x)$ und $g \in \D'(\real^m_y)$.\\
Dann ist die Distribution $f \otimes g \in \D'(\real^{n+m}_{(x,y)})$
definiert durch $(f \otimes g, \varphi) := (f(x), (g(y), \varphi(x, y)))$
für alle $\varphi \in \D(\real^{n+m}_{(x,y)})$.

\textbf{Lemma}:
Seien $g \in \D'(\real^m_y)$ und $\varphi \in \D(\real^{n+m}_{(x,y)})$.
Dann gilt:
\begin{enumerate}
    \item
    $\psi(x) = (g(y), \varphi(x, y)) \in \D(\real^n_x)$

    \item
    $\partial^\alpha_x \psi(x) = (g(y), \partial^\alpha_x \varphi(x, y))$
    für alle $\alpha \in \natural_0^n$

    \item
    Gilt $\varphi_k \xrightarrow{\D(\real^{n+m}_{(x,y)})} 0$, so gilt
    $\psi_k(x) = (g(y), \varphi_k(x, y)) \xrightarrow{\D(\real^n_x)} 0$.
\end{enumerate}

\textbf{Korrektheit}:
Wohldefiniertheit folgt aus 1.
Linearität ist klar und Stetigkeit folgt aus 3.

\emph{Beispiel}:
Für $\delta_x \in \D'(\real^n_x)$ und $\delta_y \in \D'(\real^m_y)$ gilt
$(\delta_x \otimes \delta_y, \varphi) = (\delta_x, (\delta_y, \varphi(x, y))) =
(\delta_x, \varphi(x, 0))$\\
$= \varphi(0, 0) = (\delta_{(x,y)}, \varphi)$
für alle $\varphi \in \D(\real^{n+m}_{(x,y)})$.

\linie

\textbf{Eigenschaften}: $f \in \D'(\real^n_x)$, $g \in \D'(\real^m_y)$
\begin{enumerate}
    \item
    \emph{"`Kommutativität"'}:
    Für $f \in L^1_\loc(\real^n_x)$ und $g \in L^1_\loc(\real^m_y)$ gilt
    $(f(x) \cdot g(y), \varphi(x, y))$\\
    $= (f(x), (g(y), \varphi(x, y))) =
    (g(y), (f(x), \varphi(x, y))) = (g(y) \cdot f(x), \varphi(x, y))$.
    Es stellt sich heraus, dass dies auch allgemein für Distributionen
    $f, g \in \D'$ gilt, d.\,h. es gilt
    $(f \otimes g, \varphi)$\\
    $= (f(x), (g(y), \varphi(x, y))) = (g(y), (f(x), \varphi(x, y)))$.
    Es gilt allerdings \emph{nicht} $f \otimes g = g \otimes f$, da sich die
    Variablenreihenfolge in $\varphi$ nicht ändert.

    \item
    \emph{Dif"|ferenzierbarkeit}:
    Es gilt $\partial_x^\alpha (f \otimes g) = (\partial_x^\alpha f) \otimes g$
    und $\partial_y^\beta (f \otimes g) = f \otimes (\partial_y^\beta g)$,
    denn mit 2. von oben gilt
    $(\partial_x^\alpha (f \otimes g), \varphi) =
    (-1)^{|\alpha|} (f \otimes g, \partial_x^\alpha \varphi) =
    (-1)^{|\alpha|} (f(x), (g(y), \partial_x^\alpha \varphi(x, y)))$\\
    $= (-1)^{|\alpha|} (f(x), \partial_x^\alpha (g(y), \varphi(x, y))) =
    (\partial_x^\alpha f(x), (g(y), \varphi(x, y))) =
    ((\partial_x^\alpha f) \otimes g, \varphi)$.

    \item
    \emph{Stetigkeit}:
    Die Abbildung $\tau_g\colon \D'(\real^n_x) \rightarrow
    \D'(\real^{n+m}_{(x,y)})$ mit $\tau_g f = f \otimes g$ ist stetig, d.\,h.\\
    aus $f_k \xrightarrow{\D'(\real^n_x)} f$ folgt
    $f_k \otimes g \xrightarrow{\D'(\real^{n+m}_{(x,y)})} f \otimes g$.\\
    Analog ist $\tau_f\colon \D'(\real^m_y) \rightarrow
    \D'(\real^{n+m}_{(x,y)})$ mit $\tau_f g = f \otimes g$ stetig.

    \item
    \emph{Assoziativität}:
    $(f \otimes g) \otimes h = f \otimes (g \otimes h)$

    \item
    \emph{skalare Assoziativität}:
    $(\alpha f) \otimes g = \alpha (f \otimes g)$ für
    $\alpha \in \C^\infty(\real^n_x)$

    \item
    \emph{Translation}:
    $f(x + h) \cdot g(y) = (f \otimes g)(x + h, y)$
\end{enumerate}

\pagebreak

\section{%
    Faltung von Distributionen%
}

Für $f, g \in L^1(\real^n)$ gilt $f \ast g \in L^1(\real^n)$ mit
$(f \ast g)(x) = \int_{\real^n} f(\tau)g(x - \tau)\d\tau$.\\
Allerdings folgt aus $f, g \in L^1_\loc(\real^n)$ nicht, dass
$f \ast g \in L^1_\loc(\real^n)$
(ein Gegenbeispiel ist $f = g \equiv 1$).

Man kann zeigen, dass für $f \in L^1_\loc(\real^n)$ und $g \in L^1(\real^n)$
mit $\supp g = K$ kompakt gilt, dass $f \ast g \in L^1(\real^n)$ existiert.

\emph{Motivation}:
Um eine Definition für Distributionen herzuleiten, betrachtet man
$f, g \in L^1(\real^n)$.
Dann ist $(f \ast g, \varphi) = \int_{\real^n} (f \ast g)(x)\varphi(x)\dx =
\int_{\real^n} \int_{\real^n} f(\tau)g(x - \tau)\varphi(x)\d\tau \dx$\\
$= \int_{\real^n} g(y) \cdot
\left(\int_{\real^n} f(\tau)\varphi(y + \tau)\d\tau\right)\dy =
(g(y), (f(\tau), \varphi(y + \tau)))$.\\
Man würde nun gern schreiben, dass dies gleich
$(g(y) f(\tau), \varphi(y + \tau))$ ist, allerdings ist\\
$\psi(y, \tau) = \varphi(y + \tau) \notin \D(\real^{2n}_{(y,\tau)})$,
da $\psi$ i.\,A. nicht kompakt getragen ist\\
(sei z.\,B. $\varphi(0) \not= 0$, dann ist $\psi(y, -y) = \varphi(0) \not= 0$
für alle $y \in \real^n$).\\
Man muss daher vorher ein geeignetes Abschneiden durchführen, um
eine Definition der Faltung für Distributionen zu ermöglichen.

\textbf{$\eta_k \to 1$}:
Seien $\eta_k \in \D(\real^{2n})$ für $k \in \natural$.\\
Man schreibt $\eta_k \to 1$, falls
$\forall_{K \subset \real^{2n} \text{ kpkt.}} \exists_{N(K) \in \natural}
\forall_{n \ge N(K)} \forall_{(x, y) \in K}\; \eta_n(x, y) = 1$ und\\
$\forall_{\alpha \in \natural_0^{2n}} \exists_{C_\alpha < \infty}
\forall_{k \in \natural}\; |\partial^\alpha \eta_k(x, y)| \le C_\alpha$.

Mit dieser Definition ist nun
$(f \ast g, \varphi) = \int_{\real^n} g(y) \cdot
\left(\int_{\real^n} f(\tau)\varphi(y + \tau)\d\tau\right)\dy$\\
$= \lim_{k \to \infty} \left(g(y), \int_{\real^n}
f(\tau)\varphi(y + \tau)\eta_k(y, \tau)\d\tau\right) =
\lim_{k \to \infty} (g(y) f(\tau), \varphi(y + \tau)\eta_k(y, \tau))$.\\
Dabei ist
$\psi_k(y, \tau) = \varphi(y + \tau)\eta_k(y, \tau) \in \D(\real^{2n})$
für alle $k \in \natural$.

\textbf{Faltung von Distributionen}:
Seien $f, g \in \D'(\real^n)$ und $\eta_k \in \D(\real^{2n})$ mit
$\eta_k \to 1$.\\
Falls für alle $\varphi \in \D(\real^n)$ der Grenzwert
$\lim_{k \to \infty} (f(x)g(y), \varphi(x + y)\eta_k(x, y)) =:
\ell_{f \ast g}(\varphi)$ existiert und unabhängig von der Wahl der $\eta_k$
ist, dann ist die Distribution $f \ast g \in \D'(\real^n)$ definiert durch
$(f \ast g, \varphi) := \ell_{f \ast g}(\varphi)$ für alle
$\varphi \in \D(\real^n)$.

Die Faltung existiert nicht immer (z.\,B. $1 \ast 1$).

\emph{Beispiel}:
Seien $f, \delta \in \D'(\real^n)$.
Dann ist $f \ast \delta = \delta \ast f = f$, da
$(f(x)\delta(y), \varphi(x + y)\eta_k(x, y))$\\
$= (f(x), (\delta(y), \varphi(x + y)\eta_k(x, y))) =
(f(x), \varphi(x)\eta_k(x, 0)) =
(f, \varphi)$ für $k \ge N(K)$ mit $K = \supp \varphi$,
da $\eta_k(x, 0) = 1$ für diese $k$ und alle $x \in K$.

\linie
\pagebreak

\textbf{Eigenschaften}:
\begin{enumerate}
    \item
    \emph{Stetigkeit gilt \textbf{nicht}}:
    $\tau_f\colon T \subset \D'(\real^n) \rightarrow \D'(\real^n)$,
    $g \mapsto f \ast g$ ist linear,
    aber i.\,A. nicht stetig ($T$ sei die Teilmenge von $\D'(\real^n)$,
    sodass $f \ast g$ für $g \in T$ definiert ist).\\
    Ein Gegenbeispiel ist für $d = 1$ die Distributionenfolge
    $g_k = \delta(x - k)$ für $k \in \natural$.
    Es gilt $(g_k, \varphi) = \varphi(k) \to 0$ für $k \to 0$,
    da $\varphi$ kompakt getragen ist.
    Somit ist $g_k = \delta(x - k) \xrightarrow{\D'} 0$.
    Für $f \equiv 1$ gilt allerdings
    $f \ast g_k = f \ast \delta(x - k) = f = 1 \not\to 0$,
    d.\,h. die Abbildung $\tau_f$ ist nicht stetig.
    Analog argumentiert man für $\tau_g\colon f \mapsto f \ast g$.

    \item
    \emph{Kommutativität}:
    Für $f, g \in \D'(\real^n)$ mit $\exists f \ast g \in \D'(\real^n)$
    gibt es auch $g \ast f \in \D'(\real^n)$ und es gilt
    $g \ast f = f \ast g$, denn
    $(f \ast g, \varphi) =
    \lim_{k \to \infty} (f(x)g(y), \eta_k(x, y)\varphi(x + y))$\\
    $= \lim_{k \to \infty} (g(y)f(x), \eta_k(x, y)\varphi(x + y)) =
    \lim_{k \to \infty} (g(x)f(y), \eta_k(y, x)\varphi(x + y)) =
    (g \ast f, \varphi)$,
    da $\eta_k(y, x) \to 1$ wie $\eta_k(x, y)$.

    \item
    \emph{Dif"|ferenzierbarkeit}:
    Für $f, g \in \D'(\real^n)$ mit $\exists f \ast g \in \D'(\real^n)$
    und $\alpha \in \natural_0^n$
    gibt es auch $(\partial^\alpha f) \ast g,
    f \ast (\partial^\alpha g) \in \D'(\real^n)$ und es gilt
    $(\partial^\alpha f) \ast g = f \ast (\partial^\alpha g) =
    \partial^\alpha (f \ast g)$.\\
    Die Umkehrung gilt nicht:
    Aus der Existenz von $\theta' \ast 1$ und $\theta \ast 1'$
    kann man nicht folgern, dass $\theta \ast 1$ existiert
    (sonst gäbe es $\theta' \ast 1 = \delta \ast 1 = 1$ und
    $\theta \ast 1' = \theta \ast 0 = 0$ und
    die beiden Ausdrücke wären gleich).

    \item
    \emph{Assoziativität gilt \textbf{nicht}}:
    Sonst wäre $(\theta \ast \delta') \ast 1 = \theta \ast (\delta' \ast 1)$,
    allerdings ist die linke Seite
    $(\theta \ast \delta') \ast 1 = (\theta \ast \delta)' \ast 1 =
    (\theta)' \ast 1 = \delta \ast 1 = 1$ und die rechte Seite\\
    $\theta \ast (\delta' \ast 1) = \theta \ast (\delta \ast 1)' =
    \theta \ast (1)' = \theta \ast 0 = 0$.

    \item
    \emph{Translation}:
    Existiert $f \ast g$, so existiert auch
    $f(x + h) \ast g = (f \ast g)(x + h)$.

    \item
    \emph{Existenzkriterium bei kompaktem Träger}:
    Seien $f, g \in \D'(\real^n)$ mit $\supp g = K$ kompakt.\\
    Dann existiert die Faltung $f \ast g$.

    \item
    \emph{Stetigkeit bei kompaktem Träger}:
    Seien $f_k, f, g \in \D'(\real^n)$ mit $f_k \xrightarrow{\D'(\real^n)} f$
    und $\supp g = K$ kompakt.
    Dann gilt $f_k \ast g \xrightarrow{\D'(\real^n)} f \ast g$.
    Umgekehrt seien $g_k, f, g \in \D'(\real^n)$ mit
    $g_k \xrightarrow{\D'(\real^n)} g$
    und $\exists_{R < \infty} \forall_{k \in \natural}\;
    \supp g_k \subset U_R(0)$.
    Dann gilt $f \ast g_k \xrightarrow{\D'(\real^n)} f \ast g$.

    \item
    \emph{Faltung mit Testfunktion}:
    Sei $\psi \in \D(\real^n)$.
    Dann ist $(f \ast \psi)(y) = (f(x), \psi(y - x)) \in \C^\infty(\real^n)$.\\
    Als Beispiel betrachtet man eine Delta-Folge $\psi_k \in \D$.
    Dann gilt $f_k = f \ast \psi_k \in \C^\infty$ und
    $f_k \xrightarrow{\D'} f \ast \delta = f$ aufgrund der Stetigkeit.
    Damit ist $\D$ dicht in $\D'$.
\end{enumerate}

\section{%
    Fundmental"|lösungen für PDE%
}

\textbf{Dif"|ferentialausdruck}:
Seien $m \in \natural$ und $a_\alpha \in \complex$ konstant für alle
$\alpha \in \natural_0^d$ mit $|\alpha| \le m$.\\
Dann heißt $L(\partial) = \sum_{|\alpha| \le m} a_\alpha \partial^\alpha$
\emph{Dif"|ferentialausdruck}.

\textbf{Fundamental"|lösung}:
$\varepsilon \in \D'(\real^d)$ heißt \emph{Fundamental"|lösung} von
$L(\partial)$, falls $L(\partial) \varepsilon = \delta$.

\emph{Beispiel}: Sei $L(\partial) = \Delta =
\frac{\partial^2}{\partial x_1^2} + \dotsb +
\frac{\partial^2}{\partial x_d^2}$ der Laplace-Operator.\\
Dann ist $\varepsilon_2 = \frac{1}{2\pi} \ln |x|$
eine Fundamental"|lösung für $d = 2$ und
$\varepsilon_d = \frac{-|x|^{2-d}}{(d - 2)\sigma_d}$
eine Fundamental"|lösung für $d \ge 2$ mit
$\sigma_d = \frac{2\pi^{d/2}}{\Gamma(d/2)}$
der Oberfläche der $d$-dimensionalen Einheitskugel.

\textbf{Anmerkung}:
Die Fundamental"|lösung ist i.\,A. nicht eindeutig, denn für
$u_0 \in \D'$ mit $L(\partial) u_0 = 0$ gilt
$L(\partial)(\varepsilon + u_0) = L(\delta)\varepsilon + L(\partial)u_0 =
L(\delta)\varepsilon = \delta$.

\textbf{Satz}:
Seien $\varepsilon$ eine Fundamental"|lösung von $L(\partial)$ und
$f \in \D'$, sodass $u = \varepsilon \ast f \in \D'$ existiert.\\
Dann gilt $L(\partial)u = f$ und jede Lösung $u$ von $L(\partial)u = f$
ist eindeutig in der Klasse der $u$, für welche $u \ast \varepsilon$ existiert.

\pagebreak

\section{%
    Der Raum der temperierten Distributionen \texorpdfstring{$\S'$}{S'}%
}

Für Anwendungen wie die Fourier-Transformation sieht man, dass die bisher
betrachtete Räume $\D$ und $\D'$ von Testfunktionen und Distributionen
zu weit gefasst sind.
Daher werden nun andere Räume $\S$ und $\S'$ von Testfunktionen und
Distributionen eingeführt, um die Fourier-Transformationen auf $\S'$
zu verallgemeinern.

\textbf{Raum der Testfunktionen $\S = \S(\real^d)$}:
Als Raum der Testfunktionen betrachtet man nun\\
$\S = \S(\real^d) := \{\varphi \in \C^\infty(\real^d) \;|\;
\forall_{\alpha, \beta \in \natural_0^d} \exists_{C(\alpha, \beta) < \infty}\;
|(1 + x^\alpha)\partial^\beta \varphi| \le C(\alpha, \beta)\}$.

\textbf{Konvergenz auf $\S$}:
Für eine Folge von Testfunktionen $\{\varphi_k\}_{k \in \natural}$ und
$\varphi$ in $\S$ schreibt man $\varphi_k \xrightarrow{\S} \varphi$, falls
$\forall_{\alpha, \beta \in \natural_0^d}\;
\sup_{x \in \real^d} |(1 + x^\alpha) \partial^\beta (\varphi_k - \varphi)|
\to 0$.

\textbf{Bemerkung}:
Es gilt $\D \subset \S$ dicht und aus
$\varphi_k \xrightarrow{\D} \varphi$ folgt
$\varphi_k \xrightarrow{\S} \varphi$.

\textbf{Eigenschaften}:
\begin{enumerate}
    \item
    $\partial^\alpha\colon \S \rightarrow \S$ ist linear und stetig.

    \item
    $\pi_{A,b}\colon \S \rightarrow \S$ ist linear und stetig.

    \item
    Für $\alpha \in \C^\infty$ und $\varphi \in \S$ gilt i.\,A. \emph{nicht}
    $\alpha \cdot \varphi \in \S$
    (wenn $\alpha$ schneller wächst wie $\varphi$ abfällt).
    Daher geht man über zu $\Theta_M := \{\alpha \in \C^\infty \;|\;
    \forall_{\beta \in \natural_0^d} \exists_{C(\beta) < \infty}\;
    |\partial^\beta \alpha(x)| \le C(\beta) (1 + |x|^{m_\beta})\}$.\\
    In diesem Fall folgt aus $\alpha \in \Theta_M$ und $\varphi \in \S$,
    dass $\alpha \cdot \varphi \in \S$ und die Abbildung
    $\varphi \mapsto \alpha \cdot \varphi$ ist stetig in $\S$.
\end{enumerate}

\linie

\emph{Motivation}:
Für $f \in L^1_\loc$ und $\int f(x) (1 + |x|)^{-m} \dx < \infty$ für ein
geeignetes $m \in \natural$ definiert
$(f, \varphi) = \int f(x) \varphi(x)\dx$ ein lineares stetiges Funktional auf
$\S$.

\textbf{Raum der temperierten Distributionen $\S'$}:\\
$\S'$ ist der Raum der linearen stetigen Funktionale auf $\S$.

\textbf{Konvergenz auf $\S'$}:
Für eine Folge von Distributionen $\{f_k\}_{k \in \natural}$ und
$f$ in $\S'$ schreibt man $f_k \xrightarrow{\S'} f$, falls
$\forall_{\varphi \in \S}\; (f_k, \varphi) \to (f, \varphi)$.

Es gilt $\D_f' \subset \S' \subset \D'$, wobei
$\D_f'$ der Raum der Distributionen aus $\D'$ mit kompaktem Träger ist.
Somit können alle Operationen
(Ableitung, Tensorprodukt, Faltung usw.)
für $\S'$ analog wie für $\D'$ definiert werden,
die Rechenregeln bleiben dabei dieselben.

\pagebreak

\section{%
    Die \name{Fourier}-Transformation für temperierte Distributionen%
}

Sei $\varphi \in \S(\real^d)$.\\
Dann ist die Fourier-Transformation definiert durch
$\F[\varphi](\xi) = \frac{1}{(2\pi)^{d/2}}
\int_{\real^d} e^{-\i\sp{x, \xi}}\varphi(x)\dx$.\\
Die Fourier-Transformation $\F\colon \S \rightarrow \S$ ist wie schon gezeigt
eine bijektive Abbildung.

\textbf{Lemma}: $\F\colon \S \rightarrow \S$ ist eine stetige Bijektion.\\
Aus $\varphi_n \xrightarrow{\S} 0$ folgt
$\partial^\alpha_\xi (\xi^\beta \F[\varphi_n]) \to 0$ gleichmäßig.

Man nun den Begriff der Fourier-Transformation auf Distributionen erweitern.\\
Beispielsweise soll für die Delta-Distribution gelten, dass\\
$\F[\delta(x - x_0)](\xi) = \frac{1}{(2\pi)^{d/2}}
\int_{\real^d} e^{-\i\sp{x, \xi}}\delta(x - x_0)\dx =
\frac{1}{(2\pi)^{d/2}} e^{-\i\sp{x_0, \xi}}$.

\linie

\emph{Motivation}:
Für $f \in L^1(\real^d)$ gilt nach Fubini
$(\F[f], \varphi) = \frac{1}{(2\pi)^{d/2}} \int_{\real^d} \varphi(\xi)
\left(\int_{\real^d} e^{-\i\sp{x, \xi}}f(x)\dx\right) \d\xi$\\
$= \frac{1}{(2\pi)^{d/2}} \int_{\real^d} f(x)
\left(\int_{\real^d} e^{-\i\sp{\xi, x}}\varphi(\xi)\d\xi\right) \dx =
(f, \F[\varphi])$.

\textbf{\name{Fourier}-Transformation}:
Sei $f \in \S'(\real^d)$.\\
Dann ist die Distribution $\F[f] \in \S'(\real^d)$ definiert durch
$(\F[f], \varphi) := (f, \F[\varphi])$ für alle\\
$\varphi \in \S(\real^d)$.

\textbf{Korrektheit}:
Für $\varphi \in \S$ ist $\F[\varphi] \in \S$, d.\,h.
$(f, \F[\varphi])$ ist wohldefiniert.\\
Die Linearität folgt aus der Linearität von $\F\colon \S \rightarrow \S$.\\
Die Stetigkeit folgt aus obigem Lemma:\\
Für $\varphi_k \xrightarrow{\S} \varphi$ gilt
$\F[\varphi_k] \xrightarrow{\S} \F[\varphi]$, d.\,h.
$(f, \F[\varphi_k]) \to (f, \F[\varphi])$.

\emph{Beispiel}:
Für die Fourier-Transformation der Delta-Distribution gilt
$(\F[\delta(x - x_0)](\xi), \varphi(\xi))$\\
$= (\delta(x - x_0), \F[\varphi](x)) = \F[\varphi](x_0) =
\frac{1}{(2\pi)^{d/2}} \int_{\real^d} e^{-\i\sp{x_0, \xi}}\varphi(\xi)\d\xi =
(\frac{1}{(2\pi)^{d/2}} e^{-\i\sp{x_0, \xi}}, \varphi(\xi))$, also\\
$\F[\delta(x - x_0)] = \frac{1}{(2\pi)^{d/2}} e^{-\i\sp{x_0, \xi}}$.

Wichtige Formeln sind $\F[\delta] = \frac{1}{(2\pi)^{d/2}}$ und
$\F[1] = \frac{1}{(2\pi)^{d/2}} \delta$.

\linie

\emph{Motivation}:
Für $\psi \in \S$ gilt $\F^{-1}[\psi](x) =
\frac{1}{(2\pi)^{d/2}} \int_{\real^d} e^{\i\sp{x, \xi}}\psi(\xi)\d\xi =
\F[\psi \circ \pi](x)$ mit\\
$(\psi \circ \pi)(\xi) = \psi(-\xi)$.

\textbf{inverse \name{Fourier}-Transformation}:
Sei $f \in \S'(\real^d)$.\\
Dann ist die Distribution $\F^{-1}[f]$ definiert durch
$\F^{-1}[f] := \F[f \circ \pi]$ mit $\pi\colon \real^d \rightarrow \real^d$,
$\xi \mapsto -\xi$.

\linie

\textbf{Eigenschaften}
\begin{enumerate}
    \item
    \emph{FT und inverse FT sind invers zueinander}:
    Sei $f \in \S'(\real^d)$.\\
    Dann gilt $\F[\F^{-1}[f]] = \F^{-1}[\F[f]] = f$, denn\\
    $(\F^{-1}[\F[f]], \varphi) =
    (\F[\F[f] \circ \pi], \varphi) =
    (\F[f] \circ \pi, \F[\varphi]) =
    (\F[f], \F[\varphi] \circ \pi) =
    (\F[f], \F^{-1}[\varphi])$\\
    $= (f, \F[\F^{-1}[\varphi]]) =
    (f, \varphi)$
    aufgrund $\F[\F^{-1}[\varphi]] = \varphi$ für $\varphi \in \S(\real^d)$.

    \item
    \emph{FT ist eine Bijektion}:
    $\F\colon \S' \rightarrow \S'$ ist eine Bijektion, denn sie ist\\
    surjektiv
    (für $g \in \S'$ gilt $\F[f] = g$ mit $f = \F^{-1}[g] \in \S$) und\\
    injektiv
    (aus $f \in \S'$ mit $\F[f] = 0$ folgt
    $\forall_{\varphi \in \S}\; (\F[f], \varphi) = (f, \F[\varphi]) = 0$,
    also\\
    $\forall_{\psi \in \S}\; (f, \psi) = 0$ und daher $f = 0$,
    indem man $\varphi = \F^{-1}(\psi) \in \S$ setzt).

    \item
    \emph{Ableitung der FT}:
    Für $f \in \S'(\real^d)$ und $\alpha \in \natural_0^d$ gilt
    $\partial_\xi^\alpha \F[f] = \F[(-\i x)^\alpha f]$, denn\\
    $(\partial_\xi^\alpha \F[f](\xi), \varphi(\xi)) =
    (-1)^{|\alpha|} (\F[f](\xi), \partial_\xi^\alpha \varphi(\xi)) =
    (-1)^{|\alpha|} (f(x), \F[\partial_\xi^\alpha \varphi(\xi)](x))$\\
    $= (-1)^{|\alpha|} (f(x), (ix)^\alpha \F[\varphi(\xi)](x)) =
    (-1)^{|\alpha|} ((ix)^\alpha f(x), \F[\varphi(\xi)](x))$\\
    $= ((-ix)^\alpha f(x), \F[\varphi(\xi)](x)) =
    (\F[(-ix)^\alpha f](\xi), \varphi(\xi))$.

    \item
    \emph{FT der Ableitung}:
    Analog beweist man $\F[\partial_x^\alpha f] = (i\xi)^\alpha \F[f]$.

    \item
    \emph{FT einer skalierten Funktion}:
    Sei $c \in \real$ mit $c \not= 0$.
    Dann ist $\F[f(cx)](\xi) = |c|^{-d} \F[f](\frac{\xi}{c})$.

    \item
    \emph{FT vom Tensorprodukt}:
    Mit $x, \xi \in \real^n$ und $y, \eta \in \real^m$ gilt
    $\F_{(x, y) \to (\xi, \eta)}[f(x) \cdot g(y)](\xi, \eta)$\\
    $= \F_{x \to \xi}[f](\xi) \cdot \F_{y \to \eta}[g](\eta) =
    \F_{y \to \eta}[\F_{x \to \xi}[f](\xi) \cdot g(y)](\eta) =
    \F_{x \to \xi}[\F_{y \to \eta}[g](\eta) \cdot f(x)](\xi)$.

    \item
    \emph{FT bei kompaktem Träger}:
    Für $g \in \D'(\real^d)$ mit $\supp g = K$ kompakt
    (d.\,h. insbesondere $g \in \S'(\real^d)$) gilt
    $\F[g] \in \Theta_M$ mit
    $\F[g] = (g(x), \eta(x)\frac{e^{-\i\sp{x, y}}}{(2\pi)^{d/2}})$.\\
    Dabei ist $\eta \in \D(\real^d)$ mit $\eta \equiv 1$ auf $K$.

    \item
    \emph{FT der Faltung}:
    Seien $f, g \in \S'(\real^d)$ mit $\supp g = K$ kompakt.\\
    Dann ist $\frac{1}{(2\pi)^{d/2}} \F[f \ast g] = \F[f] \cdot \F[g]$
    mit $\F[f] \in \S'(\real^d)$ und $\F[g] \in \Theta_M$, denn\\
    $(\F[f \ast g], \varphi) = (f \ast g, \F[\varphi]) =
    (f(x), (g(y)\eta(y) \cdot \frac{1}{(2\pi)^{d/2}}
    \int_{\real^d} e^{-\i\sp{x + y, \xi}}\varphi(\xi)\d\xi))$\\
    $= (f(x), \int_{\real^d}
    \left(\frac{1}{(2\pi)^{d/2}} (g(y), \eta(y)e^{-\i\sp{y, \xi}})\right)
    \varphi(\xi)e^{-\i\sp{x, \xi}}\d\xi) =
    (f(x), \int_{\real^d} \F[g](\xi)\varphi(\xi)e^{-\i\sp{x, \xi}}\d\xi)$\\
    $= (2\pi)^{d/2} (f(x), \F[\F[g](\xi) \cdot \varphi(\xi)](x)) =
    (2\pi)^{d/2} (\F[f], \F[g] \cdot \varphi) =
    (2\pi)^{d/2} (\F[g] \cdot \F[f], \varphi)$.\\
    Dabei ist $\eta(y)$ gleich $1$ auf $\supp g$ und es wurde die Formel aus
    7. angewandt.
\end{enumerate}

\section{%
    Die \name{Fourier}-Transformation zur Berechnung von Fundamental"|lösungen%
}

Sei $L(\partial) = \sum_{|\alpha| \le m} a_\alpha \partial^\alpha$ ein
Dif"|ferentialausdruck auf $\real^d$ mit konstanten $a_\alpha \in \complex$.\\
Gesucht ist ein $\varepsilon \in \S'(\real^d)$ mit
$L(\partial)\varepsilon = \delta$ (Fundmental"|lösung).\\
Für dieses $\varepsilon$ gilt dann $\F[L(\partial)\varepsilon] = \F[\delta]$,
d.\,h. $L(\i\xi) \F[\varepsilon] = \frac{1}{(2\pi)^{d/2}}$.

\emph{Beispiel}:
Für $d = 2$ und $L(\partial) = \Delta = \frac{\partial^2}{\partial x_1^2} +
\frac{\partial^2}{\partial x_2^2}$ gilt
$L(\i\xi) = (\i\xi_1)^2 + (\i\xi_2)^2 = -(\xi_1^2 + \xi_2^2)$.\\
Daher gilt für $\varepsilon \in \S'(\real^2)$, dass
$\Delta \varepsilon = \delta$ genau dann, wenn
$-(\xi_1^2 + \xi_2^2) \F[\varepsilon] = \frac{1}{2\pi}$.

Aus der Gleichung $L(\i\xi) \F[\varepsilon] = \frac{1}{(2\pi)^{d/2}}$
kann man $\varepsilon$ herleiten:
Falls $\frac{1}{L(\i\xi)} \in L^1_\loc$ gilt, so ist
$\F[\varepsilon] = \frac{1}{(2\pi)^{d/2} L(\i\xi)}$.
Andernfalls führt man eine geeignete Regularisation durch
(z.\,B. Annähern von $\frac{1}{x}$ durch $P\frac{1}{x}$ oder
$\frac{1}{x \pm \i \cdot 0}$).
Der \textbf{Satz von Hörmander} besagt, dass
obige Gleichung immer eine distributionelle Lösung $X$ besitzt.
Dann kann man $\varepsilon = \F^{-1}[X]$ berechnen.

\linie

\emph{Beispiel}:
Um die Fundmental"|lösung für $d = 3$ und den Laplace-Operator $\Delta$
zu finden, verwendet man wieder die Gleichung
$-|\xi|^2 \F[\varepsilon] = \frac{1}{(2\pi)^{3/2}}$.
Dabei ist $|\xi|^2 = \xi_1^2 + \xi_2^2 + \xi_3^2$ und
$X := \F[\varepsilon]$.\\
Man erhält also $X = -\frac{1}{(2\pi)^{3/2} |\xi|^2}$.
Dies ist allerdings nur lokal integrierbar (nicht im $L^1$).\\
Man verwendet daher die Approximation (Regularisierung)
$X_\nu := -\frac{\chi|_{|\xi| \le R(\nu)}}
{(2\pi)^{3/2} (|\xi|^2 + |\nu|^2)}$.\\
Für $\nu \to 0$ und $R(\nu) \to 0$ gilt
$X_\nu(\xi) \xrightarrow{(\cdot)} X(\xi)$,
also $X_\nu \xrightarrow{\S'} X$ und
$F^{-1}[X_\nu] \xrightarrow{\S'} \F^{-1}[X] = \varepsilon$.\\
Daher ist $\varepsilon = \lim_{\nu \to 0, R(\nu) \to \infty} \F^{-1}[X_\nu]$.\\
Für die Berechnung von
$\F^{-1}[X_\nu] = \frac{1}{(2\pi)^3} \int_{\real^3,\; |\xi| \le R(\nu)}
\frac{e^{\i\sp{x, \xi}}}{|\xi|^2 + \nu^2}d^3\xi$ führt man eine
Koordinatentransformation in Kugelkoordinaten $(R, \varphi, \theta)$
durch, sodass $x$ auf der $z$-Achse liegt.
Mit $r := |x|$ und $R := |\xi|$ ist dann $\sp{x, \xi} = rR\cos \theta$
mit $\theta$ dem Winkel zwischen $x$ und $\xi$.\\
Damit ist dann
$\F^{-1}[X_\nu] = \frac{1}{(2\pi)^3} \int_{\real^3,\; |\xi| \le R(\nu)}
\frac{e^{\i\sp{x, \xi}}}{|\xi|^2 + \nu^2}d^3\xi =
\frac{1}{(2\pi)^3} \int_0^{2\pi} \int_0^\pi \int_0^{R(\nu)}
\frac{e^{\i rR\cos \theta}}{R^2 + \nu^2} R^2 \dr \sin \theta d\theta d\varphi$\\
$= \frac{1}{(2\pi)^2} \int_0^\pi \int_0^{R(\nu)}
\frac{e^{\i rR\cos \theta}}{R^2 + \nu^2} R^2 \dr \sin \theta d\theta =
\frac{1}{(2\pi)^2} \int_{-1}^1 \int_0^{R(\nu)} \frac{R^2}{R^2 + \nu^2}
e^{\i rRy} dR\dy$\\
$= \frac{1}{(2\pi)^2} \int_0^{R(\nu)} \frac{R^2}{R^2 + \nu^2} \cdot
\frac{1}{\i rR} (e^{\i rR} - e^{-\i rR}) dR =
\frac{1}{(2\pi)^2} \cdot 2 \cdot \frac{1}{r}
\int_0^{R(\nu)} \frac{R}{R^2 + \nu^2} \cdot \sin(rR) dR$\\
$= \frac{1}{(2\pi)^2} \cdot \frac{1}{r}
\int_{-R(\nu)}^{R(\nu)} \frac{R}{R^2 + \nu^2} \cdot \sin(rR) dR =
\frac{1}{(2\pi)^2} \cdot \frac{1}{2\i r}
\int_{-R(\nu)}^{R(\nu)} \frac{R}{R^2 + \nu^2} (e^{\i rR} - e^{-\i rR}) dR$
mit $y = \cos \theta$.\\
Per Integration über einen Halbkreis in der oberen bzw. unteren Halbebene
sieht man\\
$I_\nu^\pm := \int_{-R(\nu)}^{R(\nu)}
\frac{R}{R^2 + \nu^2} e^{\pm\i rR}dR = \pm\i\pi e^{-\nu r} + o(1)$ für
$\nu \to 0$ (mit dem Lemma von Riemann).\\
Damit ist $\F^{-1}[X_\nu] =
\frac{1}{(2\pi)^2} \cdot \frac{1}{2\i r} \cdot (I_\nu^+ - I_\nu^-) =
\frac{1}{(2\pi)^2} \cdot \frac{1}{2\i r} \cdot (\i\pi - (-\i\pi))e^{\nu r} +
o(1) \xrightarrow{\nu \to 0} \frac{1}{4\pi r}$.

\pagebreak
