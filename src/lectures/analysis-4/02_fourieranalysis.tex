\chapter{%
    \name{Fourier}analysis und trigonometrische Reihen%
}

\section{%
    Motivation%
}

Sei $E$ ein euklidischer oder hermitescher Vektorraum der Dimension
$\dim E = n$, d.\,h. ein $\real$- oder $\complex$-Vektorraum, auf dem
ein Skalarprodukt $\innerproduct{\cdot, \cdot}$ gegeben ist
(hier linear im ersten Argument).

In diesem Fall existiert eine \textbf{Orthonormalbasis (ONB)}
$\{e_1, \dotsc, e_n\}$, d.\,h. eine Basis, sodass
$\innerproduct{e_j, e_k} = \delta_{jk}$ für $j, k = 1, \dotsc, n$ ist.
Ist $x \in E$ ein Vektor, so kann man $x$ eindeutig als Linearkombination
der Basis darstellen, also $x = \xi_1 e_1 + \dotsb + \xi_n e_n$
mit Skalaren $\xi_k$.\\
Es gilt $\xi_k = \innerproduct{x, e_k}$ für $k = 1, \dotsc, n$.
Die $\innerproduct{x, e_k}$ heißen \textbf{\name{Fourier}koef"|fizienten} von $x$.

Dies gilt ohne Weiteres jedoch nicht mehr für unendlich-dimensionale
Vektorräume $E$, z.\,B. ist auf $E = \C([a, b], \complex)$ die Norm
$\norm{f}_\C = \max_{x \in [a, b]} |f(x)|$ definiert, jedoch gibt es kein
Skalarprodukt, das diese Norm induziert.

\linie

Im Folgenden wird das Skalarprodukt
$\innerproduct{f, g}_{L^2} = \int_{[a, b]} f(t) \overline{g(t)} \dt$ und die davon
induzierte Norm $\norm{f}_{L^2}^2 = \int_{[a, b]} |f(t)|^2 \dt$ verwendet.\\
Der Einfachheit halber beschränkt man sich auf $[a, b] = [-\pi, \pi]$.

Betrachtet man die Funktionen $1, \sin x, \cos x, \sin(2x), \cos(2x), \dotsc$,
so stellt man fest:
\begin{itemize}
    \item
    $\int_{-\pi}^\pi 1\sin(nx)\dx = \int_{-\pi}^\pi 1\cos(mx)\dx = 0$
    für $n, m \in \natural$ und\\
    $\int_{-\pi}^\pi \cos(nx)\cos(mx)\dx = \int_{-\pi}^\pi \sin(nx)\sin(mx)\dx =
    0$
    für $n, m \in \natural$, $n \not= m$

    \item
    $\int_{-\pi}^\pi \sin(nx)\cos(mx)\dx = 0$ für $n, m \in \natural$

    \item
    $\int_{-\pi}^\pi \sin^2(nx)\dx = \int_{-\pi}^\pi \cos^2(mx)\dx = \pi$
    für $n, m \in \natural$ und
    $\int_{-\pi}^\pi 1^2 \dx = 2\pi$
\end{itemize}
Daher bildet $\frac{1}{\sqrt{2\pi}}, \frac{1}{\sqrt{\pi}} \sin x,
\frac{1}{\sqrt{\pi}} \cos x, \frac{1}{\sqrt{\pi}} \sin(2x),
\frac{1}{\sqrt{\pi}} \cos(2x), \dotsc$ ein
Orthonormalsystem.\\
Insbesondere ist dieses System linear unabhängig
(d.\,h. jede endliche Linearkombination der $0$ mit Vektoren aus diesem System
ist trivial).

\linie

Für eine gegebene Funktion $f \in \C([-\pi, \pi], \complex)$ kann man nun
die Fourierkoef"|fizienten\\
$\alpha_n := \innerproduct{f, \frac{1}{\sqrt{\pi}} \sin(nx)}_{L^2} =
\frac{1}{\sqrt{\pi}} \int_{-\pi}^\pi f(x)\sin(nx)\dx$,\\
$\beta_m := \innerproduct{f, \frac{1}{\sqrt{\pi}} \cos(mx)}_{L^2} =
\frac{1}{\sqrt{\pi}} \int_{-\pi}^\pi f(x)\cos(mx)\dx$ und\\
$\gamma := \innerproduct{f, \frac{1}{\sqrt{2\pi}}}_{L^2} =
\frac{1}{\sqrt{2\pi}} \int_{-\pi}^\pi f(x)\dx$
für $n, m \in \natural$ berechnen.

Man kann $f$ diese Fourierkoef"|fizienten
$(\{\alpha_n\}_{n \in \natural}, \{\beta_m\}_{m \in \natural}, \gamma)$
zuweisen und sich fragen, was $f$ mit der zunächst formalen
\textbf{\name{Fourier}-Reihe}
$\frac{\gamma}{\sqrt{2\pi}} +
\sum_{n=1}^\infty \left(\frac{\alpha_n}{\sqrt{\pi}} \sin(nx) +
\frac{\beta_n}{\sqrt{\pi}} \cos(nx)\right)$ zu tun hat.
Konvergiert diese Reihe (in welchem Sinn)?
Was hat der Wert der Reihe mit $f$ zu tun?
Welche Eigenschaften von $f$ korrespondieren in welcher Art mit welchen
Eigenschaften von $\{\alpha_n\}$, $\{\beta_m\}$ und $\gamma$?

\textbf{alternative Schreibweise}:
Man kann auch die "`unschönen"' Wurzeln vollständig in die Koef"|fizienten
ziehen.
Dafür schreibt man lateinische Buchstaben, d.\,h.
$f \mapsto (\{a_n\}_{n \in \natural}, \{b_m\}_{m \in \natural}, c)$,\\
$a_n := \frac{1}{\pi} \int_{-\pi}^\pi f(x)\sin(nx)\dx$,\\
$b_m := \frac{1}{\pi} \int_{-\pi}^\pi f(x)\cos(mx)\dx$ und\\
$c := \frac{1}{2\pi} \int_{-\pi}^\pi f(x)\dx$ für $n, m \in \natural$.\\
Die Fourier-Reihe vereinfacht sich dann zu
$c + \sum_{n=1}^\infty (a_n \sin(nx) + b_n \cos(nx))$.

\linie
\pagebreak

Ersetzt man $\sin(nx) = \frac{e^{\i nx} - e^{-\i nx}}{2\i}$ und
$\cos(mx) = \frac{e^{\i mx} + e^{-\i mx}}{2}$, so gilt wegen\\
$\innerproduct{e^{\i nx}, e^{\i mx}}_{L^2} = \int_{-\pi}^\pi e^{\i nx} e^{-\i mx}\dx =
\int_{-\pi}^\pi e^{\i (n - m)x}\dx =$
\matrixsize{$\begin{cases}0 & n \not= m\\2\pi & n = m\end{cases}$}
für $n, m \in \integer$,\\
dass $\left\{\frac{1}{\sqrt{2\pi}} e^{\i nx}\right\}_{n \in \integer}$
ein Orthonormalsystem ist.

Definiert man
$\gamma_n := \innerproduct{f, \frac{1}{\sqrt{2\pi}} e^{\i nx}}_{L^2} =
\frac{1}{\sqrt{2\pi}} \int_{-\pi}^\pi f(x) e^{-\i nx}\dx$, so kann man wieder
die formale Fourier-Reihe
$\sum_{n \in \integer} \frac{\gamma_n}{\sqrt{2\pi}} e^{\i nx}$ definieren.\\
Analog wie eben schreibt man auch oft
$c_n := \frac{1}{2\pi} \int_{-\pi}^\pi f(x) e^{-\i nx} \dx$ bzw.
$\sum_{n \in \integer} c_n e^{\i nx}$.

\linie

Diese Reihe ist eine Laurent-Reihe
$\widetilde{f}(z) = \sum_{n=-\infty}^{+\infty} c_n z^n$ um $z_0 = 0$.
Falls $0 \le r < 1 < R \le \infty$, so gilt für $z = e^{\i x}$ und
$x \in [-\pi, \pi]$, dass
$\widetilde{f}(e^{\i x}) = \sum_{n=-\infty}^{+\infty} c_n e^{\i nx}$.\\
In diesem Fall lässt sich die Formel
$c_n = \frac{1}{2\pi\i} \oint_{|z| = 1} \frac{\widetilde{f}(z)}{z^{n+1}}\dz$
anwenden.\\
Man erhält dadurch wieder die Definition der
$c_n = \frac{1}{2\pi} \int_{-\pi}^\pi f(x) e^{-\i nx} \dx$.

\section{%
    Das Kriterium von \name{Dini}%
}

Im Folgenden betrachtet man die Partialsummen
$S_N(t) = c + \sum_{k=1}^N (a_k \sin(kt) + b_k \cos(kt))$ für $N \in \natural$.
Es gilt
$S_N(t) = \sum_{k=-N}^N c_k e^{\i kt}$,
denn\\
$c_k e^{\i kt} + c_{-k} e^{-\i kt} =
(c_k + c_{-k}) \cos(kt) + \i (c_k - c_{-k}) \sin(kt) =
\frac{1}{2\pi} \int_{-\pi}^\pi f(x) (e^{-\i kx} + e^{\i kx}) \dx \cos(kt) +
\frac{\i}{2\pi} \int_{-\pi}^\pi f(x) (e^{-\i kx} - e^{\i kx}) \dx \sin(kt) =
b_k \cos(kt) + a_k \sin(kt)$.

Daraus folgt dann\\
$S_N(t) = \sum_{k=-N}^N \left(\frac{1}{2\pi}
\int_{-\pi}^{\pi} f(\tau) e^{-\i k\tau}\d\tau\right) e^{ikt}
= \int_{-\pi}^{\pi} f(\tau)
\left(\frac{1}{2\pi} \sum_{k=-N}^N e^{\i k(t - \tau)}\right)\d\tau$.\\
Dabei ist
$\sum_{k=-N}^N e^{\i ks} =
e^{-\i Ns} \sum_{k=0}^{2N} e^{\i ks} =
e^{-\i Ns} \cdot
\frac{1 - e^{\i (2N + 1)s}}{1 - e^{\i s}}$\\
$= e^{-\i Ns} \cdot
\frac{e^{\i (N + 1/2)s} \cdot
(e^{-\i (N + 1/2)s} - e^{\i (N + 1/2)s})}
{e^{\i s/2} \cdot (e^{-\i s/2} - e^{\i s/2})} =
\frac{\sin((N + 1/2) s)}{\sin(s/2)}$.

Daher ist $S_N(t) = \int_{-\pi}^\pi f(\tau) \D_N(t - \tau)\d\tau$ mit
dem \textbf{\name{Dirchlet}-Kern}\\
$\D_N(s) = \frac{1}{2\pi} \cdot \frac{\sin((N + 1/2)s)}{\sin(s/2)} =
\frac{1}{2\pi} \sum_{k=-N}^N e^{\i ks}$.

\linie

Es gilt $\int_{-\pi}^\pi \D_N(s)\ds = 1$
(dies sieht man schnell mit der Summenformel).\\
Außerdem ist $\D_N$ $2\pi$-periodisch, d.\,h. $\D_N(s) = \D_N(s + 2k\pi)$ für
alle $s \in \real$ und $k \in \integer$.

Außerdem setzt man $f\colon [-\pi, \pi] \rightarrow \complex$
$2\pi$-periodisch zu $f\colon \real \rightarrow \complex$ fort, d.\,h.
$f(t + 2k\pi) = f(t)$ für alle $t \in [-\pi, \pi]$ und $k \in \integer$.

Damit sind $\D_N$ und $f$ $2\pi$-periodisch und
$S_N(t) = \int_{-\pi}^\pi f(\tau) \D_N(t - \tau) \d\tau =
\int_{-\pi}^\pi \D_N(s) f(s + t) \ds$
(aufgrund der Symmetrie von $\D_N(s)$).

Um die Konvergenz von $S_N(t)$ gegen $f(t)$ zu verifizieren, nutzt man
$f(t) = \int_{-\pi}^\pi \D_N(s) f(t) \ds$ aus und berechnet
$S_N(t) - f(t) = \int_{-\pi}^\pi \D_N(s) (f(s + t) - f(t)) \ds =
\int_{-\pi}^\pi \frac{f(s + t) - f(t)}{2\pi \cdot \sin(s/2)} \cdot
\sin((N + \frac{1}{2})s) \ds$.\\
Der erste Faktor ist eine Funktion $F(s, t)$, die unabhängig von $N$ ist.
Der zweite Faktor ist eine Sinus-Funktion $\sin(\omega s)$ mit für
$N \to \infty$ immer schneller werdender Frequenz $\omega$.\\
Die übliche betragsmäßige Abschätzung kann hier nicht verwendet werden, da der
Sinus nur mit $1$ abgeschätzt werden kann.
Stattdessen kann man sich die Konvergenz bildhaft mit der in der
Signalübertragung verwendeten Amplitudenmodulation überlegen, bei der
eine Information (hier $F(s, t)$) in der Amplitude eines Trägersignals
mit konstanter Frequenz (hier $\sin(\omega s)$) kodiert wird.
Für eine genügend hohe Frequenz $\omega$ löschen sich positive und negative
Anteile annäherend aus, sodass Konvergenz (unter gewissen Bedingungen)
vorliegt.

\linie
\pagebreak

\textbf{Lemma}:
Sei $F \in \C^1([-\pi, \pi], \complex)$.
Dann gilt $\lim_{\omega \to \infty}
\left(\int_{-\pi}^\pi F(s) \sin(\omega s) \ds\right) = 0$.

\textbf{Lemma}:
Die Menge $\C_0^\infty([-\pi, \pi], \complex)$ der unendlich oft
dif"|ferenzierbaren Funktionen, die auf dem Rand von $[-\pi, \pi]$
verschwinden, liegt dicht in $L^1([-\pi, \pi], dx)$ mit dem Lebesgue-Maß $dx$,
d.\,h. $\forall_{F \in L^1}
\exists_{\{F_n\}_{n \in \natural},\; F_n \in \C_0^\infty}\;
\norm{F_n - F}_{L^1} \xrightarrow{n \to \infty} 0$.

\textbf{\name{Riemann}-Lemma}:
Sei $F \in L^1([-\pi, \pi], dx)$.
Dann gilt $\lim_{\omega \to \infty}
\left(\int_{[-\pi, \pi]} F(s) \sin(\omega s) \ds\right) = 0$.

\linie

\textbf{Kriterium von \name{Dini} zur punktweisen Konvergenz der
\name{Fourier}-Reihe}:\\
Seien $f \in L^1([-\pi, \pi], dx)$ und $t_0 \in [-\pi, \pi]$.\\
Es existiere ein $\delta = \delta(t_0) > 0$ mit
$\int_{[-\delta, \delta]}
\left|\frac{f(t_0 + \tau) - f(t_0)}{\tau}\right| \d\tau < \infty$.\\
Dann gilt $\lim_{N \to \infty} S_N(t_0) = f(t_0)$.

\textbf{Bemerkung}:
Die zweite Bedingung ist erfüllt, wenn $f$ in $t_0$ dif"|ferenzierbar ist.\\
Die zweite Bedingung ist erfüllt, wenn
$|f(t_0 + \tau) - f(\tau)| \le M |\tau|^\alpha$ für ein
$\alpha > 0$ und $|\tau| < \delta$.

Die Stetigkeit von $f$ in $t_0$ reicht im Allgemeinen nicht!

\linie

Was passiert, wenn $f$ in $t_0$ einen Sprung besitzt?
In diesem Fall kann $f$ als Summe einer stetigen Funktion und einer
charakteristischen Funktion dargestellt werden.
Falls die Fourier-Reihe der stetigen Funktion konvergiert, reicht es, die
Konvergenz der Fourier-Reihe für die charakteristische Funktion zu prüfen.
Es zeigt sich, dass dabei Konvergenz gilt.
Der Grenzwert befindet sich genau in der "`Mitte"' des Sprungs.

\textbf{modifiziertes Kriterium von \name{Dini} für Sprungstellen}:\\
Seien $f \in L^1([-\pi, \pi], dx)$ und $t_0 \in [-\pi, \pi]$.
Es existieren $f(t_0 - 0)$, $f(t_0 + 0)$ und\\
ein $\delta = \delta(t_0) > 0$ mit
$\int_{[-\delta, 0]}
\left|\frac{f(t_0 + \tau) - f(t_0 - 0)}{\tau}\right| \d\tau < \infty$ und
$\int_{[0, \delta]}
\left|\frac{f(t_0 + \tau) - f(t_0 + 0)}{\tau}\right| \d\tau < \infty$.\\
Dann gilt $\lim_{N \to \infty} S_N(t_0) = \frac{f(t_0 - 0) + f(t_0 + 0)}{2}$.

\textbf{Bemerkung}:
Die Bedingung $f \in L^1([-\pi, \pi], dx)$ ist so zu verstehen, dass
ein Repräsentant aus der Äquivalenzklasse von $f$ gewählt wird,
der diese Bedingung erfüllt.
Die Existenz des Sprunges und seine Höhe ist dann invariant für alle
äquivalenten Funktionen.

\linie

\textbf{Satz}:
Sei $f\colon \real \rightarrow \complex$ auf $[-\pi, \pi]$
$\ell$-fach dif"|ferenzierbar mit
$f^{(j)}(-\pi) = f^{(j)}(\pi)$ für $j = 0, \dotsc, \ell - 1$.\\
Außerdem sei $f^{(\ell)}$ Riemann-integrierbar auf $[-\pi, \pi]$.\\
Dann gilt $a_n = o(n^{-\ell})$,
$b_n = o(n^{-\ell})$ und
$c_n = o(n^{-\ell})$ für $n \to \infty$.

\textbf{Bemerkung}:
Für solche Funktionen fallen die Fourierkoef"|fizienten also schnell ab.
Dies ist wichtig, damit z.\,B. ein Tiefpass
(Weglassen der hohen Frequenzen) bei periodischen Signalen keine
allzu großen Störungen mit sich bringt.

\linie

Gilt auch die Umkehrung, d.\,h. folgt aus dem schnellen Abfallen der
Koef"|fizienten, dass $f$ glatt ist?
Dazu sei $\{c_k\}_{k \in \integer} \in \ell_1$
(also $\sum_{k=1}^\infty |c_k| < \infty$).
Die Summanden der Reihe $S(t) = \sum_{k \in \integer} c_k e^{\i kt}$,
$t \in [-\pi, \pi]$ können durch
$|c_k e^{\i k t}| = |c_k|$ gleichmäßig abgeschätzt werden.
Da $\{c_k\}_{k \in \integer} \in \ell_1$, konvergiert $S(t)$ absolut und
gleichmäßig.
Jeder der Summanden ist stetig,
also ist $S(t)$ stetig.

Falls sogar $\{k^\ell c_k\}_{k \in \integer} \in \ell_1$ gilt,
folgt nach $\ell$-maligem Dif"|ferenzieren, dass
$S(t)$ $\ell$-fach dif"|ferenzierbar ist.

\pagebreak

\section{%
    \name{Fourier}-Integral und \name{Fourier}-Transformation%
}

\textbf{\name{Fourier}-Transformation}:
Sei $f\colon \real \rightarrow \complex$ mit $f \in L^1(\real, dx)$.\\
Dann ist $\widehat{f}(\lambda) = \F[f](\lambda) :=
\frac{1}{\sqrt{2\pi}} \int_\real f(\tau) e^{-\i \lambda\tau} \d\tau$
für $\lambda \in \real$ die \emph{\name{Fourier}transformierte} von $f$.

\textbf{Eigenschaften von $\F[f]$}:
\begin{enumerate}
    \item
    Für $f \in L^1(\real, dx)$ existiert $\widehat{f}(\lambda)$ für alle
    $\lambda \in \real$, denn
    $|\widehat{f}(\lambda)| \le
    \frac{1}{\sqrt{2\pi}} \int_\real |f(\tau)| \d\tau =
    \frac{\norm{f}_{L^1}}{\sqrt{2\pi}}$.\\
    Daraus folgt insbesondere, dass
    $\widehat{f}$ eine beschränkte Funktion ist.

    \item
    $\widehat{f}$ ist stetig, denn für eine Folge
    $\{\lambda_n\}$ mit $\lambda_n \to \lambda$ gilt
    $f(\tau) e^{-i \lambda_n \tau} \to f(\tau) e^{-i \lambda \tau}$.
    Wegen $|f(\tau) e^{-\i \lambda_n \tau}| = |f(\tau)|$ ist
    $f(\tau)$ eine integrierbare Majorante für alle $\tau \in \real$ und
    $n \in \natural$.
    Aus dem Satz von Lebesgue zur majorisierten Konvergenz folgt daher
    $\widehat{f}(\lambda_n) \to \widehat{f}(\lambda)$.

    \item
    Es gilt $\lim_{\lambda \to \infty} \widehat{f}(\lambda) = 0$, denn
    aus der $\sigma$-Additivität des Lebesgue-Integrals folgt\\
    $\int_\real f \dx =
    \sum_{j \in \integer} (\int_{\left]j, j + 1\right]} f\dx) =
    \lim_{R \to \infty} (\int_{[-R, R]} f\dx)$, also gibt es für alle
    $\varepsilon > 0$ ein $R(\varepsilon) > 0$ mit
    $\int_{|x| > R(\varepsilon)} |f|\dx < \varepsilon$.
    Man teilt nun $\int_\real f(\tau) e^{-\i \lambda\tau} \d\tau =
    (\int_{|x| > R(\varepsilon)} + \int_{|x| \le R(\varepsilon)})
    f(\tau) e^{-\i \lambda\tau} \d\tau$ auf.
    Der erste Summand ist vom Betrag her nach eben Gesagtem
    $\le \int_{|x| > R(\varepsilon)} |f(\tau)| \d\tau < \varepsilon$,
    der zweite Summand geht für $\lambda \to \infty$ nach dem Lemma von
    Riemann gegen $0$, ist also $< \varepsilon$ für $\lambda$ groß genug.

    \item
    Die Fourier-Transformation $\F$ ist linear, d.\,h. für
    $f, g \in L^1(\real, dx)$ und $\alpha, \beta \in \complex$ gilt
    $\F[\alpha f + \beta g] = \alpha \F[f] + \beta \F[g]$.
    $\F$ ist zusätzlich stetig, d.\,h. $\F \in \L(L^1, L^\infty)$.
\end{enumerate}

\linie

Unter welchen Umständen existiert die inverse Fouriertransformierte, d.\,h.
wann ist\\
$f(t) = \frac{1}{\sqrt{2\pi}}
\int \F[f](\lambda) e^{\i \lambda t} d\lambda$?

\textbf{Kriterium von \name{Dini} für die Konvergenz der inversen
\name{Fourier}transformierten}:\\
Seien $f \in L^1(\real, dx)$ und $t_0 \in \real$.\\
Es existiere ein $\delta = \delta(t_0) > 0$ mit
$\int_{[-\delta, \delta]}
\left|\frac{f(t_0 + x) - f(t_0)}{x}\right| \dx < \infty$.\\
Dann konvergiert das uneigentliche Riemann-Integral
$\frac{1}{\sqrt{2\pi}}
\int_{-\infty}^{+\infty} \F[f](\lambda) e^{\i \lambda t_0} d\lambda = f(t_0)$.

\pagebreak

\section{%
    \name{Hilbert}räume und \name{Fourier}reihen%
}

Sei $H$ ein \emph{$K$-Vektorraum} mit $K = \complex$.

Sei $\innerproduct{\cdot, \cdot}\colon H \times H \rightarrow \complex$
ein \emph{Skalarprodukt}, d.\,h.
$\innerproduct{x, x} \ge 0$,
$\innerproduct{x, x} = 0 \iff x = 0$,
$\innerproduct{x, y} = \overline{\innerproduct{y, x}}$ und
$\innerproduct{\alpha_1 x_1 + \alpha_2 x_2, y} =
\alpha_1 \innerproduct{x_1, y} + \alpha_2 \innerproduct{x_2, y}$
(also $\innerproduct{x, \beta_1 y_1 + \beta_2 y_2} =
\overline{\beta_1} \innerproduct{x, y_1} + \overline{\beta_2} \innerproduct{x, y_2}$).

Das Skalarprodukt definiert eine Norm $\norm{\cdot}\colon H \rightarrow \real$
mit $\norm{x} = \sqrt{\innerproduct{x, x}} \ge 0$.\\
Es gilt die \emph{\name{Cauchy}-\name{Schwarz}sche Ungleichung}, d.\,h.
$|\innerproduct{x, y}| \le \norm{x} \cdot \norm{y}$.

\textbf{\name{Hilbert}raum}:
Sei $H$ ein $K$-Vektorraum mit Skalarprodukt $\innerproduct{\cdot, \cdot}$.\\
Dann heißt $(H, \innerproduct{\cdot, \cdot})$ \emph{\name{Hilbert}raum}, falls
$(H, \norm{\cdot})$ vollständig ist.

Die \emph{Parallelogramgleichung}
$\norm{x + y}^2 + \norm{x - y}^2 = 2 \norm{x}^2 + 2 \norm{y}^2$ ist
erfüllt genau dann, wenn eine gegebene Norm ein Skalarprodukt induziert
(in diesem Fall gilt z.\,B. für $K = \real$, dass
$4 \innerproduct{x, y} = \norm{x + y}^2 - \norm{x - y}^2$).

\linie

\emph{Beispiel}:
Ein Beispiel für einen Hilbertraum ist
$\ell^2(\natural)$ (oder auch $\ell^2(\integer)$).
Es gilt\\
$\ell^2(\natural) = \{\{a_n\}_{n \in \natural} \;|\;
\sum_{n \in \natural} |a_n|^2 < \infty\}$,
das Skalarprodukt ist
$\innerproduct{\{a_n\}, \{b_n\}}_{\ell^2(\natural)} =
\sum_{n \in \natural} a_n \overline{b_n}$.\\
$\ell^2(\natural)$ und $\ell^2(\integer)$ sind \emph{separabel}, d.\,h.
es gibt eine abzählbare dichte Teilmenge.

\emph{Beispiel}:
Die Verallgemeinerung ist $L^2(X, \mu) =
\{f\colon X \rightarrow \complex \;|\;
\norm{f}_{L^2}^2 = \int_X |f|^2 d\mu < \infty\}$ mit dem Skalarprodukt
$\innerproduct{f, g}_{L^2} = \int_X f \overline{g} d\mu$.
Falls $(X, \mu)$ ein separabler Maßraum ist, so ist auch
$L^2(X, \mu)$ separabel (z.\,B. Lebesgue-Maß).

\linie

\textbf{Orthogonalität}:
Sei $(H, \innerproduct{\cdot, \cdot})$ ein Hilbertraum.
Man definiert eine Relation $\bot$ auf $H$ mit
$f \orth g$, falls $\innerproduct{f, g} = 0$
($f$ und $g$ sind zueinander \emph{orthogonal}).

Für $f \orth g$ gilt der \emph{Satz des Pythagoras}, d.\,h.
$\norm{f + g}^2 = \innerproduct{f + g, f + g} = \norm{f}^2 + \norm{g}^2$.

Für $f_n \xrightarrow{\norm{\cdot}} f$ und $g_n \xrightarrow{\norm{\cdot}} g$
gilt $\innerproduct{f_n, g_n} \to \innerproduct{f, g}$, da
$|\innerproduct{f_n, g_n} - \innerproduct{f, g}|$\\
$= |\innerproduct{f_n, g_n - g} + \innerproduct{f_n - f, g}| \le
\norm{f_n} \cdot \norm{g_n - g} + \norm{f_n - f} \cdot \norm{g}
\le C \cdot \norm{g_n - g} + \norm{f_n - f} \cdot C \to 0$.

\linie

\textbf{orthonormiertes System (ONS)}:
Sei $(H, \innerproduct{\cdot, \cdot})$ ein Hilbertraum.\\
Ein System $\{\varphi_n\}_n \subset H$ heißt
\emph{orthonormiertes System (ONS)}, falls
$\innerproduct{\varphi_n, \varphi_k} = \delta_{nk}$.

\textbf{linear unabhängig}:
Sei $(H, \innerproduct{\cdot, \cdot})$ ein Hilbertraum.\\
Ein System $\{\varphi_n\}_n \subset H$ heißt
\emph{linear unabhängig}, falls jedes endliche Teilsystem lin.\,unabh. ist.

Jedes ONS ist linear unabhängig
(aus $\alpha_1 \varphi_1 + \dotsb + \alpha_n \varphi_n = 0$ folgt\\
$\innerproduct{\alpha_1 \varphi_1 + \dotsb + \alpha_n \varphi_n, \varphi_k}
= \alpha_k = \innerproduct{0, \varphi_k} = 0$ für alle $k = 1, \dotsc, n$).

\textbf{vollständig}:
Sei $(H, \innerproduct{\cdot, \cdot})$ ein Hilbertraum.\\
Ein System $\{\varphi_n\}_{n \in \natural} \subset H$ heißt
\emph{vollständig}, falls\\
$\forall_{x \in H} \forall_{\varepsilon > 0}
\exists_{N(\varepsilon, x) \in \natural}
\exists_{\{\alpha_k(\varepsilon, x)\}_{k=1,\dotsc,N(\varepsilon,x)}}\;
\norm{x - \sum_{k=1}^{N(\varepsilon,x)} \alpha_k(\varepsilon, x) \varphi_k}
< \varepsilon$.

\textbf{Basis, Orthonormalbasis (ONB)}:
Sei $(H, \innerproduct{\cdot, \cdot})$ ein Hilbertraum.\\
Ein System $\{\varphi_n\}_{n \in \natural} \subset H$ heißt
\emph{Basis}, falls es vollständig und linear unabhängig ist.\\
Ein ONS heißt \emph{Orthonormalbasis (ONB)}, falls es vollständig ist.

\linie

\emph{Beispiel}:
Für $H = \ell^2(\natural)$ ist $\{\varphi_k\}_{k \in \natural}$ mit
$\varphi_k = (0, \dotsc, 0, 1, 0, \dotsc)$ eine Basis (sogar eine ONB).

\emph{Beispiel}:
Für $H = L^2([-\pi, \pi], dx)$ ist $\{\varphi_k\}_{k \in \integer}$ mit
$\varphi_k(x) = \frac{1}{\sqrt{2\pi}} e^{\i kx}$ ein ONS.\\
Frage: Ist dies auch eine Basis?

\linie
\pagebreak

Dafür erweitert man die Orthogonalität $\bot$ auf Mengen, d.\,h.
für $f \in H$ und $M \subset H$ soll $f \orth M$ gelten, falls
$\forall_{g \in M}\; f \orth g$.
Außerdem bezeichnet im Folgenden $\bigvee M$ die Menge aller endlichen
Linearkombinationen von $M$.
Aus $f \orth M$ folgt $f \orth \bigvee M$.

\textbf{Projektion auf Unterraum}:\\
Seien $(H, \innerproduct{\cdot, \cdot})$ ein Hilbertraum,
$\{\varphi_n\}_{n \in \natural}$ ein ONS und
$L = \bigvee \{\varphi_1, \dotsc, \varphi_N\}$ mit $N \in \natural$.\\
Dann ist $P_L\colon H \rightarrow H$,
$P_L x := \sum_{k=1}^N \innerproduct{x, \varphi_k} \varphi_k$
die \emph{Projektion auf den Unterraum $L$}.

\textbf{Eigenschaf"|ten}:
\begin{enumerate}
    \item
    $P_L$ ist ein linearer Operator (d.\,h. ein Endomorphismus)

    \item
    $P_L(H) = L$ (d.\,h. $P_L\colon H \rightarrow L$)

    \item
    $x - P_L x =: h \orth L$,
    denn für $f = \beta_1 \varphi_1 + \dotsb + \beta_N \varphi_N \in L$ gilt\\
    $\innerproduct{h, f} = \innerproduct{x, f} - \innerproduct{P_L x, f} =
    \sum_{k=1}^N \innerproduct{x, \varphi_k} \overline{\beta_k} -
    \sum_{k=1}^N \sum_{j=1}^N \innerproduct{\innerproduct{x, \varphi_j} \varphi_j, \varphi_k}
    \overline{\beta_k} = 0$

    \item
    $\norm{x}^2 = \norm{P_L x}^2 + \norm{h}^2 \ge \norm{P_L x}^2$
    (da $P_L x \in L$ und $h \orth L$), d.\,h.
    $P_L$ ist beschränkt

    \item
    $\norm{P_L x} = \norm{x} \iff \norm{h} = 0 \iff P_L x = x \iff x \in L$\\
    (für $\norm{h} = 0$ gilt $h = x - P_L x = 0$, also $x = P_L x \in L$,\\
    für $x \in L$ ist $h \in L$, da $P_L x$ in $L$,
    aus $h \orth L$ folgt $h \orth h$, also $\norm{h}^2 = 0$)

    \item
    $P_L (P_L x) = P_L x$
    (da $P_L x \in L$ und $P_L y = y$ für $y = P_L x \in L$)
\end{enumerate}

\linie

\textbf{Folgerung}:
Für $x \in H$ und $f \in L$ gilt $\norm{x - f} \ge \norm{x - P_L x}$.

Anschaulich besagt die Folgerung, dass der Abstand von $x$ zur
senkrechten Projektion von $x$ auf $L$ am kürzesten ist.

Also besitzt das Problem $\{f \in L \;|\;
\norm{x - f} = \min_{y \in L} \norm{x - y}\}$ genau eine Lösung
$f = P_L x$.

\textbf{Satz (\name{Bessel}sche Ungleichung)}:
Seien $(H, \innerproduct{\cdot, \cdot})$ ein Hilbertraum,
$\{\varphi_k\}_{k \in \natural}$ ein ONS,
$x \in H$ und $c_k := \innerproduct{x, \varphi_k}$ die Fourierkoef"|fizienten.\\
Dann gilt $\sum_{k=1}^\infty |c_k|^2 \le \norm{x}^2$.

\textbf{Satz}:
Es gilt $\sum_{k=1}^\infty |c_k|^2 = \norm{x}^2$ genau dann, wenn
$\sum_{k=1}^N c_k \varphi_k \xrightarrow{\norm{\cdot}} x$.

\textbf{abgeschlossen}:
Seien $(H, \innerproduct{\cdot, \cdot})$ ein Hilbertraum.
Ein ONS $\{\varphi_k\}_{k \in \natural}$ heißt \emph{abgeschlossen}, falls
für alle $x \in H$ die \emph{Gleichung von \name{Parseval}} gilt, d.\,h.
$\forall_{x \in H}\; \sum_{k=1}^\infty |c_k|^2 = \norm{x}^2$\\
(das ist nach dem vorherigen Satz äquivalent zu
$\forall_{x \in H}\; \sum_{k=1}^\infty c_k \varphi_k = x$).

Die Gleichung von Parseval ist eine unendliche Verallgemeinerung des
Satzes des Pythagoras.

\textbf{total}:
Seien $(H, \innerproduct{\cdot, \cdot})$ ein Hilbertraum.
Ein ONS $\{\varphi_k\}_{k \in \natural}$ heißt \emph{total}, falls\\
$\forall_{x \in H}\; (\forall_{k \in \natural}\;
c_k = \innerproduct{x, \varphi_k} = 0) \;\Rightarrow\; (x = 0)$.

\textbf{Satz}:
Seien $(H, \innerproduct{\cdot, \cdot})$ ein Hilbertraum und
$\{\varphi_k\}_{k \in \natural}$ ein ONS.\\
Dann ist das ONS abgeschlossen $\iff$ vollständig $\iff$ total.

\linie
\pagebreak

\textbf{Zuordnung von Vektoren und Folgen}:\\
Seien $(H, \innerproduct{\cdot, \cdot})$ ein Hilbertraum und
$\{\varphi_n\}_{n \in \natural}$ ein ONS.\\
Definiere eine Abbildung $\phi\colon H \rightarrow \ell^2(\natural)$,
$x \mapsto \{c_k\}_{k \in \natural}$ mit $c_k = \innerproduct{x, \varphi_k}$.

\textbf{Eigenschaf"|ten}:
\begin{enumerate}
    \item
    $\phi\colon H \rightarrow \ell^2(\natural)$ ist linear

    \item
    $\norm{\phi x}_{\ell^2(\natural)}^2 =
    \sum_{k=1}^\infty |c_k|^2 \le \norm{x}^2$,
    d.\,h. $\norm{\phi}_{\L(H, \ell^2(\natural))} \le 1$.\\
    Es gilt sogar $\norm{\phi}_{\L(H, \ell^2(\natural))} = 1$
    (wähle $x = \varphi_k$ für ein $k \in \natural$).

    \item
    $\phi\colon H \rightarrow \ell^2(\natural)$ ist surjektiv, denn:\\
    Sei $\{c_k\}_{k \in \natural} \in \ell^2(\natural)$ gegeben.
    Für $S_N = \sum_{k=1}^N c_k \varphi_k$ ergibt sich dann\\
    $\norm{S_M - S_N}^2 = \norm{\sum_{k=N+1}^M c_k \varphi_k}^2 \le
    \sum_{k=N+1}^M |c_k|^2 < \varepsilon$
    (da die Reihe $\sum_{k=1}^\infty |c_k|^2$ konvergiert, d.\,h. die
    Partialsummen bilden eine Cauchy-Folge).
    Also ist $\{S_N\}_{N \in \natural}$ eine Cauchy-Folge und wegen der
    Vollständigkeit von $(H, \innerproduct{\cdot, \cdot})$ existiert ein $x \in H$ mit
    $\sum_{k=1}^\infty c_k \varphi_k = x$.\\
    Es gilt $\innerproduct{x, \varphi_k} =
    \lim_{N \to \infty} \innerproduct{S_N, \varphi_k} = c_k$, also
    $\phi x = \{c_k\}_{k \in \natural}$.

    \item
    Falls $\{\varphi_n\}_{n \in \natural}$ eine ONB ist, so ist
    $\phi\colon H \rightarrow \ell^2(\natural)$ injektiv, denn dann gilt\\
    $\norm{x}^2 = \sum_{k=1}^\infty |c_k|^2$, d.\,h.
    $\Kern \phi = \{0\}$.
\end{enumerate}

Für einen Hilbertraum und eine Orthonormalbasis erhält man also eine
1:1-Beziehung (Bijektion) zwischen den Vektoren und den Folgen der
Fourierkoef"|fizienten.

\section{%
    Delta-Folgen%
}

Seien $a < 0 < b$ und $g_n \in L^1([a, b], dx)$ für $n \in \natural$.

\textbf{Delta-Folge}:
$\{g_n\}_{n \in \natural}$ heißt \emph{Delta-Folge}, falls
\begin{enumerate}
    \item
    $\forall_{n \in \natural}\; g_n \ge 0$,

    \item
    $\int_{[a, b]} g_n(t)\dt \xrightarrow{n \to \infty} 1$ und

    \item
    $\forall_{\delta > 0} \forall_{\varepsilon > 0}
    \exists_{N(\varepsilon, \delta) \in \natural}
    \forall_{n \ge N(\varepsilon, \delta)}\;
    \left(\int_{[a, -\delta]} + \int_{[\delta, b]}\right)
    g_n(t) \dt < \varepsilon$.
\end{enumerate}

\linie

\textbf{Satz}:
Seien $f \in \C([a, b], \complex)$ und $\{g_n\}_{n \in \natural}$
eine Delta-Folge.\\
Dann gilt $\int_{[a, b]} f(t)g_n(t) \dt \xrightarrow{n \to \infty} f(0)$.

\linie

\emph{Beispiel}:
Seien $[a, b] = [-\pi, \pi]$ und $\phi_n(x) :=
\frac{1}{2\pi n} \left(\frac{\sin(nx/2)}{\sin(x/2)}\right)^2$.
Dann ist $\{\phi_n\}_{n \in \natural}$ eine Delta-Folge:
\begin{enumerate}
    \item
    $\phi_n \ge 0$

    \item
    Mit $\D_k(x) = \frac{1}{2\pi} \frac{\sin((k + 1/2)x)}{\sin(x/2)}$
    gilt $\int_{-\pi}^\pi \D_k(x) \dx = 1$ und
    $\phi_n(x) = \frac{1}{n} \sum_{k=0}^{n-1} \D_k(x)$ (s.\,u.).\\
    Daraus folgt $\int_{-\pi}^\pi \phi_n(x)\dx =
    \frac{1}{n} \sum_{k=0}^{n-1} \int_{-\pi}^\pi \D_k(x) \dx = 1$
    für alle $n \in \natural$.

    \item
    Für $\delta > 0$ und $\varepsilon > 0$ beliebig gilt
    $\left(\int_{-\pi}^{-\delta} + \int_\delta^\pi\right) \phi_n(x)\dx \le
    \left(\int_{-\pi}^{-\delta} + \int_\delta^\pi\right)
    \frac{1}{2\pi n} \frac{1}{\sin^2(|\delta|/2)} \dx$\\
    $\le 2\pi \cdot \frac{1}{2\pi n} \frac{1}{\sin^2(|\delta|/2)} =
    \frac{1}{n \sin^2(\delta/2)} < \varepsilon$
    für $n \ge N(\varepsilon, \delta) :=
    \frac{1}{\varepsilon \sin^2(\delta/2)}$.
\end{enumerate}

\linie

Begründung für $\phi_n(x) = \frac{1}{n} \sum_{k=0}^{n-1} \D_k(x)$:
$\phi_n(x) = \frac{1}{2\pi n} \left(\frac{\sin(nx/2)}{\sin(x/2)}\right)^2 =
\frac{\sin(nx/2)}{2\pi n \sin^2(x/2)} \Im(e^{\i nx/2})$\\
$= \frac{1}{2\pi n \sin(x/2)}
\Im\left(\frac{\sin(nx/2)}{\sin(x/2)} e^{\i nx/2}\right) =
\frac{1}{2\pi n \sin(x/2)}
\Im\left(\frac{e^{\i nx} - 1}{e^{\i x/2} - e^{-\i x/2}}\right) =
\frac{1}{2\pi n \sin(x/2)} \Im(e^{\i x/2} \cdot \sum_{k=0}^{n-1} e^{\i kx})$\\
$= \frac{1}{2\pi n \sin(x/2)} \Im(\sum_{k=0}^{n-1} e^{\i (k + 1/2)x}) =
\frac{1}{n} \sum_{k=0}^{n-1} \frac{1}{2\pi}
\frac{\sin((k + 1/2)x)}{\sin(x/2)} =
\frac{1}{n} \sum_{k=0}^{n-1} \D_k(x)$.

\pagebreak

\section{%
    Der Satz von \name{Fejer}%
}

Im Folgenden bezeichnet
$\C_p([-\pi, \pi], \complex) :=
\{f \in \C([-\pi, \pi], \complex) \;|\; f(-\pi) = f(\pi)\}$
den Raum der stetigen Funktionen auf $[-\pi, \pi]$, die $2\pi$-periodisch sind.

\textbf{Satz von \name{Fejer}}:
Sei $f \in \C_p([-\pi, \pi], \complex)$.\\
Dann gilt $\sigma_N(x) \xrightarrow{N \to \infty} f(x)$ gleichmäßig,
wobei $\sigma_N(x) := \frac{1}{N} \sum_{n=0}^{N-1} S_n(x)$ das arithmetische
Mittel der ersten $N$ Fourier-Partialsummen
$S_n(x) = \sum_{k=-n}^n c_k e^{\i kx}$,
$c_k = \frac{1}{2\pi} \int_{-\pi}^\pi f(x) e^{-\i kx} \dx$ ist.

\linie

\textbf{Folgerung}:
\begin{enumerate}
    \item
    $f \in \C_p([-\pi, \pi], \complex)$ ist durch die Fourierkoef"|fizienten
    eindeutig bestimmt, denn falls\\
    $f, \widetilde{f} \in \C_p([-\pi, \pi], \complex)$
    die gleichen Fourierkoef"|fizienten
    $\{c_k\}_{k \in \integer}$ besitzen, konvergiert jeweils
    $\sigma_N(x)$ bzw. $\widetilde{\sigma}_N(x)$ gegen $f(x)$ bzw. $g(x)$.
    Aufgrund der gleichen Fourierkoef"|fizienten gilt jedoch
    $\sigma_N(x) = \widetilde{\sigma}_N(x)$ für alle $N \in \natural$,
    aus der Eindeutigkeit des Grenzwerts folgt dann $f(x) \equiv g(x)$
    für $x \in [-\pi, \pi]$.

    \item
    $\{e^{\i nx}\}_{n \in \integer}$ ist ein vollständiges System in
    $L^2([-\pi, \pi], dx)$.
    Dies lässt sich aus der Dichtheit von $\C_p([-\pi, \pi], \complex)$ in
    $L^2([-\pi, \pi], dx)$ folgern (Funktionalanalysis):
    Für eine gegebene Funktion $f \in L^2([-\pi, \pi], dx)$
    gibt es eine Funktion $f_\varepsilon \in \C_p([-\pi, \pi], \complex)$
    mit $\norm{f - f_\varepsilon}_{L^2} < \varepsilon$.\\
    Nach dem Satz von Fejer gibt es ein $N(\varepsilon)$ mit
    $|f_\varepsilon(t) - \sigma_{N(\varepsilon)}(t)| < \varepsilon$ für alle
    $t \in [-\pi, \pi]$.\\
    Daraus folgt
    $\norm{f_\varepsilon - \sigma_{N(\varepsilon)}}_{L^2}^2 =
    \int_{-\pi}^\pi |f_\varepsilon(t) - \sigma_{N(\varepsilon)}(t)|^2 \dt \le
    \varepsilon^2 \cdot 2\pi$ bzw.\\
    $\norm{f - \sigma_{N(\varepsilon)}}_{L^2} \le
    \norm{f - f_\varepsilon}_{L^2} +
    \norm{f_\varepsilon - \sigma_{N(\varepsilon)}}_{L^2} \le
    \varepsilon + \varepsilon \sqrt{2\pi} < \widetilde{\varepsilon}$.\\
    Dabei ist $\sigma_{N(\varepsilon)}(t) =
    \frac{1}{N(\varepsilon)} \sum_{\ell=0}^{N(\varepsilon)-1}
    \left(\sum_{k=-\ell}^\ell c_k(f_\varepsilon) e^{\i kt}\right)$
    eine Linearkombination von\\
    $\{e^{\i kt} \;|\; k = -(N-1),\; \dotsc,\; (N-1)\}$, d.\,h.
    $\{e^{\i nx}\}_{n \in \integer}$ ist vollständig.
\end{enumerate}

\linie

\textbf{\name{Gibbs}-Ef"|fekt}:
Dieser tritt bei der punktweisen Approximation eines Signals mit
Sprungstellen durch die Fouriersumme $S_N(t)$ auf.
Auch wenn $N$ groß gewählt wird, verbleibt immer ein Überschwinger von ca.
$9\,\%$ der Sprunghöhe vor und nach dem Sprung.
Dieser Ef"|fekt heißt \emph{\name{Gibbs}-Ef"|fekt} und kann durch die
Approximation durch die Mittelwerte $\sigma_N(t)$ vermieden werden.
Diese ist zwar schlechter in der $L^2$-Norm, aber dafür konvergiert sie
gleichmäßig, d.\,h. solche Überschwinger können nicht auf"|treten.
Dies liegt daran, dass in $\sigma_N(t) = \sum_{k=-N}^N \alpha_k(N) e^{\i kt}$
die Gewichte $\alpha_k(N)$ für jedes $N$ unterschiedlich sind.

\linie

\textbf{Zusammenfassung zur Konvergenz von \name{Fourier}-Reihen}:
\begin{enumerate}
    \item
    Für $f \in L^2([-\pi, \pi], dx)$ ist $\{\varphi_n\}_{n \in \integer}$
    mit $\varphi_n(t) = \frac{1}{\sqrt{2\pi}} e^{\i nt}$ eine ONB.
    Für die Fourierkoeff. $\gamma_k = \innerproduct{f, \varphi_k}_{L^2} =
    \frac{1}{\sqrt{2\pi}} \int_{[-\pi, \pi]} f(t) e^{-\i kt}\dt$ gilt
    $\sum_{k \in \integer} |\gamma_k|^2 = \norm{f}_{L^2}^2 =
    \int_{[-\pi, \pi]} |f(t)|^2 \dt$.\\
    Außerdem konvergiert $f(x) = \sum_{k \in \integer}
    \frac{\gamma_k}{\sqrt{2\pi}} e^{\i kx}$ absolut im $L^2$.
    Es gilt:
    Für $f \in L^2([-\pi, \pi], dx)$ konvergiert
    $S_N(t) \xrightarrow{(\cdot)} f(t)$ punktweise Lebesgue-fast-überall
    (\textbf{Satz von \name{Carleson}}).

    \item
    Für $f \in \C_p([-\pi, \pi], \complex)$ lassen sich wegen
    $\C_p \subset L^2$ die gleichen Schlussfolgerungen ziehen.
    Im Allgemeinen weiß man zwar nicht, wo die Lebesgue-Nullmenge liegt,
    auf der die Fourier-Reihe nicht konvergiert.
    Allerdings lässt sich hier der Satz von Fejer anwenden, der eine
    gleichmäßige Konvergenz von $\sigma_N$ gibt
    (d.\,h. $\sigma_N \xrightarrow{\norm{\cdot}_{\C_p}} f$).
    Wichtig ist, dass für den Satz gebraucht wurde, dass $\phi_n$ eine
    Delta-Folge ist -- mit $\D_k$ geht das nicht ($\not\ge 0$).
    Konvergenz von $S_N(t)$ lässt sich somit nur über die Dini-Bedingung mit
    zusätzlichen Voraussetzungen beweisen (Stetigkeit reicht nicht aus).

    \item
    Für $f \in L^1([-\pi, \pi], dx)$ existieren zwar die
    Fourier-Koef"|f.
    $c_k = \frac{1}{2\pi} \int_{[-\pi, \pi]} f(t) e^{-\i kt} \dt$,\\
    aber wegen $L^1 \not\subset L^2$ lässt sich die $L^2$-Theorie nicht
    verallgemeinern.\\
    Es gibt aber einen \textbf{$L^1$-Satz von \name{Fejer}}
    (für $f \in L^1([-\pi, \pi], dx)$ gilt
    $\sigma_N \xrightarrow{\norm{\cdot}_{L^1}} f$).
\end{enumerate}

\section{%
    Wichtige Eigenschaften der \name{Fourier}-Transformation%
}

Für $f \in L^1(\real, dx)$ und $\lambda \in \real$ ist die
\emph{\name{Fourier}transformierte} definiert als\\
$\widehat{f}(\lambda) = \F[f](\lambda) =
\frac{1}{\sqrt{2\pi}} \int_\real f(x) e^{-\i\lambda x}\dx$.\\
$\F[f]$ ist stetig auf $\real$ und es gilt $\F[f](\lambda) \to 0$ für
$\lambda \to \pm\infty$.
Falls $f$ in $t = t_0$ die Dini-Bedingung erfüllt, dann gilt
$f(t_0) = \frac{1}{\sqrt{2\pi}}
\int_{-\infty}^{+\infty} \F[f](\lambda) e^{\i\lambda t_0} d\lambda$.
Man schreibt auch $f(t_0) = \F^{-1}[\F[f]](t_0)$.\\
Wegen $|\F[f](\lambda)| \le \frac{1}{\sqrt{2\pi}} \cdot \norm{f}_{L^1}$ ist
$\F\colon L^1(\real, dx) \rightarrow \C(\real, \complex)$ ein linearer und
stetiger Operator.

\linie

\textbf{Lemma}:
Sei $f(t), t f(t), \dotsc, t^n f(t) \in L^1(\real, dx)$.\\
Dann ist $\widehat{f} = \F[f]$ $n$-mal stetig dif"|ferenzierbar und es gilt
$\F^{(k)}[f](\lambda) = \F[(-\i t)^k f(t)](\lambda)$,\\
insbesondere gilt $\F^{(k)}[f](\lambda) \to 0$ für $\lambda \to \pm\infty$ und
$k = 0, \dotsc, n$.

\textbf{Lemma}:
Seien $f$ $n$-fach stetig dif"|ferenzierbar und
$f, f', \dotsc, f^{(n)} \in L^1(\real, dx)$.\\
Dann gilt $\F[f^{(k)}](\lambda) = (\i\lambda)^k \F[f](\lambda)$,
insbesondere gilt $\F[f](\lambda) = o(|\lambda|^{-n})$ für
$\lambda \to \pm\infty$.

\linie

\textbf{\name{Schwartz}sche Funktionenklasse $\S(\real)$}:
Für $f \colon \real \rightarrow \complex$ beliebig oft dif"|ferenzierbar sei\\
$f \in \S(\real)$, falls
$\forall_{p, q \in \natural_0} \exists_{C(p, q) < \infty}
\forall_{x \in \real}\; |x^p f^{(q)}(x)| \le C(p, q)$.\\
$\S(\real)$ heißt \emph{\name{Schwartz}sche Funktionenklasse} und wird zur
Betonung der Variablen auch manchmal als $\S_t(\real)$ geschrieben.

\textbf{\name{Fourier}-Transformation als Bijektion zwischen
$\S(\real)$-Räumen}:\\
Für $f \in \S_t(\real)$ gilt
$\F[f] \in \S_\lambda(\real)$ und
$\F\colon \S_t(\real) \rightarrow \S_\lambda(\real)$ ist eine Bijektion.

\linie

\textbf{Faltung}:
Seien $f, g \in L^1(\real, dx)$.\\
Dann ist die Faltung $f \ast g\colon \real \rightarrow \complex$ definiert als
$(f \ast g)(t) := \int_\real f(\tau) g(t - \tau) \d\tau$.

\textbf{In welchem Sinn existiert $f \ast g$?}\\
Der \emph{Satz von \name{Fubini}} lässt aus
$h(t, \tau) \in L^1(\real^2, d(t, \tau))$ folgern, dass\\
$\int_{\real^2} h(t, \tau) d(t, \tau) =
\int_\real \left(\int_\real h(t, \tau) \dt\right) \d\tau =
\int_\real \left(\int_\real h(t, \tau) \d\tau\right) \dt$.\\
Die Funktionen in Klammern existieren jeweils fast überall und sind
Lebesgue-integrierbar.\\
Für $h(t, \tau) = f(\tau) g(t - \tau)$ folgt, dass
$\norm{f \ast g}_{L^1} =
\int_\real |(f \ast g)(t)| \dt =
\int_\real |\int_\real f(\tau) g(t - \tau) \d\tau| \dt$\\
$\le \int_\real \left(\int_\real |f(\tau)| |g(t - \tau)| \d\tau\right) \dt =
\int_\real |f(\tau)| \left(\int_\real |g(t - \tau)| \dt\right) \d\tau =
\int_\real |f(\tau)| \d\tau \cdot \norm{g}_{L^1}$\\
$= \norm{f}_{L^1} \cdot \norm{g}_{L^1}$,
daraus folgt die \textbf{Ungleichung von \name{Hausdorff}-\name{Young}}\\
$\norm{f \ast g}_{L^1} \le \norm{f}_{L^1} \cdot \norm{g}_{L^1}$
und die Faltung ist eine Lebesgue-integrierbare Funktion, die in $t$ bis auf
eine Lebesgue-Nullmenge existiert.

\textbf{Lemma}:
Für $f, g \in L^1(\real, dx)$ gilt
$\frac{1}{\sqrt{2\pi}} \F[f \ast g](\lambda) =
\F[f](\lambda) \cdot \F[g](\lambda)$.

\linie
\pagebreak

\textbf{\name{Fourier}-Transformation im $L^2(\real, dx)$}:\\
Für beschränkte Intervalle ist $L^2 \subset L^1$, da zum Beispiel\\
$\norm{f}_{L^1} = \int_{[-\pi, \pi]} |f| \cdot 1 \dx \le
\left(\int_{[-\pi, \pi]} |f|^2 \dx\right)^{1/2} \cdot
\left(\int_{[-\pi, \pi]} 1^2 \dx\right)^{1/2} =
\sqrt{2\pi} \norm{f}_{L^2} < \infty$\\
aufgrund der \emph{\name{Hölder}schen Ungleichung},\\
es gilt also $f \in L^2([-\pi, \pi], dx) \;\Rightarrow\;
f \in L^1([-\pi, \pi], dx)$.

Allerdings gilt im Allgemeinen $f \in L^2(\real, dx) \;\not\Rightarrow\;
f \in L^1(\real, dx)$!
Somit kann das Fourier-Integral evtl. nicht definiert sein.

Im Folgenden nutzt man aus, dass man zeigen kann, dass
$\S(\real)$ dicht in $L^2(\real, dx)$ ist.\\
Für $f, g \in \S(\real)$ ist
$\F[f], \F[g] \in \S(\real) \subset L^2(\real, dx)$.

\textbf{Satz von \name{Plancherel} für $\S(\real)$}:
Für $f, g \in \S(\real)$ gilt
$\innerproduct{\F[f], \F[g]}_{L^2} = \innerproduct{f, g}_{L^2}$,\\
d.\,h. $\F\colon \S_t(\real) \rightarrow \S_\lambda(\real)$ ist ein
\emph{unitärer Operator}.
Insbesondere gilt $\norm{\F[f]}_{L^2} = \norm{f}_{L^2}$.

\linie

\textbf{Herleitung, Existenz}:
Um nun die Fourier-Transformation für eine Funktion $f \in L^2(\real, dx)$ zu
bestimmen, nutzt man die Existenz einer Folge $\{f_n\}_{n \in \natural}$,
$f_n \in \S(\real)$ mit $f_n \xrightarrow{L^2} f$
aus.
Für die Fourier-Transformationen der Folgenglieder
$g_n := \F[f_n]$ gilt aufgrund des Satzes von Plancherel
$\norm{g_n - g_m}_{L^2} = \norm{\F[f_n] - \F[F_m]}_{L^2} =
\norm{\F[f_n - f_m]}_{L^2} = \norm{f_n - f_m}_{L^2} < \varepsilon$,
da $\{f_n\}_{n \in \natural}$ eine Cauchy-Folge im $L^2$ ist.
Also ist auch $\{\F[f_n]\}_{n \in \natural}$ eine Cauchy-Folge im $L^2$ und
aufgrund der Vollständigkeit von $L^2(\real, dx)$ gibt es ein
$g \in L^2(\real, dx)$ mit $g_n \xrightarrow{L^2} g$.
Dieses $g$ wird als Fourier-Transformation von $f$ definiert.

\textbf{\name{Fourier}-Transformation für $f \in L^2(\real, dx)$}:
Sei $f \in L^2(\real, dx)$.\\
Dann ist $\F[f]$ definiert als
$\F[f] \overset{L^2}{:=} \lim_{n \to \infty} \F[f_n]$
für $f_n \in \S(\real)$ mit $f_n \xrightarrow{L^2} f$.

\textbf{Eindeutigkeit}:
Die Definition könnte evtl. nicht eindeutig sein, da die $f_n$ nicht eindeutig
sein müssen.
Für $f_n, \widetilde{f}_n \in \S(\real)$ mit
$f_n, \widetilde{f}_n \xrightarrow{L^2} f$ gilt
mit $g_n = \F[f_n]$ und $\widetilde{g}_n = \F[\widetilde{f}_n]$, dass
$\norm{g_n - \widetilde{g}_n}_{L^2} = \norm{\F[f_n - \widetilde{f}_n]}_{L^2} =
\norm{f_n - \widetilde{f}_n}_{L^2} < \varepsilon$.
Somit müssen die $g_n$ und $\widetilde{g}_n$ gegen den gleichen Grenzwert
konvergieren.
Daraus folgt die Eindeutigkeit von $\F[f]$.

\textbf{Abschluss eines Operators}:
Diese Vorgehensweise der Verallgemeinerung eines Operators und anschließender
Verifikation der gewünschten Eigenschaften wird öfters angewandt und heißt
\emph{Abschluss eines Operators}.
Allgemein gibt es für einen Hilbertraum $H$, eine dichte Teilmenge
$D \subset H$ und einen linearen und beschränkten Operator
$T\colon D \rightarrow H$ eine lineare und beschränkte Fortsetzung
$\widetilde{T}\colon H \rightarrow H$ mit $\widetilde{T}|_D = T$.

\linie

Für zwei Funktionen $f, g \in L^2(\real, dx)$ lässt sich der Satz
von Plancherel verallgemeinern:
Ist $f_n, g_n \in \S(\real)$ mit $f_n \xrightarrow{L^2} f$ und
$g_n \xrightarrow{L^2} g$,
so gilt einerseits $\innerproduct{\F[f_n], \F[g_n]}_{L^2} = \innerproduct{f_n, g_n}_{L^2} \to
\innerproduct{f, g}_{L^2}$
aufgrund des Satzes von Plancherel für $\S(\real)$ und der Stetigkeit des
Skalarprodukts, andererseits gilt aber
$\innerproduct{\F[f_n], \F[g_n]}_{L^2} \to \innerproduct{\F[f], \F[g]}_{L^2}$
aufgrund der Stetigkeit des Skalarprodukts und der Defintion von $\F[f]$ bzw.
$\F[g]$.
Also gilt $\innerproduct{\F[f], \F[g]}_{L^2} = \innerproduct{f, g}_{L^2}$.

\textbf{Satz von \name{Plancherel} für $L^2(\real, dx)$}:
Für $f, g \in L^2(\real, dx)$ gilt
$\innerproduct{\F[f], \F[g]}_{L^2} = \innerproduct{f, g}_{L^2}$,\\
d.\,h. $\F\colon L^2(\real, dx) \rightarrow L^2(\real, dx)$ ist ein
\emph{unitärer Operator}.
Insbesondere gilt $\norm{\F[f]}_{L^2} = \norm{f}_{L^2}$.

\linie

\textbf{\name{Fourier}-Transformation als Bijektion zwischen
$L^2(\real, dx)$-Räumen}:\\
$\F\colon L^2(\real, dx) \rightarrow L^2(\real, dx)$ ist eine Bijektion.

\linie
\pagebreak

\textbf{\name{Fourier}-Transformation im $\real^d$}:
Sei $d \in \natural$.

\textbf{Multiindex}:
Man bezeichnet Elemente
$\alpha = (\alpha_1, \dotsc, \alpha_d) \in \natural_0^d$
als \emph{Multiindex}.\\
$|\alpha| := \sum_{j=1}^d \alpha_j$ heißt die \emph{Ordnung} von $\alpha$.\\
Für einen Vektor $\xi = (\xi_1, \dotsc, \xi_d)$ und einen Multiindex
$\alpha$ schreibt man $\xi^\alpha := \xi_1^{\alpha_1} \dotsm \xi_d^{\alpha_d}$.

\textbf{mehrfache partielle Ableitungen}:\\
Falls die Ableitungen vertauscht werden können, schreibt man
$\frac{\partial^{|\alpha|}}{\partial x^\alpha} = \partial^\alpha :=
\frac{\partial^{|\alpha|}}{\partial x_1^{\alpha_1} \dotsm
\partial x_d^{\alpha_d}}$.

\textbf{\name{Schwartz}sche Funktionenklasse $\S(\real^d)$}:
Für $f \colon \real^d \rightarrow \complex$ beliebig oft dif"|ferenzierbar
sei\\
$f \in \S(\real^d)$, falls
$\forall_{m, n \in \natural_0} \exists_{C(m, n) < \infty}
\forall_{\alpha, \beta \in \natural_0^d,\; |\alpha| \le n,\; |\beta| \le m}
\forall_{x \in \real}\; |x^\alpha \partial^\beta f(x)| \le C(n, m)$.\\
$\S(\real^d)$ heißt \emph{\name{Schwartz}sche Funktionenklasse}.

\linie

\textbf{\name{Fourier}-Transformation im $\real^d$}:\\
Für $f \in \S(\real^d)$ sei $\F[f](\xi) := \frac{1}{(2\pi)^{d/2}}
\int_{\real^d} f(x) e^{-\i\innerproduct{x, \xi}} \dx$ für $\xi \in \real^d$.

Wegen $\innerproduct{x, \xi}_{\real^d} = x_1 \xi_1 + \dotsb + x_d \xi_d$ gilt
$e^{-\i\innerproduct{x, \xi}} = e^{-\i x_1 \xi_1} \dotsm e^{-i x_d \xi_d}$.
Nach dem Satz von Fubini gilt daher
$\F[f](\xi) = \frac{1}{\sqrt{2\pi}} \int_\real e^{-\i x_d \xi_d}
\Big(\frac{1}{\sqrt{2\pi}} \int_\real e^{-\i x_{d-1} \xi_{d-1}}
\Big(\dotsb$\\
$\Big(\frac{1}{\sqrt{2\pi}} \int_\real e^{-\i x_1 \xi_1}
f(x_1, \dotsc, x_d) \dx_1\Big)\dotsb\Big)\dx_{d-1}\Big)\dx_d$, d.\,h.\\
$\F_{x \to \xi}[f] = \F_{x_d \to \xi_d}[\F_{x_{d-1} \to \xi_{d-1}}[\dotsb[
\F_{x_1 \to \xi_1}[f]]\dotsb]]$
gilt für $f \in L^1(\real^d, dx)$.

\textbf{Satz}:
Für $f \in \S(\real^d)$ ist $\F[f] \in \S(\real^d)$.

\textbf{wichtige Formeln}:
Für $f \in \S(\real^d)$ gilt
$\frac{\partial^{|\alpha|}}{\partial \xi^\alpha} \F[f](\xi) =
(-\i)^{|\alpha|} \F[x^\alpha f(x)](\xi)$ sowie\\
$\F[\frac{\partial^{|\beta|}}{\partial x^\beta} f(x)](\xi) =
\i^{|\beta|} \xi^\beta \F[f](\xi)$.
Außerdem gilt
$\norm{\F[f]}_\C \le \frac{1}{(2\pi)^{d/2}} \norm{f}_{L^1}$,
also $\F\colon L^1(\real^d, dx) \rightarrow \C$.

\textbf{Satz}:
$\F\colon \S(\real^d) \rightarrow \S(\real^d)$ ist eine Bijektion.

\textbf{Satz}:
Für $f, g \in \S(\real^d)$ gilt
$\innerproduct{\F[f], \F[g]}_{L^2(\real^d, dx)} = \innerproduct{f, g}_{L^2(\real^d, dx)}$.

Man kann analog wie eben $\F$ zu
$\F\colon L^2(\real^d, dx) \rightarrow L^2(\real^d, dx)$ erweitern.
Die Formel von Plancherel gilt dann für alle $f, g \in L^2(\real^d, dx)$.

\textbf{Satz}:
$\F\colon L^2(\real^d, dx) \rightarrow L^2(\real^d, dx)$ ist eine Bijektion.

\pagebreak
