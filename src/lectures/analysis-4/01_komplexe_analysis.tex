\chapter{%
    Funktionen in einer komplexen Variablen%
}

\section{%
    Die Topologie der erw. kompl. Zahlenebene und die
    \name{Möbius}-Transf.%
}

Komplexe Zahlen lassen sich mittels $z = (x, y) = x + \i y \in \complex$,
$x, y \in \real$
als Element der komplexen Zahlenebene schreiben.
Dabei ist $\i = (0, 1)$ mit $\i^2 = -1 = (-1, 0)$.
Die zu $z$ komplex konjugierte Zahl ist $\overline{z} = x - \i y$ und
der Betrag von $z$ ist $|z| = \sqrt{x^2 + y^2} = \sqrt{z \overline{z}} \ge 0$.
Dieser definiert die Abstandsfunktion $d(z_1, z_2) = |z_1 - z_2|$.
Damit ist Konvergenz in $\complex$ definiert, da die Abstandsfunktion
$\varepsilon$-Umgebungen definiert durch
$U_\varepsilon(z) = \{w \in \complex \;|\; |w - z| < \varepsilon\}$.

\linie

\textbf{Wie versteht man "`$\infty$"' im komplexen Fall?}

Im Reellen sagt man, eine Folge $z_n \in \real$ läuft gegen $+\infty$,
falls $z_n > R$ für $n \ge N_R$ und jedes beliebige $R> 0$, d.\,h.
$z_n \in U_R(+\infty) := \left]R, +\infty\right[$.
Analog gilt $z_n \to -\infty$,
falls $z_n < -R$ für $n \ge N_R$ und jedes beliebige $R > 0$, d.\,h.
$z_n \in U_R(-\infty) := \left]-\infty, -R\right[$.\\
Allgemein schreibt man $z_n \to \infty$,
falls $|z_n| > R$ für $n \ge N_R$ und jedes beliebige $R > 0$, d.\,h.
$z_n \in U_R(\infty) := U_R(+\infty) \cup U_R(-\infty)$.
Geometrisch kann man $U_R(\infty)$ durch die \emph{stereographische Projektion}
als "`Umgebung von $\infty$"' interpretieren:\\
Zeichnet man einen Kreis auf die reelle Achse, der diese in $0$ berührt, so
kann man jeder reellen Zahl einen Punkt auf dem Kreis zuweisen,
indem man die reelle Zahl mit dem "`Nordpol"'
(der der $0$ gegenüber liegende Punkt) verbindet und der reellen Zahl
den Schnittpunkt der Verbindungsstrecke mit dem Kreis zuweist.
Auf diese Weise entsprechen Umgebungen einer reellen Zahl $z \in \real$
wieder Umgebungen des Bildes von $z$ auf dem Kreis.
Den "`Nordpol"' kann man als
"`$\infty$"' bezeichnen, da das Bild von $U_R(\infty)$ eine Umgebung von
$\infty$ darstellt (nur ohne $\infty$).

Im Komplexen ist dies nicht ganz so einfach, da es dort viele verschiedene
Richtungen gibt.

\linie

\textbf{\name{Riemann}sche Zahlenkugel}:
Auf die komplexe Zahlenebene wird eine Kugel mit Radius $\frac{1}{2}$ gelegt,
die die Ebene in $(0, 0)$ berührt.
Mithilfe der stereographischen Projektion entspricht jeder Punkt $z = (x, y)$
der komplexen Zahlenebene ein Punkt $(\xi, \eta, \zeta)$ der Kugel
$\mathbb{S}^2$
(man verbinde den Nordpol $\infty := (0, 0, 1)$ mit $(x, y)$ und
$(\xi, \eta, \zeta)$ ist dann der Schnittpunkt der Verbindungsgeraden mit der
Kugel).
Umgebungen auf der Kugel werden (umgekehrt) auf Umgebungen auf der Ebene
abgebildet.
Man schreibt für eine Folge $z_n \in \complex$, dass $z_n \to \infty$,
falls $|z_n| > R$ für alle $n \ge N_R$ und jedes beliebige $R > 0$, d.\,h.
$z_n \in U_R(\infty) := \{z \in \complex \;|\; |z| > R\}$.
Das Bild von $U_R(\infty)$ ist wieder eine Umgebung von $\infty$ (nur ohne
$\infty$).

$\mathbb{S}^2 \setminus \{(0, 0, 1)\}$ und $\complex$ lassen sich stetig und
bijektiv durch die stereographische Projektion aufeinander abbilden,
wenn man nun kanonischerweise wie eben $(0, 0, 1)$ mit $\infty$ identifiziert,
erhält man eine stetige Bijektion zwischen $\mathbb{S}^2$ und
$\complex^\ast := \complex \cup \{\infty\}$.

Insbesondere gilt $z_n \to \infty$ genau dann, wenn $\frac{1}{z_n} \to 0$,
sowie $\xi^2 + \eta^2 + (\zeta - \frac{1}{2})^2 = \frac{1}{4}$\\
$\iff\; \xi^2 + \eta^2 = \zeta (1 - \zeta)$.
Aus der Geradengleichung $\frac{\xi - 0}{x - 0} = \frac{\eta - 0}{y - 0} =
\frac{\zeta - 1}{0 - 1}$ für $(0, 0, 1)$, $(\xi, \eta, \zeta)$ und
$(x, y, 0)$ folgt, dass $x = \frac{\xi}{1 - \zeta}$ und
$y = \frac{\eta}{1 - \zeta}$.
Umgekehrt gilt $x^2 + y^2 = \frac{\xi^2 + \eta^2}{(1 - \zeta)^2} =
\frac{\xi}{1 - \zeta}$, daraus folgt dann
$\zeta = \frac{x^2 + y^2}{1 + x^2 + y^2}$,
$\xi = \frac{x}{1 + x^2 + y^2}$ und $\eta = \frac{y}{1 + x^2 + y^2}$.

\linie

\textbf{Kreis in $\complex^\ast$}:
Ein Kreis in $\complex^\ast$ ist definiert als ein Kreis in $\complex$ oder
eine Gerade in $\complex$, die zusätzlich $\infty$ enthält.

\textbf{Eigenschaft}:
Die stereographische Projektion erhält Kreise und Winkel.

\linie
\pagebreak

\textbf{lineare Abbildung}:
Eine \emph{lineare Abbildung} $f\colon \complex \rightarrow \complex$
(oder $f\colon \complex^\ast \rightarrow \complex^\ast$) hat die Form
$z \mapsto w = az + b$ mit $a, b \in \complex$ und $a \not= 0$.

\textbf{Spezialfall}:
$a = 1$, d.\,h. $z \mapsto w = z + b$
ist eine Verschiebung, erhält Kreise und Winkel

\textbf{Spezialfall}:
$b = 0$, $a = e^{\i \beta}$ mit $\beta \in \left[0, 2\pi\right[$,
d.\,h. Drehung um den Winkel $\beta$, erhält Kreise und Winkel

\textbf{Spezialfall}:
$b = 0$, $a \in \real$ mit $a = r > 0$,
d.\,h. Streckung/Stauchung um den Faktor $r$, erhält Kreise und Winkel

Die Abbildungen dieser drei Spezialfälle nennen sich \textbf{elementar}.
Jede lineare Abbildung\\
$z \mapsto az + b$ ist also Komposition von elementaren Abbildungen
und erhält Kreise und Winkel.

Es gibt auch die Abbildung $f\colon \complex^\ast \rightarrow \complex^\ast$,
$z \mapsto w = \frac{1}{z}$.
Diese ist eine Inversion ("`Spiegelung"') am Einheitskreis:
Der neue Betrag ist
der Kehrwert des alten Betrags, anschließend wird an der reellen
Achse gespiegelt (das neue Argument ist die Negation des alten Arguments).
Auch diese nicht-lineare Abbildung erhält Kreise und Winkel.

\linie

\textbf{\name{Möbius}-Transformation}:
Eine \emph{\name{Möbius}-Transformation} ist eine Abbildung
$f\colon \complex^\ast \rightarrow \complex^\ast$,\\
$z \mapsto w = \frac{az + b}{cz + d}$ mit $a, b, c, d \in \complex$ und
$ad - bc \not= 0$.
Aus der letzten Bedingung folgt, dass die Möbius-Transformation bijektiv
ist (außerdem ist sie stetig).
Sie erhält Kreise und Winkel, denn
$w = \frac{a}{c} + \frac{bc - ad}{c (cz + d)}$ ist eine Komposition von
linearen Abbildungen und $\frac{1}{z}$.

Die Umkehrabbildung einer MT ist
$z = \frac{dw + (-b)}{(-c)w + a}$, d.\,h. wieder eine MT.

Die Komposition von MTs ist wieder eine MT.

\linie

Es gibt keine 1:1-Beziehung zwischen den MTs $w = \frac{az + b}{cz + d}$ und
den komplexen Matrizen
\matrixsize{$\begin{pmatrix}a & b\\c & d\end{pmatrix}$},
denn eine MT wird schon durch drei komplexe Parameter
bestimmt, d.\,h. drei Gleichungen sind notwendig.
Genauer:
Seien $z_j, w_j \in \complex$ für $j = 1, 2, 3$ gegeben, dabei seien
$z_j \not= z_k$ und $w_j \not= w_k$ für $j \not= k$.
Dann gibt es genau eine Möbius-Transformation $\MT$, sodass
$w_j = \MT(z_j)$ für $j = 1, 2, 3$.

Dies beweist man, indem man $z_1, z_2, z_3$ durch den eindeutig bestimmten
Kreis verbindet, analog $w_1, w_2, w_3$.
Für einen beliebigen Punkt $z$, für den $\MT(z)$
bestimmt werden soll, lässt man den eindeutig bestimmten Kreis durch
$z, z_1, z_2$ mittels $\MT$ abbilden.
Da die MT kreistreu ist, wird der Kreis auf einen Kreis
abgebildet, der durch $w_1$ und $w_2$ geht.
Aufgrund der Winkeltreue bleibt der Schnittwinkel der beiden Kreise bei
$z_1$ und $z_2$ erhalten, d.\,h. der Kreis durch $w_1$ und $w_2$ kann eindeutig
bestimmt werden.
In der gleichen Weise verfährt man mit $z, z_1, z_3$.
Auf dem Schnittpunkt der beiden Kreise liegt das gesuchte Bild
$w = \MT(z)$.

$w$ kann auch rechnerisch bestimmt werden:
Ist $w = \frac{az + b}{cz + d}$ und $w_j = \frac{a z_j + b}{c z_j + d}$, so
gilt\\
$w_k - w_j = \frac{(ad - bc)(z_k - z_j)}{(c z_k + d)(c z_j + d)}$ für
$k, j = 1, 2, 3$.\\
Daraus folgt, dass
$(w_1, w_2, w, w_3) := \frac{(w - w_1)/(w - w_2)}{(w_3 - w_1)/(w_3 - w_2)} =
\frac{(z - z_1)/(z - z_2)}{(z_3 - z_1)/(z_3 - z_2)} =: (z_1, z_2, z, z_3)$
eine Invariante ist, aus der $w$ berechnet werden kann.

\linie

Eine Möbius-Transformation bildet im nicht-entarteten Fall entweder das Innere
eines Kreises auf das Innere oder auf das Äußere des Bilds ab.
Würde die Transformation einen Teil auf das Innere und einen Teil auf das
Äußere abbilden, so könnte man (MT stetig) einen Pfad definieren, der
vollständig im Inneren des Urbilds liegt, dessen Bild aber Endpunkte besitzt,
von denen einer im Inneren und einer im Äußeren liegt.
Dann würde das Bildes des Pfads aber den Bild-Kreis schneiden, was aufgrund
der Bijektivität und der Kreistreue nicht möglich ist.
Im entarteten Fall können Kreisinnere auf Halbebenen und Halbebenen auf Inneres
bzw. Äußeres von Kreisen abgebildet werden.

\section{%
    Mehrwertige Abbildungen und \name{Riemann}sche Flächen%
}

Es gibt Zuordnungen wie $\complex \rightarrow \complex$,
$z \mapsto w = \sqrt[n]{z}$, $n \in \natural$,
die an sich keine Abbildungen sind, da es mehrere Werte geben kann.
Im Beispiel gilt für $z = r e^{\i\varphi}$, dass die $n$ Werte\\
$w_k = r^{1/n} e^{\i(\varphi/n + 2\pi k/n)}$, $k = 0, \dotsc, n - 1$
die Gleichung $w_k^n = z$ erfüllen.\\
Hier hat man es also mit einer sog. \textbf{mehrwertigen Abbildung} zu tun.\\
Welchen Wert soll man auswählen, um eine möglichst sinnvolle Abbildung
zu definieren?

Wenn man z.\,B. immer die Lösung für $k = 0$ auswählt, so ergibt sich das
Problem, dass die Abbildung nicht stetig ist:
Nimmt man als Beispiel $n = 3$ an und "`läuft"' von $1$ aus
einmal gegen den Uhrzeigersinn um den Ursprung, so läuft das Bild nur bis
zum Argument $\frac{2\pi}{3}$;
überquert man die reelle Achse, so "`springt"' die Lösung wieder zurück
zum Argument $0$.
Man kann also keinen stetigen Zweig der Wurzeldefinition definieren.

Die Lösung besteht darin, für jeden Zweig (im Beispiel für jedes mögliche $k$)
eine Kopie von $\complex$ einzuführen.
Für $n = 3$ gibt es dann im Beispiel drei (nummerierte) Kopien der
komplexen Ebene;
wenn man die Abbildungen auf sie anwendet, verwendet man den der Nummer der
Kopie entsprechenden Zweig der mehrwertigen Abbildung.
Nun muss man die Zweige noch sinnvoll "`verkleben"', damit ein Zusammenhang
zwischen den Kopien hergestellt ist.

Bei der Wurzelfunktion geschieht dies z.\,B. darin, dass man in jeder Kopie
die positive reelle Halbachse als Schnitt wählt und die beiden Seiten
so nummeriert, dass man beim Durchlauf gegen den Uhrzeigersinn eine "`Ebene"'
höher kommt.
Wenn man ganz "`oben"' noch eine Ebene weiter geht, gelangt man wieder zurück
auf die "`unterste"'.
Der Schnitt kann auch anders gewählt werden, die geeignete Wahl ist vom Zweck
abhängig.
Die resultierende Fläche (eine eindimensionale komplexe Mannigfaltigkeit)
nennt man \textbf{\name{Riemann}-Fläche}, die "`Ebenen"' heißen
\textbf{\name{Riemann}-Blätter}.

\linie

Ein weiteres Beispiel ist $f(z) = \sqrt{z(1 - z)}$.
Obwohl $f(z) = \sqrt{z} \sqrt{1 - z}$ ist, gibt es nur zwei Zweige statt vier,
da sich die anderen beiden wegkürzen
(wählt man z.\,B. bei beiden Wurzeln $k = 1$, so ist die Summe der
resultierenden Phasensprünge in den Argumenten $2\pi$).
"`Erlaubte"' Wege in der komplexen Ebene sind die Wege, die entweder keinen
oder beide der Punkte $0$ und $1$ umlaufen.
"`Nicht erlaubt"' ist ein Umlaufen nur von $0$ oder $1$, da
so wieder ein Sprung auftritt (wie oben).

Eine mögliche Lösung besteht darin, zwischen $0$ und $1$ einen Schnitt zu
machen und die beiden Riemann-Blätter an gegenüberliegenden Seiten zu
verkleben.

\linie

Für $w = \Ln(z)$ gilt $e^w = z$, d.\,h. mit $z = r e^{\i\varphi}$ ist
$w = u + \i v$ ($u, v \in \real$) mit $u = \ln(r)$ und
$v = \arg(z) = \varphi + 2k\pi$, $k \in \integer$.
Es gibt also abzählbar unendlich viele Lösungen.
Dementsprechend gibt es auch unendlich viele Riemann-Blätter, die analog wie
bei der Wurzelfunktion verklebt werden müssen.

\pagebreak

\section{%
    Dif"|ferenzierbarkeit%
}

Ist $U \subset \complex$ of"|fen, $z_0 \in U$ und
$f\colon U \rightarrow \complex$, so ist die komplexe Ableitung von $f$ in
$z_0$ definiert als
$f'(z_0) := \lim_{h \to 0,\; h \in \complex} \frac{f(z_0 + h) - f(z_0)}{h}$.
$f$ kann als Funktion $f(z) = (u(x, y), v(x, y))$\\
$= u(x, y) + \i v(x, y)$
aufgefasst werden, wobei $u, v\colon \real^2 \rightarrow \real$ reelle
Funktionen sind.

\linie

\textbf{Spezialfall}:
Für $h = x \to 0$ ($x \in \real$) und
$f'$ in $z_0$ komplex dif"|ferenzierbar gilt\\
$f'(z_0)
= \lim_{x \to 0} \frac{u(z_0 + x) + \i v(z_0 + x) - u(z_0) - \i v(z_0)}{x}
= \lim_{x \to 0} \left(\frac{u(z_0 + x) - u(z_0)}{x} +
\i \frac{v(z_0 + x) - v(z_0)}{x}\right)$,\\
d.\,h. $f'(z_0) = \frac{\partial u}{\partial x}(z_0) +
\i \frac{\partial v}{\partial x}(z_0)$.

\textbf{Spezialfall}:
Für $h = \i y \to 0$ ($y \in \real$) und
$f'$ in $z_0$ komplex dif"|ferenzierbar gilt\\
$f'(z_0)
= \lim_{y \to 0} \frac{u(z_0 + \i y) + \i v(z_0 + \i y) -
u(z_0) - \i v(z_0)}{\i y}
= \lim_{y \to 0} \left(\frac{v(z_0 + \i y) - v(z_0)}{y} -
\i \frac{u(z_0 + \i y) - u(z_0)}{y}\right)$,\\
d.\,h. $f'(z_0) = \frac{\partial v}{\partial y}(z_0) -
\i \frac{\partial u}{\partial y}(z_0)$.

\linie

Also gilt:
Falls $f$ in $z_0$ komplex dif"|ferenzierbar ist, so sind $u$ und $v$ partiell
dif"|ferenzierbar und es gilt
$\frac{\partial u}{\partial x} = \frac{\partial v}{\partial y}$ sowie
$\frac{\partial v}{\partial x} = -\frac{\partial u}{\partial y}$
(\textbf{\name{Cauchy}-\name{Riemann}-Gleichungen}).

In diesem Falle gilt $\left.\frac{df}{dz}\right|_{z=z_0} = u_x' + \i v_x' =
v_y' - \i u_y' = u_x' - \i u_y' = v_y' + \i v_x'$, d.\,h. $f$ kann
dif"|ferenziert werden, ohne den Real- oder Imaginärteil zu kennen.

\linie

Sei $f$ in $z_0$ komplex dif"|fb.\\
Dann gilt $f(z) = f(z_0) + w_0 h + o(|h|)$ mit $w_0 := f'(z_0)$
und $z - z_0 =: h = h_x + \i h_y$.\\
Setzt man $u_0 := u(x_0, y_0) = \Re(f(z_0))$,
$v_0 := v(x_0, y_0) = \Im(f(z_0))$,
$f = u + \i v$,
$f(z_0) = u_0 + \i v_0$ und
$|h|_\complex =$
\matrixsize{$\norm{\begin{pmatrix}h_x\\h_y\end{pmatrix}}_{\real^2}$},
so ergibt sich bei Betrachtung der Real- und Imaginärteile\\
$u = u_0 + (w_{0,r} h_x - w_{0,i} h_y) + o\Big($\matrixsize{$\norm{
\begin{pmatrix}h_x\\h_y\end{pmatrix}}$}$\Big)$ sowie
$v = v_0 + (w_{0,i} h_x + w_{0,r} h_y) + o\Big($\matrixsize{$\norm{
\begin{pmatrix}h_x\\h_y\end{pmatrix}}$}$\Big)$
mit $w_{0,r} := \Re(w_0)$ und $w_{0,i} := \Im(w_0)$,
also
$\begin{pmatrix}u\\v\end{pmatrix} = \begin{pmatrix}u_0\\v_0\end{pmatrix} +
\begin{pmatrix}w_{0,r} & -w_{0,i}\\w_{0,i} & w_{0,r}\end{pmatrix}
\begin{pmatrix}h_x\\h_y\end{pmatrix} + o(\norm{h})$.\\
Wegen den CR-Gleichungen gilt $w_{0,r} = u_x' = v_y'$ und
$w_{0,i} = -u_y' = v_x'$, d.\,h.\\
$\begin{pmatrix}u\\v\end{pmatrix} = \begin{pmatrix}u_0\\v_0\end{pmatrix} +
\begin{pmatrix}u_x' & u_y'\\v_x' & v_y'\end{pmatrix}
\begin{pmatrix}h_x\\h_y\end{pmatrix} + o(\norm{h})$.
Somit ist \matrixsize{$\begin{pmatrix}u\\v\end{pmatrix}$}
Frechet-dif"|ferenzierbar.

Es gilt $\frac{D(u, v)}{D(x, y)} =
\begin{pmatrix}w_{0,r} & -w_{0,i}\\w_{0,i} & w_{0,r}\end{pmatrix} =
\sqrt{w_{0,r}^2 + w_{0,i}^2}
\begin{pmatrix}\cos \theta & -\sin \theta\\
\sin \theta & \cos \theta\end{pmatrix}$,
da $\left(\frac{w_{0,r}}{|w|}\right)^2 +
\left(\frac{w_{0,i}}{|w|}\right)^2 = 1$.
Also ist die Jacobi-Matrix eine Drehung mit anschließender Streckung.
Man nennt Transformationen, deren Jacobi-Matrix gleich einer Rotationsmatrix
multipliziert mit einem Skalar ist, \textbf{konform}.

Es gilt also:
Eine Funktion $f\colon \complex \rightarrow \complex$ ist in
$z_0 \in \complex$ komplex dif"|ferenzierbar genau dann, wenn die
Cauchy-Riemann-Gleichungen erfüllt sind und
\matrixsize{$\begin{pmatrix}u\\v\end{pmatrix}$} Frechet-dif"|ferenzierbar ist
(die Umkehrung beweist man wie eben, nur umgekehrt).

Anders gesagt ist $f$ komplex dif"|ferenzierbar genau dann, wenn alle
partiellen Ableitungen existieren und stetig sind sowie die
Cauchy-Riemann-Gleichungen erfüllt sind.

\linie

Angenommen, es existieren alle partiellen Ableitungen zweiter Ordnung und
diese sind stetig.
Dann folgt aus den CR-Gleichungen
$\frac{\partial}{\partial x} \left(\frac{\partial u}{\partial x}\right) =
\frac{\partial}{\partial x} \left(\frac{\partial v}{\partial y}\right) =
\frac{\partial}{\partial y} \left(\frac{\partial v}{\partial x}\right) =
\frac{\partial^2 v}{\partial x \partial y}$ und analog\\
$\frac{\partial}{\partial y} \left(\frac{\partial u}{\partial y}\right) =
-\frac{\partial}{\partial y} \left(\frac{\partial v}{\partial x}\right) =
-\frac{\partial^2 v}{\partial x \partial y}$, d.\,h.
$\Delta u = \left(\frac{\partial^2}{\partial x^2} +
\frac{\partial^2}{\partial y^2}\right)u = 0$ und
$\Delta v = \left(\frac{\partial^2}{\partial x^2} +
\frac{\partial^2}{\partial y^2}\right)v = 0$.\\
Solche Funktionen (zweifach stetig dif"|fb. mit $\Delta u = 0$)
nennt man \textbf{harmonisch}.

\linie
\pagebreak

\textbf{"`partielle Ableitungen"' $\frac{\partial}{\partial z}$ und
$\frac{\partial}{\partial \overline{z}}$}:

Für $f(z) = f(x + \i y) = u(x, y) + \i v(x, y)$ ist
$z = x + \i y$ und $\overline{z} = x - \i y$, d.\,h.\\
$x = \frac{z + \overline{z}}{2}$ und $y = \frac{z - \overline{z}}{2 \i}$.
Damit kann $f(z) = f(x, y) =
f(\frac{z + \overline{z}}{2}, \frac{z - \overline{z}}{2 \i}) =:
\widetilde{f}(z, \overline{z})$ als eine Funktion von zwei voneinander
abhängigen Variablen $z$ und $\overline{z}$ betrachtet werden.
Tut man so, als wären $z$ und $\overline{z}$ voneinander unabhängig, dann ist
$\frac{D(z, \overline{z})}{D(x, y)} =$
\matrixsize{$\begin{pmatrix}1 & \i\\1 & -\i\end{pmatrix}$} und\\
\matrixsize{$\begin{pmatrix}\partial/\partial z\\
\partial/\partial \overline{z}\end{pmatrix}$} $=
\left(\frac{D(x, y)}{D(z, \overline{z})}\right)^t$
\matrixsize{$\begin{pmatrix}\partial/\partial x\\
\partial/\partial y\end{pmatrix}$}.
Damit gilt $\frac{\partial}{\partial z} :=
\frac{1}{2} \left(\frac{\partial}{\partial x} -
\i \frac{\partial}{\partial y}\right)$ und
$\frac{\partial}{\partial \overline{z}} :=
\frac{1}{2} \left(\frac{\partial}{\partial x} +
\i \frac{\partial}{\partial y}\right)$, d.\,h. zum Beispiel
$\frac{\partial}{\partial z} \widetilde{f}(z, \overline{z}) :=
\frac{1}{2} \left(\frac{\partial}{\partial x} -
\i \frac{\partial}{\partial y}\right) f(x, y)$ und
$\frac{\partial}{\partial \overline{z}} \widetilde{f}(z, \overline{z}) :=
\frac{1}{2} \left(\frac{\partial}{\partial x} +
\i \frac{\partial}{\partial y}\right) f(x, y)$.

Im Beispiel $f(z) = z$ gilt
$\frac{\partial}{\partial z} f = \frac{1}{2} + \frac{1}{2} = 1$ und
$\frac{\partial}{\partial \overline{z}} f = \frac{1}{2} - \frac{1}{2} = 0$.

Existieren die part. Ableitungen, dann gilt
$\frac{\partial}{\partial \overline{z}} (u(x, y) + \i v(x, y)) =
\frac{1}{2} \left(\frac{\partial u}{\partial x} -
\frac{\partial v}{\partial y}\right) +
\frac{\i}{2} \left(\frac{\partial u}{\partial y} +
\frac{\partial v}{\partial x}\right)$.\\
Die beiden Ausdrücke in den Klammern sind $0$ genau dann, wenn
die CR-Gl. erfüllt sind.

Damit gilt:
$u + \i v$ erfüllt die Cauchy-Riemann-Gleichungen genau dann, wenn
$\frac{\partial}{\partial \overline{z}} f = 0$.

Man schreibt auch kurz
$\partial = \frac{\partial}{\partial z}$ und
$\overline{\partial} = \frac{\partial}{\partial \overline{z}}$.
$\overline{\partial}$ heißt
\textbf{\name{Cauchy}-\name{Riemann}-Operator}.

\linie

\textbf{Beispiele}:
$f(z) = z^n$ ist komplex dif"|fb.,
dagegen ist $f(z) = |z|^2 = z \overline{z}$ nicht komplex dif"|fb.\\
Die Potenzreihe $f(z) = \sum_{k=0}^\infty a_k z^k$ konvergiert für
$|z| < R$ und divergiert für $|z| > R$, wobei
$R = \frac{1}{\limsup_{k \to \infty} \sqrt[k]{|a_k|}}$.
Für $|z| < R$ konvergiert die Reihe absolut, ist $|z| \le R' < R$,
so konvergiert die Reihe sogar gleichmäßig.
Die Reihe ist in jedem Kreis $|z| \le R' < R$ gliedweise komplex dif"|fb.,
d.\,h. sie ist für $|z| < R$ komplex dif"|fb. und
$f'(z) = \sum_{k=1}^\infty k a_k z^{k-1}$.

\linie

\textbf{holomorph}:
Seien $U \subset \complex$ of"|fen und $f\colon U \rightarrow \complex$
eine Funktion.
$f$ heißt \emph{holomorph in $z_0 \in U$}, falls es ein $\varepsilon > 0$ gibt,
sodass $f$ in allen $z$ mit $|z - z_0| < \varepsilon$ komplex dif"|ferenzierbar
ist.\\
$f$ heißt \emph{holomorph in $U$} ($f \in \A(U)$), falls
$f$ in allen $z_0 \in U$ holomorph ist.

\section{%
    Gebiete%
}

\textbf{zusammenhängend}:
Sei $G \subset \complex$ of"|fen mit $G \not= \emptyset$.
Dann heißt $G$ \emph{zusammenhängend}, falls\\
$\lnot(\exists_{G_1, G_2 \subset \complex \text{ of"|fen}}\quad
G_1 \not= \emptyset, G_2 \not= \emptyset,\quad
G_1 \cup G_2 = G,\quad
G_1 \cap G_2 = \emptyset)$.

\linie

\textbf{Polygonzug in $G$}:
Sei $G \subset \complex$ of"|fen mit $a, b \in G$.
Sei außerdem $z_j \in G$ für $j = 0, \dotsc, n$,
wobei $z_0 := a$ und $z_n := b$.
Die $k$-te \emph{Teilstrecke} ist $\overline{z_k z_{k+1}} :=
\{z_k + t(z_{k+1} - z_k) \;|\; t \in [0,1]\}$ mit $k = 0, \dotsc, n - 1$.
Der \emph{Polygonzug} in $G$ von $a$ nach $b$ über $\delta = \{z_k\}_{k=0}^n$
ist $\Gamma_{ab}^\delta := \bigcup_{k=0}^{n-1} \overline{z_k z_{k+1}}$.

\textbf{polygonial zusammenhängend}:
Sei $G \subset \complex$ of"|fen mit $G \not= \emptyset$.
Dann heißt $G$ \emph{polygonial zusammenhängend}, falls
$\forall_{a, b \in G} \exists_{\delta = \{z_k\}_{k=0}^n}\;
\Gamma_{ab}^\delta \subset G$.

$G$ ist zusammenhängend genau dann, wenn $G$ polygonial zusammenhängend ist.

\linie

\textbf{Gebiet}:
Eine nicht-leere, of"|fene, zusammenhängende Menge $G \subset \complex$ heißt
\emph{Gebiet}.

\linie

\textbf{Satz}:
Seien $G \subset \complex$ ein Gebiet und $f \in \A(G)$.
Dann sind äquivalent:
\begin{enumerate}
    \item
    $f'(z) \equiv 0$

    \item
    $\Re f(z) \equiv \const$

    \item
    $\Im f(z) \equiv \const$

    \item
    $|f(z)| \equiv \const$

    \item

    $f(z) \equiv \const$
\end{enumerate}

\section{%
    Kurvenintegrale%
}

\textbf{\name{Jordan}-Kurve}:
Sei $\gamma\colon [0, T] \rightarrow \complex$ stetig und injektiv
(bis auf ggf. $\gamma(0) = \gamma(T)$).\\
Für $\gamma \in \C^1$ und $\dot{\gamma}(t) \not= 0$ für alle $t \in [0, T]$
heißt $\Gamma_\gamma := \{z = \gamma(t) \;|\; t \in [0, T]\}$
\emph{\name{Jordan}-Kurve}.

\textbf{Zerlegung und Stützstellen}:
$\delta = \{t_k\}_{k=0}^n$ heißt \emph{Zerlegung} von $[0, T]$, falls\\
$0 = t_0 < t_1 < \dotsb < t_n = T$.
$\xi = \{\tau_k\}_{k=0}^{n-1}$ heißt \emph{Satz von Stützstellen}, falls
$\tau_k \in [t_k, t_{k+1}]$ für $k = 0, \dotsc, n - 1$.
Man definiert $z_k := \gamma(t_k)$ für $k = 0, \dotsc, n$ und
$w_j = \gamma(\tau_j)$ für $j = 0, \dotsc, n - 1$.

\textbf{\name{Riemann}-Summe}:
Sei $f\colon \Gamma_\gamma \subset \complex \rightarrow \complex$ stetig.
Dann gilt für die \emph{\name{Riemann}-Summe}\\
$\sum_{k=0}^{n-1} f(w_k) (z_{k+1} - z_k)
= \sum_{k=0}^{n-1} f(\gamma(\tau_k)) (\gamma(t_{k+1}) - \gamma(t_k))
\frac{t_{k+1} - t_k}{t_{k+1} - t_k}
= \sum_{k=0}^{n-1} \widetilde{f}(\tau_k)
\frac{\gamma(t_{k+1}) - \gamma(t_k)}{t_{k+1} - t_k} \Delta t_k$\\
mit $\widetilde{f} := f \circ \gamma$ und $\Delta t_k := t_{k+1} - t_k$.
Lässt man den Rang der Zerlegung $\lambda(\delta)$ gegen $0$ laufen, so sieht
man, dass folgende Definition Sinn ergibt.

\textbf{Kurvenintegral}:
Für eine Jordan-Kurve $\Gamma_\gamma$ und eine stetige Funktion
$f\colon \Gamma_\gamma \subset \complex \rightarrow \complex$ ist das
\emph{Kurvenintegral} definiert als $\int_{\Gamma_\gamma} f(z)\dz
:= \int_0^T (f \circ \gamma)(t) \dot{\gamma}(t) \dt$.

Diese Definition ist unabhängig von der konkreten Parametrisierung $\gamma$\\
(bei Erhalt der Richtung).

\linie

\textbf{Eigenschaften}:
\begin{enumerate}
    \item
    $\int_{\Gamma_{ab}} f(z)\dz = -\int_{\Gamma_{ba}} f(z)\dz$

    \item
    $\int_{\Gamma_{ab}} (\alpha f(z) + \beta g(z))\dz =
    \alpha \int_{\Gamma_{ab}} f(z)\dz + \beta \int_{\Gamma_{ab}} g(z)\dz$

    \item
    Man kann Jordan-Kurven
    $\overrightarrow{\Gamma_1}, \dotsc, \overrightarrow{\Gamma_n}$
    aneinanderhängen.
    Es ergibt sich ein sog. \emph{gerichteter Pfad} $\overrightarrow{\Gamma}$,
    der jedoch nur rein symbolisch als
    "`$\overrightarrow{\Gamma} =
    \overrightarrow{\Gamma_1} \cup \dotsb \cup \overrightarrow{\Gamma_n}$"'
    geschrieben werden kann, da dieser sich z.\,B. selbst überschneiden darf
    (in verschiedenen Jordan-Kurven).\\
    Es sei dann $\int_{\overrightarrow{\Gamma}} f(z)\dz :=
    \sum_{k=1}^n \int_{\overrightarrow{\Gamma_k}} f(z)\dz$.
\end{enumerate}

\textbf{Beispiele}:
Für $f(z) \equiv 1$ gilt $\int_{\Gamma_{ab}} f(z)\dz = b - a \in \complex$.\\
Für $f(z) = z^n$, $n \in \integer$ und
$\Gamma = \{z \in \complex \;|\; |z| = 1\}$ mit einfach mathematisch positivem
Umlauf (gegen den Uhrzeigersinn) kann man $\Gamma$ durch
$\gamma\colon [0, 2\pi] \rightarrow \complex$, $\gamma(t) = e^{\i t}$
parametrisieren.
Das entstehende Integral bezeichnet man auch als \emph{Ringintegral}
und man schreibt $\oint$ dafür, dass über einen geschlossenen Pfad
mit einfach mathematisch positivem Umlauf integriert wird.\\
Es gilt $\oint_\Gamma z^n \dz = \int_0^{2\pi} e^{\i tn} \i e^{\i t} \dt
= \i \cdot \int_0^{2\pi} e^{\i t(n+1)} \dt
= \i \cdot \int_0^{2\pi} (\cos((n+1)t) + \i \sin((n+1)t)) \dt$.
Daraus ergibt sich die wichtige Formel
$\oint_\Gamma z^n \dz =$
\matrixsize{$\begin{cases} 2\pi\i & n = -1\\0 & n \not= -1\end{cases}$}.

\textbf{Anmerkung zur Abschätzung von Kurvenintegralen}:
Im Allgemeinen gilt die Formel
$|\int_{\Gamma_{ab}} f(z)\dz| \le \sup_{z \in \Gamma_{ab}} |f(z)| \cdot |b - a|$
nicht
(ein Gegenbeispiel ist das Beispiel mit $z^n$ von eben).
Dies liegt daran, dass sich in
$|\sum_{k=0}^{n-1} f(w_k) (z_{k+1} - z_k)|
\le \sum_{k=0}^{n-1} |f(w_k)| |z_{k+1} - z_k|$\\
$\le \sup_{z \in \Gamma_{ab}} |f(z)| \cdot \sum_{k=0}^{n-1} |z_{k+1} - z_k|$
die letzte Summe aufgrund der Beträge keine Teleskopsumme ist
(im Gegensatz dazu, wie es im Reellen der Fall wäre).
Jedoch erhält man im Grenzübergang $\lambda(\delta) \to 0$ die richtige Formel
$|\int_\Gamma f(z)\dz| \le \sup_{z \in \Gamma} |f(z)| \cdot \ell(\Gamma)$
mit $\ell(\Gamma)$ der Länge von $\Gamma$.

\linie

\textbf{Satz (Formel von \name{Newton}-\name{Leibniz})}:\\
Seien $G \subset \complex$ ein Gebiet,
$a, b \in G$, $\Gamma_{ab} \subset G$ eine (stückweise) $\C^1$-Jordan-Kurve,
$f\colon G \rightarrow \complex$ eine Funktion, wobei
$f$ für alle $z \in \Gamma_{ab}$ komplex dif"|ferenzierbar und
$f'|_{\Gamma_{ab}}$ stetig ist.\\
Dann gilt $\int_{\Gamma_{ab}} f'(z) \dz = f(b) - f(a)$.

\textbf{Anmerkung}:
Ist $a = b$, d.\,h. $\Gamma$ ein geschlossener Pfad, so gilt
$\oint_\Gamma f'(z)\dz = 0$\\
(wenn die Stammfunktion existiert).

\linie
\pagebreak

\textbf{Warum gilt dann $\int_\Gamma \frac{1}{z} \dz = 2\pi\i$?}\\
(dabei ist $\Gamma$ der mathematisch positive einfache Umlauf von $|z| = 1$)

Dann müsste man eine Stammfunktion von $f'(z) = \frac{1}{z}$ finden.
Versucht man $f(z) = \Ln z$, so muss man einen Zweig auswählen.
Allerdings ist es nicht möglich, einen auf ganz $\Gamma$ dif"|ferenzierbaren
Zweig von $\Ln z$ anzugeben
(irgendwo muss der "`Schnitt"' sein).

\linie

\emph{Beispiel}:
Für das Polynom $p(z) = c_0 + c_1 z + \dotsb + c_n z^n$ und
$q(z) = c_0 z + \frac{c_1}{2} z^2 + \dotsb + \frac{c_n}{n + 1} z^{n+1}$ gilt
$p(z) = q'(z)$, d.\,h. $\int_{\Gamma_{ab}} p(z)\dz = q(b) - q(a)$, insbesondere
gilt für $a = b$ $\oint_\Gamma p(z)\dz = 0$.

\emph{Beispiel}:
Für die Potenzreihe $p(z) = c_0 + \sum_{k=1}^\infty c_k z^k$ mit
Konvergenzradius $R > 0$ und\\
$q(z) = \sum_{k=1}^\infty \frac{c_{k-1}}{k} z^k$ gilt $p(z) = q'(z)$
(dabei hat $q(z)$ den gleichen Konvergenzradius).\\
Insbesondere gelten also für
$\Gamma_{ab}, \Gamma \subset U_R(0) = \{z \in \complex \;|\; |z| < R\}$
obige Formeln.

\section{%
    Der Integralsatz von \name{Cauchy}%
}

\textbf{Satz (Integralsatz von \name{Cauchy} für Dreiecke)}:\\
Seien $G \subset \complex$ ein Gebiet und $\Delta \subset G$ ein Dreieck
(d.\,h. Rand und Inneres liegen in $G$).\\
Außerdem seien $f \in \A(G)$ und $\Gamma = \partial \Delta$.\\
Dann gilt $\oint_\Gamma f(z)\dz = 0$.

\textbf{Anmerkung}:
Man weiß hier i.\,A. nicht, ob $f$ eine Stammfunktion besitzt.

\linie

\textbf{sternförmiges Gebiet}:
Ein Gebiet $G \subset \complex$ heißt \emph{sternförmig}, falls es ein
$a \in G$ gibt, sodass
$\forall_{z \in G}\; \overline{az} \subset G$.
$a$ heißt in diesem Fall \emph{zentraler Punkt}.

\textbf{Satz (Integralsatz von \name{Cauchy} für sternförmige Gebiete)}:\\
Seien $G \subset \complex$ ein sternförmiges Gebiet und $f \in \A(G)$.\\
Dann gilt $F \in \A(G)$ mit $F(z) := \int_{\overline{az}} f(w)\dw$ und
$F'(z) = F(z)$.

Holomorphe Funktionen auf sternförmigen Gebieten haben also Stammfunktionen.

\linie

\textbf{Zusammenfassung}:
\begin{enumerate}
    \item
    Seien $G \subset \complex$, $f \in \A(G)$ und
    $\exists_{F \in \A(G)}\; F' = f$.
    Dann gilt $\int_{\Gamma_{ab}} f(z)\dz =
    \int_{\widetilde{\Gamma}_{ab}} f(z)\dz$ ($\ast$).\\
    Für $\Gamma \subset G$ geschlossen gilt außerdem $\oint_\Gamma f(z)\dz = 0$
    ($\ast\ast$).\\
    Die Formeln ($\ast$) und ($\ast\ast$) sind äquivalent, d.\,h.
    ($\ast$) gilt für alle Pfade
    $\Gamma_{ab}, \widetilde{\Gamma}_{ab} \subset G$
    genau dann, wenn ($\ast\ast$) für alle geschlossenen Pfade
    $\Gamma \subset G$ gilt.

    \item
    Seien $G \subset \complex$ sternförmig und $f \in \A(G)$.
    Dann gilt $\exists_{F \in \A(G)}\; F' = f$.
    Also gelten ($\ast$), ($\ast\ast$).
\end{enumerate}

Ist $G \subset \complex$ nicht sternförmig, so muss $f$ i.\,A. keine
Stammfunktion besitzen.\\
Ein Gegenbeispiel ist $\int_\Gamma \frac{1}{z} \dz = 2\pi\i$
(hier fehlt die $0$ in dem Gebiet).

Ist das Gebiet $G$ nicht sternförmig, aber der geschlossene Pfad $\Gamma$
zusammenziehbar (z.\,B. wenn das Gebiet keine "`Löcher"' hat --
weiter unten wird dies genauer erklärt), so kann man folgendermaßen vorgehen:
Wähle ein sternförmiges Gebiet $G' \subset G$, das einen Teil vom Pfad enthält.
Trenne nun einen Teil des Pfades ab (dabei beachte man die Umlaufrichtung),
wobei man den Schnitt als $\gamma$ bezeichnet.
Dann ist $\oint_\Gamma = \oint_\Gamma +
\int_{\overrightarrow{\gamma}} + \int_{\overleftarrow{\gamma}} =
\oint_{\Gamma'} + \oint_{\Gamma''}$, wobei $\Gamma'$, $\Gamma''$ die zwei
geschlossenen Teilpfade sein sollen ($\Gamma'' \subset G'$ soll der Teil sein,
der ganz in $G'$ liegt).
Es gilt $\oint_{\Gamma''} = 0$, da $G'$ sternförmig ist.
Also ist $\oint_\Gamma = \oint_{\Gamma'}$, man hat also den Pfad
"`verkleinert"'.
In Gebieten ohne Löcher kann man dies iterativ durchführen, so dass
schließlich der Pfad vollständig in einem sternförmigen Gebiet liegt
und das Integral somit $0$ ist.

\linie
\pagebreak

\textbf{elementare Deformation eines Pfades in $G$}:\\
Seien $G \subset \complex$ ein Gebiet und $G' \subset G$ sternförmig.
Die Ersetzung eines Teils eines geschlossenen Pfads $\Gamma \subset G$
im sternförmigen Gebiet $G'$ heißt \emph{elementare Deformation}
von $\Gamma$.\\
Für nicht-geschlossene Pfade $\Gamma_{ab} \subset G$ ist dies analog definiert,
nur müssen hier die Anfangs- und Endpunkte $a, b \in G$ erhalten bleiben.

\textbf{wichtig}:
Das Pfadintegral bleibt für $f \in \A(G)$ bei el. Deformationen des
Pfads erhalten.

\textbf{homotop}:
Zwei Pfade $\Gamma, \Gamma' \subset G$ sind in $G$ \emph{homotop}, falls
$\Gamma'$ aus $\Gamma$ durch eine endliche Anzahl von elementaren Deformationen
hervorgeht.

\textbf{Satz (Deformationssatz)}:\\
Seien $G \subset \complex$ ein Gebiet, $f \in \A(G)$ und
$\Gamma, \Gamma' \subset G$ in $G$ homotope Pfade.\\
Dann gilt $\int_\Gamma f(z)\dz = \int_{\Gamma'} f(z)\dz$.

Betrachtet man einen einzelnen Punkt als Pfad (sog. \textbf{Nullpfad}),
so gilt $\int_\Gamma f(z)\dz = 0$, falls $\Gamma$ ein geschlossener Pfad in $G$
homotop zum Nullpfad ist.

\linie

\textbf{einfach zusammenhängend}:
Ein Gebiet $G \subset \complex$ heißt \emph{einfach zusammenhängend}, falls
jeder geschlossene Pfad $\Gamma \subset G$ in $G$ homotop zum Nullpfad ist.

\textbf{Satz (Integralsatz von \name{Cauchy}
für einfach zusammenhängende Gebiete)}:\\
Seien $G \subset \complex$ ein einfach zusammenhängendes Gebiet und
$f \in \A(G)$.\\
Dann gilt $\oint_\Gamma f(z)\dz = 0$ für jeden
geschlossenen Pfad $\Gamma \subset G$.\\
Zudem gilt für $a \in G$, dass $F' = f$ mit
$F(z) := \int_{\Gamma_{az}} f(w)\dw$, d.\,h.
$f$ hat eine Stammfunktion.

Die Umkehrung gilt ebenfalls:
Wenn jede holomorphe Funktion $f \in \A(G)$ eine Stammfunktion besitzt,
dann ist $G$ einfach zusammenhängend.

\linie

\textbf{weitere Modifikationen}:\\
In einem Gebiet, das ein "`Loch"' hat, kann man einen
Pfad homotop zum Nullpfad so elementar deformieren, dass er sehr nahe an
den Rand des Gebiets kommt.
Vom äußeren Rand zum Loch läuft dabei der Pfad einmal
hin und einmal wieder zurück (auf derselben Linie).
Weil die Umlaufrichtungen auf dieser Linie gegenläufig sind, heben sich die
Integrale auf und man erhält
$\oint_{\Gamma_+} f(z)\dz + \oint_{\Gamma_-} f(z)\dz = 0$,
falls $\Gamma_+$ bzw. $\Gamma_-$ den Pfad um den äußeren bzw. inneren Rand
bezeichnet.
(Beachte: $\Gamma_-$ wird im mathematisch negativem Sinne umlaufen!)

Allgemeiner gilt für ein Gebiet $G$ mit $k$ "`Löchern"'
$\oint_{\Gamma_+} f(z)\dz +
\sum_{j=1}^k \left(\oint_{\Gamma_{-,j}} f(z)\dz\right) = 0$\\
für alle $f \in \A(G)$.
Dabei bezeichnet $\Gamma_+$ den Pfad um den äußeren Rand (positiv umlaufen)
und $\Gamma_{-,j}$ den Pfad um das $j$-te Loch (negativ umlaufen).

\linie

\textbf{Windungszahl}:
Für $w \in \complex$ und einen geschlossenen Pfad $\Gamma \subset \complex$,
der $w$ nicht enthält, bezeichnet man
$n(\Gamma, w) := \frac{1}{2\pi\i} \cdot \oint_\Gamma \frac{1}{z - w} \dz$ als
die \emph{Windungszahl} von $\Gamma$ um $w$.

Parametrisiert man einen nicht-geschlossenen Pfad
$\Gamma_T$ durch $\gamma\colon [0, T] \rightarrow \complex$, so gilt\\
$\Re(\frac{1}{2\pi\i} \oint_{\Gamma_T} \frac{1}{z - w} \dz) =
\frac{1}{2\pi} \Im(\oint_{\Gamma_T} \frac{1}{z - w} \dz) =
\frac{1}{2\pi} (\arg(\gamma(T) - w) - \arg(\gamma(0) - w))$.

\textbf{Satz (äquivalente Beschreibungen)}:
Sei $G \subset \complex$ ein Gebiet.
Dann sind äquivalent:
\begin{enumerate}
    \item
    $G$ ist einfach zusammenhängend.

    \item
    $\forall_{\Gamma \subset G \text{ geschlossen}} \forall_{w \notin G}\;
    n(\Gamma, w) = 0$

    \item
    $\forall_{\Gamma \subset G \text{ geschlossen}} \forall_{f \in \A(G)}\;
    \oint_\Gamma f(z)\dz = 0$

    \item
    $\forall_{f \in \A(G)} \exists_{F \in \A(G)}\; F' = f$

    \item
    $\forall_{f \in \A(G),\; \forall_{z \in G}\; f(z) \not= 0\;}
    \exists_{g \in \A(G)}\; e^g = f$
\end{enumerate}

\section{%
    Die Integralformel von \name{Cauchy}%
}

\textbf{Satz (Integralformel von \name{Cauchy})}:
Seien $G \subset \complex$ ein Gebiet, $a \in G$ und $\varepsilon > 0$
mit $\overline{U_\varepsilon(a)} \subset G$.
Sei außerdem $f \in \A(G)$ und $\Gamma$ homotop in
$G \setminus \{a\}$ zum einfachen, mathematisch positiven Umlauf von
$\partial U_\varepsilon(a)$.\\
Dann gilt $\frac{1}{2\pi\i} \oint_\Gamma \frac{f(z)}{z - a} \dz = f(a)$.

\emph{Beispiel}:
$\oint_\Gamma \frac{\cos z}{z} \dz = 2\pi\i$ für jeden geschlossenen Pfad
$\Gamma \in \complex \setminus \{0\}$, der homotop zum einfachen,
math. positiven Umlauf des Einheitskreises ist.

\textbf{Spezialfall (Mittelwertsatz)}:
Seien $G \subset \complex$ ein Gebiet und $f \in \A(G)$.\\
Dann gilt $\frac{1}{2\pi} \int_0^{2\pi} f(a + Re^{i\theta}) d\theta = f(a)$.

\linie

\textbf{analytisch}:
$f$ heißt \emph{analytisch} im Punkt $a \in \complex$, falls\\
$\exists_{\varepsilon > 0} \forall_{z \in U_\varepsilon(a)}\;
f(z) = \sum_{k=0}^\infty c_k (z - a)^k$ konvergiert
(mit bestimmten $c_k \in \complex$).

\textbf{Folgerung}:
Ist $f$ analytisch im Punkt $a$, so hat die Potenzreihe einen Konvergenzradius
$R$ mit $0 < \varepsilon \le R$.
Für $|z - a| < \varepsilon$ kann man gliedweise dif"|ferenzieren und erhält
$c_k = \frac{f^{(k)}(a)}{k!}$.

\textbf{Satz}:
Seien $G \subset \complex$ ein Gebiet.
Definiere für $a \in G$ den Abstand\\
$R_a := \dist(a, \partial G)
= \inf_{u \in \partial G} |a - u|$ von $a$ zum Rand von $G$.\\
Dann gilt $f \in \A(G)$ genau dann, wenn $f$ analytisch in allen $a \in G$ ist.
Die Potenzreihe hat in diesem Fall einen Konvergenzradius $\ge R_a$
und es gilt $c_k = c_k(a) = \frac{1}{2\pi\i}
\oint_{\partial U_r(a)} \frac{f(z)}{(z - a)^{k+1}} \dz$ mit $0 < r < R_a$.
Mit obiger Formel ergibt sich damit
$\frac{1}{2\pi\i} \oint_{\partial U_r(a)} \frac{f(z)}{(z - a)^{k+1}} \dz =
\frac{f^{(k)}(a)}{k!}$.

\textbf{Folgerung}:
Ist $f \in \A(G)$, so ist $f$ beliebig oft dif"|ferenzierbar.

Die $c_k$ erfüllen die Abschätzung $|c_k| \le M r^{-k}$ mit
$M := \sup_{z \in U_r(a)} |f(z)|$.

\linie

\textbf{Satz von \name{Liouville}}:
Sei $f \in \A(\complex)$ beschränkt, d.\,h.
$\exists_{M > 0} \forall_{z \in \complex}\; |f(z)| \le M$.\\
Dann ist $f(z) \equiv \const$.

\textbf{Modifikation}:
Sei $f \in \A(\complex)$ mit
$\exists_{M > 0} \exists_{N \in \natural} \forall_{z \in \complex}\;
|f(z)| \le M (|z|^N + 1)$.\\
Dann ist $f$ ein Polynom vom Grad $\le N$.

Mit dieser Modifikation kann man relativ einfach den Hauptsatz der Algebra
beweisen.

\textbf{Satz (Multiplikation von Potenzreihen)}:\\
Seien $p(z) = \sum_{k=0}^\infty c_k' z^k$ und
$q(z) = \sum_{k=0}^\infty c_k'' z^k$ Potenzreihen
mit Konvergenzradius $R', R'' > 0$.\\
Dann ist $p(z) q(z) = \sum_{n=0}^\infty d_n z^n$ eine Potenzreihe
mit Konvergenzradius
$R \ge \min\{R', R''\}$, wobei
$d_n = \sum_{k=0}^n c_k' c_{n-k}''$.

\linie

\textbf{Satz von \name{Morera}}:
Seien $G \subset \complex$ ein Gebiet und $f\colon G \rightarrow \complex$
stetig.
$\Delta$ bezeichne ein Dreieck (mit Rand und Innerem).
Für alle $\Delta \subset G$ gelte $\oint_{\partial\Delta} f(z)\dz = 0$.\\
Dann ist $f \in \A(G)$.

\textbf{komplexe Halbebenen}:
$\complex_+ := \{z \in \complex \;|\; \Im z > 0\}$,
$\complex_- := \{z \in \complex \;|\; \Im z < 0\}$,\\
$\overline{\complex_+} := \{z \in \complex \;|\; \Im z \ge 0\}$,
$\overline{\complex_-} := \{z \in \complex \;|\; \Im z \le 0\}$

\textbf{Teilraumtopologie}:
$G \subset \overline{\complex_+}$ heißt
\emph{of"|fen in der induzierten Topologie}, falls\\
$\exists_{\widetilde{G} \subset \complex \text{ of"|fen}}\;
G = \overline{\complex_+} \cap \widetilde{G}$.

\textbf{Satz (\name{Schwarz}sches Spiegelungsprinzip)}:\\
Seien $G \subset \overline{\complex_+}$ of"|fen in der induzierten Topologie,
$f \colon G \rightarrow \complex$ stetig und
$f|_{G \cap \complex_+} \in \A(G \cap \complex_+)$.
Seien außerdem $\overline{G} := \{\overline{z} \;|\; z \in G\}$
und $\widetilde{G} := G \cup \overline{G}$.\\
Falls $\forall_{z \in \real \cap G}\; f(z) \in \real$ gilt, dann gibt es
$\widetilde{f}\colon \widetilde{G} \rightarrow \complex$ mit
$\widetilde{f} \in \A(\widetilde{G})$, wobei
$\widetilde{f}(z) =$ \matrixsize{$\begin{cases}f(z) & z \in G\\
\overline{f(\overline{z})} & z \in \overline{G}\end{cases}$}.

\section{%
    Nullstellen analytischer Funktionen%
}

\textbf{Nullstellenmenge}:
Seien $G \subset \complex$ ein Gebiet und $f \in \A(G)$.\\
Dann heißt $Z(f) := \{z \in G \;|\; f(z) = 0\}$
\emph{Nullstellenmenge} von $f$.

\textbf{Ordnung von Nullstellen}:
Die Nullstelle $a \in Z(f)$ besitzt die
\emph{(endliche) Ordnung $m \in \natural$}, falls
$f(a) = f'(a) = \dotsb = f^{(m-1)}(a) = 0$ und
$f^{(m)}(a) \not= 0$.

\linie

\textbf{Lemma}:
Seien $a \in \complex$, $f \in \A(U_r(a))$ für ein $r > 0$ mit
$f(z) = \sum_{k=0}^\infty c_k (z - a)^k$ für $z \in U_r(a)$.\\
Dann sind äquivalent:
\begin{enumerate}
    \item
    $a$ ist eine Nullstelle der Ordnung $m$.

    \item
    $f(z) = \sum_{k=m}^\infty c_k (z - a)^k$ mit $c_m \not= 0$

    \item
    $f(z) = (z - a)^m g(z)$ mit $g \in \A(U_r(a))$ und $g(a) \not= 0$

    \item
    $\exists \lim_{z \to a} (z - a)^{-m} f(z) \not= 0$
\end{enumerate}

\textbf{Folgerung}:
Seien $f \in \A(U_r(a))$ und $a \in Z(f)$.\\
Dann ist entweder $a$ eine Nullstelle endlicher Ordnung oder
$f(z) \equiv 0$ für $z \in U_r(a)$.

\textbf{Folgerung}:
Seien $f \in \A(U_r(a))$ und $a \in Z(f)$.\\
Dann ist $a$ eine isolierte Nullstelle
(d.\,h. $\exists_{\varepsilon > 0} \forall_{z \in U_r(a),\, z \not= a}\;
f(z) \not= 0$) genau dann, wenn $a$ eine Nullstelle endlicher Ordnung ist.

\linie

\textbf{Identitätssatz}:
Seien $G \subset \complex$ ein Gebiet und $f \in \A(G)$ mit
$\acc(Z(f)) \cap G \not= \emptyset$.\\
Dann gilt $f(z) \equiv 0$ für $z \in G$.

\textbf{Folgerung}:\\
Seien $G \subset \complex$ ein Gebiet und $f, g \in \A(G)$ mit
$\exists_{M \subset G}\; f|_M = g|_M$ und $\acc(M) \cap G \not= \emptyset$.\\
Dann gilt $f(z) \equiv g(z)$ für $z \in G$.

\textbf{analytische Fortsetzung}:
Seien $G, \widetilde{G} \subset \complex$ Gebiete mit $G \subset \widetilde{G}$,
$f \in \A(G)$ und $\widetilde{f} \in \A(\widetilde{G})$.\\
Dann heißt $\widetilde{f}$ \emph{analytische Fortsetzung} von $f$,
falls $\widetilde{f}|_G = f|_G$.

Falls zu gegebenen Gebieten $G, \widetilde{G} \subset \complex$ und
$f \in \A(G)$ eine analytische Fortsetzung
$\widetilde{f} \in \A(\widetilde{G})$ von $f$
auf $\widetilde{G}$ existiert, so ist diese eindeutig bestimmt.

\linie

\emph{Beispiel}:
Die \emph{\name{Riemann}sche Zeta-Funktion}
$\zeta(z) = \sum_{n=1}^\infty \frac{1}{n^z}$ für $\Re(z) > 1$ kann
analytisch auf $\complex$ fortgesetzt werden.
Die berühmte \emph{\name{Riemann}sche Vermutung} besagt, dass alle
nicht-trivialen Nullstellen
der Fortsetzung auf der Geraden mit $\Re(z) = \frac{1}{2}$ liegen.

\emph{Beispiel}:
Die Funktion $f(z) = \sum_{k=0}^\infty z^k$ für $|z| < 1$ bzw.
$G = \{z \in \complex \;|\; |z| < 1\}$ kann mittels
$\widetilde{f}(z) = \frac{1}{1 - z}$ auf
$\widetilde{G} = \complex \setminus \{1\}$ analytisch fortgesetzt werden.

\section{%
    Das Maximumsprinzip%
}

\textbf{Maximumsprinzip für Kreise}:
Seien $a \in \complex$, $R > 0$ und $f \in \A(U_R(a))$ mit\\
$\forall_{z \in U_R(a)}\; |f(a)| \ge |f(z)|$.
Dann gilt $f(z) \equiv f(a)$ für $z \in U_R(a)$.

\textbf{Maximumsprinzip für allgemeine Gebiete}:\\
Seien $G \subset \complex$ ein beschränktes Gebiet und
$f\colon \overline{G} \rightarrow \complex$ stetig mit $f|_G \in \A(G)$.\\
Dann nimmt $|f(z)|$ ein globales Maximum auf dem Rand $\partial G$ an.

\textbf{Folgerung}:
Für $f \in \A(U_R(0))$ mit $f(0) = 0$ und $|f(z)| \le M$ für alle
$z \in U_R(0)$ gilt die Abschätzung $|f(z)| \le \frac{M}{R} |z|$
für alle $z \in U_R(0)$.

\section{%
    Singularitäten%
}

\textbf{Menge der isolierten Singularitäten}:
Seien $G \subset \complex$ ein Gebiet und $f \in \A(G)$.\\
Dann heißt $J = \iso(\complex \setminus G)$ die
\emph{Menge der isolierten Singularitäten} von $f$.

Es gilt $a \in J$ genau dann, wenn
$a \notin G$ und
$\exists_{\varepsilon > 0}\; U_\varepsilon(a) \setminus \{a\} \subset G$.

\textbf{Arten der Singularität}:
\begin{itemize}
    \item
    $a \in J$ heißt \emph{hebbar}, falls
    $\exists_{w \in \complex}\; \widetilde{f} \in \A(G \cup \{a\})$ mit
    $\widetilde{f}(a) := w$ und $\widetilde{f}(z) := f(z)$ für $z \not= a$.

    \item
    $a \in J$ heißt \emph{Polstelle der Ordnung $m \in \natural$}, falls
    $a$ hebbare Singularität von $(z - a)^m f(z)$ und $m$ kleinstmöglich ist.

    \item
    $a \in J$ heißt \emph{wesentlich}, falls $a$ weder hebbar noch
    Polstelle endlicher Ordnung ist.
\end{itemize}

\textbf{meromorph}:
Besitzt $f \in \A(G)$ nur isolierte Singularitäten, welche hebbar bzw.
Polstellen endlicher Ordnung sind, so nennt man $f$ \emph{meromorph} auf
$G \cup J$.

\emph{Beispiel}:
$f(z) = \frac{p(z)}{q(z)}$ ist meromorph auf $\complex$, wenn
$p$ und $q$ Polynome mit $q(z) \not\equiv 0$ sind.

\emph{Beispiel}:
Ist $f(z) = \frac{g(z)}{h(z)}$, wobei
$g, h \in \A(U_r(a))$, $r > 0$ mit
$a$ Nullstelle der Ordnung $m$ für $h$ und
$a$ Nullstelle der Ordnung $n$ für $g$ ist, so gilt für
\begin{itemize}
    \item
    $m > n$, dass $a$ Polstelle der Ordnung $m - n$ für $f$ ist,

    \item
    $m = n$, dass $f$ eine hebbare Singularität in $a$ besitzt und
    $\widetilde{f}(a) \not= 0$, und

    \item
    $m < n$, dass $f$ eine hebbare Singularität in $a$ und
    $\widetilde{f}$ in $a$ eine Nullstelle der Ordnung\\
    $n - m$ besitzt.
\end{itemize}

\linie

\textbf{\name{Laurent}-Reihe}:
Eine \emph{\name{Laurent}-Reihe} ist eine Reihe der Form
$f(z) = \sum_{k=-\infty}^{+\infty} c_k (z - a)^k$.\\
Sie kann geschrieben werden als
$f(z) = \sum_{k=-\infty}^{-1} c_k (z - a)^k +
\sum_{k=0}^{+\infty} c_k (z - a)^k$, wobei
der erste Summand als \emph{Hauptteil} und der zweite Summand als
\emph{Nebenteil} bezeichnet wird.

Die Laurent-Reihe konvergiert genau dann, wenn Haupt- und Nebenteil jeweils
für sich konvergieren.

Der Nebenteil ist eine gewöhnliche Potenzreihe mit Konvergenzradius
$R = \frac{1}{\limsup_{k \to \infty} \sqrt[k]{|c_k|}}$.\\
Der Hauptteil ist ebenfalls eine Potenzreihe in $w = \frac{1}{z - a}$
mit Konvergenzradius $\frac{1}{r}$, wobei\\
$r = \limsup_{k \to \infty} \sqrt[|k|]{|c_k|}$.

In $z$ konvergiert somit der Nebenteil für $|z - a| < R$ und der Hauptteil
für $|z - a| > r$.
Im Falle $r < R$ bildet sich somit ein Kreisring
$K_{rR}(a) := \{z \in \complex \;|\; r < |z - a| < R\}$,
in dem die Laurent-Reihe konvergiert
(außerhalb divergiert sie, unbestimmtes Verhalten auf dem Rand).
Für $r > R$ divergiert die Laurent-Reihe überall.

Zusätzlich gilt $f \in \A(K_{rR})$
(da Haupt- und Nebenteil dort holomorph sind) und
die Laurent-Reihe ist gliedweise dif"|ferenzierbar mit
$f'(z) = \sum_{k=-\infty,\;k\not=0}^{+\infty} k c_k (z - a)^{k-1}$.

\textbf{Stammfunktion von \name{Laurent}-Reihen}:
Eine Laurent-Reihe $f(z) = \sum_{k=-\infty}^{+\infty} c_k (z - a)^k$
besitzt eine Stammfunktion $F(z)$ in $K_{rR}$ genau dann, wenn
$c_{-1} = 0$ ist.
Durch gliedweises Auf"|leiten erhält man für diesen Fall
$F(z) = \sum_{k=-\infty,\;k\not=-1}^{+\infty} \frac{c_k}{k + 1} (z - a)^{k+1}$.

\linie
\pagebreak

\textbf{Berechnung der Koef"|fizienten $c_k$ aus $f$}:
Für $a = 0$ und $0 \le r < R$ lässt sich die Laurent-Reihe schreiben als
$f(z) = \sum_{k=-\infty}^{+\infty} c_k z^k =
\sum_{k=-\infty}^{-2} c_k z^k + \frac{c_{-1}}{z} +
\sum_{k=0}^{+\infty} c_k z^k$.
Ist $\Gamma$ ein Pfad in $K_{rR}(0)$, der homotop zum einfachen, mathematisch
positiven Umlauf von $0$ in $K_{rR}(0)$ ist, so gilt aufgrund der
gleichmäßigen Konvergenz der Laurent-Reihe\\
$\oint_\Gamma f(z)\dz =
\oint_\Gamma (\sum_{k=-\infty}^{-2} c_k z^k)\dz +
\oint_\Gamma (\sum_{k=0}^{+\infty} c_k z^k)\dz +
\oint_\Gamma \frac{c_{-1}}{z}\dz$\\
$= \sum_{k=-\infty,\;k\not=-1}^{+\infty} c_k \oint_\Gamma z^k \dz +
\oint_\Gamma \frac{c_{-1}}{z}\dz =
c_{-1} \cdot 2\pi\i$,
da $\oint_\Gamma z^k \dz = 0$ für $k \not= -1$
und $\oint_\Gamma \frac{1}{z}\dz = 2\pi\i$.

\textbf{Residuum}:
Man bezeichnet
$c_{-1} = \frac{1}{2\pi\i} \cdot \oint_\Gamma f(z) \dz =: \Res(f)$ als
das \emph{Residuum} von $f$.

Analog kann man durch Indexverschiebung die Formel
$c_k = \frac{1}{2\pi\i} \cdot \oint_\Gamma f(z)(z - a)^{-1-k}\dz$ herleiten.\\
Für $r < \varrho < R$ ergibt sich daraus direkt die Abschätzung\\
$|c_k| \le \frac{1}{2\pi}
\left|\oint_{\partial U_\varrho(a)} f(z)(z - a)^{-1-k}\dz\right| \le
M \varrho^{-k}$ mit $M : = \sup_{z \in \partial U_\varrho(a)} |f(z)|$.

\textbf{Satz}:
Sei $f \in \A(K_{rR}(a))$ mit $a \in \complex$ und $0 \le r < R$.\\
Dann ist $f$ in $K_{rR}(a)$ als Laurent-Reihe darstellbar.

\linie

\textbf{Zusammenfassung}:
\begin{itemize}
    \item
    \textbf{Potenzreihen}:
    Es ist $f \in \A(U_R(a))$ genau dann, wenn
    $f(z) = \sum_{k=0}^\infty c_k (z - a)^k$ als Potenzreihe darstellbar ist.
    Diese konvergiert mindestens in $U_r(a)$ und ist gliedweise
    dif"|ferenzierbar.
    Es existiert immer eine Stammfunktion (durch gliedweises Auf"|leiten).

    \item
    \textbf{\name{Laurent}-Reihen}:
    Es ist $f \in \A(K_{rR}(a))$ genau dann, wenn
    $f(z) = \sum_{k=-\infty}^\infty c_k (z - a)^k$ als
    Laurent-Reihe darstellbar ist.
    Diese konvergiert mindestens in $K_{rR}(a)$ und ist gliedweise
    dif"|ferenzierbar.
    Es existiert eine Stammfunktion genau dann, wenn $c_{-1} = 0$.
\end{itemize}

\linie

\textbf{Spezialfall $r = 0$}:
In diesem Fall ist $K_{0R} = U_R(a) \setminus \{a\}$
und für $f \in \A(U_R(a) \setminus \{a\})$ ist $f$ als Laurent-Reihe
$f(z) = \sum_{k=-\infty}^{+\infty} c_k (z - a)^k$ für $z \not= a$ darstellbar.
\begin{itemize}
    \item
    \emph{hebbare Singularität in $a$}:
    Falls die Singularität von $f$ in $a$ hebbar ist,
    gilt $\widetilde{f} \in \A(U_R(a))$ mit
    $\widetilde{f}(z) = f(z)$ für $z \not= a$ und $\widetilde{f}(a) = B$.
    $\widetilde{f}$ ist eine analytische Fortsetzung von $f$, daher stimmen
    die Potenzreihen überein, also
    $f(z) = \widetilde{f}(z) = \sum_{k=0}^\infty c_k (z - a)^k$.\\
    Daher hat $f$ eine hebbare Singularität in $a$ genau dann, wenn der
    Hauptteil verschwindet.

    \item
    \emph{Polstelle der Ordnung $m$ in $a$}:
    In diesem Fall hat $(z - a)^m f(z)$ eine hebbare Singularität in $a$,
    d.\,h. $(z - a)^m f(z) = \sum_{k=0}^\infty \widetilde{c}_k (z - a)^k$
    ist als Potenzreihe darstellbar.
    Daher gilt $f(z) = \sum_{k=-m}^\infty c_k (z - a)^k$ mit
    $c_k = \widetilde{c}_{k+m}$.\\
    Daher hat $f$ eine Polstelle der Ordnung $m$ in $a$ genau dann, wenn der
    Hauptteil nur endlich viele Summanden besitzt und der Term bei
    $(z - a)^{-m}$ nicht verschwindet.
    Für $z \to a$ geht $|f(z)| \to \infty$ (und zwar wie $(z - a)^{-m}$).

    \item
    \emph{wesentliche Singularität in $a$}:
    Dieser Fall tritt ein genau dann, wenn die anderen beiden Fälle nicht
    gelten, d.\,h. genau dann, wenn
    der Hauptteil unendlich viele Summanden besitzt.\\
    Man kann zeigen:
    Besitzt $f$ in $a$ eine wesentliche Singularität, dann liegt das Bild\\
    $f(U_\varepsilon(a) \setminus \{a\})$ jeder beliebig kleinen
    $\varepsilon$-Umgebung um $a$ dicht in der komplexen Ebene $\complex$.
\end{itemize}

\pagebreak

\section{%
    Residuensatz und Residuenkalkül%
}

Seien $G \subset \complex$ ein Gebiet, $f \in \A(G \setminus \{a\})$
und $\Gamma \subset G \setminus \{a\}$ homotop in $G \setminus \{a\}$ zum
einfachen, mathematisch positiven Umlauf von $a$.
$f$ lässt sich als Laurent-Reihe
$f(z) = \sum_{k=-\infty}^{+\infty} c_k (z - a)^k$
darstellen.
Dabei gilt $c_{-1} = \frac{1}{2\pi\i} \oint_\Gamma f(z)\dz$.

\textbf{Residuum}:
Man bezeichnet $c_{-1} =: \Res_a(f)$ als
das \emph{Residuum} von $f$ im Punkt $a$.

Man kann die Integralformel auch umkehren und bei bekanntem Residuum das
Integral berechnen durch
$\oint_\Gamma f(z)\dt = 2\pi\i \cdot \Res_a(f)$.

Falls $\Gamma$ mehrere isolierte Singularitäten umläuft, kann man den
Pfad auf"|teilen und die entstehenden Integrale summieren.
Falls Singularitäten mehrfach umlaufen werden, müssen diese natürlich auch
entsprechend der Windungszahl (Umlaufrichtung beachten!) gezählt werden.
Somit gelangt man zum folgenden Satz.

\textbf{Residuensatz}:
Seien $G \subset \complex$ ein Gebiet, $J = \{a_1, \dotsc, a_N\} \subset G$,
$f \in \A(G \setminus J)$ und\\
$\Gamma \subset G \setminus J$ in $G$ homotop zum Nullpfad.\\
Dann gilt $\oint_\Gamma f(z)\dz =
\sum_{k=1}^N (2\pi\i) \cdot n(\Gamma, a_k) \cdot \Res_{a_k}(f)$.

\linie

\textbf{Residuenkalkül (Bestimmung des Residuums)}:
\begin{itemize}
    \item
    Falls $f$ in $a_k$ eine hebbare Singularität hat, gilt
    $\Res_{a_k}(f) = c_{-1}(a_k) = 0$ (siehe oben).

    \item
    Falls $f$ in $a_k$ eine Polstelle der Ordnung $m$ hat, gilt\\
    $(z - a_k)^m f(z) =
    c_{-m} + (z - a_k) c_{-m+1} + \dotsb + (z - a_k)^{m-1} c_{-1} + \dotsb$.\\
    Man erhält also $c_{-1}$ durch $(m - 1)$-fache Dif"|ferentiation
    und Grenzwertbildung:\\
    $\lim_{z \to a_k} \frac{d^{m-1}}{dz^{m-1}} (z - a_k)^m f(z) =
    (m - 1)! \cdot c_{-1}$.
    Man erhält die wichtige Formel\\
    $\Res_{a_k}(f) = \frac{1}{(m - 1)!} \cdot
    \lim_{z \to a_k} \frac{d^{m-1}}{dz^{m-1}} (z - a_k)^m f(z)$.

    \item
    Für wesentliche Singularitäten gibt es keine einheitliche Vorgehensweise.
\end{itemize}

\linie

\emph{Beispiel}:
Seien $p(z)$ und $q(z)$ Polynome mit $\deg q(z) \ge \deg p(z) + 2$.
$J$ seien die Nullstellen von $q$, wobei $J \cap \real = \emptyset$ gelten
soll, d.\,h. keine Nullstelle ist reell.
Man betrachtet nun die Funktion $f(z) = \frac{p(z)}{q(z)}$ und möchte
das Integral $I = \int_{-\infty}^{+\infty} \frac{p(x)}{q(x)} \dx$ berechnen.
Das uneigentliche Integral existiert, da $f$ auf $\real$ stetig ist
($J \cap \real = \emptyset$) und $|f(z)| = \O(|x|^{-2})$ für $|x| \to \infty$.
Dabei ist $I = \lim_{R \to +\infty} I_R$ mit
$I_R = \int_{-R}^{+R} \frac{p(x)}{q(x)} \dx$.

Man definiert nun $\Gamma_R^{(1)}$ als die Kurve in $\complex$
von $-R$ bis $+R$ und $\Gamma_R^{(2)}$ als den Halbkreis mit Mittelpunkt $0$
und Radius $R$ von $+R$ bis $-R$.
Dann ist $\Gamma_R = \Gamma_R^{(1)} \cup \Gamma_R^{(2)}$ ein
geschlossener Pfad.
Man stellt nun drei Beobachtungen an:
\begin{itemize}
    \item
    Für alle $R > R_1$ mit $R_1$ groß genug gilt
    $J \cap \Gamma_R^{(2)} = \emptyset$, da $J$ endlich ist.

    \item
    Es gilt $\int_{\Gamma_R^{(1)}} f(z)\dz + \int_{\Gamma_R^{(2)}} f(z)\dz =
    \oint_{\Gamma_R} f(z)\dz = 2\pi\i \cdot \sum_{\Im a_k > 0} \Res_{a_k}(f)$
    für $R > R_1$.

    \item
    Für alle $R = |z| > R_2$ mit $R_2$ groß genug gilt
    $|f(z)| \le C |z|^{n-m}$, denn\\
    $f(z) = \frac{c_n z^n + \dotsb + c_1 z + c_0}
    {\widetilde{c}_m z^m + \dotsb \widetilde{c}_1 z + \widetilde{c}_0} =
    \frac{c_n}{\widetilde{c}_m} z^{n-m} \cdot \frac{1 + \O(1/z)}{1 + \O(1/z)}$
    für $|z| \to \infty$.
    Daraus folgt mit $n - m \le -2$, dass
    $\left|\int_{\Gamma_R^{(2)}} f(z)\dz\right| \le \pi R \cdot CR^{-2} \to 0$
    für $R \to \infty$.
\end{itemize}

Damit gilt für $I_R + \int_{\Gamma_R^{(2)}} f(z)\dz =
2\pi\i \cdot \sum_{\Im a_k > 0} \Res_{a_k}(f)$ im Grenzwertübergang für
$R \to \infty$, dass
$I = \int_{-\infty}^{+\infty} \frac{p(x)}{q(x)} \dx =
2\pi\i \cdot \sum_{\Im a_k > 0} \Res_{a_k}(f)$.

\linie

\textbf{Hinweis}:
Ist $a_k$ NS von $q$ mit Ordnung $1$, so gilt
$\Res_{a_k}(f) = \lim_{z \to a_k} (z - a_k) \frac{p(z)}{q(z) - q(a_k)} =
\frac{p(a_k)}{q'(a_k)}$.

\emph{Beispiel}:
Mit eben Gesagtem gilt $\int_{-\infty}^{+\infty} \frac{1}{1 + x^2} \dx =
2\pi\i \cdot \frac{1}{2\i} = \pi$.

\pagebreak

\section{%
    Das Zählen von Pol- und Nullstellen%
}

"`Verschiebt"' man eine reelle Funktion ein wenig, dann ändert sich die
Zahl der Nullstellen meistens nicht (wenn sie nicht mehrfach sind).
Das Beispiel $z^2 + c$ zeigt allerdings, dass bei größeren Störungen
die Zahl der reellen Nullstellen zwischen $2$, $1$ und $0$ variieren kann.
Nicht so in der komplexen Ebene:
Hier gibt es immer zwei Nullstellen, die sich zunächst auf der reellen Achse
befinden, zum Ursprung wandern, sich dort vereinigen und dann wieder auf
der imaginären Achse trennen.

\linie

\textbf{Residuen der logarithmischen Ableitung}:
Für $\varepsilon > 0$ betrachtet man $f \in \A(U_\varepsilon(a))$, wobei
$a$ eine Nullstelle von $f$ der Ordnung $m$ sein soll.
Man will nun die sog. \textbf{logarithmische Ableitung} $\frac{f'(z)}{f(z)}$
betrachten (der Name kommt daher, weil dies die Ableitung von $\ln(f(z))$ ist).
Die Potenzreihe von $f$ hat die Form
$f(z) = c_m (z - a)^m + c_{m+1} (z - a)^{m+1} + \dotsb$,
die von $f'$ ist dann
$f'(z) = m c_m (z - a)^{m-1} + (m + 1) c_{m+1} (z - a)^m + \dotsb$.
Daraus folgt\\
$\frac{f'(z)}{f(z)} =
\frac{m c_m (z - a)^{m-1} + (m + 1) c_{m+1} (z - a)^m + \dotsb}
{c_m (z - a)^m + c_{m+1} (z - a)^{m+1} + \dotsb} =
\frac{m c_m (z - a)^{m-1}}{c_m (z - a)^m} \cdot
\frac{1 + \frac{m + 1}{m} \frac{c_{m+1}}{c_m} (z - a) + \dotsb}
{1 + \frac{c_{m+1}}{c_m} (z - a) + \dotsb} =
\frac{m}{z - a} (1 + r(z))$ mit $r \in \A(U_\varepsilon(a))$, $r(a) = 0$ und
$\varepsilon$ klein genug.

Daher ist $\Res_a(\frac{f'}{f}) = m$ die Ordnung der Nullstelle von $f$.\\
Ist $a$ dagegen eine Polstelle von $f$ der Ordnung $n$, so erhält man analog
$\Res_a(\frac{f'}{f}) = -n$.

\linie

\textbf{Lemma}:
Sei $G \subset \complex$ ein einfach zusammenhängendes Gebiet und
$\Omega \subset G$ ebenfalls\\
ein einfach zusammenhängendes Gebiet mit
$\C^1$-Rand $\Gamma = \partial \Omega$.\\
Seien außerdem $f \in \A(G \setminus J)$ mit
$J = \{a_1, \dotsc, a_k\} \subset G$ die
Polstellen von $f$ der Ordnung $n_1, \dotsc, n_k$ und
$\{b_1, \dotsc, b_r\} \subset G$ die Nullstellen von $f$
der Ordnung $m_1, \dotsc, m_r$.\\
Dann gilt $\frac{1}{2\pi\i} \oint_\Gamma \frac{f'(z)}{f(z)} \dz =
\sum_{b_\ell \in \Omega} m_\ell - \sum_{a_\ell \in \Omega} n_\ell$.

\textbf{Satz von \name{Rouché}}:
Sei $G \subset \complex$ ein einfach zusammenhängendes Gebiet und
$\Omega \subset G$ ebenfalls\\
ein einfach zusammenhängendes Gebiet mit
$\C^1$-Rand $\Gamma = \partial \Omega$.\\
Seien außerdem $f, g \in \A(G)$ mit
$\forall_{z \in \Gamma}\; |g(z)| < |f(z)|$.\\
Dann gilt $\sum_{b_\ell(f) \in \Omega} m_\ell(f) =
\sum_{b_\ell(f + g) \in \Omega} m_\ell(f + g)$, wenn
$b_\ell(h)$ die Nullstellen einer Funktion $h$ und $m_\ell(h)$ deren Ordnungen
bezeichnen.

\section{%
    Harmonische Funktionen%
}

Seien $G \subset \complex$ ein Gebiet und $f \in \A(G)$.
$f$ lässt sich darstellen als\\
$f(z) = f(x, y) = u(x, y) + \i v(x, y)$ mit
reellwertigen Funktionen $u = \Re f$ und $v = \Im f$.\\
Aufgrund der komplexen Dif"|ferenzierbarkeit gelten die
Cauchy-Riemann-Gleichungen $u'_x = v'_y$ und $u'_y = -v'_x$.
Wegen der zweifachen stetigen Dif"|ferenzierbarkeit von $u$ und $v$
($f$ ist beliebig oft komplex dif"|fb.) gilt daher
$(u'_x)'_x = (v'_y)'_x = (v_x')'_y = (-u'_y)'_y$, also
$u'_{xx} = -u'_{yy}$ und $\Delta u = 0$
(mit dem Laplace-Operator $\Delta u := u'_{xx} + u'_{yy}$).
Analog zeigt man $\Delta v = 0$.

\textbf{harmonische Funktion}:
Eine Funktion $u$ heißt \emph{harmonisch},
falls alle zweiten partiellen Ableitungen
existieren und stetig sind sowie $\Delta u = 0$.\\
Die Menge der harmonischen Funktionen auf einem Gebiet $G$ bezeichnet man
mit $\H(G)$.

\emph{Beispiel}:
Ist $f = u + \i v \in \A(G)$, so ist $u, v \in \H(G)$ mit $u = \Re f$ und
$v = \Im f$.

Aus $u, v \in \H(G)$ folgt i.\,A. aber nicht $f = u + \i v \in \A(G)$.
Ein Gegenbeispiel ist\\
$f(x, y) = (x^2 - y^2) (1 + \i)$, d.\,h.
$u(x, y) = v(x, y) = x^2 - y^2$ sowie
$u'_x = v'_x = 2x$ und $u'_y = v'_y = 2y$.\\
Damit die CR-Gleichungen erfüllt sind, muss $2x = 2y$ und
$2x = -2y$ gelten, also $x = y = 0$.
Somit ist $f$ in keinem Punkt komplex dif"|ferenzierbar
(keine Umgebung vorhanden).

\linie

\textbf{harmonisch konjugiert}:
Sei $u \in \H(G)$.\\
Dann heißt eine Funktion $v \in \H(G)$ \emph{harmonisch konjugiert} zu $u$,
falls $f = u + \i v \in \A(G)$.

Das harmonische Konjugat von $u \in \H(G)$ ist bis auf Konstanten eindeutig:
Falls $v_1$ und $v_2$ harmonisch konjugiert zu $u$ sind, gilt
$f_1 = u + \i v_1 \in \A(G)$ und $f_2 = u + \i v_2 \in \A(G)$, also
$f = f_1 - f_2 = \i (v_1 - v_2) \in \A(G)$.
Wegen $\Re f \equiv 0$ ist $f \equiv \const$, also $v_1 - v_2 \equiv c$.\\
Man kann also aus Kenntnis des Realteils einer Funktion (falls existent) den
Imaginärteil bis auf Konstanten rekonstruieren.

\textbf{Satz}:
Sei $G \subset \complex$ ein einfach zusammenhängendes Gebiet.
Dann existiert zu jedem $u \in \H(G)$ eine harmonisch konjugierte Funktion
$v \in \H(G)$ (d.\,h. $f = u + \i v \in \A(G)$).

Der Beweis gibt eine Methode zur Rekonstruktion des Imaginärteils
(analog geht das natürlich mit dem Realteil).
Als Beispiel wird $G = \real^2$ und $u(x, y) = xy - x$ verwendet.

\begin{enumerate}
    \item
    verifizieren, dass die gegebene Funktion harmonisch ist:\\
    $u'_{xx} = 0 = u'_{yy}$

    \item
    Funktion $g = w_r + \i w_i$ mit $w_r = u'_x$ und $w_i = -u'_y$
    konstruieren:\\
    $w_r = y - 1$, $w_i = -x$, also
    $g(x, y) = y - 1 - \i x$

    \item
    $g$ in Abhängigkeit von $z = x + \i y$ schreiben:\\
    $g(z) = -\i z - 1$

    \item
    $g$ auf"|leiten:\\
    $f(z) = -z - \frac{\i}{2} z^2$ (plus Konstante), also
    $f(z) = -(x + \i y) - \frac{\i}{2} (x + \i y)^2$\\
    $= x(y - 1) + \i (-y - \frac{1}{2} x^2 + \frac{1}{2} y^2)$,
    dies ist eine holomorphe Funktion mit $u$ als Realteil
\end{enumerate}

\linie

\textbf{Integralformel von \name{Poisson}}:\\
Seien $G \subset \complex$ ein Gebiet, $u \in \H(G)$,
$\overline{U_R(0)} \subset G$, $0 \le r < R$ und $0 \le \theta < 2\pi$.\\
Dann gilt $u(r e^{\i\theta}) = \frac{1}{2\pi}
\int_0^{2\pi} \frac{(R^2 - r^2) u(R e^{\i t})}
{R^2 - 2rR \cos(\theta - t) + r^2} \dt$.

\textbf{Spezialfall (Mittelwertsatz für harmonische Funktionen)}:
$u(0) = \frac{1}{2\pi} \int_0^{2\pi} u(R e^{\i t}) \dt$

Insbesondere gilt, dass auf $u \in \H(G)$ und $u|_{\partial U_R} = 0$ folgt,
dass $u \equiv 0$ in $U_R$ ist, denn aus dem Mittelwertsatz kann
man das Maximumsprinzip für harmonische Funktionen folgern
(analog dem für holomorphe Funktionen).

\linie

\textbf{Transfer-Lemma}:
Seien $G, \widetilde{G} \subset \complex$ einfach zusammenhängende
Gebiete, $\psi\colon G \rightarrow \widetilde{G}$ bijektiv mit $\psi \in \A(G)$
und $\widetilde{u} \in \H(\widetilde{G})$.
Dann ist $u := \widetilde{u} \circ \psi \in \H(G)$.

\textbf{\name{Riemann}scher Abbildungssatz}:
Jedes einfach zusammenhängende Gebiet $G \subsetneqq \complex$ lässt sich
bijektiv mit einer holomorphen Funktion auf den Einheitskreis $U_1(0)$
abbilden.

\textbf{\name{Dirichlet}-Problem im Einheitskreis}:
Gesucht ist eine harmonische Funktion $u \in \H(G)$,
d.\,h. $\Delta u = 0$, wobei $u \in \C(\overline{G})$ und
$u$ auf dem Rand gegeben ist durch $u|_{\partial G} = u_0$.\\
Hat man zwei Lösungen des Problems, so ist die Dif"|ferenz harmonisch und
verschwindet auf dem Rand.
Nach dem Maximumsprinzip verschwindet sie auch im Inneren, d.\,h. die Lösung
ist eindeutig.\\
Für $G = U_1(0)$, also $u_0 = u_0(e^{\i t})$, ist die Lösung
$u(r e^{\i \theta}) = u_0(e^{i \theta})$ für $r = 1$ und\\
$u(r e^{\i \theta}) = \frac{1}{2\pi}
\int_0^{2\pi} P_{r,\theta}(t) u_0(e^{\i t}) \dt$ für $0 \le r < 1$ mit
dem \emph{\name{Poisson}-Kern}
$P_{r,\theta}(t) = \frac{1 - r^2}{1 - 2r\cos(\theta - t) + r^2}$.

\pagebreak
