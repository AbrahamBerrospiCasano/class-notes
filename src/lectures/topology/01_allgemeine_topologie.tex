\chapter{%
    Allgemeine Topologie%
}

\section{%
    Metrische Räume%
}

\subsection{%
    Euklidische Räume%
}

Betrachtet man die \textbf{stufenweise Erweiterung des Zahlensystems}
$\natural \rightarrow \integer \rightarrow \rational \rightarrow
\real \rightarrow \complex$, so sieht man, dass jeder der Schritte
$\natural \rightarrow \integer \rightarrow \rational$ und
$\real \rightarrow \complex$ durch Probleme motiviert wird,
die rein algebraischer Natur sind.
Anders beim Schritt $\rational \rightarrow \real$:
Hier handelt es sich um eine metrische/topologische Vervollständigung.
Dass dieser Schritt weitreichender ist, erkennt man auch daran, dass
$\rational$ im Gegensatz zu $\real$ abzählbar ist und in Computern gespeichert
werden kann.

Auf dem \textbf{euklidischen Raum} $\real^n$, ein $\real$-Vektorraum, kann man
ein \textbf{Skalarprodukt} \\
$\sp{-, -}\colon \real^n \times \real^n \rightarrow \real$
definieren durch $\sp{(x_1, \dotsc, x_n), (y_1, \dotsc, y_n)} =
x_1 y_1 + \dotsb + x_n y_n$. \\
Es erfüllt die Skalarprodukt-Axiome
\begin{enumerate}[label=(S\arabic*)]
    \item
    $\sp{x, x} \ge 0$ und $\sp{x, x} = 0 \;\Leftrightarrow\; x = 0$,

    \item
    $\sp{x, y} = \sp{y, x}$ sowie

    \item
    $\sp{x, \lambda y + \mu z} = \lambda \sp{x, y} + \mu \sp{x, z}$.
\end{enumerate}

Das Skalarprodukt induziert eine \textbf{Norm} $\norm{x} = \sqrt{\sp{x, x}}$,
die wiederum die Norm-Axiome
\begin{enumerate}[label=(N\arabic*)]
    \item
    $\norm{x} \ge 0$ und $\norm{x} = 0 \;\Leftrightarrow\; x = 0$,

    \item
    $\norm{\lambda x} = |\lambda| \norm{x}$ sowie

    \item
    $\norm{x + y} \le \norm{x} + \norm{y}$ erfüllt.
\end{enumerate}

Die Norm induziert dann eine \textbf{Metrik} $d(x, y) = \norm{x - y}$ mit den
Metrik-Axiomen
\begin{enumerate}[label=(M\arabic*)]
    \item
    $d(x, y) \ge 0$ und $d(x, y) = 0 \;\Leftrightarrow\; x = y$,

    \item
    $d(x, y) = d(y, x)$ sowie

    \item
    $d(x, z) \le d(x, y) + d(y, z)$.
\end{enumerate}

Man kann auch von den Axiomen ausgehen und auf einem $\real$-Vektorraum $V$ ein
\textbf{euklidisches Skalarprodukt} definieren
(eine positiv definite, symmetrische Bilinearform $\sp{-, -}$ auf $V$).
Damit wird $(V, \sp{-, -})$ zum \textbf{euklidischen Vektorraum}. \\
Analog kann man \textbf{normierte Vektorräume} definieren. Jedes Skalarprodukt
induziert eine Norm, diese erfüllt die
\textbf{Ungleichung von \name{Cauchy}-\name{Schwarz}}
$|\sp{x, y}| \le \norm{x} \cdot \norm{y}$.

\emph{Beispiele für euklidische Räume}: \\
$\Omega = \natural$,
$\ell^2(\Omega) = \{f\colon \Omega \rightarrow \real \;|\;
\sum_{x \in \Omega} f(x)^2 < \infty\}$
(Menge aller quadrat-summierbaren Abbildungen)
mit Skalarprodukt
$\sp{f, g} = \sum_{x \in \Omega} f(x)g(x)$.
Für $\Omega = \{1, \dotsc, n\}$ erhält man den
$\real^n$ mit üblichem Skalarprodukt. \\
$\C([0,1], \real)$ mit Skalarprodukt $\sp{f, g} = \int_0^1 f(x)g(x)\dx$.

\emph{Beispiele für normierte Räume}: \\
Jede euklidische Raum ist normiert mittels $\norm{x} = \sqrt{\sp{x, x}}$
(\begriff{induzierte Norm}). \\
Auf $\real^n$ wird für $1 \le p < \infty$ die Norm
$\norm{x}_p = (|x_1|^p + \dotsb + |x_n|^p)^{1/p}$ (\begriff{$p$-Norm})
definiert, ebenso $\norm{x}_\infty = \sup\{|x_1|, \dotsc, |x_n|\}$
(\begriff{Supremums-Norm}). \\
$\Omega = \natural$,
$\ell^\infty(\Omega) = \{f\colon \Omega \rightarrow \real \;|\;
\sup_{x \in \Omega} |f(x)| < \infty\}$
(Menge aller beschränkten Abbildungen),
$\norm{f}_\infty = \sup_{x \in \Omega} |f(x)|$ sowie
$\ell^p(\Omega) = \{f\colon \Omega \rightarrow \real \;|\;
\sum_{x \in \Omega} |f(x)|^p < \infty\}$
(Menge aller $p$-summierbaren Abbildungen, $1 \le p < \infty$),
$\norm{f}_p = (\sum_{x \in \Omega} |f(x)|^p)^{1/p}$.

\subsection{%
    Metrische Räume%
}

\begin{Def}{Metrik}
    Sei $X$ eine Menge.
    Eine Abbildung $d\colon X \times X \rightarrow \real$ heißt
    \begriff{Metrik}, falls sie M1, M2 und M3 erfült.
    Das Paar $(X, d)$ heißt dann \begriff{metrischer Raum}.
\end{Def}

\begin{Bsp}
    Jeder normierte Raum $(V, \norm{-})$ induziert eine Metrik durch
    $d(x, y) = \norm{x - y}$. \\
    $d\colon \real^n \times \real^n \rightarrow \real$,
    $d(x, y) = \norm{x - y}$ für $\real x = \real y$ und
    $d(x, y) = \norm{x} + \norm{y}$ für $\real x \not= \real y$
    ist eine Metrik (\begriff{französische Eisenbahn-Metrik}). \\
    $d\colon X \times X \rightarrow \real, d(x, y) = 0$ für $x = y$ und
    $d(x, y) = 1$ für $x \not= y$ ist eine Metrik \\
    (\begriff{diskrete Metrik}). \\
    Ist $d\colon X \times X \rightarrow \real$ eine Metrik, so auch
    $d^\ast(x, y) = \min\{d(x, y), 1\}$ (\begriff{gestutzte Metrik}) und
    $d'(x, y) = \frac{d(x, y)}{1 + d(x, y)}$ (\begriff{gestauchte Metrik}).
    $d$, $d^\ast$ und $d'$ sind topologisch äquivalent (s.\,u.). \\
    Ist $d\colon X \times X \rightarrow \real$ eine Metrik und
    $Y \subset X$, so auch $d_Y\colon Y \times Y \rightarrow \real$,
    $d_Y(x, y)= d(x, y)$ für $x, y \in Y$ (\begriff{Teilraum}).
\end{Bsp}

\subsection{%
    Konvergenz und Stetigkeit%
}

\begin{Def}{of"|fener/abgeschlossener Ball}
    Sei $(X, d)$ ein metrischer Raum.
    Für $x \in X$ und $r \in \real_{> 0}$ sei
    $B(x, r) := \{y \in X \;|\; d(x, y) < r\}$ bzw.
    $\overline{B}(x, r) := \{y \in X \;|\; d(x, y) \le r\}$
    der \begriff{of"|fene bzw. abgeschlossene Ball} um $x$ mit Radius $r$
    ($B(x, \varepsilon)$ heißt auch \begriff{$\varepsilon$-Umgebung um x}).
\end{Def}

\begin{Def}{of"|fene/abgeschlossene Menge}\\
    $O \subset X$ heißt \begriff{of"|fen} (bzgl. der Metrik $d$), falls
    $\forall_{x \in O} \exists_{\varepsilon > 0}\;
    B(x, \varepsilon) \subset O$. \\
    $A \subset X$ heißt \begriff{abgeschlossen} (bzgl. der Metrik $d$),
    falls $X \setminus A$ of"|fen ist.
\end{Def}

\begin{Def}{Umgebung}
    $U \subset X$ heißt \begriff{Umgebung} von $x \in X$, falls
    $\exists_{\varepsilon > 0}\; B(x, \varepsilon) \subset U$.
\end{Def}

\begin{Bsp}
    $O \subset X$ ist of"|fen genau dann, wenn sie Umgebung jedes ihrer Punkte
    ist. \\
    $B(x, r)$ ist of"|fen und $\overline{B}(x, r)$ ist abgeschlossen.
\end{Bsp}

\begin{Satz}{System aller of \!\!fenen Mengen}
    Das System $J \subset P(X)$ aller of"|fenen Mengen erfüllt
    \begin{enumerate}[label=(O\arabic*)]
        \item
        $\emptyset, X \in J$,

        \item
        für alle $O_1, \dotsc, O_n \in J$, $n \in \natural$ gilt, dass
        $O_1 \cap \dotsb \cap O_n \in J$, sowie

        \item
        für alle $O_i \in J$, $i \in I$ gilt, dass
        $\bigcup_{i \in I} O_i \in J$.
    \end{enumerate}
\end{Satz}

\linie

\begin{Def}{Konvergenz}
    Seien $(X, d)$ ein metrischer Raum und $a \in X$. \\
    Eine Folge $\{x_n\}_{n \in \natural}$ in $X$
    \begriff{konvergiert} gegen $a$, falls
    $\forall_{\varepsilon > 0} \exists_{m \in \natural} \forall_{n \ge m}\;
    d(a, x_n) < \varepsilon$. \\
    Das ist der Fall genau dann, wenn jede Umgebung von $a$ \begriff{fast alle}
    Folgenglieder enthält \\
    (d.\,h. alle bis auf endlich viele).
    Der Grenzwert ist (falls existent) eindeutig.
\end{Def}

\begin{Def}{Stetigkeit}
    Seien $(X, d)$ und $(Y, e)$ metrische Räume. \\
    Eine Abbildung $f\colon X \rightarrow Y$ heißt \begriff{stetig}, falls
    $\forall_{x \in X} \forall_{\varepsilon > 0} \exists_{\delta > 0}
    \forall_{x' \in B(x, \delta)}\; f(x') \in B(f(x), \varepsilon)$. \\
    Das ist der Fall genau dann, wenn für jede of"|fene Menge $O$ in $Y$
    $f^{-1}(O)$ of"|fen in $X$ ist.
\end{Def}

\linie

\begin{Def}{(topologisch) äquivalente Metriken}
    Zwei Metriken $d, e\colon X \times X \rightarrow \real$ heißen \\
    \begriff{(topologisch) äquivalent}, falls jede Teilmenge $Y \subset X$
    genau dann of"|fen bzgl. $d$ ist, wenn sie of"|fen bzgl. $e$ ist.
    Dies ist der Fall genau dann, wenn durch $d$ und $e$ derselbe
    Konvergenzbegriff definiert wird
    (was der Fall ist genau dann, wenn derselbe Stetigkeitsbegriff
    definiert wird).
\end{Def}

\begin{Bsp}
    Auf $\real^n$ sind die $p$-Metriken ($1 \le p \le \infty$) äquivalent.
    Auf $\C([0,1], \real)$ gilt dies nicht!
\end{Bsp}

\pagebreak

\section{%
    Topologische Räume%
}

\subsection{%
    Topologische Räume%
}

\begin{Def}{Topologie}
    Sei $X$ eine Menge.
    Ein System von Teilmengen $\T \subset P(X)$ heißt \begriff{Topologie},
    falls
    \begin{enumerate}[label=(O\arabic*)]
        \item
        $\emptyset, X \in \T$,

        \item
        für alle $U_1, \dotsc, U_n \in \T$, $n \in \natural$ gilt, dass
        $U_1 \cap \dotsb \cap U_n \in \T$, sowie

        \item
        für alle $U_i \in \T$, $i \in I$ gilt, dass
        $\bigcup_{i \in I} U_i \in \T$.
    \end{enumerate}
    Das Paar $(X, \T)$ heißt dann \begriff{topologischer Raum}.
    Die Elemente $U \in \T$ heißen \begriff{of"|fene Mengen},
    ihre Komplemente heißen \begriff{abgeschlossene Mengen}.
\end{Def}

\begin{Bsp}
    Jede Metrik $d$ eines metrischen Raums $(X, d)$ induziert eine Topologie \\
    $\T_d = \{U \subset X \;|\; U \text{ of"|fen bzgl. d}\}$.
    Für $X = \real^n$ bzw. $X \subset \real^n$ ist $(X, \T_d)$
    Topologie bzgl. der euklidischen Metrik bzw. bzgl. der eingeschränkten
    Metrik.
\end{Bsp}

\begin{Def}{metrisierbar}
    Eine Topologie $\T$ heißt \begriff{metrisierbar}, falls
    sie von einer Metrik induziert wird.
\end{Def}

\begin{Bsp}
    Die \begriff{diskrete Topologie} auf $X$ ist $\T = P(X)$ und wird
    von der diskreten Metrik induziert.
    Die \begriff{indiskrete Topologie} auf $X$ ist $\T = \{\emptyset, X\}$,
    sie ist nicht metrisierbar ($|X| \ge 2$).
\end{Bsp}

\linie

\begin{Def}{(of"|fene) Umgebung}
    Sei $a \in X$.
    Eine of"|fene Menge $O \in \T$ mit $a \in O$ heißt
    \begriff{of"|fene Umge\-bung} von $a$.
    $U \subset X$ heißt \begriff{Umgebung} von $a$, falls $U$ eine of"|fene
    Umgebung von $a$ enthält.
\end{Def}

\begin{Def}{Konvergenz}
    Eine Folge $\{x_n\}_{n \in \natural}$ in $X$ \begriff{konvergiert} gegen
    $a \in X$, falls jede Umgebung $U$ von $a$ fast alle Folgenglieder enthält
    (also $\exists_{m \in \natural} \forall_{n \ge m}\; x_n \in U$).
\end{Def}

\begin{Bsp}
    Für metrische Räume ist dies die übliche Konvergenz. \\
    In der diskreten Topologie konvergieren genau die
    fast-konstanten Folgen. \\
    In der indiskreten Topologie konvergiert jede Folge gegen jeden Punkt.
\end{Bsp}

\linie

\begin{Def}{Vergleich von Topologien}
    Seien $\T_1, \T_2 \subset P(X)$ Topologien.
    $\T_1$ heißt \begriff{feiner} bzw. \begriff{echt feiner} als
    $\T_2$, falls $\T_1 \supset \T_2$ bzw.
    $\T_1 \supsetneqq \T_2$.
    In diesem Fall heißt $\T_2$ heißt \begriff{gröber} bzw.
    \begriff{echt gröber}.
\end{Def}

\begin{Bsp}
    Auf $\C([0,1], \real)$ ist für $1 \le p < q \le \infty$ die Topologie der
    $q$-Norm echt feiner als die Topologie der $p$-Norm.
    Auf der Menge $\C_C(\real, \real)$ der kompakt getragenen Funktionen
    (d.\,h. Funktionen, bei denen der Abschluss der Nichtnullstellenmenge
    kompakt ist) sind die $p$-Norm und die $q$-Norm nicht vergleichbar.
    Auf $X = \{a, b\}$ gibt es die vier Topologien
    $\{\emptyset, X\}$, $\{\emptyset, \{a\}, X\}$, $\{\emptyset, \{b\}, X\}$
    und $\{\emptyset, \{a\}, \{b\}, X\}$, die rautenförmig angeordnet
    werden können.
\end{Bsp}

\subsection{%
    Beispiele%
}

Sei $(X, \le)$ eine geordnete Menge. \\
Die \begriff{Ordnungstopologie} auf $X$ ist $\T =
\{U \subset X \;|\; \forall_{x \in U} \exists_{a < x < b,\; a, b \in X}\;
\left]a,b\right[ \subset U\}$.

Sei $X$ eine Menge.
Die \begriff{koendliche Topologie} ist
$\T = \{U \subset X \;|\; U^C \text{ endlich}\} \cup
\{\emptyset\}$. \\
Hier konvergieren genau die Folgen, die jeden Wert $\not=$ GW nur endlich oft
annehmen.

Sei $X$ eine Menge.
Die \begriff{koabzählbare Topologie} ist
$\T = \{U \subset X \;|\; U^C \text{ abzählbar}\} \cup
\{\emptyset\}$. \\
Hier konvergieren genau die fast-konstanten Folgen.

Sei $S \subset \complex[X_1, \dotsc, X_n]$.
Die \begriff{Nullstellenmenge} von $S$ ist definiert durch \\
$V(S) := \{x \in \complex^n \;|\; \forall_{f \in S}\; f(x) = 0\}$.
Die $V(S)$ heißen \begriff{\name{Zariski}-abgeschlossen},
ihre Komplemente \begriff{\name{Zariski}-of"|fen}.
Die \begriff{\name{Zariski}-Topologie} ist
$\T = \{\complex^n \setminus V(S) \;|\;
S \subset \complex[X_1, \dotsc, X_n]\}$.

\subsection{%
    Funktionenräume%
}

\begin{Def}{punktweise Konvergenz}
    Auf $\real^\real$ definiert man die
    \begriff{Topologie der punktweisen Konvergenz} wie folgt:
    Eine Folge $f_n\colon \real \rightarrow \real$
    \begriff{konvergiert punktweise}
    gegen $f\colon \real \rightarrow \real$, falls $f_n(x) \to f(x)$ für jedes
    $x \in \real$, d.\,h.
    $\forall_{x \in \real} \forall_{\varepsilon > 0} \exists_{m \in \natural}
    \forall_{n \ge m}\; |f(x) - f_n(x)| < \varepsilon$. \\
    Für jede endliche Menge $J = \{x_1, \dotsc, x_n\} \subset \real$ und
    $\varepsilon > 0$ sei die $(J, \varepsilon)$-Umgebung von $f$
    $U(f, J, \varepsilon) := \{g\colon \real \rightarrow \real \;|\;
    \forall_{x \in J}\; |f(x) - g(x)| < \varepsilon\}$. \\
    Eine Menge $O \subset \real^\real$ heißt of"|fen, falls es für alle
    $f \in O$ ein $J = \{x_1, \dotsc, x_n\}$ und $\varepsilon > 0$ gibt, sodass
    $U(f, J, \varepsilon) \subset O$. \\
    Dies definiert eine Topologie.
    Eine Folge $f_n$ konvergiert gegen $f$ bzgl. dieser Topologie genau dann,
    wenn $f_n$ gegen $f$ punktweise konvergiert.
\end{Def}

\linie

\begin{Def}{gleichmäßige Konvergenz}
    Analog definiert man die
    \begriff{Topologie der gleichmäßigen Konver\-genz}:
    $f_n$ \begriff{konvergiert gleichmäßig} gegen $f$, falls
    $\forall_{\varepsilon > 0} \exists_{m \in \natural} \forall_{x \in \real}
    \forall_{n \ge m}\; |f(x) - f_n(x)| < \varepsilon$. \\
    Die $\varepsilon$-Umgebung von $f$ ist
    $U(f, \varepsilon) := \{g\colon \real \rightarrow \real \;|\;
    \forall_{x \in \real}\; |f(x) - g(x)| < \varepsilon\}$. \\
    Eine Menge $O \subset \real^\real$ heißt of"|fen, falls es für alle
    $f \in O$ ein $\varepsilon > 0$ gibt, sodass
    $U(f, \varepsilon) \subset O$. \\
    Dies definiert eine Topologie.
    Eine Folge $f_n$ konvergiert gegen $f$ bzgl. dieser Topologie genau dann,
    wenn $f_n$ gegen $f$ gleichmäßig konvergiert.
\end{Def}

\linie

Bei \textbf{Potenzreihen}, z.\,B. der Exponentialfunktion
$\exp\colon \real \rightarrow \real$,
$\exp(x) = \sum_{k=0}^\infty \frac{x^k}{k!}$, hat man manchmal das Problem, das
sie zwar punktweise konvergiert (d.\,h. die Folge der Partialsummen
konvergiert punktweise), jedoch nicht gleichmäßig.
Daher führt man einen weiteren Konvergenzbegriff ein, sozusagen ein
"`Kompromiss"' zwischen punktweiser und gleichmäßiger Konvergenz.

\begin{Def}{gleichmäßige Konvergenz auf jedem Kompaktum}
    $f_n$ \begriff{konvergiert gleichmäßig auf jedem Kompaktum} gegen $f$,
    falls $\forall_{K \subset \real \text{ kompakt}} \forall_{\varepsilon > 0}
    \exists_{m \in \natural} \forall_{x \in K} \forall_{n \ge m}\;
    |f(x) - f_n(x)| < \varepsilon$. \\
    Für $K \subset \real$ und $\varepsilon > 0$ definiert man
    $U(f, K, \varepsilon) := \{g\colon \real \rightarrow \real \;|\;
    \forall_{x \in K}\; |f(x) - g(x)| < \varepsilon\}$. \\
    Eine Menge $O \subset \real^\real$ heißt of"|fen, falls es für alle
    $f \in O$ eine kompakte Menge $K \subset \real$ und ein $\varepsilon > 0$
    gibt, sodass $U(f, K, \varepsilon) \subset O$. \\
    Dies definiert eine Topologie.
    Eine Folge $f_n$ konvergiert gegen $f$ bzgl. dieser Topologie genau dann,
    wenn $f_n$ gegen $f$ gleichmäßig auf jedem Kompaktum konvergiert.
\end{Def}

\linie

\begin{Bem}
    Die Topologie $\Tglm$ der gleichmäßigen Konvergenz ist echt feiner
    als die Topologie $\Tkpkt$ der gleichmäßigen Konvergenz auf jedem
    Kompaktum, welche echt feiner als die Topologie $\Tpw$ der
    punktweisen Konvergenz ist.
    $\Tglm$, $\Tkpkt$ sind metrisierbar, dagegen ist
    $\Tpw$ nicht metrisierbar (entsprechender Satz s.\,u.).
\end{Bem}

\pagebreak

\subsection{%
    Topologische Grundbegrif"|fe%
}

\begin{Def}{Menge der Umgebungen}
    Sei $(X, \T)$ ein topologischer Raum.
    $\U_x$ bezeichnet die Menge aller Umgebungen von $x$ in $(X, \T)$ und
    $\U_x^\circ \subset \U_x$ bezeichnet die Menge aller of"|fenen Umgebungen
    von $x$ in $(X, \T)$.
\end{Def}

\begin{Lemma}{Menge of"|fen $\Leftrightarrow$ Umgebung jedes ihrer Punkte}\\
    $U \subset X$ ist of"|fen genau dann, wenn $U$ Umgebung jedes ihrer
    Punkte ist.
\end{Lemma}

\begin{Def}{(of"|fene) Umgebung von Mengen}
    Sei $M \subset X$.
    Eine of"|fene Menge $O \subset X$ mit $M \subset O$ heißt
    \begriff{of"|fene Umgebung} von $M$.
    Eine Menge $U \subset X$, die eine of"|fene Umgebung von $M$ enthält,
    heißt \begriff{Umgebung} von $M$.
    $\U_M$ bezeichnet die Menge aller Umgebungen von $M$.
\end{Def}

\linie

\begin{Satz}{Umgebungsaxiome}
    Sei $(X, \T)$ ein topologischer Raum.
    Dann gilt (\begriff{Umgebungsaxiome}):
    \begin{enumerate}[label=(U\arabic*)]
        \item
        $X \in \U_x$, $\forall_{U \in \U_x}\; x \in U$

        \item
        $\forall_{U, V \in \U_x}\; U \cap V \in \U_x$

        \item
        $\forall_{U \in \U_x} \forall_{U \subset V \subset X}\; V \in \U_x$

        \item
        $\forall_{V \in \U_x} \exists_{U \in \U_x} \forall_{y \in U}\;
        V \in \U_y$
    \end{enumerate}
    Umgekehrt:
    Ist $\{\U_x \;|\; x \in X\}$ eine Familie von Mengensystemen,
    die (U1) bis (U4) erfüllt,
    dann existiert genau eine Topologie $\T$ auf $X$, für die
    $\U_x$ das Umgebungssystem für jedes $x \in X$ ist, nämlich
    $\T = \{O \subset X \;|\; \forall_{x \in O}\; O \in \U_x\}$.
\end{Satz}

\linie

\begin{Def}{topologische Grundbegrif"|fe}
    Bezüglich einer Teilmenge $M \subset X$ heißt $x \in X$ \\
    \begriff{innerer Punkt}, falls $M \in \U_x$, \qquad
    \begriff{äußerer Punkt}, falls $M^C \in \U_x$, \\
    \begriff{Randpunkt}, falls $\forall_{U \in \U_x}\; U \cap M \not= \emptyset,\;
    U \cap M^C \not= \emptyset$, \qquad
    \begriff{Berührpunkt}, falls
    $\forall_{U \in \U_x}\; U \cap M \not= \emptyset$, \\
    \begriff{Häufungspunkt}, falls $\forall_{U \in \U_x}\;
    (U \cap M) \setminus \{x\} \not= \emptyset$, \qquad
    \begriff{isolierter Punkt}, falls
    $\exists_{U \in \U_x}\; U \cap M = \{x\}$. \\
    Das \begriff{Innere} ist
    $\inneres{M} = \bigcup_{O \subset M,\; O \text{ of"|fen}} O$
    (die größte of"|fene Menge, die in $M$ enthalten ist,
    also die Menge aller inneren Punkte),
    der \begriff{Abschluss} ist
    $\abschluss{M} = \bigcap_{A \supset M,\; A \text{ abgeschlossen}} A$
    (die kleinste abgeschlossene Menge, in der $M$ enthalten ist,
    also die Menge aller Berührpunkte)
    und der \begriff{Rand} ist $\rand{M} = \abschluss{M} \setminus \inneres{M}$
    (also die Menge aller Randpunkte).
\end{Def}

\begin{Satz}{topologische Grundbegrif \!\!fe}
    $M \subset X$ ist of"|fen genau dann, wenn $M$ keinen ihrer Randpunkte
    enthält.
    $M$ ist abgeschlossen genau dann, wenn $M$ alle ihre Randpunkte enthält. \\
    Wenn $M$ of"|fen oder abgeschlossen ist, dann hat der Rand keine inneren
    Punkte. \\
    Es gilt $(\inneres{M})^C = \abschluss{M^C}$,
    $(\abschluss{M})^C = \inneres{(M^C)}$,
    $\inneres{M} \subset M$ und $\inneres{(\inneres{M})} = \inneres{M}$.
    Aus $M \subset N$ folgt $\inneres{M} \subset \inneres{N}$ und es gilt
    $\inneres{(M \cap N)} = \inneres{M} \cap \inneres{N}$.
    Für den Abschluss gilt $\abschluss{M} \supset M$ und
    $\abschluss{\abschluss{M}} = \abschluss{M}$.
    Aus $M \subset N$ folgt $\abschluss{M} \subset \abschluss{N}$ und es gilt
    $\abschluss{M \cup N} = \abschluss{M} \cup \abschluss{N}$.
\end{Satz}

\begin{Bsp}
    Der \begriff{Einheitsball}
    $\dball^n := \{x \in \real^n \;|\; \norm{x} \le 1\}$
    ist abgeschlossen im $\real^n$. \\
    Sein Inneres ist der \begriff{of"|fene Einheitsball}
    $\ball^n := \{x \in \real^n \;|\; \norm{x} < 1\}$, \\
    sein Rand (sowie der von $\ball^n$) ist die \begriff{Einheitssphäre}
    $\sphere^{n-1} := \{x \in \real^n \;|\; \norm{x} = 1\}$.
\end{Bsp}

\linie

\begin{Def}{dicht, diskret}
    $M \subset X$ heißt \begriff{dicht in $X$}, falls $\abschluss{M} = X$ ist.
    \\
    $A \subset X$ heißt \begriff{diskret}, falls jeder Punkt $a \in A$ isoliert
    ist.
\end{Def}

\begin{Bsp}
    $\rational \subset \real$ ist dicht (aber nicht diskret) und
    $\integer \subset \real$ ist diskret (aber nicht dicht). \\
    $M$ ist dicht in $X$ genau dann, wenn jeder Punkt $x \in X$ ein Berührpunkt
    von $M$ ist, also wenn jede nicht-leere of"|fene Menge mindestens
    einen Punkt von $M$ enthält. \\
    $X$ ist diskret genau dann, wenn keine Teilmenge $M \subset X$ Randpunkte
    besitzt.
\end{Bsp}

\pagebreak

\subsection{%
    Abzählbarkeitsaxiome%
}

\begin{Def}{Umgebungsbasis}
    Seien $(X, \T)$ ein topologischer Raum und $a \in X$. \\
    Ein System $\B_a \subset \U_a$ von Umgebungen heißt
    \begriff{Umgebungsbasis} von $a$, falls jede Umgebung von $a$ eine
    Umgebung aus $\B_a$ enthält.
\end{Def}

\begin{Bsp}
    Jede Umgebungsbasis $(U_i)_{i \in I}$ kann durch Übergang zu
    $(\inneres{U_i})_{i \in I}$ als of"|fen angenommen werden.
    In einem metrischen Raum $(X, d)$ bilden $B\!\left(a, \frac{1}{n}\right)$,
    $n \in \natural$ eine Umgebungsbasis von $a$.
\end{Bsp}

\begin{Lemma}{bei Konvergenz nur Umgebungsbasis betrachten reicht}
    Seien $(x_n)_{n \in \natural}$ eine Folge in $X$ und
    $(U_i)_{i \in I}$ eine Umgebungsbasis von $a$ in $X$.
    Dann gilt $x_n \to a$ genau dann, wenn jede Umgebung $U_i$, $i \in I$
    fast alle Folgenglieder $x_n$ enthält.
\end{Lemma}

\linie

\begin{Def}{erstes Abzählbarkeitsaxiom}
    Ein Punkt $a \in X$ \begriff{erlaubt eine abzählbare Umgebungsbasis}, falls
    es eine abzählbare Umgebungsbasis
    $\{U_n \;|\; n \in \natural\} \subset \U_a$ von $a$ gibt. \\
    Gilt dies für alle $a \in X$, so
    \begriff{erfüllt $(X, \T)$ das erste Abzählbarkeitsaxiom}.
\end{Def}

\begin{Bsp}
    Jeder metrisierbare topologische Raum $(X, \T)$ erfüllt das erste
    Abzählbarkeitsaxiom.
\end{Bsp}

\begin{Satz}{$\T_{pw}$ erfüllt nicht 1. Abzählbarkeitsaxiom}
    Die Topologie $\T_{pw}$ der punktweisen Konvergenz auf
    $\real^\real$ erfüllt nicht das erste Abzählbarkeitsaxiom und ist daher
    nicht metrisierbar.
\end{Satz}

\linie

\begin{Def}{Basis}
    Ein System $\B \subset \T$ heißt \begriff{Basis} der Topologie $\T$, falls
    sich jede of"|fene Menge $U \in \T$ als Vereinigung von Mengen aus $\B$
    darstellen lässt, d.\,h. $U = \bigcup_{i \in I} B_i$ mit $B_i \in \B$,
    $i \in I$.
\end{Def}

\begin{Satz}{Äquivalenz zur Basis}
    Für $\B \subset \T$ ist Folgendes äquivalent:
    \begin{enumerate}
        \item
        $\B$ ist Basis, d.\,h.
        $\forall_{U \in \T} \exists_{S \subset \B}\; U = \bigcup_{B \in S} B$.

        \item
        $\forall_{U \in \T}\; U = \bigcup_{B \in \B,\; B \subset U} B$.

        \item
        $\forall_{U \in \T} \forall_{x \in U}
        \exists_{B \in \B,\; B \subset U}\; x \in B$.
    \end{enumerate}
\end{Satz}

\linie

\begin{Def}{zweites Abzählbarkeitsaxiom} \\
    $(X, \T)$ \begriff{erfüllt das zweite Abzählbarkeitsaxiom}, wenn $\T$ eine
    abzählbare Basis erlaubt.
\end{Def}

\begin{Kor}
    Ist $\B \subset \T$ eine Basis, dann ist
    $\phi\colon \T \rightarrow P(\B)$,
    $U \mapsto \{B \in \B \;|\; B \subset U\}$
    injektiv, denn $U = \bigcup_{B \in \B,\; B \subset U} B$.
    Insbesondere ist $\card(\T) \le \card(P(\B))$. \\
    Erlaubt also $\T$ eine abzählbare Basis, so gilt
    $\card(\T) \le \card(P(\natural)) = \card(\real)$.
\end{Kor}

\begin{Def}{separabel}
    Ein Raum heißt \begriff{separabel}, falls er eine abzählbare dichte
    Teilmenge besitzt.
\end{Def}

\begin{Bsp}
    $\real = \abschluss{\rational}$,
    $\real^n = \abschluss{\rational^n}$.
\end{Bsp}

\begin{Satz}{Topologie mit 2. Abzählbarkeitsaxiom ist separabel}\\
    Erlaubt $\T$ eine abzählbare Basis, dann existiert eine
    abzählbare dichte Teilmenge $A \subset X$.
\end{Satz}

\begin{Satz}{separable metrische Räume erfüllen das 2. Abzählbarkeitsaxiom}\\
    Sei $(X, d)$ ein metrischer Raum.
    Ist $(X, \T_d)$ separabel, dann erlaubt $\T_d$ eine abzählbare Basis
    $\B = \{B\!\left(a, \frac{1}{k}\right) \;|\; a \in A,\; k \in \natural\}$
    (für $A \subset X$ abzählbar, $\abschluss{A} = X$).
\end{Satz}

\begin{Kor}
    Die euklidische Topologie $\T$ auf $\real^n$ erlaubt eine abzählbare Basis,
    z.\,B. \\
    $\B = \{B\!\left(a, \frac{1}{k}\right) \;|\; a \in \rational^n,\;
    k \in \natural\}$
    und es gilt $\card(\T) = \card(\real)$.
\end{Kor}

\linie
\pagebreak

\begin{Satz}{Zusammenhang zwischen den Abzählbarkeitsaxiomen}\\
    Das zweite Abzählbarkeitsaxiom impliziert das erste.
\end{Satz}

\begin{Bem}
    Das erste Abzählbarkeitsaxiom impliziert jedoch nicht das zweite. \\
    Ein Gegenbeispiel ist $\real$ mit der diskreten Topologie.
\end{Bem}

\begin{Lemma}{in Topologie mit 2. Abzählbarkeitsaxiom ist jede diskrete
              Teilmenge abzählbar}\\
    Ist $\B \subset \T$ eine Basis, dann gilt für jede diskrete Teilmenge
    $A \subset X$ die Bedingung $\card(A) \le \card(\B)$.
    Erlaubt also $\T$ eine abzählbare Basis, dann ist jede diskrete Teilmenge
    abzählbar.
\end{Lemma}

\begin{Satz}{Beispiel für metrisierbare Topologie,
             die nicht beide Abzählbarkeitsaxiome erfüllt}\\
    Sei $\C_b(\real) = \{f\colon \real \rightarrow \real \;|\;
    f \text{ stetig, beschränkt}\}$ mit der Supremumsnorm
    $\norm{f}_\infty = \sup_{x \in \real} |f(x)|$.
    Dann erfüllt $(\C_b(\real), \norm{-}_\infty)$ als metrischer Raum das
    erste Abzählbarkeitsaxiom, aber nicht das zweite.
\end{Satz}

\linie

\begin{Lemma}{Durchschnitt von Topologien}
    Sei $X$ eine Menge und $(\T_\lambda)_{\lambda \in \Lambda}$ eine Familie
    von Topologien auf $X$.
    Dann ist $\bigcap_{\lambda \in \Lambda} \T_\lambda$ ebenfalls eine
    Topologie auf $X$.
\end{Lemma}

\begin{Def}{erzeugte Topologie, Erzeugendensystem}
    Sei $X$ eine Menge und $\S \subset P(X)$ ein System von Teilmengen.
    Man definiert
    $\B := \{S_1 \cap \dotsb \cap S_n \;|\;
    n \in \natural_0,\; S_1, \dotsc, S_n \in \mathcal{S}\}$ und \\
    $\mathcal{T} := \{\bigcup_{i \in I} B_i \;|\; I \text{ Indexmenge},\;
    B_i \in \B\}$. \\
    Dann ist $\T$ die gröbste Topologie auf $X$, die $\S$ enthält,
    und $\B$ ist eine Basis von $\T$. \\
    $\T$ heißt die von $\S$ \begriff{erzeugte Topologie} und
    $\S$ heißt \begriff{Erzeugendensystem} oder \begriff{Subbasis} von $\T$.
\end{Def}

\begin{Bsp}
    Jede Basis von $\T$ ist ein Erzeugendensystem. \\
    $\mathcal{S} =
    \{\left]-\infty, a\right[, \left]a, +\infty\right[ \;|\; a \in \real\}$
    führt zu $\B = \{\left]a, b\right[ \;|\; a, b \in \real\}
    \cup \{\real\} \cup \mathcal{S}$
    und erzeugt die übliche Topologie $\T$ auf $\real$. \\
    Auf $\real^n$ betrachtet man die Halbräume
    $\{x \in \real^n \;|\; x_k > a,\; k = 1, \dotsc, n\}$ und \\
    $\{x \in \real^n \;|\; x_k < a,\; k = 1, \dotsc, n\}$.
    Dieses System $\E$ erzeugt die euklidische Topologie auf $\real^n$.
\end{Bsp}

\subsection{%
    Folgen und Konvergenz%
}

\begin{Def}{separiert/\name{hausdorff}sch}
    Ein topologischer Raum $(X, \T)$ heißt
    \begriff{separiert \\
    (\name{hausdorff}sch, \name{Hausdorff}-Raum)},
    falls es für alle Punkte $x, y \in X$, $x \not= y$
    disjunkte of"|fene Umgebungen von $x$ und $y$ gibt, d.\,h.
    $\forall_{x, y \in X,\; x \not= y} \exists_{U, V \in \T}\;
    x \in U,\; y \in V,\; U \cap V = \emptyset$.
\end{Def}

\begin{Bsp}
    Ist $(X, d)$ ein metrischer Raum, dann ist $(X, \T_d)$ hausdorffsch. \\
    Allerdings ist nicht jeder Hausdorff-Raum metrisierbar.
    Bspw. ist $\real^\real$ mit der Topologie der punktweisen Konvergenz
    hausdorffsch, aber nicht metrisierbar
    (erfüllt nicht das 1. Abzählbarkeitsaxiom).
\end{Bsp}

\linie

\begin{Satz}{Eindeutigkeit des Grenzwerts}
    Sei $(X, \T)$ ein topologischer Raum. \\
    Ist $X$ hausdorffsch,
    dann hat jede Folge in $X$ höchstens einen Grenzwert. \\
    Die Umkehrung gilt, wenn $X$ das 1. Abzählbarkeitsaxiom erfüllt.
\end{Satz}

\begin{Bem}
    Auf das 1. Abzählbarkeitsaxiom kann man hier nicht verzichten.
    Zum Beispiel konvergieren in $X$ mit der koabzählbaren Topologie genau die
    fast-konstanten Folgen.
    Hier gilt daher die Eindeutigkeit des Grenzwerts.
    Allerdings ist $X$ für $X$ überabzählbar nicht separiert.
\end{Bem}

\linie
\pagebreak

\begin{Satz}{Abschluss als folgenabgeschlossene Menge} \\
    Seien $(X, \T)$ ein topologischer Raum und $A \subset X$, $x \in X$. \\
    Wenn eine Folge $(a_n)_{n \in \natural}$, $a_n \in A$ gegen $x$
    konvergiert, dann ist $x \in \abschluss{A}$. \\
    Die Umkehrung gilt, wenn $x$ eine abzählbare Umgebungsbasis erlaubt.
\end{Satz}

\begin{Def}{folgenabgeschlossen}
    Sei $(X, \T)$ ein topologischer Raum.
    $A \subset X$ heißt \begriff{folgenabgeschlossen}, falls für alle Folgen
    $(a_n)_{n \in \natural}$ in $A$ mit $a_n \to x$, $x \in X$
    auch $x \in A$ gilt.
\end{Def}

\begin{Kor}
    Sei $(X, \T)$ ein topologischer Raum.
    Dann ist jede abgeschlossene Teilmenge $A \subset X$ folgenabgeschlossen.
    Die Umkehrung gilt, falls $X$ dem 1. Abzählbarkeitsaxiom genügt.
\end{Kor}

\begin{Bem}
    Auf das 1. Abzählbarkeitsaxiom kann man auch hier nicht verzichten.
    Bspw. ist in der koabzählbaren Topologie jede Menge
    folgen-abgeschlossen, aber i.\,A. nicht abgeschlossen.
\end{Bem}

\subsection{%
    Stetige Abbildungen%
}

\begin{Def}{Stetigkeit in einem Punkt}
    Seien $X$ und $Y$ topologische Räume. \\
    Eine Abbildung $f\colon X \rightarrow Y$ heißt
    \begriff{stetig in $a \in X$}, falls für jede Umgebung $V$ von $f(a)$ in
    $Y$ das Urbild $f^{-1}(V)$ eine
    Umgebung von $a$ in $X$ ist, d.\,h.
    $\forall_{V \in \U_{f(a)}}\; f^{-1}(V) \in \U_a$.
\end{Def}

\begin{Bem}
    Es reicht, statt $\U_{f(a)}$ eine Umgebungsbasis von $f(a)$ zu betrachten.
\end{Bem}

\begin{Bsp}
    $f\colon \rational \rightarrow \rational$, $f(x) = 0$ für $x^2 > 2$,
    $f(x) = 1$ für $x^2 \le 2$, ist stetig in jedem Punkt.
    $g\colon \real \rightarrow \real$, $g(x) = 0$ für $x^2 > 2$,
    $g(x) = 1$ für $x^2 \le 2$, ist nicht stetig in $\pm \sqrt{2}$.
\end{Bsp}

\begin{Satz}{Stetigkeit bei Komposition}
    Seien $X$, $Y$ und $Z$ topologische Räume. \\
    Ist $f\colon X \rightarrow Y$ stetig in $a$ und
    $g\colon Y \rightarrow Z$ stetig in $f(a)$, dann ist
    $g \circ f\colon X \rightarrow Z$ stetig in $a$.
\end{Satz}

\linie

\begin{Satz}{Äquivalenz für Stetigkeit} \\
    Seien $X$ und $Y$ topologische Räume.
    Für $f\colon X \rightarrow Y$ sind äquivalent:
    \begin{enumerate}
        \item
        $f$ ist stetig in jedem Punkt $a \in X$.

        \item
        Für alle $V \subset Y$ of"|fen ist $f^{-1}(V) \subset X$ of"|fen.

        \item
        Für alle $B \subset Y$ abgeschlossen ist $f^{-1}(B) \subset X$
        abgeschlossen.

        \item
        Für alle $A \subset X$ gilt
        $f(\abschluss{A}) \subset \abschluss{f(A)}$.

        \item
        Für alle $B \subset Y$ gilt
        $f^{-1}(\inneres{B}) \subset \inneres{(f^{-1}(B))}$.
    \end{enumerate}
\end{Satz}

\begin{Def}{Stetigkeit}
    Seien $(X, \T_X)$ und $(Y, \T_Y)$ topologische Räume. \\
    Eine Abbildung $f\colon X \rightarrow Y$ heißt \begriff{stetig}, falls
    $\forall_{V \in \T_Y}\; f^{-1}(V) \in \T_X$.
\end{Def}

\begin{Satz}{Stetigkeit bei Komposition}
    Seien $X$, $Y$ und $Z$ topologische Räume. \\
    Ist $f\colon X \rightarrow Y$ stetig und
    $g\colon Y \rightarrow Z$ stetig, dann ist
    $g \circ f\colon X \rightarrow Z$ stetig.
\end{Satz}

\linie
\pagebreak

\begin{Def}{Homöomorphismus}
    Seien $f\colon X \rightarrow Y$ und $g\colon Y \rightarrow X$ stetige
    Abbildungen. \\
    Gilt $g \circ f = \id_X$ und $f \circ g = \id_Y$, dann heißen $f$ und $g$
    \begriff{zueinander inverse Homöomorphismen}. \\
    $f$ heißt \begriff{Homöomorphismus} ($f\colon X \homoe Y$), falls es einen
    zu $f$ inversen Homöomorphismus gibt. \\
    $X$ und $Y$ heißen \begriff{homöomorph} ($X \cong Y$), falls
    es einen Homöomorphismus $f\colon X \homoe Y$ gibt.
\end{Def}

\begin{Bem}
    Homöomorphe topologische Räume besitzen die gleichen Eigenschaften.
    Man nennt diese Eigenschaften deshalb \begriff{topologisch invariant}.
    Dazu zählen z.\,B. Anzahl der Elemente in $X$ und $\T$,
    erstes/zweites Abzählbarkeitsaxiom, Separabilität und
    Hausdorff-Eigenschaft. \\
    Später werden Zusammenhang, Wegzusammenhang, Kompaktheit usw.
    dazu kommen. \\
    Da zwei homöomorphe Räume die gleichen topologischen Eigenschaften
    besitzen, sind zwei Räume, die sich in einer der Eigenschaften
    unterscheiden, nicht zueinander homöomorph.
\end{Bem}

\begin{Bsp}
    $[0,2] \cong [3,7]$, denn ein Homöomorphismus ist
    $f\colon [0,2] \rightarrow [3,7]$, $f(x) = 2x + 3$ mit inversem
    Homöomorphismus $g\colon [3,7] \rightarrow [0,2]$,
    $g(y) = \frac{y - 3}{2}$ \\
    (alternativ z.\,B. auch $f(x) = x^2 + 3$, $g(y) = \sqrt{y - 3}$).
    Es gilt jedoch $\left[0,1\right[ \not\cong [0,1]$ sowie
    $\left[0,1\right[ \not\cong \left]0,1\right[$
    (das kann später durch Kompaktheit und Zusammenhang gezeigt werden).
\end{Bsp}

\begin{Bem}
    Eine stetige Bijektion muss noch kein Homömorphismus sein
    (erst wenn die Umkehrung auch stetig ist).
    Beispielsweise ist $f\colon \left[0, 2\pi\right[ \rightarrow \sphere^1$,
    $f(t) = e^{it}$ eine stetige Bijektion, aber die Umkehrabbildung
    $g\colon \sphere^1 \rightarrow \left[0, 2\pi\right[$,
    $g(e^{it}) = t$ ist nicht stetig in $1 + 0\i \in \sphere^1$.
\end{Bem}

\linie

\begin{Def}{of"|fene/abgeschlossene Abbildung}
    $f\colon X \rightarrow Y$ heißt of"|fen bzw. abgeschlossen, falls das
    Bild jeder of"|fenen bzw. abgeschlossenen Menge wieder of"|fen
    bzw. abgeschlossen ist.
\end{Def}

\begin{Satz}{Kriterium für Homöomorphismen}
    Sei $f\colon X \rightarrow Y$ eine stetige Bijektion. \\
    Dann sind äquivalent:
    \begin{enumerate}
        \item
        $f$ ist ein Homöomorphismus (d.\,h. $f^{-1}$ ist stetig).

        \item
        $f$ ist of"|fen.

        \item
        $f$ ist abgeschlossen.
    \end{enumerate}
\end{Satz}

\pagebreak

\subsection{%
    Filter%
}

\emph{Motivation}:
Sei $\{x_n\}_{n \in \natural}$ eine Folge in $X$.
Dann ist $E_m := \{x_n \;|\; n \ge m\}$ das $m$-te Endstück
und $\E := \{E_m \;|\; m \in \natural\}$ System aller Endstücke.
$\E$ erfüllt (FB1), (FB2) von unten.
Es gilt: $x_n \to a \;\Leftrightarrow\;
\forall_{U \in \U_a} \exists_{m \in \natural}\; E_m \subset U$.
Der von $\E$ erzeugte Filter erfüllt (F1), (F2), (F3) von unten.

\begin{Def}{Filterbasis}
    Sei $X$ ein topologischer Raum.
    $\E \subset P(X)$ heißt \begriff{Filterbasis}, falls Folgendes gilt:
    \begin{enumerate}[label=(FB\arabic*)]
        \item
        $\E \not= \emptyset$, $\emptyset \notin \E$

        \item
        $\forall_{U, V \in \E} \exists_{W \in \E}\; W \subset U \cap V$
    \end{enumerate}
\end{Def}

\begin{Def}{Filter}
    $\F \subset P(X)$ heißt \begriff{Filter} auf $X$, falls Folgendes gilt:
    \begin{enumerate}[label=(F\arabic*)]
        \item
        $X \in \F$, $\emptyset \notin \F$

        \item
        $\forall_{U, V \in \F}\; U \cap V \in \F$

        \item
        $\forall_{U \in \F,\; U \subset V \subset X}\; V \in \F$
    \end{enumerate}
    Jede Filterbasis $\E$ \begriff{erzeugt einen Filter}
    $\F = \aufspann{\E}_X := \{F \subset X \;|\;
    \exists_{E \in \E}\; E \subset F\}$.
\end{Def}

\linie

\begin{Bsp}
    Für $A \subset X$ mit $A \not= \emptyset$ ist $\{A\}$ eine Filterbasis,
    der erzeugte Filter ist der \begriff{Hauptfilter}
    $\aufspann{A}_X := \{F \subset X \;|\; A \subset F\}$. \\
    Für $a \in X$ (d.\,h. $A = \{a\}$) ist dies entsprechend
    $\aufspann{a}_X := \{F \subset X \;|\; a \in F\}$. \\
    Aus $A \subset B$ folgt stets $\aufspann{A}_X \supset \aufspann{B}_X$. \\
    Ein Filter $\F$ hat ein kleinstes Element $A$ genau dann, wenn
    $A = \bigcap_{F \in \F} F$ in $\F$ ist. \\
    In diesem Fall ist $\F = \aufspann{A}_X$ ein Hauptfilter.
\end{Bsp}

\begin{Bsp}
    Sei $X$ unendlich.
    Dann ist $\F = \{F \subset X \;|\; F^C \text{ endlich}\}$ der
    \begriff{koendliche Filter} auf $X$.
    Es gilt $\bigcap_{F \in \F} F = \emptyset \notin \F$.
\end{Bsp}

\begin{Bsp}
    Das System $\U_a$ der Umgebungen von $a \in X$ in einem topologischen Raum
    $(X, \T)$ ist ein Filter, der \begriff{Umgebungsfilter} von $a$.
    Jede Umgebungsbasis $\B = \{U_i \;|\; i \in I\} \subset \U_a$ ist
    eine Filterbasis und der erzeugte Filter ist gerade $\U_a$.
\end{Bsp}

\begin{Bsp}
    Jede Folge $\{x_n\}_{n \in \natural}$ in $X$ definiert eine Filterbasis
    $\E$ und einen Filter $\F$ wie oben. \\
    Es gilt $x_n \to a$ genau dann, wenn $\F \supset \U_a$.
\end{Bsp}

\begin{Def}{Filter-Konvergenz}
    Ein Filter $\F$ \begriff{konvergiert} gegen $a \in X$ ($\F \to a$), falls
    $\F \supset \U_a$ ist.
\end{Def}

\linie

\emph{Exkurs: Filter und Ideale} \\
Sei $(R, +, \cdot)$ ein kommutativer Ring mit Eins
(z.\,B. $(\integer, +, \cdot)$). \\
Für $m \in R$ sei
$I := \{rm \;|\; r \in R\}$. Diese Menge erfüllt folgende
Eigenschaften: \\
(I1) $0 \in I$, \quad
(I2) $\forall_{u, v \in I}\; u + v \in I$, \quad
(I3) $\forall_{u \in I,\; a \in R}\; au \in I$. \\
Ein \begriff{Ideal} in $R$ ist eine Teilmenge $I \subset R$,
die (I1), (I2), (I3) erfüllt.
Beispiele sind $\{0\}$ (Nullideal) und $\aufspann{m} := \{rm \;|\; r \in R\}$,
wobei $\aufspann{1} = R$.
In der Tat gilt $I = R$ genau dann, wenn $1 \in I$ ist. \\
Ein \begriff{echtes Ideal} in $R$ ist ein Ideal $I \subsetneqq R$.
Für solche gilt (I1') $0 \in I$, $1 \notin I$, (I2), (I3) wie oben.

Statt $(R, +, \cdot)$ mit $0$ und $1$ betrachte nun $(P(X), \cap, \cup)$ mit
$X$ und $\emptyset$. \\
Ein echtes Ideal in $(P(X), \cap, \cup)$ ist nichts anderes als ein
Filter auf $X$!

\linie
\pagebreak

\begin{Bsp}
    Seien $\F_1, \F_2$ Filter auf $X$.
    Dann ist $\F_1 \cap \F_2$ ein Filter auf $X$.
    $\F_1 \cup \F_2$ ist i.\,A. kein Filter, z.\,B. für $a, b \in X$ mit
    $a \not= b$ ist $\aufspann{a} \cup \aufspann{b}$ kein Filter, da
    $\{a\} \cap \{b\} = \emptyset$. \\
    Genauer:
    Es gibt keinen Filter, der sowohl $\aufspann{a}$ als auch $\aufspann{b}$
    enthält.
\end{Bsp}

\begin{Def}{fremd, verträglich}
    Zwei Filter $\F_1, \F_2$ auf $X$ heißen \begriff{fremd},
    falls es $U_1 \in \F_1$ und $U_2 \in \F_2$ gibt mit
    $U_1 \cap U_2 = \emptyset$.
    Andernfalls heißen sie \begriff{verträglich}, d.\,h. falls
    $\forall_{U_1 \in \F_1,\; U_2 \in \F_2}\; U_1 \cap U_2 \not= \emptyset$.
\end{Def}

\begin{Lemma}{Filter sind verträglich $\Leftrightarrow$
              es gibt einen Filter, der beide enthält}
    Es existiert ein Filter auf $X$, der $\F_1$ und $\F_2$ enthält, genau dann,
    wenn $\F_1$ und $\F_2$ verträglich sind.
    In diesem Fall ist $\aufspann{\F_1, \F_2} :=
    \{U_1 \cap U_2 \;|\; U_1 \in \F_1,\; U_2 \in \F_2\}$
    der gröbste Filter, der $\F_1$ und $\F_2$ enthält.
\end{Lemma}

\linie

\begin{Satz}{$X$ hausdorffsch $\;\Leftrightarrow\;$ Filter-GW eindeutig}
    Sei $X$ ein topologischer Raum.
    Dann ist $X$ hausdorffsch genau dann, wenn kein Filter auf $X$ gegen
    zwei verschiedene Punkte konvergiert.
\end{Satz}

\begin{Satz}{Abschluss und Filter-GW}
    Seien $X$ ein topologischer Raum, $A \subset X$ und $x \in X$. \\
    Dann ist $x \in \abschluss{A}$ genau dann, wenn es einen Filter $\E$ auf
    $A$ gibt, dessen zugehöriger Filter $\F = \aufspann{\E}_X$ auf $X$
    gegen $x$ konvergiert.
\end{Satz}

\linie

\begin{Bem}
    Ist $\F$ ein Filter auf $X$ und $f\colon X \rightarrow Y$ eine Abbildung,
    dann ist \\
    $\E = \{f(U) \;|\; U \in \F\}$ ein Filter auf $f(X)$.
    Für $f$ surjektiv ist dies ein Filter auf $Y$. \\
    Andernfalls ist $\E$ nur eine Filterbasis auf $Y$.
    Um im allgemeinen Fall ($f(X) \subsetneqq Y$) auch von einem Filter auf $Y$
    sprechen zu können,
    geht man nun zum erzeugten Filter $\aufspann{E}_Y$ über.
\end{Bem}

\begin{Def}{Bildfilter}
    Seien $\F$ ein Filter auf $X$ und $f\colon X \rightarrow Y$ eine
    Abbildung.
    Dann ist \\
    $f(\F) := \{V \subset Y \;|\;
    \exists_{U \in \F}\; f(U) \subset V\} =
    \aufspann{\{f(U) \;|\; U \in \F\}}_Y$
    der \begriff{Bildfilter} von $\F$ unter $f$.
\end{Def}

\begin{Satz}{Äquivalenz für Stetigkeit}\\
    Seien $X, Y$ topologische Räume und $f\colon X \rightarrow Y$ eine
    Abbildung.
    Dann sind äquivalent:
    \begin{enumerate}
        \item
        $f$ ist stetig in $a \in X$.

        \item
        Für jeden Filter $\F$ mit $\F \to a$ gilt $f(\F) \to f(a)$.

        \item
        $f(\U_a) \to f(a)$.
    \end{enumerate}
\end{Satz}

\linie

\begin{Def}{Ultrafilter}
    Ein Filter $\F$ auf $X$ heißt \begriff{Ultrafilter} oder
    \begriff{maximaler Filter}, \\
    falls für alle Filter $\F'$ auf $X$ mit $\F' \supset \F$ gilt,
    dass $\F' = \F$.
\end{Def}

\begin{Bsp}
    Für $a \in X$ ist $\aufspann{a}_X$ ein Ultrafilter.
\end{Bsp}

\begin{Satz}{Kriterium für Ultrafilter}
    Ein Filter $\F$ auf $X$ ist Ultrafilter genau dann, wenn für alle
    $A \subset X$ entweder $A \in \F$ oder $A^C \in \F$ ist.
\end{Satz}

\begin{Satz}{jeder Filter ist in einem Ultrafilter enthalten}
    Jeder Filter $\F$ auf $X$ ist in einem Ultrafilter enthalten,
    wenn man das Lemma von Zorn voraussetzt.
\end{Satz}

\pagebreak

\section{%
    Konstruktion topologischer Räume%
}

\subsection{%
    Teilräume%
}

\begin{Def}{Teilraumtopologie}
    Seien $(X, \T)$ ein topologischer Raum und $A \subset X$ eine Teilmenge. \\
    Dann ist $\T_A = \{A \cap U \;|\; U \in \T\}$ eine Topologie auf $A$,
    die \begriff{Teilraumtopologie}. \\
    Der topologische Raum $(A, \T_A)$ heißt \begriff{Teilraum} von $(X, \T)$.
\end{Def}

\begin{Bsp}
    Die Teilraumtopologie von $\real$ in $\complex$ ist die übliche Topologie
    auf $\real$.
\end{Bsp}

\begin{Bem}
    Sei $(X, d)$ ein metrischer Raum mit der induzierten Topologie $\T$. \\
    Auf $A \subset X$ induziert die Teilraummetrik $d_A$ die Teilraumtopologie
    $\T_A$.
\end{Bem}

\begin{Satz}{Charakterisierung der Teilraumtopologie}
    \begin{enumerate}
        \item
        $\T_A$ ist die gröbste Topologie auf $A$, für die die Inklusion
        $\iota\colon A \rightarrow X$ stetig ist.

        \item
        Für jeden Raum $Y$ ist $f\colon Y \rightarrow A$ stetig bzgl. $\T_A$
        genau dann, wenn $g := \iota \circ f\colon Y \rightarrow X$ stetig ist.

        \item
        Für jeden Raum $Y$ ist $\phi\colon \C(Y, A) \rightarrow \C(Y, X)$,
        $f \mapsto \iota \circ f$ eine Bijektion auf die Teilmenge der
        stetigen Abbildungen $g\colon Y \rightarrow X$, $g(Y) \subset A$.
    \end{enumerate}
    Die Teilraumtopologie $\T_A$ auf $A$ wird durch jede dieser Eigenschaften
    charakterisiert, d.\,h. $\T_A$ ist die einzige Topologie mit diesen
    Eigenschaften.
\end{Satz}

\begin{Bem}
    Ist $A \subset X$ of"|fen und $U \subset A$,
    dann gilt $U \in \T_A \;\Leftrightarrow\; U \in \T$.
    Ist $A \subset X$ abgeschlossen und $M \subset A$,
    dann ist $M$ abgeschlossen in $A$ genau dann, wenn $M$ abgeschlossen
    in $X$ ist. \\
    Jeder Teilraum eines Teilraums $(Y, \T_Y)$ von $(X, \T_X)$ mit
    $Y \subset X$ ist auch ein Teilraum von $(X, \T_X)$.
    Wenn $(X, \T)$ hausdorffsch ist/dem ersten/zweiten Abzählbarkeitsaxiom
    genügt, dann auch jeder Teilraum $(A, \T_A)$.
\end{Bem}

\linie

\begin{Def}{Einbettungstopologie}
    Seien $(X, \T_X)$ ein topologischer Raum, $A$ eine Menge und \\
    $f\colon A \rightarrow X$ eine injektive Abbildung.
    Die \begriff{Einbettungstopologie} $\T_f := \{f^{-1}(U) \;|\; U \in \T\}$
    auf $A$ ist die gröbste Topologie auf $A$, für die $f$ stetig ist. \\
    Ist $A$ mit $\T_f$ ausgestattet, dann heißt die stetige Abbildung
    $f\colon (A, \T_f) \rightarrow (X, \T_X)$ \begriff{Einbettung}. \\
    Die Teilraumtopologie auf $A \subset X$ ist die Einbettungstopologie
    bzgl. der Inklusion $\iota\colon A \rightarrow X$.
\end{Def}

\begin{Bem}
    Eine injektive Abbildung $f\colon A \rightarrow X$ zwischen zwei
    topologischen Räumen $(X, \T_X)$ und $(A, \T_A)$
    ist genau dann eine Einbettung, falls
    $V \subset A$ of"|fen in $A$ ist genau dann, wenn es eine in $X$ of"|fene
    Menge $U \subset X$ gibt mit $f^{-1}(U) = V$. \\
    Eine injektive Abbildung $f\colon X \rightarrow Y$ zwischen topologischen
    Räumen ist eine Einbettung genau dann, wenn
    $f\colon X \rightarrow f(X)$ ein Homöomorphismus ist.
\end{Bem}

\begin{Bsp}
    Für $m < n$ ist $f\colon \real^m \rightarrow \real^n$,
    $f(x_1, \dotsc, x_m) = (x_1, \dotsc, x_m, 0, \dotsc, 0)$
    Einbettung.
\end{Bsp}

\begin{Satz}{Kriterium für Einbettungen}
    Sei $f\colon X \rightarrow Y$ stetig und injektiv. \\
    Ist $f$ of"|fen oder abgeschlossen, so ist $f$ eine Einbettung.
\end{Satz}

\pagebreak

\subsection{%
    Quotientenräume%
}

\begin{Def}{Äquivalenzrelation}
    Sei $X$ eine Menge. \\
    Eine Relation $R \subset X \times X$ heißt \begriff{Äquivalenzrelation},
    falls Folgendes gilt:
    \begin{itemize}
        \item
        \begriff{Reflexivität}:
        $\forall_{x \in X}\; xRx$

        \item
        \begriff{Symmetrie}:
        $\forall_{x, y \in X}\; (xRy \;\Leftrightarrow\; yRx)$

        \item
        \begriff{Transitivität}:
        $\forall_{x, y, z \in X}\; (xRy \land yRz \;\Rightarrow\; xRz)$
    \end{itemize}
    Die \begriff{Äquivalenzklasse} von $x \in X$ ist
    $\cl_R(x) := \{x' \in X \;|\; x'Rx\}$.
    Die \begriff{Quotientenmenge} von $X$ bzgl. $R$ ist
    $X/R := \{\cl_R(x) \;|\; x \in X\}$.
    Die \begriff{Quotientenabbildung} ist
    $q\colon X \rightarrow X/R$, $x \mapsto \cl_R(x)$.
    Die Menge der Äquivalenzklassen bildet eine Partition von $X$.
\end{Def}

\begin{Bem}
    Jede Abbildung $f\colon X \rightarrow Y$ induziert eine
    Äquivalenzrelation \\
    $R_f := \{(x, x') \in X \times X \;|\; f(x) = f(x')\}$. \\
    Damit erhält man eine injektive Abbildung
    $\overline{f}\colon X/R_f \rightarrow Y$, $\cl(x) \mapsto f(x)$.
\end{Bem}

\begin{Lemma}{Faktorisierung von Abbildungen mithilfe von
              Äquivalenzrelationen}\\
    Sei $R \subset X \times X$ eine Äquivalenzrelation auf $X$ und
    $f\colon X \rightarrow Y$ eine Abbildung.
    Dann gibt es ein $\overline{f}\colon X/R \rightarrow Y$ mit
    $f = \overline{f} \circ q$ genau dann, wenn $R \subset R_f$ ist.
    In diesem Fall gilt $\overline{f}(\cl_R(x)) = f(x)$, insbesondere ist
    $\overline{f}$ eindeutig.
    $\overline{f}$ ist injektiv genau dann, wenn $R = R_f$ ist. \\
    Es gilt $\overline{f}(X/R) = f(X)$, insbesondere ist
    $\overline{f}$ surjektiv genau dann, wenn $f$ surjektiv ist.
\end{Lemma}

\begin{Satz}{Faktorisierung von Abbildungen}
    In der Kategorie der Mengen faktorisiert jede Abbildung
    $f\colon X \rightarrow Y$ zu
    $f = \iota \circ \overline{f} \circ q$ mit
    $q\colon X \rightarrow X/R_f$ surjektiv,
    $\overline{f}\colon X/R_f \rightarrow f(X)$ bijektiv und
    $\iota\colon f(X) \rightarrow Y$ injektiv:
    \displaymathother
    \begin{align*}
        \begin{xy}
            \xymatrix{
                %& B \ar[rd]^g & \ar @{=>} @/^ 2mm/ [r]^G & &
                %G(B) \ar[ld]_{G(g)} \\
                %A \ar[ru]^f \ar[rr]_h & & C &
                %G(A) & & G(C) \ar[lu]_{G(g)} \ar[ll]^{G(h)}
                X \ar[r]^f \ar @{>>} [d]_q &
                Y \\
                X/R_f \ar[r]^\sim_{\overline{f}} &
                f(X) \ar @{^{(}->} [u]_\iota
            }
        \end{xy}
    \end{align*}
    \displaymathnormal
\end{Satz}

\linie

\begin{Def}{Quotiententopologie}
    Sei $(X, \T)$ ein topologischer Raum, $R \subset X \times X$
    eine Äquivalenzrelation sowie $q\colon X \rightarrow Q := X/R$ die
    Quotientenabbildung. \\
    Auf $Q$ definiert man die \begriff{Quotiententopologie}
    $\T_q := \{U \subset Q \;|\; q^{-1}(U) \in \T\}$.
\end{Def}

\begin{Satz}{Charakterisierung der Quotiententopologie}
    \begin{enumerate}
        \item
        $\T_q$ ist die feinste Topologie auf $Q$,
        für die die Quotientenabbildung $q\colon X \rightarrow Q$ stetig ist.

        \item
        Für jeden Raum $Y$ ist $f\colon Q \rightarrow Y$ stetig bzgl. $\T_q$
        genau dann, wenn $g := f \circ q\colon X \rightarrow Y$ stetig ist.

        \item
        Für jeden Raum $Y$ ist $\phi\colon \C(Q, Y) \rightarrow \C(X, Y)$,
        $f \mapsto f \circ q$ eine Bijektion auf die Teilmenge der
        stetigen Abbildungen $g\colon X \rightarrow Y$ mit $R \subset R_g$.
    \end{enumerate}
    Die Quotiententopologie $\T_q$ auf $Q$ wird durch jede dieser Eigenschaften
    charakterisiert, d.\,h. $\T_q$ ist die einzige Topologie mit diesen
    Eigenschaften.
\end{Satz}

\linie
\pagebreak

\begin{Def}{Identifizierungstopologie}
    Seien $(X, \T_X)$ ein topologischer Raum, $Y$ eine Menge und \\
    $f\colon X \rightarrow Y$ surjektiv.
    Die \begriff{Identifizierungstopologie}
    $\T_f := \{U \subset Y \;|\; f^{-1}(U) \in \T_X\}$
    ist die feinste Topologie auf $Y$, sodass $f$ stetig ist.
    Ist $Y$ mit dieser Topologie ausgestattet, dann nennt man
    $f\colon (X, \T_X) \rightarrow (Y, \T_f)$ eine \begriff{Identifizierung} oder
    \begriff{identifzierend}. \\
    Die Quotiententopologie auf $X/R$ ist die Identifizierungstopologie
    bzgl. der Quotientenabbildung $q\colon X \rightarrow X/R$.
\end{Def}

\begin{Bem}
    Eine surjektive Abbildung $f\colon X \rightarrow Q$ zwischen zwei
    topologischen Räumen ist genau dann identifizierend, falls
    $V \subset Q$ of"|fen in $Q$ ist genau dann, wenn $f^{-1}(V)$ of"|fen
    in $X$ ist.
\end{Bem}

\begin{Bsp}
    Für $n > m$ ist $f\colon \real^n \rightarrow \real^m$,
    $f(x_1, \dotsc, x_m, \dotsc, x_n) = (x_1, \dotsc, x_m)$ identifizierend.
\end{Bsp}

\linie

\begin{Satz}{kanonische Faktorisierung}
    Jede stetige Abbildung $f\colon X \rightarrow Y$ faktorisiert zu
    $f = \iota \circ \overline{f} \circ q$ mit $q\colon X \rightarrow X/R_f$
    stetig, surjektiv, $\overline{f}\colon X/R_f \rightarrow f(X)$
    stetig, bijektiv und $\iota\colon f(X) \rightarrow Y$ stetig, injektiv.
    Dabei ist $\overline{f}$ ein Homöomorphismus genau dann, wenn
    $f$ identifizierend ist.
\end{Satz}

\begin{Satz}{Kriterium für Identifizierungen}
    Sei $f\colon X \rightarrow Y$ stetig und surjektiv. \\
    Ist $f$ of"|fen oder abgeschlossen, so ist $f$ identifizierend.
\end{Satz}

\begin{Bem}
    Dieses Kriterium ist hinreichend, aber nicht notwendig:
    $f\colon \real^2 \rightarrow \real$, $f(x, y) = x$ ist identifizierend,
    of"|fen, aber nicht abgeschlossen,
    z.\,B. ist die Hyperbel \\
    $A = \{(x, y) \in \real^2 \;|\; xy = 1\}$ abgeschlossen in $\real^2$, aber
    $f(A) = \real \setminus \{0\}$ ist nicht abgeschlossen in $\real$.
    $f\colon [0, 1] \rightarrow \sphere^1$, $f(x) = e^{2\pi \i x}$ ist
    identifizierend, abgeschlossen, aber nicht of"|fen,
    z.\,B. ist $\left]\frac{1}{2}, 1\right]$ of"|fen in $[0,1]$, aber
    $f\left(\left]\frac{1}{2}, 1\right]\right)$ ist nicht of"|fen in
    $\sphere^1$.
    Kombiniert man diese beiden Beispiele, so erhält man eine Abbildung
    $f\colon [0, 1] \times \real \rightarrow \sphere^1$ mit
    $f(x, y) = e^{2\pi \i x}$, die identifizierend,
    aber weder of"|fen noch abgeschlossen ist.
\end{Bem}

\subsection{%
    Erste Beispiele%
}

\begin{Def}{Zusammenschlagen eines Teilraums}
    Sei $(X, \T)$ ein topologischer Raum und $A \subset X$.
    Man definiert eine Äquivalenzrelation auf $X$ mit
    $x \underset{A}{\equiv} y \;\Leftrightarrow\;
    (x = y) \lor (x, y \in A)$. \\
    Die Äquivalenzklassen sind $A$ und $\{x\}$ für alle
    $x \in X \setminus A$. \\
    Der entsprechende Quotientenraum wird als
    $X/\!/A := X/\!\underset{A}{\equiv}$ bezeichnet.
\end{Def}

\begin{Bsp}
    $[0, 1]/\!/\{0,1\} \cong \sphere^1$, ebenso
    $\dball^n/\!/\sphere^{n-1} \cong \sphere^n$.
\end{Bsp}

\begin{Bsp}
    Auf dem komplexen Einheitskreis
    $\sphere^1 = \{z \in \complex \;|\; |z| = 1\}$ definiert man die
    komplexen Einheitswurzeln $W_n = \{z \in \complex \;|\; z^n = 1\} =
    \{e^{2 \pi \i k/n} \;|\; k = 0, \dotsc, n - 1\}$.
    Dann entspricht $\sphere^1 /\!/ W_n$ einem Bouquet von $n$ Kreislinien.
    Es gilt $\sphere^1 /\!/ W_n \cong
    \{z \in \complex \;|\; |z| = |z^n - 1|\}$. \\
    Entsprechend entsteht $\real /\!/ \integer$ durch Verheften aller
    ganzzahligen Punkte in $\real$
    (unendliches Bouquet).
\end{Bsp}

\pagebreak

\subsection{%
    Summen topologischer Räume%
}

\begin{Def}{finale Abbildungsfamilien}
    Seien $X$ eine Menge, $X_i$ Mengen für $i \in I$ ($I$ Indexmenge)
    und $F = (f_i\colon X_i \rightarrow X)_{i \in I}$ eine Familie von
    Abbildungen. \\
    $F$ heißt \begriff{finale Familie}, falls
    $X = \bigcup_{i \in I} f_i(X_i)$. \\
    Zu $g\colon X \rightarrow Y$ ($Y$ Menge) definiert man
    $g_i := g \circ f_i\colon X_i \rightarrow Y$. \\
    Die Familie $G = (g_i)_{i \in I}$ ist mit $F = (f_i)_{i \in I}$
    \begriff{kompatibel}, d.\,h. aus $f_i(x_i) = f_j(x_j)$ folgt immer \\
    $g_i(x_i) = g_j(x_j)$ für alle $i, j \in I$, $x_i \in X_i$
    und $x_j \in X_j$. \\
    $\prod_{i \in I}^{(F)} \Abb(X_i, Y) \subset \prod_{i \in I} \Abb(X_i, Y)$
    sei die Teilmenge der mit $F$ kompatiblen Familien.
\end{Def}

\begin{Satz}{Bijektion}
    Sei $F = (f_i\colon X_i \rightarrow X)_{i \in I}$ eine finale Familie. \\
    Dann ist
    $\phi\colon \Abb(X, Y) \rightarrow \prod_{i \in I}^{(F)} \Abb(X_i, Y)$,
    $g \mapsto (g \circ f_i)_{i \in I}$ eine Bijektion.
\end{Satz}

\linie

\begin{Def}{finale Topologie}
    Seien $(X_i, \T_i)$ topologische Räume, $X$ eine Menge und
    $F = (f_i)_{i \in I}$ eine finale Familie mit
    $f_i\colon X_i \rightarrow X$ für $i \in I$. \\
    Die von $F$ induzierte \begriff{Finaltopologie} $\T_F$ auf $X$ ist
    $\T_F := \{U \subset X \;|\; \forall_{i \in I}\; f_i^{-1}(U) \in \T_i\}$.
\end{Def}

\begin{Bsp}
    Für $f\colon Y \rightarrow X$ surjektiv und $F = (f)$
    ist die Finaltopologie auf $X$
    genau die Identifizierungstopologie bzgl. $f$.
\end{Bsp}

\begin{Satz}{Eigenschaften der Finaltopologie}
    \begin{enumerate}
        \item
        $\T_F$ ist die feinste Topologie auf $X$, sodass alle
        $f_i\colon X_i \rightarrow X$ stetig sind.

        \item
        Für jeden Raum $Y$ ist $g\colon X \rightarrow Y$ stetig (bzgl. $\T_F$)
        genau dann, wenn für alle $i \in I$ die Komposition
        $g_i = g \circ f_i$ stetig ist.

        \item
        $\phi\colon \C(X, Y) \rightarrow
        \prod_{i \in I}^{(F)} \C(X_i, Y)$, $g \mapsto (g \circ f_i)_{i \in I}$
        ist eine Bijektion.
    \end{enumerate}
\end{Satz}

\linie

\begin{Def}{disjunkte Vereinigung zweier Mengen}
    Seien $X$ und $Y$ Mengen mit $X \cap Y = \emptyset$. \\
    Für $X \cap Y \not= \emptyset$ geht man zu $X' := X \times \{0\}$,
    $Y' := Y \times \{1\}$ über, damit $X' \cap Y' = \emptyset$. \\
    Die \begriff{disjunkte Vereinigung} bezeichnet man mit
    $X \sqcup Y := X \;\dot{\cup}\; Y$. \\
    Außerdem definiert man die \begriff{Inklusionen}
    $i\colon X \rightarrow X \sqcup Y$, $i(x) = x$
    sowie $j\colon Y \rightarrow X \sqcup Y$, $j(x) = x$.
\end{Def}

\begin{Satz}{$(i, j)$ als finale Familie}
    $(i, j)$ ist eine finale Familie, und zwar die freie: \\
    $\phi\colon \Abb(X \sqcup Y, Z) \rightarrow
    \Abb(X, Z) \times \Abb(Y, Z)$,
    $f \mapsto (f_1 := f \circ i, f_2 := f \circ j) = (f|_X, f|_Y)$
    ist eine Bijektion für jede Menge $Z$.
\end{Satz}

\begin{Bem}
    Man schreibt manchmal $Y^X := \Abb(X, Y)$ und $X + Y := X \sqcup Y$. \\
    Damit ist nämlich $Z^{(X + Y)} \cong Z^X \times Z^Y$.
    Außerdem kann man dann auch $X + Y = Y + X$, $(X + Y) + Z = X + (Y + Z)$
    und $X + \emptyset = \emptyset + X = X$ schreiben usw.
\end{Bem}

\linie

\begin{Def}{Summe zweier topologischer Räume}\\
    Seien $(X, \T_X)$, $(Y, \T_Y)$ topologische Räume mit
    $X \cap Y = \emptyset$. \\
    Auf $X \sqcup Y$ definiert man die \begriff{Summentopologie}
    $\T := \{U \sqcup V \;|\; U \in \T_X,\; V \in \T_Y\}$. \\
    Die Summentopologie auf $X$ ist die Finaltopologie zu
    $(i\colon X \rightarrow X \sqcup Y, j\colon Y \rightarrow X \sqcup Y)$.
\end{Def}

\linie
\pagebreak

\begin{Def}{disjunkte Vereinigung beliebig vieler Mengen}\\
    Sei $(X_\lambda)_{\lambda \in \Lambda}$ ein Mengensystem mit
    $X_\lambda \cap X_\mu = \emptyset$ für $\lambda \not= \mu$. \\
    Für $X_\lambda \cap X_\mu \not= \emptyset$, $\lambda \not= \mu$
    geht man zu $X_\lambda' := X_\lambda \times \{\lambda\}$
    über, damit $X_\lambda' \cap X_\mu' = \emptyset$ für
    $\lambda \not= \mu$. \\
    Die \begriff{disjunkte Vereinigung} bezeichnet man mit
    $X = \bigsqcup_{\lambda \in \Lambda} X_\lambda :=
    \bigcup_{\lambda \in \Lambda} (X_\lambda \times \{\lambda\})$. \\
    Außerdem definiert man die \begriff{Inklusionen}
    $i_\lambda\colon X_\lambda \rightarrow X$, $i_\lambda(x) = x$
    für alle $x \in X_\lambda$.
\end{Def}

\begin{Satz}{$(i_\lambda)_{\lambda \in \Lambda}$ als finale Familie}
    $(i_\lambda)_{\lambda \in \Lambda}$ ist eine finale Familie,
    und zwar die freie: \\
    $\phi\colon \Abb(X, Y) \rightarrow
    \prod_{\lambda \in \Lambda} \Abb(X_\lambda, Y)$,
    $f \mapsto (f \circ i_\lambda)_{\lambda \in \Lambda}$
    ist eine Bijektion für jede Menge $Y$.
\end{Satz}

\linie

\begin{Def}{Summe beliebig vieler topologischer Räume}
    Seien $(X_\lambda, \T_\lambda)$, $\lambda \in \Lambda$, topologische Räume.
    Dann ist die \begriff{Summentopologie} auf
    $X = \bigsqcup_{\lambda \in \Lambda} X_\lambda$ gegeben durch
    $\T := \{\bigsqcup_{\lambda \in \Lambda} U_\lambda \;|\;
    U_\lambda \in T_\lambda\}$. \\
    Die Summentopologie auf $X$ ist die Finaltopologie zu
    $(i_\lambda\colon X_\lambda \rightarrow X)_{\lambda \in \Lambda}$.
\end{Def}

\subsection{%
    Produkte topologischer Räume%
}

\begin{Def}{initiale Abbildungsfamilien}
    Seien $X$ eine Menge, $X_i$ Mengen für $i \in I$ ($I$ Indexmenge)
    und $F = (f_i\colon X \rightarrow X_i)_{i \in I}$ eine Familie von
    Abbildungen. \\
    $F$ heißt \begriff{initiale Familie}, falls es
    zu jedem Paar $x \not= x'$ in $X$ ein $i \in I$ gibt mit
    $f_i(x) \not= f_i(x')$. \\
    Zu $g\colon Y \rightarrow X$ ($Y$ Menge) definiert man
    $g_i := f_i \circ g\colon Y \rightarrow X_i$. \\
    Die Familie $G = (g_i)_{i \in I}$ ist mit $F = (f_i)_{i \in I}$
    \begriff{kompatibel}, d.\,h. es gilt
    $\forall_{y \in Y} \exists_{x \in X} \forall_{i \in I}\;
    f_i(x) = g_i(y)$. \\
    $\prod_{i \in I}^{(F)} \Abb(Y, X_i) \subset \prod_{i \in I} \Abb(Y, X_i)$
    sei die Teilmenge der mit $F$ kompatiblen Familien.
\end{Def}

\begin{Satz}{Bijektion}
    Sei $F = (f_i\colon X \rightarrow X_i)_{i \in I}$ eine initiale Familie. \\
    Dann ist
    $\phi\colon \Abb(Y, X) \rightarrow \prod_{i \in I}^{(F)} \Abb(Y, X_i)$,
    $g \mapsto (f_i \circ g)_{i \in I}$ eine Bijektion.
\end{Satz}

\linie

\begin{Def}{initiale Topologie}
    Seien $(X_i, \T_i)$ topologische Räume, $X$ eine Menge und
    $F = (f_i)_{i \in I}$ eine initiale Familie mit
    $f_i\colon X \rightarrow X_i$ für $i \in I$. \\
    Für $i \in I$ ist $\T_{f_i} := \{f_i^{-1}(U) \;|\; U \in \T_i\}$ eine
    Topologie auf $X$, und zwar die gröbste, sodass
    $f_i\colon X \rightarrow X_i$ stetig ist.
    Die von $F$ induzierte \begriff{Initialtopologie} $\T_F$ auf $X$ ist
    die von \\
    $\E = \bigcup_{i \in I} \T_{f_i} =
    \{f_i^{-1}(U) \;|\; i \in I,\; U \in \T_i\}$ erzeugte Topologie auf $X$.
\end{Def}

\begin{Satz}{Eigenschaften der Initialtopologie}
    \begin{enumerate}
        \item
        $\T_F$ ist die gröbste Topologie auf $X$, sodass alle
        $f_i\colon X \rightarrow X_i$ stetig sind.

        \item
        Für jeden Raum $Y$ ist $g\colon Y \rightarrow X$ stetig (bzgl. $\T_F$)
        genau dann, wenn für alle $i \in I$ die Komposition
        $g_i = f_i \circ g$ stetig ist.

        \item
        $\phi\colon \C(Y, X) \rightarrow
        \prod_{i \in I}^{(F)} \C(Y, X_i)$, $g \mapsto (f_i \circ g)_{i \in I}$
        ist eine Bijektion.
    \end{enumerate}
\end{Satz}

\linie

\begin{Def}{Produkt zweier Mengen}
    Seien $X$ und $Y$ Mengen. \\
    Das \begriff{Produkt} von $X$ mit $Y$ bezeichnet man mit
    $X \times Y := \{(x, y) \;|\; x \in X,\; y \in Y\}$. \\
    Außerdem definiert man die \begriff{Projektionen}
    $p\colon X \times Y \rightarrow X$, $p(x, y) = x$,
    sowie $q\colon X \times Y \rightarrow Y$, $q(x, y) = y$.
\end{Def}

\begin{Satz}{$(p, q)$ als initiale Familie}
    $(p, q)$ ist eine initiale Familie,
    und zwar die freie: \\
    $\phi\colon \Abb(Z, X \times Y) \rightarrow \Abb(Z, X) \times \Abb(Z, Y)$,
    $f \mapsto (p \circ f, q \circ f)$ ist eine Bijektion für jede Menge $Z$.
\end{Satz}

\begin{Bem}
    Man schreibt manchmal $Y^X := \Abb(X, Y)$. \\
    Damit ist nämlich $(X \times Y)^Z \cong X^Z \times Y^Z$.
    Außerdem kann man dann auch $X \times Y \cong Y \times X$,
    $(X \times Y) \times Z \cong X \times (Y \times Z)$,
    $X \times \{a\} \cong \{a\} \times X \cong X$ und
    $(X \sqcup Y) \times Z = (X \times Z) \sqcup (Y \times Z)$ schreiben usw.
\end{Bem}

\linie

\begin{Def}{Produkt zweier topologischer Räume}
    Seien $(X, \T_X)$, $(Y, \T_Y)$ topologische Räume.
    Eine Menge $W \subset X \times Y$ heißt
    \begriff{of"|fen in der Produkttopologie}, falls es zu allen $(x, y) \in W$
    of"|fene Umgebungen $U \in \T_X$, $V \in \T_Y$, $x \in U$, $y \in V$ gibt,
    sodass $U \times V \subset W$ ist. \\
    Die Produkttopologie auf $X \times Y$ ist die Initialtopologie zu
    $(p\colon X \times Y \rightarrow X, q\colon X \times Y \rightarrow Y)$.
\end{Def}

\begin{Bem}
    Die \begriff{of"|fenen Kästchen} $U \times V$ mit
    $U \in \T_X$, $V \in \T_Y$ definieren selbst noch keine Topologie, denn
    sie sind zwar unter dem Durchschnitt, aber nicht unter der Vereinigung
    abgeschlossen.
    Die of"|fenen Kästchen bilden jedoch eine Basis der Produkttopologie.
\end{Bem}

\linie

\begin{Def}{Produkt beliebig vieler Mengen}
    Sei $(X_\lambda)_{\lambda \in \Lambda}$ ein Mengensystem. \\
    Das \begriff{Produkt} aller $X_\lambda$ bezeichnet man mit \\
    $X = \prod_{\lambda \in \Lambda} X_\lambda :=
    \{(x_\lambda)_{\lambda \in \Lambda} \;|\;
    \forall_{\lambda \in \Lambda}\; x_\lambda \in X_\Lambda\} =
    \{x\colon X \rightarrow \bigcup_{\lambda \in \Lambda} X_\lambda \;|\;
    \forall_{\lambda \in \Lambda}\; x(\lambda) \in X_\Lambda\}$. \\
    Außerdem definiert man die \begriff{Projektionen}
    $p_\lambda\colon X \rightarrow X_\lambda$,
    $(x_\mu)_{\mu \in \Lambda} \mapsto x_\lambda$.
\end{Def}

\begin{Satz}{$(p_\lambda)_{\lambda \in \Lambda}$ als initiale Familie}
    $(p_\lambda)_{\lambda \in \Lambda}$ ist eine initiale Familie,
    und zwar die freie: \\
    $\phi\colon \Abb(Y, X) \rightarrow
    \prod_{\lambda \in \Lambda} \Abb(Y, X_\lambda)$,
    $f \mapsto (p_\lambda \circ f)_{\lambda \in \Lambda}$ ist eine Bijektion
    für jede Menge $Y$.
\end{Satz}

\linie

\begin{Def}{Produkt beliebig vieler topologischer Räume}
    Seien $(X_\lambda, \T_\lambda)$, $\lambda \in \Lambda$, topologische Räume.
    Für $\lambda \in \Lambda$ definiert man
    $\T_{p_\lambda} := \{p_\lambda^{-1}(U) \;|\; U \in \T_\lambda\}$.
    Dann ist die \begriff{Produkttopologie} auf
    $X = \prod_{\lambda \in \Lambda} X_\lambda$ die von
    $\E := \bigcup_{\lambda \in \Lambda} T_{p_\lambda} =
    \{p_\lambda^{-1}(U) \;|\; \lambda \in \Lambda,\; U_\lambda \in T_\lambda\}$
    erzeugte Topologie auf $X$. \\
    Eine Basis der Produkttopologie ist also \\
    $\B = \left\{p_{\lambda_1}^{-1}(U_1) \cap \dotsb \cap
    p_{\lambda_n}^{-1}(U_n) \;|\; n \in \natural,\;
    \lambda_1, \dotsc, \lambda_n \in \Lambda,\;
    U_1 \in \T_{\lambda_1}, \dotsc, U_n \in \T_{\lambda_n}\right\}$ \\
    $= \left\{\prod_{\lambda \in \Lambda} U_\lambda \;|\;
    U_\lambda \in \T_\lambda,\;
    U_\lambda = X_\lambda \text{ für fast alle }
    \lambda \in \Lambda\right\}$. \\
    Die Produkttopologie auf $X$ ist die Initialtopologie zu
    $(p_\lambda\colon X \rightarrow X_\lambda)_{\lambda \in \Lambda}$.
\end{Def}

\begin{Bsp}
    Für $\Lambda = \{1, 2\}$ oder $\Lambda$ endlich ist die Produkttopologie
    wie oben die Produkttopologie von endlich vielen Räumen.
    Auf $\real^\real$ ist die Produkttopologie genau die Topol. der
    pktw. Konv.
\end{Bsp}

\linie

\begin{Bem}
    Sind $(X_1, d_1), \dotsc, (X_n, d_n)$ metrische Räume, dann definiert
    $d(x, y) :=$ \\
    $\sup\{d_i(x_i, y_i)\}$
    eine Metrik auf $X := X_1 \times \dotsb \times X_n$, die die
    Produkttopologie auf $X$ induziert.
\end{Bem}

\begin{Satz}{Metrisierbarkeit von Produkträumen}
    Seien $(X_i, \T_i)$ topologische Räume, die mindestens zwei Elemente
    $a_i \not= b_i$ enthalten ($i \in I$).
    Das Produkt $X = \prod_{i \in I} X_i$ wird mit der Produkttopologie
    auf $X$ versehen.
    \begin{enumerate}
        \item
        Ist $X$ metrisierbar, dann auch alle $X_i$.

        \item
        Sind alle $X_i$ metrisierbar und $I$ abzählbar, dann ist $X$
        metrisierbar.

        \item
        Ist $I$ überabzählbar, dann ist $X$ nicht metrisierbar.
    \end{enumerate}
\end{Satz}

\begin{Satz}{Metrisierbarkeit der Initialtopologie}
    Seien $(X_i, d_i)$, $i \in \natural$ metrische Räume und \\
    $f_i\colon X \rightarrow X_i$ eine initiale Familie.
    $d(x, y) := \sum_{i=0}^\infty 2^{-i} d_i^\ast(f_i(x), f_i(y))$ induz.
    die Initialtopologie.
\end{Satz}

\linie

\begin{Def}{Zylinder, Kegel, Einhängung}
    Sei $X$ ein topologischer Raum.
    Der \begriff{Zylinder über $X$} ist der Raum $ZX := X \times [0,1]$.
    Der \begriff{Kegel über $X$} ist der Raum
    $CX := ZX/\!/(X \times \{1\})$.
    Die \begriff{Einhängung} oder der \begriff{Doppelkegel} ist der Raum
    $\Sigma X := (X \times [-1,1])/\!/(X \times \{1\})/\!/(X \times \{-1\})$.
\end{Def}

\begin{Bsp}
    $C \sphere^{n-1} \cong \dball^n$,
    $\Sigma \sphere^{n-1} \cong \sphere^n$
\end{Bsp}

\begin{Def}{Verheften von Räumen}\\
    Seien $X, Y$ topologische Räume  mit $A \subset X$, $B \subset Y$ und
    $f\colon A \rightarrow B$ eine Abbildung. \\
    Auf $X \sqcup Y$ definiert man $\sim$ als die von $a \sim f(a)$, $a \in A$
    erzeugte Äquivalenzrelation. \\
    Die \begriff{Verheftung} von $X$ und $Y$ entlang $f$ ist der Quotientenraum
    $X \cup_f Y := (X \sqcup Y)/\sim$.
\end{Def}

\pagebreak

\section{%
    Kompaktheit%
}

\subsection{%
    Kompakte topologische Räume%
}

\begin{Bem}
    Kompaktheit ist ein topologisches Analogon zur Endlichkeit, denn \\
    für jede endliche Menge $X$ \emph{(jeden kompakten Raum $X$)} gilt:
    \begin{enumerate}
        \item
        Jede \emph{(of"|fene)} Überdeckung $X = \bigcup_{i \in I} U_i$ enthält
        eine endliche Teilüberdeckung, d.\,h.
        $X = U_{i_1} \cup \dotsb \cup U_{i_n}$.

        \item
        Jede Folge $(x_n)_{n \in \natural}$ in $X$ besitzt eine konstante
        Teilfolge \emph{(einen Häufungspunkt)}.

        \item
        Jede \emph{(stetige)} Funktion $f\colon X \rightarrow \real$ ist
        beschränkt und nimmt ihre Extrema an, d.\,h.
        $\exists_{a, b \in X} \forall_{x \in X}\; f(a) \le f(x) \le f(b)$.
    \end{enumerate}
\end{Bem}

\linie

\begin{Def}{Überdeckung}
    Seien $(X, \T)$ ein topologischer Raum und $A \subset X$ eine Teilmenge. \\
    Eine Familie $(U_i)_{i \in I}$ mit $U_i \subset X$ heißt
    \begriff{Überdeckung} von $A$, falls $A \subset \bigcup_{i \in I} U_i$. \\
    Sie heißt \begriff{of"|fen}, falls $\forall_{i \in I}\; U_i \in \T$,
    und \begriff{endlich}, falls $I$ endlich ist. \\
    Eine \begriff{Teilüberdeckung} ist eine Familie $(U_i)_{i \in J}$ mit
    $J \subset I$ und $A \subset \bigcup_{i \in J} U_i$.
\end{Def}

\begin{Def}{kompakt}
    $(X, \T)$ heißt \begriff{kompakt}, falls jede of"|fene Überdeckung
    $(U_i)_{i \in I}$ von $X$ eine endliche Teilüberdeckung enthält, d.\,h.
    $\exists_{i_1, \dotsc, i_n \in I}\; X = U_{i_1} \cup \dotsb \cup U_{i_n}$.
\end{Def}

\begin{Bsp}
    Falls $X$ diskret ist, ist $X$ kompakt genau dann, wenn $X$ endlich ist. \\
    Falls $X$ indiskret ist, so ist $X$ immer kompakt. \\
    Der euklidische Raum $\real^n$ ist nicht kompakt, denn
    $\real^n = \bigcup_{r \in \natural} B(0, r)$ ist eine of"|fene Überdeckung,
    aber es gibt keine endliche Teilüberdeckung.
\end{Bsp}

\linie

\begin{Def}{kompakte Teilmenge}
    $A \subset X$ heißt \begriff{kompakt in $X$}, falls $(A, \T_A)$
    kompakt ist (bzgl. der Teilraumtopologie $\T_A$),
    und \begriff{relativ kompakt in $X$}, falls $\abschluss{A}$ kompakt in $X$
    ist.
\end{Def}

\begin{Bem}
    Nach Definition der Teilraumtopologie ist $V \subset A$ of"|fen in $\T_A$
    genau dann, wenn $\exists_{U \in \T_X}\; V = A \cap U$.
    Damit sind die folgenden Bedingungen äquivalent: \\
    Jede Überdeckung $A = \bigcup_{i \in I} V_i$ mit $V_i \in \T_A$
    enthält eine endliche Teilüberdeckung. \\
    Jede Überdeckung $A \subset \bigcup_{i \in I} U_i$ mit $U_i \in \T_X$
    enthält eine endliche Teilüberdeckung.
\end{Bem}

\begin{Bsp}
    Jede endliche Menge $A \subset X$ ist kompakt in $X$. \\
    $\integer \subset \real$, $\rational \subset \real$ sind nicht kompakt,
    denn
    $\integer, \rational \subset \bigcup_{r \in \natural} \left]-r, r\right[$
    enthält keine endliche Teilüberdeckung.
    $\left]0, 1\right]$ ist nicht kompakt, denn
    $\left]0, 1\right] \subset
    \bigcup_{n \in \natural} \left]\frac{1}{n}, 2\right[$
    enthält keine endliche Teilüberdeckung
    (aber $\abschluss{\left]0, 1\right]} = [0, 1]$ ist kompakt, s.\,u.).
\end{Bsp}

\linie

\begin{Satz}{stetiges Bild einer kompakten Menge ist kompakt}\\
    Seien $X$ kompakt und $f\colon X \rightarrow Y$ stetig.
    Dann ist $f(X)$ kompakt.
\end{Satz}

\linie

\begin{Satz}{$[0, 1]$ ist kompakt}
    $[0, 1] = \{x \in \real \;|\; 0 \le x \le 1\}$ ist kompakt in $\real$.
\end{Satz}

\linie
\pagebreak

\begin{Satz}{abgeschlossene Teilmengen von kompakten Räumen sind kompakt}\\
    Seien $X$ kompakt und $A \subset X$ abgeschlossen.
    Dann ist $A$ kompakt in $X$.
\end{Satz}

\begin{Satz}{kompakte Teilmengen von Hausdorff-Räumen sind abgeschlossen}\\
    Seien $X$ hausdorffsch und $A \subset X$ kompakt.
    Dann ist $A$ abgeschlossen.
\end{Satz}

\begin{Lemma}{disjunkte Umgebungen bei kompakten Mengen in Hausdorff-Räumen}\\
    Seien $X$ hausdorffsch und $A \subset X$ kompakt. \\
    Dann gibt es für jedes $b \in X \setminus A$ of"|fene Umgebungen
    $U$ von $A$ und $V$ von $b$ mit $U \cap V = \emptyset$.
\end{Lemma}

\begin{Satz}{disjunkte Umgebungen von zwei kompakten Mengen in
             Hausdorff-Räumen}\\
    Seien $X$ hausdorffsch und $A, B \subset X$ kompakt. \\
    Dann gibt es of"|fene Umgebungen $U$ von $A$ und $V$ von $b$ mit
    $U \cap V = \emptyset$.
\end{Satz}

\linie

\begin{Satz}{Produkt von kompakten Räumen ist kompakt}
    Seien $(X, \T_X)$ und $(Y, \T_Y)$ kompakt. \\
    Dann ist auch $X \times Y$ in der Produkttopologie kompakt.
\end{Satz}

\begin{Kor}
    Jedes endliche Produkt kompakter Räume ist kompakt.
\end{Kor}

\begin{Bsp}
    Jeder Quader $[a_1, b_1] \times \dotsb \times [a_n, b_n] \subset \real^n$
    ist kompakt.
\end{Bsp}

\begin{Satz}{\name{Heine}-\name{Borel}}\\
    $A \subset \real^n$ ist kompakt genau dann, wenn $A$ beschränkt und
    abgeschlossen ist.
\end{Satz}

\linie

\begin{Satz}{stetige Funktion von einem kompaktem Raum ist beschränkt,
             nimmt ihre Extrema an} \\
    Seien $X$ kompakt und $f\colon X \rightarrow \real$ stetig. \\
    Dann gibt es $a, b \in X$, sodass $f(a) \le f(x) \le f(b)$ für alle
    $x \in X$.
\end{Satz}

\begin{Satz}{stetige Abbildungen von kompakten in Hausdorff-Räumen sind
             abgeschlossen} \\
    Seien $X$ kompakt, $Y$ hausdorffsch und $f\colon X \rightarrow Y$ stetig.
    Dann ist $f$ abgeschlossen. \\
    Insbesondere gilt:
    Ist $f$ injektiv/surjektiv/bijektiv, \\
    dann ist $f$ Einbettung/Identifizierung/Homöomorphismus.
\end{Satz}

\begin{Satz}{Vergleich von kompakten Hausdorff-Räumen}
    Sei $(X, \T)$ kompakt und hausdorffsch. \\
    Jede echt feinere Topologie $\T' \supsetneqq \T$ ist hausdorffsch,
    aber nicht kompakt. \\
    Jede echt gröbere Topologie $\T' \subsetneqq \T$ ist kompakt,
    aber nicht hausdorffsch.
\end{Satz}

\subsection{%
    Der Satz von \name{Tychonoff}%
}

\begin{Satz}{topologischer Raum kompakt $\;\Leftrightarrow\;$
             jeder Ultrafilter konvergiert} \\
    Ein topologischer Raum $(X, \T)$ ist kompakt genau dann, wenn
    jeder Ultrafilter $\F$ auf $X$ konvergiert
    ($\exists_{x \in X}\; \F \supset \U_x$).
\end{Satz}

\begin{Lemma}{Filter auf Produkt konvergiert $\;\Leftrightarrow\;$
              Bildfilter der Projektionen konvergieren gegen
              jede einzelne Komponente}
    Ein Filter $\F$ auf $X = \prod_{i \in I} X_i$ konvergiert gegen
    $x = (x_i)_{i \in I}$ auf $X$ genau dann, wenn jeder
    Bildfilter $p_i(\F)$ auf $X_i$ gegen $x_i$ konvergiert, wobei
    $p_i\colon X \rightarrow X_i$, $p_i(x) = x_i$
    die Projektion auf die $i$-te Komponente ist ($i \in I$).
\end{Lemma}

\begin{Satz}{\name{Tychonoff}}
    $X = \prod_{i \in I} X_i$ ist kompakt genau dann, wenn alle $X_i$ kompakt
    sind.
\end{Satz}

\pagebreak

\subsection{%
    Erste Anwendungen%
}

\begin{Def}{konvex}
    Eine Teilmenge $X \subset \real^n$ heißt \begriff{konvex}, falls
    für alle $a, b \in X$ auch $[a, b] \subset X$ gilt, wobei
    $[a, b] := \{(1 - t)a + tb \;|\; t \in [0, 1]\}$.
\end{Def}

\begin{Def}{sternförmig}
    Eine Teilmenge $X \subset \real^n$ heißt
    \begriff{sternförmig bzgl. $a \in X$},
    falls für alle $b \in X$ auch $[a, b] \subset X$ gilt.
\end{Def}

\begin{Bem}\\
    Eine Menge $X \subset \real^n$ ist konvex genau dann, wenn sie sternförmig
    bzgl. jeden ihrer Punkte ist. \\
    $I \subset \real$ ist sternförmig $\;\Leftrightarrow\;$
    $I$ ist konvex $\;\Leftrightarrow\;$
    $I$ ist ein Intervall.
\end{Bem}

\begin{Satz}{kompakte, konvexe Menge mit Innerem
             ist homöomorph zur Vollkugel}\\
    Sei $X \subset \real^n$ kompakt, konvex und $\inneres{X} \not= \emptyset$.
    Dann ist $X \cong \dball^n$.
\end{Satz}

\begin{Satz}{allgemeinere Version}\\
    Sei $X \subset \real^n$ kompakt und sternförmig bzgl. allen
    $a \in B(a_0, \varepsilon)$ für ein $a_0 \in X$ und ein
    $\varepsilon > 0$. \\
    Dann ist $X \cong \dball^n$.
    Es gilt sogar:
    Es gibt ein $h\colon \real^n \homoe \real^n$ mit $h(X) = \dball^n$.
\end{Satz}

\linie

\begin{Def}{topologischer Vektorraum}
    Ein \begriff{topologischer Vektorraum} $(V, +, \cdot, \T)$ über $\real$ ist
    ein \\
    $\real$-Vektorraum $(V, +, \cdot)$ mit einer Topologie $\T$ auf $V$,
    wobei gilt:
    \begin{enumerate}
        \item
        $+\colon V \times V \rightarrow V$ ist stetig.

        \item
        $\cdot\colon \real \times V \rightarrow V$ ist stetig.

        \item
        $(V, \T)$ ist hausdorffsch.
    \end{enumerate}
\end{Def}

\begin{Bsp}
    $(\real^n, +, \cdot)$ mit der euklidischen Topologie ist ein topologischer
    Vektorraum. \\
    Auf $\C([0, 1], \real)$ definieren die $p$-Normen unendlich viele
    verschiedene Vektorraum-Topolo"-gien.
\end{Bsp}

\begin{Satz}{lineare Abbildungen stetig, Homöomorphismen}
    \begin{enumerate}
        \item
        Jede lineare Abbildung $f\colon \real^n \rightarrow V$ ist stetig
        ($\real^n$ mit eukl., $V$ mit VR-Topologie).

        \item
        Für $x \in \real \setminus \{0\}$ ist $\mu_x\colon V \rightarrow V$,
        $\mu_x(v) = xv$ ein Homöomorphismus.

        \item
        Für $a \in V$ ist $\tau_a\colon V \rightarrow V$, $\tau_a(v) = v + a$
        ein Homöomorphismus.
    \end{enumerate}
\end{Satz}

\begin{Def}{ausgeglichene Umgebung}\\
    Eine Umgebung $A \in \U_0$ von $0 \in V$ heißt \begriff{ausgeglichen},
    falls $tA \subset A$ für alle $t \in [-1, 1]$.
\end{Def}

\begin{Lemma}{ausgeglichene Umgebungen bilden eine Umgebungsbasis der $0$}\\
    Jede Umgebung $U$ von $0$ in $V$ enthält eine of"|fene,
    ausgeglichene Umgebung $A$ von $0$ in $V$.
\end{Lemma}

\begin{Satz}{auf endlich-dimensionalen $\real$-Vektorräumen gibt es genau
             eine Vektorraum-Topologie}\\
    Auf jedem endlich-dimensionalen $\real$-Vektorraum $(V, +, \cdot)$
    gibt es genau eine Vektorraum-Topo\-logie.
    Insbesondere ist die euklidische Topologie auf $\real^n$ die einzige
    Vektorraum-Topologie auf $(\real^n, +, \cdot)$.
\end{Satz}

\begin{Bem}
    Die Kompaktheit ist stets wesentlich!
    Der Satz gilt nicht für unendlich-dimensionale Vektorräume
    und auch nicht für topologische $\rational$-Vektorräume:
    Versieht man z.\,B. $\rational^2 \subset \real^2$ mit der Produkttopologie
    und $\rational[\sqrt{2}] := \{a + b\sqrt{2} \;|\; a, b \in \rational\}$
    mit der Teilraumtopologie, so ist $\rational^2 \cong \rational[\sqrt{2}]$
    als $\rational$-Vektorraum (bspw. durch $(a, b) \mapsto a + b \sqrt{2}$),
    aber nicht als topologische $\rational$-Vektorräume.
\end{Bem}

\pagebreak

\subsection{%
    Verwandte Kompaktheitsbegrif"|fe%
}

\begin{Def}{\name{Lebesgue}-Zahl}
    Seien $(X, d)$ ein metrischer Raum und $(U_i)_{i \in I}$ eine of"|fene
    Überdeckung von $X$.
    Eine Zahl $\delta > 0$ heißt \begriff{\name{Lebesgue}-Zahl} zu
    $(U_i)_{i \in I}$, falls
    $\forall_{x \in X} \exists_{i \in I}\; B(x, \delta) \subset U_i$.
\end{Def}

\begin{Bsp}
    Zu $(\left]n, n + 2\right[)_{n \in \natural}$ ist $\delta = \frac{1}{2}$
    eine Lebesgue-Zahl. \\
    Sei $x_n = \ln n$ für $n \in \natural$, $x_0 = -\infty$.
    Die Überdeckung $(\left]x_n, x_{n+2}\right[)_{n \in \natural}$
    von $\real$ erlaubt keine Lebesgue-Zahl bzgl. $d(x, y) = |x - y|$. \\
    Man kann allerdings auf $\real$ auch die zu $d(x, y) = |x - y|$
    äquivalente Metrik $e(x, y) = |e^x - e^y|$ betrachten.
    Bzgl. $e$ erlaubt $(\left]x_n, x_{n+2}\right[)_{n \in \natural}$
    die Lebesgue-Zahl $\delta = \frac{1}{2}$, d.\,h. die Lebesgue-Zahl
    und auch schon die Existenz einer solchen hängt wesentlich von der
    gewählten Metrik ab.
\end{Bsp}

\begin{Lemma}{\name{Lebesgue}}
    Sei $(X, d)$ ein kompakter metrischer Raum. \\
    Dann existiert zu jeder of"|fenen Überdeckung $(U_i)_{i \in I}$
    von $X$ eine Lebesgue-Zahl.
\end{Lemma}

\begin{Bem}
    Die Umkehrung gilt nicht, d.\,h. die Existenz einer Lebesgue-Zahl
    für jede of"|fene Überdeckung impliziert nicht die Kompaktheit.
    Ein Gegenbeispiel ist $(X, d)$ mit der diskreten Metrik.
    $\delta = 1$ ist eine Lebesgue-Zahl für jede of"|fene Überdeckung,
    aber $(X, d)$ ist nicht kompakt, wenn $X$ unendlich ist.
\end{Bem}

\linie

\begin{Def}{totalbeschränkt}
    Ein metrischer Raum $(X, d)$ heißt \begriff{totalbeschränkt}, falls
    es zu jedem $\varepsilon > 0$ eine endliche Familie
    $a_1, \dotsc, a_n \in X$ gibt mit
    $X = B(a_1, \varepsilon) \cup \dotsb \cup B(a_n, \varepsilon)$.
\end{Def}

\begin{Satz}{metrischer Raum kompakt $\;\Leftrightarrow\;$
             totalbeschränkt, erlaubt immer Lebesgue-Zahl}\\
    Ein metrischer Raum $(X, d)$ ist kompakt genau dann, wenn er
    totalbeschränkt ist und jede of"|fene Überdeckung eine Lebesgue-Zahl
    erlaubt.
\end{Satz}

\begin{Satz}{Charakterisierungen kompakter metrischer Räume}\\
    Für jeden metrischen Raum $(X, d)$ sind äquivalent:
    \begin{enumerate}
        \item
        \textbf{Kompaktheit}:
        Jede of"|fene Überdeckung von $X$ enthält eine endliche
        Teilüberdeckung.

        \item
        \textbf{abzählbare Kompaktheit}:
        Jede Folge $(x_n)_{n \in \natural}$ in $X$ hat einen Häufungspunkt
        in $X$.

        \item
        \textbf{Folgenkompaktheit}:
        Jede Folge $(x_n)_{n \in \natural}$ in $X$ hat eine konvergente
        Teilfolge.

        \item
        \textbf{Pseudokompaktheit}:
        Jede stetige Funktion $f\colon X \rightarrow \real$ ist beschränkt.

        \item
        \textbf{\name{Lebesgue}-Kompaktheit}: \\
        $X$ ist totalbeschränkt und jede of"|fene Überdeckung erlaubt eine
        Lebesgue-Zahl.

        \item
        \textbf{\name{Heine}-\name{Borel}-\name{Lebesgue}-Kompaktheit}: \\
        $X$ ist totalbeschränkt und vollständig.
    \end{enumerate}
\end{Satz}

\begin{Bem}
    Während die Kompaktheitsbegrif"|fe 1., 2. und 3. nur eine topologische
    Beschreibung darstellen, d.\,h. sich auch auf topologische Räume
    ausweiten lassen,
    sind die Definitionen 5. und 6. metrischer Natur, denn sie basieren
    auf Totalbeschränktheit und Lebesgue-Zahlen. \\
    Die 4. Definition der Pseudokompaktheit stellt eine Mischung dar. \\
    Für topologische Räume gelten folgende Beziehungen:
    \displaymathother
    \begin{align*}
        \begin{xy}
            \xymatrix@C=2cm{
                *+++[F]\txt{kompakt}
                \ar @<-1ex> @{=>} [r] &
                *+++[F]\txt{abz. kompakt}
                \ar @<-1ex> @{=>} [r]_{\text{1. AA}}
                \ar @<-1ex> @{=>} [l]_{\text{2. AA}\quad} &
                *+++[F]\txt{folgenkompakt}
                \ar @<-1ex> @{=>} [l]
            }
        \end{xy}
    \end{align*}
    \displaymathnormal
    %Wenn man 1., 2. und 3. auf topologische Räume verallgemeinert, gehen
    %dabei die Äquivalenzen verloren und man erhält:
    %Kompakt $\;\Rightarrow\;$ abzählbar kompakt, \\
    %aber abzählbar kompakt $\;\Rightarrow\;$ kompakt
    %nur mit dem 2. Abzählbarkeitsaxiom, \\
    %folgenkompakt $\;\Rightarrow\;$ abzählbar kompakt, \\
    %aber abzählbar kompakt $\;\Rightarrow\;$ folgenkompakt
    %nur mit dem 1. Abzählbarkeitsaxiom, \\
    %und i.\,A. kompakt $\;\not\Rightarrow\;$ folgenkompakt,
    %i.\,A. folgenkompakt $\;\not\Rightarrow\;$ kompakt.
    %\displaymathother
    %\begin{align*}
    %    \begin{xy}
    %        \xymatrix{
    %            \text{kompakt}
    %            \ar@{=>}[dr] & &
    %            \text{folgenkompakt}
    %            \ar@{=>}[dl] \\
    %            & \text{abz. kompakt}
    %            \ar@{=>}@<2ex>[ul]_{\text{2. Abz.axiom}}
    %            \ar@{=>}@/_ 10mm/[ur]_{\text{1. Abz.axiom}}
    %        }
    %    \end{xy}
    %\end{align*}
    %\displaymathnormal
\end{Bem}

\pagebreak

\subsection{%
    Lokal-kompakte Räume und Alexandroff-Kompaktifizierung%
}

\begin{Def}{Kompaktifizierung}
    Sei $X$ ein topologischer Raum.
    Eine Einbettung $\iota\colon X \rightarrow Y$ in einen kompakten
    Raum $Y$ mit $\abschluss{\iota(X)} = Y$
    heißt \begriff{Kompaktifizierung} auf $X$.
\end{Def}

\begin{Bsp}
    Ist $X$ kompakt, so ist $\id\colon X \rightarrow X$ eine
    Kompaktifizierung. \\
    $\iota\colon \real \rightarrow \overline{\real} :=
    \real \cup \{+\infty\} \cup \{-\infty\}$ ist eine Kompaktifizierung, wobei
    $\overline{\real}$ mit der fortgesetzten Ordnungstopologie auf $\real$
    versehen wird, d.\,h. $-\infty \le a \le +\infty$ für alle
    $a \in \overline{\real}$. \\
    $\iota\colon \real \rightarrow \real\projective^1$, $\iota(x) = [x:1]$,
    dabei ist $\real\projective^1 \setminus \iota(\real) = \{\infty\}$ mit
    $\infty := [1:0]$. \\
    $\iota\colon \complex \rightarrow \complex\projective^1$,
    $\iota(x) = [x:1]$,
    dabei ist $\complex\projective^1 \setminus \iota(\complex) = \{\infty\}$
    mit $\infty := [1:0]$.
\end{Bsp}

\linie

\begin{Def}{lokal-kompakt}
    Sei $X$ ein topologischer Raum.
    $X$ heißt \begriff{lokal-kompakt}, falls jede Umgebung eines Punktes
    $x \in X$ eine kompakte Umgebung von $x$ enthält.
\end{Def}

\begin{Bsp}
    $\real^n$ ist lokal-kompakt (z.\,B. mit
    $\abschluss{B(x, \varepsilon)}$).
    Jede of"|fene Menge $X \subset \real^n$ ist lokal-kompakt. \\
    $\rational$ ist nicht lokal-kompakt.
\end{Bsp}

\begin{Satz}{of \!\!fene/abgeschlossene Teilmengen}
    Jede of"|fene/abgeschlossene Teilmenge $Y \subset X$
    eines lokal-kompakten Raums $X$ ist lokal-kompakt.
\end{Satz}

\begin{Satz}{in Hausdorff-Räumen reicht eine Umgebung für lokale Kompaktheit}\\
    Sei $(X, \T)$ ein Hausdorff-Raum, in dem jeder Punkt $x \in X$ eine
    kompakte Umgebung besitzt.
    Dann ist $X$ lokal-kompakt.
\end{Satz}

\begin{Kor}
    Jeder kompakte Hausdorff-Raum ist lokal-kompakt.
\end{Kor}

\begin{Satz}{Umgebungen von kompakten Mengen in lokal-kompakten Räumen}\\
    Seien $X$ lokal-kompakt und $K \subset X$ kompakt.
    Dann enthält jede Umgebung $U$ von $K$ eine kompakte Umgebung $V$ von $K$,
    d.\,h. $K \subset \inneres{V} \subset V \subset U$.
\end{Satz}

\linie

\begin{Satz}{Alexandroff-Kompaktifizierung}\\
    Sei $(X, \T)$ ein lokal-kompakter Hausdorff-Raum.
    \begin{enumerate}
        \item
        Es existiert ein kompakter Hausdorff-Raum $(\widehat{X}, \widehat{\T})$
        und eine Einbettung $\iota\colon X \rightarrow \widehat{X}$, sodass
        $\widetilde{X} \setminus \iota(X) =: \{\infty\}$ nur aus einem
        Punkt besteht.

        \item
        Ist $\kappa\colon X \rightarrow Y$ eine of"|fene Einbettung in einen
        kompakten Hausdorff-Raum $Y$, dann ist die Abbildung
        $f\colon Y \rightarrow \widehat{X}$, $f \circ \kappa = \iota$ und
        $f(Y \setminus \kappa(X)) = \{\infty\}$ stetig.

        \item
        Sind $\iota\colon X \rightarrow \widehat{X}$ und
        $\kappa\colon X \rightarrow \widetilde{X}$ Einbettungen mit der
        Eigenschaft 1., dann sind $\widehat{X}$ und $\widetilde{X}$ homöomorph
        durch $h\colon \widehat{X} \homoe \widetilde{X}$,
        $h \circ \iota = \kappa$ und
        $h(\widehat{\infty}) = \widetilde{\infty}$.
    \end{enumerate}
\end{Satz}

\begin{Lemma}{Konstruktion von $(\widehat{X}, \widehat{\T})$}
    Sei $\infty \notin X$. \\
    Definiere $\widehat{X} := X \cup \{\infty\}$ und
    $\widehat{\T} := \T \cup \{\widehat{X} \setminus K \;|\;
    K \subset X \text{ abgeschlossen und kompakt}\}$. \\
    Dann ist $\widehat{\T}$ eine Topologie auf $\widehat{X}$.
\end{Lemma}

\begin{Lemma}{Inklusion als Einbettung}
    Die Inklusion $\iota\colon X \rightarrow \widehat{X}$ ist eine Einbettung.
\end{Lemma}

\begin{Lemma}{$\widehat{X}$ ist kompakt}
    $(\widehat{X}, \widehat{\T})$ ist kompakt.
    Für $X$ nicht kompakt gilt $\abschluss{X} = \widehat{X}$.
\end{Lemma}

\begin{Lemma}{$(\widehat{X}, \widehat{\T})$ hausdorffsch $\;\Leftrightarrow\;$
              $(X, \T)$ hausdorffsch und lokal-kompakt}\\
    $(\widehat{X}, \widehat{\T})$ ist hausdorffsch genau dann, wenn
    $(X, \T)$ hausdorffsch und lokal-kompakt ist.
\end{Lemma}

\begin{Lemma}{Eindeutigkeit bis auf Homöomorphie}\\
    Seien $\widehat{X}$ und $\widetilde{X}$ Alexandroff-Kompaktifizierungen
    mit den Einbettungen $\iota$ und $\kappa$. \\
    Dann ist
    $h\colon \widehat{X} \rightarrow \widetilde{X}$
    mit $h \circ \iota = \kappa$ und
    $h(\widehat{\infty}) = \widetilde{\infty}$ ein Homöomorphismus.
\end{Lemma}

\linie
\pagebreak

\begin{Bsp}
    Sei $X = \real^n$ (lokal-kompakt).
    Dann ist der kompakte Hausdorff-Raum nach 1. $\widehat{X} = \sphere^n$
    und die Einbettung $\iota\colon \real^n \rightarrow \sphere^n$
    ist die Umkehrung der stereographischen Projektion \\
    ($\sphere^n \setminus \iota(\real^n) = \{\infty\}$ besteht nur aus
    einem Punkt, dem Nordpol $\infty$). \\
    Definiert man $\kappa\colon \real \rightarrow \real\projective^1$
    wie oben, also $\kappa(x) = [x:1]$,
    dann ist $\kappa$ ebenfalls eine Einbettung, sodass
    $\real\projective^1 \setminus \kappa(\real)$ nur aus einem Punkt besteht,
    also ist $\sphere^1 \cong \real\projective^1$. \\
    Analog gilt mit $\kappa\colon \complex \rightarrow \complex\projective^1$,
    $\kappa(x) = [x:1]$, dass
    $\sphere^2 \cong \complex\projective^1$.
\end{Bsp}

\linie

\begin{Def}{$\sigma$-kompakt}
    Ein lokal-kompakter Hausdorff-Raum $X$ heißt \begriff{$\sigma$-kompakt},
    falls $X$ eine abzählbare Vereinigung kompakter Mengen ist, d.\,h.
    $X = \bigcup_{n \in \natural} K_n$ mit $K_n \subset X$ kompakt.
\end{Def}

\begin{Bsp}
    $\real^n = \bigcup_{n \in \natural} \abschluss{B(0, n)}$ ist
    $\sigma$-kompakt. \\
    $\rational = \bigcup_{x \in \rational} \{x\}$ ist nicht $\sigma$-kompakt,
    denn $\rational$ ist nicht lokal-kompakt.
\end{Bsp}

\begin{Satz}{äquivalente Beschreibungen}\\
    Für jeden lokal-kompakten Hausdorff-Raum $X$ ist Folgendes äquivalent:
    \begin{enumerate}
        \item
        $X$ ist $\sigma$-kompakt, d.\,h. $X = \bigcup_{n \in \natural} K_n$ mit
        $K_n \subset X$ kompakt.

        \item
        $X = \bigcup_{n \in \natural} U_n$ mit $U_n \subset X$ of"|fen,
        $\overline{U_n}$ kompakt, $\overline{U_n} \subset U_{n+1}$.

        \item
        $X$ ist \begriff{abzählbar im Unendlichen}, d.\,h.
        $\widehat{X}$ hat in $\infty$ eine abzählbare Umgebungsbasis.
    \end{enumerate}
\end{Satz}

\linie

\begin{Satz}{Teilmengen von lokal-kompakten Räumen}
    Sei $X$ ein lokal-kompakter Raum. \\
    Eine Teilmenge $U \subset X$ ist of"|fen in $X$ genau dann, wenn
    $U \cap K$ of"|fen in $K$ ist für alle $K \subset X$ kompakt
    (analog mit $U \subset X$ und $U \cap K$ abgeschlossen).
\end{Satz}

\begin{Def}{kompakt erzeugt}
    Sei $(X, \T)$ ein topologischer Raum. \\
    Die Topologie $\T$ heißt \begriff{kompakt erzeugt}, falls für jede
    Teilmenge $A \subset X$ gilt: \\
    Ist $A \cap K$ of"|fen in $K$ für alle $K \subset X$ kompakt,
    so ist $A$ of"|fen in $X$ \\
    (alternativ auch äquivalent dazu "`abgeschlossen"' statt "`of"|fen"').
\end{Def}

\begin{Bem}
    Das heißt, dass $\T$ die Finaltopologie bzgl. der finalen Familie \\
    $\{\iota_K\colon K \hookrightarrow X \;|\; K \subset X \text{ kompakt}\}$
    ist.
\end{Bem}

\begin{Bsp}
    Beispiele für kompakt erzeugte Räume sind lokal-kompakte Räume,
    metrische Räume und Räume mit dem 1. Abzählbarkeitsaxiom.
\end{Bsp}

\linie

\begin{Def}{eigentliche Abbildung}
    Seien $X, Y$ lokal-kompakte Hausdorff-Räume. \\
    Eine stetige Abbildung $f\colon X \rightarrow Y$ heißt
    \begriff{eigentlich}, falls für jede kompakte Menge $K \subset Y$
    das Urbild $f^{-1}(K) \subset X$ kompakt ist.
\end{Def}

\begin{Bem}
    Man kann eigentliche Abbildung auch zwischen beliebigen Räumen definieren.
    In diesem Fall muss $f$ stetig, abgeschlossen und das Urbild jedes
    Punkts muss kompakt sein. \\
    Obige Definition wird dann mit "`genau dann, wenn"' statt "`falls"'
    zum Satz.
\end{Bem}

\begin{Bsp}
    $f\colon \natural \rightarrow \real^d$, $f(n) = x_n$ ist eigentlich
    genau dann, wenn $\norm{x_n} \to \infty$ für $n \to \infty$. \\
    Eine stetige Abbildung $f\colon \real_{\ge 0} \rightarrow \real^d$ ist
    eigentlich genau dann, wenn
    $\norm{f(x)} \to \infty$ für $x \to \infty$, d.\,h.
    $\forall_{r \ge 0} \exists_{x_0 \ge 0}
    \forall_{x > x_0}\; \norm{f(x)} > r$.
\end{Bsp}

\begin{Satz}{Fortsetzung von stetigen Abbildungen auf die
             Alexandroff-Kompaktifizierungen}\\
    Seien $X, Y$ lokal-kompakte Hausdorff-Räume und
    $\widehat{X} := X \cup \{\infty\}$,
    $\widehat{Y} := Y \cup \{\widetilde{\infty}\}$
    die entsprechenden Alexandroff-Kompaktifizierungen. \\
    Eine stetige Abbildung $f\colon X \rightarrow Y$ lässt sich zu einer
    stetigen Abbildung $\widehat{f}\colon \widehat{X} \rightarrow \widehat{Y}$
    mit $\widehat{f}(\infty) = \widetilde{\infty}$ fortsetzen genau dann, wenn
    $f$ eigentlich ist.
\end{Satz}

\pagebreak

\subsection{%
    Die Kompakt-Of"|fen-Topologie%
}

\begin{Bem}
    Das Ziel dieses Abschnitts ist, $\C(X, Y)$ mit einer "`vernünftigen"'
    Topologie auszustatten.
    Man kann die Definitionen für reellwertige Funktionen
    verallgemeinern und auf $\C(X, Y)$ die Topologien der punktweisen,
    gleichmäßigen und kompakten Konvergenz definieren:
\end{Bem}

\begin{Def}{Topologie der punktweisen Konvergenz}
    Seien $X, Y$ topologische Räume.
    Auf $Y^X$ wird die Topologie $\Tpw$ der pktw. Konv. erzeugt
    von den Mengen $[x, O] := \{g\colon X \rightarrow Y \;|\; g(x) \in O\}$
    mit $x \in X$ und $O \subset Y$ of"|fen.
    Dies ist auch die Produkttopologie auf $Y^X$.
    Man erhält die Topologie der punktweisen Konvergenz auf $\C(X, Y)$
    als Teilraumtopologie.
\end{Def}

\begin{Def}{Topologie der gleichmäßigen Konvergenz}
    Sei $X$ ein topologischer und $(Y, d)$ ein metrischer Raum.
    Auf $Y^X$ wird die Topologie $\Tglm$ der gleichmäßigen Konvergenz
    induziert durch die Metrik
    $d_X(f, g) := \sup_{x \in X} d^\ast(f(x), g(x))$.
    Die Topologie wird also erzeugt von den Mengen
    $B(f, \varepsilon) :=
    \{g\colon X \rightarrow Y \;|\; d_X(f, g) < \varepsilon\}$
    mit $f\colon X \rightarrow Y$ und $\varepsilon > 0$.
    Man erhält die Topologie der gleichmäßigen Konvergenz auf $\C(X, Y)$
    als Teilraumtopologie.
\end{Def}

\begin{Def}{Topologie der kompakten Konvergenz}
    Sei $X$ ein topologischer und $(Y, d)$ ein metrischer Raum.
    Auf $Y^X$ wird die Topologie $\Tkpkt$ der kompakten Konvergenz
    erzeugt von den Mengen $B_K(f, \varepsilon) :=
    \{g\colon X \rightarrow Y \;|\; d_K(f, g) < \varepsilon\}$
    für $K \subset X$ kompakt, $f\colon X \rightarrow Y$ und $\varepsilon > 0$.
    Man erhält die Topologie der kompakten Konvergenz auf $\C(X, Y)$
    als Teilraumtopologie.
\end{Def}

\begin{Bem}
    Es gilt $\Tpw \subset \Tkpkt \subset \Tglm$.
\end{Bem}

\linie

\begin{Def}{Kompakt-Of"|fen-Topologie}
    Seien $X, Y$ topologische Räume. \\
    Für $A \subset X$, $B \subset Y$ definiert man
    $[A, B] = \C(X, A; Y, B) := \{f\colon X \rightarrow Y \text{ stetig} \;|\;
    f(A) \subset B\}$. \\
    Die \begriff{Kompakt-Of"|fen-Topologie} oder \begriff{KO-Topologie}
    auf $\C(X, Y)$ ist die von
    den Mengen $[K, O]$ mit $K \subset X$ kompakt, $O \subset Y$ of"|fen
    erzeugte Topologie.
\end{Def}

\begin{Bem}
    Die Mengen $[K, O]$ bilden i.\,A. keine Basis der KO-Topologie, sondern
    nur ein Erzeugendensystem.
    Eine Basis sind alle Mengen der Form
    $[K_1, O_1] \cap \dotsb \cap [K_n, O_n]$
    mit $n \in \natural$, $K_i \subset X$ kompakt und $O_i \subset Y$ of"|fen
    für $i = 1, \dotsc, n$. \\
    Ist $X$ diskret, dann ist $\C(X, Y) = Y^X$ und die KO-Topologie
    stimmt mit der Produkttopologie (Topologie der punktweisen Konvergenz)
    überein. \\
    Ist $X$ kompakt und $Y$ ein metrischer Raum, dann stimmt die
    KO-Topologie auf $\C(X, Y)$ mit der
    Topologie der gleichmäßigen Konvergenz überein. \\
    Ist $Y$ ein metrischer Raum, dann stimmt die
    KO-Topologie auf $\C(X, Y)$ mit der
    Topologie der kompakten Konvergenz überein.
\end{Bem}

\linie

\begin{Satz}{Einbettungen}
    Seien $X \not= \emptyset$ und $Y$ topologische Räume.
    \begin{enumerate}
        \item
        Ist $B \subset Y$ ein Teilraum, dann ist
        $\C(X, B) \subset \C(X, Y)$ ein Teilraum.

        \item
        $j\colon Y \rightarrow \C(X, Y)$, $j(y) = \const_X^y$ ist eine
        Einbettung.

        \item
        $\C(X, Y)$ ist hausdorffsch genau dann, wenn $Y$ hausdorffsch ist.
    \end{enumerate}
\end{Satz}

\linie
\pagebreak

\begin{Satz}{Stetigkeit der Komposition}
    \begin{enumerate}
        \item
        Für jede Abbildung $g \in \C(Y, Z)$ ist
        $g_\ast\colon \C(X, Y) \rightarrow \C(X, Z)$, $f \mapsto g \circ f$
        stetig.

        \item
        Für jede Abbildung $f \in \C(X, Y)$ ist
        $f^\ast\colon \C(Y, Z) \rightarrow \C(X, Z)$, $g \mapsto g \circ f$
        stetig.

        \item
        Ist $Y$ lokal-kompakt, so ist
        $\circ\colon \C(Y, Z) \times \C(X, Y) \rightarrow \C(X, Z)$,
        $(g, f) \mapsto g \circ f$ stetig \\
        (in $g$ und $f$ gleichzeitig).
    \end{enumerate}
\end{Satz}

\begin{Kor}
    Für jedes $y \in Y$ ist die Auswertung
    $e_y\colon \C(Y, Z) \rightarrow Z$, $g \mapsto g(y)$ stetig. \\
    Ist $Y$ lokal-kompakt, so ist die Auswertung
    $e\colon \C(Y, Z) \times Y \rightarrow Z$, $(g, y) \mapsto g(y)$ stetig.
\end{Kor}

\begin{Bem}
    \emph{Lokale Kompaktheit ist wesentlich!} \\
    Gegenbeispiel: Die Auswertung
    $e\colon \C(\rational, [0, 1]) \times \rational$, $(f, x) \mapsto f(x)$
    ist nicht stetig.
\end{Bem}

\linie

\begin{Def}{Adjunktion}
    Seien $X$, $Y$ und $Z$ Mengen.
    Zu einer Abbildung $f\colon X \times Y \rightarrow Z$ und $x \in X$
    ist $f(x, -)\colon Y \rightarrow Z$, $y \mapsto f(x, y)$ die
    Einschränkung $Y \xrightarrow{\sim} \{x\} \times Y \xrightarrow{f} Z$. \\
    Dies definiert $\adj{f}\colon X \rightarrow Z^Y = \Abb(Y, Z)$ mit
    $(\adj{f}(x))(y) := f(x, y)$. \\
    Die Abbildungen $f$ und $\adj{f}$ heißen \begriff{zueinander adjungiert}
    und $Z^{X \times Y} \rightarrow (Z^Y)^X$, $f \mapsto \adj{f}$
    heißt \begriff{Adjunktion}.
    Dies ist eine Bijektion.
\end{Def}

\begin{Satz}{Adjunktion in topologischen Räumen}
    Seien $X$, $Y$ und $Z$ topologische Räume. \\
    Ist $f\colon X \times Y \rightarrow Z$ stetig, dann auch
    $\widetilde{f}\colon X \rightarrow \C(Y, Z)$. \\
    Ist $Y$ lokal-kompakt und $\widetilde{f}\colon X \rightarrow \C(Y, Z)$,
    dann auch $f\colon X \times Y \rightarrow Z$.
\end{Satz}

\begin{Kor}
    Ist $X$ lokal-kompakt, dann ist die KO-Topologie auf $\C(X, Y)$ die
    gröbste Topologie, sodass die Auswertung $\C(X, Y) \times X \rightarrow Y$,
    $(f, x) \mapsto f(x)$ stetig ist.
\end{Kor}

\linie

\begin{Bem}
    Seien $X$, $Y$ und $Z$ Mengen. \\
    Dann ist $Z^{X + Y} \xrightarrow{\sim} Z^X \times Z^Y$,
    $f \mapsto (f|_X = f \circ \iota_X, f|_Y = f \circ \iota_Y)$
    eine Bijektion. \\
    Für topologische Räume induziert dies eine Bijektion
    $\C(X \sqcup Y, Z) \xrightarrow{\sim} \C(X, Z) \times \C(Y, Z)$ (1). \\
    Analog induziert die Bijektion
    $(Y \times Z)^X \xrightarrow{\sim} Y^X \times Z^X$,
    $f \mapsto (p_Y \circ f, p_Z \circ f)$ die Bijektion
    $\C(X, Y \times Z) \xrightarrow{\sim} \C(X, Y) \times \C(X, Z)$ (2). \\
    Analog induziert die Bijektion
    $Z^{X \times Y} \xrightarrow{\sim} (Z^Y)^X$,
    $f \mapsto \adj{f}$ die Bijektion \\
    $\C(X \times Y, Z) \xrightarrow{\sim} \C(X, \C(Y, Z))$ (3), wenn
    man voraussetzt, das $Y$ lokal-kompakt ist.
\end{Bem}

\begin{Satz}{Homöomorphismen bzgl. der KO-Topologie}
    Die natürlichen Bijektionen (1), (2) und (3) sind Homöomorphismen
    bzgl. der KO-Topologie.
    Hierbei setzt man für (2) $X$ als hausdorffsch und
    für (3) $X$ als hausdorffsch und $Y$ als lokal-kompakt und hausdorffsch
    voraus.
\end{Satz}

\linie

\begin{Def}{Erzeugendensystem der Kompakta}
    Sei $X$ ein topologischer Raum.
    Eine Familie $\A$ kompakter Mengen in $X$ heißt
    \begriff{Erzeugendensystem der Kompakta} in $X$, falls
    zu $K \subset X$ kompakt und $U \supset K$ of"|fen
    $A_1, \dotsc, A_n \in \A$ existieren mit
    $K \subset A_1 \cup \dotsb \cup A_n \subset U$.
\end{Def}

\begin{Bsp}
    Für Hausdorff-Räume $Y$ und $Z$ ist die Familie
    $\A := \{A \times B \;|\; A \subset Y \text{ kompakt},\;$ \\
    $B \subset Z \text{ kompakt}\}$ ein Erzeugendensystem der Kompakta in
    $Y \times Z$.
\end{Bsp}

\begin{Lemma}{kompakte Hausdorff-Räume}
    Sei $X$ ein kompakter Hausdorff-Raum.
    Dann existieren zu jeder of"|fenen Überdeckung
    $X = U_1 \cup \dotsb \cup U_n$ kompakte Teilmengen $K_1 \subset U_1$,
    \dots, $K_n \subset U_n$ mit
    $X = \inneres{K_1} \cup \dotsb \cup \inneres{K_n}$.
\end{Lemma}

\begin{Satz}{Erzeugendensystem der KO-Topologie}
    Seien $X$ und $Y$ topologische Räume mit $X$ hausdorffsch,
    $\A$ ein Erzeugendensystem der Kompakta in $X$ und
    $\B$ ein Erzeugendensystem der Topologie auf $Y$.
    Dann wird die KO-Topologie auf $\C(X, Y)$ erzeugt von
    $\{[A, B] \;|\; A \in \A, B \in \B\}$.
\end{Satz}

\begin{Kor}
    Sei $X$ ein lokal-kompakter Hausdorff-Raum.
    Erlauben die Topologien auf $X$ und $Y$ abzählbare Basen, dann auch
    die KO-Topologie auf $\C(X, Y)$.
\end{Kor}

\pagebreak

\section{%
    Trennung%
}

\subsection{%
    Trennung durch of"|fene Mengen%
}

\begin{Def}{Trennungsaxiome}
    Sei $X$ ein topologischer Raum. \\
    Die folgenden Eigenschaften heißen \begriff{Trennungsaxiome}:
    \begin{enumerate}[label=\textbf{T\arabic*}:]
        \item
        Zu je zwei Punkten $a \not= b$ in $X$ existieren \\
        of"|fene Umgebungen
        $U \in \U_a$, $V \in \U_b$ mit $b \notin U$, $a \notin V$.

        \item
        Zu je zwei Punkten $a \not= b$ in $X$ existieren \\
        disjunkte of"|fene Umgebungen $U \in \U_a$, $V \in \U_b$.

        \item
        Zu $A \subset X$ abgeschlossen und $b \in X \setminus A$ existieren \\
        disjunkte of"|fene Umgebungen $U \in \U_A$, $V \in \U_b$.

        \item
        Zu $A, B \subset X$ abgeschlossen und disjunkt existieren \\
        disjunkte of"|fene Umgebungen $U \in \U_A$, $V \in \U_B$.
    \end{enumerate}
\end{Def}

\begin{Bsp}
    $\real^n$ erfüllt T1, T2, T3 und T4. \\
    Allgemein erfüllt jeder metrische Raum T1, T2, T3 und T4
    (Folgerung s.\,u.).
    Es gilt:
    \displaymathother
    \begin{align*}
        \begin{xy}
            \xymatrix@C=2cm{
                *+++[o][F]\txt{T1}
                \ar @<-1ex> @{=>} [r]_{X \text{ endlich}} &
                *+++[o][F]\txt{T2}
                \ar @<-1ex> @{=>} [l]
                \ar @<-1ex> @{=>} [r]_{X \text{ kompakt}} &
                *+++[o][F]\txt{T3}
                \ar @<-1ex> @{=>} [l]_{\text{T1}}
                \ar @<-1ex> @{=>} [r]_{X \text{ kompakt}} &
                *+++[o][F]\txt{T4}
                \ar @<-1ex> @{=>} [l]_{\text{T1}}
            }
        \end{xy}
    \end{align*}
    \displaymathnormal
\end{Bsp}

\begin{Def}{hausdorffsch, regulär, normal}
    Ein topologischer Raum $X$ heißt \begriff{hausdorffsch}, falls T2 gilt,
    \begriff{regulär}, falls T1 und T3 gelten, und
    \begriff{normal}, falls T1 und T4 gelten.
\end{Def}

\begin{Bem}
    T1 ist erfüllt genau dann, wenn $\{x\}$ für jeden Punkt $x \in X$
    abgeschlossen ist.
\end{Bem}

\subsection{%
    Trennung durch stetige Funktionen%
}

\begin{Bem}
    Gegeben seien zwei Teilmengen $A, B \subset X$ eines
    topologischen Raums $X$. \\
    Unter welchen Umständen existiert eine stetige Funktion
    $f\colon X \rightarrow \real$ mit $f|_A = 1$ und $f|_B = 0$? \\
    Es muss zunächst einmal $A \cap B = \emptyset$ gelten
    (sonst wäre $f$ nicht wohldefiniert). \\
    Außerdem ist die Fragestellung nicht-trivial, da z.\,B. für
    $A := \rational$, $B := \real \setminus \rational$ in $\real$
    keine stetige Funktion $f\colon \real \rightarrow \real$
    mit $f|_A = 1$ und $f|_B = 0$ existiert.
\end{Bem}

\begin{Def}{Abstand zwischen Mengen}
    Sei $(X, d)$ ein metrischer Raum.
    Für $A \subset X$ und $x \in X$ ist $d(x, A) := \inf_{a \in A} d(x, a)$
    der \begriff{Abstand von $x$ zu $A$}.
\end{Def}

\begin{Lemma}{Abstand zwischen Mengen stetig}
    Für $A \not= \emptyset$ ist $d_A\colon X \rightarrow \real$,
    $d_A(x) := d(x, A)$ stetig. \\
    Für $A \subset X$ abgeschlossen gilt $d_A(x) = 0$ genau dann, wenn
    $x \in A$.
\end{Lemma}

\begin{Bem}
    Für $A = \emptyset$ ist $d_\emptyset(x) = +\infty$ für alle $x \in X$.
    Auf die Abgeschlossenheit kann man nicht verzichten:
    z.\,B. $A := \rational$ in $X := \real$, es gilt aber $d_\rational(x) = 0$
    für alle $x \in \real$.
\end{Bem}

\begin{Kor}
    Jeder metrische Raum $(X, d)$ erfüllt T1, T2, T3 und T4.
\end{Kor}

\begin{Bem}
    Im Beweis der Folgerung sieht man:
    $g\colon X \rightarrow [-1, +1]$ mit
    $g(x) := \frac{d_A(x) - d_B(x)}{d_A(x) + d_B(x)}$ ist für $A, B \subset X$
    abgeschlossen mit $A \cap B = \emptyset$ wohldefiniert und stetig. \\
    Die Abbildung $f := \frac{1}{2}(1 - g)\colon X \rightarrow [0, 1]$ ist
    stetig und es gilt $f|_A = 1$, $f|_B = 0$.
\end{Bem}

\begin{Satz}{\name{Urysohn}}
    Sei $X$ ein T4-Raum.
    Dann existiert zu $A, B \subset X$ abgeschlossen und disjunkt eine
    stetige Funktion $f\colon X \rightarrow [0, 1]$ mit $f|_A = 1$ und
    $f|_B = 0$.
\end{Satz}

\linie

\begin{Bem}
    Gegeben sei $f\colon A \rightarrow Y$ auf $A \subset X$. \\
    Unter welchen Umständen existiert eine stetige Funktion
    $F\colon X \rightarrow Y$ mit $F|_A = f$? \\
    Die Fragestellung ist nicht-trivial, da z.\,B.
    für $A := \rational$ die Funktion $f\colon \rational \rightarrow \real$,
    $f(x) = 0$ für $x^2 > 2$ und $f(x) = 1$ für $x^2 \le 2$ keine stetige
    Fortsetzung $F\colon \real \rightarrow \real$ auf $\real$ erlaubt.
\end{Bem}

\begin{Lemma}{Näherungsfortsetzung}
    Seien $A \subset X$ abgeschlossen und $\varphi\colon A \rightarrow [-s, s]$
    stetig.
    Dann existiert eine \begriff{Näherungsfortsetzung}
    $\Phi\colon X \rightarrow \left[-\frac{s}{3}, \frac{s}{3}\right]$ mit
    $|\varphi(a) - \Phi(a)| \le \frac{2}{3}s$ für alle $a \in A$.
\end{Lemma}

\begin{Satz}{\name{Tietze}}
    Seien $X$ ein T4-Raum und $A \subset X$ abgeschlossen.
    Zu jeder stetigen Funktion $f\colon A \rightarrow [a, b]$ existiert
    eine stetige Fortsetzung $F\colon X \rightarrow [a, b]$ mit $F|_A = f$.
\end{Satz}

\begin{Kor}
    Seien $X$ ein T4-Raum und $A \subset X$ abgeschlossen.
    Zu jeder stetigen Funktion $f\colon A \rightarrow \real$ existiert
    eine stetige Fortsetzung $F\colon X \rightarrow \real$ mit $F|_A = f$
    (auch für $\real^n$).
\end{Kor}

\linie

\begin{Bem}
    Aus $A, B \subset \real^n$ mit $A \cong B$ folgt im Allgemeinen nicht,
    dass $\real^n \setminus A \cong \real^n \setminus B$.
    Für $n = 1$ ist ein Beispiel $A := \real \setminus \{0\}$ und
    $B := \real \setminus [-1, 1]$.
    Allerdings gilt die obige Aussage auch für $A, B$ abgeschlossen:
    Für $n = 2$, $A := \sphere^1 \sqcup 2\sphere^1$ und
    $B := 2\sphere^1 \sqcup (\sphere^1 + 4)$ gilt
    $A \cong B$, aber $\real^2 \setminus A \not\cong \real^2 \setminus B$
    (Beweis später).
    Durch Hinzufügung zusätzlicher Freiheitsgrade (Stabilisierung)
    können die Komplemente durch den zusätzlichen Platz homöomorph werden:
    Für $A' := A \times \{0\} \subset \real^3$ und
    $B' := B \times \{0\} \subset \real^3$ gilt $A' \cong B'$ und
    $\real^3 \setminus A' \cong \real^3 \setminus B'$.
    Es gibt sogar einen Homöomorphismus $h\colon \real^3 \homoe \real^3$
    mit $h(A') = B'$, daraus folgt dann
    $\real^3 \setminus A' \xrightarrow[h|_{\real^3 \setminus A'}]{\cong}
    \real^3 \setminus B'$.
\end{Bem}

\begin{Satz}{Komplemente nach Stabilisierung homöomorph}
    Seien $A \subset \real^m$ und $B \subset \real^n$ abgeschlossene
    homöomorphe Teilmengen sowie
    $A' := A \times \{0\} \subset \real^m \times \real^n$,
    $B' := \{0\} \times B \subset \real^m \times \real^n$. \\
    Dann sind $\real^{m+n} \setminus A'$ und $\real^{m+n} \setminus B'$
    homöomorph.
\end{Satz}

\subsection{%
    Parakompaktheit%
}

\begin{Def}{Träger}
    Zu $f\colon X \rightarrow \real$ heißt
    $\supp(f) := \abschluss{\{x \in X \;|\; f(x) \not= 0\}}$ der
    \begriff{Träger} von $f$. \\
    $\C_C(X, \real)$ sei die Menge der stetigen Funktionen von $X$ nach $\real$
    mit kompaktem Träger.
\end{Def}

\begin{Bem}
    Das Integral definiert eine lineare Abbildung
    $\int_{\real^n}\colon \C_C(\real^n, R) \rightarrow \real$. \\
    Man will nun auch stetige Funktionen
    $f\colon \sphere^n \rightarrow \real$ integrieren. \\
    Zur Definition eines Integrals
    $\int_{\sphere^n}\colon \C(\sphere^n, \real) \rightarrow \real$
    bedient man sich den Parametrisierungen
    $\varphi_\pm\colon \real^n \homoe U_\pm :=
    \sphere^n \setminus \{\pm e_1\}$
    mit $\varphi_\pm$ der Umkehrung der stereographischen Projektion
    (bzgl. Nord-/Südpol).
    Ist nun $f\colon \sphere^n \rightarrow \real$ mit $\supp(f) \subset U_+$
    gegeben, so kann man auf $U_+$ integrieren, d.\,h.
    $\int_{U_+}\colon \C_C(U_+, \real) \rightarrow \real$,
    $\int_{U_+} f := \int_{\real^n} f(\varphi_+(x)) \vol_+(x) \dx$.
    Ebenso ist für $\supp(f) \subset U_-$ das Integral
    $\int_{U_-}\colon \C_C(U_-, \real) \rightarrow \real$,
    $\int_{U_-} f := \int_{\real^n} f(\varphi_-(x)) \vol_-(x) \dx$
    definiert.
    Die Funktionen $\vol_\pm$ sorgen für den Ausgleich der von
    $\varphi_\pm$ erzeugten Verzerrung. \\
    Was ist aber zu tun, wenn für $f\colon \sphere^n \rightarrow \real$
    weder $\supp(f) \subset U_+$ noch $\supp(f) \subset U_-$ gilt?
    Die Funktionen $f \circ \varphi_\pm\colon \real^n \rightarrow \real$
    hat dann keinen kompakten Träger mehr, liegt also nicht in
    $\C_C(\real^n, \real)$ und lässt sich somit auch nicht direkt
    integrieren. \\
    Hier wendet man die Methode der "`Teilung der Eins"' an: \\
    Es gibt stetige Funktionen $\tau_\pm\colon \sphere^n \rightarrow [0, 1]$
    mit $\supp(\tau_+) \subset U_+$ und $\supp(\tau_-) \subset U_-$ sowie
    $\tau_+(x) + \tau_-(x) = 1$ für alle $x \in \sphere^n$.
    Damit kann man jede stetige Funktion $f\colon \sphere \rightarrow \real$
    in die Summe $f = f_+ + f_-$ mit $f_+ := \tau_+ \cdot f$ und
    $f_- := \tau_- \cdot f$ zerlegen.
    Diese Funktionen sind wiederum stetig und erfüllen
    $\supp(f_\pm) \subset U_\pm$, d.\,h. sie können wie oben integriert werden:
    $\int_{\sphere^n}\colon \C(\sphere^n, \real) \rightarrow \real$,
    $\int_{\sphere^n} f := \int_{U_+} (\tau_+ \cdot f) +
    \int_{U_-} (\tau_- \cdot f)$. \\
    Anschließend kann man nachweisen, dass $\int_{\sphere^n}$ alle
    gewünschten Eigenschaften besitzt
    (Unabhängigkeit von der Zerlegung $\tau_\pm$ und der Parametrisierung
    $\varphi_\pm$, Monotonie, Linearität usw.).
\end{Bem}

\linie
\pagebreak

\begin{Def}{lokal-endlich}
    Eine Familie $(V_i)_{i \in I}$ von Teilmengen $V_i \subset X$ heißt
    \begriff{lokal-endlich}, falls zu jedem Punkt $x \in X$ eine of"|fene
    Umgebung $U$ existiert, sodass
    $I_U := \{i \in I \;|\; V_i \cap U \not= \emptyset\}$ endlich ist.
\end{Def}

\begin{Bsp}
    Die of"|fene Überdeckung $(\left]k - 1, k + 1\right[)_{k \in \integer}$
    von $\real$ ist lokal-endlich. \\
    Die of"|fene Überdeckung $(\left]-\infty, k\right[)_{k \in \integer}$
    von $\real$ ist nicht lokal-endlich.
\end{Bsp}

\begin{Lemma}{Abschluss lokal-endlicher Familien}
    Sei $(V_i)_{i \in I}$ lokal-endlich. \\
    Dann ist $(\abschluss{V_i})_{i \in I}$ lokal-endlich sowie
    der Abschluss der Vereinigung $V := \bigcup_{i \in I} V_i$ ist
    $\abschluss{V} = \bigcup_{i \in I} \abschluss{V_i}$.
\end{Lemma}

\begin{Bem}
    Auf lokale Endlichkeit kann hier nicht verzichtet werden: \\
    Zum Beispiel gilt für $V = \rational$, dass
    $\rational = \bigcup_{x \in \rational} \{x\}$, aber
    $\real = \abschluss{\rational} \supsetneqq \bigcup_{x \in \rational}
    \abschluss{\{x\}} = \rational$.
\end{Bem}

\begin{Lemma}{Summe einer Familie von Funktionen mit lokal-endlichem Träger}\\
    Sei $(f_i)_{i \in I}$ eine Familie stetiger Funktionen
    $f_i\colon X \rightarrow \real$ mit $(\supp(f_i))_{i \in I}$
    lokal-endlich. \\
    Dann ist $f\colon X \rightarrow \real$, $f(x) := \sum_{i \in I} f_i(x)$
    wohldefiniert und stetig.
\end{Lemma}

\begin{Def}{Teilung der Eins}
    Eine Familie $(f_i)_{i \in I}$ stetiger Funktionen
    $f_i\colon X \rightarrow [0, 1]$ heißt
    \begriff{Zerlegung/\hspace{0mm}Teilung der Eins},
    falls $(\supp(f_i))_{i \in I}$ lokal-endlich ist und
    $\sum_{i \in I} f_i(x) = 1$ für alle $x \in X$. \\
    Eine Teilung der Eins $(f_i)_{i \in I}$ heißt einer of"|fenen Überdeckung
    $(U_i)_{i \in I}$ von $X$ \begriff{untergeordnet}, falls
    $\supp(f_i) \subset U_i$ für alle $i \in I$.
\end{Def}

\linie

\begin{Def}{Verfeinerung}
    Sei $(U_i)_{i \in I}$ eine (of"|fene) Überdeckung von $X$.
    Eine \begriff{(of"|fene) Verfeinerung} von $(U_i)_{i \in I}$ ist eine
    (of"|fene) Überdeckung $(V_j)_{j \in J}$ von $X$, sodass jedes $V_j$
    in einem $U_i$ enthalten ist, d.\,h.
    $\exists_{\nu\colon J \rightarrow I} \forall_{j \in J}\;
    V_j \subset U_{\nu(j)}$.
\end{Def}

\begin{Def}{parakompakt}
    Ein topologischer Raum $X$ heißt \begriff{parakompakt}, falls zu jeder
    of"|fenen Überdeckung $(U_i)_{i \in I}$ eine lokal-endliche of"|fene
    Verfeinerung $(V_j)_{j \in J}$ existiert.
\end{Def}

\begin{Bem}
    Man kann stets $J = I$, $V_i \subset U_i$ für alle $i \in I$ annehmen
    ($\nu = \id)$. \\
    Aus Kompaktheit oder $\sigma$-Kompaktheit folgt Parakompaktheit.
\end{Bem}

\begin{Satz}{\name{Stone}}
    Jede metrische Raum ist parakompakt.
\end{Satz}

\linie

\begin{Lemma}{Trennungsaxiome mit parakompakt}
    Aus T2 folgen mit Parakompaktheit T3 und T4.
\end{Lemma}

\linie

\begin{Lemma}{Existenz einer Schrumpfung}
    Sei $X$ ein parakompakter Hausdorff-Raum. \\
    Dann existiert zu jeder of"|fenen Überdeckung $(U_i)_{i \in I}$ eine
    of"|fene Überdeckung $(V_i)_{i \in I}$ mit $\abschluss{V_i} \subset U_i$
    für alle $i \in I$ (\begriff{Schrumpfung}).
\end{Lemma}

\begin{Satz}{Existenz einer Zerlegung der Eins}
    Ein Hausdorff-Raum ist parakompakt genau dann, wenn jede of"|fene
    Überdeckung $(U_i)_{i \in I}$ eine untergeordnete Zerlegung der Eins
    erlaubt.
\end{Satz}

\pagebreak

\section{%
    Zusammenhang%
}

\subsection{%
    Zusammenhängende topologische Räume%
}

\begin{Bem}
    In jedem topologischen Raum $X$ sind $\emptyset$ und $X$ sowohl of"|fen
    als auch abgeschlossen.
    Allgemein sind für $A \subset X$ und $B := X \setminus A \subset X$
    äquivalent:
    \begin{enumerate}
        \item
        $A \subset X$ ist of"|fen und abgeschlossen.

        \item
        $B \subset X$ ist of"|fen und abgeschlossen.

        \item
        $X = A \sqcup B$ zerfällt in zwei disjunkte of"|fene Mengen.

        \item
        $X = A \sqcup B$ zerfällt in zwei disjunkte abgeschlossene Mengen.
    \end{enumerate}
\end{Bem}

\begin{Def}{zusammenhängend}
    Ein topologischer Raum $X$ heißt
    \begriff{zusammenhängend}, falls für jede Zerlegung
    $X = A \sqcup B$ in disjunkte of"|fene Teilmengen $A, B \subset X$ gilt,
    dass $A = \emptyset$ oder $B = \emptyset$. \\
    $X$ heißt \begriff{unzusammenhängend}, falls es eine Zerlegung
    $X = A \sqcup B$ in disjunkte of"|fene Teilmengen $A, B \subset X$ gibt
    mit $A, B \not= \emptyset$.
\end{Def}

\begin{Bsp}
    $\rational$ ist nicht zusammenhängend, denn es gilt
    $\rational = A \sqcup B$ mit $A := \{x \in \rational \;|\; x^2 < 2\}$
    und $B := \{x \in \rational \;|\; x^2 > 2\}$ of"|fen und disjunkt.
    $\real$ ist zusammenhängend (nach folgendem Satz).
\end{Bsp}

\linie

\begin{Def}{Intervall}
    $I \subset \real$ heißt \begriff{Intervall}, falls
    für $a, b \in I$ und $a < x < b$ auch $x \in I$ gilt.
\end{Def}

\begin{Satz}{Intervalle sind zush.}
    Jedes Intervall $I \subset \real$ ist zusammenhängend.
\end{Satz}

\begin{Satz}{Umkehrung}
    Jede zusammenhängende Menge $I \subset \real$ ist ein Intervall.
\end{Satz}

\linie

\begin{Satz}{stetiges Bild eines zush. Raums ist zush.}\\
    Ist $f\colon X \rightarrow Y$ stetig und $X$ zusammenhängend,
    dann ist auch $f(X)$ zusammenhängend.
\end{Satz}

\begin{Satz}{äquivalente Formulierungen}
    Für jeden topologischen Raum $X$ sind äquivalent:
    \begin{enumerate}
        \item
        $X$ ist zusammenhängend.

        \item
        Jede stetige Abbildung $f\colon X \rightarrow Y$ mit $Y$ diskret
        ist konstant.

        \item
        Jede stetige Abbildung $f\colon X \rightarrow \{0, 1\}$
        ist konstant.

        \item
        Jede stetige Funktion $f\colon X \rightarrow \real$ hat die
        Zwischenwerteigenschaft, d.\,h.
        für alle $a, b \in X$, $y \in \real$ mit $f(a) \le y \le f(b)$ gibt
        es ein $x \in X$ mit $f(x) = y$.
    \end{enumerate}
\end{Satz}

\linie

\begin{Lemma}{Vereinigung zush.}
    Sind für $i \in I$ die Teilmengen $A_i \in X$ zusammenhängend und
    paarweise nicht-disjunkt
    (d.\,h. $A_i \cap A_j \not= \emptyset$ für alle $i, j \in I$),
    dann ist auch $A := \bigcup_{i \in I} A_i$ zusammenhängend.
\end{Lemma}

\begin{Lemma}{größere Menge zush.}\\
    Ist $A \subset B \subset \abschluss{A}$ mit $A$ zusammenhängend,
    dann ist auch $B$ zusammenhängend.
\end{Lemma}

\begin{Lemma}{endliches Kreuzprodukt zush.}\\
    Sind $X_1, \dotsc, X_n$ zusammenhängend, dann ist auch
    $X_1 \times \dotsb \times X_n$ zusammenhängend.
\end{Lemma}

\begin{Satz}{beliebiges Produkt zush. $\;\Leftrightarrow\;$
             alle Räume zush.} \\
    Sei $(X_i)_{i \in I}$ eine Familie nicht-leerer topologischer Räume.
    Der Produktraum $X := \prod_{i \in I} X_i$ ist zusammenhängend
    genau dann, wenn $X_i$ für alle $i \in I$ zusammenhängend ist.
\end{Satz}

\linie
\pagebreak

\begin{Def}{Zusammenhangskomponente}
    Sei $X$ ein topologischer Raum. \\
    Für $x \in X$ ist die \begriff{Zusammenhangskomponente}
    $\zhk(x)$ die Vereinigung aller zusammenhängenden Teilmengen $A \subset X$
    mit $x \in A$.
\end{Def}

\begin{Satz}{Eigenschaften der ZHK}
    \begin{enumerate}
        \item
        $\zhk(x)$ ist die größte zusammenhängende Menge, die $x$ enthält.

        \item
        $\zhk(x)$ ist abgeschlossen in $X$.

        \item
        Die Familie $\zhk(X) := \{\zhk(x) \;|\; x \in X\}$ ist eine
        Zerlegung von $X$.
    \end{enumerate}
\end{Satz}

\begin{Bsp}
    $X$ ist zusammenhängend genau dann, wenn $\zhk(x) = X$ für alle
    $x \in X$ gilt. \\
    Die Zusammenhangskomponenten von $\real \setminus \{0\}$ sind
    $\zhk(\real \setminus \{0\}) = \{\real_{<0}, \real_{>0}\}$. \\
    In $\rational$ gilt $\zhk(x) = \{x\}$ für alle $x \in \rational$
    (solche Räume heißen \begriff{total unzusammenhängend}).
\end{Bsp}

\begin{Satz}{ZHKs von stetigen Abbildungen}
    \begin{enumerate}
        \item
        Ist $f\colon X \rightarrow Y$ stetig, dann gilt
        $f(\zhk(x)) \subset \zhk(f(x))$ für alle $x \in X$.

        \item
        Man erhält so die Abbildung
        $\zhk(f)\colon \zhk(X) \rightarrow \zhk(Y)$,
        $\zhk(x) \mapsto \zhk(f(x))$.

        \item
        Es gilt $\zhk(\id_X) = \id_{\zhk(X)}$ und
        $\zhk(g \circ f) = \zhk(g) \circ \zhk(f)$ für
        $f\colon X \rightarrow Y$, $g\colon Y \rightarrow Z$ stetig.

        \item
        Jeder Homöomorphismus $f\colon X \homoe Y$ induziert eine Bijektion
        $\zhk(f)\colon \zhk(X) \bij \zhk(Y)$.
    \end{enumerate}
\end{Satz}

\subsection{%
    Wegzusammenhang%
}

\begin{Def}{Weg}
    Sei $X$ ein topologischer Raum.
    Eine stetige Abbildung $\gamma\colon [0, 1] \rightarrow X$ heißt
    \begriff{Weg} in $X$. \\
    Dabei heißt $a = \gamma(0)$ der \begriff{Anfangspunkt} und
    $b = \gamma(1)$ der \begriff{Endpunkt} von $\gamma$. \\
    Man sagt, $\gamma$ verbindet $a$ und $b$ in $X$
    oder $\gamma$ ist ein Weg in $X$ von $a$ nach $b$. \\
    $PX := \C([0,1], X)$ ist die Menge aller Wege in $X$. \\
    $PX(a, b) := \{\gamma \in PX \;|\; \gamma(0) = a,\; \gamma(1) = b\}$
    ist die Menge aller Wege von $a$ nach $b$. \\
    $X$ kann durch $X \hookrightarrow PX$,
    $x \mapsto \const_{[0, 1]}^x \in PX(x, x)$ in $PX$ eingebettet werden. \\
    Läuft ein Weg $\gamma\colon [0, 1] \rightarrow X$ von $a$ nach $b$,
    dann läuft der \begriff{inverse Weg} \\
    $\overline{\gamma}\colon [0, 1] \rightarrow X$,
    $\overline{\gamma}(t) := \gamma(1 - t)$ von $b$ nach $a$.
    Dies definiert $-\colon PX(a, b) \rightarrow PX(b, a)$. \\
    Laufen $\gamma_1$ von $a$ nach $b$ und $\gamma_2$ von $b$ nach $c$, dann
    läuft $\gamma_1 \ast \gamma_2$ von $a$ nach $c$, wobei \\
    $\gamma_1 \ast \gamma_2\colon [0, 1] \rightarrow X$,
    $(\gamma_1 \ast \gamma_2)(t) := $\matrixsize{$\;\begin{cases}
        \gamma_1(2t) & 0 \le t \le 1/2 \\
        \gamma_2(2t - 1) & 1/2 < t \le 1
    \end{cases}$}. \\
    Dies definiert $\ast\colon PX(a, b) \times PX(b, c) \rightarrow PX(a, c)$.
\end{Def}

\linie

\begin{Def}{verbindbar, wegzusammenhängend}\\
    Zwei Punkte $a, b \in X$ heißen \begriff{verbindbar} in $X$, falls
    $PX(a, b) \not= \emptyset$. \\
    Der Raum $X$ heißt \begriff{wegzusammenhängend}, falls je zwei Punkte in
    $X$ verbindbar sind.
\end{Def}

\begin{Bsp}
    Jedes Intervall in $\real$ in $\real$ ist wegzusammenhängend. \\
    $\real \setminus \{0\}$ ist nicht wegzusammenhängend.
\end{Bsp}

\begin{Satz}{wegzush. $\Rightarrow$ zush.}
    Jeder wegzusammenhängende Raum ist zusammenhängend.
\end{Satz}

\begin{Bem}
    Die Umkehrung gilt nicht! \\
    Ein Gegenbeispiel ist $C := A \cup B$ mit
    $A := \{(x, \sin(\frac{\pi}{x}) \;|\; x \in \left]0, 1\right]\}$ und
    $B := \{0\} \times [-1, +1]$. \\
    $C$ ist zusammenhängend, aber nicht wegzusammenhängend
    (\begriff{topologische Sinuskurve}).
\end{Bem}

\begin{Bsp}
    $\real \setminus \{x\}$ ist nicht wegzusammenhängend,
    aber $\real^2 \setminus \{x\}$ ist wegzusammenhängend. \\
    $\real^2 \setminus \integer^2$ und $\real^2 \setminus \rational^2$
    sind wegzusammenhängend. \\
    Allgemein: Für $A \subset \real^n$ abzählbar und $n \ge 2$ ist
    $\real^n \setminus A$ wegzusammenhängend.
\end{Bsp}

\linie
\pagebreak

\begin{Satz}{stetiges Bild eines wegzush. Raums ist wegzush.}
    Ist $f\colon X \rightarrow Y$ stetig und $X$ wegzusammenhängend,
    dann ist auch $f(X)$ wegzusammenhängend.
\end{Satz}

\begin{Satz}{beliebiges Produkt wegzush. $\;\Leftrightarrow\;$
             alle Räume wegzush.} \\
    Sei $(X_i)_{i \in I}$ eine Familie nicht-leerer topologischer Räume.
    Der Produktraum $X := \prod_{i \in I} X_i$ ist wegzusammenhängend
    genau dann, wenn $X_i$ für alle $i \in I$ wegzusammenhängend ist.
\end{Satz}

\linie

\begin{Satz}{Verbindbarkeit als Äquivalenzrelation}
    Verbindbarkeit ist eine Äquivalenzrelation.
\end{Satz}

\begin{Lemma}{Vereinigung zush.}
    Sind für $i \in I$ die Teilmengen $A_i \in X$ wegzusammenhängend und
    paarweise nicht-disjunkt
    (d.\,h. $A_i \cap A_j \not= \emptyset$ für alle $i, j \in I$),
    dann ist auch $A := \bigcup_{i \in I} A_i$ wegzusammenhängend.
\end{Lemma}

\begin{Def}{Wegkomponente}\\
    Für $a \in X$ sei
    $[a] := \{b \in X \;|\; a \text{ ist mit } b \text{ in }
    X \text{ verbindbar}\}$. \\
    $[a]$ heißt \begriff{Wegkomponente}
    (oder \begriff{Wegzusammenhangskomponente}) von $a$ in $X$. \\
    $\pi_0(X) := \{[a] \;|\; a \in X\}$ ist die Menge alle Wegkomponenten
    in $X$.
\end{Def}

\begin{Bsp}
    $X$ ist wegzusammenhängend genau dann, wenn $\pi_0(X) = \{X\}$. \\
    Es gilt $\pi_0(\real \setminus \{0\}) = \{\real_{<0}, \real_{>0}\}$. \\
    Wenn $X$ diskret ist, dann gilt $\pi_0(X) = \{\{x\} \;|\; x \in X\}$. \\
    Die Umkehrung gilt nicht, bspw. ist
    $\pi_0(\rational) = \{\{x\} \;|\; x \in \rational\}$,
    aber $\rational$ ist nicht diskret. \\
    Für die topologische Sinuskurve $C = A \cup B$ gilt $\pi_0(C) = \{A, B\}$.
\end{Bsp}

\begin{Satz}{Wegkomponenten von stetigen Abbildungen}
    \begin{enumerate}
        \item
        Ist $f\colon X \rightarrow Y$ stetig, dann gilt
        $f([x]) \subset [f(x)]$ für alle $x \in X$.

        \item
        Man erhält so die Abbildung
        $\pi_0(f)\colon \pi_0(X) \rightarrow \pi_0(Y)$,
        $[x] \mapsto [f(x)]$.

        \item
        Es gilt $\pi_0(\id_X) = \id_{\pi_0(X)}$ und
        $\pi_0(g \circ f) = \pi_0(g) \circ \pi_0(f)$ für
        $f\colon X \rightarrow Y$, $g\colon Y \rightarrow Z$ stetig.

        \item
        Jeder Homöomorphismus $f\colon X \homoe Y$ induziert eine Bijektion
        $\pi_0(f)\colon \pi_0(X) \bij \pi_0(Y)$.
    \end{enumerate}
\end{Satz}

\subsection{%
    Lokaler (Weg-)Zusammenhang%
}

\begin{Def}{lokal (weg)zusammenhängend}
    Ein Raum $X$ heißt \begriff{lokal (weg)zusammenhängend in \\
    $x \in X$}, falls
    jede Umgebung von $x$ eine (weg)zusammenhängende Umgebung von $x$
    enthält. \\
    $X$ heißt \begriff{lokal (weg)zusammenhängend}, falls $X$ für alle
    $x \in X$ (weg)zusammenhängend in $x$ ist.
\end{Def}

\begin{Bsp}
    Jede of"|fene Menge $X \subset \real^n$ ist lokal (weg)zusammenhängend. \\
    $[0, 1] \cup [2, 3] \subset \real$ ist lokal (weg)zusammenhängend,
    aber nicht (weg)zusammenhängend. \\
    Die topologische Sinuskurve $C$ ist zusammenhängend,
    aber nicht lokal zusammenhängend. \\
    Der \begriff{topologische Kamm} $X := ([0, 1] \times \{0\}) \cup
    ((\rational \cap [0, 1]) \times [0, 1]) \subset \real^2$
    ist wegzusammenhängend, aber nicht lokal wegzusammenhängend.
\end{Bsp}

\pagebreak

\subsection{%
    Kategorien%
}

\begin{Def}{Kategorie}
    Eine \begriff{Kategorie} $\cat{C} = (\Ob, \Mor, \circ)$ ist ein Tripel
    bestehend aus
    \begin{enumerate}
        \item
        einer Klasse $\Ob = \Ob(\cat{C})$ von \begriff{Objekten},

        \item
        zu je zwei Objekten $A, B \in \Ob$ einer Menge $\Mor(A, B)$ von
        \begriff{Morphismen} sowie

        \item
        zu je drei Objekten $A, B, C \in \Ob$ einer \begriff{Verknüpfung} \\
        $\circ\colon \Mor(B, C) \times \Mor(A, B) \rightarrow \Mor(A, C)$,
        $(g, f) \mapsto g \circ f$, sodass
        \begin{enumerate}
            \item
            $h \circ (g \circ f) = (h \circ g) \circ f$
            für alle $A, B, C, D \in \Ob$ mit $f \in \Mor(A, B)$,
            $g \in \Mor(B, C)$ und $h \in \Mor(C, D)$ (Assoziativität) und

            \item
            zu jedem $B \in \Ob$ ein Morphismus $\id_B \in \Mor(B, B)$
            existiert, sodass $\id_B \circ f = f$ und $g \circ \id_B = g$
            für alle $A, C \in \Ob$ mit
            $f \in \Mor(A, B)$, $g \in \Mor(B, C)$.
        \end{enumerate}
    \end{enumerate}
\end{Def}

\begin{Bem}
    Für jedes Objekt $B \in \Ob$ ist $\id_B$ eindeutig bestimmt:
    Erfüllen $\id_B$ und $\id_B'$ die Bedingung b), so gilt
    $\id_B = \id_B \circ \id_B' = \id_B'$.
    $\id_B$ heißt auch \begriff{Identität von $B$}.
\end{Bem}

\begin{Bsp}
    Beispiele für Kategorien sind
    $\cat{Top} := (\text{top. Räume}, \text{stetige Abb.},
    \text{übl. Verkn.})$, \\
    $\cat{Set} := (\text{Mengen}, \text{Abbildungen},
    \text{übl. Verkn.})$, \\
    $K\text{-}\cat{Vec} := (K\text{-Vektorräume}, K\text{-lineare Abb.},
    \text{übl. Verkn.})$, \\
    $(\natural, m \times n\text{-Matrizen}, \text{Matrixmult.})$ und
    $(X, \le, \text{Transitivität})$ \\
    (dabei ist $(X, \le)$ eine geordnete Menge).
\end{Bsp}

\linie

\begin{Def}{kommutatives Diagramm}\\
    Einen Morphismus $f \in \Mor(A, B)$ schreibt man kurz als Pfeil
    $f\colon A \rightarrow B$ oder $A \xrightarrow{f} B$. \\
    Die Komposition von $A \xrightarrow{f} B$ und $B \xrightarrow{g} C$
    schreibt man dann als kommutatives Diagramm:
    \displaymathother
    \begin{align*}
        \begin{xy}
            \xymatrix{
                & B \ar[rd]^g \\
                A \ar[ru]^f \ar[rr]_{g \circ f} & & C
            }
        \end{xy}
    \end{align*}
    \displaymathnormal
    Ein \begriff{Diagramm} in der Kategorie $\cat{C}$ ist ein Graph, dessen
    Ecken mit Objekten aus $\cat{C}$ und dessen Kanten mit passenden
    Morphismen aus $\cat{C}$ beschriftet sind.
    Ein Diagramm heißt \begriff{kommutativ}, falls zwischen je zwei Ecken des
    Diagramms die Komposition entlang aller Pfade denselben Morphismus
    in $\cat{C}$ ergibt.
\end{Def}

\begin{Bsp}
    Assoziativität ($h \circ (g \circ f) = (h \circ g) \circ f$)
    und Identität ($\id_B \circ f = f$, $g \circ \id_B = g$) lassen
    sich durch folgende kommutative Diagarmme ausdrücken:
    \displaymathother
    \begin{align*}
        \begin{xy}
            \xymatrix{
                A \ar[r]^f \ar[rd]_{g \circ f} &
                B \ar[d]_g \ar[rd]^{h \circ g} \\
                & C \ar[r]_h & D
            }
        \end{xy} \qquad
        \begin{xy}
            \xymatrix{
                A \ar[r]^f \ar[rd]_f & B \ar[d]_{\id_B} \ar[rd]^g \\
                & B \ar[r]_g & C
            }
        \end{xy}
    \end{align*}
    \displaymathnormal
\end{Bsp}

\linie

\begin{Def}{Isomorphismus}
    Seien $f\colon X \rightarrow Y$ und $g\colon Y \rightarrow X$
    Morphismen in $\cat{C}$.
    \begin{enumerate}
        \item
        Gelten $g \circ f = \id_X$ und $f \circ g = \id_Y$
        so heißen $g$ und $f$ \begriff{zueinander invers}.

        \item
        $f$ heißt \begriff{invertierbar} oder
        \begriff{Isomorphismus in $\cat{C}$}, falls es
        einen zu $f$ inversen Morphismus gibt.
        (In diesem Fall ist dieser eindeutig bestimmt und heißt $f^{-1}$.)

        \item
        Zwei Objekte $A$ und $B$ in $\cat{C}$ heißen \begriff{isomorph}
        (man schreibt $A \underset{\cat{C}}{\cong} B$ oder $A \cong B$), \\
        falls es einen Isomorphismus $f\colon A \rightarrow B$ gibt.
    \end{enumerate}
\end{Def}

\pagebreak

\subsection{%
    Funktoren%
}

\begin{Def}{kovarianter Funktor}
    Seien $\cat{C}$ und $\cat{D}$ Kategorien.
    Ein \begriff{kovarianter Funktor} $F\colon \cat{C} \rightarrow \cat{D}$
    ordnet jedem Objekt $X$ in $\cat{C}$ ein Objekt $F(X)$ in $\cat{D}$
    und jedem Morphismus $f\colon X \rightarrow Y$ in $\cat{C}$ einen
    Morphismus $F(f)\colon F(X) \rightarrow F(Y)$ in $\cat{D}$ zu,
    sodass $F(\id_X) = \id_{F(X)}$ für alle Objekte $X$ in $\cat{C}$ und
    $F(g \circ f) = F(g) \circ F(f)$ für alle Morphismen
    $f\colon X \rightarrow Y$ und $g\colon Y \rightarrow Z$ in $\cat{C}$.
\end{Def}

\begin{Bem}
    Ein kovarianter Funktor $F\colon \cat{C} \rightarrow \cat{D}$
    überführt die Identität $\id_X$ in die
    Identität $\id_{F(X)}$ und kommutative Diagramme in $\cat{C}$
    in kommutative Diagramme in $\cat{D}$:
    \displaymathother
    \begin{align*}
        \begin{xy}
            \xymatrix{
                & B \ar[rd]^g & \ar @{=>} @/^ 2mm/ [r]^F & &
                F(B) \ar[rd]^{F(g)} \\
                A \ar[ru]^f \ar[rr]_h & & C &
                F(A) \ar[ru]^{F(f)} \ar[rr]_{F(h)} & & F(C)
            }
        \end{xy}
    \end{align*}
    \displaymathnormal
\end{Bem}

\begin{Bsp}\\
    $\zhk\colon \cat{Top} \rightarrow \cat{Set}$,
    $X \mapsto \zhk(X)$,
    $(f\colon X \rightarrow Y) \mapsto
    (f_\ast\colon \zhk(X) \rightarrow \zhk(Y),
    f_\ast(\zhk(x)) = \zhk(f(x)))$ \\
    $\pi_0\colon \cat{Top} \rightarrow \cat{Set}$,
    $X \mapsto \pi_0(X)$,
    $(f\colon X \rightarrow Y) \mapsto
    (f_\ast\colon \pi_0(X) \rightarrow \pi_0(Y), f_\ast([x]) = [f(x)])$ \\
    $P_\ast\colon \cat{Set} \rightarrow \cat{Set}$,
    $X \mapsto P(X)$,
    $(f\colon X \rightarrow Y) \mapsto (f_\ast\colon P(X) \rightarrow P(Y),
    f_\ast(A) = \{f(a) \;|\; a \in A\})$
\end{Bsp}

\linie

\begin{Def}{kontravarianter Funktor}
    Seien $\cat{C}$ und $\cat{D}$ Kategorien.
    Ein \begriff{kontravarianter Funktor} $G\colon \cat{C} \rightarrow \cat{D}$
    ordnet jedem Objekt $X$ in $\cat{C}$ ein Objekt $G(X)$ in $\cat{D}$
    und jedem Morphismus $f\colon X \rightarrow Y$ in $\cat{C}$ einen
    Morphismus $G(f)\colon G(Y) \rightarrow G(X)$ in $\cat{D}$ zu,
    sodass $G(\id_X) = \id_{G(X)}$ für alle Objekte $X$ in $\cat{C}$ und
    $G(g \circ f) = G(f) \circ G(g)$ für alle Morphismen
    $f\colon X \rightarrow Y$ und $g\colon Y \rightarrow Z$ in $\cat{C}$.
\end{Def}

\begin{Bem}
    Ein kontravarianter Funktor $G\colon \cat{C} \rightarrow \cat{D}$
    überführt die Identität $\id_X$ in die
    Identität $\id_{F(X)}$ und kommutative Diagramme in $\cat{C}$
    in kommutative Diagramme in $\cat{D}$:
    \displaymathother
    \begin{align*}
        \begin{xy}
            \xymatrix{
                & B \ar[rd]^g & \ar @{=>} @/^ 2mm/ [r]^G & &
                G(B) \ar[ld]_{G(f)} \\
                A \ar[ru]^f \ar[rr]_h & & C &
                G(A) & & G(C) \ar[lu]_{G(g)} \ar[ll]^{G(h)}
            }
        \end{xy}
    \end{align*}
    \displaymathnormal
\end{Bem}

\begin{Bsp}
    $P^\ast\colon \cat{Set} \rightarrow \cat{Set}$,
    $X \mapsto P(X)$, \\
    $(f\colon X \rightarrow Y) \mapsto
    (f^\ast\colon P(Y) \rightarrow P(X),
    f^\ast(B) = \{x \in X \;|\; f(x) \in B\})$ \\
    $\Hom_K(X, -)\colon K\text{-}\cat{Vec} \rightarrow K\text{-}\cat{Vec}$,
    $V \mapsto \Hom_K(X, V)$, \\
    $(f\colon V \rightarrow W) \mapsto
    (f_\ast\colon \Hom_K(X, V) \rightarrow \Hom_K(X, W),
    f_\ast(g) = f \circ g)$ \\
    $\Hom_K(-, X)\colon K\text{-}\cat{Vec} \rightarrow K\text{-}\cat{Vec}$,
    $V \mapsto \Hom_K(V, X)$, \\
    $(f\colon V \rightarrow W) \mapsto
    (f^\ast\colon \Hom_K(V, X) \rightarrow \Hom_K(W, X),
    f^\ast(g) = g \circ f)$ \\
    Für $X = K$ erhält man den Dualraum $V^\ast = \Hom_K(V, K)$ und das
    übliche "`Sternen"' von Abbildungen.
    Analog geht das für beliebige Kategorien.
\end{Bsp}

\linie

\begin{Bem}
    Wozu nützen Kategorien? \\
    Will man zum Beispiel feststellen, ob $X \cong Y$ als topologische Räume
    mit $X := [0, 1]$ und $Y := [0, 1] \cup [2, 3]$, so benutzt man die
    Annahme, dass $X \cong Y$ mit zueinander inversen Homöomorphismen
    $f\colon X \rightarrow Y$ und $g\colon Y \rightarrow X$, d.\,h.
    $X$ und $Y$ sind in $\cat{Top}$ isomorph.
    Dann müssen nach Anwendung des kovariaten Funktors $\pi_0$ auch
    $\pi_0(X)$ und $\pi_0(Y)$ isomorph in $\cat{Set}$ sein, wobei die
    zueinander inversen Isomorphismen $f_\ast$ und $g_\ast$ sind.
    Dies kann allerdings nicht gelten, da $\pi_0(X)$ ein- und
    $\pi_0(Y)$ zweielementig ist.
    Daher gilt $X \not\cong Y$.
\end{Bem}

\pagebreak
