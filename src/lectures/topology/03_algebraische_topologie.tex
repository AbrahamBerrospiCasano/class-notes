\chapter{%
    Algebraische Topologie%
}

\section{%
    Gruppen%
}

\subsection{%
    Gruppen und Untergruppen%
}

\begin{Def}{Gruppe}
    Eine \begriff{Gruppe} ist ein Paar $(G, \ast)$ mit einer Menge $G$ und
    einer Abbildung \\
    $\ast\colon G \times G \rightarrow G$, sodass
    \begin{enumerate}
        \item
        für alle $a, b, c \in G$ die Gleichung
        $(a \ast b) \ast c = a \ast (b \ast c)$ gilt
        (\begriff{Assoziativität}),

        \item
        es ein $e \in G$ gibt mit $e \ast a = a \ast e = a$ für alle
        $a \in A$ (\begriff{neutrales Element}) sowie

        \item
        es für alle $a \in G$ ein $b \in G$ gibt mit
        $a \ast b = b \ast a = e$ (\begriff{inverses Element}).
    \end{enumerate}
    Die Gruppe heißt \begriff{kommutativ} oder \begriff{abelsch}, falls
    \begin{enumerate}[resume]
        \item
        für alle $a, b \in G$ die Gleichung $a \ast b = b \ast a$ gilt
        (\begriff{Kommutativität}).
    \end{enumerate}
\end{Def}

\begin{Bem}\\
    Das neutrale Element $e$ und das zu $a \in G$ inverse Element $b$
    sind eindeutig bestimmt.
\end{Bem}

\begin{Bsp}
    $(\integer, +)$, $(\GL_n(\real), \cdot)$, $(\mathfrak{S}_n, \circ)$ \\
    Sind $(G_1, \ast_1), \dotsc, (G_n, \ast_n)$ Gruppen, so ist
    $(G, \ast)$ eine Gruppe, wobei $G := G_1 \times \dotsb \times G_n$ und
    $\ast\colon G \times G \rightarrow G$,
    $(a_1, \dotsc, a_n) \ast (b_1, \dotsc, b_n) :=
    (a_1 \ast_1 b_1, \dotsc, a_n \ast_n b_n)$.
\end{Bsp}

\begin{Bem}
    Man schreibt Gruppen meistens multiplikativ (oder additiv), d.\,h. statt
    $\ast$ benutzt man oft das Symbol $\cdot$.
    Man spricht dann von der \begriff{Multiplikation}
    $\cdot\colon G \times G \rightarrow G$ und $ab = a \cdot b$ heißt das
    \begriff{Produkt} von $a$ und $b$.
    Dabei bezeichnet $1 = 1_G$ das neutrale Element und $a^{-1}$ das zu
    $a$ inverse Element.
\end{Bem}

\linie

\begin{Def}{Operationen mit Mengen}
    Sind $(G, \cdot)$ eine Gruppe, $a, b \in G$ und $S, T \subset G$, so ist \\
    $a \cdot T := \{a \cdot t \;|\; t \in T\}$, \qquad
    $T \cdot a := \{t \cdot a \;|\; t \in T\}$, \qquad
    $S \cdot T := \{s \cdot t \;|\; s \in S,\; t \in T\}$ und \\
    $S^{-1} := \{s^{-1} \;|\; s \in S\}$.
\end{Def}

\begin{Def}{Untergruppe}
    $U \subset G$ heißt \begriff{Untergruppe} ($U < G$), falls
    $1 \in U$, $U \cdot U \subset U$ und $U^{-1} \subset U$.
\end{Def}

\begin{Satz}{Untergruppen von $(\integer, +)$}
    $(\integer, +)$ hat nur Untergruppen der Form $n \cdot \integer$,
    $n \in \natural$.
\end{Satz}

\linie

\begin{Def}{erzeugte Untergruppe}
    Sei $S \subset G$.
    Dann ist die von $S$ \begriff{erzeugte Untergruppe} \\
    $\aufspann{S} := \{s_1^{e_1} \dotsm s_n^{e_n} \;|\; n \in \natural,\;
    s_1, \dotsc, s_n \in S,\; e_1, \dotsc, e_n \in \integer\}$
    die kleinste Untergruppe von $G$, die $S$ enthält.
\end{Def}

\begin{Bsp}
    In $(\integer, +)$ gilt $\aufspann{3} = 3\integer$ und
    $\aufspann{3, 5} = \integer$.
\end{Bsp}

\begin{Def}{zyklisch}
    $G$ heißt zyklisch, falls $G = \aufspann{a}$ für ein $a \in G$.
\end{Def}

\begin{Bsp}
    Die Gruppen $(\integer, +)$ und $(\integer/n\integer, +)$ sind zyklisch.
\end{Bsp}

\pagebreak

\subsection{%
    Nebenklassen und Quotientenmenge%
}

\begin{Def}{Äquivalenzrelation auf $G$}
    Sei $H < G$.
    Dann kann man auf $G$ eine Äquivalenzrelation definieren durch
    $a \sim b$, falls $a^{-1} b \in H$.
    Für $a \in G$ ist die Äquivalenzklasse $aH$ (\begriff{Linksnebenklasse})
    und die Menge aller Äquivalenzklassen ist
    $G/H := \{a \cdot H \;|\; a \in G\}$ mit der
    Projektion $\pi\colon G \rightarrow G/H$, $\pi(a) := aH$.
    $|G : H| := |G/H|$ heißt \begriff{Index} der Untergruppe $H$ in $G$.
\end{Def}

\begin{Bem}
    Im Allgemeinen ist $G/H$ keine Gruppe.
\end{Bem}

\linie

\begin{Satz}{\name{Lagrange}}
    Für $H < G$ gilt $|G| = |H| \cdot |G/H|$
    (d.\,h. insbesondere $|H| \;|\; |G|$).
\end{Satz}

\begin{Kor}
    Ist $|G| = p$ eine Primzahl, so ist $G$ zyklisch, d.\,h.
    besitzt keine echte nicht-triviale Untergruppe.
\end{Kor}

\subsection{%
    Gruppenhomomorphismen, Bild und Kern%
}

\begin{Def}{Gruppenhomomorphismus}
    Seien $(G, \ast)$ und $(H, \circ)$ Gruppen.
    Eine Abbildung $h\colon G \rightarrow H$ heißt \begriff{Homomorphismus},
    falls $h(a \ast b) = h(a) \circ h(b)$ für alle $a, b \in G$.
    Injektive, surjektive bzw. bijektive Homomorphismen heißen
    \begriff{Monomorphismen}, \begriff{Epimorphismen} bzw.
    \begriff{Isomorphismen}.
    Homomorphismen einer Gruppe in sich heißen \begriff{Endomorphismen},
    im bijektiven Fall \begriff{Auto\-morphismen}.
\end{Def}

\begin{Bem}
    Für einen Gruppenhomomorphismus $h$ gilt $h(1_G) = 1_H$ und
    $h(a^{-1}) = (h(a))^{-1}$.
\end{Bem}

\begin{Bsp}
    Sind $V$ und $W$ $K$-Vektorräume und $h\colon V \rightarrow W$ linear,
    so ist $h\colon (V, +) \rightarrow (W, +)$ ein Homomorphismus.
    Andere Beispiele sind
    $\det\colon \GL_n(\real) \rightarrow (\real \setminus \{0\}, \cdot)$ und
    $\exp\colon (\real, +) \rightarrow (\real, \cdot)$.
\end{Bsp}

\begin{Def}{Kategorie der Gruppen}
    Gruppen und ihre Homomorphismen bilden die Kategorie der Gruppen
    $\cat{Grp}$:
    Objekte sind die Gruppen, Morphismen sind die Homomorphismen und
    die Verknüpfung ist die übliche Verknüpfung (die Komposition zweier
    Homomorphismen ist wieder ein Homomorphismus).
\end{Def}

\linie

\begin{Def}{Bild und Kern}
    Sei $f\colon G \rightarrow H$ ein Homomorphismus. \\
    Dann heißen $\im f := f(G)$ \begriff{Bild} von $f$ und
    $\ker f := f^{-1}(\{1_H\})$ \begriff{Kern} von $f$.
\end{Def}

\subsection{%
    Normale Untergruppen und Quotientengruppen%
}

\begin{Def}{normale Untergruppe}\\
    $K < G$ heißt \begriff{normal} ($K \vartriangleleft G$), falls
    für alle $g \in G$ gilt, dass $gKg^{-1} = K$.
\end{Def}

\begin{Bem}
    Diese Bedingung ist äquivalent zu $gK = Kg$ für alle $g \in G$, d.\,h.
    $K < G$ ist normal genau dann, wenn für jedes $g \in G$ die
    Linksnebenklasse $gK$ mit der Rechtsnebenklasse $Kg$ übereinstimmt.
\end{Bem}

\begin{Satz}{Kern ist normale Untergruppe}
    Ist $f\colon G \rightarrow H$ ein Homomorphismus, so ist
    $\ker f \vartriangleleft G$.
\end{Satz}

\linie

\begin{Lemma}{$\sim$ für Untergruppen verträglich mit Multiplikation}\\
    Ist $K \vartriangleleft G$ eine normale Untergruppe, so folgt aus
    $a \sim b$ und $a' \sim b'$, dass $aa' \sim bb'$.
\end{Lemma}

\begin{Satz}{Faktorgruppe}
    Ist $K \vartriangleleft G$ eine normale Untergruppe, so gibt es genau
    eine Gruppenstruktur auf $G/K$, die $\pi$ zu einem Homomorphismus macht,
    nämlich $(a \cdot K) \cdot (b \cdot K) := (a \cdot b) \cdot K$.
\end{Satz}

\linie

\begin{Satz}{Homomorphiesatz}
    Seien $K \vartriangleleft G$ eine normale Untergruppe und
    $f\colon G \rightarrow H$ ein Homomorphismus.
    Dann gibt es einen Homomorphismus $\overline{f}\colon G/K \rightarrow H$
    mit $f = \overline{f} \circ \pi$ genau dann, wenn $K < \ker f$.
    $\overline{f}$ ist eindeutig und es gilt $\im(\overline{f}) = \im(f)$
    sowie $\ker(\overline{f}) = \ker(f) / K$.
\end{Satz}

\pagebreak

\subsection{%
    Isomorphiesätze%
}

\begin{Satz}{erster Isomorphiesatz}
    Jeder Homomorphismus $f\colon G \rightarrow H$ faktorisiert zu \\
    $G \xrightarrow{\pi} G/\ker(f) \xrightarrow{\overline{f}} \im(f)
    \xrightarrow{\iota} H$
    mit $\pi$ Epi-, $\overline{f}$ Iso- und $\iota$ Monomorphismus:
    \displaymathother
    \begin{align*}
        \begin{xy}
            \xymatrix{
                G \ar[r]^f \ar @{>>} [d]_\pi &
                H \\
                G/\ker(f) \ar[r]^\sim_{\overline{f}} &
                \im(f) \ar @{^{(}->} [u]_\iota
            }
        \end{xy}
    \end{align*}
    \displaymathnormal
\end{Satz}

\begin{Kor}
    Jede zyklische Gruppe ist isomorph zu $\integer/n\integer$.
\end{Kor}

\linie

\begin{Def}{Kommutator}
    Seien $G$ eine Gruppe und $a, b \in G$.
    Dann heißt $[a, b] := aba^{-1}b^{-1}$ \begriff{Kommutator} von $a$ und $b$.
    Die \begriff{Kommutatoruntergruppe} von $G$ ist
    $[G, G] := \aufspann{[a, b] \;|\; a, b \in G}$.
\end{Def}

\begin{Satz}{\name{Abel}schmachung}\\
    Es gilt $[G, G] \vartriangleleft G$ und
    die \begriff{\name{Abel}schmachung} $G_{\ab} := G/[G, G]$ ist eine
    abelsche Gruppe. \\
    Jeder Homomorphismus $f\colon G \rightarrow A$ in eine
    abelsche Gruppe $A$ induziert einen Homomorphismus
    $\overline{f}\colon G_{\ab} \rightarrow A$ mit
    $f = \overline{f} \circ \alpha_G$, wobei
    $\alpha_G\colon G \rightarrow G_{\ab}$ die Quotientenabbildung ist.
\end{Satz}

\subsection{%
    Freie Gruppen%
}

\begin{Def}{freie Gruppe}
    Eine Gruppe $G$ heißt \begriff{frei} über einer Teilmenge $S \subset G$,
    falls sich jedes $a \in G$ eindeutig schreiben lässt als
    $a = s_1^{e_1} \dotsm s_n^{e_n}$, $n \in \integer$,
    $s_1, \dotsc, s_n \in S$, $e_1, \dotsc, e_n \in \integer$. \\
    In diesem Fall heißt $S$ \begriff{Basis} von $G$.
\end{Def}

\begin{Bsp}
    $(\integer, +)$ ist frei über $S = \{1\}$.
    $(\integer/n\integer, +)$ ist nicht frei.
\end{Bsp}

\begin{Satz}{Existenz einer freien Gruppe}
    Zu jeder Menge $S$ existiert eine freie Gruppe $F(S)$.
\end{Satz}

\begin{Bem}
    Konstruktion:
    Definiere $A := S \times \{\pm 1\}$ mit
    $(s, \varepsilon)^{-1} = (s, -\varepsilon)$ für $(s, \varepsilon) \in A$.
    Sei $A^\ast := \{\text{endl. Wörter in A}\}$ und
    $\cdot\colon A^\ast \times A^\ast \rightarrow A^\ast$ die Verknüpfung
    von Wörtern.
    Auf $A^\ast$ wird die Äquivalenzrelation $\equiv$ erzeugt durch
    $uaa^{-1}v \equiv uv$ mit $u, v \in A^\ast$, $a \in A$, d.\,h. zwei
    Wörter aus $A^\ast$ sind äquivalent genau dann, wenn sie durch eine
    endliche Folge von Einfügen oder Entfernen von Unterwörtern der Form
    $aa^{-1}$ mit $a \in A$ ineinander übergehen.
    $(F(S), \cdot)$ mit $F(S) := A^\ast/\equiv$ und
    $\cdot\colon F(S) \times F(S) \rightarrow F(S)$
    der durch $\cdot$ auf $A^\ast$ induzierten Multiplikation ist dann
    nach Konstruktion eine freie Gruppe.
\end{Bem}

\linie

\begin{Satz}{universelle Eigenschaft}
    Eine Gruppe $F$ ist frei über $S \subset F$ genau dann, wenn es für alle
    Abbildungen $f\colon S \rightarrow G$ genau einen Homomorphismus
    $h\colon F \rightarrow G$ gibt mit $h|_S = f$.
\end{Satz}

\begin{Kor}
    Ist $S \subset G$, dann induziert die Inklusion
    $\iota\colon S \rightarrow G$ einen Homomorphismus \\
    $\phi\colon F(S) \rightarrow G$.
\end{Kor}

\begin{Kor}
    Jede Gruppe ist isomorph zu einem Quotienten einer freien Gruppe.
\end{Kor}

\pagebreak

\section{%
    Fundamentalgruppe und Überlagerungen%
}

\subsection{%
    Fundamentalgruppe%
}

\begin{Def}{homotop bei festem $A$}
    Seien $X$ und $Y$ topologische Räume sowie $A \subset X$.
    Zwei stetige Abbildungen $f, g\colon X \rightarrow Y$ heißen
    \begriff{homotop bei festem $A$} ($f \simeq g \text{ fix } A$ oder
    $f \simeq_A g$), falls es eine Homotopie
    $H\colon [0, 1] \times X \rightarrow Y$ von $H_0 = f$ nach $H_1 = g$ gibt
    mit $H_s|_A = f|_A$ für alle $s \in [0, 1]$.
\end{Def}

\begin{Lemma}{Äquivalenzrelation}
    Homotopie bei festem $A$ ist eine Äquivalenzrelation.
\end{Lemma}

\linie

\begin{Def}{äquivalente Wege}
    Zwei Wege $\alpha, \beta\colon [0, 1] \rightarrow X$ heißen
    \begriff{äquivalent} ($\alpha \sim \beta$), falls es eine Homotopie
    $H\colon [0, 1] \times [0, 1] \rightarrow X$ von $H_0 = \alpha$ und
    $H_1 = \beta$ gibt mit $H(s, 0) = \alpha(0)$ und $H(s, 1) = \alpha(1)$
    für alle $s \in [0, 1]$.
\end{Def}

\begin{Lemma}{Äquivalenzrelation}
    Die Äquivalenz von Wegen ist eine Äquivalenzrelation. \\
    Die Quotientenmenge sei $\Pi X(a, b) := PX(a, b) / \sim$. \\
    Aus $\alpha \sim \beta$ folgt $\overline{\alpha} \sim \overline{\beta}$,
    d.\,h. man erhält $-\colon \Pi X(a, b) \rightarrow \Pi X(b, a)$,
    $[\gamma] \mapsto \overline{[\gamma]} := [\overline{\gamma}]$. \\
    Aus $\alpha \sim \alpha'$ in $PX(a, b)$ und $\beta \sim \beta'$ in
    $PX(b, c)$ folgt $\alpha \ast \beta \sim \alpha' \ast \beta'$ in
    $PX(a, c)$, d.\,h. man erhält
    $\ast\colon \Pi X(a, b) \times \Pi X(b, c) \rightarrow \Pi X(a, c)$,
    $([\alpha], [\beta]) \mapsto [\alpha] \ast [\beta] := [\alpha \ast \beta]$.
\end{Lemma}

\linie

\begin{Def}{Wegekategorie}\\
    Jeder topologische Raum $X$ definiert eine Kategorie,
    die \begriff{Wegekategorie} $\cat{\Pi X}$:
    \begin{itemize}
        \item
        Objekte sind die Punkte $a \in X$,

        \item
        Morphismen zu $a, b \in X$ sind die Klassen $[\gamma] \in \Pi X(a, b)$
        und

        \item
        die Verknüpfung ist die Komposition $\ast$ wie oben.
    \end{itemize}
    In $\cat{\Pi X}$ ist jeder Morphismus ein Isomorphismus
    (invertierbar durch $[\gamma] \mapsto [\overline{\gamma}]$).
\end{Def}

\linie

\begin{Def}{$f_\sharp$}
    Ist $f\colon X \rightarrow Y$ eine stetige Abbildung, dann kann
    man jedem Weg $\gamma$ von $a$ nach $b$ in $X$
    den Weg $f \circ \gamma$ von $f(a)$ nach $f(b)$ in $Y$ zuordnen. \\
    Dies definiert eine Abbildung
    $f_\sharp\colon PX(a, b) \rightarrow PY(f(a), f(b))$,
    $\gamma \mapsto f \circ \gamma$.
    Sie ist auch wohldefiniert auf Homotopieklassen, d.\,h.
    $f_\sharp\colon \Pi X(a, b) \rightarrow \Pi Y(f(a), f(b))$,
    $[\gamma] \mapsto [f \circ \gamma]$.
\end{Def}

\begin{Satz}{$f_\sharp$ als Funktor}\\
    Jede stetige Abbildung $f\colon X \rightarrow Y$ induziert einen Funktor
    $f_\sharp\colon \cat{\Pi X} \rightarrow \Pi Y$:
    \begin{itemize}
        \item
        Jedem Punkt $a \in X$ wird der Punkt $f(a) \in Y$ zugeordnet.

        \item
        Jeder Homotopieklasse $[\gamma] \in \Pi X(a, b)$ wird die
        Homotopieklasse \\
        $f_\sharp([\gamma]) := [f \circ \gamma] \in \Pi Y(f(a), f(b))$
        zugeordnet.

        \item
        Es gilt $f_\sharp([1_a]) = [1_{f(a)}]$ und
        $f_\sharp([\alpha] \ast [\beta]) =
        f_\sharp([\alpha]) \ast f_\sharp([\beta])$.
    \end{itemize}
\end{Satz}

\linie

\begin{Def}{Fundamentalgruppe}
    Seien $X$ ein topologischer Raum und $x_0 \in X$. \\
    Dann heißt $\pi_1(X, x_0) := \Pi X(x_0, x_0)$
    die \begriff{Fundamentalgruppe} von $X$ in $x_0$.
    Dies ist eine Gruppe.
\end{Def}

\begin{Satz}{induzierter Isomorphismus}
    Jeder Weg $\gamma\colon [0, 1] \rightarrow X$ von $x_0$ nach $x_1$
    induziert einen Isomorphismus
    $h_\gamma\colon \pi_1(X, x_0) \rightarrow \pi_1(X, x_1)$,
    $h_\gamma([\alpha]) := [\overline{\gamma} \ast \alpha \ast \gamma]$
    mit $h_\gamma^{-1} = h_{\overline{\gamma}}$.
\end{Satz}

\linie
\pagebreak

\begin{Def}{einfach zusammenhängend}
    Sei $X$ ein topologischer Raum.
    $X$ heißt \begriff{einfach zusammen\-hängend}, falls $X$ wegzusammenhängend
    und $\pi_1(X, x_0)$ für ein $x_0 \in X$ trivial ist.
\end{Def}

\begin{Bem}
    In diesem Fall ist $\pi_1(X, x_0)$ automatisch
    für alle $x_0 \in X$ trivial. \\
    $X$ ist wegzusammenhängend genau dann, wenn für alle $x, y \in X$
    $\Pi X(x, y)$ genau aus einem Element besteht.
\end{Bem}

\begin{Bsp}
    $\real^n$ ist einfach zusammenhängend.
\end{Bsp}

\begin{Satz}{$\sphere^n$, $n \ge 2$ einfach zush.}
    Für $n \ge 2$ ist $\sphere^n$ einfach zusammenhängend.
\end{Satz}

\begin{Bem}
    $\sphere^1$ ist nicht einfach zusammenhängend, da
    $\pi_1(\sphere^1, 1) \cong \integer$ (siehe unten).
\end{Bem}

\linie

\begin{Def}{punktierter Raum}
    Ein \begriff{punktierter Raum} ist ein Paar $(X, x_0)$ mit
    einem topologischen Raum $X$ und einem Punkt $x_0 \in X$.
    Analog zu $\cat{Top}$ ist die Kategorie $\cat{Top}_\ast$ der punktierten
    Räume definiert.
    Eine Abbildung $f\colon (X, x_0) \rightarrow (Y, y_0)$ zwischen
    punktierten Räumen ist eine Abbildung $f\colon X \rightarrow Y$ mit
    $f(x_0) = y_0$.
\end{Def}

\begin{Satz}{Fundamentalgruppe als Funktor}
    Die Fundamentalgruppe ist ein Funktor
    $\cat{Top}_\ast \rightarrow \cat{Grp}$:
    \begin{itemize}
        \item
        Jedem punktierten Raum $(X, x_0)$ wird die Gruppe $\pi_1(X, x_0)$
        zugeordnet.

        \item
        Jeder stetigen Abbildung $f\colon (X, x_0) \rightarrow (Y, y_0)$ wird
        der Gruppenhomomorphismus \\
        $f_\sharp =: \pi_1(f)\colon \pi_1(X, x_0) \rightarrow \pi_1(Y, y_0)$,
        $f_\sharp([\alpha]) = [f \circ \alpha]$ zugeordnet.

        \item
        Es gilt $\pi_1(\id_{(X, x_0)}) = \id_{\pi_1(X, x_0)}$ und
        $\pi_1(f \circ g) = \pi_1(f) \circ \pi_1(g)$.
    \end{itemize}
\end{Satz}

\begin{Kor}\\
    Aus $f\colon (X, x_0) \homoe (Y, y_0)$ folgt, dass
    $f_\sharp\colon \pi_1(X, x_0) \bij \pi_1(Y, y_0)$ ein
    Gruppenisomorphismus ist.
\end{Kor}

\begin{Satz}{$f \sim g \;\Rightarrow\; f_\sharp = g_\sharp$}\\
    Sind $f, g\colon (X, x_0) \rightarrow (Y, y_0)$ homotop bei festem $x_0$,
    dann gilt $f_\sharp = g_\sharp$.
\end{Satz}

\subsection{%
    Überlagerungen%
}

\begin{Def}{triviale Überlagerung}\\
    Seien $X$ und $\widetilde{X}$ topologische Räume sowie
    $p\colon \widetilde{X} \rightarrow X$ stetig und surjektiv. \\
    Ein Teilraum $U \subset X$ heißt \begriff{von $p$ trivial überlagert},
    falls $p^{-1}(U) = \bigsqcup_{i \in I} \widetilde{U}_i$
    mit of"|fenen Mengen $\widetilde{U}_i \subset \widetilde{X}$, wobei
    $p_i := p|_{U_i}\colon \widetilde{U}_i \rightarrow U$ für alle $i \in I$
    ein Homöomorphismus ist.
\end{Def}

\begin{Def}{Überlagerung}
    $p\colon \widetilde{X} \rightarrow X$ heißt \begriff{Überlagerung}, falls
    jeder Punkt $x \in X$ eine of"|fene Umgebung $U \subset X$ besitzt,
    die von $p$ trivial überlagert wird. \\
    In diesem Fall heißt $\widetilde{X}$ der \begriff{Überlagerungsraum} und
    $X$ der \begriff{überlagerte Raum}.
\end{Def}

\begin{Bsp}
    $\id\colon X \rightarrow X$ ist eine Überlagerung. \\
    Jeder Homöomorphismus $p\colon \widetilde{X} \rightarrow X$
    ist eine Überlagerung. \\
    Ist $F$ ein diskreter Raum, dann ist $\pr\colon X \times F \rightarrow X$,
    $\pr(x, y) = x$ eine (triviale) Überlagerung. \\
    Sind $p_i\colon \widetilde{X}_i \rightarrow X_i$ Überlagerungen, dann auch
    $\bigsqcup_{i \in I} p_i\colon \bigsqcup_{i \in I} \widetilde{X}_i
    \rightarrow \bigsqcup_{i \in I} X_i$.
\end{Bsp}

\begin{Def}{Faser, Blätter}
    Für $x \in X$ heißt $p^{-1}(x) := p^{-1}(\{x\}) \subset \widetilde{X}$ die
    \begriff{Faser} über $x$. \\
    Jede Faser $p^{-1}(x)$ ist diskret in $\widetilde{X}$.
    Die Kardinalität $|p^{-1}(x)|$ heißt
    \begriff{Anzahl der Blätter} über $x$. \\
    Gilt $|p^{-1}(x)| = k \in \natural$ für alle $x \in X$, so heißt $p$ eine
    \begriff{$k$-blättrige Überlagerung}.
\end{Def}

\begin{Satz}{$p(t) = e^{2\pi\i t}$ ist Überlagerung}\\
    Die Abbildung $p\colon \real \rightarrow \sphere^1$, $p(t) := e^{2\pi\i t}$
    ist eine Überlagerung.
\end{Satz}

\linie
\pagebreak

\begin{Def}{Hochhebung}
    Seien $p\colon (\widetilde{X}, \widetilde{x}_0) \rightarrow (X, x_0)$
    und $f\colon (W, w_0) \rightarrow (X, x_0)$ stetige Abbildungen mit
    gleichem Zielraum.
    Dann heißt eine stetige Abbildung
    $\widetilde{f}\colon (W, w_0) \rightarrow (\widetilde{X}, \widetilde{x}_0)$
    \begriff{Hochhebung} von $f$ bzgl. $p$, falls $p \circ \widetilde{f} = f$.
    \displaymathother
    \begin{align*}
        \begin{xy}
            \xymatrix{
                &
                (\widetilde{X}, \widetilde{x}_0) \ar[d]^p \\
                (W, w_0) \ar[r]_f \ar@{-->}[ru]^{\widetilde{f}} &
                (X, x_0)
            }
        \end{xy}
    \end{align*}
    \displaymathnormal
\end{Def}

\begin{Satz}{Fundamentalsatz der Überlagerungstheorie}
    Sei $p\colon (\widetilde{X}, \widetilde{x}_0) \rightarrow (X, x_0)$
    eine Überlagerung. \\
    Dann existiert zu jeder stetigen Abbildung
    $f\colon ([0, 1]^n, 0) \rightarrow (X, x_0)$
    genau eine Hochhebung $\widetilde{f}\colon ([0, 1]^n, 0)
    \rightarrow (\widetilde{X}, \widetilde{x}_0)$.
\end{Satz}

\begin{Bem}
    Für $n = 1$ besagt der Satz, dass zu jedem Weg
    $\gamma\colon ([0, 1], 0) \rightarrow (X, x_0)$ genau eine Hochhebung
    $\widetilde{\gamma}\colon ([0, 1], 0) \rightarrow
    (\widetilde{X}, \widetilde{x}_0)$ existiert. \\
    Für $n = 2$ besagt der Satz, dass zu jeder Homotopie
    $H\colon ([0, 1]^2, 0) \rightarrow (X, x_0)$ von
    $H_0 = \gamma$ nach $H_1 = \gamma'$ eine Homotopie
    $\widetilde{H}\colon ([0, 1]^2, 0) \rightarrow
    (\widetilde{X}, \widetilde{x}_0)$ von
    $\widetilde{H}_0 = \widetilde{\gamma}$ nach
    $\widetilde{H}_1 = \widetilde{\gamma}'$ existiert \\
    (dabei sind $\gamma, \gamma'\colon ([0, 1], 0) \rightarrow (X, x_0)$ Wege).
\end{Bem}

\begin{Def}{Menge aller Wege, die in einem Punkt beginnen}
    Für einen topologischen Raum $X$ sei
    $P(X, x_0) := \bigcup_{x \in X} PX(x_0, x)$ die Menge aller Wege in $X$,
    die in $x_0$ beginnen. \\
    Entsprechend ist $\Pi(X, x_0) := \bigcup_{x \in X} \Pi X(x_0, x) =
    P(X, x_0)/\sim$ die Menge aller Äquivalenzklassen von Wegen in $X$,
    die in $x_0$ beginnen.
\end{Def}

\begin{Satz}{induzierte Bijektionen $p_\sharp$}
    Jede Überlagerung
    $p\colon (\widetilde{X}, \widetilde{x}_0) \rightarrow (X, x_0)$
    induziert Bijektionen
    $p_\sharp\colon P(\widetilde{X}, \widetilde{x}_0) \bij P(X, x_0)$,
    $\alpha \mapsto p \circ \alpha$ und
    $p_\sharp\colon \Pi(\widetilde{X}, \widetilde{x}_0) \bij \Pi(X, x_0)$,
    $[\alpha] \mapsto [p \circ \alpha]$.
\end{Satz}

\begin{Kor}
    $p_\sharp\colon \pi_1(\widetilde{X}, \widetilde{x}_0) \rightarrow
    \pi_1(X, x_0)$ ist injektiv.
\end{Kor}

\linie

\begin{Def}{Fasertransport}
    Sei $p\colon \widetilde{X} \rightarrow X$ eine Überlagerung.
    Für $x \in X$ ist $F_x := p^{-1}(x)$ die Faser über dem Punkt $x$.
    Zu jedem Startwert $\widetilde{x} \in F_x$ und jedem Weg
    $\gamma \in PX(x, y)$ existiert genau eine Hochhebung
    $\widetilde{\gamma}\colon ([0, 1], 0) \rightarrow
    (\widetilde{X}, \widetilde{x})$.
    Der Endpunkt $\widetilde{y} = \widetilde{\gamma}(1)$ ergibt sich aus
    dem Startwert $\widetilde{x}$ und dem Verlauf von $\gamma$.
    Man setzt $\widetilde{x} \cdot \gamma := \widetilde{y}$.
    Dies ist wohldefiniert auf $[\gamma]$, d.\,h. man kann
    $\widetilde{x} \cdot [\gamma] := \widetilde{y}$ schreiben.
\end{Def}

\begin{Satz}{Fasertransport als Funktor}\\
    Jede Überlagerung $p\colon \widetilde{X} \rightarrow X$ definiert einen
    Funktor $F\colon \cat{\Pi X} \rightarrow \cat{Set}$:
    \begin{itemize}
        \item
        Jedem Punkt $x \in X$ wird seine Faser $F_x = p^{-1}(x)$ zugeordnet.

        \item
        Jedem Morphismus $[\gamma] \in \Pi X(x, y)$ wird die Abbildung
        $F_{[\gamma]}\colon F_x \rightarrow F_y$,
        $\widetilde{x} \mapsto \widetilde{x} \cdot [\gamma]$ zugeordnet.

        \item
        Es gilt $\widetilde{x} \cdot [1_x] = \widetilde{x}$ und
        $(\widetilde{x} \cdot [\alpha]) \cdot [\beta] =
        \widetilde{x} \cdot ([\alpha] \ast [\beta])$.
    \end{itemize}
    Da jeder Morphismus $[\gamma] \in \Pi X(x, y)$ in der Kategorie
    $\cat{\Pi X}$ invertierbar ist, ist die Abbildung
    $F_{[\gamma]}\colon F_x \rightarrow F_y$
    eine Bijektion zwischen den Fasern.
\end{Satz}

\begin{Kor}
    Sei $p\colon \widetilde{X} \rightarrow X$ eine Überlagerung.
    Dann operiert die Fundamentalgruppe \\
    $G := \pi_1(X, x_0)$ auf der
    Faser $F := p^{-1}(x_0)$ gemäß $F \times G \rightarrow F$,
    $(\widetilde{x}, [\gamma]) \mapsto \widetilde{x} \cdot [\gamma]$.
\end{Kor}

\linie

\begin{Satz}{Fundamentalgruppe der Kreislinie}
    Die Überlagerung $p\colon (\real, 0) \rightarrow (\sphere^1, 1)$ mit
    $p(t) = e^{2\pi\i t}$ induziert einen Gruppenisomorphismus
    $h\colon \pi_1(\sphere^1, 1) \rightarrow \integer$ mit
    $h([\gamma]) := 0 \cdot [\gamma]$.
\end{Satz}

\pagebreak

\subsection{%
    Quotienten%
}

\begin{Def}{Operation einer Gruppe}
    Seien $X$ ein topologischer Raum und $G$ eine Gruppe. \\
    Eine \begriff{(Links-)Operation} von $G$ auf $X$ ist eine Abbildung
    $\varphi\colon G \times X \rightarrow X$, $(g, x) \mapsto g \cdot x = gx$,
    sodass $1x = x$ und $(gh)x = g(hx)$ für alle $g, h \in G$ und $x \in X$
    gilt. \\
    Analog sind \begriff{Rechts-Operationen}
    $\varphi\colon X \times G \rightarrow X$ definiert. \\
    Eine Operation heißt \begriff{stetig}, falls
    $\varphi_g\colon X \rightarrow X$, $x \mapsto xg$ stetig ist
    für alle $g \in G$.
\end{Def}

\begin{Def}{Bahn}
    Für $x \in X$ heißt $Gx := \{gx \;|\; g \in G\}$ die \begriff{Bahn} von $x$
    unter der Operation von $G$.
    Zwei Bahnen sind entweder gleich oder disjunkt.
    Die \begriff{Quotientenmenge} ist $X/G := \{Gx \;|\; x \in X\}$ mit
    der \begriff{Quotientenabbildung} $q\colon X \rightarrow X/G$,
    $x \mapsto Gx$.
    Die Quotiententopologie macht $X/G$ zu einem topologischen Raum und
    $q$ zu einer stetigen Abbildung.
\end{Def}

\begin{Def}{freie (diskontinuierliche) Operation}
    Sei $\varphi\colon G \times X \rightarrow X$ eine Operation. \\
    $\varphi$ heißt \begriff{frei}, falls $gx \not= x$ für jeden Punkt
    $x \in X$ und alle $g \in G$ mit $g \not= 1$. \\
    $\varphi$ heißt \begriff{frei diskontinuierlich}, falls jeder Punkt
    $x \in X$ eine of"|fene Umgebung $U \subset X$ besitzt, sodass
    $U \cap gU = \emptyset$ für alle $g \in G$ mit $g \not= 1$.
\end{Def}

\begin{Bsp}
    $(\integer, +)$ operiert auf $\real$ durch
    $\integer \times \real \rightarrow \real$, $(k, x) \mapsto k + x$
    (Translation).
    Diese Operation ist frei diskontinuierlich.
    Analog operiert $(\integer^n, +)$ auf $\real^n$ durch
    $\integer^n \times \real^n \rightarrow \real^n$, $(k, x) \mapsto k + x$. \\
    Der Quotient
    $q\colon \sphere^n \rightarrow \real\projective^n = \sphere^n/\{\pm 1\}$
    entsteht durch die Operation
    ${\pm 1} \times \sphere^n \rightarrow \sphere^n$, $(g, x) \mapsto gx$
    (Punktspiegelung am Ursprung im $\real^{n+1}$).
    Die Operation ist frei diskontinuierlich. \\
    Die nicht-orientierbaren Flächen $F_g^- = F_g^+/\{\pm 1\}$ entstehen
    ebenso als Quotienten aus den orientierbaren Flächen $F_g^+$. \\
    Sei $a \in \real \setminus \rational$ und $\xi := e^{2\pi\i a}$.
    Dann ist $\integer \times \sphere^1 \rightarrow \sphere^1$,
    $(k, x) \mapsto \xi^k x$ eine freie Operation, aber nicht
    frei diskontinuierlich.
\end{Bsp}

\linie

\begin{Satz}{Homomorphismus durch Fasertransport}
    Sei $G \times \widetilde{X} \rightarrow \widetilde{X}$ eine stetige, freie
    diskontinuierliche Operation einer Gruppe $G$ auf einem topologischen Raum
    $\widetilde{X}$. Dann gilt:
    \begin{enumerate}
        \item
        Die Quotientenabbildung
        $q\colon \widetilde{X} \rightarrow X := \widetilde{X}/G$
        ist eine Überlagerung.

        \item
        Die Operation von $G$ kommutiert mit dem Fasertransport durch
        $\cat{\Pi X}$, d.\,h. \\
        $(g \cdot \widetilde{x}) \cdot [\gamma] =
        g \cdot (\widetilde{x} \cdot [\gamma])$
        für alle $g \in G$, $[\gamma] \in \Pi(X, x)$ und
        $\widetilde{x} \in q^{-1}(x)$.

        \item
        Für jeden Basispunkt $\widetilde{x}_0 \in \widetilde{X}$ und
        $x_0 := q(\widetilde{x}_0)$ existiert der Gruppenhomomorphismus
        $h\colon \pi_1(X, x_0) \rightarrow G$ mit
        $h([\alpha]) \cdot \widetilde{x}_0 = \widetilde{x}_0 \cdot [\alpha]$.

        \item
        Ist $\widetilde{X}$ wegzusammenhängend, dann ist $h$ surjektiv. \\
        Allgemein gilt $\im(h) = \{g \in G \;|\; \widetilde{x}_0 \text{ und }
        g \cdot \widetilde{x}_0 \text{ sind in } \widetilde{X}
        \text{ verbindbar}\}$.

        \item
        Ist $\widetilde{X}$ einfach zusammenhängend, dann ist $h$ bijektiv. \\
        Allgemein gilt
        $\ker(h) = q_\sharp(\pi_1(\widetilde{X}, \widetilde{x}_0))$.
    \end{enumerate}
\end{Satz}

\pagebreak

\subsection{%
    Hochhebungen%
}

\begin{Satz}{Eindeutigkeit von Hochhebungen auf wegzush. Räumen}\\
    Sei $p\colon (\widetilde{X}, \widetilde{x}_0) \rightarrow (X, x_0)$
    eine Überlagerung.
    Ist $(W, w_0)$ wegzusammenhängend, dann existiert zu jeder stetigen
    Abbildung $f\colon (W, w_0) \rightarrow (X, x_0)$
    höchstens eine Hochhebung \\
    $\widetilde{f}\colon (W, w_0) \rightarrow
    (\widetilde{X}, \widetilde{x}_0)$.
\end{Satz}

\begin{Satz}{Existenz von Hochhebungen auf wegzush.
             und lokal wegzush. Räumen}\\
    Seien $p\colon (\widetilde{X}, \widetilde{x}_0) \rightarrow (X, x_0)$
    eine Überlagerung und $(W, w_0)$ ein wegzusammenhängender und lokal
    wegzusammenhängender Raum.
    Dann erlaubt eine stetige Abbildung $f\colon (W, w_0) \rightarrow (X, x_0)$
    eine Hochhebung $\widetilde{f}\colon (W, w_0) \rightarrow
    (\widetilde{X}, \widetilde{x}_0)$ genau dann, wenn
    $f_\sharp(\pi_1(W, w_0)) \subset
    p_\sharp(\pi_1(\widetilde{X}, \widetilde{x}_0))$. \\
    (In diesem Fall ist die Hochhebung gemäß obigem Satz eindeutig.)
\end{Satz}

\subsection{%
    Decktransformationen und normale Überlagerungen%
}

\begin{Def}{Automorphismus}
    Sei $p\colon \widetilde{X} \rightarrow X$ eine Überlagerung.
    Ein Homöomorphismus $f\colon \widetilde{X} \homoe \widetilde{X}$ mit
    $p \circ f = p$ heißt \begriff{Automorphismus} oder
    \begriff{Decktransformation} der Überlagerung $p$.
    Die Menge $\Aut(p) := \{g\colon \widetilde{X} \homoe \widetilde{X} \;|\;
    p \circ g = p\}$ heißt \begriff{Automorphismengruppe} der Überlagerung $p$.
\end{Def}

\begin{Bsp}
    Für die Überlagerung $p\colon \real \rightarrow \sphere^1$ mit
    $p(t) = e^{2\pi\i t}$ ist die Translation
    $\tau\colon \real \rightarrow \real$ mit
    $\tau(x) = x + 1$ eine Decktransformation.
    Es gilt $\Aut(p) = \aufspann{\tau} \cong \integer$. \\
    Für die Überlagerung $p\colon \sphere^1 \rightarrow \sphere^1$ mit
    $p(z) = z^k$ ist die Rotation
    $\rho\colon \sphere^1 \rightarrow \sphere^1$ mit
    $\rho(z) = e^{2\pi\i/k} z$ eine Decktransformation.
    Es gilt $\Aut(p) = \aufspann{\rho} \cong \integer/k$.
\end{Bsp}

\begin{Satz}{Automorphismengruppe}
    Sei $\widetilde{X}$ wegzusammenhängend und $G < \Homeo(\widetilde{X})$
    operiere frei diskontinuierlich auf $\widetilde{X}$.
    Für die Überlagerung
    $q\colon \widetilde{X} \rightarrow X := \widetilde{X}/G$ gilt dann
    $\Aut(q) = G$.
\end{Satz}

\linie

\begin{Satz}{Transitivität der Decktransformationsgruppe}
    Sei $p\colon \widetilde{X} \rightarrow X$ eine
    wegzusammenhängende Überlagerung (d.\,h. $\widetilde{X}$ und $X$ sind
    wegzusammenhängend). Dann gilt:
    \begin{enumerate}
        \item
        Die Automorphismengruppe $\Aut(p)$ operiert frei diskontinuierlich auf
        $\widetilde{X}$.

        \item
        Operiert $\Aut(p)$ transitiv auf einer Faser, dann operiert
        $\Aut(p)$ transitiv auf jeder Faser und $p$ ist homöomorph zum
        Quotienten $q\colon \widetilde{X} \rightarrow \widetilde{X}/\Aut(p)$.
    \end{enumerate}
\end{Satz}

\linie

\begin{Def}{normale Überlagerung}
    Eine Überlagerung $p\colon \widetilde{X} \rightarrow X$ heißt
    \begriff{normal} oder \begriff{galoisch}, falls $\widetilde{X}$
    wegzusammenhängend ist und $\Aut(p)$ transitiv auf jeder Faser operiert.
\end{Def}

\begin{Bsp}
    $p\colon \real \rightarrow \sphere^1$ mit $p(t) = e^{2\pi\i t}$ ist eine
    normale Überlagerung. \\
    Jede zweiblättrige, wegzusammenhängende Überlagerung
    $p\colon \widetilde{X} \rightarrow X$ ist normal.
\end{Bsp}

\begin{Satz}{Kriterium für Normalität}
    Sei $p\colon (\widetilde{X}, \widetilde{x}_0) \rightarrow (X, x_0)$ eine
    Überlagerung wegzusammenhängender und lokal wegzusammenhängender Räume
    $\widetilde{X}$ und $X$. \\
    Dann ist $p$ normal genau dann, wenn die Untergruppe
    $p_\sharp(\pi_1(\widetilde{X}, \widetilde{x}_0))$ in $\pi_1(X, x_0)$
    normal ist.
\end{Satz}

\pagebreak

\subsection{%
    \name{Galois}-Korrespondenz%
}

\begin{Def}{Kategorie der wegzush. Überlagerungen}\\
    Sei $(X, x_0)$ wegzusammenhängend und lokal wegzusammenhängend. \\
    Die wegzusammenhängenden Überlagerungen bilden eine Kategorie
    $\cat{Cor(X, x_0)}$:
    \begin{itemize}
        \item
        Die Objekte sind die wegzusammenhängenden Überlagerungen
        $p\colon (Y, y_0) \rightarrow (X, x_0)$.

        \item
        Die Morphismen zwischen wegzusammenhängenden Überlagerungen \\
        $p\colon (Y, y_0) \rightarrow (X, x_0)$ und
        $q\colon (Z, z_0) \rightarrow (X, x_0)$ sind die stetigen
        Abbildungen $f\colon (Y, y_0) \rightarrow (Z, z_0)$ mit
        $q \circ f = p$.

        \item
        Die Komposition ist die für stetige Abbildungen übliche.
    \end{itemize}
\end{Def}

\begin{Bem}
    Aufgrund der Eindeutigkeit von Hochhebungen enthält jede Morphismenmenge
    $\Mor(p, q)$ höchstens ein Element.
    Im Falle $\Mor(p, q) \not= \emptyset$ schreibt man kurz
    $f\colon p \rightarrow q$ oder $p \rightarrow q$.
    Dies definiert eine Ordnung auf $\cat{Cor(X, x_0)}$,
    denn es gilt $p \rightarrow p$ (durch die Identität),
    aus $p \rightarrow q$ und $q \rightarrow r$ folgt $p \rightarrow r$
    (durch die Komposition) und
    aus $p \rightarrow q$ und $q \rightarrow p$ folgt $p \cong q$.
\end{Bem}

\linie

\begin{Bsp}
    Über der Kreislinie $(\sphere^1, 1)$ gibt es die Überlagerungen
    $p_0\colon (\real, 0) \rightarrow (\sphere^1, 1)$ mit \\
    $p_0(t) = e^{2\pi\i t}$ und
    $p_k\colon (\sphere^1, 1) \rightarrow (\sphere^1, 1)$
    mit $p_k(z) = z^k$ für $k \in \integer$, $k \not= 0$. \\
    Es gilt $p_0 \rightarrow p_k$ für alle $k \in \integer$, $k \not= 0$. \\
    Für $k, \ell \in \integer$ gilt $p_k \rightarrow p_\ell$ genau dann,
    wenn $\ell \;|\; k$.
    Genauer: Aus $k = m\ell$ folgt $z^k = z^{m\ell} = (z^m)^{\ell}$.
\end{Bsp}

\begin{Satz}{Faktorisierung von Überlagerungen}
    Seien $X$ lokal wegzusammenhängend, $r\colon Y \rightarrow Z$ und
    $q\colon Z \rightarrow X$ stetige, surjektive Abbildungen sowie
    $p := q \circ r\colon Y \rightarrow X$ ihre Komposition.
    \begin{enumerate}
        \item
        Sind $p$ und $q$ Überlagerungen, dann auch $r$.

        \item
        Sind $p$ und $r$ Überlagerungen, dann auch $q$.
    \end{enumerate}
\end{Satz}

\begin{Bem}
    Im Allgemeinen ist $p = q \circ r$ keine Überlagerung, wenn
    $q$ und $r$ Überlagerungen sind.
\end{Bem}

\begin{Kor}
    Sei $X$ wegzusammenhängend und lokal wegzusammenhängend,\\
    $p\colon (Y, y_0) \rightarrow (X, x_0)$ und
    $q\colon (Z, z_0) \rightarrow (X, x_0)$ wegzusammenhängende Überlagerungen
    sowie $K := p_\sharp(\pi_1(Y, y_0))$ und $H := q_\sharp(\pi_1(Z, z_0))$
    die zugehörigen Untergruppen in $\pi_1(X, x_0)$.
    \begin{enumerate}
        \item
        Ein Morphismus $f\colon p \rightarrow q$ existiert genau dann, wenn
        $K < H$ gilt. \\
        In diesem Fall ist $f$ eine Überlagerung mit Blätterzahl gleich dem
        Index von $K$ in $H$.

        \item
        Die Überlagerugn $f\colon p \rightarrow q$ ist normal genau dann, wenn
        $K \vartriangleleft H$ gilt. \\
        In diesem Fall gibt es einen Gruppenisomorphismus, sodass
        $\Aut(f) \cong H/K$.
    \end{enumerate}
\end{Kor}

\linie

\begin{Satz}{\name{Galois}-Korrespondenz}
    Sei $X$ wegzusammenhängend und lokal wegzusammenhängend,
    $p\colon (Y, y_0) \rightarrow (X, x_0)$ eine normale Überlagerung und
    $K := p_\sharp(\pi_1(Y, y_0))$ die zugehörige normale Untergruppe in der
    Fundamentalgruppe $G := \pi_1(X, x_0)$.
    Dann gibt es folgende Korrespondenz von wegzusammenhängenden Überlagerungen
    und Untergruppen:
    \begin{enumerate}
        \item
        Zu jeder \begriff{Zwischenüberlagerung}
        $q\colon (Z, z_0) \rightarrow (X, x_0)$ mit
        $p \rightarrow q$ gehört die \begriff{Zwischengruppe}
        $H := q_\sharp(\pi_1(Z, z_0))$ mit $K < H < G$.

        \item
        Zu jeder Zwischengruppe $H$ mit $K < H < G$ gehört eine
        (bis auf Homöomorphie eindeutige) Zwischenüberlagerung
        $q\colon (Z, z_0) \rightarrow (X, x_0)$ mit
        $q_\sharp(\pi_1(Z, z_0)) = H$.
    \end{enumerate}
\end{Satz}

\pagebreak

\subsection{%
    Universelle Überlagerung%
}

\begin{Def}{universelle Überlagerung}
    Sei $X$ wegzusammenhängend und lokal wegzusammenhängend.
    Eine Überlagerung
    $p\colon (\widetilde{X}, \widetilde{x}_0) \rightarrow (X, x_0)$ heißt
    \begriff{universell}, falls $\widetilde{X}$ einfach zusammenhängend ist.
\end{Def}

\begin{Bem}
    In diesem Fall ist auch
    $p\colon (\widetilde{X}, \widetilde{x}) \rightarrow (X, x)$ für alle
    $\widetilde{x} \in \widetilde{X}$ und $x := p(\widetilde{x})$ eine
    universelle Überlagerung.
\end{Bem}

\linie

\begin{Satz}{notwendige Bedingung}
    Ist $p\colon \widetilde{X} \rightarrow X$ eine universelle Überlagerung,
    dann existiert zu jedem $x \in X$ eine of"|fene Umgebung $U \subset X$,
    sodass $\iota\colon (U, x) \rightarrow (X, x)$ den trivialen Homomorphismus
    $\iota_\sharp\colon \pi_1(U, x) \rightarrow \pi_1(X, x)$ induziert.
\end{Satz}

\begin{Def}{semilokal einfach zusammenhängend}
    Ein topologischer Raum $X$ heißt
    \begriff{semilokal einfach zusammenhängend} in $x \in X$, falls eine
    Umgebung $U \subset X$ von $x$ in $X$ existiert, sodass jede Schleife in
    $(U, x)$ in $(X, x)$ zusammenziehbar ist.
    Äquivalent dazu ist, dass die Inklusion
    $\iota\colon (U, x) \rightarrow (X, x)$ den trivialen Homomorphismus
    $\iota_\sharp\colon \pi_1(U, x) \rightarrow \pi_1(X, x)$ induziert.
\end{Def}

\begin{Bsp}
    Der \begriff{Hawaiianische Ohrring}
    $W := \bigcup_{n \in \natural} \frac{1}{n} (\sphere^1 - 1)$ ist
    wegzusammenhängend und lokal wegzusammenhängend, aber nicht
    semilokal einfach zusammenhängend. \\
    Der \begriff{Hawaiianische Kegel} $CW$ ist semilokal einfach
    zusammenhängend, aber nicht lokal einfach zusammenhängend.
\end{Bsp}

\linie

\begin{Satz}{Konstruktion der universellen Überlagerung}
    Sei $X$ wegzusammenhängend und lokal wegzusammenhängend.
    Eine universelle Überlagerung
    $p\colon (\widetilde{X}, \widetilde{x}_0) \rightarrow (X, x_0)$
    existiert genau dann, wenn $X$ semilokal einfach zusammenhängend ist.
\end{Satz}

\pagebreak
