\chapter{%
    Geometrische Topologie%
}

\section{%
    Homotopie und Abbildungsgrad%
}

\begin{Bem}
    Im Folgenden sollen Methoden entwickelt werden, mit denen gezeigt werden
    können, dass $\real^2 \not\cong \real^3$.
    Im Allgemeinen gilt sogar $\real^n \not\cong \real^m$ für $n \not= m$.
\end{Bem}

\subsection{%
    Homotope Abbildungen%
}

\begin{Def}{Homotopie}
    Seien $X$ und $Y$ topologische Räume. \\
    Eine \begriff{Homotopie} ist eine stetige Abbildung
    $H\colon [0, 1] \times X \rightarrow Y$. \\
    Für $t \in [0, 1]$ definiert man die stetige Abbildung
    $H_t\colon X \rightarrow Y$, $H_t(x) = H(t, x)$.
\end{Def}

\begin{Def}{homotop}
    Zwei stetige Abbildungen $f, g\colon X \rightarrow Y$ heißen
    \begriff{homotop} in $Y$ ($f \simeq g$), falls es eine Homotopie
    $H\colon [0, 1] \times X \rightarrow Y$
    mit $H_0 = f$ und $H_1 = g$ gibt. \\
    Eine stetige Abbildung $f\colon X \rightarrow Y$ heißt
    \begriff{nullhomotop}/\begriff{zusammenziehbar} ($f \simeq \ast$),
    falls $f$ zu einer konstanten Abbildung
    $\const_X^\ast\colon X \rightarrow \{\ast\}$
    mit $\ast \in Y$ homotop ist. \\
    Der Raum $X$ heißt \begriff{zusammenziehbar} ($X \simeq \ast$),
    falls $\id_X \simeq \ast$.
\end{Def}

\begin{Bsp}
    Sei $X \subset \real^n$ sternförmig bzgl. $a \in \real^n$,
    z.\,B. $X = \real^n$ und $a = 0$. \\
    Dann ist $X$ zusammenziehbar durch $H(t, x) = (1 - t)x + ta$. \\
    Jede stetige Abbildung $f\colon X \rightarrow Y$ ist nullhomotop
    durch $H(t, x) = f((1 - t)x + ta)$. \\
    Jede stetige Abbildung $f\colon Y \rightarrow X$ ist nullhomotop
    durch $H(t, y) = (1 - t)f(y) + ta$.
\end{Bsp}

\begin{Satz}{homotope Abbildungen auf Einheitssphäre}
    Seien $X$ ein topologischer Raum und \\
    $f, g\colon X \rightarrow \sphere^n$ stetige Abbildungen, die
    nirgends \begriff{antipodal} sind, d.\,h.
    $\forall_{x \in X}\; f(x) \not= -g(x)$. \\
    Dann gilt $f \simeq g$.
\end{Satz}

\linie

\begin{Bem}
    Man kann jede Homotopie $H\colon [0, 1] \times X \rightarrow Y$ als einen
    Weg $h\colon [0, 1] \rightarrow \C(X, Y)$ mit $(h(t))(x) := H(t, x)$
    betrachten.
    Ist $H$ stetig, dann ist auch $h$ stetig. \\
    Ist $h$ stetig und $X$ lokal-kompakt, dann ist auch $H$ stetig.
\end{Bem}

\begin{Satz}{Homotopie als Äquivalenzrelation}
    Homotopie ist eine Äquivalenzrelation auf $\C(X, Y)$.
\end{Satz}

\begin{Def}{Homotopieklassen}
    Seien $X$ und $Y$ topologische Räume.
    Der Quotientenraum der \\
    \begriff{Homotopieklassen} (Äquivalenzklassen bzgl. der Homotopie) heißt
    $[X, Y] := \C(X, Y) /\! \simeq$.
\end{Def}

\begin{Bsp}
    Für $X$ lokal-kompakt gilt $[X, Y] = \pi_0(\C(X, Y))$. \\
    Für jeden topologischen Raum $X$ gilt $[\{\ast\}, X] = \pi_0(X)$.
\end{Bsp}

\begin{Satz}{Homotopie bei Kompositionen}
    Seien $f_0, f_1\colon X \rightarrow Y$ und $g_0, g_1\colon Y \rightarrow Z$
    stetige Abbildungen.
    Gilt $f_0 \simeq f_1$ und $g_0 \simeq g_1$, dann gilt auch
    $g_0 \circ f_0 \simeq g_1 \circ f_1$.
\end{Satz}

\begin{Def}{Kategorie $\cat{hTop}$}
    Man kann eine Kategorie $\cat{hTop}$ definieren.
    Die Objekte sind die topologischen Räume,
    die Morphismen $[X, Y]$ sind die Homotopieklassen $[f]$ stetiger
    Abbildungen $f\colon X \rightarrow Y$ und
    die Komposition ist $[g] \circ [f] := [g \circ f]$
    (wohldefiniert nach obigem Satz).
\end{Def}

\linie
\pagebreak

\begin{Def}{homotopie-äquivalent}
    $X$ und $Y$ heißen \begriff{homotopie-äquivalent} ($X \simeq Y$),
    falls es stetige Abbildungen $f\colon X \rightarrow Y$ und
    $g\colon Y \rightarrow X$ gibt mit $g \circ f \simeq \id_X$ und
    $f \circ g \simeq \id_Y$.
\end{Def}

\begin{Bem}
    So wie Homöomorphie die Isomorphie in $\cat{Top}$ ist, so ist
    Homotopie-Äquivalenz die Isomorphie in $\cat{hTop}$
    (insbesondere ist Homotopie-Äquivalenz eine Äquivalenzrelation).
\end{Bem}

\begin{Bsp}
    Aus $X \cong Y$ folgt $X \simeq Y$, die Umkehrung gilt nicht: \\
    Zum Beispiel sind $\real$ und $\real^2$ nicht homöomorph, aber
    homotopie-äquivalent. \\
    Ein Raum $X$ ist zusammenziehbar genau dann, wenn $X$ zu $\{\ast\}$
    homotopie-äquivalent ist.
\end{Bsp}

\begin{Satz}{$\sphere^n \simeq \real^{n+1} \setminus \{0\}$}
    $\sphere^n$ und $\real^{n+1} \setminus \{0\}$
    sind homotopie-äquivalent.
\end{Satz}

\begin{Satz}{Funktoren durch Homotopieklassen}
    Seien $X, Y, Z$ topologische Räume.
    \begin{enumerate}
        \item
        Jede stetige Abbildung $f\colon X \rightarrow Y$ induziert
        $f_\ast\colon [Z, X] \rightarrow [Z, Y]$,
        $f_\ast([h]) := [f \circ h]$. \\
        Dies definiert einen kovarianten Funktor
        $[Z, -]\colon \cat{Top} \rightarrow \cat{Set}$.

        \item
        Jede stetige Abbildung $f\colon Y \rightarrow X$ induziert
        $f^\ast\colon [Y, Z] \rightarrow [X, Z]$,
        $f^\ast([h]) := [h \circ f]$. \\
        Dies definiert einen kontravarianten Funktor
        $[-, Z]\colon \cat{Top} \rightarrow \cat{Set}$.

        \item
        Aus $f \simeq g$ folgt $f_\ast = g_\ast$ und $f^\ast = g^\ast$.
    \end{enumerate}
\end{Satz}

\linie

\begin{Def}{Retraktion}
    Seien $X$ ein topologischer Raum und $A \subset X$ eine Teilmenge. \\
    Eine \begriff{Retraktion} von $X$ auf $A \subset X$ ist eine stetige
    Abbildung $r\colon X \rightarrow A$ mit $r|_A = \id_A$. \\
    $A \subset X$ heißt \begriff{Retrakt} von $X$, falls es eine Retraktion
    von $X$ auf $A$ gibt.
\end{Def}

\begin{Def}{Deformationsretraktion}
    Eine \begriff{Deformationsretraktion} von $X$ auf $A \subset X$
    ist eine Homotopie
    $H\colon [0, 1] \times X \rightarrow X$ mit $H_0 = \id_X$ und
    $H_1$ eine Retraktion von $X$ auf $A$. \\
    $A \subset X$ heißt \begriff{Deformationsretrakt} von $X$, falls es eine
    Deformationsretraktion von $X$ auf $A$ gibt.
\end{Def}

\begin{Def}{starke Deformationsretraktion}
    Eine \begriff{starke Deformationsretraktion} von $X$ auf $A \subset X$
    ist eine
    Deformationsretraktion $H$ mit $H_t|_A = \id_A$ für alle $t \in [0, 1]$. \\
    $A \subset X$ heißt \begriff{starker Deformationsretrakt} von $X$, falls es
    eine starke Deformationsretraktion von $X$ auf $A$ gibt.
\end{Def}

\begin{Bsp}
    $\sphere^n \subset \real^{n+1} \setminus \{0\}$ ist ein
    starker Def.retrakt. \\
    Für $a \in X$ ist $\{a\} \subset X$ ein Retrakt,
    und ein Def.retrakt genau dann, wenn $X$ zusz.bar ist. \\
    Ist $X$ nicht wegzsh. (z.\,B. $X = \real \setminus \{0\}$), dann ist
    $\{a\} \subset X$ ein Retrakt, aber kein Def.retrakt. \\
    Es gibt keine Retraktion $[0, 1] \rightarrow \{0, 1\}$.
\end{Bsp}

\pagebreak

\subsection{%
    Der Abbildungsgrad%
}

\begin{Bem}\\
    Für $k \in \integer$ kann man
    $\varphi_k\colon \sphere^1 \rightarrow \sphere^1$, $z \mapsto z^k$
    definieren, also
    $\varphi_k(\cos(t), \sin(t)) := (\cos(kt), \sin(kt))$. \\
    Anschaulich gesagt wickelt diese Abbildung die Kreislinie $k$-mal
    um den Nullpunkt. \\
    Dabei ist $\varphi_0$ konstant, $\varphi_1$ die Identität und
    $\varphi_{-1}$ die Spiegelung an der $x$-Achse. \\
    Allgemeiner ist $\phi_k\colon \real^{n+1} \rightarrow \real^{n+1}$
    definiert durch \\
    $\phi_k(r \cos(t), r \sin(t), x_3, \dotsc, x_{n+1}) :=
    (r \cos(kt), r \sin(kt), x_3, \dotsc, x_{n+1})$. \\
    Es gilt $|\phi_k(x)| = |x|$, d.\,h. man erhält die Einschränkung
    $\varphi_k := \phi_k|_{\sphere^n}\colon \sphere^n \rightarrow \sphere^n$.
\end{Bem}

\begin{Satz}{\name{Brouwer}-\name{Hopf}}
    Für $k \in \integer$ ist die Abbildung
    $\integer \rightarrow [\sphere^n, \sphere^n]$,
    $k \mapsto [\varphi_k]$ eine Bijektion, d.\,h. jede stetige Abbildung
    $f\colon \sphere^n \rightarrow \sphere^n$ ist zu genau einer
    Abbildung $\varphi_k$ homotop.
\end{Satz}

\begin{Def}{Umlaufzahl}\\
    Die Umkehrabbildung ist der \begriff{Abbildungsgrad}/die
    \begriff{Umlaufzahl}
    $\deg\colon [\sphere^n, \sphere^n] \bij \integer$,
    $[\varphi_k] \mapsto k$.
\end{Def}

\begin{Kor}
    Der Abbildungsgrad ist multiplikativ, d.\,h.
    $\deg(f \circ g) = \deg(f) \cdot \deg(g)$.
\end{Kor}

\linie

\begin{Kor}
    $\sphere^n$ ist nicht zusammenziehbar
    ($\sphere^n \not\simeq \{\ast\}$).
\end{Kor}

\begin{Kor}
    $\sphere^n \subset \dball^{n+1}$ ist kein Retrakt.
\end{Kor}

\begin{Bem}
    $\sphere^n \subset \dball^{n+1} \setminus \{0\}$ ist ein Retrakt.
\end{Bem}

\begin{Satz}{\name{Brouwer}scher Fixpunktsatz}\\
    Jede stetige Abbildung $f\colon \dball^n \rightarrow \dball^n$
    besitzt mindestens einen Fixpunkt, d.\,h.
    $\exists_{a \in \dball^n}\; f(a) = a$.
\end{Satz}

\linie

\begin{Def}{tangentiales Vektorfeld}
    Ein \begriff{tangentiales Vektorfeld} auf $\sphere^n$ ist eine
    stetige Abbildung $v\colon \sphere^n \rightarrow \real^{n+1}$
    mit $\innerproduct{v(x), x} = 0$ für alle $x \in \sphere^n$.
\end{Def}

\begin{Bsp}
    Sei $n = 2m - 1$, $m \in \natural$ ungerade.
    Dann ist $v(x_1, x_2, \dots, x_{2m-1}, x_{2m}) =$ \\
    $(x_2, -x_1, \dotsc, x_{2m}, -x_{2m-1})$ ein tangentiales Vektorfeld auf
    $\sphere^{2m-1}$, das nirgends verschwindet.
\end{Bsp}

\begin{Bem}
    Sind solche Vektorfelder auch für $n = 2m$ möglich?
\end{Bem}

\begin{Satz}{Satz vom gekämmten Igel}\\
    Jedes tangentiale Vektorfeld
    $v\colon \sphere^{2m} \rightarrow \real^{2m+1}$
    besitzt mindestens eine Nullstelle.
\end{Satz}

\begin{Lemma}{Grad linearer Abbildungen}
    Für $A \in \GL_{n+1}(\real)$ besitzt
    $f_A\colon \sphere^n \rightarrow \sphere^n$,
    $x \mapsto \frac{Ax}{\norm{Ax}}$ den Abbildungsgrad
    $\deg(f_A) = \sign(\det A) \in \{\pm 1\}$.
\end{Lemma}

\linie

\begin{Satz}{stetige Abbildungen von $\sphere^m$ nach
             $\sphere^n$ mit $m < n$ sind nullhomotop}\\
    Für $m < n$ ist jede stetige Abbildung
    $f\colon \sphere^m \rightarrow \sphere^n$ nullhomotop.
\end{Satz}

\begin{Kor}
    Für $m \not= n$ ist $\sphere^m \not\simeq \sphere^n$, d.\,h.
    insbesondere $\sphere^m \not\cong \sphere^n$.
\end{Kor}

\begin{Kor}
    Für $m \not= n$ ist $\real^m \not\cong \real^n$.
\end{Kor}

\begin{Satz}{Invarianz der Dimension}
    Seien $U \subset \real^m$ of"|fen, $V \subset \real^n$ of"|fen und
    $U, V \not= \emptyset$. \\
    Gilt $U \cong V$, so ist $m = n$.
\end{Satz}

\begin{Lemma}{Umgebungen der $0$}
    Ist $V \subset \real^n$ eine Umgebung der $0$ mit $V \cong \dball^m$,
    so gilt $n = m$.
\end{Lemma}

\pagebreak

\section{%
    Simpliziale Komplexe%
}

\subsection{%
    Simpliziale Komplexe%
}

\begin{Def}{Standard-Simplex}
    $\Delta^n := \{(t_0, \dotsc, t_n) \in \real^{n+1} \;|\;
    t_0, \dotsc, t_n \ge 0,\; t_0 + \dotsb + t_n = 1\}$ heißt
    \begriff{Standard-Simplex} der Dimension $n$ ($n \in \natural$).
\end{Def}

\begin{Bem}
    $\Delta^n$ ist die konvexe Hülle von $e_0, \dotsc, e_n \in \real^{n+1}$,
    wobei $(e_0, \dotsc, e_n)$ die kanonische Basis des $\real^{n+1}$ ist,
    d.\,h. $\Delta^0$ ist ein Punkt, $\Delta^1$ eine Strecke,
    $\Delta^2$ ein Dreieck, $\Delta^3$ ein Tetraeder usw.
    Da $\Delta^n$ kompakt und sternförmig bzgl. der $\varepsilon$-Umgebung
    einer ihrer Punkte ist, gilt $\Delta^n \cong \dball^n$.
\end{Bem}

\begin{Def}{af"|fin unabhängig}
    Sei $V$ ein $\real$-Vektorraum.
    Eine Familie $(v_0, v_1, \dotsc, v_n)$ in $V$ heißt
    \begriff{af"|fin unabhängig}, falls $v_1 - v_0, \dotsc, v_n - v_0$
    linear unabhängig sind.
\end{Def}

\begin{Def}{Simplex}
    Seien $V$ ein $\real$-Vektorraum und $(v_0, v_1, \dotsc, v_n)$
    af"|fin unabhängig. \\
    $\Delta = [v_0, v_1, \dotsc, v_n] :=
    \{t_0 v_0 + t_1 v_1 + \dotsb + t_n v_n \;|\; t \in \Delta^n\}$ heißt
    der von $(v_0, v_1, \dotsc, v_n)$ aufgespannte af"|fine
    \begriff{$n$-Simplex}.
    Die Punkte $v_0, v_1, \dotsc, v_n$ heißen \begriff{Ecken} des Simplex
    $\Delta$. \\
    Für die kanonische Basisvektoren vom $\real^{n+1}$ gilt
    $\Delta^n = [e_0, e_1, \dotsc, e_n]$.
\end{Def}

\begin{Def}{baryzentrische Koordinaten}
    Für jeden Punkt $x = t_0 v_0 + t_1 v_1 + \dotsb + t_n v_n$
    heißen die Koordinaten
    $(t_0, t_1, \dotsc, t_n)$ \begriff{baryzentrische Koordinaten}
    von $x$ bzgl. $(v_0, v_1, \dotsc, v_n)$. \\
    Die Abbildung
    $h\colon \Delta^n \rightarrow \Delta$, $t \mapsto \sum_{i=0}^n t_i v_i$
    ist eine Bijektion, \\
    d.\,h. die Koordinaten sind eindeutig.
\end{Def}

\begin{Satz}{Ecken, Dim. eindeutig}
    $v\colon [v_0, v_1, \dotsc, v_n] \mapsto \{v_0, v_1, \dotsc, v_n\}$,
    $\dim\colon [v_0, v_1, \dotsc, v_n] \mapsto n$ sind wohldefinierte
    Zuordnungen auf der Menge aller af"|finen Simplizes in
    einem Vektoraum $V$, d.\,h.
    jeder af"|fine Simplex $\Delta = [v_0, v_1, \dotsc, v_n]$ in $V$
    bestimmt eindeutig seine Eckenmenge.
\end{Satz}

\linie

\begin{Def}{Seite}
    Sei $\Delta = [v_0, v_1, \dotsc, v_n]$ ein $n$-Simplex.
    Für jede nicht-leere Teilmenge \\
    $F \subset \{v_0, v_1, \dotsc, v_n\}$ mit
    $d + 1$ Elementen heißt der $d$-Simplex
    $[F]$ \begriff{Seite} von $\Delta$ der
    \begriff{Dimension} $d$ und der \begriff{Kodimension} $n - d$.
    Eine Seite der Kodimension $\ge 1$ heißt \begriff{echt}.
\end{Def}

\begin{Def}{Rand, Inneres}
    Der \begriff{Rand} eines Simplex $\Delta$ ist
    die Vereinigung all seiner echten Seiten, d.\,h. $\partial \Delta :=
    \bigcup_{\emptyset \not= F \subsetneqq \{v_0, v_1, \dotsc, v_n\}} [F]$.
    Das \begriff{Innere} ist $\Int \Delta := \Delta \setminus \partial \Delta$.
\end{Def}

\begin{Bem}
    Das Innere des Simplex $\Delta = [v_0, v_1, \dotsc, v_n]$ besteht aus
    allen Punkten \\
    $x = t_0 v_0 + t_1 v_1 + \dotsb + t_n v_n$ mit
    $t_0 + t_1 + \dotsb + t_n = 1$ sowie $t_0, t_1, \dotsc, t_n > 0$.
    Der Rand besteht aus allen Punkten, für die mindestens eine baryzentrische
    Koordinate $t_k$ verschwindet.
\end{Bem}

\linie

\begin{Def}{af"|finer simplizialer Komplex}
    Sei $V$ ein $\real$-Vektorraum.
    Ein \begriff{(af"|finer) simplizialer Komplex} in $V$ ist eine Menge
    $\K$ von Simplizes in $V$, sodass
    \begin{enumerate}
        \item
        für alle Simplizes $\Delta \in \K$
        auch alle Seiten von $\Delta$ ein Element von $\K$ sind und

        \item
        für alle Simplizes $\Delta_1, \Delta_2$ mit Durchschnitt
        $\Delta := \Delta_1 \cap \Delta_2 \not= \emptyset$
        gilt, dass $\Delta$ eine gemeinsame Seite ist
        (d.\,h. eine Seite sowohl von $\Delta_1$ als auch $\Delta_2$).
    \end{enumerate}
    Die Vereinigung $|\K| := \bigcup_{\Delta \in \K} \Delta$ heißt
    \begriff{Träger} von $\K$. \\
    Die \begriff{(af"|fine) Dimension} von $\K$ ist
    $\dim \K := \sup\{\dim \Delta \;|\; \Delta \in \K\}$.
\end{Def}

\begin{Bsp}
    Ist $\Delta$ ein af"|finer $n$-Simplex, dann bildet die Menge $\K$ aller
    Seiten von $\Delta$ einen simplizialen Komplex der Dimension $n$
    mit Träger $|\K| = \Delta$.
    Die Menge $\L$ aller echten Seiten von $\Delta$ bildet einen simplizialen
    Komplex der Dimension $n - 1$ mit Träger $|\L| = \partial \Delta$.
\end{Bsp}

\pagebreak

\begin{Def}{simpliziale Topologie}
    Sei $\K$ ein simplizialer Komplex in $V$.
    Jeder Simplex $\Delta \in \K$ wird mit seiner euklidischen Topologie
    ausgestattet, sodass $h\colon \Delta^n \rightarrow \Delta$
    ein Homöomorphismus ist.
    $|\K|$ wird mit der finalen Topologie ausgestattet, d.\,h.
    $U \subset |\K|$ ist of"|fen in $|\K|$ genau dann, wenn $U \cap \Delta$
    of"|fen in $\Delta$ ist für alle $\Delta \in \K$.
    Dies heißt \begriff{simpliziale Topologie} auf $|\K|$.
\end{Def}

\begin{Bem}
    Für $\K$ (lokal-)endlich in einem topologischen Vektorraum $V$ stimmen
    simpliziale Topologie und Teilraumtopologie überein.
\end{Bem}

\linie

\begin{Bem}
    Ein af"|finer Simplex $\Delta = [v_0, \dotsc, v_n]$ in einem Vektorraum $V$
    ist durch seine Eckenmenge $v(\Delta) = \{v_0, \dotsc, v_n\}$ festgelegt.
    Ein af"|finer simplizialer Komplex $\K$ in $V$ ist durch seine Simplizes
    $\Delta \in \K$ festgelegt.
    Zu seiner Beschreibung reicht es also aus, die Familie
    $K = v(\K) := \{v(\Delta) \;|\; \Delta \in \K\}$
    aller Eckenmengen anzugeben.
\end{Bem}

\begin{Def}{kombinatorischer simplizialer Komplex}\\
    Eine Familie $K$ endlicher nicht-leerer Mengen heißt
    \begriff{kombinatorischer simplizialer Komplex},
    falls für alle $S \in K$ und $\emptyset \not= S' \subset S$ auch
    $S' \in K$ gilt. \\
    In diesem Fall heißt $S = \{s_0, \dotsc, s_n\}$
    \begriff{kombinatorischer Simplex}
    der \begriff{Dimension} $\dim S := n$. \\
    $\Omega(K) := \bigcup_{S \in K} S$ heißt die \begriff{Eckenmenge} von $K$,
    ihre Elemente heißen \begriff{Ecken}.
\end{Def}

\linie

\begin{Def}{Darstellung}\\
    Eine \begriff{Darstellung} von $K$ in einem Vektorraum $V$ ist
    eine Abbildung $f\colon \Omega(K) \rightarrow V$, sodass
    \begin{enumerate}
        \item
        für alle $S \in K$ das Bild $f(S)$ af"|fin unabhängig in $V$ ist und

        \item
        für alle $S, T \in K$ gilt $[f(S)] \cap [f(T)] = [f(S \cap T)]$.
    \end{enumerate}
    In diesem Fall ist $\K = \{[f(S)] \;|\; S \in K\}$
    ein af"|finer simplizialer Komplex in $V$. \\
    $|K|_f := |\K|$ heißt die \begriff{topologische Realisierung} von $K$
    mittels $f$.
\end{Def}

\begin{Bem}
    Diese Bedingungen gelten insbesondere dann, wenn die Vektoren
    $(f(s))_{s \in \Omega}$ linear unabhängig sind.
\end{Bem}

\begin{Def}{kanonische Realisierung}\\
    Sei $K$ ein kombinatorischer simplizialer Komplex mit
    Eckenmenge $\Omega$.
    In der Menge $\real^{(\Omega)}$ aller Abbildungen
    $g\colon \Omega \rightarrow \real$ mit endlichem Träger
    (d.\,h. $\supp(g) = \{x \in \Omega \;|\; g(x) \not= 0\}$ ist endlich)
    definiert man die \begriff{kanonische Basis}
    $(\delta_s)_{s \in \Omega}$ mit $\delta_s\colon \Omega \rightarrow \real$,
    $\delta_s(t) := \delta_{st}$ (Kronecker-Delta). \\
    Die Abbildung $f\colon \Omega \rightarrow \real^{(\Omega)}$,
    $f(s) = \delta_s$ ist eine Darstellung von $K$, die
    \begriff{kanonische Darstellung}. \\
    Der so definierte Komplex $\K := \{[f(S)] \;|\; S \in K\}$ heißt
    \begriff{kanonischer af"|finer Komplex} von $K$. \\
    Der Raum $|K| := |\K|$ mit der simplizialen Topologie heißt
    \begriff{kanonische Realisierung} von $K$.
\end{Def}

\begin{Bem}
    Eine andere Schreibweise ist
    $|K| = \{x\colon \Omega \rightarrow [0, 1] \;|\;
    \supp(x) \in K,\; \sum_{s \in \Omega} x(s) = 1\}$. \\
    Jedem kombinatorischen Simplex $S \in K$ entspricht der af"|fine Simplex \\
    $|S| = \{x \in |K| \;|\; \supp(x) \subset S\}$, d.\,h.
    $\K = \{|S| \;|\; S \in K\}$.
\end{Bem}

\begin{Satz}{jede Real. ist zur kanon. Real. homöomorph}
    Seien $K$ ein kombinatorischer simplizialer Komplex mit Eckenmenge $\Omega$
    und $f\colon \Omega \rightarrow V$ eine Darstellung in einen Vektorraum
    $V$. \\
    Dann ist die Abbildung $h\colon |K| \homoe |K|_f$,
    $h(x) = \sum_{s \in \Omega} x(s) f(s)$ ein Homöomorphismus.
\end{Satz}

\linie
\pagebreak

\begin{Def}{kombinatorische simpliziale Abbildung}
    Seien $K$ und $L$ kombinatorische simpliziale Komplexe.
    Eine \begriff{kombinatorische simpliziale Abbildung}
    $f\colon K \rightarrow L$ ist eine Abbildung \\
    $f\colon \Omega(K) \rightarrow \Omega(L)$ der Eckenmengen, sodass
    für jeden Simplex $S \in K$ auch $f(S) \in L$ gilt.
\end{Def}

\begin{Def}{af"|fine simpliziale Abbildung}
    Seien $\K$ und $\L$ af"|fine simpliziale Komplexe.
    Eine \begriff{af"|fine simpliziale Abbildung}
    $g\colon \K \rightarrow \L$ ist eine Abbildung
    $g\colon \Omega(\K) \rightarrow \Omega(\L)$ der Eckenmengen,
    af"|fin fortgesetzt auf jeden Simplex, sodass
    für jeden Simplex $\Delta \in \K$ auch $g(\Delta) \in \L$ gilt.
\end{Def}

\begin{Bem}
    Seien $\K$ und $\L$ af"|fine simpliziale Komplexe sowie
    $K = v(\K)$ und $L = v(\L)$ die zugehörigen kombinatorischen simplizialen
    Komplexe. \\
    Jede kombinatorische simpliziale Abbildung $f\colon K \rightarrow L$
    definiert eine af"|fine simpliziale Abbildung $g\colon \K \rightarrow \L$
    durch $g(\sum_{s \in \Omega(K)} x(s) \cdot s) =
    \sum_{s \in \Omega(K)} x(s) \cdot f(s)$. \\
    Jede af"|fine simpliziale Abbildung $g\colon \K \rightarrow \L$ definiert
    eine kombinatorische simpliziale Abbildung $f\colon K \rightarrow L$
    durch Einschränkung auf die Eckenmengen.
\end{Bem}

\linie

\begin{Def}{kombinatorischer Teilkomplex}
    Sei $K$ ein kombinatorischer simplizialer Komplex.
    Ein \begriff{Teilkomplex} von $K$ ist eine Teilmenge $L \subset K$,
    die selbst ein komb. simplizialer Komplex ist.
\end{Def}

\begin{Def}{af"|finer Teilkomplex}
    Sei $\K$ ein af"|finer simplizialer Komplex.
    Ein \begriff{Teilkomplex} von $\K$ ist eine Teilmenge $\L \subset \K$,
    die selbst ein af"|finer simplizialer Komplex ist.
\end{Def}

\begin{Def}{$n$-Skelett}
    Sei $K$ ein simplizialer Komplex.
    Für $n \in \natural$ heißt
    $K_{\le n} := \{S \in K \;|\; \dim S \le n\}$ \\
    \begriff{$n$-Skelett} von $K$
    (Teilkomplex von $K$ der Dimension $\le n$).
\end{Def}

\subsection{%
    Triangulierung topologischer Räume%
}

\begin{Def}{Triangulierung}
    Sei $X$ ein topologischer Raum.
    Eine \begriff{Triangulierung} von $X$ ist ein Paar $(K, h)$, wobei
    $K$ ein simplizialer Komplex und $h\colon |K| \homoe X$ ein
    Homöomorphismus ist. \\
    $X$ heißt \begriff{triangulierbar},
    falls es eine Triangulierung von $X$ gibt.
\end{Def}

\begin{Bsp}
    Jede diskrete Menge $X$ kann trianguliert werden durch
    $K = \{\{x\} \;|\; x \in X\}$ (Komplex der Dimension $0$). \\
    Komplexe der Dimension $1$ heißen \begriff{kombinatorische Graphen},
    dazu homöomorphe topologische Räume heißen \begriff{topologische Graphen}.
\end{Bsp}

\begin{Satz}{$\dball^n$ und $\sphere^{n-1}$ triangulierbar}
    $\dball^n$ und $\sphere^{n-1}$ sind triangulierbar.
\end{Satz}

\begin{Bem}
    Die top. Realisierung $|K|$ jedes simplizialen Komples $K$ ist
    lokal zusammenziehbar, d.\,h. topologische Räume, die nicht
    lokal zusammenziehbar sind, sind nicht triangulierbar.
\end{Bem}

\linie

\begin{Satz}{Invarianz der Dimension}\\
    Für simpliziale Komplexe $K$ und $L$ mit $|K| \cong |L|$ gilt
    $\dim K = \dim L$.
\end{Satz}

\begin{Def}{Dimension}
    Sei $X$ ein durch $|K| \cong X$ triangulierbarer topologischer Raum. \\
    Dann heißt $\dim X := \dim K$ seine \begriff{(simpliziale) Dimension}.
\end{Def}

\begin{Bsp}
    Es gilt $\dim \dball^n = n$ und $\dim \sphere^n = n$.
\end{Bsp}

\pagebreak

\subsection{%
    Simpliziale Approximation%
}

\begin{Def}{simpliziale Metrik}
    Sei $K$ ein kombinatorischer simplizialer Komplex mit Eckenmenge $\Omega$.
    Auf der kanonischen Realisierung $|K| \subset \real^{(\Omega)}$
    ist die \begriff{simpliziale Metrik} definiert durch \\
    $d(x, y) := \max\{|x(s) - y(s)| \;|\; s \in \Omega\}$.
\end{Def}

\begin{Satz}{Vergleich mit metrischer Topologie}
    Die metrische Topologie auf $|K|$ ist gröber als die simpliziale Topologie.
    Ist $K$ (lokal-)endlich, so stimmen beide Topologien überein.
\end{Satz}

\begin{Kor}
    Für jeden simplizialen Komplex $K$ ist die Realisierung $|K|$ hausdorffsch.
\end{Kor}

\linie

\begin{Def}{Stern}
    Seien $K$ ein simplizialer Komplex und $|K|$ seine kanonische
    Realisierung. \\
    Für jede Ecke $a \in \Omega$ ist $\st(a) := \{x \in |K| \;|\; x(a) > 0\}$
    der \begriff{Stern} um $a$.
\end{Def}

\begin{Bem}
    Es gilt $\st(a) = B(a, 1) = \bigcup_{S \in K,\; a \in S} (\Int |S|)
    = |K| \setminus \{|T| \;|\; T \in K,\; a \notin T\}$.
\end{Bem}

\begin{Satz}{Stern of\!\! fen und zusammenziehbar}\\
    Für jede Ecke $a \in \Omega$ und jeden Radius $r$ mit $0 < r \le 1$ ist
    $B(a, r) = \{x \in |K| \;|\; x(a) > 1 - r\}$
    eine zusammenziehbare of"|fene Umgebung von $a$ in $|K|$,
    d.\,h. insbesondere auch $\st(a) = B(a, 1)$.
\end{Satz}

\linie

\begin{Def}{baryzentrische Unterteilung}
    Sei $K$ ein kombinatorischer Simplex.
    Dann heißt \\
    $\beta K := \{\{S_0, S_1, \dotsc, S_n\} \subset K \;|\;
    S_0 \subsetneqq S_1 \subsetneqq \dotsb \subsetneqq S_n\}$
    \begriff{baryzentrische Unterteilung} von $K$.
\end{Def}

\begin{Bem}\\
    $\beta K$ ist ein kombinatorischer Komplex, dessen Ecken genau die
    Simplizes von $K$ sind. \\
    Man kann $\beta K$ auf $|\K|$ wie folgt realisieren:
    Für $S \in K$ wähle man $\mu(S) \in \Int |S|$
    (z.\,B. für $S = \{s_0, \dotsc, s_n\}$ den Mittelpunkt
    $\mu(S) = \frac{1}{n + 1} s_0 + \dotsb + \frac{1}{n + 1} s_n$).
    Die Abbildung $\mu\colon \Omega(\beta K) = K \rightarrow |K|$ ist
    eine Darstellung von $\beta K$ und induziert einen Homöomorphismus
    $h\colon |\beta K| \homoe |K|$.
\end{Bem}

\begin{Kor}
    In $|K|$ ist jeder Punkt $a$ starker Deformationsretrakt einer of"|fenen
    Umgebung.
\end{Kor}

\begin{Kor}
    Sei $K$ ein simplizialer Komplex.
    Die topologische Realisierung $|K|$ ist kompakt genau dann, wenn
    $K$ endlich ist.
\end{Kor}

\linie

\begin{Lemma}{simpliziale Approximation}
    Seien $f\colon |K| \rightarrow |L|$ eine stetige Abbildung und \\
    $\varphi\colon \Omega(K) \rightarrow \Omega(L)$ eine Abbildung,
    sodass $f(\st(a)) \subset \st(\varphi(a))$ für alle $a \in \Omega(K)$ ist.
    Dann gilt:
    \begin{enumerate}
        \item
        Die Abbildung $\varphi$ ist simplizial, d.\,h.
        für alle $S \in K$ gilt $\varphi(S) \in L$.

        \item
        Die topologische Realisierung $g\colon |K| \rightarrow |L|$ von
        $\varphi\colon K \rightarrow L$ erfüllt: \\
        Für jedes $x \in |K|$ liegen $g(x)$ und $f(x)$ in einem gemeinsamen
        Simplex in $|L|$.

        \item
        Es gilt $g \simeq f$ durch
        $H(t, x) = (1 - t) \cdot g(x) + t \cdot f(x)$.
    \end{enumerate}
\end{Lemma}

\begin{Satz}{simpliziale Approximation}
    Seien $K$ und $L$ simpliziale Komplexe, wobei $K$ endlich ist. \\
    Dann ist jede Abbildung $f\colon |K| \rightarrow |L|$ homotop zu einer
    simplizialen Abbildung \\
    $g\colon |K| = |\beta^n K| \rightarrow |L|$ für $n$ genügend groß.
\end{Satz}

\begin{Kor}
    Jede stetige Abbildung $f\colon \sphere^m \rightarrow \sphere^n$
    mit $m < n$ ist nullhomotop.
\end{Kor}

\pagebreak

\subsection{%
    \name{Euler}-Charakteristik%
}

\begin{Bem}
    Gegeben sei ein endlicher simplizialer Komplex $K$.
    Gesucht wird eine topologische Invariante $I(K)$, z.\,B. eine ganze Zahl,
    sodass aus $|K| \cong |L|$ stets $I(K) = I(L)$ folgt.
    Die Anzahl $a_i$ der $i$-Simplizes eignet sich dafür nicht, da bspw. die
    baryzentrischen Unterteilungen die Zahlen $a_0, a_1 \dotsc$ verändern.
\end{Bem}

\begin{Def}{\name{Euler}-Charakteristik}
    Sei $K$ ein kombinatorischer simplizialer Komplex, der endlich ist. \\
    Dann heißt $\chi(K) := \sum_{S \in K} (-1)^{\dim S}$
    \begriff{\name{Euler}-Charakteristik}, d.\,h. \\
    $\chi(K) = +\; \text{Anzahl 0-Simplizes (Ecken)}
    \;-\; \text{Anzahl 1-Simplizes (Kanten)}$ \\
    $\;+\; \text{Anzahl 2-Simplizes (Dreiecke)}
    \;-\; \text{Anzahl 3-Simplizes (Tetraeder)} + \dotsb$.
\end{Def}

\begin{Satz}{\name{Euler}-Charakteristik von $D^n$, $S^n$}
    Es gilt $\dball^n \cong |D^n|$ und $\sphere^n \cong |S^n|$ mit \\
    $D^n := P(\{0, \dotsc, n\}) \setminus \{\emptyset\}$ und
    $S^n := D^{n+1} \setminus \{\{0, \dotsc, n, n + 1\}\}$. \\
    Dabei ist $\chi(D^n) = 1$ und $\chi(S^n) = 1 + (-1)^n$ für alle
    $n \in \natural$.
\end{Satz}

\begin{Satz}{Teilkomplexe}
    Seien $K$ ein endlicher simplizialer Komplex und $A, B$
    Teilkomplexe von $K$.
    Dann sind auch $A \cap B$ und $A \cup B$ Teilkomplexe und es gilt
    $\chi(A \cup B) = \chi(A) + \chi(B) - \chi(A \cap B)$.
\end{Satz}

\linie

\begin{Satz}{\name{Euler}scher Polyedersatz}\\
    Jede Triangulierung der Sphäre $\sphere^2$ hat Euler-Charakteristik $2$.
\end{Satz}

\begin{Bsp}
    Bspw. haben die Triangulierungen regelmäßiger Oktaeder und
    regelmäßiger Ikosaeder die Euler-Charakteristiken
    $\chi(\text{Oktaeder}) = 6 - 12 + 8 = 2$ und
    $\chi(\text{Ikosaeder}) = 12 - 30 + 20 = 2$.
\end{Bsp}

\linie

\begin{Satz}{\name{Euler}-Charakteristik Homöomorphie-invariant}\\
    Seien $K$ und $L$ endliche simpliziale Komplexe.
    Aus $|K| \cong |L|$ folgt $\chi(K) = \chi(L)$.
\end{Satz}

\begin{Satz}{\name{Euler}-Charakteristik Homotopie-invariant}\\
    Seien $K$ und $L$ endliche simpliziale Komplexe.
    Aus $|K| \simeq |L|$ folgt $\chi(K) = \chi(L)$.
\end{Satz}

\begin{Def}{\name{Euler}-Charakteristik von top. Räumen}
    Sei $X$ ein topologischer Raum. \\
    Ist $X$ homöomorph (oder auch nur homotopie-äquivalent) zur Realisierung
    $|K|$ eines endlichen simplizialen Komplexes $K$, dann heißt
    $\chi(X) := \chi(K)$ \begriff{\name{Euler}-Charakteristik} von $X$.
\end{Def}

\begin{Bsp}
    Es gilt $\chi(\dball^n) = 1$ und $\chi(\sphere^n) = 1 + (-1)^n$.
\end{Bsp}

\pagebreak

\section{%
    Flächen%
}

\subsection{%
    Topologische Mannigfaltigkeiten%
}

\begin{Def}{lokal euklidisch}
    Ein topologischer Raum $M$ heißt \begriff{lokal euklidisch} der
    Dimension $n$, falls es zu jedem Punkt $x \in M$ eine of"|fene Umgebung
    $U \subset M$ und einen Homöomorphismus $h\colon U \rightarrow V$
    mit $V \subset \real^n$ of"|fen gibt.
\end{Def}

\begin{Bsp}
    $M$ ist diskret genau dann, wenn $M$ lokal euklidisch der Dimension $0$
    ist. \\
    Jede of"|fene Menge $M \subset \real^n$ ist lokal euklidisch der
    Dimension $n$. \\
    $\sphere^n \subset \real^{n+1}$ ist lokal euklidisch der Dimension $n$
    (mithilfe der stereographischen Projektion). \\
    $\dball^n \subset \real^n$ ist nicht lokal euklidisch.
\end{Bsp}

\begin{Bem}
    Aus lokal euklidisch folgt nicht hausdorffsch.
    Ein Gegenbeispiel ist die Gerade mit doppeltem Ursprung
    (lokal euklidisch der Dimension $1$, aber nicht hausdorffsch).
\end{Bem}

\linie

\begin{Def}{Mannigfaltigkeit}
    Für $n \in \natural$ sei
    $\real^n_+ := \{(x_1, \dotsc, x_n) \in \real^n \;|\; x_1 \ge 0\}$, \\
    $\partial \real^n_+ := \{(x_1, \dotsc, x_n) \in \real^n \;|\; x_1 = 0\}$
    und $\Int \real^n_+ := \{(x_1, \dotsc, x_n) \in \real^n
    \;|\; x_1 > 0\}$. \\
    Ein topologischer Raum $M$ heißt \begriff{$n$-Mannigfaltigkeit}, falls
    \begin{enumerate}
        \item
        $M$ hausdorffsch ist und eine abzählbare Basis besitzt und

        \item
        es zu jedem Punkt $x \in M$ eine of"|fene Umgebung $U \subset M$ und
        einen Homöomorphismus $h\colon U \rightarrow V$ gibt mit
        $V \subset \real^n_+$ of"|fen
        ($h$ heißt dann \begriff{lokale Karte} von $M$).
    \end{enumerate}
    Gilt dabei $h(x) \in \Int \real^n_+$,
    dann heißt $x$ \begriff{innerer Punkt von $M$} ($x \in \Int M$),
    gilt stattdessen $h(x) \in \partial \real^n_+$,
    dann heißt $x$ \begriff{Randpunkt von $M$} ($x \in \partial M$).
\end{Def}

\begin{Def}{of"|fene/geschlossene Mannigfaltigkeit}\\
    Eine $n$-Mannigfaltigkeit mit $\partial M = \emptyset$ heißt
    \begriff{$n$-Mannigfaltigkeit ohne Rand}.
    Eine $n$-Mannigfaltig\-keit ohne Rand heißt \begriff{geschlossen},
    falls $M$ kompakt ist, und \begriff{of"|fen}, falls $M$ nicht kompakt ist.
\end{Def}

\begin{Bsp}
    $M$ ist diskret und abzählbar genau dann, wenn $M$ eine
    $0$-Mannigfaltigkeit ist. \\
    $\ball^n \subset \real^n$
    ($\Int \ball^n = \ball^n$, $\partial \ball^n = \emptyset$), \qquad
    $\dball^n \subset \real^n$
    ($\Int \dball^n = \ball^n$, $\partial \dball^n = \sphere^{n-1}$) und \\
    $\sphere^n \subset \real^{n+1}$
    ($\Int \sphere^n = \sphere^n$, $\partial \sphere^n = \emptyset$)
    sind $n$-Mannigfaltigkeiten. \\
    $\emptyset$ ist eine $n$-Mannigfaltigkeit für alle $n \in \natural$.
\end{Bsp}

\linie

\begin{Satz}{Eindeutigkeit der Dimension}\\
    Ist $M \not= \emptyset$ sowohl $m$- als auch $n$-Mannigfaltigkeit,
    dann gilt $m = n$.
\end{Satz}

\begin{Def}{Dimension}
    Sei $M \not= \emptyset$ eine $n$-Mannigfaltigkeit. \\
    Dann heißt $\dim M := n$ die \begriff{Dimension} von $M$.
\end{Def}

\linie

\begin{Satz}{Disjunktheit von Innerem und Rand}\\
    Ist $M$ eine $n$-Mannigfaltigkeit, dann gilt
    $\Int M \cap \partial M = \emptyset$.
\end{Satz}

\begin{Satz}{Inneres/Rand als Mannigfaltigkeit}
    Für jede $n$-Mannigfaltigkeit $M \not= \emptyset$ gilt: \\
    $\Int M \not= \emptyset$ und $\Int M$ ist eine $n$-Mfkt. ohne Rand.
    $\partial M$ ist eine $n - 1$-Mfkt. ohne Rand. \\
    $\Int M \subset M$ ist of"|fen und
    $\partial M \subset M$ ist abgeschlossen.
    Aus $M$ kompakt folgt $\partial M$ kompakt.
\end{Satz}

\begin{Satz}{Produktmannigfaltigkeit}
    Sind $M$ bzw. $N$ $m$- bzw. $n$-Mannigfaltigkeiten, so ist
    $M \times N$ eine $m + n$-Mannigfaltigkeit mit
    $\partial (M \times N) = (\partial M \times N) \cup (M \times \partial N)$.
\end{Satz}

%\linie
%\pagebreak

%\begin{Def}{Atlas}
    %Ein \begriff{Atlas} einer $n$-Mannigfaltigkeit $M$ ist eine Familie
    %$\A = (h_i\colon U_i \homoe V_i)_{i \in I}$ von Karten, die $M$ überdecken
    %(d.\,h. $U_i \subset M$ of"|fen, $V_i \subset \real^n_+$ of"|fen,
    %$M = \bigcup_{i \in I} U_i$).
%\end{Def}

%\begin{Def}{Kartenwechsel-Homöomorphismus}\\
    %Zu je zwei Karten $h_i$, $h_j$ ($i, j \in I$) sei
    %$U_{ij} := U_i \cap U_j$ sowie
    %$V_{ij} = h_i(U_{ij})$ und $V_{ji} = h_j(U_{ij})$. \\
    %Die Abbildungen $h_{ij}\colon V_{ij} \homoe V_{ji}$,
    %$h_{ij} := h_j \circ h_i^{-1}|_{V_{ij}}$ und $h_{ji}$ (analog)
    %heißen \begriff{Kartenwechsel-Ho\-möomorphismen}.
    %Der Fall $U_{ij} = \emptyset$ ist dabei nicht ausgeschlossen.
%\end{Def}

%\begin{Def}{orientierbar}
    %Ein Atlas $\A$ einer $n$-Mannigfaltigkeit $M$ heißt \begriff{orientiert},
    %falls alle Kartenwechsel zwischen ?????????????????
    %$M$ heißt \begriff{orientierbar}, falls ein orientierter Atlas $\A$ auf
    %$M$ existiert.
    %Ist $\A$ orientiert, dann kann man zu einem maximalen orientierten Atlas
    %übergehen.
    %Eine Orientierung von $M$ ist ein maximaler orientierter Atlas $\A$ von
    %$M$.
%\end{Def}

%\begin{Bsp}
    %Die Kreislinie $\sphere^1$ ist eine orientierbare $1$-Mannigfaltigkeit. \\
    %Der Zylinder $\sphere^1 \times [0, 1]$ ist eine orientierbare
    %$2$-Mannigfaltigkeit. \\
    %Das Möbius-Band ist eine nicht-orientierbare $2$-Mannigfaltigkeit.
%\end{Bsp}

\pagebreak

\subsection{%
    Beispiele und Klassifikationssätze%
}

\begin{tabular}{l|cc}
    & ohne Rand & mit Rand \\ \hline
    kompakt & $\sphere^1$ & $[0, 1]$ \\
    nicht kompakt & $\real$ & $\left[0, 1\right[$
\end{tabular}

\begin{Satz}{Klassifikation der $1$-Mannigfaltigkeiten}
    Jede zusammenhängende $1$-Mannigfaltigkeit ist homöomorph zu genau einer
    dieser Repräsentanten.
\end{Satz}

\linie

\begin{Def}{Fläche}
    Eine \begriff{Fläche} ist eine $2$-Mannigfaltigkeit.
\end{Def}

\begin{Def}{geschlossene Fläche}
    Man startet mit der $2$-Sphäre $F_0 := \sphere^2$ und dem Einheitstorus \\
    $F_1 := \sphere^1 \times \sphere^1$.
    Anschließend verklebt man für $g \ge 1$ die Flächen $F_g$ und $F_1$ zu
    einer neuen Fläche $F_{g+1}$ mit $g + 1$ Löchern.
    $F_g$ heißt \begriff{orientierbare geschlossene Fläche vom
    Geschlecht $g$}. \\
    Identifiziert man in $F_g$ gegenüberliegende Punkte paarweise miteinander,
    so erhält man die \begriff{nicht-orientierbare geschlossene Fläche
    $N_g := F_g/\pm 1$ vom Geschlecht $g$}. \\
    Für $g = 1$ erhält man den projektiven Raum
    $\real\projective^2 = F_0/\pm 1$.
    $F_1/\pm 1$ ist die \begriff{\name{Klein}sche Flasche}.
\end{Def}

\begin{Satz}{Klassifikation der $2$-Mannigfaltigkeiten}
    Jede zusammenhängende geschlossene Fläche $F$
    ist homöomorph zu genau einer dieser
    Repräsentanten ($F_g$ oder $N_g$ für ein $g \in \natural$).
\end{Satz}

\pagebreak

\subsection{%
    Klassifikation geschlossener Flächen%
}

\begin{Def}{Modellflächen}
    Mit $Q_0 := [-2, 2] \times [-2, 2]$,
    $Q_1 := Q_0 \setminus ([-1, 1] \times [-1, 1])$ und \\
    $Q_g := \bigcup_{k=1}^g (Q_1 - 2 - 2g + 4k)$, $g \ge 2$
    werden kompakte Flächen mit Rand definiert
    ($Q_g$ ist ein Rechteck mit $g$ Löchern).
    Der Produktraum $H_g := Q_g \times [-1, 1]$ heißt
    \begriff{Henkelkörper vom Geschlecht $g$}
    ($3$-Mannigfaltigkeit mit Rand).
    $H_g \subset \real^3$ ist punktsymmetrisch, d.\,h. $-H_g = H_g$.
\end{Def}

\begin{Def}{Orientierbarkeit im triangulierten Fall}
    Eine Mannigfaltigkeit heißt orientierbar, falls es eine Triangulierung
    gibt, sodass man jedem Dreieck eine Orientierung zuordnen kann, wobei
    jede Kante von den benachbarten Dreiecken gegenläufige Orientierungen erbt.
\end{Def}

\begin{Satz}{Rand der Modellflächen}
    Der Rand $F_g^+ := \partial H_g$ ist eine zusammenhängende
    geschlossene Fläche.
    Sie ist orientierbar und hat Euler-Charakteristik
    $\chi(F_g^+) = 2 - 2g$. \\
    Der Quotientenraum $F_g^- := F_g^+ / \{\pm\}$ ist ebenfalls eine
    zusammenhängende geschlossene Fläche.
    Sie ist nicht-orientierbar und hat Euler-Charakteristik
    $\chi(F_g^-) = 1 - g$.
\end{Satz}

\linie

\begin{Satz}{Klassifikationssatz}
    Jede zusammenhängende geschlossene Fläche $F$ ist homöomorph zu genau einer
    der Modellflächen $F_g^\pm$.
    Genauer gilt:
    \begin{itemize}
        \item
        Ist $F$ orientierbar ($\varepsilon := +$),
        dann ist $\chi(F) = 2 - 2g$ für ein $g \in \natural$.

        \item
        Ist $F$ nicht-orientierbar ($\varepsilon := -$),
        dann ist $\chi(F) = 1 - g$ für ein $g \in \natural$.
    \end{itemize}
    Allein aus diesen beiden Informationen folgt bereits die Homöomorphie
    $F \cong F_g^\varepsilon$.
\end{Satz}

\linie

\begin{Satz}{Triangulierbarkeit topologischer Flächen}\\
    Jede topologische Mannigfaltigkeit der Dimension $\le 3$ lässt sich
    triangulieren.
\end{Satz}

\begin{Satz}{triangulierte Flächen}\\
    Sei $K$ ein endlicher simplizialer Komplex.
    $|K|$ ist eine Fläche genau dann, wenn
    \begin{enumerate}
        \item
        jeder Simplex in einem $2$-Simplex enthalten ist,

        \item
        jeder $1$-Simplex in höchstens zwei $2$-Simplizes enthalten ist und

        \item
        für jede Ecke $a$ die $2$-Simplizes $\Delta_1, \dotsc, \Delta_k$,
        die $a$ enthalten, sich so anordnen lassen, dass jeweils
        $\Delta_i$ und $\Delta_{i+1}$ eine gemeinsame Kante haben.
    \end{enumerate}
\end{Satz}

\linie

\begin{Def}{Polygonmodell}
    Sei $n \in \natural$ mit $n \ge 2$.
    Die Kreislinie $\sphere^1 = \rand{\dball^2}$ wird in $n$ gleichlange
    Segmente $\gamma_k\colon [0,1] \rightarrow \sphere^1$ mit
    $\gamma_k(t) = \exp(\frac{2\pi\i}{n} (k - 1 + t))$, $k = 1, \dotsc, n$
    unterteilt. \\
    Sei $w = w_1 \dotsb w_n$ ein Wort über dem Alphabet
    $a^{\pm 1}, b^{\pm 1}, \dotsc$.
    Für $w_k = w_\ell$ wird $\gamma_k(t) \sim \gamma_\ell(t)$ für alle
    $t \in [0, 1]$ identifiziert,
    für $w_k = w_\ell^{-1}$ wird $\gamma_k(t) \sim \gamma_\ell(1 - t)$ für alle
    $t \in [0, 1]$ identifiziert. \\
    Dies erzeugt eine Äquivalenzrelation $\sim$.
    Der Quotientenraum ist
    $\dball^2 / \aufspann{w} := \dball^2 / \sim$.
\end{Def}

\begin{Bem}
    Ist $n \ge 3$, so kann man das Polygonmodell auch durch ein regelmäßiges
    $n$-Eck realisieren, an dessen Kanten die Buchstaben des Worts stehen.
\end{Bem}

\begin{Satz}{Polygonmodell geschlossener Flächen}
    Der Raum $\dball^2 / \aufspann{w}$ ist eine geschlossene Fläche genau dann,
    wenn jeder Buchstabe in $w$ genau zweimal vorkommt.
    In diesem Fall heißt $w$ \begriff{Flächenwort}.
    Tritt ein Buchstabe in $w$ zweimal mit gleichem Exponenten auf,
    dann ist $\dball^2 / \aufspann{w}$ nicht-orientierbar,
    andernfalls ist $\dball^2 / \aufspann{w}$ orientierbar.
\end{Satz}

\begin{Bsp}
    Der Raum $\dball^2 / \aufspann{a_1 b_1 a_1^{-1} b_1^{-1}
    \dotsb a_g b_g a_g^{-1} b_g^{-1}}$ ist eine zusammenhängende,
    orientierbare, geschlossene Fläche mit Euler-Charakteristik $2 - 2g$.
    Der Raum $\dball^2 / \aufspann{c_0 c_0 \dotsb c_g c_g}$
    ist eine zusammenhängende,
    nicht-orientierbare, geschlossene Fläche mit
    Euler-Charakteristik $1 - g$. \\
    Es gilt $\dball^2 / \aufspann{a a^{-1}} \cong \sphere^2$, \qquad
    $\dball^2 / \aufspann{a a} \cong \real\projective^2 =
    \sphere^2 / \{\pm 1\}$, \qquad
    $\dball^2 / \aufspann{a b a^{-1} b^{-1}} \cong \sphere^1 \times \sphere^1$
    und \\
    $\dball^2 / \aufspann{a b a b^{-1}} \cong
    (\sphere^1 \times \sphere^1) / \{\pm 1\}$
\end{Bsp}

\linie

\begin{Lemma}{zusammenhängende, geschlossene Fläche homöomorph
              zu einem Polygonmodell}\\
    Jede zusammenhängende, geschlossene Fläche ist homöomorph zu einem Raum
    $\dball^2 / \aufspann{w}$ für ein geeignetes Flächenwort $w$.
\end{Lemma}

\begin{Lemma}{Umformungen}
    Folgende Umformungen sind möglich ($\varepsilon, \delta \in \{\pm 1\}$):
    \begin{itemize}
        \item
        $\dball^2 / \aufspann{w_1 w_2 \dotsb w_n} \cong
        \dball^2 / \aufspann{w_2 \dotsb w_n w_1}$
        (zyklische Umordnung)

        \item
        $\dball^2 / \aufspann{\dotsb a^\varepsilon \dotsb
        a^\delta \dotsb} \cong
        \dball^2 / \aufspann{\dotsb b^\varepsilon \dotsb
        b^\delta \dotsb}$
        (wobei $a$ und $b$ sonst nicht vorkommen)

        \item
        $\dball^2 / \aufspann{\dotsb a^\varepsilon \dotsb
        a^\delta \dotsb} \cong
        \dball^2 / \aufspann{\dotsb a^{-\varepsilon} \dotsb
        a^{-\delta} \dotsb}$

        \item
        $\dball^2 / \aufspann{\dotsb a b b^{-1} c \dotsb} \cong
        \dball^2 / \aufspann{\dotsb a c \dotsb}$
        (Einklappen)

        \item
        $\dball^2 / \aufspann{\dotsb c \dotsb c \dotsb} \cong
        \dball^2 / \aufspann{\dotsb c c \dotsb}$
        (Zusammenfassen von \begriff{Kreuzhauben})

        \item
        $\dball^2 / \aufspann{\dotsb c c x \dotsb} \cong
        \dball^2 / \aufspann{\dotsb x c c \dotsb}$
        (Verschieben von Kreuzhauben)

        \item
        $\dball^2 / \aufspann{\dotsb a \dotsb b \dotsb a^{-1}
        \dotsb b^{-1} \dotsb} \cong
        \dball^2 / \aufspann{\dotsb a b a^{-1} b^{-1} \dotsb}$
        (Zusammenfassen von \begriff{Henkeln})

        \item
        $\dball^2 / \aufspann{\dotsb a b a^{-1} b^{-1} x \dotsb} \cong
        \dball^2 / \aufspann{\dotsb x a b a^{-1} b^{-1} \dotsb}$
        (Verschieben von Henkeln)
    \end{itemize}
\end{Lemma}

\begin{Satz}{Umformung in normalisierte Form}
    Mit obigen Umformungen kann jedes Flächenwort überführt werden in
    $w = c_1 c_1 \dotsb c_k c_k a_1 b_1 a_1^{-1} b_1^{-1} \dotsb
    a_\ell b_\ell a_\ell^{-1} b_\ell^{-1}$. \\
    Jede zusammenhängende geschlossene Fläche $F$ erfüllt demnach \\
    $F \cong \dball^2 /
    \aufspann{c_1 c_1 \dotsb c_k c_k a_1 b_1 a_1^{-1} b_1^{-1} \dotsb
    a_\ell b_\ell a_\ell^{-1} b_\ell^{-1}}$ für geeignete
    $k, \ell \in \natural$. \\
    Im Falle $k \ge 1$ kann man dies weiter vereinfachen zu
    $F \cong \dball^2 / \aufspann{c_1 c_1 \dotsb c_{k'} c_{k'}}$
    mit $k' = k + 2\ell$. \\
    Für den Fall $k = 0$ erhält man
    $F \cong \dball^2 / \aufspann{a_1 b_1 a_1^{-1} b_1^{-1} \dotsb
    a_\ell b_\ell a_\ell^{-1} b_\ell^{-1}}$.
\end{Satz}

\subsection{%
    Klassifikation kompakter Flächen mit Rand%
}

\begin{Def}{Modellflächen}
    Als Modell betrachtet man die Flächen $F_{g,r}^\pm$ mit $g \ge 0$ und
    $r \ge 1$, wobei
    $F_{g,r}^+$ ein Band mit $g$ angeklebten Paaren von ineinander
    verschränkten Bändern und $r - 1$ zusätzliche angeklebte Bänder sowie
    $F_{g,r}^-$ ein Band mit $g + 1$ angeklebten einmal verdrehten Bändern
    und $r - 1$ zusätzliche angeklebte Bänder.
\end{Def}

\begin{Satz}{Klassifikation kompakter Flächen mit Rand}
    Jede zusammenhängende, kompakte Fläche $M$ mit Rand
    $\partial M \not= \emptyset$ ist homöomorph zu genau einem der Modelle
    $F_{g,r}^\pm$.
\end{Satz}

\pagebreak
