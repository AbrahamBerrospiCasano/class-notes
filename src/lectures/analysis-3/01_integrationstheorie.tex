\chapter{%
    Elemente der Integrationstheorie%
}

\section{%
    Ring, Algebra und Maß%
}

Im Folgenden sei $X$ die Grundmenge und $2^X = P(X)$ bezeichne die Potenzmenge.

\textbf{Ring}:
$\emptyset \not= \mathcal{R} \subset 2^X$ heißt \emph{Ring}, falls für alle
$A, B \in \mathcal{R}$ gilt, dass $A \cup B \in \mathcal{R}$ sowie
$A \setminus B \in \mathcal{R}$.

\emph{Folgerung}:
Für $A, B \in \mathcal{R}$ ist auch
$A \setminus A = \emptyset \in \mathcal{R}$ sowie
$A \setminus (A \setminus B) = A \cap B \in \mathcal{R}$.

\textbf{$\sigma$-Ring}:
$\mathcal{R} \subset 2^X$ heißt \emph{$\sigma$-Ring}, falls $\mathcal{R}$
ein Ring ist sowie für jedes abzählbare System von Mengen
$\{A_k\}_{k \in \natural}$, $A_k \in \mathcal{R}$ gilt,
dass $\bigcup_{k \in \natural} A_k \in \mathcal{R}$.

\textbf{Algebra}:
$\mathcal{R} \subset 2^X$ heißt \emph{Algebra}, falls $\mathcal{R}$ ein Ring
ist sowie $X \in \mathcal{R}$.

\textbf{$\sigma$-Algebra}:
$\mathcal{R} \subset 2^X$ heißt \emph{$\sigma$-Algebra}, falls
$\mathcal{R}$ ein $\sigma$-Ring ist sowie $X \in \mathcal{R}$.

\emph{Folgerung}:
Es gilt $\bigcap_{k \in \natural} A_k =
A_1 \setminus \left(\bigcup_{k \in \natural} (A_1 \setminus A_k)\right)$,
d.\,h. für einen $\sigma$-Ring $\mathcal{R}$ und einem abzählbaren System
von Mengen $\{A_k\}_{k \in \natural}$, $A_k \in \mathcal{R}$ gilt, dass
$\bigcap_{k \in \natural} A_k \in \mathcal{R}$.

\linie

Nun seien $\mathcal{R}$ ein Ring und
$\varphi\colon \mathcal{R} \rightarrow \real^\ast :=
\{-\infty\} \cup \real \cup \{+\infty\}$ eine Funktion,
sodass höchstens einer der Werte $\pm \infty$ angenommen wird.
Eine solche Funktion heißt auch \textbf{Mengenfunktion}.

\textbf{additiv}:
$\varphi\colon \mathcal{R} \rightarrow \real^\ast$ heißt \emph{additiv}, falls
$\mathcal{R}$ ein Ring ist und
für alle $A, B \in \mathcal{R}$ mit $A \cap B = \emptyset$ gilt,
dass $\varphi(A \cup B) = \varphi(A) + \varphi(B)$.

\textbf{$\sigma$-additiv}:
$\varphi\colon \mathcal{R} \rightarrow \real^\ast$ heißt
\emph{$\sigma$-additiv}, falls $\mathcal{R}$ ein $\sigma$-Ring ist sowie für
jedes abzählbare System von Mengen $\{A_k\}_{k \in \natural}$,
$A_k \in \mathcal{R}$ mit $A_k \cap A_j = \emptyset$
für alle $k \not= j$ gilt, \\
dass $\varphi\left(\bigcup_{k \in \natural} A_k\right) =
\sum_{k=1}^\infty \varphi(A_k)$.

$A = \bigcup_{k \in \natural} A_k$ hängt nicht von der Reihenfolge der $A_k$
ab, d.\,h. $\sum_{k=1}^\infty \varphi(A_k)$ ist unabhängig von der Reihenfolge
der Summanden.
Damit konvergiert die Reihe nach dem Umordnungssatz von Riemann absolut,
wenn sie überhaupt konvergiert (ansonsten divergiert sie für
$\varphi \ge 0$ bestimmt).

\textbf{Eigenschaften additiver Funktionen}:
\begin{enumerate}
    \item
    $\varphi(\emptyset) = 0$, denn mit $A \in \mathcal{R}$ gilt
    $A \cap \emptyset = \emptyset$, also
    $\varphi(A) = \varphi(A \cup \emptyset) =
    \varphi(A) + \varphi(\emptyset)$ \\
    (wenn $\varphi(A)$ für mindestens ein $A \in \mathcal{R}$ endlich ist).

    \item
    Aus $A_1, \dotsc, A_n \in \mathcal{R}$, $A_j \cap A_k = \emptyset$ für alle
    $j \not= k$ folgt induktiv
    $\varphi\left(\bigcup_{k=1}^n A_k\right) = \sum_{k=1}^n \varphi(A_k)$.

    \item
    Es gilt $\varphi(A_1 \cap A_2) + \varphi(A_1 \cup A_2) =
    \varphi(A_1) + \varphi(A_2)$ für $A_1, A_2 \in \mathcal{R}$.
\end{enumerate}

\textbf{Eigenschaften nicht-negativer additiver Funktionen}:
\begin{enumerate}[resume]
    \item
    $\varphi(A_1 \cup A_2) \le \varphi(A_1) + \varphi(A_2)$, d.\,h.
    $\varphi\left(\bigcup_{k=1}^n A_k\right) \le \sum_{k=1}^n \varphi(A_k)$.

    \item
    Ist $A_1 \subset A_2$, so ist $\varphi(A_1) \le \varphi(A_2)$, denn \\
    $\varphi(A_2) = \varphi((A_2 \setminus A_1) \cup A_1) =
    \varphi(A_2 \setminus A_1) + \varphi(A_1) \ge \varphi(A_1)$.
\end{enumerate}

\textbf{Satz}:
Seien $\mathcal{R} \subset 2^X$ ein $\sigma$-Ring,
$\varphi\colon \mathcal{R} \rightarrow \real^\ast$ $\sigma$-additiv sowie
$A_k \in \mathcal{R}$ für $k \in \natural$ mit \\
$A_1 \subset A_2 \subset \dotsb \subset A_k \subset A_{k+1} \subset \dotsb$
und $A = \bigcup_{k \in \natural} A_k$. \\
Dann ist $\lim_{k \to \infty} \varphi(A_k) = \varphi(A)$.

\linie

\textbf{Maß}:
Ein \emph{Maß} ist eine auf einem $\sigma$-Ring $\mathcal{R}$ gegebene
nicht-negative, $\sigma$-additive Funktion
$\varphi\colon \mathcal{R} \rightarrow \real^\ast$.
Das Tripel $(X, \mathcal{R}, \varphi)$ heißt \emph{Maßraum}.
$A \subset X$ heißt \emph{messbar}, falls $A \in \mathcal{R}$.

Ein Beispiel für eine $\sigma$-Algebra auf $X \not= \emptyset$ ist
$\mathcal{R} = \{\emptyset, X\}$ oder auch $\mathcal{R} = 2^X$.
Im letzten Fall ist mit $\varphi(A) = \#\{x \in A\}$ (Anzahl der Elemente für
$A \subset X$ endlich, $+\infty$ für $A$ unendlich) ein Maß gegeben, das sog.
\textbf{Zählmaß}.

\section{%
    Zur Konstruktion des \name{Lebesgue}-Maßes%
}

Im Folgenden betrachtet man die Grundmenge $X = \real^d$.
Im $\real^d$ definiert man (abgeschlossene) \textbf{Quader}
$Q := \{x \in \real^d \;|\; a_j \le x_j \le b_j,\; j = 1, \dotsc, d\}$.
Analog kann man of"|fene bzw. halb\-of"|fene Quader definieren, falls man alle
bzw. manche Ungleichheitszeichen zu "`echt kleiner"' verstärkt.
Dies sind nicht alle Quader des $\real^d$, sondern genau solche, deren Kanten
und Flächen parallel zu den Koordinatenachsen sind.
Als "`Quader"' sollen nun alle of"|fenen, halbof"|fenen und abgeschlossenen
Quader $Q$ der obigen Form bezeichnet werden.

Falls $Q \not= \emptyset$, d.\,h. $a_j \le b_j$ für $j = 1, \dotsc, d$, so
definiert man $m(Q) := \prod_{j=1}^d (b_j - a_j)$
als \textbf{Volumen des Quaders}.
Man kann sehen, dass diese Definition absichtlich nicht berücksichtigt, ob
ein Quader of"|fen, halbof"|fen oder abgeschlossen ist
(diese Quader sollen dasselbe Volumen besitzen).

Eine \textbf{elementare Menge} ist nun eine Teilmenge $E \subset \real^d$,
die als endliche disjunkte Vereinigung von solchen Quadern dargestellt werden
kann.
$\mathcal{E}$ sei die Menge aller elementaren Mengen des $\real^d$, d.\,h.
$\mathcal{E} = \{E \subset \real^d \;|\; E \text{ ist elementare Menge}\}$.

\textbf{Eigenschaften von $\mathcal{E}$}:
$\mathcal{E}$ ist ein Ring. \\
$m$ ist wohldefiniert auf $\mathcal{E}$ durch
$m(E) = \sum_{i=1}^n m(Q_i)$ für $E \in \mathcal{E}$,
d.\,h. $E = \bigcup_{i=1}^n Q_i$ mit $Q_i$ paarweise disjunkt
(die Zerlegung ist nicht eindeutig, trotzdem ist $m$ wohldefiniert, also
unabhängig von der konkreten Darstellung von $E$ als
endliche disjunkte Vereinigung von Quadern). \\
Damit ist $m$ additiv auf $\mathcal{E}$.

\linie

\textbf{regulär}:
Eine erweiterte nicht-negative additive Funktion
$\varphi\colon \mathcal{E} \rightarrow \real^\ast$ heißt \emph{regulär}, falls
für alle $A \in \mathcal{E}$ und $\delta > 0$ Mengen $F = F_{\delta,A}$ und
$G = G_{\delta,A}$ existieren mit \\
$F, G \in \mathcal{E}$, \qquad
$F$ abgeschlossen, $G$ of"|fen, \qquad
$F \subset A \subset G$ \quad und \quad
$\varphi(G) - \delta \le \varphi(A) \le \varphi(F) + \delta$.

Es stellt sich heraus, dass $m$ eine reguläre Funktion ist.

Ein weiteres Beispiel ist für $d = 1$ und
eine beliebige, monoton steigende Funktion
$\alpha\colon \real \to \real$, $\alpha\ms$ die Funktion
$\mu\colon \mathcal{E} \rightarrow \real$ mit
$\mu(\left[a,b\right]) = \alpha(b + 0) - \alpha(a - 0)$,
$\mu(\left]a,b\right]) = \alpha(b + 0) - \alpha(a + 0)$,
$\mu(\left[a,b\right[) = \alpha(b - 0) - \alpha(a - 0)$ und
$\mu(\left]a,b\right[) = \alpha(b - 0) - \alpha(a + 0)$.
$\mu$ ist regulär auf $\mathcal{E}$ (über $\real$).
Die Grenzwerte existieren alle, da monotone Funktionen Unstetigkeiten höchstens
in Form von Sprungstellen haben.

Ist eine Menge $\{\mathcal{R}_\tau\}_{\tau \in T}$ von $\sigma$-Ringen
$\mathcal{R}_\tau \subset 2^X$ gegeben, so ist
$\mathcal{R} = \bigcap_{\tau \in T} \mathcal{R}_\tau$ ebenfalls ein
$\sigma$-Ring.
Insbesondere gibt es einen minimalen $\sigma$-Ring $\mathcal{R}$, der
$\mathcal{E}$ enthält.

\linie

\textbf{äußeres Maß $\mu^\ast$}:
Seien $\mu\colon \mathcal{E} \rightarrow \real^\ast$ regulär
sowie $E \subset \real^d$ beliebig. \\
Außerdem seien $\bigcup_{n=1}^\infty A_n \supset E$
($A_n \in \mathcal{E}$ of"|fen) abzählbare of"|fene Überdeckungen von $E$. \\
Dann ist mit $\mu^\ast(E) := \inf_{\bigcup_{n=1}^\infty A_n \supset E}
\left\{\sum_{n=1}^\infty \mu(A_n)\right\}$ das \emph{äußere Maß} von
$E$ definiert.

\textbf{Eigenschaften von $\mu^\ast$}:
$\mu^\ast$ ist monoton, d.\,h. für $E_1 \subset E_2$ gilt
$\mu^\ast(E_1) \le \mu^\ast(E_2)$. \\
Außerdem ist $\mu^\ast \ge 0$ nicht-negativ.

\textbf{Satz}:
Für $E \in \mathcal{E}$ ist $\mu^\ast(E) = \mu(E)$. \\
Ist $E = \bigcup_{n=1}^\infty E_n$ abzählbare Vereinigung von Mengen
$E_n \subset \real^d$, so ist
$\mu^\ast(E) \le \sum_{n=1}^\infty \mu^\ast(E_n)$, \\
d.\,h. $\mu^\ast$ ist \textbf{$\sigma$-subadditiv}.

Man sagt auch, $\mu^\ast$ ist eine
\textbf{$\sigma$-subadditive Fortsetzung} von
$\mu\colon \mathcal{E} \rightarrow \real^\ast$ auf $2^{\real^d}$.

\textbf{Lemma von \name{Heine}-\name{Borel}}:
Sei $(M, d)$ ein metrischer Raum.
Dann ist eine Teilmenge $F \subset M$ kompakt genau dann, wenn aus jeder
abzählbaren Überdeckung $F \subset \bigcup_{n=1}^\infty A_n$ von $F$
mit of"|fenen Mengen $A_n$ eine endliche Überdeckung
$F \subset \bigcup_{j=1}^m A_{n_j}$ ausgewählt werden kann.

\linie
\pagebreak

Sind $A$ und $B$ Mengen, so ist die symmetrische Dif"|ferenz
$A \symmdiff B := (A \setminus B) \cup (B \setminus A) =
(A \cup B) \setminus (A \cap B)$ definiert.
Ist also $\mathcal{R}$ ein Ring, so gilt für $A, B \in \mathcal{R}$ auch
$A \symmdiff B \in \mathcal{R}$.

Für $A, B \subset \real^d$ sei $d^\ast(A, B) := \mu^\ast(A \symmdiff B)$.
Außerdem führt man einen Konvergenzbegriff ein: \\
Für $A_n \subset \real^d$, $n \in \natural$ und $A \subset \real^d$ konvergiert
$A_n$ gegen $A$ ($A_n \xrightarrow{d^\ast} A$), falls $d^\ast(A_n, A) \to 0$.

\textbf{Eigenschaften von $d^\ast$}:
$d^\ast(A, B) = d^\ast(B, A)$ wegen $A \symmdiff B = B \symmdiff A$. \\
$d^\ast(A, A) = 0$ wegen $A \symmdiff A = \emptyset$ sowie
$d^\ast(A, B) \ge 0$. \\
$d^\ast(A, B) = \mu^\ast(A \symmdiff B) \le
\mu^\ast((A \symmdiff C) \cup (B \symmdiff C)) \le$ \\
$\mu^\ast(A \symmdiff C) + \mu^\ast(B \symmdiff C) =
d^\ast(A, C) + d^\ast(C, B)$
wegen $A \symmdiff B \subset (A \symmdiff C) \cup (B \symmdiff C)$.

Jedoch ist $d^\ast$ keine Metrik, da
$d^\ast(A, B) = 0 \;\not\Rightarrow\; A = B$! \qquad
Zum Beispiel für $B = A \cup \{x\}$ mit $A$ beliebige Menge, $x \notin A$ ist
$A \symmdiff B = \{x\}$, aber $d^\ast(A, B) = \mu^\ast(\{x\}) = 0$.

Also ist der obige Konvergenzbegriff nicht eindeutig, denn zwei Grenzwerte
einer Folge können sich um etwas unterscheiden, dessen äußeres Maß $0$ ist.

\linie

\textbf{Lemma}:
Für $\circ \in \{\cup, \cap, \setminus\}$ gilt aufgrund
$(A_1 \circ A_2) \symmdiff (B_1 \circ B_2) \subset
(A_1 \symmdiff B_1) \cup (A_2 \symmdiff B_2)$, dass
$d^\ast(A_1 \circ A_2, B_1 \circ B_2) \le
d^\ast(A_1, B_1) + d^\ast(A_2, B_2)$. \\
Außerdem gilt für $A, B \subset \real^d$, $\mu^\ast(A) < \infty$
die Ungleichung $|\mu^\ast(A) - \mu^\ast(B)| \le d^\ast(A, B)$.

\textbf{(endlich) $\mu$-messbar}:\\
$A \subset \real^d$ heißt \emph{endlich $\mu$-messbar}
($A \in \mathcal{M}_F(\mu)$), falls $A_n \xrightarrow{d^\ast} A$ mit einer Folge von
$A_n \in \mathcal{E}$. \\
$A \subset \real^d$ heißt \emph{$\mu$-messbar}
($A \in \mathcal{M}(\mu)$), falls $A = \bigcup_{n=1}^\infty A_n$ mit
$A_n \in \mathcal{M}_F(\mu)$.

\textbf{Satz}:
$\mathcal{M}(\mu)$ ist eine $\sigma$-Algebra.
$\mu^\ast$ ist $\sigma$-additiv auf $\mathcal{M}(\mu)$.

Im Beweis dieses Satzes wird zusätzlich
$A \in \mathcal{M}_F(\mu) \;\Leftrightarrow\;
A \in \mathcal{M}(\mu) \land \mu^\ast(A) < \infty$
gezeigt.

\linie

Nimmt man eine beliebige of"|fene, nicht-leere Menge $A \subset \real^d$,
$A \not= \emptyset$, so kann man für einen Punkt $x \in A$ eine
$\varepsilon$-Umgebung $U_\varepsilon(x) \subset A$ finden mit
$\varepsilon > 0$.
In diese $\varepsilon$-Umgebung kann man immer einen Würfel der Seitenlänge
$\varepsilon_1 > 0$ einbeschreiben.
Da $\rational^d$ in $\real^d$ dicht liegt, gibt es ein $r \in \rational^d$
mit $\norm{x - r} < \frac{\varepsilon_1}{4}$.
Im Würfel um $r$ mit Seitenlänge $\varepsilon_2 = \frac{\varepsilon_1}{2} > 0$
ist $x$ enthalten. \\
Also ist jede of"|fene Menge $A \subset \real^d$ darstellbar als
abzählbare Vereinigung von Würfeln. \\
Da Würfel elementare Mengen und somit (finit) messbar sind, sind
of"|fene als auch abgeschlossene Mengen $A \subset \real^d$ $\mu$-messbar,
d.\,h. $A \in \mathcal{M}(\mu)$.

\textbf{\name{Borel}-Algebra}:
Die \emph{\name{Borel}-Algebra} $\mathcal{B}$
ist die kleinste $\sigma$-Algebra, welche alle of"|fenen Mengen enthält.
Diese existiert, da z.\,B. $\mathcal{M}(\mu)$ eine $\sigma$-Algebra ist,
die alle of"|fenen Mengen enthält (s.\,o.). \\
Die Borel-Algebra enthält genau die \emph{\name{Borel}-Mengen}, das sind
die Mengen, die sich aus of"|fenen Mengen durch die Operationen
"`abzählbare Vereinigung"', "`abzählbarer Durchschnitt"' sowie "`Komplement"'
bilden lassen.
Sind z.\,B. $G_n$ of"|fen, $F_n$ abgeschlossen für $n \in \natural$, dann sind
$G^\sigma = \bigcap_{n=1}^\infty G_n$ und $F^\sigma = \bigcup_{n=1}^\infty F_n$
Borelmengen 1. Typs usw. \\
Die Borel-Algebra auf $X$ bezeichnet man auch mit $\mathcal{B}(X)$.

Es gilt $\mathcal{B} \subset \mathcal{M}(\mu)$, da ja $\mathcal{M}(\mu)$ eine
$\sigma$-Algebra ist, die alle of"|fenen Mengen enthält. \\
Allerdings gilt $\mathcal{B} \not= \mathcal{M}(\mu)$, d.\,h. es gibt
messbare Mengen, die keine Borel-Mengen sind.

\pagebreak

\section{%
    Messbare Funktionen%
}

Seien $(X, \mathcal{R}, \mu)$ ein Maßraum mit $\mathcal{R}$ $\sigma$-Algebra
und $f\colon X \rightarrow \real^\ast$ eine erweiterte Funktion. \\
Eine Menge $A \subset X$ heißt \emph{messbar}, falls $A \in \mathcal{R}$ ist.

\textbf{messbare Funktion}:
$f$ heißt \emph{messbar}, falls
$\forall_{a \in \real}\; E_a(f) := \{x \in X \;|\; f(x) > a\}$ messbar ist.

\textbf{Satz}:
$\forall_{a \in \real}\; \{x \in X \;|\; f(x) > a\}$ messbar \\
$\;\Leftrightarrow\;$
$\forall_{a \in \real}\; \{x \in X \;|\; f(x) \ge a\}$ messbar \\
$\;\Leftrightarrow\;$
$\forall_{a \in \real}\; \{x \in X \;|\; f(x) < a\}$ messbar \\
$\;\Leftrightarrow\;$
$\forall_{a \in \real}\; \{x \in X \;|\; f(x) \le a\}$ messbar

\textbf{Folgerung}:
Ist $f$ messbar und $I \subset \real$ ein Intervall,
so ist $f^{-1}(I)$ messbar.

Jede of"|fene Menge $A \subset \real$ ist eine disjunkte Vereinigung von
höchstens abzählbar vielen of"|fenen Intervallen $I_k$.
Also ist $f^{-1}(A) = \bigcup_{k=1}^\infty f^{-1}(I_k)$
messbar, falls $A$ of"|fen und $f$ messbar ist. \\
Daher ist $f^{-1}(A)$ messbar für Borel-Mengen $A \subset \real$.

\textbf{\name{Borel}-messbar}:
Eine Funktion $f\colon X \rightarrow \real^\ast$ heißt
\emph{\name{Borel}-messbar}, falls \\
$\forall_{a \in \real}\; E_a(f) = \{x \in X \;|\; f(x) > a\}
\in \mathcal{B}(X)$ ist. \\
Dies ist der Fall genau dann, wenn $f^{-1}(B) \in \mathcal{B}(X)$ für alle
$B \in \mathcal{B}(\real^\ast)$.

\linie

\textbf{Satz}:
Ist $f$ messbar, so ist auch $|f|$ messbar.

\textbf{Satz}:
Ist $\{f_n\}_{n \in \natural}$ eine Folge messbarer Funktionen, dann sind
$g$ und $h$ messbar mit \\
$g(x) = \sup_{n \in \natural} f_n(x)$ (auch $\inf$) und
$h(x) = \limsup_{n \to \infty} f_n(x)$ (auch $\liminf$).

\textbf{Folgerung}:
Ist $\{f_n\}_{n \in \natural}$ eine Folge messbarer Funktionen sowie
für alle $x \in X$ gibt es einen Grenzwert $f(x) = \lim_{n \to \infty} f_n(x)$,
dann ist $f$ messbar. \\
Ist $f$ messbar, dann sind $f^+$, $f^-$ und $-f$ messbar mit \\
$f^+(x) := \max\{0, f(x)\}$, $f^- := -\min\{0, f(x)\}$,
$f = f^+ - f^-$, $f^\pm \ge 0$.

\textbf{Satz}:
Sind $f$ und $g$ messbar mit $c \in \real$, so sind auch
$f + g$, $f - g$, $f \cdot g$ und $c \cdot f$ messbar. \\
Außerdem ist $\{x \in X \;|\; f(x) < g(x)\}$ messbar.

\textbf{Folgerung}:
Ist $p$ ein Polynom mit reellen Koef"|fizienten und $f$ messbar,
so ist auch $p(f(x))$ messbar.

\linie

\textbf{Treppenfunktion}:
Eine \emph{Treppenfunktion} (eigentlich \emph{einfache Funktion})
ist eine Funktion $t\colon X \rightarrow \real$
mit endlichem Wertevorrat, d.\,h. $t(X) = \{c_1, \dotsc, c_N\}$ mit
$c_j \not= c_k$ für $j \not= k$. \\
Für $E \subset X$ ist $\chi_E\colon X \rightarrow \real$ mit
$\chi_E(x) = 1$ für $x \in E$ und $\chi_E(x) = 0$
für $x \notin E$ die \emph{charakteristische Funktion} von $E$.
$E$ ist messbar genau dann, wenn $\chi_E$ messbar ist. \\
Eine Treppenfunktion $t$ kann als Linearkombination von charakteristischen
Funktionen \\
$t(x) = \sum_{k=1}^N c_i \chi_{E_i}(x)$ mit $E_i \subset X$
paarweise disjunkt dargestellt werden, nämlich \\
$E_i = \{x \in X \;|\; t(x) = c_i\} = t^{-1}(\{c_i\})$.
$t$ ist messbar genau dann, wenn alle $E_i$ messbar sind.

\textbf{Satz}:
Sei $f\colon X \rightarrow \real^\ast$ messbar. \\
Dann gibt es eine Folge von Treppenfunktionen $t_n$, sodass
$\forall_{x \in X}\; \lim_{n \to \infty} t_n(x) = f(x)$. \\
Für $f \ge 0$ kann man $t_n$ monoton wählen, d.\,h. für fixes $x \in X$
ist $t_n(x)$ monoton steigend für wachsendes $n$.

\pagebreak

\section{%
    Das \name{Lebesgue}-Integral%
}

Sei $(X, \mathcal{R}, \mu)$ ein Maßraum.
Ist $t(x) = \sum_{i=1}^n c_i \chi_{E_i}(x)$ eine messbare Treppenfunktion
und \\
$E \in \mathcal{R}$, so sei $I_E(t) := \sum_{i=1}^n c_i \mu(E_i \cap E)$
das Integral von $t$ über $E$ mit Maß $\mu$.

\textbf{\name{Lebesgue}-Integral}:
Sei $f\colon X \rightarrow \real^\ast$, $f \ge 0$ messbar. \\
Dann ist
$\int_E f d\mu := \sup_{t \text{ Treppenfkt.},\; 0 \le t \le f} I_E(t)$
das \emph{\name{Lebesgue}-Integral} von $f$ über $E$ mit Maß $\mu$. \\
Für $f\colon X \rightarrow \real^\ast$ messbar ist
$\int_E f d\mu := \int_E f^+ d\mu - \int_E f^- d\mu$, falls mindestens eines
der Integrale $\int_E f^\pm d\mu$ endlich ist.

\emph{Beispiel}: Für eine Treppenfunktion $f = t$ gilt
$\int_E f d\mu = I_E(f)$.

\textbf{\name{Lebesgue}-integrierbar}:
Sei $f\colon X \rightarrow \real^\ast$ messbar. \\
$f$ heißt \emph{\name{Lebesgue}-integrierbar} ($f \in \L(E, \mu)$), falls
$\int_E f^+ d\mu < \infty$ und $\int_E f^- d\mu < \infty$.

\linie

\textbf{\name{Lebesgue}-Maß}:
Für Quader $Q \subset \real^d$ ist das Volumen definiert als
$m(Q) = \prod_{j=1}^d (b_j - a_j)$. \\
$m$ lässt sich auf $\mathcal{E} \subset P(\real^d)$ ausweiten
(unabhängig von der konkreten Zerlegung). \\
Wegen $m \ge 0$ additiv und regulär ist das äußere Maß
$m^\ast: P(\real^d) \rightarrow \real^\ast$ auf
beliebigen Teil\-mengen $E \subset \real^d$ wohldefiniert.
$\mathcal{M}(m)$ ist eine $\sigma$-Algebra, auf der $m^\ast$ $\sigma$-additiv
ist. \\
Daher bildet $(\real^d, \mathcal{M}(m), m^\ast)$ einen Maßraum.
$m^\ast$ heißt \emph{\name{Lebesgue}-Maß} und die Mengen aus
der \emph{\name{Lebesgue}-Algebra} $\mathcal{M}(m)$
heißen \emph{\name{Lebesgue}-messbar}.

\linie

\textbf{Eigenschaften}:\vspace{-2mm}
\begin{tabbing}
    \hspace{9cm}\=\kill

    1. $f$ messbar, beschränkt, $\mu(E) < \infty$ \>
    $\Rightarrow\; f \in \L(E, \mu)$ \\[2mm]

    2. $f$ messbar, $\mu(E) < \infty$,
    $\forall_{x \in X}\; a \le f(x) \le b$ \>
    $\Rightarrow\; a \cdot \mu(E) \le \int_E f d\mu \le b \cdot \mu(E)$ \\[2mm]

    3. $f, g \in \L(E, \mu)$, $\forall_{x \in X}\; f(x) \le g(x)$ \>
    $\Rightarrow\; \int_E f d\mu \le \int_E g d\mu$ \\[2mm]

    4. $f \in \L(E, \mu)$, $c \in \real$ \>
    $\Rightarrow\; c \cdot f \in \L(E, \mu)$,
    $\int_E (c \cdot f) d\mu = c \cdot \int_E f d\mu$ \\[2mm]

    5. $f$ messbar, $\mu(E) = 0$ \>
    $\Rightarrow\; \int_E f d\mu = 0$ \\[2mm]

    6. $f \in \L(E, \mu)$, $A \in \mathcal{R}$, $A \subset E$ \>
    $\Rightarrow\; f \in \L(A, \mu)$
\end{tabbing}

\linie

\textbf{Satz ($\sigma$-Additivität des Integrals bzgl. des
Integrationsbereichs)}: \\
Seien $f\colon X \rightarrow \real^\ast$ messbar, $f \ge 0$
sowie $\varphi(A) := \int_A f d\mu$ für $A \in \mathcal{R}$. \\
Dann ist $\varphi\colon \mathcal{R} \rightarrow \real^\ast$
$\sigma$-additiv. \\
Für $f \in \L(X, \mu)$ ist $\varphi$ ebenfalls
$\sigma$-additiv.

\textbf{Folgerungen}:
Ist $f$ messbar, $f \ge 0$ und $A, B \in \mathcal{R}$ mit $B \subset A$,
so ist $\int_B f d\mu \le \int_A f d\mu$. \\
Ist $f$ messbar und $A, B \in \mathcal{R}$ mit $B \subset A$,
$\mu(A \setminus B) = 0$, so ist $\int_A f d\mu = \int_B f d\mu$.

\linie

\textbf{$\mu$-fast-überall}:
Sei $H(\cdot)$ eine Aussageform. \\
$H$ gilt \emph{$\mu$-fast-überall ($\mu$-f.ü.)},
falls es ein $E \in \mathcal{R}$, $\mu(E) = 0$
gibt, sodass $\{x \in X \;|\; \lnot H(x)\} \subset E$.

\emph{Beispiel}:
Sei $f\colon \real \rightarrow \real$ mit $f(x) = 1$ für $x \in \rational$
und $f(x) = 0$ für $x \notin \rational$, wobei
$\mu$ das Lebesgue-Maß ist
($f = \chi_\rational$ heißt \emph{\name{Dirichlet}-Funktion}).
Dann ist $\mu(\rational) = \sum_{n=1}^\infty \mu(\{q_n\}) = 0$, wobei
$\{q_n\}_{n \in \natural}$ eine Abzählung von $\rational$ ist.
Daher ist $\int_\real f d\mu = \int_{\real \setminus \rational} f d\mu = 0$.
$f$ ist $\mu$-fast-überall gleich $0$.

\textbf{Satz}:
Sei $f \in \L(X, \mu)$.
Dann ist auch $|f| \in \L(X, \mu)$ und
$\left|\int_X f d\mu\right| \le \int_X |f| d\mu$.

\textbf{Folgerungen}: \\
Ist $g \in \L(E, \mu)$, $f$ messbar mit $|f(x)| \le g(x)$ $\mu$-fast-überall,
so ist $f \in \L(E, \mu)$. \\
Ist $f \in \L(X, \mu)$, dann ist $f$ endlich $\mu$-fast-überall.

\pagebreak

\section{%
    Das \name{Lebesgue}-Integral und Konvergenz%
}

Im Folgenden sei $(X, \mathcal{R}, \mu)$ ein Maßraum.

\textbf{Satz von \name{Lebesgue} zur monotonen Konvergenz}:
Seien $f_n\colon X \rightarrow \real^\ast$ ($n \in \natural$)
messbare Funktionen mit
$f(x) := \lim_{n \to \infty} f_n(x)$
(bei bestimmter Divergenz $+\infty$) und \\
$0 \le f_1(x) \le f_2(x) \le \dotsb$ für alle $x \in X$.
Dann ist $\int_X f d\mu = \lim_{n \to \infty} \left(\int_X f_n d\mu\right)$.

\textbf{Satz (Additivität bzgl. des Integranden)}:
Seien $f_1, f_2 \in \L(X, \mu)$. \\
Dann ist auch $f_1 + f_2 \in \L(X, \mu)$ und
$\int_X (f_1 + f_2) d\mu = \int_X f_1 d\mu + \int_X f_2 d\mu$.

\textbf{Folgerung}:
Seien $f_k \ge 0$ messbar ($k \in \natural$)
und $f(x) = \sum_{k=1}^\infty f_k(x)$. \\
Dann ist $\int_X f d\mu = \sum_{k=1}^\infty \left(\int_X f_k d\mu\right)$.

\textbf{Lemma von \name{Fatou}}:
Seien $f_n \ge 0$ messbar ($n \in \natural$) und
$f(x) = \liminf_{n \to \infty} f_n(x)$. \\
Dann ist $\int_X f d\mu \le
\liminf_{n \to \infty} \left(\int_X f_n d\mu\right)$.

\textbf{Satz von \name{Lebesgue} zur majorisierten Konvergenz}:
Seien $f_n\colon X \rightarrow \real^\ast$ ($n \in \natural$),
messbare Funktionen mit
$f(x) := \lim_{n \to \infty} f_n(x)$ und
$|f_n(x)| \le g(x)$ für alle $x \in X$, $n \in \natural$
mit $g \in \L(X, \mu)$. \\
Dann ist $f \in \L(X, \mu)$ und
$\int_X f d\mu = \lim_{n \to \infty} \left(\int_X f_n d\mu\right)$.

\section{%
    Das \name{Lebesgue}- und das \name{Riemann}-Integral%
}

Sei nun $X = [a, b] \subset \real$ mit dem Lebesgue-Maß $\mu$ auf $\real$.
Im Folgenden bezeichnen $\int_a^b f(x)\dx$ das Riemann-Integral und
$\int_{[a,b]} f d\mu$ das Lebesgue-Integral.

\textbf{Satz von \name{Lebesgue}}: \\
a) Sei $f \in \R[a, b]$ Riemann-integrierbar.
Dann ist $f \in \L([a,b], \mu)$ sowie
$\int_{[a,b]} f d\mu = \int_a^b f(x)\dx$. \\
b) Sei $f$ messbar.
Dann ist $f \in \R[a, b]$ genau dann, wenn $f$ auf $[a, b]$ $\mu$-f.ü.
stetig ist. \\
Für $f \ge 0$ und $\int_X f d\mu = 0$ gilt $f(x) = 0$ $\mu$-f.ü.

\textbf{Anmerkung}:
Ist $f \in \L([a, b], \mu)$ mit $F(x) := \int_{[a, x]} f d\mu$, so ist
$F'(x) = f(x)$ $\mu$-f.ü. \\
\emph{Die Umkehrung gilt mit $\mu$-f.ü. nicht!}

\linie

\textbf{Vergleich von \name{Riemann}- und \name{Lebesgue}-Integral}:

\begin{tabular}{p{8cm}p{8cm}}
    \textbf{\name{Riemann}-Integral} &
    \textbf{\name{Lebesgue}-Integral} \\ \hline

    gerichtet, d.\,h. $\int_a^b \dx = -\int_b^a \dx$ &
    nicht gerichtet, $\int_{[a,b]} d\mu$ \vspace{2mm}\\

    Ist $f \in \R[a, b]$, so ist $f$ beschränkt. &
    Ist $f \in \L([a, b], \mu)$, so ist $|f|$ endlich $\mu$-f.ü. \vspace{2mm}\\

    nur auf $X = [a, b]$ definiert &
    auch $\mu(X) = \infty$ möglich, für $f \in \L(X, \mu)$ gilt dabei, dass
    $\{x \in X \;|\; f(x) \not= 0\}$ \textbf{$\sigma$-finit} ist, d.\,h.
    abzählbare Vereinigung von Mengen mit endlichem Maß \vspace{2mm}\\

    $\int_{-\infty}^{+\infty} = \lim_{R_1 \to \infty} \int_0^{R_1} +
    \lim_{R_2 \to \infty} \int_{-R_2}^0$, z.\,B.
    $\frac{\sin x}{x}$ ist uneigentlich Riemann-integrierbar, da sich Bereiche
    unterschiedlichen Vorzeichens auslöschen können &
    $\frac{\sin x}{x} \notin \L(\real, \mu)$, denn das Lebesgue-Integral
    kennt keine Auslöschungsef"|fekte ($f^\pm$ werden getrennt betrachtet)
\end{tabular}

Beim Lebesgue-Integral lässt sich mit den drei Konvergenzsätzen
(monotone/majorisierte Konvergenz, Lemma von Fatou) aus der punktweisen
Konvergenz und zusätzlichen Bedingungen ein Zusammenhang zwischen
$\lim_{n \to \infty} \left(\int f_n d\mu\right)$ und $\int f d\mu$ herstellen.
Weil sich die $\int f_n d\mu$ und damit auch $\int f d\mu$ bei einer
Änderung von $f_n$ auf einer Nullmenge nicht verändern, müssen die
Voraussetzungen nur für $x \in X \setminus E$, $\mu(E) = 0$ erfüllt sein,
d.\,h. nur $\mu$-fast-überall.

\section{%
    Die Funktionenräume \texorpdfstring{$L^p(X, \mu)$}{Lp(X, µ)}%
}

Wie immer sei $(X, \mathcal{R}, \mu)$ ein Maßraum.
Man kann eine Äquivalenzrelation $\sim$ auf der Menge der messbaren Funktionen
von $X$ nach $\real^\ast$ definieren mit
$f \sim g \;\Leftrightarrow\; f(x) = g(x)$ $\mu$-f.ü.
mit $f, g\colon X \rightarrow \real^\ast$ messbar.
Die Äquivalenzklasse von $f$ sei $\widehat{f}$.

\textbf{$L^p$-Raum}:
$\widehat{f} \in L^p(X, \mu) \;\Leftrightarrow\; |f|^p \in \L(X, \mu)$ für
$f \in \widehat{f}$, \\
$\widehat{f} \in L^\infty(X, \mu) \;\Leftrightarrow\;
\exists_{c \in \real}\; |f(x)| \le c$ $\mu$-f.ü. für $f \in \widehat{f}$.

Man schreibt auch $f \in L^p = L_p$ und $f \in L^\infty$.

\linie

\textbf{Norm im $L^p$-Raum}:
$\norm{f}_p := \left(\int_X |f|^p d\mu\right)^{1/p}$ und
$\norm{f}_\infty := \inf \{c \in \real \;|\; |f(x)| \le c\; \mu\text{-f.ü.}\}$.

\textbf{Satz (\name{Hölder}sche Ungleichung)}:
Seien $1 \le p, q \le \infty$ mit $\frac{1}{p} + \frac{1}{q} = 1$
und $f \in L^p$, $g \in L^q$. \\
Dann ist $f \cdot g \in L^1$ und
$\norm{f \cdot g}_1 \le \norm{f}_p \norm{g}_q$, d.\,h.
$\int_X |fg| d\mu \le \left(\int_X |f|^p d\mu\right)^{1/p} \cdot
\left(\int_X |g|^q d\mu\right)^{1/q}$.

\textbf{Satz (\name{Minkowski}sche Ungleichung)}:
Seien $1 \le p \le \infty$ und $f, g \in L^p$. \\
Dann ist $f + g \in L^p$ und
$\norm{f + g}_p \le \norm{f}_p + \norm{g}_p$.

Also ist $\norm{f}_p$ bzw. $\norm{f}_\infty$ eine Norm auf $L^p(X, \mu)$ bzw.
$L^\infty(X, \mu)$.

\linie

\textbf{Satz}:
$L^p(X, \mu)$ ist mit der Norm $\norm{\cdot}_p$ vollständig, d.\,h. \\
für $1 \le p \le \infty$ ist $L^p(X, \mu)$ ein Banachraum.

\linie

\textbf{wichtige Spezialfälle}:
\begin{itemize}
    \item
    \textbf{$L^1(X, \mu)$} enthält genau die integrierbaren Funktionen, d.\,h.
    "`$L^1(X, \mu) = \L(X, \mu)$"', wobei $\mu$-f.ü. identische Funktionen
    miteinander identifiziert werden.

    \item
    Auf \textbf{$L^2(X, \mu)$} kann man ein Skalarprodukt
    $\langle f, g \rangle:= \int_X f \overline{g} d\mu$ definieren mit \\
    $f, g \in L^2(X, \mu)$.
    (Alles bisher Genannte für Lebesgue-Integrale lässt sich auch
    komponentenweise auf komplex- und vektorwertige Funktionen in den
    $\field^n$ übertragen.) \\
    Damit wird $L^2(X, \mu)$ zu einem Hilbertraum.

    \item
    Wählt man $X = \natural$ als Grundmenge, $\R = 2^\natural$ als
    $\sigma$-Algebra und das Zählmaß $\mu$ auf $\R$
    (d.\,h. $\mu(M) = \#\{m \in M\} = \card(M)$ für $M \subset \natural$),
    so ist jede Funktion $f: X = \natural \rightarrow \real$ eine Folge
    $f_n = f(n)$ und für die $p$-Norm ergibt sich dann
    $\norm{f}_p = \left(\int_\natural |f|^p d\mu\right)^{1/p} =
    \left(\sum_{n=1}^\infty |f_n|^p\right)^{1/p}$.
    Den entstehenden Raum $L^p(\natural, \mu)$ bezeichnet man
    als \textbf{$\ell^p$-Raum}.

    \item
    Für $X = [a,b]$ und dem Lebesgue-Maß $\mu$ ist \textbf{$\C_0^\infty[a,b]$}
    die Menge der glatten Funktionen mit kompaktem Träger.
    Diese liegen dicht in $L^p([a,b], \mu)$ für $1 \le p < \infty$.
\end{itemize}

\pagebreak

\section{%
    Weitere Konvergenzaussagen%
}

Sei $(X, \R, \mu)$ ein Maßraum mit $\mu(X) < \infty$. \\
Eine Funktionenfolge $\{f_n\}_{n \in \natural}$
\textbf{konvergiert gleichmäßig gegen
eine Funktion $f$} ($f_n \rightrightarrows f$), falls
$\forall_{\varepsilon > 0} \exists_{N_\varepsilon \in \natural}
\forall_{n \ge N_\varepsilon} \forall_{x \in X}\;
|f_n(x) - f(x)| < \varepsilon$.
$\{f_n\}_{n \in \natural}$
\textbf{konvergiert punktweise gegen $f$ $\mu$-fast-überall},
($f_n \xrightarrow[\mu\text{-f.ü.}]{(\cdot)} f$), falls
$\mu(\{x \in X \;|\; f_n(x) \not\to f(x)\}) = 0$.

\textbf{Satz von \name{Egorov}}:
Seien $\mu(X) < \infty$ sowie $f_n \xrightarrow[\mu\text{-f.ü.}]{(\cdot)} f$
messbare Funktionen. \\
Dann gilt $\forall_{\delta > 0} \exists_{E_\delta \in \R}$
$\mu(X \setminus E_\delta) < \delta$,
$f_n|_{E_\delta} \rightrightarrows f|_{E_\delta}$.

\linie

\textbf{Konvergenz im Maß}:
$f_n$ \emph{konvergiert im Maß} gegen $f$, d.\,h.
$f_n \xrightarrow{\mu} f$, falls \\
$\forall_{\delta > 0}\;
\lim_{n \to \infty} \mu(\{x \in X \;|\; |f_n(x) - f(x)| \ge \delta\}) = 0$.

\textbf{Satz}:
Aus $f_n \xrightarrow[\mu\text{-f.ü.}]{(\cdot)} f$ folgt
$f_n \xrightarrow{\mu} f$.
\emph{Die Umkehrung gilt nicht!}

\textbf{Satz}:
Aus $f_n \xrightarrow{\mu} f$ folgt, dass es eine Teilfolge $f_{n_k}$ gibt
mit $f_{n_k} \xrightarrow[\mu\text{-f.ü.}]{(\cdot)} f$.

Für $\mu(X) < \infty$ ergibt sich also folgendes \textbf{Diagramm}:
\begin{align*}
    \begin{xy}
        %\xymatrix{
        %    & & f_n \rightrightarrows f \ar[dll] \ar[drr] & & \\
        %    f_n \xrightarrow{L^2} f & & & &
        %    f_n \xrightarrow[\mu\text{-f.ü.}]{(\cdot)} f \\
        %    & f_n \xrightarrow{L^1} f & & f_n \xrightarrow{\mu} f &
        %}
        \xymatrix{
            & f_n \rightrightarrows f \ar@{=>}[dl] \ar@{=>}[dr] & \\
            f_n \xrightarrow{L^2} \ar@{=>}[d] f & &
            f_n \xrightarrow[\mu\text{-f.ü.}]{(\cdot)} f \ar@{=>}[d]
            \ar@/_ 10mm/[ul]_{\text{auf } E_\delta \text{ (Egorov})} \\
            f_n \xrightarrow{L^1} f \ar@{=>}[rr] & &
            f_n \xrightarrow{\mu} f \ar@/_ 10mm/[u]_{\text{auf TF}}
        }
    \end{xy}
\end{align*}

\pagebreak
