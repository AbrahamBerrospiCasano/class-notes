\chapter{%
    Oberflächen- und Volumenintegrale, Elemente der Vektoranalysis%
}

\section{%
    Produktmaß, Satz von \name{Fubini}%
}

Im Folgenden seien $(X, \mathcal{A}_X, \mu)$ und $(Y, \mathcal{A}_Y, \nu)$
Maßräume, wobei $\mathcal{A}_X$ bzw. $\mathcal{A}_Y$ $\sigma$-Algebren auf
$X$ bzw. $Y$ sein sollen.
Dabei ist $a_X$ eine Menge $a_X \in \mathcal{A}_X$ und
$a_Y$ eine Menge $a_Y \in \mathcal{A}_Y$. \\
Man will nun ein Maß $\mu \otimes \nu$ auf $X \times Y$ konstruieren.
Dabei sollen die Rechtecke $a_X \times a_Y$ mit $a_X \in \mathcal{A}_X$ und
$a_Y \in \mathcal{A}_Y$ messbar sein, wobei
$(\mu \otimes \nu)(a_X \times a_Y) = \mu(a_X) \cdot \nu(a_Y)$.

\textbf{Algebra $\mathcal{A}_{X \times Y}$ auf $X \times Y$}: \\
Sind $(X, \mathcal{A}_X, \mu)$ und $(Y, \mathcal{A}_Y, \nu)$ Maßräume,
so definiert man die $\sigma$-Algebra $\mathcal{A}_{X \times Y}$ auf
$X \times Y$ als die kleinste $\sigma$-Algebra,
die alle Rechtecke $a_X \times a_Y$ mit
$a_X \in \mathcal{A}_X$, $a_Y \in \mathcal{A}_Y$ enthält.

\linie

\textbf{Satz}:
Sei $E \in \mathcal{A}_{X \times Y}$ mit
$E_x := \{y \in Y \;|\; (x, y) \in E\}$ für $x \in X$ und
$E_y := \{x \in X \;|\; (x, y) \in E\}$ für $y \in Y$.
Dann ist $\forall_{x \in X}\; E_x \in \mathcal{A}_Y$ und
$\forall_{y \in Y}\; E_y \in \mathcal{A}_X$.

Für $E \in \mathcal{A}_{X \times Y}$ sind also
$f(x) := \nu(E_x) \ge 0$ für $x \in X$ und
$g(y) := \mu(E_y) \ge 0$ für $y \in Y$ wohldefiniert,
da $E_x \in \mathcal{A}_Y$ und $E_y \in \mathcal{A}_X$.

\linie

Für alle Rechtecke $E = a_X \times a_Y$, $a_X \in \mathcal{A}_X$,
$a_Y \in \mathcal{A}_Y$ gilt $f(x) = \nu(a_Y)$ für $x \in a_X$ und
$f(x) = 0$ für $x \notin a_X$, analog $g(y) = \mu(a_X)$ für $y \in a_Y$ und
$g(y) = 0$ für $y \notin a_Y$. \\
Daher gilt $\int_X f(x) d\mu = \int_{a_X} \nu(a_Y) d\mu =
\mu(a_X) \cdot \nu(a_Y) = \int_{a_Y} \mu(a_X) d\nu = \int_Y g(y) d\nu$,
also $\int_X f(x) d\mu = \int_Y g(y) d\nu$.
Die Frage ist, ob dies für alle $E \in \mathcal{A}_{X \times Y}$ gilt.

\textbf{$\sigma$-finit}:
Ein Maßraum heißt $\sigma$-finit, falls die Grundmenge sich durch
höchstens abzählbar viele Mengen von endlichem Maß überdecken lässt.

\emph{Lemma}: Enthält ein monotones Mengensystem $\mathcal{D}$
(d.\,h. aus $E_n \in \mathcal{D}$, $E_1 \subset E_2 \subset \dotsb$ folgt
$E = \bigcup_{n=1}^\infty E_n \in \mathcal{D}$ und aus
$E_n' \in \mathcal{D}$, $E_1' \supset E_2' \supset \dotsb$ folgt
$E' = \bigcap_{n=1}^\infty E_n' \in \mathcal{D}$)
einen Ring $\mathcal{F}$, so enthält $\mathcal{D}$ auch den minimalen
$\sigma$-Ring $\mathcal{F}^\ast$, der von $\mathcal{F}$ erzeugt wird.

\linie

\textbf{Satz}:
Seien $\mu$ und $\nu$ $\sigma$-finite Maße.
Dann gilt $\int_X \nu(E_x) d\mu = \int_Y \mu(E_y) d\nu$
für alle $E \in \mathcal{A}_{X \times Y}$.

\textbf{Satz}:
Seien $(X, \mathcal{A}_X, \mu)$ und $(Y, \mathcal{A}_Y, \nu)$ $\sigma$-finite
Maßräume.
Dann ist
$\mu \otimes \nu\colon \mathcal{A}_{X \times Y} \rightarrow [0, +\infty]$
mit $(\mu \otimes \nu)(E) := \int_X \nu(E_x) d\mu = \int_Y \mu(E_y) d\nu$
ein Maß auf $\mathcal{A}_{X \times Y}$ (\textbf{Produktmaß}).

\linie

\textbf{Satz}:
Sei $h\colon X \times Y \rightarrow \real$ bzgl. $\mathcal{A}_{X \times Y}$
messbar. \\
Dann sind auch die Funktionen $h(x, \cdot)\colon Y \rightarrow \real$ bzgl.
$\mathcal{A}_Y$ und $h(\cdot, y)\colon X \rightarrow \real$ bzgl.
$\mathcal{A}_X$ messbar.

Für eine messbare Funktion $h\colon X \times Y \rightarrow \real$
kann man das \textbf{Doppelintegral}
$\int_{X \times Y} h(x, y) d(\mu \otimes \nu)$
und die \textbf{iterierten Integrale}
$\int_Y (\int_X h(x, y) d\mu) d\nu$ bzw. $\int_X (\int_Y h(x, y) d\nu) d\mu$
betrachten.

\textbf{Satz von \name{Fubini}}: \\
Sei $h\colon X \times Y \rightarrow \real$ messbar mit $h(x, y) \ge 0$
bzw. $h \in L^1(X \times Y, \mu \otimes \nu)$. \\
Dann sind für fast alle $x \in X$ die Funktion
$h(x, \cdot)$ und für fast alle $y \in Y$ die Funktion
$h(\cdot, y)$ messbar und
nicht-negativ bzw. integrierbar und es gilt \\
$\int_X (\int_Y h(x, y) d\nu) d\mu =
\int_{X \times Y} h(x, y) d(\mu \otimes \nu) =
\int_Y (\int_X h(x, y) d\mu) d\nu$.

\textbf{Satz von \name{Fubini-Tonelli}}: \\
Sei $h\colon X \times Y \rightarrow \real$ messbar mit
$\int_X (\int_Y |h(x, y)|d\nu) d\mu < \infty$ oder
$\int_Y (\int_X |h(x, y)|d\mu) d\nu < \infty$. \\
Dann gilt $h \in L^1(X \times Y, \mu \otimes \nu)$
und der Satz von Fubini lässt sich anwenden.

\linie
\pagebreak

\textbf{Folgerung}:
Seien $h(x, y) \ge 0$ mit $\int_{X \times Y} h(x, y)
d(\mu \otimes \nu) < \infty$, $f(x) := \int_Y h(x, y) d\nu$ und \\
$g(y) := \int_X h(x, y) d\mu$.
Für $h(x, y) \ge 0$ bzw. $h \in L^1(X \times Y, \mu \otimes \nu)$
gilt nach dem Satz von Fubini $f \in L^1(X, \mu)$ und $g \in L^1(Y, \nu)$,
d.\,h. $f(x) < \infty$ $\mu$-f.-ü. und $g(y) < \infty$ $\nu$-f.-ü. sowie
$\int_Y h(x, y) d\nu < \infty$ $\mu$-f.-ü. und
$\int_X h(x, y) d\mu < \infty$ $\nu$-f.-ü.

\linie

\emph{Falsch ist dagegen folgende Aussage}:
Aus $|\int_X (\int_Y h(x, y) d\nu) d\mu| < \infty$ und \\
$|\int_Y (\int_X h(x, y) d\mu) d\nu| < \infty$ folgt, dass die
Integrale gleich sind. \\
Ein Gegenbeispiel ist $X = Y = [0, 1]$ mit $\mu = \nu$ dem Lebesgue-Maß
und $h(x, y) = \frac{x^2 - y^2}{(x^2 + y^2)^2}$. \\
Es gilt $\int_0^1 \left(\int_0^1 h(x, y) \dx\right) \dy =
\int_0^1 \left(-\frac{1}{1 + y^2}\right) \dx =
-\frac{\pi}{2} \not= \frac{\pi}{2} =
\int_0^1 \frac{1}{1 + x^2} \dx = \int_0^1 \left(\int_0^1 h(x, y) \dy\right) \dx$.

\section{%
    Zur Substitution der Integrationsvariablen%
}

Seien $(X, \mathcal{A}_X, \mu)$ ein Maßraum und $f\colon D_X \rightarrow \real$
eine messbare Funktion mit $D_X \subset X$. \\
Außerdem seien eine Menge $Y$ und eine bijektive Abbildung
$\varphi\colon D_Y \rightarrow D_X$ mit $D_Y \subset Y$ gegeben.
Gesucht ist nun ein Maßraum $(Y, \mathcal{A}_Y, \nu)$, sodass
$\int_{D_Y} (f \circ \varphi)(y) d\nu = \int_{D_X} f(x) d\mu$ gilt.

Für die Teilmengen $F \subset Y$ von $Y$ soll dabei gelten, dass
$F \in \mathcal{A}_Y$ gilt
genau dann, wenn $E := \varphi(F) \in \mathcal{A}_X$ ist.

Da $f$ nur auf $D_X$ definiert ist, reduziert man
die Maßräume auf $(D_X, \widetilde{\mathcal{A}}_X, \mu)$ und
$(D_Y, \widetilde{\mathcal{A}}_Y, \nu)$,
wobei für $\widetilde{E} \subset D_X$ gilt, dass
$\widetilde{E} \in \widetilde{\mathcal{A}}_X$ genau dann, wenn
$\widetilde{E} = E \cap D_X$ für ein $E \in \mathcal{A}_X$, \\
d.\,h. $\widetilde{\mathcal{A}}_X := \{E \cap D_X \;|\; E \in \mathcal{A}_X\}$.

Daraus leitet man folgende Konstruktion für den Maßraum
$(D_Y, \widetilde{\mathcal{A}}_Y, \nu)$ ab:
\begin{enumerate}
    \item
    $\widetilde{\mathcal{A}}_Y :=
    \{F \subset D_Y \;|\; \varphi(F) \in \widetilde{\mathcal{A}}_X\}$

    \item
    $\nu(F) := \mu(\varphi(F))$
\end{enumerate}

\linie

\emph{Beispiel}:
Seien $f\colon [a, b] \rightarrow \real$ eine Funktion und
$\varphi\colon [\alpha, \beta] \rightarrow [a, b]$ bijektiv
mit $\varphi \in \C^1$. \\
Beim Riemann-Integral gilt der Transformationssatz
$\int_\alpha^\beta (f \circ \varphi)(y) \varphi'(y) \dy = \int_a^b f(x)\dx$. \\
Seine Entsprechung für das Lebesgue-Integral lautet
$\int_{[\alpha, \beta]} (f \circ \varphi)(y) |\varphi'(y)| \dy =
\int_{[a, b]} f(x)\dx$.
Dabei ist $dx = d\mu$ als das "`alte Maß"' (Lebesgue-Maß)
und $|\varphi'(y)| dy = d\nu$ als das "`neue"',
substituierte Maß zu betrachten. \\
Der Absolutbetrag beim Lebesgue-Integral kommt daher, dass das
Lebesgue-Integral ungerichtet ist und für $\varphi$ monoton fallend
(bijektiv, stetig) ändert sich das Integrationsgebiet nicht
(von der Orientierung her). \\
Beim Riemann-Integral werden bei der Integration mit monoton fallendem
$\varphi$ die Grenzen vertauscht (d.\,h. aus $\alpha < \beta$ folgt
$a > b$), also ist ein zusätzliches Vorzeichen nötig, um die Vertauschung
rückgängig zu machen.

\linie

\pagebreak

Ein wichtiger Spezialfall des Transformationssatzes
ist $\varphi\colon \real^n \rightarrow \real^n$
mit $x = \varphi(y) = Ay$, wobei $x, y \in \real^n$ und
$A$ eine lineare Abbildung ist.
Gesucht ist ein Maß $\nu$ auf $\real^n$, sodass
$\nu(F) = \mu(E)$ für $E = \varphi(F)$.
Das Volumen von $E$ ist $\vol(E) = |\det A| \cdot \vol(F)$.
Daraus folgt mit
$d\nu = \left|\frac{D(x_1, \dotsc, x_n)}{D(y_1, \dotsc, y_n)}\right| d^n y$
für das Maß von $F$
$\nu(F) = \int_F 1 d\nu =
\int_F 1 \left|\frac{D(x_1, \dotsc, x_n)}{D(y_1, \dotsc, y_n)}\right| d^n y$
bzw. als Transformationssatz
$\int_{D_y} (f \circ \varphi)(y)
\left|\frac{D(x_1, \dotsc, x_n)}{D(y_1, \dotsc, y_n)}\right| d^n y =
\int_{D_x} f(x) d^n x$, wobei $d^n x = d\mu$ das Lebesgue-Maß
und $\frac{D(x_1, \dotsc, x_n)}{D(y_1, \dotsc, y_n)}$ die Jacobi-Matrix
von $\varphi$ ist.

Bei der multiplen Substitution mit $\psi\colon \real^n \rightarrow \real^n$
und $\varphi\colon \real^n \rightarrow \real^n$ bijektiv mit
$\psi, \varphi \in \C^1$ sowie $f\colon \real^n \rightarrow \real$ gilt
$\int_{D_T} (f \circ \varphi \circ \psi)(t)
\left|\frac{D(x_1, \dotsc, x_n)}{D(t_1, \dotsc, t_n)}\right| d^n t =
\int_{D_Y} (f \circ \varphi)(y)
\left|\frac{D(x_1, \dotsc, x_n)}{D(y_1, \dotsc, y_n)}\right| d^n y =
\int_{D_X} f(x) d^n x$ aufgrund
$\left|\frac{D(x_1, \dotsc, x_n)}{D(t_1, \dotsc, t_n)}\right| =
\left|\frac{D(x_1, \dotsc, x_n)}{D(y_1, \dotsc, y_n)}\right| \cdot
\left|\frac{D(y_1, \dotsc, y_n)}{D(t_1, \dotsc, t_n)}\right|$.

\linie

\textbf{Transformationssatz für $\real^n$}:
Seien $U, V \subset \real^n$ of"|fen und $\varphi\colon U \rightarrow V$
bijektiv mit $\varphi$ und $\varphi^{-1}$ stetig dif"|ferenzierbar
(d.\,h. $\varphi$ ist Dif"|feomorphismus). \\
Dann ist eine Funktion $f\colon V \rightarrow \real$ auf
$V$ integrierbar genau dann, wenn $(f \circ \varphi) \cdot |\det \varphi'|$
auf $U$ integrierbar ist, und es gilt
$\int_U f(\varphi(x)) \cdot |\det \varphi'(x)| \dx = \int_V f(y) \dy$.

\linie

\emph{Beispiel}:
Im $\real^2$ sei der Kreisauschnitt $D_X$ mit Radius $r$ und Winkel
$\frac{2\pi}{3} = 120^\circ$ gegeben.
Wenn z.\,B. der Schwerpunkt von $D_X$ berechnet werden soll, müssen Integrale
wie $\int_{D_X} x_1 d^2 x$ berechnet werden.
Dies geht einfacher mit Koordinatentransformation in Polarkoordinaten,
d.\,h. mit der Transformation $\varphi(r, \theta) =
\begin{pmatrix}r \cos \theta \\ r \sin \theta\end{pmatrix} =
\begin{pmatrix}x_1 \\ x_2\end{pmatrix}$. \\
Die Jacobi-Matrix ist $\frac{D(x_1, x_2)}{D(r, \theta)} =
\begin{pmatrix}\cos \theta & -r \sin \theta \\
\sin \theta & r \cos \theta\end{pmatrix}$, ihre Determinate ist
$\left|\frac{D(x_1, x_2)}{D(r, \theta)}\right| = r$. \\
Das Integrationsgebiet wird dabei auf
$D_Y := \left]0, 1\right] \times [0, 2\pi/3]$ transformiert.
Die Gerade $r = 0$ fehlt dabei, weil sonst die Abbildung nicht
bijektiv wäre. \\
Nach dem Transformationssatz gilt $\int_{D_X} x_1 d^2 x =
\int_{D_y} (f \circ \varphi)(r, \theta) r d(r, \theta)$ mit
$d\nu = r d(r, \theta)$, d.\,h.
$\int_{D_X} x_1 d^2 x = \int_{[0, 2\pi/3]}
\left(\int_{\left]0, 1\right]} (f \circ \varphi)(r, \theta) r\dr\right) \d\phi =
\int_0^{2\pi/3} \left(\int_0^1 (r \cos \theta) r\dr\right) d\theta$
(Fubini) usw.

\linie

\textbf{Übersicht über verschiedene Koordinatentransformationen}:

\begin{tabular}{p{40mm}p{66mm}p{50mm}}
    \toprule

    \textbf{Name} & \textbf{Definition} & \textbf{Funktionaldeterminante} \\

    \midrule

    Polarkoordinaten &
    $\varphi\colon \left[0, +\infty\right[ \times \left[0, 2\pi\right[
    \rightarrow \real^2$, \newline
    $\varphi(r, \theta) = \begin{pmatrix}r \cos \theta \\
    r \sin \theta\end{pmatrix}$ &
    $r$ \\

    \midrule

    Zylinderkoordinaten &
    $\varphi\colon \left[0, +\infty\right[ \times
    \left[0, 2\pi\right[ \times \real \rightarrow \real^3$, \newline
    $\varphi(r, \theta, z) = \begin{pmatrix}r \cos \theta \\
    r \sin \theta \\ z\end{pmatrix}$ &
    $r$ \\

    \midrule

    Kugelkoordinaten &
    $\varphi\colon \left[0, +\infty\right[ \times
    \left[0, 2\pi\right[ \times [0, \pi] \rightarrow \real^3$, \newline
    $\varphi(r, \phi, \theta) = \begin{pmatrix}r \sin \theta \cos \phi \\
    r \sin \theta \sin \phi \\ r \cos \theta\end{pmatrix}$ &
    $r^2 \sin \theta$ \\

    \bottomrule
\end{tabular}

\pagebreak

\section{%
    \texorpdfstring{Mannigfaltigkeiten im $\real^n$}%
    {Mannigfaltigkeiten im ℝⁿ}%
}

Im Folgenden werden Teilmengen $S \subset \real^n$ des $\real^n$ betrachtet.
Für $x \in S$ und $\varepsilon > 0$ ist $U_\varepsilon(x)$ die
$\varepsilon$-Umgebung um $x$, dabei soll
$U_\varepsilon^S(x) := S \cap U_\varepsilon(x)$ der Schnitt
der $\varepsilon$-Umgebung um $x$ mit $S$ sein.
Außerdem sei $V_k := U_1(0) \subset \real^k$ die of"|fene Einheitskugel
im $\real^k$.

\textbf{Mannigfaltigkeit (ohne Rand)}:
$S \subset \real^n$ heißt \emph{$k$-dimensionale Mannigfaltigkeit}, \\
falls es für alle Punkte $x \in S$ ein $\varepsilon(x) > 0$ und einen
Homöomorphismus $\varphi_x\colon V_k \rightarrow U_{\varepsilon(x)}^S(x)$ gibt
(d.\,h. $\varphi_x$ bijektiv mit $\varphi_x$ und $\varphi_x^{-1}$ stetig). \\
Für $x \in S$ heißt das Paar $(\varphi_x, U_{\varepsilon(x)}^S(x))$
\emph{Karte} und eine Menge von Karten
$\{(\varphi_x, U_{\varepsilon(x)}^S(x)\}$ heißt \emph{Atlas}.
Es gilt $S = \bigcup_{x \in S} U_{\varepsilon(x)}^S(x)$, d.\,h. falls
$S$ kompakt ist, reichen endlich viele Karten zur Beschreibung von $S$ aus.

\linie

\textbf{Orientierung}:
Im $\real^k$ kann man eine (geordnete)
Basis betrachten, z.\,B. die Einheitsbasis.
Mit dem Vorzeichen der Determinante der Matrix,
die die Basisvektoren als Spalten enthält,
kann eine Art "`Orientierung"' der Basis bestimmt werden.
Im Spezialfall $k = 3$ geht dies z.\,B. mit der Drei-Finger-Regel:
Je nachdem, ob die rechte oder die linke Hand zur Darstellung der Basisvektoren
benutzt werden kann, wird die Basis \emph{Rechts- oder Linkssystem} genannt.

Ist nun eine $k$-dimensionale Mannigfaltigkeit $S$ und ein Punkt $x \in S$
gegeben, so kann man die Basis mittels des Homöomorphismus $\varphi_x$
in den $\real^n$ transformieren.
Wird die Orientierung der Basis dabei vertauscht, d.\,h.
ist das Vorzeichen der Determinante der Matrix, die die Basis des $\real^k$
auf den $\real^n$ abbildet, negativ, so bezeichnet man den Homöomorphismus
als \emph{orientierungsumkehrend}.
Ist das Vorzeichen positiv, so heißt er \emph{orientierungserhaltend}.

Da $\varphi_x$ stetig ist, kann die Orientierung im Punkt $x$ auf die
ganze Umgebung $U_{\varepsilon(x)}^S(x)$ ausgedehnt werden.
Betrachtet man jedoch einen weiteren Punkt $\widetilde{x} \in S$, so kann es
passieren, dass der Schnitt $U_{\varepsilon(x)}^S(x) \cap
U_{\varepsilon(\widetilde{x})}^S(\widetilde{x})$ der Umgebungen nicht-leer ist.
Wenn die Orientierung in beiden Punkten dieselbe ist, so heißen die Karten
$(\varphi_x, U_{\varepsilon(x)}^S(x))$ und
$(\varphi_{\widetilde{x}}, U_{\varepsilon(\widetilde{x})}^S(\widetilde{x}))$
\emph{kompatibel}.

Eine \emph{orientierbare Mannigfaltigkeit} ist eine Mannigfaltigkeit, zu der
es einen Atlas gibt, bei dem die Orientierung überall auf der Mannigfaltigkeit
erhalten bleibt, d.\,h. alle Karten sind miteinander kompatibel.

Beispielsweise ist die Kugel eine orientierbare Mannigfaltigkeit,
das \name{Möbius}-Band ist eine nicht-orientierbare Mannigfaltigkeit.

\linie

\textbf{Mannigfaltigkeiten mit Rand}:
Eine \emph{Mannigfaltigkeit mit Rand} ist eine Mannigfaltigkeit, bei der für
jeden Punkt $x \in S$ es nicht einen Homöomorphismus
$\varphi$ auf $V_k$ geben muss, sondern stattdessen (alternativ) es einen
Homöomorphismus
$\varphi_x^+\colon V_k^+ \rightarrow U_{\varepsilon(x)}^S(x)$ mit
$V_k^+ := \{t \in V_k \;|\; t_1 \ge 0\}$ geben kann.
Falls es für $x \in S$ den Homöomorphismus $\varphi_x$ gibt, so heißt
$x$ \emph{innerer Punkt} von $S$ ($x \in \Int S$), falls es $\varphi_x^+$ gibt,
heißt $x$ \emph{Randpunkt} von $S$ ($x \in \partial S$).

Es gilt $\Int S \cap \partial S = \emptyset$.
Der Rand von $S$ ist selbst eine $k - 1$-dimensionale Mannigfaltigkeit
mit leerem Rand, d.\,h. $\partial (\partial S) = \emptyset$.
Bei der Bestimmung der Orientierung in einem Randpunkt $x \in \partial S$ wird
ein zusätzlicher Basisvektor am Rand eingefügt, der von der Mannigfaltigkeit
wegzeigt, sodass die Orientierung mit den inneren Punkten übereinstimmt.

\pagebreak

\section{%
    \texorpdfstring{Oberflächeninhalt und Volumen im $\real^n$}%
    {Oberflächeninhalt und Volumen im ℝⁿ}%
}

\textbf{Parallelepiped}:
Ein \emph{$k$-dimensionales Parallelepiped} $P$ im $\real^n$ ist der Aufspann
von $k$ linear unabhängigen Vektoren $\xi_1, \dotsc, \xi_k \in \real^n$,
wobei die Koef"|fizienten zwischen $0$ und $1$ liegen müssen, d.\,h.
$P = \left\{\sum_{i=1}^k \alpha_i \xi_i \;|\; 0 \le \alpha_i \le 1\right\}$. \\
Für $k = 2$ bzw. $k = 3$ nennt man $P$ auch Parallelogramm bzw. Spat.

\linie

\textbf{Wie groß ist das $k$-dimensionale Volumen von $P$?}

Für $k = n$ gilt für das Volumen $\vol_n(P) = |\det J|$ mit
$J := (\xi_1, \dotsc, \xi_n)$
(die Matrix, in der die aufspannenden Vektoren $\xi_1, \dotsc, \xi_n$ als
Spalten stehen).

Für $k < n$ ist $J = (\xi_1, \dotsc, \xi_k)$ nicht quadratisch. \\
Man definiert die \textbf{\name{Gram}sche Matrix}
$G := J^t J = (g_{ij})_{i,j=1}^n$ mit
$g_{ij} = \sp{\xi_i, \xi_j}$ und das Volumen beträgt
$\vol_k(P) := \sqrt{\det J^t J} \ge 0$.
Im Spezialfall $k = n$ ist $\vol_n(P) = |\det J|$ wie oben.

\linie

Für $k$-dimensionale Mannigfaltigkeiten $S \subset \real^n$
und Abbildungen
$\varphi\colon D \rightarrow \varphi(D)$ mit $D \subset \real^k$,
$\varphi(D) \subset S$ sowie $\varphi \in \C^1$ ergibt sich die Formel
$\vol_k(\varphi(D)) := \int_D \sqrt{\det\left(\sp{\frac{\partial
\varphi}{\partial t_i}, \frac{\partial
\varphi}{\partial t_j}}\right)_{i,j=1}^k} d^k t$. \\
Für $k = n$ entspricht die innere Matrix $G$ mit $G = J^t J$
und der Jacobi-Matrix
$J = \left(\frac{\partial \varphi_i}{\partial t_j}\right)_{i,j=1}^n$. \\
Daher ist $\sqrt{\det G} = |\det J|$ und die Formel stimmt mit dem
Transformationssatz überein.

\linie

\emph{Beispiel}:
Für $k = 1$ sei $\varphi\colon D \rightarrow \varphi(D)$ mit
$\varphi \in \C^1$ bijektiv und $D \subset \real$,
$\varphi(D) \subset \real^n$.
Dann gilt $G = \left(\frac{d \varphi_1}{dt}\right)^2 + \dotsb +
\left(\frac{d \varphi_n}{dt}\right)^2$, d.\,h.
$\sqrt{G} = \norm{\dot{\varphi}}$ und es ergibt sich die korrekte Formel für
den Weg $\int_D \norm{\dot{\varphi}} \dt$
($1$-dimensionales Volumen im $\real^n$).

\linie

\emph{Beispiel}:
Gegeben sei eine Funktion $f\colon \real^2 \rightarrow \real$ mit $f \in \C^1$.
Man betrachte die Fläche, die entsteht, wenn man $f(x, y) \in \real$ über
$(x, y) \in \real^2$ aufträgt ($2$-dim. Mannigfaltigkeit).
Gesucht ist für eine Menge $D \subset \real^2$ der zweidimensionale
Flächeninhalt von $\varphi(D)$ mit
$\varphi(u, v) = (u, v, f(u, v))^t$.
Aufgrund $\frac{\partial \varphi}{\partial u} = (1, 0, f_u')^t$ und
$\frac{\partial \varphi}{\partial v} = (0, 1, f_v')^t$ gilt
$G =$ \matrixsize{$\begin{pmatrix}1 + (f_u')^2 & f_u' f_v' \\
f_u' f_v' & 1 + (f_v')^2\end{pmatrix}$}, d.\,h. \\
$\det G = 1 + (f_u')^2 + (f_v')^2$ bzw.
$\vol_2(\varphi(D)) = \int_D \sqrt{1 + (f_u')^2 + (f_v')^2} \du\dv$.

\pagebreak

\section{%
    Dif"|ferentialformen%
}

Seien $X$ ein $\real$-Vektorraum mit $\dim X =: n$.
$X^k$ bezeichne das $k$-fache kartesische Produkt von $X$ mit sich selbst.
Man definiert den \textbf{Raum der $k$-fachen Multilinearformen} \\
$\L_k(X, \real) := \{F\colon X^k \rightarrow \real \;|\;
F \text{ in jeder Komponente linear}\}$.
Zum Beispiel ist $\L_1(X, \real) =: X'$ der Dualraum und
$\L_2(X, \real)$ enthält alle bilinearen Abbildungen.

\textbf{Tensorprodukt}:
Für $p$- bzw. $q$-fache Multilinearformen $F' \in \L_p(X, \real)$
bzw. $F'' \in \L_q(X, \real)$ ist der \emph{Tensor}
$F' \otimes F'' \in \L_{p+q}(X, \real)$ eine $p + q$-fache Multilinearform,
wobei \\
$(F' \otimes F'')(\xi_1, \dotsc, \xi_p, \xi_{p+1}, \dotsc, \xi_{p+q}) :=
F'(\xi_1, \dotsc, \xi_p) \cdot F''(\xi_{p+1}, \dotsc, \xi_{p+q})$. \\
Die Abbildung $\otimes\colon \L_p(X, \real) \times \L_q(X, \real)
\rightarrow \L_{p+q}(X, \real)$ heißt \emph{Tensorprodukt}.

Das Tensorprodukt erfüllt Assoziativität
($F' \otimes (F'' \otimes F''') = (F' \otimes F'') \otimes F'''$),
Distributivität ($(F_1' + F_2') \otimes F'' =
(F_1' \otimes F'') + (F_2' \otimes F'')$ und
$F' \otimes (F_1'' + F_2'') =
(F' \otimes F_1'') + (F' \otimes F_2'')$) sowie
Assoziativität mit der skalaren Multiplikation
($(\lambda F') \otimes F'' = F' \otimes (\lambda F'') =
\lambda (F' \otimes F'')$).

\linie

Ist $e_1, \dotsc, e_n$ eine Basis von $X$, so kann man Vektoren
$\xi_\ell \in X$, $\ell = 1, \dotsc, k$ eindeutig als Linearkombination
$\xi_\ell = \sum_{j_\ell=1}^n \xi_\ell^{j_\ell} e_{j_\ell} =:
\xi_\ell^{j_\ell} e_{j_\ell}$ schreiben.
Die letzte Schreibweise entspricht der
\textbf{\name{Einstein}schen Summenkonvention}.
Sie besagt, dass bei Termen, in denen Indizes doppelt auftauchen,
über diese Indizes summiert werden muss.
Diese Notation wird im Folgenden exzessiv angewandt.

\textbf{Koordinatentransformation in $X$}: \\
Ist $\widetilde{e}_1, \dotsc, \widetilde{e}_n$ eine zweite Basis von $X$,
so kann die Basis $e_1, \dotsc, e_n$ in
$\widetilde{e}_1, \dotsc, \widetilde{e}_n$ mittels einer Basiswechselmatrix
$C = (c_j^i)_{i,j=1}^n$ überführt werden,
d.\,h. $\widetilde{e}_j = c_j^i e_i$.
Für $\xi \in X$ gibt es eindeutige Darstellungen als Linearkombination der
Basen $\xi = \xi^j e_j = \widetilde{\xi}^j \widetilde{e}_j =
\widetilde{\xi}^\ell c_\ell^j e_j$, d.\,h. es muss aufgrund der Eindeutigkeit
$\xi^j = \widetilde{\xi}^\ell c_\ell^j$ gelten.

\textbf{Koordinatentransformation in $\L_1(X, \real) = X'$}: \\
Für $F \in X'$ gilt $F[\xi] = F[\xi^j e_j] = \xi^j F[e_j] = \xi^j a_j$
mit $a_j := F[e_j]$.
Es gilt $F = a_j e^j$, wobei $e^1, \dotsc, e^n \in X'$ mit
$e^j[e_k] := \delta_{kj}$ eine Basis von $X'$ ist
(es gilt $e^j[\xi] = e^j[\xi^k e_k] = \xi^k \delta_{kj} = \xi^j$, d.\,h.
$e^j$ ist die Projektion auf die $j$-te Komponente).
Für die zweite Basis gilt nun $F = \widetilde{a}_\ell \widetilde{e}^j$ mit
$\widetilde{a}_\ell := F[\widetilde{e}_\ell] = F[c_\ell^j e_j] =
c_\ell^j F[e_j] = c_\ell^j a_j$, d.\,h.
$\widetilde{a}_\ell = c_\ell^j a_j$.

\textbf{Koordinatentransformation in $\L_k(X, \real)$}: \\
Wendet man eine $k$-fache Multilinearform $F \in \L_k(X, \real)$ auf
$\xi_1, \dotsc, \xi_k \in X$ an, so erhält man
$F[\xi_1, \dotsc, \xi_k] =
F[\xi_1^{i_1} e_{i_1}, \dotsc, \xi_k^{i_k} e_{i_k}] =
\xi_1^{i_1} \dotsm \xi_k^{i_k} F[e_{i_1}, \dotsc, e_{i_k}] =
\xi_1^{i_1} \dotsm \xi_k^{i_k} a_{i_1, \dotsc, i_k}$ \\
mit $a_{i_1, \dotsc, i_k} := F[e_{i_1}, \dotsc, e_{i_k}]$, d.\,h.
$F = a_{j_1, \dotsc, j_k} (e^{j_1} \otimes \dotsb \otimes e^{j_k})$ mit \\
$(e^{j_1} \otimes \dotsb \otimes e^{j_k})[\xi_1, \dotsc, \xi_k] =
e^{j_1}[\xi_1] \dotsm e^{j_k}[\xi_k] =
\xi_1^{j_1} \dotsm \xi_k^{j_k}$. \\
Für die zweite Basis gilt nun
$F = \widetilde{a}_{j_1, \dotsc, j_k}
(\widetilde{e}^{j_1} \otimes \dotsb \otimes \widetilde{e}^{j_k})$ mit \\
$\widetilde{a}_{i_1, \dotsc, i_k} :=
F[\widetilde{e}_{i_1}, \dotsc, \widetilde{e}_{i_k}] =
F[c_{i_1}^{j_1} e_{j_1}, \dotsc, c_{i_k}^{j_k} e_{j_k}] =
c_{i_1}^{j_1} \dotsm c_{i_k}^{j_k} a_{j_1, \dotsc, j_k}$.

\linie

\textbf{antisymmetrisch}:
Eine $k$-fache Multilinearform $F \in \L_k(X, \real)$ heißt
\emph{antisymmetrisch} oder \emph{alternierend}, falls
$F[\xi_1, \dotsc, \xi_i, \dotsc, \xi_j, \dotsc, \xi_k] =
-F[\xi_1, \dotsc, \xi_j, \dotsc, \xi_i, \dotsc, \xi_k]$ \\
für alle $\xi_1, \dotsc, \xi_k \in X$, $i \not= j$. \\
Die Menge $\Omega_k(X, \real) :=
\{F \in \L_k(X, \real) \;|\; F \text{ antisymmetrisch}\}
\subset \L_k(X, \real)$ ist ein Vektorraum.

\emph{Beispiel}:
$F[\xi_1, \dotsc, \xi_n] := \det(\xi_k^\ell)_{k,\ell=1}^n$ ist eine $n$-fache
antisymmetrische Multilinearform.

\linie
\pagebreak

\textbf{Antisymmetrisierungsabbildung}:
Für $k \in \natural$ ist die \emph{Antisymmetrisierungsabbildung} \\
$\A\colon \L_k \rightarrow \Omega_k$ definiert durch
$(\A F)[\xi_1, \dotsc, \xi_k] :=
\frac{1}{k!} F[\xi_{i_1}, \dotsc, \xi_{i_k}]
\sigma_{1,\dotsc,k}^{i_1,\dotsc,i_k}$ \\
mit dem \emph{\name{Levi}-\name{Civita}-Symbol}
$\sigma_{1,\dotsc,k}^{i_1,\dotsc,i_k} :=$ \matrixsize{$\begin{cases}
0 & i_1, \dotsc, i_k \text{ keine Permutation von } 1, \dotsc, k \\
1 & \text{gerade Permutation} \\
-1 & \text{ungerade Permutation}.
\end{cases}$} \\
Sie weist jeder $k$-fachen Multilinearform $F \in \L_k(X, \real)$ auf
kanonische Weise eine $k$-fache antisymmetrische Multilinearform
$\A F \in \Omega_k(X, \real)$ zu.

Man kann auch
$(\A F)[\xi_1, \dotsc, \xi_k] =
\frac{1}{k!} \sum_{\pi \in \mathfrak{S}_k}
\sign(\pi) \cdot F[\xi_{\pi(1)}, \dotsc, \xi_{\pi(k)}]$
schreiben, wobei $\mathfrak{S}_k$ die symmetrische Gruppe ist.

$\A$ ist linear, d.\,h. $\A(F' + F'') = \A F' + \A F''$ und
$\A (\lambda F) = \lambda (\A F)$. \\
Es gilt $\A(e^{j_1} \otimes \dotsb \otimes e^{j_k})[\xi_1, \dotsc, \xi_k]
= \frac{1}{k!} e^{j_1}[\xi_{i_1}] \dotsm e^{j_k}[\xi_{i_k}]
\sigma_{1,\dotsc,k}^{i_1,\dotsc,i_k} =
\frac{1}{k!} \xi_{i_1}^{j_1} \dotsm \xi_{i_k}^{j_k}
\sigma_{1,\dotsc,k}^{i_1,\dotsc,i_k}$, d.\,h. \\
$\A(e^{j_1} \otimes \dotsb \otimes e^{j_k})[\xi_1, \dotsc, \xi_k] =
\frac{1}{k!} \cdot
\det$\matrixsize{$\begin{pmatrix}\xi_1^{j_1} & \dots & \xi_k^{j_1} \\
\vdots & & \vdots \\
\xi_1^{j_k} & \dots & \xi_k^{j_k}\end{pmatrix}$}.

\vspace{3mm}
\linie

\textbf{äußeres Produkt}:
Für $p$- bzw. $q$-fache antisymmetrische Multilinearformen
$\omega' \in \Omega_p$ bzw. $\omega'' \in \Omega_q$ ist das
\emph{äußere Produkt} $\omega' \land \omega'' \in \Omega_{p+q}$ eine
$p + q$-fache antisymmetrische Multilinearform, wobei
$\omega' \land \omega'' :=
\frac{(p + q)!}{p! \cdot q!} \A(\omega' \otimes \omega'')$.
Dies definiert eine Abbildung
$\land\colon \Omega_p \times \Omega_q \rightarrow \Omega_{p+q}$.

Man kann auch $(\omega' \land \omega'')[\xi_1, \dotsc, \xi_p,
\xi_{p+1}, \dotsc, \xi_{p+q}]$ \\
$= \frac{1}{p! \cdot q!} \sum_{\pi \in \mathfrak{S}_{p+q}} \sign(\pi) \cdot
\omega'[\xi_{\pi(1)}, \dotsc, \xi_{\pi(p)}] \cdot
\omega''[\xi_{\pi(p+1)}, \dotsc, \xi_{\pi(p+q)}]$ schreiben.

Das äußere Produkt erfüllt Distributivität
($(\omega_1' + \omega_2') \land \omega'' =
\omega_1' \land \omega'' + \omega_2' \land \omega''$), Assoziativität,
Assoziativität mit der skalaren Multiplikation
($(\lambda \omega') \land \omega'' = \lambda (\omega' \land \omega'') =
\omega' \land (\lambda \omega'')$) und \\
Antikommutativität
($\omega' \land \omega'' = (-1)^{pq} \omega'' \land \omega'$).
Daraus folgt dann für $\omega \in \Omega_{p}$ und $p$ ungerade, dass
$\omega \land \omega = 0$
(im Falle von $p$ gerade ist i.\,A. $\omega \land \omega \not= 0$).

\emph{Beispiel}:
Für $p = q = 1$ und $e^{i_1}, e^{i_2} \in \Omega_1 = \L_1 = X'$ ist
$(e^{i_1} \land e^{i_2})[\xi_1, \xi_2] =
\frac{2!}{1! \cdot 1!} \cdot \frac{1}{2!}
\det$\matrixsize{$\begin{pmatrix}\xi_1^{i_1} & \xi_2^{i_1} \\
\xi_1^{i_2} & \xi_2^{i_2}\end{pmatrix}$} \\
$= \det$\matrixsize{$\begin{pmatrix}\xi_1^{i_1} & \xi_2^{i_1} \\
\xi_1^{i_2} & \xi_2^{i_2}\end{pmatrix}$}.
Im Allgemeinen ist
$(e^{i_1} \land \dotsb \land e^{i_k})[\xi_1, \dotsc, \xi_k] =
\det$\matrixsize{$\begin{pmatrix}\xi_1^{i_1} & \dots & \xi_k^{i_1} \\
\vdots & & \vdots \\
\xi_1^{i_k} & \dots & \xi_k^{i_k}\end{pmatrix}$}.

\vspace{3mm}
\linie

\textbf{Was ist $\dim \Omega_k$?} \\
Durch Überlegung (LAAG 2) kommt man auf $\dim \L_k = n^k$ und
$\dim \Omega_k = \binom{n}{k}$. \\
$\L_k$ hat die Basis $e^{j_1} \otimes \dotsb \otimes e^{j_k}$ mit
$j_1, \dotsc, j_k = 1, \dotsc, n$, d.\,h. $n^k$ Möglichkeiten.
Um eine Basis von $\Omega_k$ zu erhalten, müssen alle Basisvektoren entfernt
werden, sodass pro Permutation genau ein Vektor vorkommt.
Eine Basis von $\Omega_k$ ist $e^{i_1} \land \dotsb \land e^{i_k}$ mit
$1 \le i_1 < \dotsb < i_k \le n$, falls $k \le n$. \\
Für $k = n$ gilt $\dim \Omega_n = 1$ und für $k > n$ gilt
$\dim \Omega_k = 0$. \\
Es gilt $\Omega_0 = \L_0 = \{F\colon X^0 \rightarrow \real\} \cong \real$,
d.\,h. $\dim \Omega_0 = 1$.

\textbf{äußere Algebra}:
Die \emph{äußere Algebra} ist definiert als
$\Omega := \bigoplus_{k=0}^n \Omega_k = (\Omega_0, \dotsc, \Omega_n)$. \\
Es gilt $\dim \Omega = \sum_{k=0}^n \dim \Omega_k = 2^n$.

\linie

\textbf{Tangentialraum}:
Seien $S \subset \real^n$ eine $k$-dimensionale Mannigfaltigkeit und $x \in S$
mit lokaler Karte $\varphi \in \C^1(V_k, U_{\varepsilon(x)}^S(x))$
und $\varphi(0) = x$.
Sei $\gamma\colon \left]-\varepsilon, \varepsilon\right[ \rightarrow V_k$
ein Weg mit $\gamma \in \C^1$, $\gamma(0) = 0 \in V_k$.
In diesem Fall ist $\varphi \circ \gamma \in \C^1$ mit
$\varphi \circ \gamma\colon \left]-\varepsilon, \varepsilon\right[
\rightarrow U_{\varepsilon(x)}^S(x)$. \\
$\overrightarrow{t_\gamma} :=
\left.\frac{d(\varphi \circ \gamma)}{d\tau}\right|_{\tau=0}$
bezeichnet einen \emph{Tangentialvektor}. \\
$T_x S := \{\overrightarrow{t_\gamma} \;|\;
\gamma\colon \left]-\varepsilon, \varepsilon\right[ \rightarrow V_k,\;
\gamma \in \C^1,\; \gamma(0) = 0 \in V_k\}$ heißt
\emph{Tangentialraum} der Mannigfaltigkeit im Punkt $x \in S$.
Im Falle einer $2$-dimensionaler Mannigfaltigkeit spricht man auch von einer
\emph{Tangentialebene}.
Anschaulich gesagt berührt diese die Mannigfaltigkeit im Punkt $x$.
Jedoch geht die eigentliche Tangentialebene durch den Ursprung.

\linie
\pagebreak

Sei nun $f\colon D \subset \real^n \rightarrow \real$
Frechet-dif"|ferenzierbar mit $D \subset \real^n$ of"|fen.
In diesem Fall gilt für die Ableitung $\d f|_x = f'(x)$ mit dem Satz von Taylor
$f(x + h) = f(x) + \d f|_x[h] + o(\norm{h})$.
Dabei gilt $\d f[h] = \sp{F, h}$, $F := \nabla f|_x$, d.\,h.
$\d f$ wirkt auf $h \in \real^n = T_x \real^n$.
$\d f$ ist eine $1$-Form auf $T_x \real^n$, also
$\d f \in \Omega_1(T_x \real^n)$.

Jedoch ändert sich i.\,A. $F = F(x) = \nabla f|_x$, falls sich $x$ ändert.
Daher ist es besser, von einer Abbildung
$\d f\colon
D \subset \real^n \rightarrow \Omega_1(T_x \real^n)$, $x \mapsto \d f(x)$
zu sprechen.

Da die Projektion $\pi^j$ stetig und linear ist, gilt $\d \pi^j = \pi^j = e^j$
(hier ist $F = e_j$).
Daher ist $\d f(x)[h] = \frac{\partial f(x)}{\partial x^1} \xi^1 + \dotsb +
\frac{\partial f(x)}{\partial x^n} \xi^n =
\frac{\partial f(x)}{\partial x^1} \d \pi^1[h] + \dotsb +
\frac{\partial f(x)}{\partial x^n} \d \pi^n[h]$ für
$h = (\xi^1, \dotsc, \xi^n)^t$. \\
Mit der Schreibweise $\d x^j := \d \pi^j = e^j$ ergibt sich
$\d f(x) = \frac{\partial f(x)}{\partial x^1} \d x^1 + \dotsb +
\frac{\partial f(x)}{\partial x^n} \d x^n$.

Man definiert nun $1$-Formen als in der Form
$\d f(x) = a_j(x) \d x^j$ darstellbare Abbildungen. \\
Analog verfährt man für $k$-Formen, d.\,h.
$\omega(x) = a_{j_1,\dotsc,j_k}(x) \d x^{j_1} \land \dotsb \land \d x^{j_k}$.

\linie

\textbf{Dif"|ferentialform}:
Eine \emph{$k$-fache Dif"|ferentialform} oder \emph{$k$-Form} auf
$D \subset \real^n$ of"|fen ist eine Abbildung
$\omega\colon D \rightarrow \Omega_k(T_x D, \real)$,
$x \mapsto \omega(x)$. \\
Den Raum aller $k$-fachen Dif"|ferentialformen auf $D \subset \real^n$
bezeichnet man mit $\Omega_k(D)$.

\emph{Beispiel: $0$-Form} \\
Da $\Omega_0(T_x D, \real) \cong \real$,
definiert man sinnvollerweise als $0$-Formen alle Funktionen
$f\colon D \rightarrow \real$.

\emph{Beispiel: $1$-Form} \\
$1$-Formen heißen auch \emph{Energieformen}, denn sie sind von der Form
$\omega_F(x) = \sp{F(x), \cdot}$ mit
$F\colon D \rightarrow \real^n$ einer Abbildung, d.\,h.
$\omega_F(x)[h] = \sp{F(x), h}$.
(Interpretiert man $F$ als Kraft und $h$ als Strecke, so gibt $\omega_F$
die Energie an.)

\emph{Beispiel: $n - 1$-Form} \\
Für eine Abbildung $v\colon D \subset \real^n \rightarrow \real^n$ mit
$D \subset \real^n$ of"|fen definiert man
$\omega_v^{(n-1)}(x)[\xi_1, \dotsc, \xi_{n-1}] :=
\det(v(x), \xi_1, \dotsc, \xi_{n-1})$.
Es gilt $\omega_v^{(n-1)}(x) \in \Omega_{n-1}(T_x D)$.
Da $\omega_v^{(n-1)}(x)[\xi_1, \dotsc, \xi_{n-1}]$ das Volumen des durch
$\xi_1, \dotsc, \xi_{n-1}$ und $v(x)$ aufgespannten Parallelepipeds angibt,
lässt sich dies für $v(x)$ Fließgeschwindigkeit einer Flüssigkeit
als pro Zeiteinheit durchfließende Menge an Flüssigkeit durch das
Parallelepiped aufgespannt von $\xi_1, \dotsc, \xi_{n-1} \in \real^n$
interpretieren.
Man spricht daher auch von einer \emph{Flussform}.

Mithilfe der Schreibweise $\d x^j$ können Dif"|ferentialformen in eine
Koordinatendarstellung gebracht werden.
Für die Energieform $\omega_F$ gilt bspw. $\omega_F(x) = F_j(x) \d x^j$,
für die Flussform mit $n = 3$ und $v(x) = (v^1(x), v^2(x), v^3(x))^t$ gilt
$\omega^2_v(x)[\xi_1, \xi_2] = \det(v(x), \xi_1, \xi_2)$ \\
$=
v^1(x) \det\begin{pmatrix}\xi_1^2 & \xi_2^2 \\ \xi_1^3 & \xi_2^3\end{pmatrix} -
v^2(x) \det\begin{pmatrix}\xi_1^1 & \xi_2^1 \\ \xi_1^3 & \xi_2^3\end{pmatrix} +
v^3(x) \det\begin{pmatrix}\xi_1^1 & \xi_2^1 \\ \xi_1^2 & \xi_2^2\end{pmatrix}$
\\
$= v^1(x) (\d x^2 \land \d x^3)[\xi_1, \xi_2] -
v^2(x) (\d x^1 \land \d x^3)[\xi_1, \xi_2] +
v^3(x) (\d x^1 \land \d x^2)[\xi_1, \xi_2]$. \\
Allgemein gilt
$\omega_v^{(n-1)}(x) = \sum_{i=1}^n (-1)^{i+1} v^i(x)
\d x^1 \land \dotsb \land \d x^{i-1} \land \d x^{i+1}
\land \dotsb \land \d x^n$.

\linie

Sei $f\colon D \subset \real^n \rightarrow \real^n$ mit $D \subset \real^n$
of"|fen und $f \in \C^1$.
Dies ist eine $0$-Form. \\
$\d f = \frac{\partial f(x)}{\partial x^1} \d x^1 + \dotsb +
\frac{\partial f(x)}{\partial x^n} \d x^n$ ist eine $1$-Form.
Analog lässt sich jede $k$-Form \\
$\omega\colon D \subset \real^n \rightarrow \Omega_k(T_x D)$ schreiben als
$\omega(x) = a_{j_1,\dotsc,j_k}(x) \cdot \d x^{j_1} \land \dotsb \land
\d x^{j_k}$.

\textbf{äußeres Dif"|ferential}:
Das \emph{äußere Dif"|ferential} ist eine Abbildung $\d$,
die jeder $k$-Form $\omega$ eine $k + 1$-Form $\d \omega$ zuweist, wobei
für $\omega(x) = a_{j_1,\dotsc,j_k}(x) \cdot \d x^{j_1} \land \dotsb \land
\d x^{j_k}$ gilt, dass \\
$\d \omega(x) := (\d a_{j_1,\dotsc,j_k}(x)) \land \d x^{j_1} \land
\dotsb \land \d x^{j_k} =
\frac{\partial a_{j_1,\dotsc,j_k}(x)}{\partial x^\ell} \d x^\ell \land
\d x^{j_1} \land \dotsb \land \d x^{j_k}$.

Das äußere Dif"|ferential ist linear
($\d(\omega_1 + \omega_2) = \d\omega_1 + \d\omega_2$),
für jede dif"|ferenzierbare Funktion
$f\colon D \subset \real^n \rightarrow \real$ ist $\d f$ wie
oben (also $\d f(x) = \aufspann{F(x), \cdot}$ mit $F(x) = \nabla f|_x$),
es erfüllt die Produktregel $\d(\omega_1 \land \omega_2) =
\d \omega_1 \land \omega_2 + (-1)^k \omega_1 \land \d \omega_2$ für
$\omega_1 \in \Omega_k$, $\omega_2 \in \Omega_\ell$ und
es gilt $\d(\d \omega) = 0$ für $a_{j_1,\dotsc,j_k}(x) \in \C^2$.

\linie
\pagebreak

\emph{Beispiel}:
Das äußere Dif"|ferential einer $0$-Form
$\omega_f^{(0)} := f\colon D \subset \real^3 \rightarrow \real$ ist \\
$\d f = \frac{\partial f}{\partial x^1} \d x^1 +
\frac{\partial f}{\partial x^2} \d x^2 +
\frac{\partial f}{\partial x^3} \d x^3 = \sp{\nabla f, \cdot}$, \\
d.\,h. $\d \omega_f^{(0)} = \omega_F^{(1)} \in \Omega_1(D)$ mit
$\omega_F^{(1)}(x) := F_1(x) \d x^1 + F_2(x) \d x^2 + F_3(x) \d x^3$ und
dem \textbf{Gradienten} $F(x) := \nabla f(x)$.

Für eine $1$-Form $\omega_F^{(1)}$ mit
$\omega_F^{(1)}(x) = F_1(x) \d x^1 + F_2(x) \d x^2 + F_3(x) \d x^3$ und
$F\colon D \subset \real^3 \rightarrow \real^3$ gilt
$\d \omega_F^{(1)} = \left(\frac{\partial F_1}{\partial x^1} \d x^1 +
\frac{\partial F_1}{\partial x^2} \d x^2 +
\frac{\partial F_1}{\partial x^3} \d x^3\right) \land \d x^1 +
\left(\frac{\partial F_2}{\partial x^1} \d x^1 +
\frac{\partial F_2}{\partial x^2} \d x^2 +
\frac{\partial F_2}{\partial x^3} \d x^3\right) \land \d x^2 \;+$ \\
$\left(\frac{\partial F_3}{\partial x^1} \d x^1 +
\frac{\partial F_3}{\partial x^2} \d x^2 +
\frac{\partial F_3}{\partial x^3} \d x^3\right) \land \d x^3$ \\
$= \frac{\partial F_1}{\partial x^2} \d x^2 \land \d x^1 +
\frac{\partial F_1}{\partial x^3} \d x^3 \land \d x^1 +
\frac{\partial F_2}{\partial x^1} \d x^1 \land \d x^2 +
\frac{\partial F_2}{\partial x^3} \d x^3 \land \d x^2 +
\frac{\partial F_3}{\partial x^1} \d x^1 \land \d x^3 +
\frac{\partial F_3}{\partial x^2} \d x^2 \land \d x^3$ \\
$= \left(\frac{\partial F_2}{\partial x^1} -
\frac{\partial F_1}{\partial x^2}\right) \d x^1 \land \d x^2 +
\left(\frac{\partial F_3}{\partial x^1} -
\frac{\partial F_1}{\partial x^3}\right) \d x^1 \land \d x^3 +
\left(\frac{\partial F_3}{\partial x^2} -
\frac{\partial F_2}{\partial x^3}\right) \d x^2 \land \d x^3$, \\
d.\,h. $\d \omega_F^{(1)} = \omega_v^{(2)} \in \Omega_2(D)$ mit
$\omega_v^{(2)}(x) = v_1(x) \d x^2 \land \d x^3 - v_2(x) \d x^1 \land
\d x^3 + v_3(x) \d x^1 \land \d x^2$ und
dem \textbf{Rotor} $v(x) = \rot F(x) := \nabla \times F(x) =
\left(\frac{\partial F_3}{\partial x^2} - \frac{\partial F_2}{\partial x^3},
\frac{\partial F_1}{\partial x^3} - \frac{\partial F_3}{\partial x^1},
\frac{\partial F_2}{\partial x^1} -
\frac{\partial F_1}{\partial x^2}\right)^t$.

Für eine durch $v$ gegebene $2$-Form $\omega_v^{(2)}$ gilt
$\d \omega_v^{(2)} = \left(\frac{\partial v_1}{\partial x^1} +
\frac{\partial v_2}{\partial x^2} + \frac{\partial v_3}{\partial x^3}\right)
\d x^1 \land \d x^2 \land \d x^3$, d.\,h.
$\d \omega_v^{(2)} = \omega_\varrho^{(3)} \in \Omega_3(D)$ mit
$\omega_\varrho^{(3)} = \varrho(x) \d x^1 \land \d x^2 \land \d x^3$ und
der \textbf{Divergenz} \\
$\varrho(x) = \div v(x) := \sp{\nabla, v(x)} =
\frac{\partial v_1(x)}{\partial x^1} +
\frac{\partial v_2(x)}{\partial x^2} + \frac{\partial v_3(x)}{\partial x^3}$.

$\nabla \cdot = \grad$ wandelt ein Skalarfeld in ein Vektorfeld,
$\nabla \times \cdot = \rot$ wandelt ein Vektorfeld in ein Vektorfeld und
$\sp{\nabla, \cdot} = \div$ wandelt ein Vektorfeld in ein Skalarfeld um.

Für $F = \nabla f$, d.\,h. $\omega_F^{(1)} = \d \omega_f^{(0)}$, ist
$\omega_v^{(2)} = \d \omega_F^{(1)} = \d (\d \omega_f^{(0)}) = 0$, d.\,h.
\textbf{$\rot \grad = 0$}. \\
Für $F$ bel. und $v = \rot F$, d.\,h. $\omega_v^{(2)} = \d \omega_F^{(1)}$,
ist $\omega_\varrho^{(3)} = \d \omega_v^{(2)} = \d (\d \omega_F^{(1)}) = 0$,
d.\,h. \textbf{$\div \rot = 0$}.

\linie

\textbf{Pullback}:
Für eine dif"|ferenzierbare Abbildung $\varphi\colon U \rightarrow V$ mit
$U \subset \real^n$, $V \subset \real^m$ und \\
$k \in \natural$
ist das \emph{Pullback} (die zurückgezogene Abbildung) definiert durch \\
$\varphi^\ast\colon \Omega_k(V) \rightarrow \Omega_k(U)$ mit
$(\varphi^\ast \omega)(t)[\tau_1, \dotsc, \tau_k] :=
\omega(\varphi(t))\left[\frac{D\varphi}{Dt} \tau_1, \dotsc,
\frac{D\varphi}{Dt} \tau_k\right]$.

Das Pullback ist linear ($\varphi^\ast(\omega' + \omega'') =
\varphi^\ast \omega' + \varphi^\ast \omega''$,
$\varphi^\ast (\lambda \omega) = \lambda (\varphi^\ast \omega)$,
$\lambda \in \real$). \\
Es gilt $(\psi \circ \varphi)^\ast = \varphi^\ast \circ \psi^\ast$
sowie $\d (\varphi^\ast \omega) = \varphi^\ast (\d \omega)$. \\
Für $m < k \le n$ gilt $\varphi^\ast \omega = 0$
(da $k$-Form auf $\real^m$ mit $m < k$). \\
Ist $\varphi\colon U \rightarrow V$ bijektiv und glatt (d.\,h. $m = n$),
so ist $\varphi^\ast$ ebenfalls bijektiv und
$(\varphi^\ast)^{-1} = (\varphi^{-1})^\ast$. \\
Außerdem gilt $\varphi^\ast(\omega' \land \omega'') =
(\varphi^\ast \omega') \land (\varphi^\ast \omega'')$.

\textbf{Koordinatendarstellung des Pullbacks einer $2$-Form}:
Sei $\omega = \d x^{j_1} \land \d x^{j_2}$.
Dann gilt mit $\xi_1 = \frac{D\varphi}{Dt} \tau_1$ und
$\xi_2 = \frac{D\varphi}{Dt} \tau_2$, dass
$(\varphi^\ast \omega)(t)[\tau_1, \tau_2] = \omega(x) [\xi_1, \xi_2] =
(\d x^{j_1} \land \d x^{j_2}) [\xi_1, \xi_2]$ \\
$= \det$%
\matrixsize{$\begin{pmatrix}\xi_1^{j_1} & \xi_1^{j_2} \\
\xi_2^{j_1} & \xi_2^{j_2}\end{pmatrix}$}
$= \det$%
\matrixsize{$\begin{pmatrix}
\frac{\partial x^{j_1}}{\partial t^{\ell_1}} \tau_1^{\ell_1} &
\frac{\partial x^{j_2}}{\partial t^{\ell_2}} \tau_1^{\ell_2} \\
\frac{\partial x^{j_1}}{\partial t^{\ell_1}} \tau_2^{\ell_1} &
\frac{\partial x^{j_2}}{\partial t^{\ell_2}} \tau_2^{\ell_2}\end{pmatrix}$}
$= \sum_{\ell_1,\ell_2=1}^m \det$%
\matrixsize{$\begin{pmatrix}\tau_1^{\ell_1} & \tau_1^{\ell_2} \\
\tau_2^{\ell_1} & \tau_2^{\ell_2}\end{pmatrix}$}
$\frac{\partial x^{j_1}}{\partial t^{\ell_1}}
\frac{\partial x^{j_2}}{\partial t^{\ell_2}}$ \\
$= \sum_{\ell_1,\ell_2=1}^m (\d t^{\ell_1} \land \d t^{\ell_2})[\tau_1, \tau_2]
\frac{\partial x^{j_1}}{\partial t^{\ell_1}}
\frac{\partial x^{j_2}}{\partial t^{\ell_2}}
= \sum_{1 \le \ell_1 < \ell_2 \le m} \left(
\frac{\partial x^{j_1}}{\partial t^{\ell_1}}
\frac{\partial x^{j_2}}{\partial t^{\ell_2}} -
\frac{\partial x^{j_2}}{\partial t^{\ell_1}}
\frac{\partial x^{j_1}}{\partial t^{\ell_2}}\right)
\d t^{\ell_1} \land \d t^{\ell_2} [\tau_1, \tau_2]$ \\
$= \sum_{1 \le \ell_1 < \ell_2 \le m}
\det\left(\frac{D(x^{j_1}, x^{j_2})}{D(t^{\ell_1}, t^{\ell_2})}\right)
\d t^{\ell_1} \land \d t^{\ell_2} [\tau_1, \tau_2]$. \\
Im Allgemeinen gilt
$\varphi^\ast\left(\sum_{1 \le i_1 < \dotsb < i_p \le n}
a_{i_1, \dotsc, i_p}(x) \d x^{i_1} \land \dotsb \d x^{i_p}\right)$ \\
$= \sum_{1 \le i_1 < \dotsb < i_p \le n} a_{i_1, \dotsc, i_p}(\varphi(t))
\sum_{1 \le \ell_1 < \dotsb < \ell_p \le m}
\det\left(\frac{D(x^{i_1}, \dotsc, x^{i_p})}
{D(t^{\ell_1}, \dotsc, t^{\ell_p})}\right)
\d t^{\ell_1} \land \dotsb \land \d t^{\ell_p}$.

\pagebreak

\section{%
    Integration auf Mannigfaltigkeiten%
}

Gegeben sei eine Kurve im $\real^n$, d.\,h.
$\varphi\colon I = [a, b] \rightarrow \real^n$.
Die Kurve stellt eine eindimensionale Mannigfaltigkeit $S = \varphi([a, b])$
im $\real^n$ dar.
Für jeden Zeitpunkt $t \in I$ entspricht ein Tangentialvektor
$\tau \in T_t \real$ einem Tangentialvektor $\xi = \varphi'(t) \tau \in T_x S$.
Außerdem sei im $\real^n$ ein Kraftfeld
$F\colon \real^n \rightarrow \real^n$ gegeben.
Unterteilt man das Intervall $I$ in einzelne Zeitpunkte $t_i$,
entsprechend die Kurve in Punkte $x_i = \varphi(t_i)$ und
berechnet man dazu die Tangentialvektoren $\xi_i = \varphi'(t_i) \tau_i$,
so kann man die entlang des Weges verrichtete Arbeit approximieren durch
$\Delta A_i \approx \sp{F(x_i), \xi_i}$.
Dabei geht man von konstanter Kurvenrichtung $\xi_i$ und auf dieser Richtung
von konstanter Kraft Kraft $F(x_i)$ aus.
Mithilfe des Pullbacks kann man dies schreiben als
$\Delta A_i \approx \sp{F(\varphi(t_i)), \varphi'(\tau)\tau_i} =
\omega_F^{(1)}(x_i)[\xi_i] = (\varphi^\ast \omega_F^{(1)})(t_i)[\tau_i]$.

Für die Gesamtarbeit gilt $A \approx \sum_i \Delta A_i =
\sum_i (\varphi^\ast \omega)(t_i)[\tau_i] = \sum_i \omega(x_i)[\xi_i]$ mit
$\omega = \omega_F^{(1)}$. \\
Dies kann als eine Art Riemann-Summe gedeutet werden.
Lässt man den Rang der Zerlegung gegen $0$ laufen, so erhält man den exakten
Wert $A = \int_{I = [a,b]} \varphi^\ast \omega = \int_{S = \varphi(I)} \omega$.
Die letzten beiden Terme sind nur Schreibweise, werden aber weiter unten
zur Definition erhoben.
$A$ berechnet sich als \textbf{Wegintegral}
$A = \int_a^b \sp{F(\varphi(t)), \varphi'(t)} \dt =
\int_a^b \left(F_1(\varphi(t)) \frac{d \varphi_1}{\dt} \dt + \dotsb +
F_n(\varphi(t)) \frac{d \varphi_n}{\dt} \dt\right) =
\int_{S = \varphi(I)} (F_1(x) \d x^1 + \dotsb + F_n(x) \d x^n)$.

Auf 2-Mannigfaltigkeiten (z.\,B. im $\real^3$) geschieht dies ähnlich.
Dabei fällt auf, dass in beiden Fällen die Ordnung der Dif"|ferentialform
und die Dimension der Mannigfaltigkeit übereinstimmen.

\linie

\textbf{Integral über Mannigfaltigkeiten}:
Sei $S$ eine $m$-Mannigfaltigkeit, $V \subset \real^m$ eine Teilmenge und
$t = (t^1, \dotsc, t^m)$ die Koordinaten bzgl. der kanonischen Basis,
die positiv orientiert ist.
$S$ besitze einen Atlas, der nur aus einer Karte besteht, d.\,h.
es gibt eine Kartenabbildung $\varphi\colon V \rightarrow S$
mit $S = \varphi(V)$.
Außerdem sei $f\colon V \rightarrow \real$ eine Funktion mit $f \in L^1$. \\
Definiere $\int_V f(t) \dt^1 \land \dotsb \land \dt^m :=
\int_V f(t) d\mu$ (dabei ist die Orientierung von
$dt^1 \land \dotsb \land dt^m$ wichtig).
Nun definiere für eine $m$-Form $\omega$ auf der $m$-Mannigfaltigkeit $S$ das
Integral \\
$\int_S \omega = \int_{\varphi(V)} \omega := \int_V \varphi^\ast \omega =
\int_V f(t) \dt^1 \land \dotsb \land \dt^m$.

Beachte: Die Definition ist unabhängig von der konkreten Parametrisierung
der Karte. \\
Besitzt $S$ nur Atlanten mit mehreren Karten
$\varphi_i\colon V \rightarrow U_i$, so konstruiert man Funktionen
$\chi_i\colon S \rightarrow [0, 1]$ mit $\varphi_i(x) = 0$ für $x \notin U_i$
und $\sum_i \chi_i(x) \equiv 1$ (\emph{Zerlegung der Eins}). \\
Eine Frage, die dabei im Vorhinein beantwortet werden muss, ist, ob solche
Zerlegungen immer existieren (unter welchen Voraussetzungen). \\
Dann definiert man $\int_S \omega := \sum_i \int_{U_i} \chi_i \omega =
\sum_i \int_V \varphi^\ast(\chi_i \omega)$.
Dies geht jedoch nur bei orientierten Mannigfaltigkeiten, sonst ist das
Integral nicht wohldefiniert.

\linie

\textbf{Volumenform}:
Sei $e_1, \dotsc, e_k$ ein orthonomiertes System in $\real^n$ und
$S \subset \real^n$ eine $k$-Mannigfal\-tigkeit, wobei
$e_1, \dotsc, e_k \in T_x S$ positiv orientiert ist.
Die \emph{Volumenform} $\Omega$ soll eine $k$-Form sein, wobei
$\Omega(x)[e_1, \dotsc, e_k] = 1$ sein soll.

\pagebreak

\section{%
    Die Integralformeln der Analysis%
}

\textbf{Formel von \name{Green}}:
Sei $D \subset \real^2$ of"|fen mit den kartesischen Koordinaten $(x, y)$
im $\real^2$.
$\partial D$ setze sich aus stückweise glatten Kurven zusammen und
$P, Q\colon \overline{D} \rightarrow \real$ seien glatte Funktionen. \\
Dann gilt $\iint_{\overline{D}} \left(\frac{\partial Q}{\partial x} -
\frac{\partial P}{\partial y}\right) dxdy = \oint_{\partial D} (P\dx + Q\dy)$.

\emph{Beispiel}:
Für $P(x, y) = -y$ und $Q(x, y) = x$ gilt
$\left(\frac{\partial Q}{\partial x} -
\frac{\partial P}{\partial y}\right) = 2$, d.\,h. \\
$\vol(D) =  \iint_D dx \land dy =
\frac{1}{2} \oint_{\partial D} (-y\dx + x\dy)$.

\emph{Beispiel}:
Für den abgeschlossenen Einheitsball
$D = B = \{(x, y) \in \real^2 \;|\; x^2 + y^2 \le 1\}$ und
$f\colon B \rightarrow B$ glatt gilt
$f(p) = p$ für ein $p \in B$
(Fixpunktsatz von Schauder/Brouwer).

\linie

\textbf{Formel von \name{Gauß}-\name{Ostrogradskij}}:
Sei $D \subset \real^3$ of"|fen mit $\overline{D}$ kompakt.
$\partial D$ sei eine stückweise glatte $2$-Mannigfaltigkeit und
$P, Q, R\colon \overline{D} \rightarrow \real$ seien glatte Funktionen. \\
Dann gilt $\iiint_{\overline{D}} \left(\frac{\partial P}{\partial x} +
\frac{\partial Q}{\partial y} + \frac{\partial R}{\partial z}\right) dxdydz =
\iint_{\partial D} (Pdy \land dz + Qdz \land dx + Rdx \land dy)$.

\textbf{verallgemeinerter Satz von \name{Gauß}}:
Sei $V \subset \real^n$ kompakt mit
abschnittsweise glattem Rand $S = \partial V$
und $F\colon \real^n \rightarrow \real^n$ ein stetig dif"|ferenzierbares
Vektorfeld. \\
Dann gilt $\iiint_V (\nabla \cdot F) dV =
\iint_S\!\!\!\!\!\!\!\!\!\!\!\subset\!\supset (F \cdot n) dS$
mit $n$ dem nach außen gerichteten Einheitsnormalenfeld auf dem Rand $S$.
Man schreibt auch $\int_V \sp{\nabla, F} d\vec{x} =
\int_{\partial V} \sp{F, \vec{n}} \d\sigma$.

\linie

\textbf{Formel von \name{Stokes}}:
Sei $S \subset \real^3$ eine stückweise glatte $2$-Mannigfaltigkeit mit Rand
$\partial S$ und $P, Q, R\colon S \rightarrow \real$ seien glatte Funktionen.
Dann gilt \\
$\iint_S \left(\frac{\partial R}{\partial y} -
\frac{\partial Q}{\partial z}\right) dy \land dz +
\Big(\frac{\partial P}{\partial z} -
\frac{\partial R}{\partial x}\Big) dz \land dx +
\left(\frac{\partial Q}{\partial x} -
\frac{\partial P}{\partial y}\right) dx \land dy =
\int_{\partial S} (P\dx + Q\dy + R\dz)$.

\textbf{verallgemeinerte Formel von \name{Stokes}}:
Sei $S$ eine glatte, orientierbare $k$-Mannigfaltigkeit mit Rand $\partial S$.
Außerdem sei $\omega$ eine glatte $k - 1$-Form auf $S$. \\
Dann gilt $\int_{\partial S} \omega = \int_S \d \omega$.

\section{%
    Elemente der Vektoranalysis%
}

Im Folgenden seien $A\colon D \subset \real^3 \rightarrow \real^3$ ein
Vektorfeld und $\alpha\colon D \subset \real^3 \rightarrow \real$ ein
Skalarfeld im $\real^3$.
Zu Vektorfeldern $A$ und $B$ gehören die Energieform $\omega_A^{(1)}$ mit
$\omega_A^{(1)}(x)[\xi] = \sp{A(x), \xi}_{\real^3}$ sowie
die Flussform $\omega_B^{(2)}$ mit
$\omega_B^{(2)}(x)[\xi_1, \xi_2] = \det(B(x), \xi_1, \xi_2)$.
Außerdem bezeichnen $\cdot$ bzw. $\times$ Skalar- bzw. Kreuzprodukt.

\textbf{Lemma}:
$\omega_{A'}^{(1)} \land \omega_{A''}^{(1)} = \omega_{A' \times A''}^{(2)}$,
$\omega_{A'}^{(1)} \land \omega_{A''}^{(2)} = \omega_{A' \cdot A''}^{(3)}$ \\
(dabei ist $\omega_\alpha^{(3)}(x)[\xi_1, \xi_2, \xi_3] = \alpha(x)
\det(\xi_1, \xi_2, \xi_3)$)

Für $f\colon D \subset \real^3 \rightarrow \real$ gilt
für $\omega_f^{(0)}(x) = f(x)$, dass
$\omega_F^{(1)} = \d \omega_f^{(0)}$ mit $F = \grad f = \nabla f$. \\
Für $A\colon D \subset \real^3 \rightarrow \real^3$ gilt
für $\omega_A^{(1)}(x) = \sp{A(x), \cdot}$, dass
$\omega_B^{(2)} = \d \omega_A^{(1)}$ mit $B = \rot A = \nabla \times A$. \\
Für $H\colon D \subset \real^3 \rightarrow \real^3$ gilt
für $\omega_H^{(2)}(x) = \det(H(x), \cdot, \cdot)$, dass \\
$\omega_\rho^{(3)} = \d \omega_H^{(2)}$ mit $\rho = \div H = \sp{\nabla, H}$.

Bekanntermaßen ist $\rot F = 0$ für $F = \nabla f$ und
$\div H = 0$ für $H = \rot A$.

\textbf{weitere Rechenregeln}:
$\rot(f \cdot A) = (\grad f) \times A + f \cdot \rot A$, \\
$\div(f \cdot A) = (\grad f) \cdot A + f \cdot \div A$, \qquad
$\div(A \times B) = B \cdot \rot A - A \cdot \rot B$

\linie
\pagebreak

\textbf{Formel von \name{Newton}-\name{Leibniz}}:
Seien ein stetig dif"|ferenzierbarer Weg
$\gamma\colon [a, b] \rightarrow \real^n$ mit $\Gamma = \gamma([a, b])$
und eine Energieform $\omega_F^{(1)}$ mit $F = \nabla f$ für eine Funktion
$f\colon \real^n \rightarrow \real$ gegeben. \\
Dann gilt $\int_\Gamma \omega_F^{(1)} =
\int_a^b \sp{F \circ \gamma, \dot{\gamma}} \dt =
(f \circ \gamma)(b) - (f \circ \gamma)(a)$. \\
Insbesondere gilt $\int_{\Gamma_1} \omega_F^{(1)} =
\int_{\Gamma_2} \omega_F^{(1)}$, falls $F = \nabla f$.

\linie

\textbf{Formel von \name{Stokes}}:
Sei $A\colon \real^3 \rightarrow \real^3$ ein glattes Vektorfeld und
$S \subset \real^3$ eine $2$-dimensionale Mannigfaltigkeit.
Dann gilt $\iint_S \rot \vec{A} \cdot d\vec{\sigma} =
\oint_{\partial S} \vec{A} d\vec{s}$.

Weitere Varianten sind
$\int_S d\vec{\sigma} (\nabla \times A) =
\int_{\partial S} d\vec{s} \cdot \vec{A}$, \\
$\int_S d\vec{\sigma} \times (\nabla \times A) =
\int_{\partial S} d\vec{s} \times \overline{A}$ \quad und \quad
$\int_S d\vec{\sigma} \times \nabla f =
\int_{\partial S} d\vec{s} f$. \\
Die letzten beiden Formeln kann man sich mit der Schreibweise
$\int_S d\vec{\sigma} \times \nabla = \int_{\partial S} d\vec{s}$
merken.

\linie

\textbf{Formel von \name{Gauß}-\name{Ostrogradskij}}:
Seien $B\colon \real^3 \rightarrow \real^3$ ein Vektorfeld und
$V \subset \real^3$ of"|fen mit stückweise glattem Rand $\partial V$.
Dann gilt $\iiint_V \div \vec{B} dV =
\iint_{\partial V} \vec{B} \cdot d\vec{s}$.

Weitere Varianten sind
$\int_V dV \nabla \cdot \vec{B} =
\int_{\partial V} d\vec{\sigma} \cdot \vec{B}$, \\
$\int_V dV (\nabla \times \vec{B}) =
\int_{\partial V} d\vec{\sigma} \cdot \vec{B}$ \quad und \quad
$\int_V dV \nabla f = \int_{\partial V} d\vec{\sigma} \cdot f$. \\
Alle drei Formeln kann man sich mit der Schreibweise
$\int_V dV \nabla = \int_{\partial V} d\vec{\sigma}$ merken.

\linie

\textbf{Formel von \name{Green}}:
Seien $V \subset \real^n$ mit stückweise glattem Rand $\partial V$ und
$f, g\colon \real^n \rightarrow \real$ Funktionen.
Dann gilt $\int_V (\nabla f) \cdot (\nabla g) dV =
\int_{\partial V} g \nabla f d\vec{\sigma} - \int_V g \Delta f dV$ sowie \\
$\int_V (g \Delta f - f \Delta g) dV =
\int_{\partial V} (g \nabla f - f \nabla g) d\vec{\sigma}$.

\section{%
    Zur Theorie der Potentialfelder%
}

Im Folgenden sei $D \subset \real^n$ of"|fen mit stückweise glattem Rand,
$A\colon D \rightarrow \real^n$ ein Vektorfeld und
$U\colon D \rightarrow \real$ ein Skalarfeld.

\textbf{Potential}:
Ein Skalarfeld $U$ ist das \emph{Potential} vom Vektorfeld $A$ in $D$,
falls $A = \nabla U$. \\
$A$ heißt in diesem Fall \emph{konservatives Feld} oder \emph{Gradientenfeld}.

Es gibt nicht zu jedem Vektorfeld $\vec{A}$ ein Potential $U$.
Eine notwendige Bedingung ist \\
$\rot \vec{A} = \rot \grad U = 0$.
Allerdings ist diese Bedingung nicht hinreichend.

\textbf{Satz}:
$\vec{A}$ ist konservativ genau dann, wenn $\oint_\Gamma \vec{A} d\vec{s} = 0$
für jeden geschlossenen Pfad $\Gamma$ gilt.

\linie

\textbf{exakt}: \\
Eine $k$-Form $\omega \in \Omega_k(D)$ heißt \emph{exakt}, falls
es eine $k - 1$-Form $\widetilde{\omega} \in \Omega_{k-1}(D)$ gibt mit
$\omega = \d\widetilde{\omega}$.

\textbf{abgeschlossen}:
Eine $k$-Form $\omega \in \Omega_k(D)$ heißt \emph{abgeschlossen}, falls
$\d\omega = 0$.

Jede exakte Form ist auch abgeschlossen.
Die Umkehrung gilt i.\,A. nicht.

\textbf{Lemma von Poincaré}:
Ist $D$ homotop zur Kugel (d.\,h. $D$ hat keine "`Löcher"'),
dann ist jede abgeschlossene Form auch exakt.

\linie

Eine notwendige Bedingung, dass es zu
$\vec{B}\colon D \subset \real^3 \rightarrow \real^3$ ein Vektorpotential
$\vec{A}$ mit $\vec{B} = \rot \vec{A}$ gibt, ist, dass $\div \vec{B} = 0$
erfüllt ist.
Enthält $D$ keine "`Löcher"', dann ist diese Bedingung sogar hinreichend.

\pagebreak
