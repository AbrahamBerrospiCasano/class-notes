\section{%
    Kompakte Operatoren und adjungierte Operatoren auf Banachräumen%
}

\subsection{%
    \name{Jordan}sche Normalform für kompakte Operatoren%
}

\begin{Satz}{\name{Jordan}sche Normalform für kompakte Operatoren}\\
    Seien $X$ ein Banachraum und $T \in \K(X)$.
    Dann gilt:
    \begin{enumerate}
        \item
        Die Aussagen aus dem Satz über das Spektrum kompakter Operatoren gelten, d.\,h.\\
        $\sigma(T) \setminus \{0\} = \sigma_p(T) \setminus \{0\}$
        besteht aus höchstens abzählbar vielen Eigenwerten
        mit $0$ als einzigem Häufungspunkt (falls $|\sigma(T)| = \infty$),
        die Vielfachheit von $\lambda \in \sigma(T) \setminus \{0\}$ ist endlich und
        für $\dim X = \infty$ ist $0 \in \sigma(T)$.
        
        \item
        Für $\lambda \in \sigma(T) \setminus \{0\}$ und
        $n_\lambda := \max\{n \in \natural \;|\; \Kern((\lambda\id - T)^{n-1}) \not=
        \Kern((\lambda\id - T)^n)\}$ der \begriff{Ordnung} von $\lambda$ gilt
        $1 \le n_\lambda < \infty$.
        
        \item
        Für $\lambda \in \sigma(T) \setminus \{0\}$ gilt
        $X = \Kern((\lambda\id - T)^{n_\lambda}) \oplus \Bild((\lambda\id - T)^{n_\lambda})$
        (\begriff{\name{Riesz}-Zerlegung}).
        Beide Unterräume sind abgeschlossen und $T$-invariant.
        Der \begriff{charakteristische Unterraum} $\Kern((\lambda\id - T)^{n_\lambda})$
        von $T$ zum Eigenwert $\lambda$ ist endlich-dimensional.
        
        \item
        Für $\lambda \in \sigma(T) \setminus \{0\}$ gilt:
        Für $n = 1, \dotsc, n_\lambda$ gibt es Unterräume
        $E_n \subset \Kern((\lambda\id - T)^n)$ mit
        $E_n \cap \Kern((\lambda\id - T)^{n-1}) = \{0\}$, sodass
        $\Kern((\lambda\id - T)^{n_\lambda}) = \bigoplus_{k=1}^{n_\lambda} N_k$ mit\\
        $N_k := \bigoplus_{\ell=0}^{k-1} (\lambda\id - T)^\ell (E_k)$.
        
        \item
        $N_k$, $k = 1, \dotsc, n_\lambda$, ist $T$-invariant und die Dimensionen
        $d_k := \dim((\lambda\id - T)^{\ell} (E_k))$ sind unabhängig von
        $\ell \in \{0, \dotsc, k - 1\}$.
        
        \item
        Ist $\{e_{k,j} \;|\; j = 1, \dotsc, d_k\}$ eine Basis von $E_k$ für
        $k = 1, \dotsc, n_\lambda$, dann ist\\
        $\{(\lambda\id - T)^\ell e_k \;|\; 0 \le \ell < k \le n_\lambda,\; 1 \le j \le d_k\}$
        eine Basis von $\Kern((\lambda\id - T)^{n_\lambda})$.\\
        Mit $x = \sum_{k,j,\ell} \alpha_{k,j,\ell} (\lambda\id - T)^\ell e_{k,j}$
        und $y = \sum_{k,j,\ell} \beta_{k,j,\ell} (\lambda\id - T)^\ell e_{k,j}$ gilt\\
        $Tx = y \iff \smallpmatrix{\lambda & -1 & & \\ & \ddots & \ddots & \\
        & & \lambda & -1 \\ & & & \lambda}
        \smallpmatrix{\alpha_{k,j,0} \\ \vdots \\ \alpha_{k,j,k-1}} =
        \smallpmatrix{\beta_{k,j,0} \\ \vdots \\ \beta_{k,j,k-1}}$.
    \end{enumerate}
\end{Satz}

\vspace{3mm}
\linie

\begin{Kor}
    Seien $X$ ein Banachraum und $T \in \K(X)$.
    Dann gilt:
    \begin{enumerate}
        \item
        Für $\lambda \in \sigma(T) \setminus \{0\}$ gilt
        $\sigma(T|_{\Bild((\lambda\id - T)^{n_\lambda})}) = \sigma(T) \setminus \{\lambda\}$.
        
        \item
        Ist $P_\lambda$ für $\lambda \in \sigma(T) \setminus \{0\}$ die stetige Projektion
        auf $\Kern((\lambda\id - T)^{n_\lambda})$ gemäß der Riesz-Zerlegung,
        dann gilt $\forall_{\lambda, \mu \in \sigma(T) \setminus \{0\}}\;
        P_\lambda P_\mu = \delta_{\lambda\mu} P_\lambda$.
    \end{enumerate}
\end{Kor}

\linie

\begin{Kor}
    Seien $X$ ein Banachraum, $T \in \K(X)$ und $\lambda \in \sigma(T) \setminus \{0\}$.
    Dann hat die Resolventenfunktion $R(\cdot, T)$ in $\lambda$ einen isolierten Pol
    der Ordnung $n_\lambda$, d.\,h.
    $\mu \mapsto (\mu - \lambda)^{n_\lambda} R(\mu, T)$ kann in $\lambda$ holomorph fortgesetzt
    werden und der fortgesetzte Wert in $\lambda$ ist ungleich Null.
\end{Kor}

\pagebreak

\subsection{%
    Adjungierter Operator%
}

\begin{Def}{Adjungierte}
    Seien $X, Y$ normierte Räume und $T \in \Lin(X, Y)$.\\
    Dann heißt der Operator $T' \in \Lin(Y', X')$ definiert durch
    $(T'y')(x) := y'(Tx)$ für $y' \in Y'$ und $x \in X$ der zu $T$ \begriff{adjungierte Operator}.
\end{Def}

\begin{Satz}{Eigenschaften der Adjungierten}
    \begin{enumerate}
        \item
        $T \mapsto T'$ ist eine lineare, isometrische Einbettung
        von $\Lin(X, Y)$ nach $\Lin(Y', X')$.
        
        \item
        Seien $X, Y, Z$ normierte Räume, $T_1 \in \Lin(X, Y)$ und $T_2 \in \Lin(Y, Z)$.\\
        Dann ist $(T_2 T_1)' = T_1' T_2'$.
        
        \item
        Seien $J_X\colon X \rightarrow X''$, $x_0 \mapsto J_{x_0}$ mit
        $J_{x_0}(x') := x'(x_0)$ für $x' \in X'$ und analog $J_Y\colon Y \rightarrow Y''$.\\
        Dann gilt $T'' J_X = J_Y T$.
    \end{enumerate}
\end{Satz}

\begin{Bsp}
    \begin{enumerate}[label=\emph{(\alph*)}]
        \item
        Für $X = Y = \real^n$ mit der euklidischen Norm und
        $T = (a_{ij})_{i,j=1,\dotsc,n}$ ist\\
        $T' = (a_{ji})_{i,j=1,\dotsc,n} = T^\ast$,
        wobei $T^\ast$ die Hilbertraum-Adjungierte ist.
        
        \item
        Für $X = Y = \complex^n$ mit der euklidischen Norm und
        $T = (a_{ij})_{i,j=1,\dotsc,n}$ ist\\
        $T' = (a_{ji})_{i,j=1,\dotsc,n} \not= (\overline{a_{ji}})_{i,j=1,\dotsc,n} = T^\ast$.
        
        \item
        Für $X = Y = L^2([0,1], \complex)$ und
        $T\colon X \rightarrow X$,
        $(Tf)(y) := \int_0^1 K(x, y)f(x)\dx$ ist\\
        $(T'g)(x) := \int_0^1 K(x, y)g(y)\dy$
        (nicht gleich
        $(T^\ast g)(x) = \int_0^1 \overline{K(x, y)} g(y) \dy$).
        
        \item
        Sind $X, Y$ Hilberträume und $\R_X\colon X \rightarrow X'$ und
        $\R_Y\colon Y \rightarrow Y'$ die Isometrien aus dem Rieszschen Darstellungssatz
        (z.\,B. $(\R_X x_1)(x_2) := \sp{x_2, x_1}_X$),
        dann gilt $T^\ast = \R_X^{-1} T' \R_Y$.\\
        Für $x \in X$ und $y \in Y$ gilt nämlich
        $((T' \R_Y)(y))(x)
        = (T'(\R_Y y))(x)
        = (\R_Y y)(Tx)$\\
        $= \sp{Tx, y}_Y
        = \sp{x, T^\ast y}_X
        = (\R_X (T^\ast y))(x)
        = ((\R_X T^\ast)(y))(x)$.
    \end{enumerate}
\end{Bsp}

\pagebreak

\subsection{%
    \name{Fredholm}sche Alternative%
}

\begin{Satz}{Satz von \scshape\,\!\name{Schauder}}
    Seien $X, Y$ Banachräume und $T \in \Lin(X, Y)$.\\
    Dann gilt $T \in \K(X, Y)$ genau dann, wenn $T' \in \K(Y', X')$.
\end{Satz}

\linie

\begin{Def}{Annihilator}
    Seien $X$ ein Banachraum und $Z \subset X$ ein Unterraum.\\
    Dann heißt $Z^\circ := \{x' \in X' \;|\; x'|_Z = 0\}$
    \begriff{Annihilator} von $Z$.
\end{Def}

\begin{Def}{Kodimension}
    Seien $X$ ein $\KK$-Vektorraum und $Z \subset X$ ein Unterraum.\\
    Dann ist $\codim Z := \dim X/Z$ die \begriff{Kodimension} von $Z$ in $X$.
\end{Def}

\begin{Bem}
    Ist $Y$ ein Komplement von $Z$ in $X$ (d.\,h. $X = Y \oplus Z$), dann gilt
    $\codim Z = \dim Y$.
    %Komplemente existieren immer, allerdings sind sie i.\,A. nicht eindeutig.
\end{Bem}

\begin{Satz}{Eigenschaften des Annihilators}
    Seien $X, Y$ Banachräume und $Z \subset X$ ein Unterraum.
    \begin{enumerate}
        \item
        Ist $X$ ein Hilbertraum, dann ist $Z^\circ = \R_X(Z^\orth)$.
        
        \item
        Für $T \in \Lin(X, Y)$ gilt $\Kern(T') = \Bild(T)^\circ$.
        
        \item
        Ist $Z$ abgeschlossen und $\codim Z < \infty$, dann ist $\dim Z^\circ = \codim Z$.
    \end{enumerate}
\end{Satz}

\linie

\begin{Satz}{Inverse der Adjungierten}
    Seien $X, Y$ Banachräume und $T \in \Lin(X, Y)$.\\
    Dann existiert $T^{-1} \in \Lin(Y, X)$ genau dann,
    wenn $(T')^{-1} \in \Lin(X', Y')$ existiert.\\
    In diesem Fall gilt $(T^{-1})' = (T')^{-1}$.
\end{Satz}

\linie

\begin{Satz}{\name{Fredholm}sche Alternative}
    Seien $X$ ein Banachraum, $T \in \K(X)$ und
    $\lambda \in \KK \setminus \{0\}$.\\
    Dann gilt:
    Zu $y \in X$ besitzt die Gleichung $Tx - \lambda x = y$ eine Lösung $x \in X$ genau dann,
    wenn $x'(y) = 0$ für alle Lösungen $x' \in X'$ der
    \begriff{homogenen adjungierten Gleichung} $T'x' - \lambda x' = 0$ gilt.
    Die dadurch gegebene endliche Anzahl der Nebenbedingungen an $y$ ist gleich der Anzahl
    linear unabhängiger Lösungen $z$ der \begriff{homogenen Gleichung} $Tz - \lambda z = 0$.
\end{Satz}

\begin{Bem}
    Der Satz lässt sich auch wie folgt formulieren:
    Entweder
    \begin{itemize}
        \item
        $Tz - \lambda z = 0$ besitzt nur die triviale Lösung,
        
        \item
        $T'x' - \lambda x' = 0$ besitzt nur die triviale Lösung und
        
        \item
        $Tx - \lambda x = y$ ist für alle $y \in Y$ eindeutig lösbar
    \end{itemize}
    oder
    \begin{itemize}
        \item
        $Tz - \lambda z = 0$ besitzt $n := \dim(\Kern(\lambda\id - T))$
        ($1 \le n < \infty$) linear unabhängige Lösungen,
        
        \item
        $T'x' - \lambda x' = 0$ besitzt $n$ linear unabhängige Lösungen und
        
        \item
        $Tx - \lambda x = y$ ist für $y \in Y$ genau dann lösbar, wenn
        $x'(y) = 0$ für alle $x' \in \Kern(\lambda\id' - T')$.
    \end{itemize}
\end{Bem}

\pagebreak
