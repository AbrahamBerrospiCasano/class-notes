\chapter{%
    Der Spektralsatz für kompakte, selbstadjungierte Operatoren%
}

\section{%
    \name{Hilbert}raum-Adjungierte%
}

\begin{Bem}
    Seien $H_1, H_2$ Hilberträume und $T \in \Lin(H_1, H_2)$.
    Für $y \in H_2$ ist die Abbildung $x \mapsto \innerproduct{Tx, y}_{H_2}$ ein Element des Dualraums
    von $H_1$.
    Nach dem Rieszschen Darstellungssatz gibt es daher genau ein $T^\ast y \in H_1$ mit
    $\forall_{x \in H_1}\; \innerproduct{Tx, y}_{H_2} = \innerproduct{x, T^\ast y}_{H_1}$.
    Somit existiert die Hilbertraum-Adjungierte $T^\ast$ und ist eindeutig.
\end{Bem}

\begin{Def}{\name{Hilbert}raum-Adjungierte}
    Seien $H_1, H_2$ Hilberträume und $T \in \Lin(H_1, H_2)$.
    Dann heißt die Abbildung $T^\ast\colon H_2 \rightarrow H_1$ mit
    $\forall_{x \in H_1,\, y \in H_2}\; \innerproduct{Tx, y}_{H_2} = \innerproduct{x, T^\ast y}_{H_1}$
    \begriff{\name{Hilbert}raum-Adjungierte} von $T$.
\end{Def}

\begin{Lemma}{Eigenschaften der \name{Hilbert}raum-Adjungierten}
    \begin{enumerate}
        \item
        $T^\ast \in \Lin(H_2, H_1)$ mit $\norm{T^\ast} = \norm{T}$

        \item
        $(T + S)^\ast = T^\ast + S^\ast$, $(\alpha T)^\ast = \overline{\alpha} T^\ast$,
        $(T \circ S)^\ast = S^\ast \circ T^\ast$

        \item
        $T^{\ast\ast} = T$
    \end{enumerate}
\end{Lemma}

\linie

\begin{Bsp}
    \begin{enumerate}[label=\emph{(\alph*)}]
        \item
        Für $H_1 = H_2 = \real^n$ (mit eukl. Skalarprodukt) und
        $T = (a_{ij})_{i,j=1,\dotsc,n}$ ist $T^\ast = (a_{ji})_{i,j=1,\dotsc,n}$.

        \item
        Für $H_1 = H_2 = \complex^n$ ist $T^\ast = (\overline{a_{ji}})_{i,j=1,\dotsc,n}$.

        \item
        Für $H_1 = H_2 = \ell^2_\real$ und
        $T((x_n)_{n \in \natural}) := (a_n x_n)_{n \in \natural}$ für eine Folge
        $(a_n)_{n \in \natural}$, $\sup_{n \in \natural} |a_n| < \infty$, ist
        $T^\ast((x_n)_{n \in \natural}) = (a_n x_n)_{n \in \natural} = T((x_n)_{n \in \natural})$.

        \item
        Für $H_1 = H_2 = \ell^2_\complex$ ist
        $T^\ast((x_n)_{n \in \natural}) = (\overline{a_n} x_n)_{n \in \natural}$.
    \end{enumerate}
\end{Bsp}

\linie

\begin{Bsp}
    Seien $H_1 = L^2(\Omega_1, \complex)$ und $H_2 = L^2(\Omega_2, \complex)$,
    wobei $\Omega_1 \subset \real^n$ und $\Omega_2 \subset \real^m$ messbar seien.
    Außerdem sei $K\colon \Omega_1 \times \Omega_2 \rightarrow \complex$ messbar
    mit $\norm{K} := \left(\int_{\Omega_1} \int_{\Omega_2}
    |K(x, y)|^2 \dy \dx\right)^{1/2} < \infty$.
    Sei für $f \in L^2(\Omega_1, \complex)$ die Abbildung $Tf\colon \Omega_2 \rightarrow \complex$
    definiert durch $(Tf)(y) := \int_{\Omega_1} K(x, y)f(x)\dx$.\\
    Dann ist $T \in \Lin(H_1, H_2)$ und $\norm{T} \le \norm{K}$.
    Außerdem gilt $(T^\ast g)(x) = \int_{\Omega_2} \overline{K(x, y)} g(y) \dy$,
    wenn $n = m$ und $\Omega_1 = \Omega_2$.

    Dies sieht man wie folgt:
    Es gilt $\norm{Tf}_{H_2}^2 = \int_{\Omega_2} |(Tf)(y)|^2 \dy
    = \int_{\Omega_2} \left|\int_{\Omega_1} K(x, y)f(x)\dx\right|^2 \dy\\
    %$\le \int_{\Omega_2} \left|\int_{\Omega_1} |K(x, y)| |f(x)| \dx\right|^2 \dy
    = \int_{\Omega_2} \left|\innerproduct{K(\cdot, y), f}_{H_1}\right|^2 \dy
    \le \int_{\Omega_2} \norm{K(\cdot, y)}_{H_1}^2 \norm{f}_{H_1}^2 \dy
    = \norm{K}^2 \norm{f}_{H_1}^2$,
    also $Tf \in H_2$,\\
    $T \in \Lin(H_1, H_2)$ und $\norm{T} \le \norm{K}$.
    Die Adjungierte $T^\ast$ erhält man durch direktes Nachrechnen
    (wobei man die konjugierte Linearität im zweiten Argument beachten muss).
    Ersetzt man $\complex$ durch $\real$, so ist $T^\ast = T$.
\end{Bsp}

\linie

\begin{Def}{selbstadjungiert}
    Sei $H$ ein Hilbertraum.\\
    Dann heißt $T \in \Lin(H)$ \begriff{selbstadjungiert}, falls $T^\ast = T$
    (d.\,h. $\forall_{x, y \in H}\; \innerproduct{Tx, y} = \innerproduct{x, Ty}$).
\end{Def}

\begin{Bem}
    Ist $T \in \Lin(H)$ selbstadjungiert, so gilt für $x = y$, dass
    $\innerproduct{Tx, x} = \innerproduct{x, Tx} = \overline{\innerproduct{Tx, x}}$,
    also $\innerproduct{Tx, x} = \innerproduct{x, Tx} \in \real$ für alle $x \in H$.
    Manchmal ist Selbstadjungiertheit eine zu starke Eigenschaft,
    in diesem Fall verwendet man die Verallgemeinerung von normalen Abbildungen.
\end{Bem}

\begin{Def}{normal}
    Sei $H$ ein Hilbertraum.
    Dann heißt $T \in \Lin(H)$ \begriff{normal}, falls $T^\ast T = TT^\ast$.
\end{Def}

\begin{Lemma}{Charakterisierung}
    $T$ ist normal genau dann, wenn
    $\forall_{x \in H}\; \norm{Tx} = \norm{T^\ast x}$.
\end{Lemma}

\pagebreak

\section{%
    Kompakte Operatoren%
}

\begin{Def}{kompakter Operator}
    Seien $E, F$ Banachräume.\\
    Dann heißt $T \in \Lin(E, F)$ \begriff{kompakt}, falls
    $\overline{T B_E} = \overline{\{Tx \;|\; \norm{x}_E \le 1\}}$ kompakt in $F$ ist.\\
    Äquivalent dazu sind:
    \begin{enumerate}
        \item
        Für jede Folge $(x_n)_{n \in \natural}$ in $B_E$ besitzt $(Tx_n)_{n \in \natural}$
        eine konvergente Teilfolge in $F$.

        \item
        Für alle $\varepsilon > 0$ gibt es eine endliche Menge $M \subset F$ mit
        $TB_E \subset M + \varepsilon B_F$.
    \end{enumerate}
    Die \begriff{Menge aller kompakten Operatoren} von $E$ nach $F$ bezeichnet man mit
    $\K(E, F)$ und man schreibt $\K(E) := \K(E, E)$.
\end{Def}

\begin{Bem}
    Ist $X$ ein Banachraum, dann gilt $\id \in \K(X) \iff X \;\text{endl.-dim.}$
\end{Bem}

\begin{Lemma}{$\K(E, F)$ abg. UVR}
    $\K(E, F)$ ist ein abgeschlossener Unterraum von $\Lin(E, F)$.
\end{Lemma}

\linie

\begin{Def}{Operator mit endlichem Rang}
    Seien $E, F$ Banachräume.
    Die \begriff{Menge aller Operatoren mit endlichem Rang} ist definiert durch
    $\F(E, F) := \{T \in \Lin(E, F) \;|\; \dim TE < \infty\}$.
\end{Def}

\begin{Bsp}
    \begin{enumerate}[label=\emph{(\alph*)}]
        \item
        Für $T \in \F(E, F)$ gilt $T \in \K(E, F)$, denn
        $TB_E$ ist beschränkt in $TE$ (für alle $x \in B_E$ gilt
        $\norm{Tx}_F \le \norm{T} \norm{x}_E \le \norm{T}$) und
        somit ist $\overline{TB_E}$ beschränkt und abgeschlossen.
        Damit ist $\overline{TB_E}$ kompakt in $TE$ (wegen $\dim TE < \infty$)
        und insbesondere kompakt in $F$.

        \item
        Für $\dim E < \infty$ ist $\dim TE < \infty$ für alle $T \in \Lin(E, F)$,
        also gilt\\
        $\Lin(E, F) \subset \F(E, F) \subset \K(E, F) \subset \Lin(E, F)$,
        d.\,h. jeder lineare, stetige Operator ist kompakt, wenn $E$ endlich-dimensional ist.

        \item
        Es gilt $\overline{\F(E, F)} \subset \K(E, F)$, weil $\K(E, F)$ abgeschlossen ist.
    \end{enumerate}
\end{Bsp}

\begin{Bem}
    Lange war ungeklärt, ob die Umkehrung auch gilt,
    d.\,h. ob $\overline{\F(E, F)} = \K(E, F)$.
    Die Frage war also, ob jeder kompakte Operator durch Operatoren von endlichem Rang
    approximiert werden kann.
    Per Enflo konnte als Erster ein Gegenbeispiel liefern (1973).
    Allerdings stimmt die Aussage, wenn $F$ ein Hilbertraum ist.
\end{Bem}

\begin{Lemma}{kpkt.e Operatoren als GW von Operatoren mit endl. Rang}\\
    Seien $E$ ein Banachraum und $F$ ein Hilbertraum.
    Dann gilt $\overline{\F(E, F)} = \K(E, F)$.
\end{Lemma}

\linie
\pagebreak

\begin{Bsp}
    Obiger Integraloperator $T \in \Lin(H_1, H_2)$ ist kompakt.
    Wählt man ein vollständiges ONS $(e_k)_{k \in \natural}$ von $H_1$, dann gilt nach
    Parseval
    $\norm{K}^2 = \int_{\Omega_2} \norm{\overline{K(\cdot, y)}}_{H_1}^2 \dy$\\
    $= \int_{\Omega_2} \sum_{k \in \natural}
    \left|\innerproduct{\overline{K(\cdot, y)}, e_k}_{H_1}\right|^2 \dy
    = \int_{\Omega_2} \sum_{k \in \natural} |(Te_k)(y)|^2 \dy
    = \sum_{k \in \natural} \norm{Te_k}_{H_2}^2$.\\
    Sei $P_n$ die orthogonale Projektion von $H_1$ auf $[e_1, \dotsc, e_n]$, d.\,h.
    $P_n f := \sum_{k=1}^n \innerproduct{f, e_k}_{H_1} e_k$.\\
    Dann gilt
    $\norm{(T - TP_n) f}_{H_2}^2
    = \norm{T (f - P_n f)}_{H_2}^2
    = \norm{T\!\left(\sum_{k > n} \innerproduct{f, e_k}_{H_1} e_k\right)}_{H_2}^2$\\
    $= \norm{\sum_{k > n} \innerproduct{f, e_k}_{H_1} Te_k}_{H_2}^2
    \le \left(\sum_{k > n} \left|\innerproduct{f, e_k}_{H_1}\right| \norm{Te_k}_{H_2}\right)^2$\\
    $\le \sum_{k > n} \left|\innerproduct{f, e_k}_{H_1}\right|^2 \cdot \sum_{k > n} \norm{Te_k}_{H_2}^2$
    wegen der Cauchy-Schwarz-Ungleichung für $\ell^2$.
    Der erste Faktor ist mit Parseval durch $\norm{f}_{H_1}^2$ nach oben beschränkt, während
    der zweite für $n \to \infty$ gegen Null geht
    (weil $\sum_{k \in \natural} \norm{Te_k}_{H_2}^2 = \norm{K}^2 < \infty$).
    Damit gilt\\
    $\norm{(T - TP_n) f}_{H_2}^2 \le \sum_{k > n} \norm{Te_k}_{H_2}^2
    \norm{f}_{H_1}^2$
    und somit $\norm{T - TP_n}^2 \le \sum_{k > n} \norm{Te_k}_{H_2}^2 \to 0$
    für $n \to \infty$.
    Wegen $\Bild(TP_n) = T(\Bild(P_n))$ endlich-dimensional für alle $n \in \natural$ ist
    $T$ kompakt.
\end{Bsp}

\begin{Bem}
    Man kann bei Vorhandensein entsprechender Integrabilität von $K$ auch Integraloperatoren
    $T^{p,q} \in \Lin(L^p(\Omega_1, \KK), L^q(\Omega_2, \KK))$ für $\frac{1}{p} + \frac{1}{q} = 1$
    bekommen.
    Auch sie sind stetig (Nachweis mit Hölder statt Cauchy-Schwarz, ähnlich wie für $p = q = 2$)
    und kompakt (Nachweis mithilfe von Fréchet-Kolmogorov, Riesz).
    Man nennt diese Operatoren\\
    \begriff{\name{Hilbert}-\name{Schmidt}-Integraloperatoren}.
\end{Bem}

\linie

\begin{Lemma}{Komposition kpkt.}
    Seien $X, Y, Z$ Banachräume, $T_1 \in \Lin(X, Y)$ und $T_2 \in \Lin(Y, Z)$.\\
    Dann folgt aus $T_1$ kompakt oder $T_2$ kompakt, dass $T_2 T_1$ kompakt ist.
\end{Lemma}

\begin{Bem}
    Algebraisch lässt sich das für $X = Y = Z$ wie folgt ausdrücken:
    Mit der Verkettung $\circ$ als Multiplikation ist der Vektorraum
    $(\Lin(X), +, \circ)$ eine nicht-kommutative Algebra,
    d.\,h. $\circ$ ist assoziativ (aber i.\,A. nicht-kommutativ),
    $+$ und $\circ$ sind distributiv und für alle $\alpha \in \KK$ gilt
    $\alpha (f \circ g) = (\alpha f) \circ g = f \circ (\alpha g)$.
    Für $S, T \in \Lin(X)$ gilt außerdem $\norm{S \circ T} \le \norm{S} \cdot \norm{T}$.
    Ein Banachraum, der eine Algebra ist und dessen Multiplikation diese Beziehung erfüllt,
    heißt \begriff{\name{Banach}algebra}.
    $(\Lin(X), +, \circ)$ ist also eine Banachalgebra und obiges Lemma besagt nun,
    dass $\K(X)$ ein Ideal in $\Lin(X)$ ist.
\end{Bem}

\linie

\begin{Satz}{Eigenwerte kompakter Operatoren}\\
    Seien $X$ ein Banachraum, $T \in \K(X)$ und $\lambda \in \KK \setminus \{0\}$.
    Dann gilt:
    \begin{enumerate}
        \item
        $\dim \Kern(\lambda\id - T) < \infty$

        \item
        $\Bild(\lambda\id - T) \subset X$ abgeschlossen

        \item
        $\lambda\id - T$ injektiv $\iff$ $\lambda\id - T$ surjektiv
    \end{enumerate}
\end{Satz}

\pagebreak

\section{%
    Das Spektrum linearer Abbildungen über Banachräumen%
}

\begin{Bem}
    Im Folgenden seien $X$ ein $\complex$-Banachraum und $T \in \Lin(X)$.
\end{Bem}

\begin{Def}{Resolventenmenge}\\
    Die Menge $\varrho(T) := \{\lambda \in \complex \;|\; \lambda\id - T \text{ bijektiv}\}$
    heißt \begriff{Resolventenmenge} von $T$.
\end{Def}

\begin{Def}{Spektrum}
    Die Menge $\sigma(T) := \complex \setminus \varrho(T)$
    heißt \begriff{Spektrum} von $T$.
    Es kann zerlegt werden in
    \begin{itemize}
        \item
        das \begriff{Punktspektrum}\\
        $\sigma_p(T) := \{\lambda \in \complex \;|\; \lambda\id - T \text{ nicht injektiv}\}$,

        \item
        das \begriff{kontinuierliche Spektrum}\\
        $\sigma_c(T) := \{\lambda \in \complex \;|\; \lambda\id - T
        \text{ injektiv, aber nicht surjektiv und } \overline{\Bild(\lambda\id - T)} = X\}$ und

        \item
        das \begriff{Residualspektrum}\\
        $\sigma_r(T) := \{\lambda \in \complex \;|\; \lambda\id - T
        \text{ injektiv und } \overline{\Bild(\lambda\id - T)} \not= X\}$.
    \end{itemize}
\end{Def}

\begin{Def}{Eigenvektor, Eigenwert, Eigenraum}\\
    Für $\lambda \in \complex$ gilt $\lambda \in \sigma_p(T)$ genau dann, wenn
    $\exists_{x \in X \setminus \{0\}}\; Tx = \lambda x$.
    In diesem Fall heißt $x$ \begriff{Eigenvektor} von $T$ zum \begriff{Eigenwert} $\lambda$.
    Ist $X$ ein Funktionenraum, so heißt $x$ auch \begriff{Eigenfunktion}.
    Der Unterraum $\Kern(\lambda\id - T)$ von $X$ heißt \begriff{Eigenraum} von $T$ zum
    Eigenwert $\lambda$.
    Seine Dimension heißt \begriff{Vielfachheit} des Eigenwerts $\lambda$.
    Der Eigenraum ist ein $T$-invarianter Unterraum, d.\,h.
    $T(\Kern(\lambda\id - T)) \subset \Kern(\lambda\id - T)$.
\end{Def}

\begin{Bem}
    Für $\dim(X) < \infty$ gilt $\sigma(T) = \sigma_p(T)$ für alle $T \in \Lin(X)$
    (da in diesem Fall $\lambda\id - T$ injektiv $\iff$ $\lambda\id - T$ surjektiv für alle
    $\lambda \in \complex$).
\end{Bem}

\linie

\begin{Bem}
    Im weiteren Verlauf wird der folgende (nicht-triviale) Satz aus der Banachraum-Theorie
    benötigt, der später bewiesen wird.
\end{Bem}

\begin{Satz}{Umkehrabbildung stetig}
    Seien $E$ und $F$ Banachräume und $L \in \Lin(E, F)$ bijektiv.\\
    Dann gilt $L^{-1} \in \Lin(F, E)$.
\end{Satz}

\begin{Def}{Resolvente}
    Sei $\lambda \in \varrho(T)$.
    Dann heißt $R(\lambda, T) := (\lambda\id - T)^{-1} \in \Lin(X)$
    \begriff{Resolvente} von $T$ in $\lambda$ und
    $R(\cdot, T)\colon \varrho(T) \rightarrow \Lin(X)$,
    $\lambda \mapsto R(\lambda, T)$ heißt \begriff{Resolventenfunktion}.
\end{Def}


\begin{Satz}{Resolventenfunktion holomorph}
    $\varrho(T) \subset \complex$ ist of"|fen und
    $R(\cdot, T)\colon \varrho(T) \rightarrow \Lin(X)$ ist holomorph, d.\,h.
    $\lim_{h \to 0} \frac{R(\lambda + h, T) - R(\lambda, T)}{h}$ existiert in $\Lin(X)$.\\
    Außerdem gilt $\forall_{\lambda \in \varrho(T)}\;
    \norm{R(\lambda, T)}^{-1} \le \dist(\lambda, \sigma(T))$.
\end{Satz}

\linie

\begin{Def}{Spektralradius}
    $\sup_{\lambda \in \sigma(T)} |\lambda|$ heißt \begriff{Spektralradius} von $T$.
\end{Def}

\begin{Satz}{Spektrum kompakt}
    $\sigma(T)$ ist kompakt und für $X \not= \{0\}$ auch nicht-leer mit\\
    $\sup_{\lambda \in \sigma(T)} |\lambda| = \lim_{m \to \infty} \norm{T^m}^{1/m} \le \norm{T}$.
\end{Satz}

\begin{Satz}{Spektralradius normaler Operatoren über Hilberträume}\\
    Sei $X \not= \{0\}$ ein $\complex$-Hilbertraum und
    $T \in \Lin(X)$ normal.
    Dann gilt $\sup_{\lambda \in \sigma(T)} |\lambda| = \norm{T}$.
\end{Satz}

\pagebreak

\section{%
    Das Spektrum kompakter Operatoren und der Spektralsatz%
}

\begin{Satz}{Spektrum kompakter Operatoren}
    Sei $T \in \K(X)$.\\
    Dann stimmt $\sigma(T)$ mit den Eigenwerten $\sigma_p(T)$ bis auf $0$ überein,
    d.\,h. $\sigma(T) \setminus \{0\} = \sigma_p(T) \setminus \{0\}$.\\
    Außerdem besteht $\sigma(T) \setminus \{0\}$
    \begin{enumerate}
        \item
        aus endlich vielen Eigenwerten oder

        \item
        aus abzählbar unendlich vielen Eigenwerten mit $0$ als einzigem Häufungspunkt.
    \end{enumerate}
    Die Vielfachheit jeden von $0$ verschiedenen Eigenwerts
    $\lambda \in \sigma(T) \setminus \{0\}$ ist endlich.\\
    Für $\dim X = \infty$ ist $0 \in \sigma(T)$.
\end{Satz}

\linie

\begin{Def}{positiv semidefinit}
    Seien $H$ ein $\complex$-Hilbertraum und $T \in \Lin(H)$ selbstadjungiert.\\
    Dann heißt $T$ \begriff{positiv semidefinit}, falls
    $\forall_{x \in H}\; \innerproduct{x, Tx} \ge 0$.
\end{Def}

\begin{Satz}{Spektralsatz für kompakte, selbstadjungierte Operatoren}\\
    Seien $H$ ein $\complex$-Hilbertraum und $T \in \Lin(H) \setminus \{0\}$
    kompakt und selbstadjungiert.
    Dann gilt:
    \begin{enumerate}
        \item
        $\sigma_p(T) \setminus \{0\} = \{\lambda_k \;|\; k \in N\}$
        mit $N = \{1, \dotsc, n\}$ oder $N = \natural$ und
        $\lambda_k$ paarweise verschieden.\\
        Für alle $k \in N$ gilt $\dim(\Kern(\lambda_k \id - T)) < \infty$ und gibt es Eigenvektoren
        $e_{k,j_k}$,\\
        $j_k = 1, \dotsc, \dim(\Kern(\lambda_k \id - T))$,
        von $T$ zu $\lambda_k$, sodass
        $(e_{k,j_k})_{k,j_k}$ ein ONS in $H$ ist.\\
        Für $N = \natural$ gilt $\lim_{k \to \infty} \lambda_k = 0$.

        \item
        $H = \Kern(T) \oplus \overline{[\{e_{k,j_k} \;|\; k, j_k\}]}$
        mit $\Kern(T) \,\orth\, \overline{[\{e_{k,j_k} \;|\; k, j_k\}]}$

        \item
        $\forall_{x \in H}\; Tx = \sum_k \sum_{j_k} \lambda_k \innerproduct{x, e_{k,j_k}} e_{k,j_k}$

        \item
        $\sigma_p(T) \subset [-\norm{T}, \norm{T}] \subset \real$

        \item
        $\norm{T} \in \sigma_p(T)$ oder $-\norm{T} \in \sigma_p(T)$

        \item
        Ist $T$ positiv semidefinit, dann gilt $\sigma_p(T) \subset [0, \norm{T}]$.
    \end{enumerate}
\end{Satz}

\begin{Bem}
    Dieser Satz ist eine unendlich-dimensionale Verallgemeinerung des Theorems
    aus der linearen Algebra, dass jede
    symmetrische Matrix mithilfe von ONBen aus Eigenvektoren reell diagonalisierbar ist.
\end{Bem}

\begin{Satz}{Spektralsatz für kompakte, normale Operatoren}\\
    Seien $H$ ein $\complex$-Hilbertraum und $T \in \Lin(H) \setminus \{0\}$
    kompakt und normal.\\
    Dann gelten die Aussagen \emph{(1)}, \emph{(2)} und \emph{(3)} aus obigem Satz.
\end{Satz}

\begin{Bem}
    Anhand des Beweises erkennt man, dass die Aussagen \emph{(4)} und \emph{(6)} gelten,
    wenn $T$ nur selbstadjungiert (und stetig) ist, aber nicht kompakt.
\end{Bem}

\linie

\begin{Bem}
    Ist $X$ ein $\real$-Banachraum, so kann man $X$ \begriff{komplexifizieren},
    d.\,h. $\widetilde{X} := X \times X$ mit
    $\alpha \cdot (x_1, x_2) := (a x_1 - b x_2, a x_2 + b x_1)$ und
    $\overline{(x_1, x_2)} := (x_1, -x_2)$ für $(x_1, x_2) \in \widetilde{X}$ und
    $\alpha := a + \iu b \in \complex$ mit $a, b \in \real$.
    Damit wird $\widetilde{X}$ ein $\complex$-Vektorraum.\\
    Mit $\norm{x}_{\widetilde{X}} := \sup_{\theta \in \real}
    \left(\norm{\cos(\theta)x_1 - \sin(\theta)x_2}_X^2 +
    \norm{\sin(\theta)x_1 + \cos(\theta)x_2}_X^2\right)^{1/2}$ gilt dann\\
    $\forall_{x \in \widetilde{X}} \forall_{\theta \in \real}\;
    \norm{e^{\iu\theta}x}_{\widetilde{X}} = \norm{x}_{\widetilde{X}}$
    und $\widetilde{X}$ ist ein $\complex$-Banachraum.\\
    Falls $X$ ein $\real$-Hilbertraum ist, so ist $\widetilde{X}$ ein $\complex$-Hilbertraum
    sowie $\norm{x}_{\widetilde{X}} = \left(\norm{x_1}_X^2 + \norm{x_2}_X^2\right)^{1/2}$.\\
    Für $T \in \Lin(X)$ ist $\widetilde{T} \in \Lin(\widetilde{X})$ mit
    $\widetilde{T}x := (Tx_1, Tx_2)$.
    Zusätzliche Eigenschaften wie Kompaktheit oder Selbstadjungiertheit von $T$ übertragen sich
    auf $\widetilde{T}$.
    Somit kann man mit dieser Komplexifizierung Spektralsätze wie oben auch auf reelle
    Hilberträume übertragen (analog auch von komplexen auf reelle Banachräume).
\end{Bem}

\linie
\pagebreak

\begin{Def}{\name{Rayleigh}-Quotient}
    Seien $H$ ein $\complex$-Hilbertraum und $T \in \Lin(H)$ selbstadjungiert.\\
    Dann heißt $R_T(u) := \frac{\innerproduct{Tu, u}}{\innerproduct{u, u}}$
    der \begriff{\name{Rayleigh}-Quotient} von $u \in H \setminus \{0\}$.
\end{Def}

\begin{Bem}
    Der Rayleigh-Quotient von Eigenvektoren ist
    gleich dem jeweiligen Eigenwert.
\end{Bem}

\begin{Satz}{Eigenwerte kompakter, selbstadjungierter Operatoren}\\
    Seien $H$ ein $\complex$-Hilbertraum und $T \in \Lin(H) \setminus \{0\}$
    kompakt und selbstadjungiert.
    Dann gilt:
    \begin{enumerate}
        \item
        Wenn $\lambda \not= 0$ mit
        $\lambda := \sup_{u \in H \setminus \{0\}} R_T(u) = \sup_{u \in H,\;\norm{u}=1} \innerproduct{Tu, u}$
        gilt, dann ist\\
        $\lambda = \max(\sigma_p(T) \setminus \{0\})$.
        Das Supremum wird in diesem Fall von allen Eigenvektoren zum Eigenwert $\lambda$
        angenommen.

        \item
        Wenn $\mu \not= 0$ mit
        $\mu := \inf_{u \in H \setminus \{0\}} R_T(u) = \inf_{u \in H,\;\norm{u}=1} \innerproduct{Tu, u}$
        gilt, dann ist\\
        $\mu = \min(\sigma_p(T) \setminus \{0\})$.
        Das Infimum wird in diesem Fall von allen Eigenvektoren zum Eigenwert $\lambda$
        angenommen.

        \item
        Für $\sup_{u \in \Kern(\lambda\id - T)^\orth \setminus \{0\}} R_T(u) \not= 0$
        ist dies der zweitgrößte von $0$ verschiedene Eigenwert usw.
    \end{enumerate}
\end{Satz}

\begin{Bem}
    Für alle von $0$ verschiedenen Eigenwerte sind die Lösungen der jeweiligen
    Eigen"-wert-Gleichungen die Lösungen von Variationsproblemen mit Nebenbedingungen,
    wobei die Eigenwerte als Lagrange-Parameter auftreten.
\end{Bem}

\section{%
    Der Spektralsatz für den \name{Laplace}-Operator%
}

\begin{Def}{inverser \name{Laplace}-Operator}
    Sei $\Omega \subset \real^n$ ein beschränktes und stückweise $\C^1$-berandetes Gebiet.
    Dann ist der \begriff{(schwache) inverse \name{Laplace}-Operator
    (mit homogenen \name{Dirichlet}-RB)}\\
    $\Delta^{-1}\colon L^2(\Omega) \rightarrow H_0^1(\Omega)$ definiert durch die für
    $f \in L^2(\Omega)$ eindeutige Lösung $-\Delta^{-1} f \in H_0^1(\Omega)$ von
    $\forall_{\varphi \in H_0^1(\Omega)}\;
    \int_\Omega (\nabla (-\Delta^{-1} f) \nabla \varphi - f \varphi) \dx = 0$
    (schwache Lösung des Dirichlet-Problems für die Poisson-Gleichung mit
    homogenen Randbedingungen).
\end{Def}

\begin{Satz}{Eigenschaften von $-\Delta^{-1}$}
    $-\Delta^{-1}\colon L^2(\Omega) \rightarrow L^2(\Omega)$ ist
    linear, stetig, injektiv, kompakt, selbstadjungiert und positiv semidefinit.
\end{Satz}

\begin{Satz}{Satz von \scshape\,\!\name{Rellich}}
    Sei $\Omega \subset \real^n$ ein beschränktes und stückweise $\C^1$-berandetes Gebiet.
    Dann ist die Einbettung $\id\colon H^1(\Omega) \hookrightarrow L^2(\Omega)$ ein kompakter
    Operator, d.\,h. jede in $H^1(\Omega)$ beschränkte Folge enthält eine in
    $L^2(\Omega)$ konvergente Teilfolge.
\end{Satz}

\linie

\begin{Def}{schwacher \name{Laplace}-Operator}\\
    $\Delta := (\Delta^{-1})^{-1}\colon \Delta^{-1}(L^2(\Omega)) \rightarrow L^2(\Omega)$
    heißt \begriff{schwacher \name{Laplace}-Operator}.
\end{Def}

\begin{Satz}{Spektralsatz für den \name{Laplace}-Operator}\\
    Sei $\Omega \subset \real^n$ ein beschränktes und stückweise $\C^1$-berandetes Gebiet.
    Dann gilt:
    \begin{enumerate}
        \item
        $\sigma_p(-\Delta) = \{\lambda_k \;|\; k \in \natural\}$ mit
        $0 < \lambda_1 \le \lambda_2 \le \dotsb$,
        $\dim(\Kern(\lambda_k\id + \Delta)) < \infty$ und\\
        $\lim_{k \to \infty} \lambda_k = \infty$

        \item
        Es gibt eine Folge $(e_k)_{k \in \natural}$ in $H_0^1(\Omega)$, sodass
        $(e_k)_{k \in \natural}$ ein vollständiges ONS in $L^2(\Omega)$ aus Eigenvektoren von
        $-\Delta$ ist, d.\,h.
        $\forall_{\varphi \in H_0^1(\Omega)}\;
        \innerproduct{e_k, \varphi}_{H_0^1} = \lambda_k \innerproduct{e_k, \varphi}_{L^2}$ und\\
        $\forall_{u \in L^2(\Omega)}\;
        u \overset{L^2}{=} \sum_{k=1}^\infty \innerproduct{u, e_k}_{L^2} e_k,\;
        \norm{u}_{L^2}^2 = \sum_{k=1}^\infty |\innerproduct{u, e_k}_{L^2}|^2$.

        \item
        Für $k \in \natural$ gilt
        $\lambda_k = \min\left\{\left.\frac{\norm{u}_{H_0^1}^2}{\norm{u}_{L^2}^2} \;\right|\;
        u \in H_0^1 \setminus \{0\},\; u \;\orth\; [e_1, \dotsc, e_{k-1}]\right\}$.
    \end{enumerate}
\end{Satz}

\pagebreak
