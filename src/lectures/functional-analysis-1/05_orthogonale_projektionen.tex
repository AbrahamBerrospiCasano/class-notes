\optpart{%
    \name{Hilbert}raumtheorie%
}

\section{%
    Orthogonale Projektionen%
}

\subsection{%
    Der Projektionssatz%
}

\begin{Satz}{Existenz und Eindeutigkeit des bestappr. Elements}\\
    Seien $H$ ein Hilbertraum und
    $A \subset H$ eine nicht-leere, abgeschlossene und \begriff{konvexe} Teilmenge, d.\,h.
    $\forall_{x, y \in A} \forall_{\lambda \in [0, 1]}\; \lambda x + (1 - \lambda) y \in A$.
    Dann gilt
    $\forall_{x_0 \in H} \exists!_{y_0 \in A}\;
    \norm{x_0 - y_0} = \dist(x_0, A)$.\\
    $y_0$ heißt \begriff{bestapproximierendes Element} an $x_0$ in $A$.
\end{Satz}

\begin{Satz}{Charakterisierung des bestappr. Elements als orthogonale Projektion}\\
    Seien $H$ ein Hilbertraum und $M \subset H$ ein Unterraum.
    Dann ist $y_0 \in M$ bestapproximierend an $x_0 \in H$ in $M$ genau dann, wenn
    $\forall_{y \in M}\; \sp{x_0 - y_0, y} = 0$
    (also $x_0 - y_0 \in M^\orth$).\\
    $y_0$ heißt in diesem Fall die \begriff{orthogonale Projektion} von $x_0$ auf $M$.
\end{Satz}

\begin{Satz}{Projektionssatz}
    Seien $H$ ein Hilbertraum und $M \subset H$ ein abgeschlossener Unterraum.\\
    Dann gilt $\forall_{x_0 \in H} \exists!_{y_0 \in M} \exists!_{y_1 \in M^\orth}\;
    x_0 = y_0 + y_1$,
    also $H = M \oplus M^\orth$ (direkte Summe).\\
    Dabei ist $M^\orth := \{y \in H \;|\; \forall_{x \in M}\; \sp{x, y} = 0\}$
    das \begriff{orthogonale Komplement} von $M$ in $H$.
\end{Satz}

\begin{Kor}
    Zu jedem abgeschlossenen, echten Unterraum $M$ eines Hilbertraums $H$ ($M \not= H$)
    gibt es ein $z_0 \in M^\orth$ mit $z_0 \not= 0$ ($M^\orth \not= \{0\}$).
\end{Kor}

\begin{Bem}
    Für jeden Unterraum $M \subset H$ gilt stets $M \cap M^\orth = \{0\}$.\\
    Außerdem ist
    $M^\orth = \bigcap_{x \in M} \{y \in H \;|\; \sp{x, y} = 0\}
    = \bigcap_{x \in M} \sp{x, \cdot}^{-1}(0)$
    abgeschlossen.
\end{Bem}

\pagebreak

\subsection{%
    Orthonormalsysteme%
}

\begin{Def}{Orthonormalsystem}
    Seien $(E, \sp{\cdot, \cdot})$ ein Skalarproduktraum und $e_i \in E$ für $i \in I$
    ($I \not= \emptyset$ Indexmenge).
    Die Familie $(e_i)_{i \in I}$ heißt \begriff{Orthonormalsystem (ONS)}, falls
    $\forall_{i, j \in I}\; \sp{e_i, e_j} = \delta_{ij}$.
\end{Def}

\begin{Lemma}{orthogonale Projektion durch endliche ONS}
    Sei $(e_i)_{i \in I}$ ein endliches ONS in $E$.\\
    Dann liefert die Zuordnung $P_I\colon E \rightarrow E_I$,
    $P_I(x) := \sum_{i \in I} \sp{x, e_i} e_i$ die orthogonale Projektion von $x$ auf
    $E_I := [\{e_i \;|\; i \in I\}]$
    und es gilt $\forall_{x \in E}\; \norm{x}^2 =
    \sum_{i \in I} |\sp{x, e_i}|^2 + \norm{x - P_I(x)}^2$.\\
    Außerdem sind die $(e_i)_{i \in I}$ linear unabhängig.
\end{Lemma}

\begin{Lemma}{\name{Bessel}sche Ungleichung}
    Sei $(e_i)_{i \in I}$ ein beliebiges ONS in $E$.\\
    Dann gilt $\forall_{x \in E}\; \sum_{i \in I} |\sp{x, e_i}|^2 \le \norm{x}^2$.
\end{Lemma}

\begin{Satz}{Äquivalenzen für abzählbare ONS}
    Für jedes höchstens abzählbare ONS $(e_i)_{i \in I}$ ($I \subset \natural$)
    in einem Skalarproduktraum $(E, \sp{\cdot, \cdot})$ sind äquivalent:
    \begin{enumerate}
        \item
        $[\{e_i \;|\; i \in I\}]$ ist dicht in $E$.
        
        \item
        $\forall_{x \in E}\; x = \sum_{i \in I} \sp{x, e_i} e_i$
        
        \item
        $\forall_{x \in E}\; \norm{x}^2 = \sum_{i \in I} |\sp{x, e_i}|^2$
        (\begriff{\name{Parseval}sche Gleichung})
    \end{enumerate}
    Ist $E$ ein Hilbertraum, dann ist zusätzlich jede dieser Aussagen äquivalent zu
    \begin{enumerate}[start=4]
        \item
        $(e_i)_{i \in I}$ \begriff{maximal}, d.\,h. es gibt kein $y \in E \setminus \{0\}$ mit
        $\forall_{i \in I}\; \sp{y, e_i} = 0$.
    \end{enumerate}
\end{Satz}

\begin{Bem}
    Wenn die Parsevalsche Gleichung oder eine der äquivalenten Aussagen gilt,
    so spricht man auch oft von einer
    \begriff{Orthonormalbasis (ONB)} $(e_i)_{i \in I}$
    (i.\,A. aber keine Vektorraum-Basis)
    oder einem \begriff{vollständigen ONS}.
    In diesem Fall gilt
    $\norm{\sum_{i \in I} \alpha_i e_i}^2 = \sum_{i \in I} |\alpha_i|^2$
    für jede Folge $(\alpha_i)_{i \in I}$ in $\KK$, wie man sich leicht herleiten kann
    (Verallgemeinerung des Satzes von Pythagoras).
\end{Bem}

\linie

\begin{Def}{separabel}
    Sei $(M, d)$ ein metrischer Raum.
    Eine Teilmenge $T \subset M$ heißt \begriff{separabel}, falls es eine höchstens abzählbare
    Teilmenge $A \subset M$ gibt, die dicht in $T$ ist.
\end{Def}

\begin{Satz}{Äquivalenz für separable Hilberträume}
    Sei $H$ ein Hilbertraum. Dann sind äquivalent:
    \begin{enumerate}
        \item
        $H$ ist separabel.
        
        \item
        $H$ besitzt ein maximales, höchstens abzählbares ONS.
    \end{enumerate}
\end{Satz}

\begin{Bsp}
    \begin{enumerate}[label=\emph{(\alph*)}]
        \item
        Sei $H := L^2([0, 2\pi], \real)$.
        Dann ist $\{\frac{1}{\sqrt{2\pi}}, g_1, h_1, g_2, h_2, \dotsc\}$
        mit $g_n(x) := \frac{1}{\sqrt{\pi}} \cos(nx)$,\\
        $h_n(x) := \frac{1}{\sqrt{\pi}} \sin(nx)$
        eine abzählbare ONB.
        Es gilt für alle $f \in H$, dass\\
        $f(x) = \frac{1}{2\pi} \int_0^{2\pi} f(t)\dt +
        \frac{1}{\pi} \sum_{n=1}^\infty \left(\int_0^{2\pi} f(t)\cos(nt)\dt\right) \cos(nx) \;+$\\
        $+\; \frac{1}{\pi} \sum_{n=1}^\infty \left(\int_0^{2\pi} f(t)\sin(nt)\dt\right) \sin(nx)$,
        wobei diese Reihen bzgl. der $L^2$-Norm konvergieren.
        
        \item
        Sei $H := L^2([0, 2\pi], \complex)$.
        Dann ist $(f_n)_{n \in \integer}$ mit $f_n(x) := \frac{1}{\sqrt{2\pi}} e^{\iu nx}$
        eine abzählbare ONB.
        Es gilt für alle $f \in H$, dass
        $f(x) = \frac{1}{2\pi} \sum_{n=-\infty}^{+\infty}
        \left(\int_0^{2\pi} f(t) e^{-\iu nt} \dt\right) e^{\iu nx}$,
        wobei diese Reihe bzgl. der $L^2$-Norm konvergiert.
    \end{enumerate}
\end{Bsp}

\pagebreak

\subsection{%
    Der \name{Riesz}sche Darstellungssatz%
}

\begin{Bem}
    Jede lineare Abbildung $\ell\colon \real^n \rightarrow \real$ lässt sich durch eine
    Matrix $L = \smallpmatrix{L_1 & \cdots & L_n}$ mit $L \in \real^{1 \times n}$ darstellen,
    d.\,h. es gilt $\ell(x) = Lx = \sp{\smallpmatrix{L_1 \\ \vdots \\ L_n},
    \smallpmatrix{x_1 \\ \vdots \\ x_n}}_2 = \sp{L^T, x}$ mit $L^T \in \real^n$.
    Es ist überraschend, dass sich das auf Hilberträume verallgemeinern lässt.
\end{Bem}

\begin{Satz}{\name{Riesz}scher Darstellungssatz}
    Seien $H$ ein Hilbertraum und $\ell \in H'$.\\
    Dann gibt es genau ein $y \in H$ mit $\forall_{x \in H}\; \ell(x) = \sp{x, y}$.
    Es gilt $\norm{\ell} = \norm{y}$.
\end{Satz}

\begin{Kor}\\
    Seien $H$ ein Hilbertraum und $\R\colon H \rightarrow H'$, $y \mapsto \R y$ mit
    $(\R y)(x) := \sp{x, y}$ für $x \in H$.\\
    Dann ist $\R$ für $\KK = \real$ ein isometrischer Isomorphismus und\\
    für $\KK = \complex$ ein \begriff{isometrischer, konjugiert linearer Isomorphismus}
    (d.\,h. $\R$ ist eine Isometrie,\\
    $\forall_{y_1, y_2 \in H} \forall_{\alpha \in \complex}\;
    \R(y_1 + \alpha y_2) = \R y_1 + \overline{\alpha} \R y_2$,
    $\R$ ist bijektiv und $\R, \R^{-1}$ sind stetig).
\end{Kor}

\linie

\begin{Satz}{Charakterisierung des darstellenden Elements}\\
    Seien $H$ ein Hilbertraum, $y \in H$ und $\ell \in H'$.\\
    Dann gilt $\forall_{x \in H}\; \ell(x) = \sp{x, y}$
    genau dann, wenn\\
    $\frac{1}{2} \sp{y, y} - \Re(\ell(y)) =
    \min_{x \in H} \left(\frac{1}{2} \sp{x, x} - \Re(\ell(x))\right)$.
\end{Satz}

\linie

\begin{Satz}{Satz von \upshape\,\!\name{Lax}-\name{Milgram}}
    Seien $H$ ein Hilbertraum und $a\colon H \times H \rightarrow \KK$ \begriff{sesquilinear}\\
    (d.\,h. linear im ersten und konjugiert linear im zweiten Argument).\\
    Außerdem gebe es Konstanten $c_0, C_0 \in \real$ mit $0 < c_0 < C_0 < \infty$, sodass
    \begin{enumerate}
        \item
        $\forall_{x, y \in H}\; |a(x, y)| \le C_0 \norm{x} \norm{y}$
        (Stetigkeit von $a$) und
        
        \item
        $\forall_{x \in H}\; \Re(a(x, x)) \ge c_0 \norm{x}^2$
        (\begriff{Koerzitivität} von $a$).
    \end{enumerate}
    Dann gibt es zu jedem $\ell \in H'$ genau ein $z \in H$ mit
    $\forall_{y \in H}\; \ell(y) = a(y, z)$.
    Es gilt $\norm{z} \le \frac{1}{c_0} \norm{\ell}$.\\
    Außerdem existiert genau eine Abbildung $A\colon H \rightarrow H$ mit
    $\forall_{x, y \in H}\; a(y, x) = \sp{y, Ax}$.\\
    $A$ ist ein Isomorphismus mit $\norm{A} \le C_0$ und
    $\norm{A^{-1}} \le \frac{1}{c_0}$.
\end{Satz}

\pagebreak
