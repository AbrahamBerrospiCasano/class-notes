\chapter{%
    Anwendungen bei elliptischen RWP und \name{Sobolev}räume%
}

\section{%
    \name{Poisson}-Gleichung mit \name{Dirichlet}-Randbedingungen%
}

\begin{Bem}
    Sei $\Omega \subset \real^n$ ein beschränktes \begriff{Normalgebiet},
    d.\,h. eine beschränkte, of"|fene, nicht-leere und zusammenhängende Teilmenge von $\real^n$,
    sodass der Gaußsche Integralsatz anwendbar ist.
    Für $f \in \C^0(\overline{\Omega})$ sei außerdem
    $E(w) := \int_\Omega (\frac{1}{2} |\nabla w|^2 - fw) \dx$.\\
    Zusätzlich sei $\A_g := \C^1_g(\overline{\Omega}) \cap \C^2(\Omega)$,
    wobei $g \in \C^0(\partial\Omega)$ und
    $\C^1_g(\overline{\Omega}) := \{w \in \C^1(\overline{\Omega}) \;|\;
    w|_{\partial \Omega} = g\}$.\\
    Das \begriff{Minimumproblem} lautet nun:
    Nimmt $E$ auf $\A_g$ ein Minimum an?

    Beispiele aus der Physik beinhalten
    eingespannte Membranen im Schwerefeld der Erde,
    elektrische Potentiale oder
    stationäre Temperaturverteilungen.
\end{Bem}

\linie

\begin{Bem}
    Die Lösung (falls existent) lässt sich wie folgt charakterisieren.
\end{Bem}

\begin{Satz}{Charakterisierung der Lösung des Minimumproblems}
    Für $u \in \A_g$ sind äquivalent:
    \begin{enumerate}
        \item
        $E(u) = \min_{w \in \A_g} E(w)$

        \item
        $\forall_{\varphi \in \C^\infty_c(\Omega)}\;
        \int_\Omega (\nabla u \nabla \varphi - f \varphi)\dx = 0$

        \item
        $-\Delta u = f$ in $\Omega$, $u = g$ auf $\partial \Omega$
    \end{enumerate}
\end{Satz}

\begin{Lemma}{Fundamentallemma der Variationsrechnung}
    Sei $f \in \C^0(\Omega)$.\\
    Dann gilt $\forall_{\varphi \in \C^\infty_c(\Omega)}\; \int_\Omega f\varphi \dx = 0$
    genau dann, wenn $f \equiv 0$.
\end{Lemma}

\begin{Lemma}{\name{Green}sche Formel}
    Für alle $u, w \in \C^2(\overline{\Omega})$ gilt\\
    $\int_\Omega \nabla u \nabla w \dx = -\int_\Omega (\Delta u) w \dx
    + \int_{\partial \Omega} \frac{\partial u}{\partial \nu} w do$,
    wobei $\frac{\partial u}{\partial \nu}$ die Ableitung von $u$ in Richtung des äußeren
    Einheitsnormalenvektors ist.
\end{Lemma}

\linie

\begin{Bem}
    "`\emph{(1)} $\Rightarrow$ \emph{(2)}"' kann man wie folgt beweisen:
    Für $\varphi \in \C^\infty_c(\Omega)$ und $h > 0$ gilt\\
    $E(u) \le E(u \pm h\varphi)
    = \int_\Omega (\frac{1}{2} |\nabla (u \pm h\varphi)|^2 - f(u \pm h\varphi)) \dx$\\
    $= \int_\Omega (\frac{1}{2} |\nabla u|^2 + \frac{h^2}{2} |\nabla \varphi|^2 \pm
    h \nabla u \nabla \varphi - fu \mp h f\varphi) \dx
    = E(u) \pm h \int_\Omega (\nabla u \nabla \varphi - f\varphi) \dx +
    \frac{h^2}{2} \int_\Omega |\nabla \varphi|^2 \dx$,
    also $0 \le \pm\int_\Omega (\nabla u \nabla \varphi - f\varphi) \dx +
    \frac{h}{2} \int_\Omega |\nabla \varphi|^2 \dx$.
    Für $h \to 0$ fällt der zweite Summand weg und man erhält
    $0 \le \pm\int_\Omega (\nabla u \nabla \varphi - f\varphi) \dx$.

    "`\emph{(2)} $\iff$ \emph{(3)}"' sieht man wie folgt:
    Mit der Greenschen Formel ist (2) äquivalent zu\\
    $\forall_{\varphi \in \C^\infty_c(\Omega)}\;
    0 = \int_\Omega (\nabla u \nabla \varphi - f\varphi) \dx
    = \int_\Omega (-\Delta u - f) \varphi \dx$, weil das Integral über $\partial \Omega$
    wegfällt (da $\varphi = 0$ auf $\partial \Omega$).
    Nach dem Fundamentallemma der Variationsrechnung ist dies äquivalent zu $-\Delta u = f$
    in $\Omega$.
    $u = g$ auf $\partial G$ gilt immer, da $u \in \A_g$ nach Voraussetzung.

    "`\emph{(3)} $\Rightarrow$ \emph{(1)}"' zeigt man folgendermaßen:
    Für $w \in \A_g$ beliebig gilt nach der Greenschen Formel
    $\int_\Omega (\nabla u \nabla (u - w) - f(u - w))\dx
    = \int_\Omega ((-\Delta u)(u - w) - f(u - w)) \dx +
    \int_{\partial\Omega} \frac{\partial u}{\partial \nu} (u - w) do = 0$,
    weil $-\Delta u = f$ in $\Omega$ und $u|_{\partial\Omega} = w|_{\partial\Omega} = g$.
    Daraus folgt\\
    $\int_\Omega (|\nabla u|^2 - fu) \dx
    = \int_\Omega (\nabla u \nabla w - fw) \dx
    \le \frac{1}{2} \int_\Omega |\nabla u|^2 \dx + \frac{1}{2} \int_\Omega |\nabla w|^2 \dx -
    \int_\Omega fw \dx$ wegen der Ungleichung
    $0 \le |\nabla u - \nabla w|^2 = |\nabla u|^2 + |\nabla w|^2 - 2 \nabla u \nabla w$.\\
    Damit gilt
    $E(u) = \int_\Omega (\frac{1}{2} |\nabla u|^2 - fu) \dx \le
    \int_\Omega (\frac{1}{2} |\nabla w|^2 - fw) \dx = E(w)$.
\end{Bem}

\linie

\begin{Bem}
    Notwendige Bedingung für die Existenz einer Lösung von \emph{(3)}
    (\begriff{\name{Poisson}-Gleichung mit inhomogenen \name{Dirichlet}-Randbedingungen})
    ist die Existenz einer Funktion $u_g \in \A_g$
    (d.\,h. $\A_g = \C^1_g(\overline{\Omega}) \cap \C^2(\Omega) \not= \emptyset$).
    Existiert eine solche Funktion,
    dann ist \emph{(3)} äquivalent zu
    $-\Delta \widetilde{u} = \widetilde{f}$ in $\Omega$, $\widetilde{u} = 0$ auf $\partial \Omega$
    mit $\widetilde{u} := u - u_g$, $\widetilde{f} := f + \Delta u_g$.
    Daher genügt es, wenn im Folgenden nur homogene Dirichlet-Randbedingungen (also $g \equiv 0$)
    betrachtet werden.
    (Achtung: $\C^1_g = \C^1_0$ darf nicht mit $\C^1_c$ verwechselt werden!)
\end{Bem}

\linie
\pagebreak

\begin{Bem}
    Nun zeigt man, dass das Minimum überhaupt existiert.
\end{Bem}

\begin{Satz}{\name{Poincaré}-Ungleichung}
    Sei $\Omega \subset \real^n$ ein Gebiet, das zwischen zwei parallelen Hyperebenen
    mit Abstand $C$ liegt.
    Dann gilt $\forall_{u \in \C^1_0(\overline{\Omega})}\;
    \norm{u}_{L^2} \le \frac{C}{\sqrt{2}} \norm{\nabla u}_{L^2}$.
\end{Satz}

\begin{Bem}
    Dabei gilt $\norm{\nabla u}_{L^2}^2
    = \sum_{i=1}^n \int_\Omega |\partial_{x_i} u|^2 \dx
    = \sum_{i=1}^n \norm{\partial_{x_i} u}_{L^2}^2$.
\end{Bem}

\begin{Lemma}{$\varepsilon$-Ungleichung}
    Für $a, b \in \real$ und $\varepsilon > 0$ gilt
    $ab \le \varepsilon a^2 + \frac{b^2}{4\varepsilon}$.
\end{Lemma}

\begin{Satz}{Beschränktheit nach unten}
    $E$ ist auf $\A_0$ nach unten beschränkt.
\end{Satz}

\linie

\begin{Bem}
    Da $E$ auf $\A_0$ nach unten beschränkt ist, existiert eine Minimalfolge
    $(u_n)_{n \in \natural}$ in $\A_0$.
    Weil $\A_0$ konvex ist, kann man wie im Beweis des Projektionssatzes mithilfe der
    Parallelogrammgleichung zeigen, dass $(\partial_{x_i} u_n)_{n \in \natural}$ für alle
    $i = 1, \dotsc, n$ eine Cauchy-Folge bzgl. $\norm{\cdot}_{L^2}$ ist.
    Aufgrund der Poincaré-Ungleichung folgt, dass auch $(u_n)_{n \in \natural}$ eine
    Cauchy-Folge bzgl. $\norm{\cdot}_{L^2}$ ist.

    ($(u_n)_{n \in \natural}$ ist auch eine Cauchy-Folge bzgl. der Norm $\norm{\cdot}_{H^1}$
    mit $\norm{f}_{H^1} := \norm{f}_{L^2} + \norm{\nabla f}_{L^2}$
    sowie bzgl. der (in diesem Fall zur $H_1$-Norm äquivalenten) Norm $\norm{\cdot}_{H^1_0}$
    mit $\norm{f}_{H^1_0} := \norm{\nabla f}_{L^2}$.
    Allerdings ist $\A_0$ bzgl. dieser Normen nicht vollständig.)

    $L^2$ ist vollständig,
    daher existieren $u \in L^2$ mit $u_n \xrightarrow{\norm{\cdot}_{L^2}} u$ und
    "`$\partial_{x_i} u$"' mit $\partial_{x_i} u_n \xrightarrow{\norm{\cdot}_{L^2}}
    \partial_{x_i} u$.
    "`$\partial_{x_i} u$"' ist aber nur eine Schreibweise, i.\,A. besitzt $u$ keine partiellen
    Ableitungen.
    Zwischen $u$ und den Funktionen "`$\partial_{x_i} u$"' besteht folgende Beziehung:
    $\forall_{\varphi \in \C^\infty_c(\Omega)}\;
    \int_\Omega (\partial_{x_i} u) \varphi \dx = -\int_\Omega u \partial_{x_i} \varphi \dx$
    (weil $\int_\Omega (\partial_{x_i} u) \varphi \dx
    = \lim_{n \to \infty} \int_\Omega (\partial_{x_i} u_n) \varphi \dx
    = -\lim_{n \to \infty} \int_\Omega u_n (\partial_{x_i} \varphi) \dx
    = -\int_\Omega u \partial_{x_i} \varphi \dx$).
    Dies motiviert die Definition der Sobolevräume.
\end{Bem}

\section{%
    \name{Sobolev}räume und schwache Ableitungen%
}

\begin{Def}{\name{Sobolev}raum}
    Seien $\Omega \subset \real^n$ of"|fen, $m \in \natural$ und $p \in [1, \infty]$.\\
    Dann heißt der Vektorraum
    $W^{m,p}(\Omega) := \{f \in L^p(\Omega) \;|\; \forall_{s \in \natural_0^n,\, |s| \le m}
    \exists_{f^{(s)} \in L^p(\Omega)}\; f^{(0)} = f,$\\
    $\forall_{\varphi \in \C^\infty_c(\Omega)}\;
    \int_\Omega (\partial_x^s \varphi) f \dx = (-1)^{|s|} \int_\Omega \varphi f^{(s)} \dx\}$
    \begriff{\name{Sobolev}raum} der Ordnung $m$ mit Exponent $p$.\\
    $W^{m,p}(\Omega)$ wird mit der Norm
    $\norm{f}_{W^{m,p}(\Omega)} := \sum_{|s| \le m} \norm{f^{(s)}}_{L^p(\Omega)}$ versehen.
    Für $p = 2$ schreibt man auch $H^m(\Omega) := W^{m,2}(\Omega)$ bzw.
    $\norm{\cdot}_{H^m(\Omega)} := \norm{\cdot}_{W^{m,2}(\Omega)}$.
\end{Def}

\begin{Def}{schwache Ableitung}
    Die Funktionen $f^{(s)}$ für $|s| \ge 1$ heißen \begriff{schwache Ableitungen} von $f$
    und werden mit $\partial_x^s f := f^{(s)}$ bezeichnet.
\end{Def}

\begin{Bem}
    Eine alternative Definition der Norm lautet
    $\norm{f}_{W^{m,p}(\Omega)}' :=
    \left(\sum_{|s| \le m} \norm{\partial_x^s f}_{L^p(\Omega)}^p\right)^{1/p}$
    (bzw. für $p = \infty$ das Maximum
    $\norm{f}_{W^{m,\infty}(\Omega)}' :=
    \max_{|s| \le m} \norm{\partial_x^s f}_{L^\infty(\Omega)}$).
    Allerdings kann man zeigen, dass
    $\norm{\cdot}_{W^{m,p}(\Omega)}$ und $\norm{\cdot}_{W^{m,p}(\Omega)}'$ äquivalent sind.
    %Der Sobolevraum $W^{m,p}(\Omega)$ ist gerade die Vervollständigung bzgl.
    %$\norm{\cdot}_{W^{m,p}(\Omega)}$ derer Funktionen in $\C^\infty(\Omega)$,
    %deren partielle Ableitungen bis zur Ordnung $m$ in $L^p(\Omega)$ sind.
\end{Bem}

\begin{Def}{\name{Sobolev}raum mit Nullrandwerten}
    Der Raum $W_0^{m,p}(\Omega) := \overline{\C^\infty_c(\Omega)}^{\norm{\cdot}_{W^{m,p}(\Omega)}}$
    für\\
    $p \in [1, \infty)$ heißt
    \begriff{\name{Sobolev}raum mit (verallgemeinerten) Nullrandwerten}
    der Ordnung $m$ mit Exponent $p$.
    Für $p = 2$ schreibt man auch $H_0^m(\Omega) := W_0^{m,2}(\Omega)$.
\end{Def}

\begin{Bem}
    Für $m = 1$ gilt $W_0^{1,p}(\Omega) =
    \{f \in W^{1,p}(\Omega) \;|\; f|_{\partial\Omega} = 0\}$.\\
    Für $p = 2$ ist $\innerproduct{f, g}_{H^m(\Omega)}
    := \sum_{|s| \le m} \innerproduct{\partial_x^s f, \partial_x^s g}_{L^2(\Omega)}
    = \sum_{|s| \le m} \int_\Omega (\partial_x^s f) (\partial_x^s g) \dx$
    ein Skalarprodukt auf $H^m(\Omega)$.
    Für $m = 1$ und $p = 2$ ist
    $\innerproduct{f, g}_{H^1_0(\Omega)}
    := \innerproduct{\nabla f, \nabla g}_{L^2(\Omega)}
    := \sum_{i=1}^n \int_\Omega (\partial_x^{e_i} f) (\partial_x^{e_i} g) \dx$
    mit $e_i := (0, \dotsc, 0, 1, 0, \dotsc, 0) \in \natural_0^n$
    ein Skalarprodukt auf $H^1_0(\Omega)$.\\
    Es gilt
    $\innerproduct{f, g}_{H^1(\Omega)}
    = \innerproduct{f, g}_{L^2(\Omega)} + \innerproduct{\nabla f, \nabla g}_{L^2(\Omega)}$.
\end{Bem}

\linie
\pagebreak

\begin{Satz}{schwache Ableitungen}
    \begin{enumerate}
        \item
        Alle schwachen Ableitungen sind eindeutig bestimmt (wenn sie existieren).

        \item
        Besitzt $f \in W^{m,p}(\Omega)$ eine partielle Ableitung $\partial_x^s f$ mit $|s| \le m$,
        dann stimmt $\partial_x^s f$ fast überall mit der schwachen Ableitung $f^{(s)}$ überein.
    \end{enumerate}
\end{Satz}

\begin{Lemma}{verallgemeinertes Fundamentallemma der Variationsrechnung}\\
    Seien $\Omega \subset \real^n$ of"|fen und $f \in L^1(\Omega)$.\\
    Dann gilt $\forall_{\varphi \in \C^\infty_c(\Omega)}\; \int_\Omega f\varphi \dx = 0$
    genau dann, wenn $f = 0$ f.ü.
\end{Lemma}

\begin{Satz}{Eigenschaften der \name{Sobolev}räume}
    \begin{enumerate}
        \item
        $(W^{m,p}(\Omega), \norm{\cdot}_{W^{m,p}(\Omega)})$ ist ein Banachraum.
        $(H^m(\Omega), \norm{\cdot}_{H^m(\Omega)})$ ist ein Hilbertraum.

        \item
        Für $p \in [1, \infty)$ ist $W^{m,p}(\Omega)$ separabel.

        \item
        $(W^{m,p}(\Omega), \norm{\cdot}_{W^{m,p}(\Omega)})$ ist
        (bis auf isometrische Isomorphie) die Vervollständigung der Räume
        $W^{m,p}(\Omega) \cap \C^\infty(\Omega) =
        \{f \in \C^\infty(\Omega) \;|\; \norm{f}_{W^{m,p}(\Omega)} < \infty\}$.

        \item
        Für $p \in [1, \infty)$ und alle $f \in W^{m,p}(\Omega)$ gibt es eine Folge
        $(f_n)_{n \in \natural}$ in $W^{m,p}(\Omega) \cap \C^\infty(\Omega)$ mit
        $f_n \xrightarrow{\norm{\cdot}_{W^{m,p}(\Omega)}} f$,
        es gilt also $W^{m,p}(\Omega) =
        \overline{W^{m,p}(\Omega) \cap \C^\infty(\Omega)}^{\norm{\cdot}_{W^{m,p}(\Omega)}}$
        für $p \in [1, \infty)$.
    \end{enumerate}
\end{Satz}

\section{%
    Schwache Lösung der \name{Poisson}-Gleichung mit \name{Dirichlet}-RB%
}

\begin{Satz}{verallgemeinerte \name{Poincaré}-Ungleichung}
    Sei $\Omega \subset \real^n$ ein Gebiet, das zwischen zwei parallelen Hyperebenen
    mit Abstand $C$ liegt.
    Dann gilt $\forall_{u \in H^1_0(\Omega)}\;
    \norm{u}_{L^2} \le \frac{C}{\sqrt{2}} \norm{\nabla u}_{L^2}$,\\
    wobei $\norm{\nabla u}_{L^2} :=
    \left(\sum_{i=1}^n \norm{\partial_{x_i} u}_{L^2}^2\right)^{1/2}$.
\end{Satz}

\begin{Kor}
    Die Normen $\norm{\cdot}_{H^1(\Omega)}$ und $\norm{\cdot}_{H_0^1(\Omega)}$ auf $H_0^1(\Omega)$
    sind äquivalent, wenn $\Omega$ ein Gebiet wie im vorherigen Satz ist.
\end{Kor}

\linie

\begin{Satz}{schwache Lösung}\\
    Seien $\Omega \subset \real^n$ ein beschränktes Normalgebiet, $f \in L^2(\Omega)$ und
    $E(w) := \int_\Omega (\frac{1}{2} |\nabla w|^2 - fw)\dx$.\\
    Dann besitzt $E$ auf $H_0^1(\Omega)$ eine eindeutige Minimalstelle $u$ und $u$ ist die
    eindeutige schwache Lösung des Dirichlet-Problems für die Poisson-Gleichung
    $-\Delta u = f$ in $\Omega$, $u = 0$ auf $\partial\Omega$, d.\,h. es gilt
    $\forall_{\varphi \in H_0^1(\Omega)}\; \int_\Omega (\nabla u \nabla \varphi - f\varphi)\dx = 0$.
\end{Satz}

\begin{Bem}
    Es gibt eine nur von $\Omega$ abhängige Konstante $C > 0$ mit
    $\norm{u}_{H^1} \le C \norm{f}_{L^2}$.
\end{Bem}

\pagebreak

\section{%
    \emph{Zusatz}: \name{Poisson}-Gleichung mit \name{Neumann}-Randbedingungen%
}

\begin{Bem}
    Sei $\Omega \subset \real^n$ ein beschränktes Normalgebiet.\\
    Für $f \in \C^0(\overline{\Omega})$ und $g \in \C^0(\partial\Omega)$ sei außerdem
    $E_g(w) := \int_\Omega (\frac{1}{2} |\nabla w|^2 - fw) \dx - \int_{\partial\Omega} gw do$.\\
    Zusätzlich sei $\A := \C^1(\overline{\Omega}) \cap \C^2(\Omega)$.

    Das \begriff{Minimumproblem} lautet nun:
    Nimmt $E_g$ auf $\A$ ein Minimum an?
\end{Bem}

\linie

\begin{Satz}{Charakterisierung der Lösung des Minimumproblems}
    Sei $u \in \A$.
    Dann sind äquivalent:
    \begin{enumerate}
        \item
        $E_g(u) = \min_{w \in \A} E_g(w)$

        \item
        $\forall_{\varphi \in \C^\infty(\Omega)}\;
        \int_\Omega (\nabla u \nabla \varphi - f \varphi)\dx -
        \int_{\partial\Omega} g\varphi do = 0$

        \item
        $-\Delta u = f$ in $\Omega$, $\frac{\partial u}{\partial\nu} = g$ auf $\partial\Omega$
    \end{enumerate}
    In diesem Fall gilt notwendigerweise $\int_\Omega f\dx + \int_{\partial\Omega} g do = 0$.
\end{Satz}

\linie

\begin{Satz}{\name{Poincaré}-Ungleichung mit Mittelwert}
    Sei $\Omega \subset \real^n$ ein beschränktes und konvexes Gebiet mit Durchmesser $h$.
    Dann gibt es ein $C > 0$ mit $\forall_{u \in \C^1(\overline{\Omega})}\;
    \norm{u - Mu}_{L^2} \le Ch \norm{\nabla u}_{L^2}$,
    wobei $Mu := \frac{\int_\Omega u\dx}{\int_\Omega 1\dx}$
    der \begriff{Mittelwert} von $u$ auf $\Omega$ ist.
\end{Satz}

\begin{Satz}{Beschränktheit nach unten}
    Seien $\Omega \subset \real^n$ ein beschränktes und konvexes Normalgebiet
    und $\int_\Omega f\dx + \int_{\partial\Omega} g do = 0$.
    Dann ist $E_g$ auf $\A$ nach unten beschränkt.
\end{Satz}

\linie

\begin{Satz}{schwache Lösung}
    Seien $\Omega \subset \real^n$ ein beschränktes und konvexes Normalgebiet,\\
    $f \in L^2(\Omega)$ mit $\int_\Omega f\dx = 0$ und
    $E_0(w) := \int_\Omega (\frac{1}{2} |\nabla w|^2 - fw)\dx$.\\
    Dann besitzt $E_0$ auf $H^1(\Omega)$ eine eindeutige Minimalstelle $u$ und $u$ ist die
    eindeutige schwache Lösung des Neumann-Problems für die Poisson-Gleichung
    $-\Delta u = f$ in $\Omega$,
    $\frac{\partial f}{\partial\nu} = 0$ auf $\partial\Omega$, d.\,h. es gilt
    $\forall_{\varphi \in H^1(\Omega)}\; \int_\Omega (\nabla u \nabla \varphi - f\varphi)\dx = 0$.
\end{Satz}

\section{%
    Verallgemeinerung auf elliptische Randwertprobleme%
}

\begin{Def}{elliptische DGL}
    Sei $\Omega \subset \real^n$ ein beschränktes Normalgebiet.\\
    Gesucht sind Funktionen $u \in \C^2(\Omega)$, die die \begriff{elliptische DGL}
    $-\div(A \nabla u + h) + bu + f = 0$\\
    (d.\,h. $-\sum_{i=1}^n \partial_{x_i}
    \!\left(\sum_{j=1}^n a_{ij} \partial_{x_j} u + h_i\right) + bu + f = 0$)
    erfüllen.\\
    Dabei ist $a_{ij}, h_i \in \C^1(\Omega)$ für $i, j = 1, \dotsc, n$,
    $f, b \in \C^0(\Omega)$ und $(a_{ij}(x))_{i,j=1,\dotsc,n}$ sei
    \begriff{gleichmäßig elliptisch in $x$}, d.\,h.
    $\exists_{c_0 > 0} \forall_{x \in \Omega} \forall_{\xi \in \real^n}\;
    \xi^T A(x) \xi = \sum_{i,j=1}^n a_{ij}(x) \xi_i \xi_j \ge c_0 |\xi|^2$.
    (Für jedes $c > 0$ und $x \in \Omega$ beschreibt die Menge
    $\left\{\xi \in \real^n \;\left|\; \xi^T A(x) \xi = c\right.\right\}$
    eine Ellipse.)\\
    Die Matrix $(a_{ij}(x))_{i,j=1,\dotsc,n}$ kann auch unsymmetrisch sein.
\end{Def}

\begin{Bem}
    Ohne zusätzliche Bedingungen sind elliptische DGL nicht eindeutig lösbar.
    Meist bekommt man die eindeutige Lösbarkeit durch Einführung von Randbedingungen.
    Es folgen die beiden Randbedingungen, die in der mathematischen Physik am häufigsten
    vorkommen.
\end{Bem}

\begin{Def}{\name{Dirichlet}-Randbedingungen}\\
    $u$ löst die elliptische DGL in $\Omega$ und erfüllt
    $u = g$ auf $\partial\Omega$ mit $g \in \C^0(\partial\Omega)$.
\end{Def}

\begin{Def}{\name{Neumann}-Randbedingungen}\\
    $u$ löst die elliptische DGL in $\Omega$ und erfüllt
    $-\nu (A \nabla u + h) =
    -\sum_{i=1}^n \nu_i \left(\sum_{j=1}^n a_{ij} \partial_{x_j} u + h_i\right) = g$ auf
    $\partial\Omega$ mit $g \in \C^0(\partial\Omega)$, wobei $\nu$ der äußere
    Einheitsnormalenvektor an $\partial\Omega$ ist.
\end{Def}

\linie
\pagebreak

\begin{Bem}
    Wie bei der Poisson-Gleichung führt man den Begriff einer schwachen Lösung ein.
    Seien dafür nun $a_{ij} \in L^\infty(\Omega)$ und $(a_{ij}(x))_{i,j=1,\dotsc,n}$
    erfülle die Bedingung der gleichmäßigen Elliptizität fast überall auf $\Omega$,
    $b \in L^\infty(\Omega)$ und $h_i, f \in L^2(\Omega)$.
    Aus denselben Gründen wie bei der Poisson-Gleichung genügt es, wenn man nur den Fall $g = 0$
    betrachtet.
\end{Bem}

\begin{Def}{schwache Lösung des \name{Dirichlet}-Problems}
    $u \in H_0^1(\Omega)$ heißt \begriff{schwache Lösung des\\\name{Dirichlet}-Problems},
    falls
    $\forall_{\varphi \in H_0^1(\Omega)}\;
    \int_\Omega (\nabla \varphi (A \nabla u + h) + \varphi(bu + f))\dx = 0$.
\end{Def}

\begin{Def}{schwache Lösung des \name{Neumann}-Problems}
    $u \in H^1(\Omega)$ heißt \begriff{schwache Lösung des\\\name{Neumann}-Problems},
    falls
    $\forall_{\varphi \in H^1(\Omega)}\;
    \int_\Omega (\nabla \varphi (A \nabla u + h) + \varphi(bu + f))\dx = 0$.
\end{Def}

\begin{Bem}
    Zusätzlich sei vorausgesetzt, dass $b \ge 0$ für das Dirichlet-Problem und
    $b \ge b_0 > 0$ für das Neumann-Problem gilt.
    Dann gilt folgender Satz.
\end{Bem}

\begin{Satz}{eindeutige Lösung von elliptischen DGL}
    Unter obigen Voraussetzungen existiert genau eine schwache Lösung des Dirichlet-
    bzw. des Neumann-Problems.
\end{Satz}

\begin{Bem}
    Unter zusätzlichen Regularitätsannahmen an die Daten $a_{ij}$, $h_i$, $b$, $f$ und
    $\partial\Omega$ kann man zeigen, dass die schwache Lösung so regulär ist,
    dass sie auch eine klassische Lösung ist.
    Beispielsweise folgt aus $a_{ij} \in \C^{m,1}(\Omega)$, $h_i \in H^{m+1}(\Omega)$,
    $f \in H^m(\Omega)$ und $\partial\Omega$ lokal als Graph von $\C^{m+1,1}$-Funktionen
    darstellbar, dass $u \in H^{m+2}(\Omega)$,
    und damit für hinreichend großes $m = m(n)$, dass $u \in \C^2(\Omega)$.
    Details siehe elliptische Regularitätstheorie ($L^2$-, $L^p$- und $\C^{0,\alpha}$-Theorie)
    mithilfe der Sobolevschen Einbettungssätze (siehe Funktionalanalysis 2).
\end{Bem}

\section{%
    \name{Ritz}-\name{Galerkin}-Approximation für elliptische RWP%
}

\begin{Satz}{\name{Ritz}-\name{Galerkin}-Approximation}
    Sei $u \in H^1_0(\Omega)$ bzw. $u \in H^1(\Omega)$ die schwache Lösung des
    Dirichlet- bzw. Neumann-Problems.
    Für $N \in \natural$ sei $X_N$ ein $N$-dimensionaler
    Unterraum von $H_0^1(\Omega)$ bzw. von $H^1(\Omega)$
    mit der Basis $\{\varphi_k^{(N)} \;|\; k = 1, \dotsc, N\}$.\\
    Dann existiert genau ein $u_N \in X_N$
    (\begriff{\name{Ritz}-\name{Galerkin}-Approximation}), sodass\\
    $\forall_{\varphi \in X_n}\;
    \int_\Omega (\nabla \varphi (A \nabla u + h) + \varphi(bu_N + f))\dx = 0$.\\
    Es gilt $u_N = \sum_{k=1}^N u_{N,k} \varphi_k^{(N)}$,
    wobei sich die Koef"|fizienten $u_{N,k} \in \real$ als eindeutige Lösung des LGS
    $\sum_{\ell=1}^N a_{k\ell}^{(N)} u_{N,\ell} + c_k^{(N)} = 0$, $k = 1, \dotsc, N$ mit
    $c_k^{(N)} := \int_\Omega (\nabla \varphi_k^{(N)} h + \varphi_k^{(N)} f)\dx$ und
    $a_{k\ell}^{(N)} := \int_\Omega (A \nabla \varphi_k^{(N)} \nabla \varphi_\ell^{(N)} + b\varphi_k^{(N)} \varphi_\ell^{(N)}) \dx$
    bestimmen lassen.
\end{Satz}

\begin{Bem}
    Die Nachweis der Struktur des LGS erfolgt durch direktes Nachrechnen.
    Der Beweis der eindeutigen Existenz von $u_N$ kann man mit Lax-Milgram
    (angewendet im Hilbertraum $X_N$) durchführen
    oder man zeigt, dass die Voraussetzungen an $a_{ij}$,
    insbesondere die gleichmäßige Elliptizitätsbedingung,
    die Invertierbarkeit der Matrix des LGS implizieren.
\end{Bem}

\linie

\begin{Lemma}{\name{Céa}-Lemma}
    Es gilt $\norm{u - u_N}_{H^1} \le C \cdot \inf_{v \in X_N} \norm{u - v}_{H^1}$,
    wobei die Konstante $C > 0$ nur von den Konstanten im Satz von Lax-Milgram abhängt.
\end{Lemma}

\begin{Bem}
    Das Céa-Lemma ist die zentrale Fehlerabschätzung für Ritz-Galerkin-Approxi"-mationen.
    Es besagt, dass die Ritz-Galerkin-Approximation bis auf eine multiplikative Konstante
    die beste Approximation ist.
    Weil $H^1$ separabel ist, können die $X_N$ so gewählt werden, dass
    $\inf_{v \in X_N} \norm{u - v}_{H^1} \xrightarrow{N \to \infty} 0$.\\
    Für weitere Fehlerabschätzungen bzgl. numerischer Verfahren, die bei der numerischen Berechnung
    der Ritz-Galerkin-Approximation eingesetzt werden
    (Interpolation, numerische Integra\-tion, iterative LGS-Löser) siehe Numerik-Veranstaltungen.
\end{Bem}

\pagebreak
