\section{%
    Lokalkonvexe und schwache Topologien%
}

\subsection{%
    Grundbegrif"|fe aus der Topologie%
}

\begin{Def}{topologischer Raum}
    Seien $X$ eine Menge und $\T \subset \P(X)$.\\
    Dann heißt $(X, \T)$ \begriff{topologischer Raum}, falls
    \begin{enumerate}
        \item
        $\emptyset \in \T$, $X \in \T$,

        \item
        $\forall_{\T' \subset \T}\; \bigcup_{U \in \T'} U \in \T$ und

        \item
        $\forall_{U_1, U_2 \in \T}\; U_1 \cap U_2 \in \T$.
    \end{enumerate}
    In diesem Fall heißt $\T$ \begriff{Topologie} auf $X$ und die Elemente von $\T$ heißen
    \begriff{of"|fen}.
\end{Def}

\begin{Bem}
    Im Folgenden ist $(X, \T)$ ein topologischer Raum und $M \subset X$.
\end{Bem}

\begin{Def}{abgeschlossen}
    $M \subset X$ heißt \begriff{abgeschlossen}, falls $X \setminus M$ of"|fen ist.
\end{Def}

\begin{Def}{Inneres}
    $\interior{M} := \{x \in M \;|\; \exists_{O \in \T}\; O \subset M,\; x \in O\}$
    heißt das \begriff{Innere} von $M$.
\end{Def}

\begin{Def}{Abschluss}
    $\overline{M} := X \setminus \interior{X \setminus M}$ heißt \begriff{Abschluss} von $M$.
\end{Def}

\begin{Def}{Rand}
    $\partial M := \overline{M} \setminus \interior{M}$ heißt \begriff{Rand} von $M$.
\end{Def}

\begin{Def}{dicht}
    $M$ heißt \begriff{dicht} in $X$, falls $\overline{M} = X$.
\end{Def}

\begin{Satz}{abgeschlossene Mengen}
    $\emptyset$ und $X$ sind abgeschlossen.
    Schnitte beliebig vieler und Vereinigungen endlicher vieler abgeschlossener Mengen sind
    abgeschlossen.
\end{Satz}

\linie

\begin{Def}{Umgebung}
    Seien $(X, \T)$ ein topologischer Raum und $x \in X$.\\
    Dann heißt $U \subset X$ \begriff{Umgebung} von $x$, falls
    $\exists_{O \in \T}\; O \subset U,\; x \in O$
    (d.\,h. $x \in \interior{U}$).
\end{Def}

\begin{Def}{Umgebungsfilter}
    $\U(x) := \{U \subset X \;|\; U \text{ Umgebung von } x\}$ heißt
    \begriff{Umgebungsfilter} von $x$.
\end{Def}

\begin{Def}{Umgebungsbasis}\\
    Eine Teilfamilie $\V(x) \subset \U(x)$ heißt \begriff{Umgebungsbasis} von $x$, falls
    $\forall_{U \in \U(x)} \exists_{V \in \V(x)}\; V \subset U$.
\end{Def}

\begin{Satz}{Eigenschaften des Umgebungsfilters}
    Seien $(X, \T)$ ein topologischer Raum und $x \in X$.\\
    Dann gilt:
    \begin{enumerate}
        \item
        $\forall_{U \in \U(x)}\; x \in U$

        \item
        $\forall_{U \in \U(x)} \exists_{V \in \U(x)} \forall_{y \in V}\; U \in \U(y)$

        \item
        $\forall_{U \in \U(x)} \forall_{V \supset U}\; V \in \U(x)$

        \item
        $\forall_{U, V \in \U(x)}\; U \cap V \in \U(x)$
    \end{enumerate}
\end{Satz}

\begin{Satz}{Umgebungsfilter induziert Topologie}
    Sei $X$ eine Menge und $\U(x) \subset \P(X)$ für jedes $x \in X$,
    sodass \emph{(1)} bis \emph{(4)} von eben erfüllt sind.
    Dann gibt es genau eine Topologie $\T$ auf $X$, sodass $\U(x)$ für $x \in X$ der
    Umgebungsfilter von $x$ ist.
    Es gilt $\T = \bigcup_{x \in X} \O(x) \cup \{\emptyset\}$,
    wobei $\O(x) := \{\interior{U} \;|\; U \in \U(x)\}$ und
    $\interior{U} := \{y \in X \;|\; U \in \U(y)\}$.
\end{Satz}

\linie
\pagebreak

\begin{Satz}{Metrik induziert Topologie}
    Jeder metrische Raum induziert einen topologischen Raum.
    In diesem Fall besitzt jeder Punkt $x$ des topologischen Raums eine abzählbare
    Umgebungsbasis $\V(x)$.
    Allerdings ist nicht jeder topologische Raum \begriff{metrisierbar}
    (d.\,h. die Topologie wird nicht von einer Metrik induziert).
\end{Satz}

\begin{Def}{feiner/gröber}
    Seien $\T_1, \T_2$ Topologien auf $X$.\\
    Dann heißt $\T_2$ \begriff{stärker/feiner} als $\T_1$ bzw.
    $\T_1$ \begriff{schwächer/gröber} als $\T_2$, falls $\T_1 \subsetneqq \T_2$.
\end{Def}

\begin{Def}{\name{Hausdorff}-Raum}
    Ein topologischer Raum $(X, \T)$ heißt \begriff{\name{Hausdorff}-Raum}, falls\\
    $\forall_{x, y \in X,\; x \not= y} \exists_{U \in \U(x)} \exists_{V \in \U(y)}\;
    U \cap V = \emptyset$.
\end{Def}

\begin{Def}{Konvergenz}
    Eine Folge $(x_n)_{n \in \natural}$ in $X$ konvergiert gegen $x \in X$
    ($x_n \xrightarrow{n \to \infty} x$), falls\\
    $\forall_{U \in \U(x)} \exists_{n_U \in \natural} \forall_{n \ge n_U}\; x_n \in V$.
\end{Def}

\begin{Satz}{GWe in \name{Hausdorff}-Räumen eindeutig}\\
    Grenzwerte von Folgen in Hausdorff-Räumen sind eindeutig.
\end{Satz}

\begin{Def}{folgenabgeschlossen}\\
    $A \subset X$ heißt \begriff{folgenabgeschlossen}, falls
    $\forall_{x \in X} \forall_{(x_n)_{n \in \natural} \text{ Folge in } A,\; x_n \to x}\;
    x \in A$.
\end{Def}

\begin{Satz}{abg. $\Rightarrow$ folgenabg.}
    Wenn $A \subset X$ abgeschlossen ist,\\
    dann ist $A$ auch folgenabgeschlossen.
    Die Umkehrung gilt i.\,A. nicht.
\end{Satz}

\linie

\begin{Def}{stetig}
    Seien $(X, \T_X), (Y, \T_Y)$ topologische Räume.\\
    Eine Abbildung $T\colon X \rightarrow Y$ heißt \begriff{stetig}, falls
    $\forall_{x \in X} \forall_{V \in \U(Tx)} \exists_{U \in \U(x)}\; T(U) \subset V$.
\end{Def}

\begin{Satz}{äquivalente Beschreibungen von Stetigkeit}
    Folgende Aussagen sind äquivalent:
    \begin{enumerate}
        \item
        $T$ ist stetig.

        \item
        Für alle of"|fenen Teilmengen $O \subset Y$ ist $T^{-1}(O) \subset X$ of"|fen.

        \item
        Für alle abgeschlossenen Teilmengen $A \subset Y$ ist $T^{-1}(A) \subset X$ abgeschlossen.
    \end{enumerate}
\end{Satz}

\begin{Satz}{stetig $\Rightarrow$ folgenstetig}
    Wenn $T$ stetig ist, dann ist $T$ auch \begriff{folgenstetig}, d.\,h.\\
    $\forall_{x \in X}
    \forall_{(x_n)_{n \in \natural} \text{ Folge in } X,\; x_n \to x}\;
    T(x_n) \xrightarrow{n \to \infty} T(x)$.
    Die Umkehrung gilt i.\,A. nicht.
\end{Satz}

\linie

\begin{Def}{kompakt}
    Sei $(X, \T)$ ein topologischer Raum.
    $K \subset X$ heißt \begriff{kompakt}, falls\\
    $\forall_{I \text{ Indexmenge}} \forall_{O_i \subset X \text{ of"|fen},\;
    K \subset \bigcup_{i \in I} O_i} \exists_{i_1, \dotsc, i_n \in I}\;
    K \subset \bigcup_{j=1}^n O_{i_j}$.
\end{Def}

\begin{Def}{folgenkompakt}
    $K$ heißt \begriff{folgenkompakt}, falls\\
    $\forall_{(x_n)_{n \in \natural} \text{ Folge in } K}
    \exists_{(x_{n_k})_{k \in \natural} \text{ Teilfolge}} \exists_{x \in K}\;
    x = \lim_{k \to \infty} x_{n_k}$.
\end{Def}

\begin{Bem}
    Kompaktheit und Folgenkompaktheit sind i.\,A. nicht äquivalent.
\end{Bem}

\linie

\begin{Def}{separabel}
    Sei $(X, \T)$ ein topologischer Raum.\\
    Dann heißt $(X, \T)$ separabel, falls $X$ eine abzählbare, dichte Teilmenge enthält.
\end{Def}

\linie

\begin{Satz}{Relativtopologie}
    Seien $(X, \T)$ ein topologischer Raum und $A \subset X$.\\
    Dann ist $(A, \T_A)$ ein topologischer Raum mit der \begriff{Relativtopologie}
    $\T_A := \{U \cap A \;|\; U \in \T\}$.
\end{Satz}

\begin{Satz}{Produkttopologie}
    Seien $I$ eine Indexmenge, $(X_i, \T_i)_{i \in I}$ eine Familie topologischer Räume und
    $X := \prod_{i \in I} X_i$.
    Dann ist $(X, \T)$ ein topologischer Raum mit der \begriff{Produkttopologie} $\T$ mit Basis
    $\{\prod_{i \in I} O_i \;|\; \forall_{i \in I}\; O_i \in \T_i,\;
    \text{fast alle } O_i = X_i\}$
    (beliebige Vereinigungen hinzunehmen).
\end{Satz}

\begin{Satz}{Satz von \scshape\,\!\name{Tychonov}}
    Seien $I$ eine Indexmenge, $(X_i, \T_i)_{i \in I}$ eine Familie topologischer Räume,
    $X := \prod_{i \in I} X_i$ und $\T$ die Produkttopologie auf $X$.\\
    Dann ist $X$ kompakt genau dann, wenn $X_i$ für alle $i \in I$ kompakt ist.
\end{Satz}

\begin{Bem}
    Dieser Satz ist äquivalent zum Auswahlaxiom.
\end{Bem}

\pagebreak

\subsection{%
    Lokalkonvexe Topologie%
}

\begin{Def}{lokalkonvexe Topologie}
    Seien $X$ ein $\KK$-Vektorraum und $(p_\alpha)_{\alpha \in I}$ eine Familie von
    Halbnormen auf $X$ ($I$ Indexmenge).
    Für $x \in X$ definiert man
    \begin{itemize}
        \item
        $U_{\varepsilon,H}(x) := \{y \in X \;|\; \forall_{\alpha \in H}\;
        p_\alpha(x - y) < \varepsilon\}$
        für $\varepsilon > 0$ und $H \subset I$ endlich,

        \item
        $\V(x) := \{U_{\varepsilon,H}(x) \;|\; \varepsilon > 0,\; H \subset I \text{ endlich}\}$,

        \item
        $\U(x) := \{U \subset X \;|\; \exists_{V \in \V(x)}\; V \subset U\}$ und

        \item
        $\T := \{O \subset X \;|\; \forall_{x \in O} \exists_{V \in \V(x)}\; V \subset O\}$.
    \end{itemize}
    Man kann zeigen, dass $(X, \T)$ ein topologischer Raum ist,
    wobei $\U(x)$ der Umgebungsfilter und $\V(x)$ eine Umgebungsbasis von $x \in X$ ist.
    $\T$ heißt die von $(p_\alpha)_{\alpha \in I}$ induzierte \begriff{lokalkonvexe Topologie}
    auf $X$ und $(X, \T)$ heißt \begriff{lokalkonvexer (topologischer) Raum}.
\end{Def}

\begin{Bem}
    Die Topologie heißt deshalb lokalkonvex, weil es für jeden Punkt $x \in X$ eine
    Umgebungsbasis aus konvexen Mengen $U_{\varepsilon,H}(x)$ gibt.
\end{Bem}

\begin{Bem}
    $(X, \T)$ ist bereits eindeutig durch die Nullumgebungsbasis $\V(0)$ festgelegt,
    da $\V(x) = x + \V(0)$ und $\U(x) = x + \U(0)$
    (weil $U_{\varepsilon,H}(x) = x + U_{\varepsilon,H}(0)$).
\end{Bem}

\linie

\begin{Lemma}{Charakterisierung der Konvergenz}
    Seien $(X, \T)$ ein lokalkonvexer Raum, der durch $(p_\alpha)_{\alpha \in I}$ induziert wird,
    $(x_n)_{n \in \natural}$ eine Folge in $X$ und $x \in X$.
    Dann sind äquivalent:
    \begin{enumerate}
        \item
        $x_n \xrightarrow{n \to \infty} x$

        \item
        $x_n - x \xrightarrow{n \to \infty} 0$

        \item
        $\forall_{\alpha \in I}\; p_\alpha(x_n - x) \xrightarrow{n \to \infty} 0$
    \end{enumerate}
\end{Lemma}

\linie

\begin{Lemma}{Charakterisierung von \name{hausdorff}sch}\\
    Sei $(X, \T)$ ein lokalkonvexer Raum, der durch $(p_\alpha)_{\alpha \in I}$ induziert wird.
    Dann sind äquivalent:
    \begin{enumerate}
        \item
        $(X, \T)$ ist hausdorffsch.

        \item
        $\forall_{x \in X \setminus \{0\}} \exists_{\alpha \in I}\; p_\alpha(x) \not= 0$
    \end{enumerate}
\end{Lemma}

\linie

\begin{Lemma}{Charakterisierung von Stetigkeit}
    Seien $(X, \T_X), (Y, \T_Y)$ von den Halbnormfamilien $(p_\alpha)_{\alpha \in I_X}$ bzw.
    $(q_\beta)_{\beta \in I_Y}$ induzierte lokalkonvexe Räume und $T\colon X \rightarrow Y$
    linear.\\
    Dann sind äquivalent:
    \begin{enumerate}
        \item
        $T$ ist stetig.

        \item
        $T$ ist stetig in $0$.

        \item
        $\forall_{\beta \in I_Y} \exists_{H \subset I_X \text{ endlich}} \exists_{M \ge 0}
        \forall_{x \in X}\; q_\beta(Tx) \le M \cdot \max_{\alpha \in H} p_\alpha(x)$
    \end{enumerate}
\end{Lemma}

\begin{Kor}
    Seien $(X, \T)$ ein lokalkonvexer Raum und $T\colon X \rightarrow \KK$ linear.\\
    Dann ist $T$ stetig genau dann, wenn
    $\exists_{H \subset I \text{ endlich}} \exists_{M \ge 0}
    \forall_{x \in X}\; |Tx| \le M \cdot \max_{\alpha \in H} p_{\alpha_i}(x)$.
\end{Kor}

\begin{Def}{Dualraum}
    Sei $(X, \T)$ ein lokalkonvexer Raum.\\
    Dann heißt $X' := \{T\colon X \rightarrow \KK \;|\; T \text{ linear und stetig}\}$
    \begriff{Dualraum} von $X$.
\end{Def}

\begin{Bem}
    Es gibt Verallgemeinerungen des Satzes von Hahn-Banach und der Trennungssätze
    für lokalkonvexe Räume.
\end{Bem}

\pagebreak

\subsection{%
    Schwache Konvergenz und \texorpdfstring{Schwach$\ast$}{Schwach*}-Konvergenz%
}

\begin{Def}{schwache Topologie}
    Seien $X$ ein normierter Raum und $X'$ der Dualraum von $X$.\\
    $(p_{x'})_{x' \in X'}$ mit $p_{x'}(x) := |x'(x)|$ für $x \in X$ ist eine Familie von
    Halbnormen auf $X$.\\
    Die induzierte lokalkonvexe Topologie heißt \begriff{schwache Topologie}
    $\sigma(X, X')$ auf $X$.
\end{Def}

\begin{Def}{Schwach$\ast$-Topologie}
    Seien $X$ ein normierter Raum und $X'$ der Dualraum von $X$.\\
    $(p_x)_{x \in X}$ mit $p_x(x') := |x'(x)|$ für $x' \in X'$ ist eine Familie von
    Halbnormen auf $X'$.\\
    Die induzierte lokalkonvexe Topologie heißt \begriff{Schwach$\ast$-Topologie}
    $\sigma(X', X)$ auf $X'$.
\end{Def}

\begin{Def}{schwache Konvergenz}
    Seien $X$ ein normierter Raum, $(x_n)_{n \in \natural}$ eine Folge in $X$ und $x \in X$.\\
    Dann \begriff{konvergiert} $(x_n)_{n \in \natural}$ \begriff{schwach} gegen $x$
    ($x_n \rightharpoonup x$), falls
    $x_n \xrightarrow{n \to \infty} x$ bzgl. $\sigma(X, X')$.
\end{Def}

\begin{Def}{Schwach$\ast$-Konvergenz}
    Seien $X$ ein normierter Raum, $(x_n')_{n \in \natural}$ eine Folge in $X'$ und $x \in X'$.\\
    Dann \begriff{konvergiert} $(x_n')_{n \in \natural}$ \begriff{schwach$\ast$} gegen $x'$
    ($x_n' \xrightharpoonup{\ast} x'$), falls
    $x_n' \xrightarrow{n \to \infty} x'$ bzgl. $\sigma(X', X)$.
\end{Def}

\begin{Bem}
    Schwache Konvergenz ist äquivalent
    zu $\forall_{x' \in X'}\; x'(x_n) \xrightarrow{n \to \infty} x'(x)$.\\
    Analog ist Schwach$\ast$-Konvergenz äquivalent
    zu $\forall_{x \in X}\; x_n'(x) \xrightarrow{n \to \infty} x'(x)$.
\end{Bem}

\subsection{%
    Distributionen%
}

\begin{Def}{Distributionen}
    Seien $\Omega \subset \real^d$ of"|fen und $\D(\Omega) := \C^\infty_c(\Omega)$.
    Zunächst definiert man Halbnormen $(p_m)_{m \in \natural_0}$
    durch $p_m(\varphi) := \sup_{|\beta| \le m} \norm{\partial_x^\beta \varphi}_{\C^0(\Omega)}$.
    Anschließend definiert man $(p_\alpha)_{\alpha \in I}$ als Familie aller Halbnormen
    $p_\alpha$, sodass
    $\forall_{K \subset \Omega \text{ kpkt.}} \exists_{C \ge 0} \exists_{m \in \natural_0}
    \forall_{\varphi \in \C^\infty_c(K)}\; p_\alpha(\varphi) \le C \cdot p_m(\varphi)$.\\
    Dann heißt der Dualraum $\D'(\Omega)$ von $(\D(\Omega), \T_\D)$
    \begriff{Raum der Distributionen} auf $\Omega$,
    wobei $\T_\D$ die von $(p_\alpha)_{\alpha \in I}$ induzierte lokalkonvexe Topologie auf
    $\D(\Omega)$ ist.
\end{Def}

\begin{Bem}
    Sei $(\varphi_n)_{n \in \natural}$ eine Folge in $\D(\Omega)$ und $\varphi \in \D(\Omega)$.\\
    Dann gilt $\varphi_n \xrightarrow{n \to \infty} \varphi$ (bzgl. $\T_\D$) genau dann, wenn
    es $K \subset \Omega$ kompakt gibt mit\\
    $\supp(\varphi) \subset K$,
    $\forall_{n \in \natural}\; \supp(\varphi_n) \subset K$ und
    $\forall_{\beta \in \natural_0^d}\; \partial_x^\beta \varphi_n
    \xrightarrow{\norm{\cdot}_{\C^0(K)}} \partial_x^\beta \varphi$.
\end{Bem}

\linie

\begin{Lemma}{Eigenschaften des Testfunktionenraums}
    \begin{enumerate}
        \item
        Seien $K \subset \Omega$ kompakt und $\D_K(\Omega) := \C^\infty_c(K)$.
        Dann ist die von $(p_m)_{m \in \natural_0}$ erzeugte lokalkonvexe Topologie
        auf $\D_K(\Omega)$ gleich der
        Relativtopologie von $(D(\Omega), \T_\D)$ auf $\D_K(\Omega)$.

        \item
        $(\D(\Omega), \T_\D)$ ist hausdorffsch.
    \end{enumerate}
\end{Lemma}

\begin{Lemma}{Charakterisierung der Stetigkeit}
    Sei $T\colon \D(\Omega) \rightarrow \KK$ linear.
    Dann sind äquivalent:
    \begin{enumerate}
        \item
        $T \in \D'(\Omega)$

        \item
        $\forall_{K \subset \Omega \text{ kpkt.}}\; T|_{\D_K(\Omega)} \in (\D_K(\Omega))'$

        \item
        $\forall_{K \subset \Omega \text{ kpkt.}} \exists_{m \in \natural_0}
        \exists_{C \ge 0} \forall_{\varphi \in \D_K(\Omega)}\; |T\varphi| \le C p_m(\varphi)$

        \item
        $T$ ist \begriff{folgenstetig}, d.\,h.
        aus $\varphi_n \to \varphi$ in $(\D(\Omega), \T_D)$ folgt $T\varphi_n \to T\varphi$.

        \item
        $T$ ist folgenstetig in $0$, d.\,h.
        aus $\varphi_n \to 0$ in $(\D(\Omega), \T_D)$ folgt $T\varphi_n \to 0$.
    \end{enumerate}
\end{Lemma}

\linie

\begin{Def}{Schwach$\ast$-Topologie für Distributionen}
    Sei die Familie $(p_\varphi)_{\varphi \in \D(\Omega)}$ von Halbnormen auf $\D'(\Omega)$
    definiert durch $p_\varphi(T) := |T\varphi|$ für alle $T \in \D'(\Omega)$.\\
    Dann heißt die von $(p_\varphi)_{\varphi \in \D(\Omega)}$ induzierte lokalkonvexe Topologie
    \begriff{Schwach$\ast$-Topologie}\\
    $\sigma(\D'(\Omega), \D(\Omega))$ auf $\D'(\Omega)$.
\end{Def}

\begin{Bem}
    Sei $(T_n)_{n \in \natural}$ eine Folge in $\D'(\Omega)$ und $T \in \D'(\Omega)$.\\
    Dann gilt $T_n \xrightarrow{n \to \infty} T$ (bzgl. $\sigma(\D'(\Omega), \D(\Omega))$)
    genau dann, wenn $\forall_{\varphi \in \D(\Omega)}\;
    T_n \varphi \xrightarrow{n \to \infty} T \varphi$.
\end{Bem}

\pagebreak

\subsection{%
    Beispiele für Distributionen und distributionelle Ableitung%
}

\begin{Def}{induzierte reguläre Distribution}\\
    Sei $f \in L^1_\loc(\Omega)$
    mit $L^1_\loc(\Omega) := \{f\colon \Omega \rightarrow \real \text{ messbar} \;|\;
    \forall_{K \subset \Omega \text{ kpkt.}}\; f \in L^1(K)\}$.\\
    Dann heißt $T_f \in \D'(\Omega)$ mit $T_f(\varphi) := \int_\Omega f(y)\varphi(y)\dy$
    für $\varphi \in \D(\Omega)$
    die durch $f$ \begriff{induzierte reguläre Distribution}.
\end{Def}

\begin{Bem}
    Die Abbildung $L^1_\loc(\Omega) \rightarrow \D'(\Omega)$, $f \mapsto T_f$ ist injektiv
    (sie ist linear und aus $T_f = 0$ folgt $T_f(\varphi) = 0$ für alle $\varphi \in \D(\Omega)$,
    also $f = 0$ f.ü. nach dem Fundamentallemma der Variationsrechnung).
    Daher kann man die Funktionen $f \in L^1_\loc(\Omega)$ mit den
    induzierten regulären Distributionen $T_f \in \D'(\Omega)$ identifizieren.\\
    Eine Distribution $T \in \D'(\Omega)$ heißt \begriff{regulär},
    falls $\exists_{f \in L^1_\loc(\Omega)}\; T = T_f$,
    d.\,h. falls sie im Bild dieser Abbildung ist.
    Nicht jede Distribution ist regulär, wie die Delta-Distribution zeigt.
\end{Bem}

\begin{Def}{\name{Dirac}-/Delta-Distribution}
    Sei $x \in \Omega$.
    Dann heißt $\delta_x \in \D'(\Omega)$ mit $\delta_x(\varphi) := \varphi(x)$ für
    $\varphi \in \D(\Omega)$ \begriff{\name{Dirac}- oder Delta-Distribution} zum Punkt $x$.
    Man schreibt $\delta := \delta_0$.
\end{Def}

\begin{Bem}
    $\delta_x$ ist nicht regulär.
    Angenommen, es gilt $\delta_x = T_f$ für ein $f \in L^1_\loc(\Omega)$.
    Definiere für $\varphi \in \D(\Omega)$ die Testfunktion
    $\psi_\varphi \in \D(\Omega)$ durch $\psi_\varphi(y) := |y - x|^2 \varphi(y)$.
    Dann gilt für alle $\varphi \in \D(\Omega)$, dass
    $0 = \psi_\varphi(x) = \delta_x(\psi_\varphi) = T_f(\psi_\varphi) =
    \int_\Omega f(y) |y - x|^2 \varphi(y) \dy$.
    Nach dem Fundamentallemma der Variationsrechnung folgt,
    dass $f(y) |y - x|^2 = 0$ für fast alle $y \in \Omega$, d.\,h. $f = 0$ f.ü.
    Damit wäre aber $\delta_x = T_f = 0$, ein Widerspruch
    (es gibt $\varphi \in \D(\Omega)$ mit $\delta_x(\varphi) = \varphi(x) \not= 0$).\\
    Trotzdem schreibt man formal häufig
    $\int_\Omega \delta_x(y) \varphi(y) \dy := \varphi(x)$.
\end{Bem}

\begin{Satz}{\name{Dirac}-Folge}
    Sei $f_n \in L^1_\loc(\Omega)$ mit
    $f_n(x) := \left(\frac{n}{4\pi}\right)^{d/2} \exp\!\left(-\frac{n|x|^2}{4}\right)$
    für $n \in \natural$.\\
    Dann gilt $T_{f_n} \to \delta_0$ bzgl. $\sigma(\D'(\Omega), \D(\Omega))$.
\end{Satz}

\linie

\begin{Bem}
    Seien $f \in L^1_\loc(\Omega)$ und $\beta \in \natural_0^d$, sodass
    die partielle Ableitung $\partial^\beta_x f$ der Ordnung $\beta$ existiert.
    Dann gilt $T_{\partial^\beta_x f}(\varphi)
    = \int_{\real^d} (\partial^\beta_x f)(x) \varphi(x) \dx
    = (-1)^{|\beta|} \int_{\real^d} f(x) (\partial^\beta_x \varphi)(x) \dx$\\
    $= (-1)^{|\beta|} T_f(\partial^\beta_x \varphi)$ wegen partieller Integration.
    Die folgende Definition erklärt $T_{\partial^\beta_x f}$ zur "`Ableitung"' von $T_f$
    und verallgemeinert dies für nicht-reguläre Distributionen.
\end{Bem}

\begin{Def}{distributionelle Ableitung}
    Seien $T \in \D'(\Omega)$ und $\beta \in \natural_0^d$.\\
    Dann heißt $\partial^\beta_x T \in \D'(\Omega)$ mit
    $(\partial^\beta_x T)(\varphi) := (-1)^{|\beta|} T(\partial^\beta_x \varphi)$
    \begriff{distributionelle Ableitung} von $T$ der Ordnung $\beta$.
\end{Def}

\begin{Bsp}
    Seien $\Omega := (-1, 1)$ und $f \in L^1_\loc((-1, 1))$ mit $f(x) := |x|$.\\
    Dann gilt $T_f(\partial_x \varphi) = \int_{-1}^0 (-x)(\partial_x \varphi)(x) \dx +
    \int_0^1 x (\partial_x \varphi)(x) \dx = \int_{-1}^0 \varphi(x) \dx - \int_0^1 \varphi(x)\dx
    = -T_g(\varphi)$ für alle $\varphi \in \D((-1, 1))$ mit
    $g(x) := -1$ für $x < 0$, $g(x) := 0$ für $x = 0$ und $g(x) := 1$ für $x > 0$
    (Vorzeichenfunktion).
    Somit ist $T_g$ die distributionelle Ableitung von $T_f$
    (man identifiziert $f$ und $g$ mit $T_f$ bzw. $T_g$ und spricht oft davon,
    dass $g$ die distributionelle Ableitung von $f$ ist).
    Wegen $g \in L^2((-1, 1))$ ist $g$ auch die schwache Ableitung von $f$.

    Außerdem gilt
    $T_f(\partial_x^2 \varphi) =
    \int_{-1}^0 (\partial_x \varphi)(x)\dx - \int_0^1 (\partial_x \varphi)(x) \dx =
    2\varphi(0) = 2\delta(\varphi)$.
    Daher ist $2\delta$ die zweite distributionelle Ableitung von $T_f$ (bzw. von $f$),
    allerdings besitzt $g$ keine schwache Ableitung ($\delta$ ist keine Funktion).
\end{Bsp}

\begin{Satz}{distr. Ableitungsoperator stetig}\\
    Die Abbildung $\partial^\beta_x\colon \D'(\Omega) \rightarrow \D'(\Omega)$,
    $T \mapsto \partial^\beta_x T$ ist stetig bzgl. $\sigma(\D'(\Omega), \D(\Omega))$.
\end{Satz}

\pagebreak

\subsection{%
    Eigenschaften der schwachen Konvergenz und der Satz von \name{Alaoglu}%
}

\begin{Lemma}{schwache Konvergenz und Schwach$\ast$-Konvergenz}
    Seien $X$ ein normierter Raum,
    $(x_n)_{n \in \natural}$ eine Folge in $X$,
    $(x_n')_{n \in \natural}$ eine Folge in $X'$,
    $x \in X$ und
    $x' \in X'$.
    Dann gilt:
    \begin{enumerate}
        \item
        $x_n \rightharpoonup x \iff J_X x_n \xrightharpoonup{\ast} J_X x$ mit
        $J_X\colon X \rightarrow X''$, $(J_X x)(x') := x'(x)$ für $x \in X$, $x' \in X'$

        \item
        Aus $x_n' \rightharpoonup x'$ folgt $x_n' \xrightharpoonup{\ast} x'$.

        \item
        Der schwache Grenzwert von $(x_n)_{n \in \natural}$ (falls existent) und
        der Schwach$\ast$-Grenzwert von $(x_n')_{n \in \natural}$ (falls existent) sind eindeutig.

        \item
        Aus $x_n \to x$ (bzgl. $\norm{\cdot}_X$) folgt $x_n \rightharpoonup x$ und
        aus $x_n' \to x'$ (bzgl. $\norm{\cdot}_{X'}$) folgt $x_n' \xrightharpoonup{\ast} x'$.

        \item
        Aus $x_n' \xrightharpoonup{\ast} x'$ folgt
        $\norm{x'}_{X'} \le \liminf_{n \to \infty} \norm{x_n'}_{X'}$.

        \item
        Aus $x_n \rightharpoonup x$ folgt
        $\norm{x}_X \le \liminf_{n \to \infty} \norm{x_n}_X$\\
        (\begriff{Unterhalbstetigkeit} der Norm bzgl. der schwachen Konvergenz von Folgen).

        \item
        Konvergiert $(x_n)_{n \in \natural}$ schwach,
        so ist $(x_n)_{n \in \natural}$ beschränkt.\\
        Konvergiert $(x_n')_{n \in \natural}$ schwach$\ast$,
        so ist $(x_n')_{n \in \natural}$ beschränkt.

        \item
        Aus $x_n \to x$ und $x_n' \xrightharpoonup{\ast} x'$ folgt
        $x_n'(x_n) \to x'(x)$ in $\KK$.\\
        Aus $x_n \rightharpoonup x$ und $x_n' \to x'$ folgt
        $x_n'(x_n) \to x'(x)$ in $\KK$.
    \end{enumerate}
\end{Lemma}

\linie

\begin{Bsp}
    \begin{enumerate}[label=\emph{(\alph*)}]
        \item
        Sei $(\Omega, \Sigma, \mu)$ ein Maßraum, $p \in [1, \infty)$ und
        $p'$ mit $\frac{1}{p} + \frac{1}{p'} = 1$.
        Im Fall $p = 1$ sei $\mu$ zusätzlich $\sigma$-endlich.
        Dann ist $J_{p'}\colon L^{p'}(\Omega) \rightarrow (L^p(\Omega))'$ mit
        $(J_{p'} f)(g) := \int_\Omega g\overline{f} d\mu$ für $g \in L^p(\mu)$
        ein konjugiert linearer, isometrischer Isomorphismus.
        Für $p = 2$ ist $J_2 = \R_{L^2(\Omega)}$ gleich dem
        konjugiert linearen Isomorphismus aus dem Rieszschen Darstellungssatz.\\
        Seien $(f_k)_{k \in \natural}$ eine Folge in $L^p(\Omega)$ und $f \in L^p(\Omega)$.\\
        Dann gilt $f_k \rightharpoonup f$ in $L^p(\Omega)$ genau dann,
        wenn $\forall_{g \in L^{p'}(\Omega)}\; \int_\Omega f_k \overline{g} d\mu
        \xrightarrow{k \to \infty} \int_\Omega f\overline{g} d\mu$.

        \item
        Seien $K \subset \real^n$ kompakt und
        $\text{rca}(K)$ der Raum der signierten Borelmaße auf $K$.\\
        Dann ist $J\colon \text{rca}(K) \rightarrow (\C^0(K))'$ mit
        $(J\nu)(f) := \int_K fd\nu$ ein isometrischer Isomorphismus.\\
        Seien $(f_k)_{k \in \natural}$ eine Folge in $\C^0(K)$ und $f \in \C^0(K)$.\\
        Dann gilt $f_k \rightharpoonup f$ in $\C^0(K)$ genau dann,
        wenn $\forall_{\nu \in \text{rca}(K)}\; \int_K f_k d\nu
        \xrightarrow{k \to \infty} \int_K f d\nu$.

        \item
        Seien $\Omega \subset \real^n$ of"|fen, $m \in \natural$ und $p \in [1, \infty]$,\\
        außerdem $(u_k)_{k \in \natural}$ eine
        Folge in $W^{m,p}(\Omega)$ und $u \in W^{m,p}(\Omega)$.\\
        Dann gilt $u_k \rightharpoonup u$ in $W^{m,p}(\Omega)$ genau dann,
        wenn $\forall_{|s| \le m}\; \partial_x^s u_k \rightharpoonup \partial_x^s u$
        in $L^p(\Omega)$.\\
        Die gleiche Aussage gilt für $W^{m,p}_0(\Omega)$.
    \end{enumerate}
\end{Bsp}

\linie

\begin{Satz}{beschr. Folge in $X'$ besitzt schwach$\ast$ konv. TF für $X$ separabel}\\
    Sei $X$ ein separabler normierter Raum.\\
    Dann ist $\overline{B_1(0)} \subset X'$ schwach$\ast$ folgenkompakt.
\end{Satz}

\begin{Bem}
    In diesem Fall gilt diese Aussage auch für jede andere abgeschlossene Kugel
    $\overline{B_R(0)}$.
    Insbesondere besitzt jede beschränkte Folge in $X'$ eine schwach$\ast$ konvergente Teilfolge.\\
    Die Aussage gilt i.\,A. nicht, wenn $X$ nicht separabel ist.
\end{Bem}

\begin{Satz}{Satz von \scshape\,\!\name{Alaoglu}}
    Sei $X$ ein Banachraum.\\
    Dann ist $\overline{B_1(0)} \subset X'$ kompakt bzgl. der Schwach$\ast$-Topologie auf $X'$.
\end{Satz}

\pagebreak

\subsection{%
    Beste Approximationen in reflexiven Räumen%
}

\begin{Def}{reflexiv}
    Sei $X$ ein Banachraum.\\
    Dann heißt $X$ \begriff{reflexiv}, falls $J_X\colon X \rightarrow X''$
    (mit $(J_X x)x' = x'(x)$) surjektiv, also bijektiv ist.
\end{Def}

\begin{Satz}{(Gegen-)Beispiele für reflexive Räume}
    \begin{enumerate}
        \item
        Jeder Hilbertraum ist reflexiv.

        \item
        $L^p(\Omega)$ ist für $p \in (1, \infty)$ reflexiv.

        \item
        $W^{m,p}(\Omega)$ ist für $p \in (1, \infty)$ reflexiv.

        \item
        $\C^0(K)$ ist für $K$ kompakt und unendlich nicht reflexiv.
    \end{enumerate}
\end{Satz}

\begin{Lemma}{Eigenschaften reflexiver Räume}
    Sei $X$ ein Banachraum.
    \begin{enumerate}
        \item
        Ist $X$ reflexiv, dann stimmen schwache Konvergenz in $X'$ und
        Schwach$\ast$-Konvergenz in $X'$ überein.

        \item
        Ist $X$ reflexiv, dann ist auch jeder abgeschlossene Unterraum von $X$ reflexiv.

        \item
        Sei $Y$ ein zu $X$ isomorpher Banachraum.\\
        Dann ist $X$ reflexiv genau dann, wenn $Y$ reflexiv ist.

        \item
        $X$ ist reflexiv genau dann, wenn $X'$ reflexiv ist.
    \end{enumerate}
\end{Lemma}

\linie

\begin{Satz}{beschr. Folge in $X$ besitzt schwach konv. TF für $X$ reflexiv}\\
    Sei $X$ ein reflexiver Banachraum.\\
    Dann ist $\overline{B_1(0)} \subset X$ schwach folgenkompakt.
\end{Satz}

\begin{Bem}
    In diesem Fall gilt diese Aussage auch für jede andere abgeschlossene Kugel
    $\overline{B_R(0)}$.
    Insbesondere besitzt jede beschränkte Folge in $X$ eine schwach konvergente Teilfolge.
\end{Bem}

\begin{Lemma}{$X'$ separabel $\Rightarrow$ $X$ separabel}
    Sei $X$ ein Banachraum mit $X'$ separabel.\\
    Dann ist auch $X$ separabel.
\end{Lemma}

\linie

\begin{Def}{Vollstetigkeit}
    Seien $X, Y$ Banachräume und $T\colon X \rightarrow Y$ linear.
    Dann heißt $T$ \begriff{vollstetig}, falls
    für alle Folgen $(x_n)_{n \in \natural}$ in $X$ und $x \in X$ mit $x_n \rightharpoonup x$
    gilt, dass $Tx_n \to Tx$.
\end{Def}

\begin{Satz}{Vollstetigkeit}
    Seien $X, Y$ Banachräume und $T\colon X \rightarrow Y$ linear.
    \begin{enumerate}
        \item
        Ist $T$ kompakt, dann ist $T$ vollstetig.

        \item
        Ist $X$ reflexiv und $T$ vollstetig, dann ist $T$ kompakt.
    \end{enumerate}
\end{Satz}

\linie

\begin{Satz}{konvexe abg. Menge schwach folgenabg.}\\
    Seien $X$ ein normierter Raum und $M \subset X$
    nicht-leer, konvex und abgeschlossen.\\
    Dann ist $M$ schwach folgenabgeschlossen.
\end{Satz}

\linie

\begin{Satz}{bestapproximierendes Element für reflexive Räume}\\
    Seien $X$ ein reflexiver Banachraum und $M \subset X$
    nicht-leer, konvex und abgeschlossen.\\
    Dann gilt $\forall_{x_0 \in X} \exists_{y_0 \in M}\;
    \norm{x_0 - y_0} = \dist(x_0, M)$.
\end{Satz}

\pagebreak
