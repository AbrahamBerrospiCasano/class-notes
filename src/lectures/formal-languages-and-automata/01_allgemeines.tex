\chapter{%
    Allgemeines%
}

\section{%
    Grammatiken%
}

\begin{Def}{Alphabet}
    Eine endliche, nicht-leere Menge $\Sigma$ heißt \begriff{Alphabet}.\\
    Die Elemente von $\Sigma$ heißen \begriff{Buchstaben},
    \begriff{Zeichen} oder \begriff{Terminalsymbole}.
\end{Def}

\begin{Def}{Menge $\Sigma^\ast$ aller Wörter}
    Sei $\Sigma$ ein Alphabet.
    Dann ist $\Sigma^\ast$ die \begriff{Menge aller (endlichen) Wörter},
    die über $\Sigma$ gebildet werden können
    (dabei ist auch das \begriff{leere Wort} $\varepsilon$ zugelassen).\\
    Ein \begriff{Wort} ist dabei eine endliche Folge von
    Buchstaben aus $\Sigma$.
    Außerdem sei $\Sigma^+ := \Sigma^\ast \setminus \{\varepsilon\}$.
\end{Def}

\begin{Bem}
    Es gibt $1$ Wort der Länge $0$, $|\Sigma|$ Wörter der Länge $1$,
    $|\Sigma|^2$ Wörter der Länge $2$ usw., d.\,h.
    $|\Sigma|^k$ Wörter der Länge $k$.
    $\Sigma^\ast$ ist somit immer abzählbar unendlich.\\
    Ein \begriff{Monoid} ist eine Menge mit einer Verknüpfung $\circ$, sodass
    $\forall_{a, b, c}\; (a \circ b) \circ c = a \circ (b \circ c)$
    (Assoziativität) und $\exists_{e} \forall_{a}\; e \circ a = a \circ e = a$
    (neutrales Element).
    Man kann daher $\Sigma^\ast$ auch freies Monoid über $\Sigma$ nennen,
    wobei die Grundmenge $\Sigma$ und die Verknüpfung die Konkatenation ist
    ($e$ ist das leere Wort).
    "`Frei"' deshalb, weil sich jedes Wort aus $\Sigma^\ast$ auf eindeutige
    Weise als Verknüpfung von Buchstaben aus $\Sigma$ darstellen lässt.
\end{Bem}

\begin{Def}{formale Sprache}
    Sei $\Sigma$ ein Alphabet.
    Eine Teilmenge von $\Sigma^\ast$ heißt \begriff{formale Sprache}.
\end{Def}

\begin{Bem}
    Aufgrund $|\Sigma^\ast| = \aleph_0$ ist die Kardinalität der Menge aller
    formalen Sprachen gleich $\aleph_1$.
    Da formale Sprachen selbst auch meistens unendlich sind,
    benötigt man für sie endliche Beschreibungsmöglichkeiten.
    Dafür dienen die Grammatiken und die Automaten.
\end{Bem}

\linie

\begin{Def}{Grammatik}
    Eine \begriff{Grammatik} ist ein $4$-Tupel $G = (V, \Sigma, P, S)$, wobei
    \begin{itemize}
        \item
        $V$ eine endliche, nicht-leere Menge
        (die Menge der \begriff{Variablen}),

        \item
        $\Sigma$ eine endliche, nicht-leere Menge mit
        $V \cap \Sigma = \emptyset$
        (das \begriff{Terminalalphabet}),

        \item
        $P$ eine endliche Teilmenge von
        $(V \cup \Sigma)^+ \times (V \cup \Sigma)^\ast$\\
        (die Menge der \begriff{Regeln} oder \begriff{Produktionen}) und

        \item
        $S \in V$ (die \begriff{Startvariable}) ist.
    \end{itemize}
    Für $(u, v) \in P$ schreibt man auch $u \rightarrow v$.
\end{Def}

\begin{Def}{Satzform}
    Ein Wort $w \in (V \cup \Sigma)^\ast$ heißt \begriff{Satzform}.
\end{Def}

\begin{Def}{Übergangsrelation}
    Seien $G = (V, \Sigma, P, S)$ eine Grammatik und
    $u, v \in (V \cup \Sigma)^\ast$.\\
    Dann sei $u \Rightarrow_G v$, falls
    $u = w_1 u_1 w_2$, $v = w_1 u_2 w_2$ mit
    $w_1, w_2 \in (V \cup \Sigma)^\ast$ und $(u_1, u_2) \in P$.\\
    Dies definiert eine Relation $\Rightarrow_G$
    auf $(V \cup \Sigma)^\ast$, sie heißt \begriff{Übergangsrelation}.\\
    $\Rightarrow_G^\ast$ ist die ref"|lexive und transitive Hülle von
    $\Rightarrow_G$
    (d.\,h. $u \Rightarrow_G^\ast v$ gilt genau dann, wenn es Wörter
    $w_1, \dotsc, w_k \in (V \cup \Sigma)^\ast$, $k \in \natural_0$ gibt mit
    $u \Rightarrow_G w_1 \Rightarrow_G \dotsb \Rightarrow_G w_k
    \Rightarrow_G v$ oder wenn $u = v$).
\end{Def}

\begin{Def}{Ableitung}
    Eine Folge von Wörtern $(S, w_1, \dotsc, w_k)$ mit
    $w_k \in (V \cup \Sigma)^\ast$, $k \in \natural$ und\\
    $S \Rightarrow_G w_1 \Rightarrow_G \dotsb \Rightarrow_G w_k$
    heißt \begriff{Ableitung} von $w_k$.
\end{Def}

\begin{Def}{Linksableitung}
    Eine \begriff{Linksableitung} ist eine Ableitung, bei der immer die am
    weitesten links stehende Variable ersetzt wird.\\
    (Dies ergibt nur für kontextfreie Grammatiken Sinn, d.\,h. falls
    $P \subset V \times (V \cup \Sigma)^+$.)
\end{Def}

\begin{Def}{erzeugte Sprache}
    Die von einer Grammatik $G = (V, \Sigma, P, S)$ \begriff{erzeugte Sprache}
    ist\\
    $L(G) := \{w \in \Sigma^\ast \;|\; S \Rightarrow_G^\ast w\}$.
\end{Def}

\linie
\pagebreak

\begin{Bsp}
    $G = (\{S\}, \{a, b\}, P, S)$ mit $P = \{(S, ab), (S, aSb)\} =
    \{S \rightarrow ab \;|\; aSb\}$ (BNF) ist eine Grammatik mit
    $L(G) = \{a^n b^n \;|\; n \ge 1\}$
    (dies ist eine Kurzschreibweise).
\end{Bsp}

\begin{Bsp}
    Für ein beliebiges Alphabet $\Sigma$ erhält man mit
    $G = (\{S\}, \Sigma, P, S\}$ und\\
    $P = \{S \rightarrow a \;|\; a \in \Sigma\} \cup
    \{S \rightarrow \varepsilon,\; S \rightarrow SS\}$ eine Grammatik
    mit $L(G) = \Sigma^\ast$.\\
    $L(G) = \emptyset$ erhält man mit $P = \emptyset$.
\end{Bsp}

\begin{Bsp}
    Weitere (Anwendungs-)Beispiele für Grammatiken sind natürliche Sprachen
    (hier nicht),
    korrekte arithmetische Ausdrücke
    (z.\,B. $(a + a) \cdot a$, aber nicht $(((a((a(($ oder $(\cdot + a))($),
    Palindrome über dem Alphabet $\{a, b\}$
    (mittels $S \rightarrow aSa \;|\; bSb \;|\; a \;|\; b \;|\; \varepsilon$)
    und Wörter der Form $a^n b^n c^n$ (komplizierter, kontextsensitiv).
\end{Bsp}

\begin{Bem}
    Bei Maschinen ist Nichtdeterminismus meist nur von "`akademischem"'
    Belang, bei Grammatiken ist er jedoch essentiell
    (eine Satzform kann verschieden abgeleitet werden).\\
    Der einfache Pfeil $\rightarrow$ wird für Regeln,
    $\Rightarrow_G$ wird für Ableitungsschritte verwendet.\\
    $a^n$ ist eine Kurzform für $a \dotsb a$ ($n$-mal).\\
    Variablen werden mit Groß- und
    Terminalzeichen werden mit Kleinbuchstaben bezeichnet.
\end{Bem}

\section{%
    \name{Chomsky}-Hierarchie%
}

\begin{Def}{\name{Chomsky}-Hierarchie}
    \begin{enumerate}[label=\emph{Typ \arabic*}:,start=0,
                      leftmargin=20mm,labelsep=5mm]
        \item
        Jede Grammatik ist vom \begriff{Typ 0}.

        \item
        Eine Typ-0-Grammatik ist vom \begriff{Typ 1}
        (\begriff{kontextsensitiv}),\\
        falls für alle Regeln $w_1 \rightarrow w_2$ in $P$
        gilt, dass $|w_1| \le |w_2|$.

        \item
        Eine Typ-1-Grammatik ist vom \begriff{Typ 2}
        (\begriff{kontextfrei}),\\
        falls für alle Regeln $w_1 \rightarrow w_2$ in $P$
        gilt, dass $w_1 \in V$.

        \item
        Eine Typ-2-Grammatik ist vom \begriff{Typ 3}
        (\begriff{regulär}),\\
        falls für alle Regeln $w_1 \rightarrow w_2$ in $P$
        gilt, dass $w_2 \in \Sigma \cup \Sigma V$.
    \end{enumerate}
    Eine Sprache $L \subset \Sigma^\ast$ heißt
    vom Typ $i$, falls es eine Typ-$i$-Grammatik $G$ gibt mit $L(G) = L$
    ($i = 0, \dotsc, 3$).
\end{Def}

\begin{Bem}
    Die Ableitungen einer Typ-3-Grammatik haben die Form
    $S \Rightarrow a_1 A_1 \Rightarrow a_1 a_2 A_2$\\
    $\Rightarrow \dotsb \Rightarrow a_1 \dotsb a_n A_n$.
    Weiter geht es mit $a_1 \dotsb a_n a_{n+1} A_{n+1}$,
    terminiert wird mit $a_1 \dotsb a_n a_{n+1}$.\\
    Bei Typ-0-Grammatiken kann die Satzform beim Ableiten länger und wieder
    kürzer werden,
    bei Typ-1-Grammatiken ist dagegen die Länge der Satzform monoton
    steigend, daher heißen Typ-1-Grammatiken auch \begriff{nicht-verkürzend}.
    Das leere Wort $\varepsilon$ kann von diesen Grammatiken nie erzeugt werden
    (für alle $w_1 \rightarrow w_2$ in $P$ gilt, dass $1 \le |w_1| \le |w_2|$).
\end{Bem}

\linie
\pagebreak

\begin{Def}{$\varepsilon$-Sonderregel}
    Für Typ-1-Grammatiken kann $S \rightarrow \varepsilon$ zugelassen werden.\\
    In diesem Fall ist aber $S$ auf allen rechten Seiten verboten.
\end{Def}

\begin{Bem}
    $S$ ist auf allen rechten Seiten verboten, weil sonst die Definition
    der Typ-1-Gramma"-tik sinnlos werden würde:
    Man könnte z.\,B. eine Regel der Form $AB \rightarrow CSS$
    basteln, die wegen $S \rightarrow \varepsilon$ als $AB \rightarrow C$
    angewendet werden könnte.\\
    Damit obige Definition sinnvoll ist, sollte man zeigen,
    dass es zu jeder Typ-$i$-Grammatik\\
    $G = (V, \Sigma, P, S)$ eine Typ-$i$-Grammatik $G'$ mit
    $\varepsilon$-Sonderregel gibt, sodass
    $L' = L \cup \{\varepsilon\}$ mit $L' := L(G')$ und $L := L(G)$
    ($i = 1, 2, 3$), d.\,h. man kann jede Grammatik bei zugelassener
    $\varepsilon$-Sonderregel so verändern, dass ihre Sprache nur um
    das leere Wort ergänzt wird.

    Für Typ-1-Grammatiken verfährt man, indem man
    $G' = (V', \Sigma, P', S)$ setzt mit $V' = V \cup \{S'\}$
    und $S' \notin V$.
    $P'$ erhält man aus $P$, indem man in allen Regeln $S$ durch $S'$ ersetzt
    und die Regeln $S \rightarrow S'$, $S \rightarrow \varepsilon$
    hinzufügt.

    Für Typ-2-Grammatiken kann man genauso verfahren, da
    die hinzugefügten Regeln (insbesondere $S \rightarrow S'$)
    Typ-2-konform sind.

    Bei Typ-3-Grammatiken ergibt sich das Problem, dass $S \rightarrow S'$
    keine Typ-3-Regel ist.
    Man verfährt stattdessen folgendermaßen, um $P'$ aus $P$ zu erhalten:
    \begin{enumerate}
        \item
        ersetze in allen Regeln $S$ durch $S'$

        \item
        für jede Regel mit $S'$ auf der linken Seite füge dieselbe Regel
        mit $S$ auf der linken Seite hinzu

        \item
        füge $S \rightarrow \varepsilon$ hinzu
    \end{enumerate}
    Es ergibt sich eine Grammatik $G'$ gleichen Typs
    (bis auf $\varepsilon$-Sonderregel) mit
    $L(G') = L(G) \cup \{\varepsilon\}$.
\end{Bem}

\linie

\begin{Bem}
    Eine Typ-1-Sprache, die durch eine Typ-1-Grammatik
    (die keine Typ-3-Grammatik ist) erzeugt wird,
    kann auch vom Typ 3 sein.
\end{Bem}

\begin{Bem}
    Die Chomsky-Hierarchie hat ihren Namen von der Anordnung der
    Sprachenmengen.
    In der Menge aller Sprachen bilden die Typ-0-Sprachen eine echte Teilmenge,
    die die entscheidbaren Sprachen wieder als echte Teilmenge beinhalten.
    Diese enthalten in echter Inklusion die Typ-1-Sprachen
    (kontextsensitive Sprachen),
    die in echter Inklusion die Typ-2-Sprachen beinhalten
    (kontextfreie Sprachen),
    die in echter Inklusion die Typ-3-Sprachen beinhalten
    (reguläre Sprachen).
\end{Bem}

\pagebreak

\section{%
    Wortproblem%
}

\begin{Def}{Wortproblem}
    Gegeben seien eine Grammatik $G = (V, \Sigma, P, S)$ und ein Wort
    $w \in \Sigma^\ast$.
    Gesucht ist ein Algorithmus, der beliebige $G$ und $w$ als Eingabe hat
    sowie $1$ ausgibt, falls $w \in L(G)$, und
    $0$, falls $w \notin L(G)$
    (d.\,h. der Algorithmus implementiert die charakteristische Funktion
    von $L(G)$ auf $\Sigma^\ast$).
    Dieses Problem heißt \begriff{Wortproblem}.
\end{Def}

\begin{Satz}{Wortproblem entscheidbar für Typ-1-Sprachen}
    Das Wortproblem ist für Sprachen vom Typ 1 entscheidbar,
    d.\,h. der gesuchte Algorithmus existiert
    (falls die Sprache durch eine Typ-1-Grammatik gegeben ist).
\end{Satz}

\begin{Bem}
    Sprachen vom Typ 2 und Typ 3 sind auch vom Typ 1, d.\,h.
    das Wortproblem ist auch hier entscheidbar.
    Für Typ-0-Sprachen ist das Wortproblem i.\,A. nicht entscheidbar.
\end{Bem}

\begin{Beweis}
    Als Beweis wird ein Algorithmus angegeben.\\
    Seien also $G = (V, \Sigma, P, S)$ eine Typ-1-Grammatik und
    $x \in \Sigma^\ast$ mit $n := |x| \ge 1$.
    \begin{align*}&
        T := \{S\}\\&
        \REPEAT T_1 := T;\; T := \Abl_n(T)\\&
        \UNTIL (x \in T) \;\OR (T = T_1);\\&
        \IF x \in T \;\THEN
        \text{output}(1) \;\ELSE
        \text{output}(0);
    \end{align*}
    Für eine Menge $X \subset (V \cup \Sigma)^\ast$ ist dabei
    $\Abl_n(X)$ definiert als\\
    $\Abl_n(X) := X \cup \{w \in (V \cup \Sigma)^\ast \;|\;
    (|w| \le n) \land (\exists_{w' \in X}\; w' \Rightarrow_G w)\}$.

    Der Algorithmus terminiert stets
    (d.\,h. er bricht für jede Eingabe nach einer
    endlichen Anzahl von Schritten ab), denn:
    Aufgrund $X \subset \Abl_n(X)$ gilt am Ende jedes Schleifendurchlaufs
    entweder $T_1 = \Abl_n(T_1)$ (dann wird terminiert) oder es gilt
    $T_1 \subsetneqq \Abl_n(T_1)$.\\
    Der letzte Fall ist jedoch nur endlich oft möglich, da
    dabei $|T_1| < |\Abl_n(T_1)|$ gilt, aber
    $|\Abl_n(T_1)|$ nach oben durch $\sum_{i=1}^n t^i$ mit
    $t = |V \cup \Sigma|$ beschränkt ist.
    Somit terminiert der Algorithmus nach endlich vielen Schritten.

    Der Algorithmus ist korrekt
    (d.\,h. er gibt $1$ aus genau dann, wenn $x \in L(G)$), denn:\\
    Ist $x \in T$ im $r$-ten Schritt
    (das soll bedeuten, dass $x \notin T$ für vorherige Schritte gilt),
    so gibt es ein $w_1 \in (V \cup \Sigma)^+$ mit $w_1 \in T$
    im $(r - 1)$-ten Schritt und $w_1 \Rightarrow_G x$.
    Daraus folgt, dass es ein $w_2 \in (V \cup \Sigma)^+$ gibt mit $w_2 \in T$
    im $(r - 2)$-ten Schritt und $w_2 \Rightarrow_G w_1$ usw.\\
    Induktiv gibt es also ein $w_{r-1} \in (V \cup \Sigma)^+$ mit
    $w_{r-1} \in T$ im ersten Schritt und $w_{r-1} \Rightarrow_G w_{r-2}$.
    Daraus folgt wieder, dass es ein $w_r \in (V \cup \Sigma)^+$ mit
    $w_r \in T$ im nullten Schritt und $w_r \Rightarrow_G w_{r-1}$.
    Im nullten Schritt ist allerdings $T = \{S\}$, d.\,h. es gilt $w_r = S$.\\
    Insgesamt gilt also $S = w_r \Rightarrow_G w_{r-1} \Rightarrow_G \dotsb
    \Rightarrow_G w_2 \Rightarrow_G w_1 \Rightarrow_G x$, also
    $x \in L(G)$.
    Wenn der Algorithmus $1$ ausgibt, dann ist daher $x \in L(G)$.
    Die andere Richtung überlegt man sich analog, es gilt also
    $x \in T$ im $r$-ten Schritt genau dann, wenn $S \Rightarrow_G^\ast x$
    in $r$ Schritten.
\end{Beweis}

\pagebreak

\section{%
    Syntaxbäume%
}

\begin{Def}{Syntaxbaum}
    Sei eine Typ-2-Grammatik $G = (V, \Sigma, P, S)$ gegeben.
    Jeder Ableitung eines Wortes $x \in L(G)$ kann man einen
    \begriff{Syntaxbaum} oder \begriff{Ableitungsbaum} zuordnen:\\
    Sei dazu $S = x_0 \Rightarrow_G x_1 \Rightarrow_G \dotsb
    \Rightarrow_G x_n = x$ eine Ableitung des Wortes $x \in L(G)$.
    Man ordnet der Wurzel des (zu konstruierenden) Syntaxbaums die
    Startvariable $S$ zu.
    Für $i = 1, \dotsc, n$ führe man folgendes durch:
    Falls im $i$-ten Ableitungsschritt $x_{i-1} \Rightarrow_G x_i$
    die Variable $A$ mit der Regel $A \rightarrow z \in P$ durch ein Wort $z$
    ersetzt wird, erstelle im Syntaxbaum $|z|$ Söhne von $A$ und beschrifte
    diese mit den einzelnen Zeichen von $z$.
    Auf diese Weise entsteht ein Baum, dess Blätter gerade mit den Zeichen in
    $x$ beschriftet sind.
\end{Def}

\begin{Bem}
    Syntaxbäume für Typ-3-Grammatiken sind immer entartet, d.\,h.
    jeder Knoten hat höchstens zwei Söhne (davon immer ein Terminalzeichen und
    evtl. eine Variable).\\
    Syntaxbäume für Grammatiken, die nicht vom Typ 2 sind, sind nicht sinnvoll
    definiert.
\end{Bem}

\begin{Bem}
    Verschiedenen Ableitungen eines Wortes $x \in L(G)$ kann derselbe
    Syntaxbaum zugeordnet sein
    (beispielsweise indem man die Ableitungsreihenfolge variiert).
    Ersetzt man immer die erste vorkommende (am weitesten links stehende)
    Variable, so spricht man von einer \begriff{Linksableitung}.
    Weil man jedem Syntaxbaum eindeutig eine Linksableitung zuordnen kann,
    gibt es für jedes Wort $x \in L(G)$ eine Linksableitung.
\end{Bem}

\begin{Bem}
    Verschiedenen Syntaxbäumen eines Wortes können verschiedene Bedeutungen
    zugewiesen werden.
    Man denke dabei an die Sprache der arithmetischen Ausdrücke, in der es
    einen Sinn ergeben würde, implizit Klammern um den zuletzt abgeleiteten
    Term zu setzen.
    Man würde also den zuletzt abgeleiteten Term zuerst ausrechnen.
    Solche Interpetationen sind z.\,B. im Compilerbau sinnvoll.
    Hier wird dieser Aspekt nicht weiter verfolgt.
\end{Bem}

\linie

\begin{Def}{eindeutig/mehrdeutig}
    Eine Typ-2-Grammatik $G$ heißt \begriff{mehrdeutig}, falls es ein Wort\\
    $x \in L(G)$ gibt, dass mindestens zwei Ableitungen besitzt, deren
    Syntaxbäume verschieden sind.
    Sonst heißt die Grammatik \begriff{eindeutig}
    (jedes Wort $x \in L(G)$ besitzt genau einen Syntaxbaum).
\end{Def}

\begin{Bem}
    Es kann sein, dass es für eine Sprache mehrere die Sprache erzeugende
    Grammatiken gibt, von denen eine mehrdeutig und eine eindeutig ist.
\end{Bem}

\begin{Def}{inhärent mehrdeutig}
    Eine Typ-2-Sprache $L$ heißt \begriff{inhärent mehrdeutig}, falls
    jede Typ-2-Grammatik $G$ mit $L(G) = L$ mehrdeutig ist.
\end{Def}

\begin{Bem}
    Im Allgemeinen ist es algorithmisch unmöglich festzustellen, ob eine
    Typ-2-Gram\-matik mehrdeutig
    (oder ob eine Typ-2-Sprache inhärent mehrdeutig) ist oder nicht.
\end{Bem}

\section{%
    \name{Backus}-\name{Naur}-Form%
}

\begin{Def}{\name{Backus}-\name{Naur}-Form (BNF)}
    Backus und Naur führten einen Formalismus zum kompakten Aufschreiben
    von Typ-2-Grammatiken ein
    (\begriff{\name{Backus}-\name{Naur}-Form (BNF)}).\\
    In Grammatiken $G$ bedeutet die Regel
    $A \rightarrow \beta_1 \;|\; \beta_2 \;|\; \dotsb \;|\; \beta_n$
    den Satz von Regeln\\
    $A \rightarrow \beta_1$,
    $A \rightarrow \beta_2$, \dots,
    $A \rightarrow \beta_n$.
    (Man kann statt $\rightarrow$ auch $::=$ schreiben.)
\end{Def}

\begin{Def}{erweiterte \name{Backus}-\name{Naur}-Form (EBNF)}
    Verwendet man zusätzliche Notationen, so spricht man von der
    \begriff{erweiterten \name{Backus}-\name{Naur}-Form (EBNF)}.\\
    In Grammatiken $G$ steht die Regel $A \rightarrow \alpha [\beta] \gamma$
    für die Regeln $A \rightarrow \alpha \gamma$,
    $A \rightarrow \alpha \beta \gamma$.\\
    Die Regel $A \rightarrow \alpha \{\beta\} \gamma$ steht für die Regeln
    $A \rightarrow \alpha \gamma$, $A \rightarrow \alpha B \gamma$,
    $B \rightarrow \beta$, $B \rightarrow \beta B$.
\end{Def}

\begin{Bem}
    Da (E)BNF und kontextfreie Grammatiken gleichwertig sind, können durch\\
    (E)BNF genau die kontextfreien Sprachen dargestellt werden.
\end{Bem}

\pagebreak
