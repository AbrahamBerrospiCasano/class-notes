\chapter{%
    Kontextsensitive und Typ-0-Sprachen%
}

\section{%
    Normalform%
}

\begin{Def}{\name{Kuroda}-Normalform}
    Eine Typ-1-Grammatik heißt in \begriff{\name{Kuroda}-Normalform}, falls
    alle Regeln von der Form $A \rightarrow a$, $A \rightarrow B$,
    $A \rightarrow BC$ oder $AB \rightarrow CD$ sind
    ($A, B, C, D$ Variablen und $a$ Terminal).
\end{Def}

\begin{Satz}{\name{Kuroda}-Normalform}
    Zu jeder Typ-1-Grammatik $G$ mit $\varepsilon \notin L(G)$ gibt
    es eine Typ-1-Grammatik $G'$ in Kuroda-Normalform mit
    $L(G) = L(G')$.
\end{Satz}

\begin{Beweis}
    Zuerst führt man für jedes Terminalsymbol $a$ eine Variable $A$
    (Pseudoterminal) mit der Regel $A \rightarrow a$ ein.
    Alle $a$'s in den alten Regeln werden durch $A$ ersetzt.\\
    Nun gibt es nur Regeln der Form $A \rightarrow a$ (die sind okay) und
    $A_1 \dotsb A_m \rightarrow B_1 \dotsb B_n$ mit $1 \le m \le n$.

    Für $m = 1$ kann man wie bei der Chomsky-Normalform
    $A_1 \rightarrow B_1 \dotsb B_n$ durch mehrere Ableitungsregeln ersetzen:
    Man führt eine Variable $C_1$ ein und ersetzt die alte Regel durch
    $A_1 \rightarrow B_1 C_2$ und $C_2 \rightarrow B_2 \dotsb B_n$.
    Induktiv verfährt man genauso, bis die rechte Seite Länge $\le 2$ hat.

    Für $m \ge 2$ gilt $2 \le m \le n$, somit kann man
    $A_1 \dotsb A_m \rightarrow B_1 \dotsb B_n$ ersetzen durch die Regeln
    $A_1 A_2 \rightarrow B_1 C_2$ und
    $C_2 A_3 \dotsb A_m \rightarrow B_2 \dotsb B_n$, wobei
    $C_2$ eine neue Variable ist.
    Induktiv wiederholt man dies, bis die linke Seite Länge $2$ hat.
    Falls die entstehende letzte Regel eine rechte Seite der Länge $> 2$ hat
    kann man wie oben verfahren, andernfalls sind nun alle Regeln in
    Kuroda-Normalform.

    Man erhält also eine Typ-1-Grammatik $G'$ in Kuroda-Normalform mit
    $L(G) = L(G')$.
\end{Beweis}

\section{%
    \name{Turing}maschinen%
}

\begin{Bem}
    Eine Turingmaschine besteht bildlich gesprochen aus einem potentiell
    unendlichen Arbeitsband mit Schreib-/Lesekopf und einer endlichen
    Zustandskontrolle.
    Auf dem Arbeitsband befinden sich Symbole aus einem Bandalphabet
    (einer Obermenge des Eingabealphabets).
    Es können immer nur endlich viele Symbole des Arbeitsbandes beschrieben
    sein.
    Die anderen Symbole enthalten ein Leerzeichen $\Box$.\\
    In einem Schritt führt die Turingmaschine Folgendes durch:
    In Kenntnis von dem aktuellen Zustand $z$ und dem Zeichen $a$, an dem sich
    der Schreib-/Lesekopf gerade befindet, wird ein neuer Zustand $z'$
    angenommen, $a$ wird mit $a'$ überschrieben und der Kopf bewegt sich um
    ein Feld nach links, rechts oder verharrt in seiner Position.
\end{Bem}

\linie
\pagebreak

\begin{Def}{deterministische/nicht-deterministische
            \name{Turing}maschine (DTM/TM)}\\
    Eine \begriff{deterministische/nicht-deterministische
    \name{Turing}maschine (DTM/NTM oder TM)}
    ist ein $7$-Tupel $M = (Z, \Sigma, \Gamma, \delta, z_0, \Box, E)$, wobei
    \begin{itemize}
        \item
        $Z$ eine endliche, nicht-leere Menge
        (die Menge der \begriff{Zustände}),

        \item
        $\Sigma$ eine endliche, nicht-leere Menge
        mit $Z \cap \Sigma = \emptyset$
        (das \begriff{Eingabealphabet}),

        \item
        $\Gamma \supset \Sigma$ eine endliche, nicht-leere Menge
        mit $Z \cap \Gamma = \emptyset$
        (das \begriff{Bandalphabet}),

        \item
        $\delta\colon Z \times \Gamma \rightarrow
        Z \times \Gamma \times \{L, R, N\}$ für eine DTM und\\
        $\delta\colon Z \times \Gamma \rightarrow
        \P(Z \times \Gamma \times \{L, R, N\})$ für eine NTM
        (die \begriff{Überführungsfunktion}),

        \item
        $z_0 \in Z$ (der \begriff{Startzustand}),

        \item
        $\Box \in \Gamma \setminus \Sigma$ (das \begriff{Leerzeichen}) und

        \item
        $E \subset Z$ (die \begriff{akzeptierenden Endzustände}) ist.
    \end{itemize}
\end{Def}

\linie

\begin{Bem}
    Ein Übergang $(z', a', X) = \delta(z, a)$ bzw.
    $(z', a', X) \in \delta(z, a)$ bedeutet Folgendes:\\
    Wechsle vom Zustand $z$ in den Zustand $z'$ und schreibe das Bandsymbol
    $a'$ an die Stelle von $a$.
    Bewege anschließend den Schreib-/Lesekopf nach links ($X = L$),
    rechts ($X = R$) bzw. lasse ihn in seiner Position ($X = N$).\\
    Um eine Konfiguration (d.\,h. die aktuelle Situation der TM) darzustellen,
    müssen Bandinhalt (nur der bisher beschriebene Teil),
    Kopfposition und aktueller Zustand bekannt sein.
    Dies kann man als Elemente von $\Gamma^\ast Z \Gamma^\ast$ darstellen,
    d.\,h. für $\alpha z \beta \in \Gamma^\ast Z \Gamma^\ast$ mit $z \in Z$
    steht $\alpha$ links vor dem Schreib-/Lesekopf auf dem Band,
    der Kopf selbst zeigt auf den ersten Buchstaben $b_1$ von
    $\beta = b_1 b_2 \dotsb b_n$ ($b_i \in \Gamma$) und
    $b_2 \dotsb b_n$ stehen rechts nach dem Schreib-/Lesekopf.
\end{Bem}

\begin{Def}{Konfiguration}\\
    Eine Konfiguration $k$ der TM
    $M = (Z, \Sigma, \Gamma, \delta, z_0, \Box, E)$
    ist ein Element $k \in \Gamma^\ast Z \Gamma^\ast$.
\end{Def}

\begin{Def}{Übergangsrelation}
    Auf der Menge $\Gamma^\ast Z \Gamma^\ast$ wird eine
    Relation $\vdash$ definiert durch\\
    $a_1 \dotsb a_m z b_1 \dotsb b_n \vdash
    a_1 \dotsb a_{m-1} z' a_m c b_2 \dotsb b_n$ für
    $(z', c, L) = \delta(z, b_1)$ und $m, n \ge 1$,\\
    $a_1 \dotsb a_m z b_1 \dotsb b_n \vdash
    a_1 \dotsb a_m c z' b_2 \dotsb b_n$ für
    $(z', c, R) = \delta(z, b_1)$ und $m \ge 0$, $n \ge 2$ sowie\\
    $a_1 \dotsb a_m z b_1 \dotsb b_n \vdash
    a_1 \dotsb a_m z' c b_2 \dotsb b_n$ für
    $(z', c, N) = \delta(z, b_1)$ und $m \ge 0$, $n \ge 1$\\
    (\begriff{Übergangsrelation}).
    Außerdem ist
    $z b_1 \dotsb b_n \vdash
    z' \Box c b_2 \dotsb b_n$ für
    $(z', c, L) = \delta(z, b_1)$ und\\
    $a_1 \dotsb a_m z b_1 \vdash
    a_1 \dotsb a_m c z' \Box$ für
    $(z', c, R) = \delta(z, b_1)$.\\
    Im nicht-deterministischen Fall ersetzt man die "`$=$"'
    vor den $\delta$'s durch "`$\in$"'.\\
    $\vdash^\ast$ ist der ref"|lexive und transitive Abschluss von $\vdash$.
\end{Def}

\begin{Def}{akzeptierte Sprache}
    Die von einer TM $M = (Z, \Sigma, \Gamma, \delta, z_0, \Box, E)$
    \begriff{akzeptierte Sprache} ist
    $T(M) := \{x \in \Sigma^\ast \;|\;
    \exists_{z \in E} \exists_{\alpha, \beta \in \Gamma^\ast}\;
    z_0 x \vdash^\ast \alpha z \beta\}$.
\end{Def}

\begin{Bem}
    Falls $x \notin T(M)$ gilt, so ergibt sich entweder eine unendlich lange
    Berechnung (es wiederholen sich immer die gleichen Zustände oder immer
    mehr Speicher wird verwendet) oder die Maschine terminiert, ohne in einem
    Endzustand zu sein.
\end{Bem}

\linie

\begin{Bsp}
    Ein Beispiel für eine Turingmaschine, die die Eingabe
    $x = x_1 \dotsb x_n \in \{0, 1\}^\ast$ bitweise invertiert und den Kopf
    am Ende wieder an den linken Rand stellt, ist\\
    $M = (\{z_0, z_1, z_2\}, \{0, 1\}, \{0, 1, \Box\}, \delta, z_0, \Box,
    \{z_2\})$ mit $\delta$ gegeben durch\\
    $\delta(z_0, 0) = (z_0, 1, R)$,
    $\delta(z_0, 1) = (z_0, 0, R)$,
    $\delta(z_0, \Box) = (z_1, \Box, L)$,\\
    $\delta(z_1, 0) = (z_1, 0, L)$,
    $\delta(z_1, 1) = (z_1, 1, L)$ und
    $\delta(z_1, \Box) = (z_2, \Box, R)$.
\end{Bsp}

\pagebreak

\section{%
    Linear beschränkte \name{Turing}maschinen%
}

\begin{Bem}
    Eine linear beschränkte Turingmaschine soll genau den durch das Eingabewort
    vorbelegten Speicher benutzen (und nicht mehr), d.\,h.
    für $z_0 x \vdash^\ast \alpha z \beta$ soll immer $|\alpha \beta| = |x|$
    gelten.
    Man spricht von einer linearen Beschränkung, denn in jedem Bandbuchstaben
    können höchstens endlich viele Informationen gespeichert werden
    (z.\,B. durch $n$-Tupel), daraus ergibt sich ein Speicherplatz von
    $n \cdot |x|$.
    Damit die Turingmaschine den letzte Buchstaben erkennt, ohne über den
    Rand zu springen, wird das Bandalphabet verdoppelt und der letzte Buchstabe
    besonders markiert.
\end{Bem}

\begin{Def}{linear beschränkte \name{Turing}maschine (LBA)}
    Eine \begriff{linear beschränkte \name{Turing}maschine} oder
    \begriff{linear bounded automaton (LBA)} ist eine Turingmaschine
    $M = (Z, \Sigma', \Gamma, \delta, z_0, \Box, E)$ mit
    $\Sigma' := \Sigma \cup \{\widehat{a} \;|\; a \in \Sigma\}$,
    sodass für alle $a_1 \dotsb a_{n-1} a_n \in \Sigma^+$,
    $\alpha, \beta \in \Gamma^\ast$ und $z \in Z$ mit\\
    $z_0 a_1 \dotsb a_{n-1} \widehat{a}_{n} \vdash^\ast \alpha z \beta$ gilt,
    dass $|\alpha \beta| = n$.\\
    Die von einem LBA erkannte Sprache ist\\
    $T(M) := \{a_1 \dotsb a_{n-1} a_n \in \Sigma^+ \;|\;
    \exists_{z \in E} \exists_{\alpha, \beta \in \Gamma^\ast}\;
    z_0 a_1 \dotsb a_{n-1} \widehat{a}_n \vdash^\ast \alpha z \beta\}$.
\end{Def}

\linie

\begin{Satz}{Satz von \upshape\,\!\name{Kuroda}}
    Die Klasse der von LBAs akzeptierten Sprachen ist gleich
    der Klasse der Typ-1-Sprachen.
    (Dabei ist $\varepsilon$ in den Sprachen nicht enthalten.)
\end{Satz}

\begin{Beweis}
    Sei $L$ eine Typ-1-Sprache, d.\,h. $L = L(G)$ für eine Grammatik
    $G = (V, \Sigma, P, S)$ mit nicht-verkürzenden Regeln.
    Ein LBA, der $L$ erkennt, wird folgendermaßen konstruiert:\\
    Sei die Eingabe $x = x_1 \dotsb x_n$, zu prüfen ist, ob
    $S \Rightarrow^\ast x$ in $G$.
    \begin{enumerate}
        \item
        Wähle (nicht-deterministisch) eine Regel
        $\alpha \rightarrow \beta \in P$.

        \item
        Wähle eine (nicht-deterministisch) Position auf dem Band.

        \item
        Prüfe, ob ab der aktuellen Position $\beta$ auf dem Band steht.

        \item
        Wenn ja: Ersetze $\beta$ durch $\alpha$,
        gegebenenfalls muss rechts von $\beta$ ein Bandteil nach links
        verschoben werden (dies ist durch LBAs möglich).

        \item
        Wenn nur noch $S$ dasteht, akzeptiere, ansonst wiederhole mit 1.
    \end{enumerate}
    Der Vorgang wird so lange wiederholt, bis entweder nur $S$ auf dem
    Band steht oder kein Übergang mehr möglich ist
    (und zwar bei allen endlich vielen, durch den
    Nicht-Deterministismus entstehenden Möglichkeiten).
    Für den konstruierten LBA $M$ gilt $L(G) = T(M)$.

    Sei nun $L = T(M)$ für einen LBA $M$.
    Gesucht ist eine Typ-1-Grammatik $G$ mit $L(G) = T(M)$.
    Die Grammatik soll den Automaten "`simulieren"', indem die
    Konfigurationsübergänge durch Ableitungen modelliert werden.
    Da $\alpha z \beta$ eine größere Länge als $|\alpha \beta|$ hat und
    $G$ nur nicht-verkürzende Regeln besitzen darf,
    schreibt man statt $\alpha z \beta$ einfach
    $\alpha (z, \beta_1) \beta_2 \dotsb \beta_r$ für
    $\beta = \beta_1 \dotsb \beta_r$.
    Man erhält also ein neues Alphabet
    $\Delta := \Gamma \cup (Z \times \Gamma)$.\\
    Zum Beispiel soll sich $(z', a', L) \in \delta(z, \beta_1)$
    auf $\alpha_1 \dotsb \alpha_s (z, \beta_1) \beta_2 \dotsb \beta_r$
    auswirken durch\\
    $\alpha_1 \dotsb \alpha_{s-1} (z', \alpha_s) a' \beta_2 \dotsb \beta_r$.
    Man erhält also die Regel
    $\alpha (z, \beta) \rightarrow (z', \alpha) \gamma$
    für $(z', \gamma, L) \in \delta(z, \beta)$.

    So konstruiert man die Grammatik $G = (V, \Sigma, P, S)$ mit
    $V = \{S, A\} \cup (\Delta \times \Sigma)$.
    Die Variablen sind speziell gewählt:
    Eine Satzform hat immer zwei "`Spuren"':
    Auf der einen (sagen wir oberen) Spur steht der aktuelle Bandinhalt des
    LBA mit aktueller Kopfposition, wobei das letzte Zeichen immer markiert
    ist.
    Auf der unteren Spur steht das Wort, das erkannt werden soll.
    Die untere Spur bleibt dabei stets unverändert.
    Daher definiert man die Regeln $S \rightarrow A (\widehat{a}, a)$,
    $A \rightarrow A (a, a)$ und
    $A \rightarrow ((z_0, a), a)$ für alle $a \in \Sigma$
    (nicht-deterministisches Erzeugen aller
    möglichen Wörter aus $\Sigma^\ast$),
    Regeln wie eben im Beispiel,
    $(a, b) \rightarrow b$ für $a \in \Gamma$, $b \in \Sigma$ und
    $((z, a), b) \rightarrow b$ für $a \in \Gamma$, $b \in \Sigma$ und
    $z \in E$
    (akzeptieren durch Hinschreiben der unteren Spur, falls im Endzustand).\\
    Alle Regeln sind nicht-verkürzend und es gilt $T(M) = L(G)$.
\end{Beweis}

\linie
\pagebreak

\begin{Bem}
    Der Satz kann genauso bewiesen werden, wenn die Regeln verkürzend sein
    dürfen und das Band der Turingmaschine unbeschränkt benutzt werden darf.
    Daraus ergibt sich folgender Satz.
\end{Bem}

\begin{Satz}{$\NTM = \TypNull$}
    Die Klasse $\NTM$ der von Turingmaschinen akzeptierten Sprachen
    ist gleich der Klasse $\TypNull$ der Typ-0-Sprachen.
\end{Satz}

\begin{Bem}
    Die Simulation einer nicht-deterministischen Turingmaschine durch
    eine deterministische Turingmaschine ist möglich
    (man wende Breitensuche auf den Berechnungsbaum des Wortes an).
    Dabei ergibt sich jedoch eine exponentiell erhöhte Laufzeit.\\
    Daher ist $\DTM = \NTM = \TypNull$.
    Ob das auch für den Fall linear beschränkter Turingmaschinen gilt
    ($\DLBA = \LBA$?) ist ein of"|fenes Problem, das sogenannte
    \begriff{LBA-Problem}.
\end{Bem}

\section{%
    Der Satz von \name{Immerman} und \name{Szelepcsényi}%
}

\begin{Bem}
    Bis vor Kurzem bestand neben dem heutigen LBA-Problem ein zweites
    LBA-Problem, nämlich ob $\LBA = \coLBA$ gilt.
    Dieses Problem wurde vor ungefähr zwanzig Jahren von
    Immerman und Szelepcsényi unabhängig voneinander gelöst.\\
    Der Beweis wird hier nicht dargestellt.
\end{Bem}

\begin{Satz}{Satz von \name{Immerman} und \name{Szelepcsényi}}\\
    Die Klasse der Typ-1-Sprachen ist abgeschlossen gegen Komplement.
\end{Satz}

\pagebreak
