\section{%
    Lineare Programmierung%
}

\subsection{%
    Standardform%
}

\textbf{Diätproblem}:
Eine Motivation der linearen Programmierung ist das Diätproblem
(siehe "`Algorithmische Geometrie"').

MaxFlow-, MinCostFlow- und Kürzester-Pfad-Probleme können als lineare Programme modelliert werden
(z.\,B. für MaxFlow:
Variablen $x_e$ für jede Kante $e$,
maximiere $\sum_{e=(s,\cdot)} x_e$,
NBen $\forall_{e \in E}\; 0 \le x_e \le \Cap(e)$ und
$\forall_{v \in V}\; \sum_{e = (\cdot,v) \in E} x_e = \sum_{e = (v,\cdot) \in E} x_e$).

\linie

\textbf{Standardform eines linearen Programms}:
Seien $A \in \real^{n \times d}$, $b \in \real^n$ und $c \in \real^d$.\\
Gesucht ist $x \in \real^d$ mit $\max c^\tp x$, wobei $Ax \le b$.

Es kann auch vorkommen, dass die \begriff{Zielfunktion} $c^\tp x$ minimiert werden soll
(statt maximiert).
Außerdem kann es \begriff{Nebenbedingungen} geben mit $(Ax)_i \ge b_i$ oder $(Ax)_i = b_i$.
Weil aber diese Varianten leicht in obige Standardform überführt werden können,
wird im Folgenden meist nur die Standardform betrachtet.

\textbf{Zielbereich}:
$\{x \in \real^d \;|\; Ax \le b\}$ heißt \begriff{Zielbereich},
er enthält die \begriff{zulässigen Lösungen}.
Jede Nebenbedingung definiert einen Halbraum $(Ax)_i \le b_i$, der durch die Hyperebene
$(Ax)_i = b_i$ begrenzt wird.
Der Zielbereich ist stets konvex (aber nicht notwendigerweise nicht-leer oder beschränkt).
Genauer ist der Zielbereich ein Polyeder, dessen Ecken jeweils durch den Schnitt von
$d$ Hyperebenen, die zu NBen gehören, festgelegt werden.

\textbf{geometrische Interpretation}:
In zwei Dimensionen teilt jede Nebenbedingung die Ebene in zwei Hälften.
Der Schnitt aller Halbebenen ist der Zielbereich, ein Polyeder.
Durch Setzen von $c_1 x_1 + c_2 x_2 = z$ für $z \in \real$ verschiebt man die Gerade
$c_1 x_1 + c_2 x_2 = 0$ solange nach oben, bis sie zwar noch einen Schnitt hat
(i.\,A. einen Eckpunkt des Polyeders), danach aber nicht mehr.
Der letzte Schnitt enthält dann die optimalen Lösungen.

\linie

\textbf{Lemma}:
Wenn der Zielbereich nicht-leer und beschränkt ist,
dann gibt es eine Ecke des Polyeders, die eine optimale Lösung ist.

\begin{Beweis}
    Die Behauptung folgt trivialerweise aus der Konvexität des Polyeders und der
    Linearität der Zielfunktion.
\end{Beweis}

\subsection{%
    Simplex-Algorithmus%
}

\textbf{Idee in 2D}:
OBdA wird der unterste Punkt gesucht.
\begin{enumerate}
    \item
    Starte mit einer beliebigen Ecke des Zielbereichs.

    \item
    Wechsle zu einer benachbarten Ecke, wenn sie weiter unten ist.

    \item
    Wiederhole, bis alle benachbarten Ecken nicht weiter unten als die aktuelle Ecke sind.
\end{enumerate}

\linie

\textbf{in $d$ Dimensionen}:
Jede Ecke des Zielbereichs ist durch den Schnitt von $d$ Hyperebenen,
die die zu den NBen gehörigen Halbräume begrenzen, eindeutig festgelegt.
Eine Menge von $d$ NBen, die eine Ecke des Zielbereichs festlegen, heißt \begriff{Basis}.
Wenn man zu einer benachbarten Ecke wechselt, kann das als Basiswechsel so interpretiert werden,
dass eine NB die Basis für eine andere NB verlässt.
Wie im 2D-Fall wird solange gewechselt, bis die Ecke optimal ist.
Weil es höchstens $\binom{n}{d}$ Ecken gibt und keine Ecke mehrfach besucht wird,
terminiert der Algorithmus und arbeitet korrekt.

\linie
\pagebreak

\textbf{Fragen}:
\begin{enumerate}
    \item
    Wie findet man eine Anfangsecke?

    \item
    Wie berechnet man die Eckpunktkoordinaten, wenn eine Basis von $d$ NBen gegeben ist?

    \item
    Gegeben sei ein Eckpunkt als Basis.
    Wie findet man eine benachbarte, bessere, zulässige Ecke (falls existent)?
\end{enumerate}

\linie

\textbf{1. Frage}:
Übung

\linie

\textbf{2. Frage}:
Gegeben seien $d$ NBen
$(a_{i_k,\cdot}) x \le b_{i_k}$ für $k = 1, \dotsc, d$.
Dann berechnet sich die Ecke, die durch die NBen festgelegt ist
(falls sie in allg. Position sind), durch die Lösung des LGS
$(a_{i_k,\cdot}) x = (b_{i_k})_{k=1}^d$.

\linie

\textbf{3. Frage (Pivot-Schritt)}:
Sei $v$ eine zulässige Ecke des Zielbereichs, die durch $d$ NBen
$(a_{i_k,\cdot}) x \le b_{i_k}$ für $k = 1, \dotsc, d$ gegeben ist.
Dann gibt es $d$ Halbgeraden/Strahlen, die von $v$ aus auf den Randkanten des Zielbereichs laufen.
Gesucht ist ein Strahl, der den Zielfunktionswert verkleinert, wenn man ihm folgt
(entspricht einer NB, die man aus der Basis streicht).

Durch Umschreiben der NBen und Einführung von Schlupfvariablen $s_j$, $j = 1, \dotsc, d$,
erhält man $A_B x + s = b_B$ mit $A_B := (a_{i_k,j})_{k,j=1}^d$, $b_B := (b_{i_k})_{k=1}^d$ und
$s := (s_j)_{j=1}^d$ mit $s_j \ge 0$.
Die Koordinaten von $v$ berechnen sich durch $x = A_B^{-1} b_B - A_B^{-1} s$ für $s = 0$.
Wenn man nun sich von einer der Hyperebenen wegbewegt (auf dem Schnitt der restlichen Hyperebenen),
dann ist das nichts anderes als die Vergrößerung der entsprechenden Schlupfvariablen.

Für den Zielfunktionswert gilt $c^\tp x = c^\tp A_B^{-1} b_B + (-c^\tp A_B^{-1}) s$.
Dabei ist der erste Summand der Zielfunktionswert in $v$, während der zweite Summand eine
lineare Funktion in den $s_j$ ist.\\
Somit vergrößert das Wegbewegen von einer NB den Zielfunktionswert genau dann, wenn
der entsprechende Koef"|fizient in $-c^\tp A_B^{-1}$ positiv ist.
Sind alle Koef"|fizienten nicht-positiv, dann ist $v$ bereits optimal.

Es gibt verschiedene Strategien zur Wahl einer NB mit positivem Koef"|fizienten (siehe unten),
z.\,B. kann man die NB wählen, deren positiver Koeff. in $-c^\tp A_B^{-1}$ am größten ist.

\linie

Angenommen, eine der NBen wurde auserkoren, die Basis zu verlassen,
z.\,B. die $i$-te NB\\
$(a_{i,\cdot}) x \le b_i$.
Dann muss nun bestimmt werden, welche andere NB (von allen) die Bewegung auf dem Strahl zuerst
stoppt.

Sei $x' := A_B^{-1} b_B$ die aktuelle Ecke.
Dann ist $x'' := A_B^{-1} b_B - (A_B^{-1})_{\cdot,i}$ ein Punkt auf dem Strahl,
den wir betrachten (mit größerem Zielfunktionswert als $x'$).
Der Strahl ist daher gegeben durch
$r(\lambda) = x' + \lambda (x'' - x') = x' - \lambda (A_B^{-1})_{\cdot,i}$
für $\lambda \ge 0$.

Um den Schnittpunkt von $r(\lambda)$ mit einer anderen NB, z.\,B. der $\ell$-ten, zu
berechnen, setzt man $r(\lambda)$ in die NB ein und erhält
$(a_{\ell,\cdot}) (x' - \lambda (A_B^{-1})_{\cdot,i}) \le b_\ell
\iff (a_{\ell,\cdot}) x' - \lambda (a_{\ell,\cdot}) (A_B^{-1})_{\cdot,i} \le b_\ell$.
\begin{itemize}
    \item
    Ist nun $(a_{\ell,\cdot}) (A_B^{-1})_{\cdot,i} \ge 0$,
    dann wird diese NB den Strahl nie blockieren\\
    (es gilt in jedem Fall $(a_{\ell,\cdot}) x' \le b_\ell$,
    weil $x'$ vorher bereits zulässig war).

    \item
    Andernfalls gilt $(a_{\ell,\cdot}) (A_B^{-1})_{\cdot,i} < 0$ und erhält man die Schranke
    $\lambda \le \frac{(a_{\ell,\cdot}) x' - b_\ell}{(a_{\ell,\cdot}) (A_B^{-1})_{\cdot,i}}$,
    damit $r(\lambda)$ noch die $\ell$-te NB erfüllt.
\end{itemize}
Trifft der erste Fall auf jede NB zu, dann ist der Zielbereich in Zielfunktionsrichtung
unbeschränkt.
Ansonsten ist die $j$-te NB mit der kleinsten Schranke die NB, die als erstes vom Strahl
getrof"|fen wird.
Diese NB $j$ wird dann in der Basis gegen die NB $i$ getauscht.

\pagebreak

\subsection{%
    Pivot-Strategien%
}

\textbf{Pivot-Strategien}:
Insbesondere bei höherdimensionalen linearen Programmen gibt es oft mehrere Strahlen
ausgehend vom aktuellen Knoten, die die Zielfunktion vergrößern.
Es gibt viele Strategien, um zu entscheiden, welchem Strahl man folgt
(d.\,h. welche NB man aus der Basis "`wirft"'):
\begin{itemize}
    \item
    \begriff{Regel des steilsten Anstiegs}:
    Wähle den Strahl, der die Zielfunktion am schnellsten vergrößert
    (d.\,h. zum größten positiven Koeff. von $-c^\tp A_B^{-1}$ gehörig).

    \item
    \begriff{gieriger Ansatz}:
    Wähle den Strahl, bei dem der neue Eckpunkt den größten Zielfunktionswert besitzt
    (anders als steilster Anstieg).

    \item
    \begriff{randomisiert}:
    Wähle einen zufälligen Strahl, der den Zielfunktionswert vergrößert.
\end{itemize}
Für die meisten deterministischen Pivot-Strategien wurden Gegenbeispiele gefunden,
die dazu führen, dass der Simplex-Algorithmus eine exponentielle Zahl an Schritten durchführen
muss.
In der Praxis treten diese Extremfälle jedoch so gut wie nicht auf,
dort sind die Strategien sogar ziemlich gut.

\linie

\textbf{Basiszykel und \name{Bland}s Regel}:
Liegen die NBen nicht in allgemeiner Lage, d.\,h. haben bestimmte $> d$ der NBen einen nicht-leeren
Schnitt, dann können Strategien wie die Regel des steilsten Anstiegs sogar zu einem
\begriff{Zyklus} führen, sodass der Simplex-Algorithmus nicht terminiert.
Die Anwendung von \begriff{\name{Bland}s Regel} garantiert, dass nach einer endlichen Zahl von
Schritten die aktuelle Ecke verlassen und eine "`bessere"' (bzgl. Zielfunktionswert) Ecke erreicht
wird.
Allerdings ist die Anwendung dieser Regel nur empfehlenswert, wenn ein Zyklus vermutet wird,
andernfalls sind die anderen Strategien in der Praxis besser.

\pagebreak

\subsection{%
    Dualität%
}

\textbf{Dualität}:
Gegeben sei ein LP $\max_{Ax \le b} c^\tp x$ in Standardform,
wobei $A \in \real^{n \times d}$, $x \in \real^d$, $b \in \real^n$\\
und $c \in \real^d$.
Gesucht ist eine obere Schranke an den optimalen Zielfunktionswert $c^\tp x$,
ohne das LP direkt zu lösen.
Dazu führt man nicht-negative Variablen $y_1, \dotsc, y_n$ für jede NB ein, multipliziert
jede NB $(Ax)_i \le b_i$ mit der entsprechenden Variable $y_i$
und addiert alle anschließend die Ungleichungen.
Gesucht sind nun solche Werte der Variablen, sodass auf der linken Seite die Koef"|fizienten
der Zielfunktion entstehen und der Wert auf der rechten Seite so klein wie möglich ist.
Jeder Wert auf der rechten Seite, den man so erhält (auch wenn er nicht kleinstmöglich ist),
ist dann eine obere Schranke an den optimalen Zielfunktionswert von
$\max_{Ax \le b} c^\tp x$
(\begriff{schwache Dualität}).
Man kann sogar zeigen, dass der optimale, kleinstmögliche Wert exakt gleich dem optimalen
Zielfunktionswert von $\max_{Ax \le b} c^\tp x$ ist (\begriff{starke Dualität}).

\textbf{duales LP}:
Das zu $\max_{Ax \le b} c^\tp x$ \begriff{duale lineare Programm} ist gegeben durch\\
$\min b^\tp y$, wobei $A^\tp y = c$ und $y_1, \dotsc, y_n \ge 0$.

\linie

\textbf{wirtschaftliche Interpretation des dualen LPs zum Diätproblem}:
Man betrachte das Diätproblem $\min c^\tp x$, $Ax \ge b$,
wobei $x := \smallpmatrix{x_m\\x_t\\x_b\\x_c}$,
$c := \smallpmatrix{7\\3\\2\\4}$,
$A := \smallpmatrix{1&2&4&1\\3&2&1&4\\5&0&0&2\\&I_4}$ und
$b := \smallpmatrix{11\\7\\5\\0_4}$,
wobei $m$, $t$, $b$ und $c$ für die angebotenen Gerichte Fleisch, Tofu, Brot und Käse und
die ersten drei Gleichungen für die benötigten Größen Kohlenhydrate, Proteine und Fett stehen.
Man kann das zu diesem LP duale LP dann wie folgt interpretieren:
Angenommen, ein Produzent stellt Nahrungsergänzungsmittel-Pillen für Kohlenhydrate, Proteine
und Fett her und möchte seinen Gewinn maximieren.
Seien $y_c$, $y_p$ und $y_f$ jeweils die Preise einer der Pillen.
Der Produzent weiß, dass der Kunde $11$ der Kohlenhydrat-, $7$ der Protein- und $5$ der Fett-Pillen
benötigt, d.\,h. der Gewinn pro Kunde ist $11y_c + 7y_p + 5y_f$.

Allerdings kann der Produzent die Preise nicht beliebig hoch setzen:
Wenn die Preise zu hoch sind, wird der Kunde stattdessen eines der Nahrungsmittel kaufen.
Weil $y_c + 3y_p + 5y_f$ der Preis ist, den der Kunde ausgeben müsste, um die äquivalente Menge
an Stof"|fen zu erhalten, die in einer Einheit Fleisch enthalten ist, sollte
$y_c + 3y_p + 5y_f \le 7$ sein (wobei eine Einheit Fleisch $7$ kostet).
Analog erhält man $2y_c + 2y_p + 0y_f \le 3$ (Tofu),
$4y_c + 1y_p + 0y_f \le 2$ (Brot) und
$1y_c + 4y_p + 2y_f \le 4$ (Käse).
Zusätzlich wird noch $y_c, y_p, y_f \ge 0$ gefordert.

Dieses neue LP ist tatsächlich äquivalent zum dualen LP des Diätproblems:
Das Diätproblem lautet in der Standardform\\
$\max c^\tp x$ mit $Ax \le b$, wobei
$c := \smallpmatrix{-7\\-3\\-2\\-4}$,
$A := \smallpmatrix{-1&-2&-4&-1\\-3&-2&-1&-4\\-5&0&0&-2\\&-I_4}$,
$b := \smallpmatrix{-11\\-7\\-5\\0_4}$.\\
Setzt man $y := (y_c, y_p, y_f, y_1, y_2, y_3, y_4)^\tp$, so ist das duale Problem gegeben durch\\
$\min c^\tp y$ mit $Ay = b$ und $y \ge 0$, wobei
$c := \smallpmatrix{-11\\-7\\-5\\0_4}$,
$A := \smallpmatrix{-1&-3&-5&\\-2&-2&0&-I_4\\-4&-1&0&\\-1&-4&-2&}$,
$b := \smallpmatrix{-7\\-3\\-2\\-4}$.\\
Durch Multiplikation mit $-1$ kommt man zu\\
$\max c^\tp y$ mit $Ay = b$ und $y \ge 0$, wobei
$c := \smallpmatrix{11\\7\\5\\0_4}$,
$A := \smallpmatrix{1&3&5&\\2&2&0&I_4\\4&1&0&\\1&4&2&}$,
$b := \smallpmatrix{7\\3\\2\\4}$.\\
Jetzt eliminiert man $y_1, y_2, y_3, y_4$ als Schlupfvariablen und erhält\\
$\max c^\tp y'$ mit $Ay' \le b$ und $y' \ge 0$, wobei
$c := \smallpmatrix{11\\7\\5}$,
$A := \smallpmatrix{1&3&5\\2&2&0\\4&1&0\\1&4&2}$,
$b := \smallpmatrix{7\\3\\2\\4}$ mit
$y' := \smallpmatrix{y_c\\y_p\\y_f}$,\\
also obiges Produzenten-LP.

\pagebreak

\subsection{%
    Dualer Simplex-Algorithmus%
}

Obiger (primaler) Simplex-Algorithmus startet in einer Ecke des Zielbereichs und springt solange
zu besseren Ecken, wie es geht.
Die Idee des dualen Simplex-Algorithmus ist völlig anders.

\textbf{Idee}:
Der \begriff{duale Simplex-Algorithmus} verläuft wie folgt.
\begin{enumerate}
    \item
    Starte mit einer Ecke (V-Form), die optimal für eine Teilmenge der NBen ist.

    \item
    Solange eine NB die aktuelle V-Form verletzt, benutze diese NB, um zu einer besseren V-Form
    zu gelangen.

    \item
    Das Optimum ist erreicht, falls die aktuelle V-Form keine NB mehr verletzt
    (und daher zulässig ist).
\end{enumerate}

\textbf{V-Form}:
Eine Ecke ist der Schnitt von $d$ linear unabhängigen NBen.
Eine \begriff{V-Form} soll nun eine solche Ecke sein, sodass die NBen eine Bewegung in
Zielfunktionsrichtung (z.\,B. nach unten) verhindern.
Formal ist eine V-Form eine Teilmenge $B$ der NBen,
sodass $\exists_{y \in \real^n,\; y \ge 0}\; A^\tp y = c$
mit $y_i = 0$ für alle NBen $h_i \notin B$
(d.\,h. $c$ liegt im Kegel, der durch die NB-Normalen aufgespannt wird).
Somit ist jede V-Form ein zulässiger Punkt des dualen LPs.

\linie

\textbf{Ablauf}:
Gegeben sei ein primales LP in Standardform, wobei zur Vereinfachung der unterste Punkt gesucht
wird.
Außerdem sei eine Menge von $d$ linear unabhängigen NBen gegeben, die den Zielfunktionswert von
unten beschränken, d.\,h. eine initiale V-Form.

Eine V-Form ist eine Teilmenge $B$ von $d$ NBen, sodass der eindeutige Schnittpunkt $x_B$ zur
selben Zeit die optimale Lösung für das Teil-LP ist, wenn man nur die NBen aus $B$ verwendet.
Wenn $x_B$ zusätzlich noch zulässig für alle NBen nicht in $B$ ist, dann ist $x_B$ natürlich die
optimale Lösung für das ganze LP.

Daher braucht man nur nach einer höheren V-Form suchen, wenn $x_B$ eine NB $h_i \notin B$
verletzt, d.\,h. $(a_{i,\cdot}) x_B > b_i$
(das überprüft man, indem man $x_B$ in alle NBen außerhalb $B$ einsetzt).
Wenn eine verletzende NB $h_i$ gefunden wurde, dann kann man mit dieser NB die nächste V-Form
konstruieren, indem man den untersten zulässigen Punkt für die NBen $B \cup \{h_i\}$ sucht
(z.\,B. alle $\binom{d+1}{d} - 1 = d$ neuen Ecken durchgehen und
die unterste zulässige als nächste V-Form nehmen).

Der duale Simplex-Algorithmus läuft solange weiter, bis die V-Form für alle NBen zulässig ist.
Der Algorithmus terminiert, weil immer zu einer höheren V-Form gesprungen wird
(wenn die NBen in allgemeiner Lage sind) und es höchstens $\binom{n}{d} < \infty$
mögliche V-Formen gibt.

Die Invariante $A^\tp y = c$ ($y \ge 0$)
des dualen Simplex-Algorithmus entspricht der des primalen
Simplex-Algorithmus angewendet auf das duale LP.
Dabei ist $b^\tp y $ die "`Höhe"' der aktuellen V-Form,
weil $y_B = A_B^{-\tp} c$ und damit
$b^\tp y = y_B^\tp b_B = c^\tp (A_B^{-1} b_B) = c^\tp x_B$
mit $A_B$ der Teilmatrix mit den NBen aus $B$
(analog $b_B$ Teil der rechten Seite),
$x_B := A_B^{-1} b$ der Position der aktuellen V-Form (Schnittpunkt der NBen aus $B$) und
$y_i := (y_B)_i$ für $h_i \in B$ und $y_i := 0$ sonst.

\linie

\textbf{weitere LP-Algorithmen}:
Es ist nicht bekannt, ob die Simplex-Algorithmen in polynomieller Zeit laufen,
obwohl sie in der Praxis sehr schnell sind.
Es gibt polynomielle LP-Lösungsalgo"-rithmen, z.\,B. \begriff{Innere-Punkte-Methoden} oder
\begriff{Ellipsoid-Methoden}.

\pagebreak
