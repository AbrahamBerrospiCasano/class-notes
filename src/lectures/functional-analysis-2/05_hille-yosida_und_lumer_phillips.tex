\section{%
    Die Sätze von \name{Hille}-\name{Yosida} und \name{Lumer}-\name{Phillips}%
}

\subsection{%
    Spektrum dicht definierter, linearer Operatoren%
}

\begin{Bem}
    Im Folgenden ist $X$ ein $\complex$-Banachraum.
\end{Bem}

\begin{Def}{Resolventenmenge, -abbildung, Spektrum}\\
    Sei $T\colon D(T) \rightarrow X$ ein dicht definierter, linearer Operator.
    \begin{itemize}
        \item
        Die Menge $\varrho(T) := \{\lambda \in \complex \;|\;
        \text{$\lambda-T\colon D(T) \rightarrow X$ bijektiv, $(\lambda-T)^{-1} \in \Lin(X)$}\}$
        heißt \begriff{Resolventenmenge} von $T$.
        
        \item
        $R\colon \varrho(T) \rightarrow \Lin(X)$, $\lambda \mapsto R_\lambda := (\lambda-T)^{-1}$
        heißt \begriff{Resolventenabbildung} von $T$.
        
        \item
        $\sigma(T) := \complex \setminus \varrho(T)$ heißt \begriff{Spektrum} von $T$.
    \end{itemize}
\end{Def}

\begin{Bem}
    Ist $T$ auch noch abgeschlossen und $\lambda - T$ bijektiv, dann gilt
    $(\lambda-T)^{-1} \in \Lin(X)$ automatisch
    (Satz vom abg. Graphen für den abgeschlossenen,
    linearen Operator $(\lambda - T)^{-1}$).
\end{Bem}

\begin{Bem}
    Ist $T$ nicht abgeschlossen, dann ist $\varrho(T) = \emptyset$ bzw. $\sigma(T) = \complex$.
    Daher interessiert man sich normalerweise nur für das Spektrum abgeschlossener Operatoren.
\end{Bem}

\linie

\begin{Satz}{Eigenschaften}
    Sei $T\colon D(T) \rightarrow X$ ein dicht definierter, linearer Operator.
    Dann gilt:
    \begin{enumerate}
        \item
        $\varrho(T) \subset \complex$ ist of"|fen.
        
        \item
        $\sigma(T) \subset \complex$ ist abgeschlossen.
        
        \item
        Ist $T$ nicht stetig, dann muss $\sigma(T)$ nicht kompakt sein und es kann
        $\sigma(T) = \emptyset$ gelten.
        
        \item
        Die Resolventenabbildung $R\colon \varrho(T) \rightarrow \Lin(X)$
        ist holomorph und $R_\lambda - R_\mu = (\mu - \lambda) R_\lambda R_\mu$.
    \end{enumerate}
\end{Satz}

\subsection{%
    Der Satz von \name{Hille}-\name{Yosida} für Kontraktionshalbgruppen%
}

\begin{Satz}{Erzeuger von Kontraktionshalbgruppen}\\
    Seien $(T(t))_{t \ge 0}$ eine Kontraktionshalbgruppe mit Erzeuger $A$ und
    $\lambda \in \complex$ mit $\Re(\lambda) > 0$.\\
    Dann gilt:
    \begin{enumerate}
        \item
        $\lambda \in \varrho(A)$
        
        \item
        $\forall_{x \in X}\; (\lambda-A)^{-1} x = \int_0^\infty e^{-\lambda s} T(s) x \ds$
        
        \item
        $\norm{(\Re \lambda) (\lambda-A)^{-1}}_{\Lin(X)} \le 1$
    \end{enumerate}
\end{Satz}

\begin{Bem}
    Teil \emph{(2)} kann man als Laplace-Transformation von $s \mapsto T(s) x$ verstehen.
\end{Bem}

\linie

\begin{Satz}{Satz von \name{Hille}-\name{Yosida} für Kontraktionshalbgruppen}\\
    Ein linearer Operator $A$ ist ein Erzeuger einer Kontraktionshalbgruppe genau dann, wenn\\
    $A$ dicht definiert und abg. ist,
    $(0, \infty) \subset \varrho(A)$ gilt sowie
    $\forall_{\lambda > 0}\; \norm{\lambda (\lambda-A)^{-1}}_{\Lin(X)} \le 1$ gilt.
\end{Satz}

\pagebreak

\subsection{%
    Der Satz von \name{Hille}-\name{Yosida} für allgemeine \texorpdfstring{$\C_0$}{C₀}-Halbgruppen%
}

\begin{Satz}{Erzeuger von allg. $\C_0$-Halbgruppen}\\
    Seien $(T(t))_{t \ge 0}$ eine $\C_0$-Halbgruppe mit Erzeuger $A$,
    $M \ge 1$ und $\omega \in \real$ mit\\
    $\forall_{t \ge 0}\; \norm{T(t)}_{\Lin(X)} \le M e^{\omega t}$ und
    $\lambda \in \complex$ mit $\Re(\lambda) > \omega$.
    Dann gilt:
    \begin{enumerate}
        \item
        $\lambda \in \varrho(A)$
        
        \item
        $\forall_{x \in X}\; (\lambda-A)^{-1} x = \int_0^\infty e^{-\lambda s} T(s) x \ds$
        
        \item
        $\forall_{n \in \natural}\;
        \norm{(\Re \lambda - \omega)^n (\lambda-A)^{-n}}_{\Lin(X)} \le M$
    \end{enumerate}
\end{Satz}

\begin{Bem}
    Teil \emph{(2)} kann man als Laplace-Transformation von $s \mapsto T(s) x$ verstehen.
\end{Bem}

\linie

\begin{Satz}{Satz von \name{Hille}-\name{Yosida} für allg. $\C_0$-Halbgruppen}\\
    Ein linearer Operator $A$ ist ein Erzeuger einer $\C_0$-Halbgruppe genau dann, wenn\\
    $A$ dicht definiert und abg. ist und
    es $M \ge 1$ und $\omega \in \real$ gibt mit
    $(\omega, \infty) \subset \varrho(A)$ und\\
    $\forall_{\lambda > \omega} \forall_{n \in \natural}\;
    \norm{(\lambda - \omega)^n (\lambda-A)^{-n}}_{\Lin(X)} \le M$.\\
    In diesem Fall erfüllt die erzeugte Halbgruppe $(T(t))_{t \ge 0}$ die Abschätzung\\
    $\forall_{t \ge 0}\; \norm{T(t)}_{\Lin(X)} \le M e^{\omega t}$.
\end{Satz}

\subsection{%
    Dissipative Operatoren%
}

\begin{Def}{Dualitätsabbildung}
    Sei $X$ ein Banachraum.
    Dann heißt die Abbildung $J\colon X \to \P(X')$ mit
    $J(x) := \{x' \in X' \;|\; \norm{x'}_{X'} = \norm{x}_X,\; x'(x) = \norm{x}_X^2\}$
    \begriff{Dualitätsabbildung} von $X$.
\end{Def}

\linie

\begin{Bem}
    Nach einer Folgerung aus dem Satz von Hahn-Banach gibt es zu jedem $x \in X$
    ein $x' \in X'$ mit $\norm{x'}_{X'} = 1$ und
    $x'(x) = \norm{x}_X$.
    Daraus folgt $\norm{(\norm{x}_X x')}_{X'} = \norm{x}_X$ und
    $(\norm{x}_X x')(x) = \norm{x}_X^2$, d.\,h. $(\norm{x}_X x') \in J(x)$.
    Insbesondere gilt $J(x) \not= \emptyset$ für alle $x \in X$.
\end{Bem}
    
\begin{Bem}
    Für einen Hilbertraum $X$ erhält man $J(x) = \{\R x\}$ mit dem isometrischen Isomorphismus
    $\R\colon X \to X'$, $x \mapsto \sp{\cdot, x}$ aus dem Rieszschen Darstellungssatz:
    Einerseits gilt $\R x \in J(x)$.
    Andererseits folgt aus $x' \in J(x)$ mit $y := \R^{-1} x' \in X$,
    dass $\norm{y}_X = \norm{x}_X$ und $\sp{x, y} = \norm{x}_X^2$,
    d.\,h. $\norm{x}_X^2 = \sp{x, y} \le \norm{x}_X \norm{y}_X = \norm{x}_X^2$,
    nach C.-S. sind $x$ und $y$ linear abhängig, mit $\sp{x, y} = \norm{x}_X^2$ folgt $y = x$
    und damit $x' = \R x$.
\end{Bem}

\begin{Bsp}
    \begin{enumerate}
        \item
        Für $X = L^p$ mit $p \in (1, \infty)$ ist $J(f) \subset L^{p'} \cong (L^p)'$ ebenfalls immer
        einelementig, nämlich $J(f) = \{g\}$ mit
        $g(x) := \norm{f}_p^{2-p} \overline{f(x)} |f(x)|^{p-2}$ für $f(x) \not= 0$
        bzw. $g(x) := 0$ für $f(x) := 0$
        (wenn man $(L^p)'$ mit $L^{p'}$ mittels des konjugiert linearen, isometrischen Isomorphismus
        $J_{p'}\colon L^{p'} \to (L^p)'$, $(J_{p'} f)(g) := \int g \overline{f} d\mu$
        identifiziert).
        
        \item
        Für $X \in \{L^1, L^\infty, \C^0([0, 1])\}$ ist $J$ i.\,A. mengenwertig,
        z.\,B. gilt für $X = \C^0([0, 1])$, dass $J(x \mapsto 1)$
        isomorph zur Menge aller Wahrscheinlichkeitsmaße auf $[0, 1]$ ist.
    \end{enumerate}
\end{Bsp}

\linie

\begin{Def}{dissipativ/akkretiv}
    Sei $(A, D(A))$ ein linearer Operator.\\
    $A$ heißt \begriff{dissipativ}, falls
    $\forall_{x \in D(A)} \exists_{x' \in J(x)}\; \Re x'(Ax) \le 0$.\\
    $A$ heißt \begriff{akkretiv}, falls $-A$ dissipativ ist.
\end{Def}

\begin{Bem}
    Für $X$ Hilbertraum ist $A$ dissipativ genau dann, wenn
    $\forall_{x \in D(A)}\; \Re \sp{Ax, x} \le 0$.
\end{Bem}


\linie

\begin{Satz}{Charakterisierung von Dissipativität}\\
    Ein linearer Operator $A$ ist dissipativ genau dann, wenn
    $\forall_{\lambda > 0} \forall_{x \in D(A)}\; \norm{(\lambda - A) x}_X \ge \lambda \norm{x}_X$.
\end{Satz}

\pagebreak

\subsection{%
    Der Satz von \name{Lumer}-\name{Phillips}%
}

\begin{Satz}{Satz von \name{Lumer}-\name{Phillips}}\\
    Ein linearer Operator $A$ ist ein Erzeuger einer Kontraktionshalbgruppe genau dann, wenn\\
    $A$ dicht definiert und dissipativ ist und
    $\lambda_0 - A$ für ein $\lambda_0 > 0$ surjektiv ist.\\
    (In diesem Fall ist $\lambda - A$ für alle $\lambda > 0$ surjektiv.)
\end{Satz}

\linie

\begin{Bsp}
    \begin{enumerate}[label=\emph{(\alph*)}]
        \item
        Seien $X := \C^0_0(\real^n)$ und $A := \Delta$ mit $D(A) := \S(\real^n)$.
        Dann ist $\Delta$ dissipativ:
        Sei $\varphi \in D(\Delta)$.
        Dann gibt es ein $x_0 \in \real^n$ mit $|\varphi(x_0)| = \norm{\varphi}_X$.
        Mit $\alpha := \overline{\varphi(x_0)} \in \complex$ und
        $\ell := \alpha \delta_{x_0} \in X'$
        gilt $\ell \in J(\varphi)$,
        weil einerseits
        $\norm{\ell}_{X'} = |\alpha| \norm{\delta_{x_0}}_{X'} = |\alpha| = |\varphi(x_0)| =
        \norm{\varphi}_X$ sowie
        andererseits
        $\ell(\varphi) = \alpha \varphi(x_0) = |\varphi(x_0)|^2 = \norm{\varphi}_X^2$.
        Außerdem gilt $\Re(\ell(\Delta\varphi)) = \Re(\alpha \cdot (\Delta \varphi)(x_0)) \le 0$,
        da die reellwertige Funktion $\psi := \Re(\alpha\varphi)$ bei $x_0$ ihr Maximum annimmt,
        d.\,h. es gilt
        $\forall_{j=1,\dotsc,n}\; \frac{\partial^2\psi}{\partial x_j^2}(x_0) \le 0$.
        Somit ist $\Delta$ dissipativ.
        
        \item
        Betrachte das Anfangs-RWP
        $v_t = v_{xx}$ für $t \ge 0$ und $x \in [0, 1]$,
        $v(0, x) = f_0(x)$ für $x \in [0, 1]$ und
        $v(t, 0) = 0 = v(t, 1)$ für $t \ge 0$
        (\begriff{eindimensionale Wärmeleitungsgleichung}).
        Dieses Problem kann man wie folgt in ein abstraktes Cauchy-Problem übersetzen:
        Seien $X := \C^0_0((0, 1))$, $A := \frac{\partial^2}{\partial x^2}$ mit dem
        Definitionsbereich $D(A) := \C^0_0((0, 1)) \cap \C^2([0, 1])$ und
        $(u(t))(x) := v(t, x)$.
        Statt eine Lösung $v$ des Anfangs-RWPs zu bestimmen, kann man eine Lösung
        $u\colon [0, \infty) \to X$ von $u' = Au$, $u(0) = f_0$ bestimmen
        (jede Lösung $u$ induziert eine Lösung $v$, die Umkehrung gilt nicht).
        Eine Lösung $u$ existiert, wenn $A$ eine $\C_0$-Halbgruppe auf $X$ erzeugt.
        Dies ist nach dem Satz von Lumer-Phillips in der Tat der Fall, denn:
        \begin{itemize}
            \item
            $A$ ist dicht definiert
            (wegen $\overline{\C^\infty_c((0, 1))}^{\norm{\cdot}_{\C^0}} = X$),
            
            \item
            $A$ ist dissipativ (wie in \emph{(a)}) und
            
            \item
            $\id - A$ ist surjektiv,
            was äquivalent dazu ist, dass das RWP $f - f'' = g$ in $(0, 1)$ und $f(0) = 0 = f(1)$
            für alle $g \in X$ eindeutig in $D(A)$ lösbar ist
            (was man mithilfe von Fouriertransformation oder Regularitätstheorie zeigen kann).
        \end{itemize}
    \end{enumerate}
\end{Bsp}

\pagebreak
\ifthenelse{\equal{\standalonedoc}{true}}{\addtocontents{toc}{\protect\newpage}}{}
