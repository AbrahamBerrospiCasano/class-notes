\chapter{%
    Abstrakte \name{Cauchy}probleme%
}

\section{%
    Klassische und milde Lösung%
}

\begin{Def}{abstraktes \name{Cauchy}problem}\\
    Seien $X$ ein Banachraum, $(A, D(A))$ ein linearer Operator auf $X$ und $x \in X$.\\
    Dann heißt (ACP) mit $u'(t) = Au(t)$ für $t \ge 0$ und $u(0) = x$
    mit $u\colon [0, \infty) \to X$ \begriff{abstraktes \name{Cauchy}problem}
    mit Operator $A$ und Anfangswert $x$.
\end{Def}

\begin{Def}{klassische Lösung}
    Eine Funktion $u \in \C^1([0, \infty), X)$ heißt \begriff{klassische Lösung} von (ACP),
    falls $u$ (ACP) für alle $t \ge 0$ löst.
\end{Def}

\begin{Def}{milde Lösung}
    Eine Funktion $u \in \C^0([0, \infty), X)$ heißt \begriff{milde Lösung} von (ACP),
    falls\\
    $\forall_{t \ge 0}\; \int_0^t u(s)\ds \in D(A)$ und $u(t) = x + A\int_0^t u(s)\ds$.
\end{Def}

\begin{Bem}
    Ist $u$ eine klassische Lösung, so gilt notwendigerweise $\forall_{t \ge 0}\; u(t) \in D(A)$,
    d.\,h. insbesondere gilt $x \in D(A)$.
    Jede klassische Lösung ist für $A$ abg. auch eine milde Lösung.
    %(erhält man durch Integration der Gleichung $u'(t) = Au(t)$).
\end{Bem}

\linie

\begin{Satz}{Lösung für $A$ Erzeuger einer $\C_0$-HG}\\
    Seien $(A, D(A))$ der Erzeuger einer $\C_0$-Halbgruppe $(T(t))_{t \ge 0}$
    und $x \in X$.\\
    Dann ist $u\colon [0, \infty) \to X$, $u(t) := T(t) x$ die eind. milde Lsg. von (ACP)
    mit Op. $A$ und AW $x$.\\
    $u$ ist die eindeutige klassische Lösung von (ACP) genau dann, wenn $x \in D(A)$.
\end{Satz}

\section{%
    Wohlgestellte \name{Cauchy}probleme%
}

\begin{Def}{wohlgestellt}
    Sei $(A, D(A))$ ein abgeschlossener, linearer Operator.\\
    Dann heißt (ACP) \begriff{wohlgestellt}, falls
    \begin{itemize}
        \item
        $A$ dicht definiert ist,

        \item
        (ACP) die Existenz- und Eindeutigkeitsbedingung (EU) erfüllt, d.\,h.
        für alle $x \in D(A)$ gibt es eine eindeutige klassische Lösung
        $u(\cdot, x)$ von (ACP) zum Anfwangswert $x$, sowie

        \item
        die Lösung von (ACP) stetig von den Anfangsdaten abhängt, d.\,h.\\
        $\forall_{\text{$(x_n)_{n \in \natural}$ Folge in $D(A)$ mit $x_n \to 0$}}\;
        [\text{$u(t, x_n) \xrightarrow{n \to \infty} 0$ glm. auf kompakten $t$-Intervallen}]$.
    \end{itemize}
\end{Def}

\begin{Satz}{Charakterisierung von Erzeugern von $\C_0$-HG}\\
    Sei $(A, D(A))$ ein abgeschlossener, linearer Operator.
    Dann sind äquivalent:
    \begin{enumerate}
        \item
        (ACP) ist wohlgestellt.

        \item
        (ACP) erfüllt (EU) und es gilt $\varrho(A) \not= \emptyset$.

        \item
        $A$ erzeugt eine $\C_0$-Halbgruppe.
    \end{enumerate}
\end{Satz}

\linie

\begin{Def}{$A$-Norm}
    Sei $A\colon D(A) \to X$ ein linearer Operator.\\
    Dann ist $\norm{\cdot}_A$ mit $\norm{x}_A := \norm{x}_X + \norm{Ax}_X$ für $x \in D(A)$
    die \begriff{$A$-Norm} auf $D(A)$.
\end{Def}

\begin{Lemma}{Gen}
    Seien $(T(t))_{t \ge 0}$ eine $\C_0$-Halbgruppe auf $X$ mit Erzeuger $(A, D(A))$ und
    \\$Y \subset D(A)$ ein Unterraum mit
    $\overline{Y}^{\norm{\cdot}_X} = X$ und $\forall_{t \ge 0}\; T(t)Y \subset Y$.\\
    Dann gilt $\overline{Y}^{\norm{\cdot}_A} = D(A)$.
    In diesem Fall heißt $Y$ \begriff{Gen} von $(A, D(A))$.
\end{Lemma}

\begin{Lemma}{Fortsetzung abg. Operatoren}
    Seien $(A, D(A))$ und $(B, D(B))$ abgeschlossene, lineare Operatoren
    mit $B$ Fortsetzung von $A$ auf $D(B)$ (d.\,h. $D(A) \subset D(B)$ und $B|_{D(A)} = A$),
    wobei $\overline{D(A)}^{\norm{\cdot}_B} = D(B)$.
    Dann gilt $D(A) = D(B)$ und $A = B$.
\end{Lemma}

\pagebreak

\section{%
    Inhomogene abstrakte \name{Cauchy}probleme%
}

\begin{Def}{inhomogenes abstraktes \name{Cauchy}problem}\\
    Seien $X$ ein Banachraum, $(A, D(A))$ ein linearer Operator auf $X$ und $x \in X$.\\
    Außerdem seien $T \in \real^+ \cup \{\infty\}$ und $f\colon [0, T) \to X$ eine Funktion.\\
    Dann heißt $\text{(ACP)}_f$ mit $u'(t) = Au(t) + f(t)$ für $t \in [0, T)$ und $u(0) = x$
    mit $u\colon [0, T) \to X$ \begriff{inhomogenes abstraktes \name{Cauchy}problem}
    mit Operator $A$, rechter Seite $f$ und Anfangswert $x$.
\end{Def}

\begin{Def}{klassische Lösung}
    Eine Funktion $u \in \C^1([0, T), X)$ heißt \begriff{klassische Lösung} von $\text{(ACP)}_f$,
    falls $u$ $\text{(ACP)}_f$ für alle $t \in [0, T)$ löst.
\end{Def}

\begin{Def}{milde Lösung}
    Sei $(A, D(A))$ der Erzeuger einer $\C_0$-Halbgruppe $(T(t))_{t \ge 0}$ auf $X$.\\
    Eine Funktion $u \in \C^0([0, T), X)$ heißt \begriff{milde Lösung} von $\text{(ACP)}_f$
    mit Operator $A$, falls\\
    $\forall_{t \in [0, T)}\; u(t) = T(t)x + \int_0^t T(t-s) f(s)\ds$.
\end{Def}

\begin{Bem}
    Ist $u$ eine klassische Lösung, so gilt notwendigerweise
    $\forall_{t \in [0, T)}\; u(t) \in D(A)$, d.\,h. insbesondere gilt $x \in D(A)$.
\end{Bem}

\linie

\begin{Bem}
    Die Formel für die milde Lösung heißt auch \begriff{Variation-der-Konstanten-Formel}.
    Formal kann man sie folgendermaßen herleiten:
    Setze $u(t) := T(t) v(t)$.\\
    Dann ist $u'(t) = AT(t)v(t) + T(t)v'(t) \overset{!}{=} AT(t)v(t) + f(t)$.
    Unter der Annahme, dass $T(s)^{-1}$ existiert, ist obige Gleichung äquivalent zu
    $v(t) = v(0) + \int_0^t T(s)^{-1} f(s)\ds$, d.\,h.\\
    $u(t) = T(t) v(0) + \int_0^t T(t - s) f(s)\ds$.
\end{Bem}

\pagebreak

\section{%
    Inhomogenes Problem für stetige rechte Seiten
}

\begin{Bem}
    Seien $f \in \C^0([0, T), X)$, $u$ eine klassische Lösung von $\text{(ACP)}_f$
    und $t \in [0, T)$.\\
    Setze $v(s) := T(t - s) u(s)$ für $s \in [0, t)$.\\
    Dann gilt $\frac{d}{ds} v(s) = -AT(t - s) u(s) + T(t - s) Au(s) + T(t - s) f(s) =
    T(t - s) f(s)$.\\
    Da $f$ stetig ist, ist auch $s \mapsto T(t - s) f(s)$ stetig und somit erhält man\\
    $\int_0^t T(t - s) f(s) \ds = v(t) - v(0)$ und wegen $v(0) = T(t)u(0) = T(t)x$ und $v(t) = u(t)$
    somit\\
    $T(t)x + \int_0^t T(t - s) f(s) \ds = u(t)$.

    Daher gilt für $f \in \C^0([0, T), X)$:
    \begin{itemize}
        \item
        Jede klassische Lösung von $\text{(ACP)}_f$ ist eine milde Lösung.

        \item
        $\text{(ACP)}_f$ besitzt für jedes $x \in X$ eine eindeutige milde Lösung
        (nach Definition).
    \end{itemize}
\end{Bem}

\linie

\begin{Lemma}{milde Lsg. als klassische Lsg.}
    Seien $(A, D(A))$ der Erzeuger einer $\C_0$-Halbgruppe $(T(t))_{t \ge 0}$,
    $f \in \C^0([0, T), X)$ und $u$ eine milde Lösung von $\text{(ACP)}_f$ mit\\
    $u \in \C^0([0, T), D(A)) \cap \C^1([0, T), X)$.\\
    Dann ist $u$ eine klassische Lösung von $\text{(ACP)}_f$.
\end{Lemma}

\linie

\begin{Satz}{Charakterisierung der eind. klassischen Lösbarkeit}
    Seien $(A, D(A))$ der Erzeuger einer $\C_0$-Halbgruppe $(T(t))_{t \ge 0}$,
    $f \in \C^0([0, T), X)$ und $g(t) := \int_0^t T(t-s)f(s)\ds$ für $t \in [0, T)$.\\
    Dann sind äquivalent:
    \begin{enumerate}
        \item
        Für alle $x \in D(A)$ gibt es eine eindeutige klassische Lösung von $\text{(ACP)}_f$
        zum AW $x$.

        \item
        $g \in \C^1([0, T), X)$

        \item
        $g \in \C^0([0, T), (D(A), \norm{\cdot}_A))$
    \end{enumerate}
\end{Satz}

\begin{Bem}
    \emph{(3)} ist äquivalent zu $\Bild(g) \subset D(A)$ und $Ag \in \C^0([0, T), X)$.
\end{Bem}

\begin{Kor}
    Sei $A$ wie eben.
    Ist $f \in \C^1([0, T), X)$ oder $f \in \C^0([0, T), (D(A), \norm{\cdot}_A))$,
    dann besitzt $\text{(ACP)}_f$ für alle $x \in D(A)$ eine eindeutige klassische Lösung.
\end{Kor}

\begin{Bem}
    Gilt nur $f \in \C^0([0, T), X)$, dann besitzt $\text{(ACP)}_f$ i.\,A. nicht für alle
    $x \in D(A)$ eine klassische Lösung.
\end{Bem}

\linie

\begin{Satz}{\name{Hölder}-stetige rechte Seiten}
    Seien $(A, D(A))$ der Erzeuger einer analytischen Halbgruppe,
    $f \in \C^{0,\alpha}([0, T], X)$ mit $\alpha \in (0, 1]$
    und $u$ die milde Lösung von $\text{(ACP)}_f$.
    Dann gilt:
    \begin{enumerate}
        \item
        Für alle $x \in D(A)$ ist $u$ die eindeutige klassische Lösung von $\text{(ACP)}_f$
        zum AW $x$.

        \item
        Für alle $\delta > 0$ ist $Au, \frac{d}{dt} u \in \C^{0,\alpha}([\delta, T], X)$.

        \item
        Es gilt $Au, \frac{d}{dt} u \in \C^0([0, T), X)$.
    \end{enumerate}
\end{Satz}

\pagebreak

\section{%
    Viskose \name{Burgers}gleichung%
}

\begin{Bem}
    Im Folgenden wird die Theorie der abstrakten Cauchyprobleme zur Lösung nicht-linearer
    Anfangswertprobleme angewendet.
    Als Beispiel wird dafür die sog. viskose Burgersgleichung betrachtet.
    Diese Gleichung ähnelt der Wärmeleitungsgleichung (bis auf den quadratischen Term) und
    kann z.\,B. zur Modellierung von Verkehrsflüssen verwendet werden.
\end{Bem}

\begin{Def}{viskose \name{Burgers}gleichung}
    Die \begriff{viskose \name{Burgers}gleichung} ist gegeben durch\\
    $\partial_t u = \partial_x^2 u - \frac{1}{2} \partial_x (u^2)$ für $x \in \real$ und $t \ge 0$
    sowie $u(x, 0) = u_0(x)$ für $x \in \real$.
\end{Def}

\begin{Def}{milde Lösung}
    Sei $X := \C^0_\unif(\real)$.
    Eine Funktion $u \in \C^0([0, T_0], X)$ heißt
    \begriff{milde Lösung} der viskosen Burgersgleichung in $X$ zum AW $u_0 \in X$,
    falls
    $u(t) = T(t)u_0 + \int_0^t T(t - \tau) N(u)(\tau) \d\tau$,
    wobei $(T(t)u_0)(x) := (4\pi t)^{-1/2} \int_\real e^{-(x-y)^2/(4t)} u_0(y)\dy$ und
    $N(u) := -\frac{1}{2} \partial_x (u^2)$.
\end{Def}

\begin{Bem}
    $T(t)$ ist der Lösungsoperator der Wärmeleitungsgleichung auf $\real$ bzw.
    die eindimensionale Wärmeleitungshalbgruppe.
\end{Bem}

\linie

\begin{Satz}{eindeutige Existenz der milden Lösung}
    Sei $C_0 > 0$.\\
    Dann gibt es ein $T_0 > 0$, sodass für alle $u_0 \in X$ mit $\norm{u_0}_{\C^0} \le C_0$
    eine eindeutige milde Lösung $u \in \C^0([0, T_0], X)$ der viskosen
    Burgersgleichung zum AW $u_0$ existiert.
\end{Satz}

\begin{Bem}
    Da $T(t)$ für alle $t > 0$ glättend ist und $N(u)$ keine höheren Ableitungen als $\partial_x$
    enthält, kann man zeigen, dass die milde Lösung, deren Existenz eben behauptet wurde,
    auch eine klassische Lösung ist, wenn $u_0 \in \C^2_\unif$.
    Da jede klassische Lösung auch eine milde Lösung ist, folgt die lokale Existenz und
    Eindeutigkeit von klassischen Lösungen der viskosen Burgersgleichung.

    Mithilfe von Maximumsprinzip-Argumenten kann man auch die globale Existenz zeigen
    (d.\,h. für alle Zeiten).
    Dabei wird das im Beweis verwendete Fixpunktargument iterativ angewendet, ohne dass sich
    die Länge des zulässigen Zeitintervalls ändert.

    Die Beweisstrategie funktioniert allgemeiner für Gleichungen der Form
    $\partial_t u = \partial_x^2 u + f(u, \partial_x u)$ mit $f$ glatt, nicht aber für Gleichungen
    der Form $\partial_t u = \partial_x^2 u + f(u, \partial_x u, \partial_x^2 u)$, weil dann im
    Integral ein Faktor $(1 + (t - \tau)^{-1})$ vorkommt.
\end{Bem}

\pagebreak
