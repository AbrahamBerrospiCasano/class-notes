\optpart{%
    Regularitätstheorie%
}

\section{%
    Einbettungssätze für \name{Sobolev}- und \name{Hölder}räume%
}

\subsection{%
    Wiederholung und Motivation%
}

\begin{Bem}
    Zur Wiederholung werden die Definitionen von Sobolev- und Hölderräumen wiedergegeben.
\end{Bem}

\begin{Def}{\name{Sobolev}raum}
    Seien $\Omega \subset \real^n$ of"|fen, $m \in \natural_0$ und $p \in [1, \infty]$.\\
    Dann heißt der Vektorraum
    $W^{m,p}(\Omega) := \{f \in L^p(\Omega) \;|\; \forall_{s \in \natural_0^n,\, |s| \le m}
    \exists_{f^{(s)} \in L^p(\Omega)}\; f^{(0)} = f,$\\
    $\forall_{\varphi \in \C^\infty_c(\Omega)}\;
    \int_\Omega (\partial_x^s \varphi) f \dx = (-1)^{|s|} \int_\Omega \varphi f^{(s)} \dx\}$
    \begriff{\name{Sobolev}raum} der Ordnung $m$ mit Exponent $p$.\\
    $W^{m,p}(\Omega)$ wird mit der Norm
    $\norm{f}_{W^{m,p}(\Omega)} := \sum_{|s| \le m} \norm{f^{(s)}}_{L^p(\Omega)}$ versehen.
    Für $p = 2$ schreibt man auch $H^m(\Omega) := W^{m,2}(\Omega)$ bzw.
    $\norm{\cdot}_{H^m(\Omega)} := \norm{\cdot}_{W^{m,2}(\Omega)}$.
    Die Funktionen $f^{(s)}$ für $|s| \ge 1$ heißen \begriff{schwache Ableitungen} von $f$
    und werden mit $\partial_x^s f := f^{(s)}$ bezeichnet.
\end{Def}

\begin{Bem}
    Es gilt $W^{m,p}(\Omega) =
    \overline{W^{m,p}(\Omega) \cap \C^\infty(\Omega)}^{\norm{\cdot}_{W^{m,p}(\Omega)}}$
    für $p < \infty$.\\
    $W^{m,p}_0(\Omega) := \overline{\C^\infty_c(\Omega)}^{\norm{\cdot}_{W^{m,p}(\Omega)}}$
    ist der \begriff{\name{Sobolev}raum mit (verallg.) Nullrandwerten}.
\end{Bem}

\begin{Def}{\name{Hölder}raum}
    Seien $\Omega \subset \real^n$ of"|fen oder kompakt,
    $k \in \natural_0$ und $\alpha \in (0, 1]$.\\
    $\C^{k,\alpha}(\Omega) := \{f \in \C^k_b(\Omega) \;|\;
    \partial_x^j f \in \C^{0,\alpha}(\Omega) \text{ für } |j| = k\}$
    heißt \begriff{\name{Hölder}raum} der Ordnung $k$ mit Exponent $\alpha$.
    $\C^{k,\alpha}(\Omega)$ wird mit der Norm
    $\norm{f}_{\C^{k,\alpha}(\Omega)} := \norm{f}_{\C^k(\Omega)} +
    \sum_{|j|=k} [\partial_x^j f]_{\C^{0,\alpha}(\Omega)}$ versehen, wobei
    $[f]_{\C^{0,\alpha}(\Omega)} := \sup_{x_1, x_2 \in \Omega,\; x_1 \not= x_2}
    \frac{|f(x_1) - f(x_2)|}{|x_1 - x_2|^\alpha}$.
    (Für $\alpha = 0$ definiert man $\C^{k,0}(\Omega) := \C^k(\Omega)$.)
\end{Def}

\linie

\begin{Bem}
    Gesucht sind Bedingungen an $n, m, p, k, \alpha$, sodass
    $W^{m,p}(\real^n) \subset \C^{k,\alpha}(\real^n)$\\
    (oder sodass $W^{m,p}(\Omega) \subset \C^{k,\alpha}(\overline{\Omega})$
    mit $\Omega \subset \real^n$ of"|fen, beschränkt, Lipschitz-berandet).
    
    Für $u \in W^{1,\infty}(\real^n)$ gilt
    $\sup_{x_1, x_2 \in \real^n,\; x_1 \not= x_2} \frac{|u(x_1) - u(x_2)|}{|x_1 - x_2|}
    \le \sup_{x \in \real^n} |\nabla u(x)|$
    nach dem Mittelwertsatz (auch Hauptsatz der Dif"|ferentialrechnung).
    Für $u \in W^{1,\infty}(\real^n)$ gilt also $u \in \C^{0,1}(\real^n)$,
    d.\,h. für den Fall $(m, p, k, \alpha) = (1, \infty, 0, 1)$ gilt
    $W^{m,p}(\real^n) \subset \C^{k,\alpha}(\real^n)$.
    
    Man kann zeigen:
    Allgemeiner existieren für bestimmte $\alpha \in (0, 1)$ und $p \in [1, \infty)$
    auch Ungleichungen der Form
    $[u]_{\C^{0,\alpha}(\real^n)} \le C(n,p) \norm{\nabla u}_{L^p(\real^n)}$
    (Fall $m = 1$, $k = 0$).
    
    Um an die Beziehung zwischen $\alpha, n, p$ zu gelangen, bedient man sich eines
    Skalierungsarguments.
    Angenommen, eine solche Ungleichung existiert für $u \in W^{1,p}(\real^n)$.
    Dann ist für $\lambda > 0$ auch $u_\lambda \in W^{1,p}(\real^n)$ mit
    $u_\lambda(x) := u(\frac{x}{\lambda})$.
    Es gilt $[u_\lambda]_{\C^{0,\alpha}(\real^n)}
    = \sup_{x_1 \not= x_2} \frac{|u_\lambda(x_1) - u_\lambda(x_2)|}{|x_1 - x_2|^\alpha}$\\
    $= \lambda^{-\alpha} \cdot \sup_{x_1 \not= x_2}
    \frac{|u(x_1/\lambda) - u(x_2/\lambda)|}{|x_1/\lambda - x_2/\lambda|^\alpha}
    = \lambda^{-\alpha} \cdot [u]_{\C^{0,\alpha}(\real^n)}$
    sowie\\
    $\norm{\nabla u_\lambda}_{L^p(\real^n)}
    = \left(\int_{\real^n} |\nabla u_\lambda(x)|^p \dx\right)^{1/p}
    = \left(\int_{\real^n} \lambda^{-p} |\nabla u(\frac{x}{\lambda})|^p \dx\right)^{1/p}
    = \left(\int_{\real^n} \lambda^{n-p} |\nabla u(y)|^p \dy\right)^{1/p}$\\
    $= \lambda^{n/p-1} \cdot \norm{\nabla u}_{L^p(\real^n)}$.
    Unter der Annahme der Existenz der obigen Ungleichung gilt damit\\
    $\lambda^{-\alpha} \cdot [u]_{\C^{0,\alpha}(\real^n)}
    = [u_\lambda]_{\C^{0,\alpha}(\real^n)}
    \le C(n,p) \cdot \norm{\nabla u_\lambda}_{L^p(\real^n)}
    = C(n,p) \lambda^{n/p-1} \cdot \norm{\nabla u}_{L^p(\real^n)}$ bzw.\\
    $[u]_{\C^{0,\alpha}(\real^n)} \le \lambda^{n/p-1+\alpha}
    \cdot C(n,p) \norm{\nabla u}_{L^p(\real^n)}$.
    Diese Ungleichung kann nur für alle $\lambda > 0$ gelten, wenn
    $\frac{n}{p} - 1 + \alpha = 0$ ist,
    also $1 - \frac{n}{p} = \alpha$.
    Insbesondere muss wegen $\alpha > 0$ auch $p > n$ gelten.\\
    Für höhere Ableitungen ($m > 1$ oder $k > 0$) verfährt man ähnlich.
    
    Man vermutet daher, dass
    $W^{m,p}(\real^n) \subset \C^{k,\alpha}(\real^n)$
    für $m \in \natural$, $p \in [1, \infty)$, $k \in \natural_0$ und $\alpha \in (0, 1)$
    mit $m - \frac{n}{p} = k + \alpha$.
\end{Bem}

\linie
\pagebreak

\begin{Bem}
    Gesucht sind Bedingungen an $n, m_1, p_1, m_2, p_2$, sodass
    $W^{m_1,p_1}(\real^n) \subset W^{m_2,p_2}(\real^n)$
    (oder sodass $W^{m_1,p_1}(\Omega) \subset W^{m_2,p_2}(\Omega)$
    mit $\Omega \subset \real^n$ of"|fen, beschränkt, Lipschitz-berandet).
    
    Für $1 \le p < n$ kann man zeigen, dass es ein $p^\ast > p$ gibt mit
    $\norm{u}_{L^{p^\ast}(\real^n)} \le C(n,p) \norm{\nabla u}_{L^p(\real^n)}$
    für alle $u \in W^{1,p}(\real^n)$.
    Daraus folgt dann direkt $W^{1,p}(\real^n) \subset L^{p^\ast}(\real^n)$
    (Fall $m_1 = 1$, $m_2 = 0$).
    
    Zur Bestimmung von $p^\ast$ benutzt man wieder obiges Reskalierungsargument:\\
    $\norm{u_\lambda}_{L^{p^\ast}(\real^n)} =
    \lambda^{n/p^\ast} \cdot \norm{u}_{L^{p^\ast}(\real^n)}$
    und $\norm{\nabla u_\lambda}_{L^p(\real^n)} =
    \lambda^{n/p-1} \cdot \norm{\nabla u}_{L^p(\real^n)}$ wie oben.\\
    Damit gilt $\lambda^{n/p^\ast} \cdot \norm{u}_{L^{p^\ast}(\real^n)} =
    \norm{u_\lambda}_{L^{p^\ast}(\real^n)}
    \le C(n,p) \norm{\nabla u_\lambda}_{L^p(\real^n)} =
    \lambda^{n/p-1} \cdot C(n,p) \norm{\nabla u}_{L^p(\real^n)}$,
    also $\norm{u}_{L^{p^\ast}(\real^n)} \le
    \lambda^{n/p-1-n/p^\ast} \cdot C(n,p) \norm{\nabla u}_{L^p(\real^n)}$
    für alle $\lambda > 0$.
    Daraus folgt $\frac{n}{p} - 1 - \frac{n}{p^\ast} = 0$ bzw.
    $1 - \frac{n}{p} = -\frac{n}{p^\ast} \iff p^\ast = \frac{np}{n - p}$.
    
    Als Verallgemeinerung vermutet man $W^{m_1,p_1}(\real^n) \subset W^{m_2,p_2}(\real^n)$ für
    bestimmte $m_1, m_2 \in \natural_0$ mit $m_1 \ge m_2$ und $p_1, p_2 \in [1, \infty)$
    (genauer: für $m_1 - \frac{n}{p_1} = m_2 - \frac{n}{p_2}$ und $m_1 \ge m_2$).
\end{Bem}

\linie

\begin{Bsp}
    Wie hoch muss $m \in \natural$ sein, damit $H^m(\real^3) \subset \C^2(\real^3)$?
    (Zunächst sollen nur die Einbettungen $W^{1,p}(\real^n) \subset \C^{0,\alpha}(\real^n)$ und
    $W^{1,p}(\real^n) \subset L^{p^\ast}(\real^n)$ benutzt werden.)
    
    Sei $u \in H^m(\real^3)$.
    Dann existieren die schwachen Ableitungen $\partial_x^j u \in L^2(\real^3)$ in den Ordnungen
    $|j| \le m$.
    Für $H^m(\real^3)$ ist $p = 2$ und damit kleiner als $n = 3$.
    Daher kann die erste Einbettung aus den Bemerkungen oben nicht verwendet werden.
    Stattdessen kann man die zweite Einbettung $W^{1,p}(\real^n) \subset L^{p^\ast}(\real^n)$
    verwenden.
    Es gilt $p^\ast = \frac{np}{n - p} = \frac{3 \cdot 2}{3 - 2} = 6$,
    also $H^1(\real^3) \subset L^6(\real^3)$.
    Wegen $\forall_{|j| \le m-1}\; \partial_x^j u \in H^1(\real^3)$ gilt daher
    $\partial_x^j u \in L^6(\real^3)$ für alle $|j| \le m - 1$,
    also $u \in W^{m-1,6}(\real^3)$.
    
    Nun gilt $p^\ast > n$, daher kann man jetzt die erste Einbettung verwenden
    (für $m' := m - 1$).
    Aus der Gleichung $(m-1) - \frac{n}{p^\ast} = k + \alpha$ errechnet man
    $\alpha = (m-1) - \frac{n}{p^\ast} - k = (m-1) - \frac{3}{6} - 2 \in (0, 1)$
    zum Beispiel für $(m-1) = 3$ (mit dem gewünschten $k = 2$).
    Damit gilt $W^{3,6}(\real^3) \subset \C^{2,1/2}(\real^3)$.
    
    Insgesamt gilt also
    $H^m(\real^3) \subset H^4(\real^3) \subset W^{3,6}(\real^3) \subset \C^{2,1/2}(\real^3)
    \subset \C^2(\real^3)$ für $m \ge 4$.
    
    Wenn man $W^{m,p}(\real^n) \subset \C^{k,\alpha}(\real^n)$ mit $m - \frac{n}{p} = k + \alpha$
    verwendet, so erhält man das Resultat direkt
    (mit $(n, m, p, k, \alpha) = (3, 4, 2, 2, \frac{1}{2})$).
\end{Bsp}

\linie

\begin{Bem}
    Was kann man für beschränkte Gebiete erwarten?
    
    Sei $f_\varrho\colon \overline{B_1(0)} \subset \real^n \to \real$ mit
    $f_\varrho(x) := |x|^\varrho$ für $x \not= 0$ und $f_\varrho(0) := 0$,
    wobei $\varrho \in \real \setminus \natural_0$.\\
    Man kann direkt nachrechnen, dass dann gilt:
    \begin{enumerate}
        \item
        Für $k \in \natural_0$ und $\alpha \in (0, 1]$ gilt
        $f_\varrho \in \C^{k,\alpha}(\overline{B_1(0)}) \iff \varrho \ge k + \alpha$.
        
        \item
        Für $m \in \natural_0$ und $p \in [1, \infty)$ gilt
        $f_\varrho \in W^{m,p}(B_1(0)) \iff \varrho \ge m - \frac{n}{p}$.
    \end{enumerate}
    Dies motiviert die Vermutungen
    \begin{enumerate}
        \item
        $W^{m_1,p_1}(B_1(0)) \subset W^{m_2,p_2}(B_1(0))$ für
        $m_1 - \frac{n}{p_1} \ge m_2 - \frac{n}{p_2}$, $m_1 \ge m_2$ und
        $p_1, p_2 \in [1, \infty)$ sowie
        
        \item
        $W^{m,p}(B_1(0)) \subset \C^{k,\alpha}(\overline{B_1(0)})$ für
        $m - \frac{n}{p} \ge k + \alpha$, $p \in [1, \infty)$ und $\alpha \in (0, 1)$.
    \end{enumerate}
\end{Bem}

\pagebreak

\subsection{%
    \name{Gagliardo}-\name{Nirenberg}-\name{Sobolev}-Ungleichung%
}

\begin{Bem}
    Die Gagliardo-Nirenberg-Sobolev-Ungleichung beweist durch das anschließende Korollar
    die Einbettung $W^{m_1,p_1}(\real^n) \subset W^{m_2,p_2}(\real^n)$ für den Fall
    $m_1 = 1$, $m_2 = 0$.
\end{Bem}

\begin{Satz}{\name{Gagliardo}-\name{Nirenberg}-\name{Sobolev}-Ungleichung}\\
    Seien $p \in [1, n)$, $p^\ast := \frac{np}{n - p}$ und $u \in \C^1_c(\real^n)$.\\
    Dann ist $u \in L^{p^\ast}(\real^n)$ mit
    $\norm{u}_{L^{p^\ast}(\real^n)} \le C(n,p) \norm{\nabla u}_{L^p(\real^n)}$.
\end{Satz}

\begin{Kor}
    Seien $p \in [1, n)$, $p^\ast := \frac{np}{n - p}$ und $u \in W^{1,p}(\real^n)$.\\
    Dann ist $u \in L^{p^\ast}(\real^n)$ mit
    $\norm{u}_{L^{p^\ast}(\real^n)} \le C(n,p) \norm{\nabla u}_{L^p(\real^n)}$.
\end{Kor}

\linie

\begin{Bem}
    Für den Beweis des Korollars muss man Glättung durch Faltung
    (wenn $u$ kompakten Träger besitzt) und Abschneiden durch Multiplikation
    (wenn $u$ keinen kompakten Träger besitzt) durchführen.
\end{Bem}

\begin{Lemma}{Approximation durch Faltung}
    Seien $\varphi \in \C^\infty_c(\real^n)$ mit\\
    $\forall_{y \in \real^n}\; \varphi(y) \ge 0,\; \varphi(-y) = \varphi(y)$ und
    $\int_{\real^n} \varphi(y)\dy = 1$ sowie
    $\varphi_\varepsilon(x) := \varepsilon^{-n} \varphi(\frac{x}{\varepsilon})$
    für $\varepsilon > 0$.\\
    Außerdem seien $u \in L^p(\real^n)$ und $u_\varepsilon := \varphi_\varepsilon \ast u$.
    Dann gilt
    \begin{enumerate}
        \item
        $\supp(\varphi \ast u) \subset \overline{\supp(\varphi) + \supp(u)}$,
        
        \item
        $u_\varepsilon \in \C^\infty(\real^n)$ mit
        $\partial_x^s u_\varepsilon = (\partial_x^s \varphi_\varepsilon) \ast u$,
        
        \item
        für $u \in W^{1,p}(\real^n)$ gilt
        $\nabla u_\varepsilon = (\nabla u)_\varepsilon := \varphi_\varepsilon \ast \nabla u$,
        
        \item
        \begin{itemize}
            \item
            $\norm{u_\varepsilon}_{L^p(\real^n)} \le \norm{u}_{L^p(\real^n)}$
            (wegen $\norm{\varphi \ast u}_{L^p(\real^n)} \le
            \norm{\varphi}_{L^1(\real^n)} \norm{u}_{L^p(\real^n)}$) und
            
            \item
            für $u \in W^{1,p}(\real^n)$ gilt
            $\norm{(\nabla u)_\varepsilon}_{L^p(\real^n)} \le \norm{\nabla u}_{L^p(\real^n)}$
        \end{itemize}
        und
        
        \item
        \begin{itemize}
            \item
            $\lim_{\varepsilon \to 0} \norm{u_\varepsilon - u}_{L^p(\real^n)} = 0$,
            
            \item
            damit gilt für $u \in W^{1,p}(\real^n)$, dass
            $\lim_{\varepsilon \to 0} \norm{u_\varepsilon - u}_{W^{1,p}(\real^n)} = 0$,
            
            \item
            außerdem $\forall_{R > 0}\; \lim_{\varepsilon \to 0}
            \norm{u_\varepsilon - u}_{L^1(B_R(0))} = 0$ und
            
            \item
            damit $u_\varepsilon \to u$ f.ü. in $\real^n$.
        \end{itemize}
    \end{enumerate}
\end{Lemma}

\begin{Lemma}{Approximation durch Abschneidefunktionen}\\
    Seien $\eta \in \C^\infty(\real^n)$ mit
    $\forall_{z \in \real^n}\; \eta(z) \in [0, 1]$,
    $\eta(z) = 1$ für alle $|z| \le 1$ und
    $\eta(z) = 0$ für alle $|z| \ge 2$
    sowie $\eta_R(z) := \eta(\frac{z}{R})$ für $R > 0$.
    Außerdem seien $u \in W^{1,p}(\real^n)$ und $u_R := \eta_R \cdot u$.\\
    Dann gilt $u_R \in W^{1,p}(\real^n)$, wobei
    \begin{enumerate}
        \item
        $\norm{u_R}_{L^p(\real^n)} \le \norm{u}_{L^p(\real^n)}$,
        
        \item
        $\norm{\nabla u_R}_{L^p(\real^n)} \le \norm{\nabla u}_{L^p(\real^n)} +
        \frac{1}{R} \norm{\nabla\eta}_{L^\infty(\real^n)} \norm{u}_{L^p(\real^n)}$
        (wegen $\nabla u_R = \eta_R \nabla u + u \nabla \eta_R$).
    \end{enumerate}
\end{Lemma}

\pagebreak

\subsection{%
    Teil 1 des \name{Sobolev}schen Einbettungssatzes%
}

\begin{Lemma}{Fortsetzungsoperator}
    Seien $\Omega \subset \real^n$ of"|fen, beschränkt und Lipschitz-berandet,\\
    $p \in [1, \infty]$ und $\delta > 0$.
    Dann gibt es einen linearen und stetigen \begriff{Fortsetzungsoperator}\\
    $E\colon W^{1,p}(\Omega) \to W_0^{1,p}(B_\delta(\Omega))$ mit
    $\forall_{u \in W^{1,p}(\Omega)}\; (Eu)|_\Omega = u$.
\end{Lemma}

\begin{Satz}{Teil 1 des \name{Sobolev}schen Einbettungssatzes}\\
    Seien $m_1, m_2 \in \natural_0$ und $p_1, p_2 \in [1, \infty)$.
    \begin{enumerate}
        \item
        Ist $m_1 - \frac{n}{p_1} = m_2 - \frac{n}{p_2}$ und $m_1 \ge m_2$, dann existiert die
        Einbettung\\
        $\id\colon W^{m_1,p_1}(\real^n) \to W^{m_2,p_2}(\real^n)$ und ist stetig, d.\,h.\\
        $\exists_{C > 0} \forall_{u \in W^{m_1,p_1}(\real^n)}\;
        \norm{u}_{W^{m_2,p_2}(\real^n)} \le C \norm{u}_{W^{m_1,p_1}(\real^n)}$ mit
        $C = C(n, m_1, p_1, m_2, p_2)$.
        
        \item
        Sei $\Omega \subset \real^n$ of"|fen, beschränkt und Lipschitz-berandet.\\
        Ist $m_1 - \frac{n}{p_1} \ge m_2 - \frac{n}{p_2}$ und $m_1 \ge m_2$, dann existiert die
        Einbettung\\
        $\id\colon W^{m_1,p_1}(\Omega) \to W^{m_2,p_2}(\Omega)$ und ist stetig, d.\,h.\\
        $\exists_{C > 0} \forall_{u \in W^{m_1,p_1}(\Omega)}\;
        \norm{u}_{W^{m_2,p_2}(\Omega)} \le C \norm{u}_{W^{m_1,p_1}(\Omega)}$ mit
        $C = C(\Omega, n, m_1, p_1, m_2, p_2)$.
        
        \item
        Ist $m_1 - \frac{n}{p_1} > m_2 - \frac{n}{p_2}$ und $m_1 > m_2$, dann ist die Einbettung
        $\id\colon W^{m_1,p_1}(\Omega) \to W^{m_2,p_2}(\Omega)$ sogar kompakt.
        
        \item
        Für $\widetilde{\Omega} \subset \real^n$ nur of"|fen und beschränkt gelten die Aussagen
        \emph{(2)} und \emph{(3)} für die Räume $W^{m_i,p_i}_0(\widetilde{\Omega})$ anstatt
        $W^{m_i,p_i}(\Omega)$, wobei $W^{0,p}_0(\widetilde{\Omega}) := L^p(\widetilde{\Omega})$.
    \end{enumerate}
\end{Satz}

\pagebreak

\subsection{%
    \name{Morrey}sche Ungleichung%
}

\begin{Bem}
    Die Morreysche Ungleichung beweist durch den zweiten Teil des anschließenden Korollars
    die Einbettung $W^{m,p}(\real^n) \subset \C^{k,\alpha}(\real^n)$ für den Fall
    $m = 1$, $k = 0$.
\end{Bem}

\begin{Satz}{\name{Morrey}sche Ungleichung}
    Seien $p \in (n, \infty]$, $\alpha := 1 - \frac{n}{p}$ und $u \in \C^1(\real^n)$.\\
    Dann ist $u \in \C^{0,\alpha}(\real^n)$ mit
    $[u]_{\C^{0,\alpha}(\real^n)} \le C(n,p) \norm{\nabla u}_{L^p(\real^n)}$.
\end{Satz}

\begin{Bem}
    Die Bedingung $p > n$ ist nötig, damit keine Singularitäten auftreten
    (sonst $\alpha \le 0$).
\end{Bem}

\linie

\begin{Def}{\name{Hölder}-stetig für $L^p$-Funktionen}
    Seien $u \in L^p(\real^n)$ und $\alpha \in [0, 1]$.\\
    Dann heißt $u$ \begriff{\name{Hölder}-stetig} mit Exponent $\alpha$
    ($u \in \C^{0,\alpha}(\real^n)$), falls
    $\exists_{\widetilde{u} \in \C^{0,\alpha}(\real^n)}\;
    u = \widetilde{u} \text{ f.ü. auf } \real^n$.
    Außerdem sei $\norm{u}_{\C^{0,\alpha}(\real^n)} :=
    \norm{\widetilde{u}}_{\C^{0,\alpha}(\real^n)}$.
    Analog sind $u \in \C^{k,\alpha}(\real^n)$ und $\norm{u}_{\C^{k,\alpha}(\real^n)}$
    für $k \in \natural_0$ definiert.
    $\real^n$ kann durch $\Omega$ für $\Omega \subset \real^n$ of"|fen ersetzt werden.
\end{Def}

\begin{Kor}
    Seien $p \in (n, \infty)$ und $\alpha := 1 - \frac{n}{p}$.
    \begin{enumerate}
        \item
        Sei $u \in L^1_\loc(\real^n)$ mit $\nabla u \in L^p(\real^n)$.\\
        Dann ist $u \in \C^{0,\alpha}(\real^n)$ mit
        $[u]_{\C^{0,\alpha}(\real^n)} \le C(n,p) \norm{\nabla u}_{L^p(\real^n)}$.
        
        \item
        Sei $u \in W^{1,p}(\real^n)$.\\
        Dann ist $u \in \C^{0,\alpha}(\real^n)$ mit
        $\norm{u}_{\C^{0,\alpha}(\real^n)} \le C(n,p) \norm{u}_{W^{1,p}(\real^n)}$.
    \end{enumerate}
\end{Kor}

\subsection{%
    Teil 2 des \name{Sobolev}schen Einbettungssatzes%
}

\begin{Lemma}{Einbettungssätze für \name{Hölder}-Räume}\\
    Sei $\Omega \subset \real^n$ of"|fen, beschränkt und Lipschitz-berandet.
    Dann gilt:
    \begin{enumerate}
        \item
        Für $k \in \natural_0$ ist die Einbettung
        $\id\colon \C^{k+1}(\overline{\Omega}) \to \C^{k,1}(\overline{\Omega})$ stetig.
        
        \item
        Seien $k_1, k_2 \in \natural_0$ und $\alpha_1, \alpha_2 \in [0, 1]$ mit
        $k_1 + \alpha_1 > k_2 + \alpha_2$
        (im Fall $k_1 = 0$ kann sogar auf die Lipschitz-Berandung verzichtet werden).\\
        Dann ist die Einbettung $\id\colon \C^{k_1,\alpha_1}(\overline{\Omega}) \to
        \C^{k_2,\alpha_2}(\overline{\Omega})$ kompakt,
        wobei $\C^{k,0}(\overline{\Omega}) := \C^k(\overline{\Omega})$.
    \end{enumerate}
\end{Lemma}

\begin{Satz}{Teil 2 des \name{Sobolev}schen Einbettungssatzes}\\
    Seien $m \in \natural$, $p \in [1, \infty)$, $k \in \natural_0$ und $\alpha \in [0, 1]$.
    \begin{enumerate}
        \item
        Ist $m - \frac{n}{p} = k + \alpha$ und $\alpha \in (0, 1)$, dann existiert die
        Einbettung
        $\id\colon W^{m,p}(\real^n) \to \C^{k,\alpha}(\real^n)$ und ist stetig, d.\,h.
        $\exists_{C > 0} \forall_{u \in W^{m,p}(\real^n)}\;
        \norm{u}_{\C^{k,\alpha}(\real^n)} \le C \norm{u}_{W^{m,p}(\real^n)}$ mit
        $C = C(n, m, p, k, \alpha)$.
        
        \item
        Sei $\Omega \subset \real^n$ of"|fen, beschränkt und Lipschitz-berandet.\\
        Ist $m - \frac{n}{p} \ge k + \alpha$ und $\alpha \in (0, 1)$, dann existiert die
        Einbettung
        $\id\colon W^{m,p}(\Omega) \to \C^{k,\alpha}(\overline{\Omega})$
        und ist stetig, d.\,h.
        $\exists_{C > 0} \forall_{u \in W^{m,p}(\Omega)}\;
        \norm{u}_{\C^{k,\alpha}(\overline{\Omega})} \le C \norm{u}_{W^{m,p}(\Omega)}$ mit
        $C = C(\Omega, n, m, p, k, \alpha)$.
        
        \item
        Ist $m - \frac{n}{p} > k + \alpha$ und $\alpha \in [0, 1]$, dann existiert die Einbettung
        $\id\colon W^{m,p}(\Omega) \to \C^{k,\alpha}(\overline{\Omega})$ und ist stetig und
        kompakt.
        
        \item
        Für $\widetilde{\Omega} \subset \real^n$ nur of"|fen und beschränkt gelten die Aussagen
        \emph{(2)} und \emph{(3)} für die Räume $W^{m,p}_0(\widetilde{\Omega})$ anstatt
        $W^{m,p}(\Omega)$.
    \end{enumerate}
\end{Satz}

\begin{Satz}{Einbettung für $p = \infty$, $\alpha = 1$ ist Isomorphismus}\\
    Seien $k \in \natural_0$ sowie
    $\Omega \subset \real^n$ of"|fen, beschränkt und Lipschitz-berandet.\\
    Dann ist die Einbettung
    $\id\colon \C^{k,1}(\overline{\Omega}) \to W^{k+1,\infty}(\Omega)$ ein Isomorphismus.
\end{Satz}

\pagebreak
