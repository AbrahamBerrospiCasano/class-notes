\chapter{%
    Überblick über zentrale Resultate zu stark stetigen Halbgruppen%
}

\section{%
    Störungen%
}

\begin{Satz}{Störung mit beschränktem Operator}\\
    Seien $(A, D(A))$ der Erzeuger einer $\C_0$-Halbgruppe $(T(t))_{t \ge 0}$ auf einem
    Banachraum $X$,\\
    $M > 1$ und $\omega \in \real$ mit
    $\forall_{t \ge 0}\; \norm{T(t)}_{\Lin(X)} \le Me^{\omega t}$ und $B \in \Lin(X)$.\\
    Dann erzeugt $(A + B, D(A))$ eine $\C_0$-Halbgruppe $(S(t))_{t \ge 0}$ mit\\
    $\forall_{t \ge 0}\; \norm{S(t)}_{\Lin(X)} \le Me^{(\omega + M\norm{B}_{\Lin(X)}) t}$.\\
    Außerdem gilt $\forall_{t \ge 0} \forall_{x \in X}\;
    S(t)x = T(t)x + \int_0^t T(t-s)BS(s)x\ds$.
\end{Satz}

\begin{Bsp}
    Die Dif"|ferentialgleichung
    $\frac{\partial u}{\partial t} = \frac{\partial^2 u}{\partial x^2} + Vu$
    für $x \in \real$ und $t > 0$ sowie $u(0, x) = u_0(x)$ mit $u_0 \in H^2$ und $L^2(\real)$
    ist eindeutig klassisch lösbar, da $\frac{\partial^2}{\partial x^2} + V$ nach dem Satz von
    eben eine $\C_0$-Halbgruppe erzeugt.
\end{Bsp}

\begin{Satz}{\name{Dyson}-\name{Phillips}-Reihe}\\
    Mit den Voraussetzungen des Satzes von eben gilt
    $S(t) = \sum_{n=0}^\infty S_n(t)$ mit $S_0(t) := T(t)$ und
    $S_{n+1}(t)x := \int_0^t T(t-s)BS_n(x)x\ds$ für $x \in X$, $t \ge 0$ und $n \in \natural_0$.\\
    Die Reihe konvergiert in $\Lin(X)$ gleichmäßig für $t$ aus kompakten Intervallen in $\real^+$
    und heißt \begriff{\name{Dyson}-\name{Phillips}-Reihe}.
\end{Satz}

\linie

\begin{Def}{$A$-beschränkt}
    Seien $(A, D(A))$ und $(B, D(B))$ Operatoren auf $X$.\\
    Dann heißt $B$ \begriff{$A$-beschränkt}, falls $D(A) \subset D(B)$ und $a_B < \infty$ mit\\
    $a_B := \inf\{a \ge 0 \;|\; \exists_{b \ge 0} \forall_{x \in D(A)}\;
    \norm{Bx}_X \le a \norm{Ax}_X + b \norm{x}_X\}$.\\
    In diesem Fall heißt $a_B$ \begriff{$A$-Schranke} von $B$.
\end{Def}

\begin{Satz}{Störung mit $A$-beschränktem Operator}
    Seien $(A, D(A))$ der Erzeuger einer Kontraktionshalbgruppe sowie
    $(B, D(B))$ dissipativ und $A$-beschränkt mit $A$-Schranke $a_B < 1$.\\
    Dann erzeugt $(A + B, D(A))$ eine Kontraktionshalbgruppe.
\end{Satz}

\begin{Bsp}
    Seien $X := \C_0^0(\real)$, $D(B) := \{f \in X \cap \C^1 \;|\; f' \in X\}$ und
    $Bf := \pm f'$ für $f \in X$.\\
    Dann ist $B$ dissipativ (da Erzeuger einer Kontraktionshalbgruppe).\\
    Definiert man $D(A) := D(B^2) \subset D(B)$ und $Af := f''$ für $f \in D(A)$,
    so ist $A$ Erzeuger einer Kontraktionshalbgruppe und
    $B$ ist $A$-beschränkt mit Schranke $0$.
    Nach dem Satz erzeugt $(A + \alpha B, D(A))$ für beliebiges $\alpha \in \real$ eine
    Kontraktionshalbgruppe.\\
    Daraus folgt bspw. die Lösbarkeit von
    $\frac{\partial u}{\partial t} =
    \frac{\partial^2 u}{\partial x^2} + \alpha \frac{\partial u}{\partial x}$
    für $x \in \real$ und $t > 0$.
\end{Bsp}

\linie

\begin{Satz}{Variante für $A$ Erzeuger einer analyt. HG}\\
    Sei $(A, D(A))$ der Erzeuger einer analytischen Halbgruppe.\\
    Dann gibt es ein $\delta = \delta(A) > 0$, sodass
    $(A + B, D(A))$ für jeden $A$-beschränkten Operator mit Schranke $a_B < \delta$
    eine analytische Halbgruppe erzeugt.
\end{Satz}

\pagebreak

\section{%
    Approximationen%
}

\begin{Bem}
    Im Folgenden sei $G(M, \omega) := \{\text{$(T(t))_{t \ge 0}$ $\C_0$-HG} \;|\;
    \forall_{t \ge 0}\; \norm{T(t)}_{\Lin(X)} \le Me^{\omega t}\}$
    für $M \ge 1$ und $\omega \in \real$.
\end{Bem}

\begin{Satz}{\name{Trotter}-\name{Kato}-Approximationstheorem}\\
    Sei $(T_n(t))_{t \ge 0} \in G(M, \omega)$ mit Erzeuger $(A_n, D(A_n))$ für alle
    $n \in \natural$.
    Für ein $\lambda_0 \ge \omega$ betrachtet man die folgenden Aussagen:
    \begin{enumerate}
        \item
        Es existiert ein dicht definierter Operator $(A, D(A))$, sodass es ein Gen $D$
        von $A$ gibt mit $\forall_{x \in D}\; A_n x \xrightarrow{n \to \infty} Ax$ und
        $\overline{\Bild(\lambda_0 - A)} = X$.

        \item
        Es gibt ein $R \in \Lin(X)$ mit $R(\lambda_0, A_n) \xrightarrow{n \to \infty} R$
        punktweise in $X$ und $\overline{\Bild(R)} = X$.

        \item
        Die $\C_0$-Halbgruppen $(T_n(t))_{t \ge 0}$ konvergieren für $n \to \infty$ punktweise
        in $X$ gleichmäßig für $t \in [0, t_0]$ gegen eine $\C_0$-Halbgruppe $(T(t))_{t \ge 0}$
        mit Erzeuger $B$.
    \end{enumerate}
    Dann gilt \emph{(1)} $\implies$ \emph{(2)} $\iff$ \emph{(3)}.\\
    Falls \emph{(1)} gilt, so gilt $B = \overline{A}$.
    Falls \emph{(3)} gilt, so gilt $R = R(\lambda_0, B)$.
\end{Satz}

\begin{Bsp}
    Die Yosida-Approximation $A_n := nA R(n, A)$ mit
    $(A, D(A))$ dicht definiert,\\
    $(\omega, \infty) \subset \varrho(A)$ und
    $\norm{R(\lambda, A)^n}_{\Lin(X)} \le \frac{M}{(\lambda - \omega)^n}$ für $n \in \natural$
    ist ein Spezialfall des Trotter-Kato-Approximationstheorems.
\end{Bsp}

\linie

\begin{Satz}{\name{Chernoff}-Produktformel}
    Seien $V\colon \real_0^+ \to \Lin(X)$ stark stetig und $D \subset X$, sodass
    \begin{enumerate}
        \item
        $V(0) = \id$,

        \item
        $\forall_{t \ge 0} \forall_{m \in \natural}\; \norm{V(t)^m}_{\Lin(X)} \le M$,

        \item
        $\forall_{x \in D}\;
        [\text{$Ax := \lim_{t \to 0+0} \frac{V(t)x - x}{t}$ existiert in $X$}]$ und

        \item
        $\exists_{\lambda_0 > 0}\; [\text{$D, (\lambda_0 - A)D$ dicht in $X$}]$.
    \end{enumerate}
    Dann ist $(A, D)$ abschließbar, $\overline{A}$ erzeugt eine beschränkte $\C_0$-Halbgruppe
    $(T(t))_{t \ge 0}$ mit\\
    $T(t)x := \lim_{n \to \infty} V(\frac{t}{n})^n x$ für $x \in X$ und
    die Konvergenz ist gleichmäßig für $t$ aus kompakten Intervallen aus $\real^+_0$.
\end{Satz}

\begin{Bsp}
    Sei $(T(t))_{t \ge 0} \in G(M, \omega)$ mit Erzeuger $(A, D(A))$.\\
    Dann gilt $T(t)x = \lim_{n \to \infty} (\id - \frac{t}{n} A)^{-n} x
    = \lim_{n \to \infty} (\frac{n}{t} R(\frac{n}{t}, A))^n x$ für alle $x \in X$ und $t \ge 0$
    gleichmäßig auf kompakten $t$-Intervallen.
    In diesem Sinne gilt $T(t) = e^{tA}$.
\end{Bsp}

\linie

\begin{Satz}{\name{Trotter}-Produktformel}
    Seien $(T(t))_{t \ge 0}$ und $(S(t))_{t \ge 0}$ $\C_0$-Halbgruppen mit den Erzeugern
    $(A, D(A))$ bzw. $(B, D(B))$, sodass $\forall_{t \ge 0} \forall_{m \in \natural}\;
    \norm{(T(t)S(t))^m}_{\Lin(X)} \le Me^{\omega mt}$ und\\
    $\exists_{\lambda_0 > \omega}\; [\text{$(\lambda_0 - A - B)D, D$ dicht in $X$}]$, wobei
    $D := D(A) \cap D(B)$.\\
    Dann ist $(A + B, D)$ abschließbar und $\overline{A + B}$ erzeugt eine
    $\C_0$-Halbgruppe $(U(t))_{t \ge 0} \in G(M, \omega)$ mit
    $U(t)x := \lim_{n \to \infty} (T(\frac{t}{n}) S(\frac{t}{n}))^n x$ für $x \in X$ und $t \ge 0$.
\end{Satz}

\pagebreak

\section{%
    Spektraleigenschaften%
}

\begin{Bem}
    Sei $(A, D(A))$ der Erzeuger einer $\C_0$-Halbgruppe $(T(t))_{t \ge 0}$ auf $X$.\\
    Die Frage ist, ob $\forall_{t \ge 0}\; e^{t \sigma(A)} = \sigma(T(t)) \setminus \{0\}$
    (SMT) gilt (\begriff{spectral mapping theorem}).
\end{Bem}

\begin{Satz}{Spektralabbildungssatz}\\
    Es gilt $\forall_{t \ge 0}\; e^{t \sigma(A)} \subset \sigma(T(t)) \setminus \{0\}$,
    im Allgemeinen gilt jedoch keine Gleichheit.\\
    Für normstetige oder analytische Halbgruppen gilt jedoch Gleichheit.\\
    Gilt (SMT), dann ist die Spektralschranke
    $s(A) := \sup\{\Re \lambda \;|\; \lambda \in \sigma(A)\}$ gleich der Wachstumsschranke
    $\omega_0((T(t))_{t \ge 0})$.
    Im Allgemeinen gilt nur $s(A) \le \omega_0((T(t))_{t \ge 0})$.\\
    Gilt $s(A) = \omega_0((T(t))_{t \ge 0})$, dann gilt das
    \begriff{Stabilitätskriterium von \name{Lyapunov}}, d.\,h.\\
    $s(A) < 0 \iff \omega_0((T(t))_{t \ge 0}) < 0$
    (eine negative Spektralschranke ist äquivalent zur asymptotischen Stabilität von $0$).
\end{Satz}

\pagebreak
