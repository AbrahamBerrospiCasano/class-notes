\chapter{%
    Entstehung der Erde%
}

\section{%
    Elemententstehung%
}

\textbf{Alter des Sonnensystems}:
$T_0 = \SI{4.566}{\giga\year}$

\textbf{Ursprünge der Elemente}:
nach dem Urknall nur wenige Elemente vorhanden (Wasserstoff, Helium usw.),
in Sternen fusionieren Wasserstoff und andere Elemente zu schweren Elementen,
aber nur bis hin zum Eisen (danach nimmt chemische Bindungsenergie ab),
andere Metalle und schwerere Elemente entstehen nur in Roten Riesen und Supernovae
(in der Schockfront werden Neutronen in Atome gepresst),
die bei der Explosion dann in andere Bereiche des Weltalls getragen werden

\textbf{kosmische Anreicherung von Leichtmetallen}:
durch Konzentration von Staub und Gas bildet sich ein Stern,
in dem Wasserstoff fusioniert,
nach der Auf"|lösung des Sonnennebels bildet sich ein Planetensystem,
wenn der Stern als Roter Riese explodiert, bilden sich Leichtmetalle
(u.\,a. Sauerstoff, Magnesium, Aluminium, Silizium)
und Staub und Gas können wieder einen Stern hervorbringen usw.

\textbf{kosmische Anreicherung von Schwermetallen}:
nach der Synthese von Eisen kann ein Stern in einer Supernova explodieren,
wobei Elemente schwerer als Eisen durch Einfangen von Neutronen synthetisiert werden
(r-Prozess)

\section{%
    Kosmischer Pfad zur Erde%
}

\textbf{Anforderungen an das Sonnensystem}:
befindet sich in einer Galaxie mit hohem Gehalt an schweren Elementen,
Rote Riesensterne und Supernovae in der Nähe,
besitzt nur eine Sonne (sonst komische Planetenbahnen),
Sonne sollte so groß sein wie unsere Sonne

\textbf{Anforderungen an den Planeten}:
richtige Größe (damit richtige Wärmeproduktion im Inneren),
erdähnlicher Abstand zur Sonne (für richtige Temperatur),
Wasser darf nicht zu viel und nicht zu wenig vorhanden sein und muss flüssig sein,
es muss eine Kollision gegeben haben, die den Mond entstehen hat lassen (Stabilisierung),
großer, überhitzter, eiserner Erdkern als Wärmereservoir,
es muss ein Klima mit Jahreszeiten geben
(damit Sonnenwärme global verteilt wird),
d.\,h. stabile Ekliptikschiefe (bei Erde stabilisiert Jupiter die Ekliptik),
der Planet braucht eine schnelle Rotation (Temperatur)
und ein starkes Magnetfeld (kosmische Strahlung)

\pagebreak
