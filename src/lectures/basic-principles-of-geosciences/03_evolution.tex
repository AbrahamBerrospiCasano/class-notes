\chapter{%
    Evolution des Lebens%
}

\section{%
    Biologische Zeitenwende vor 542 Millionen Jahren
}

\textbf{Evolution vor der Zeitenwende}:
älteste Mikroben stammen von \SI{3.5}{\giga\year},
freier Sauerstoff in der Atmosphäre findet sich seit \SI{2.4}{\giga\year},
Zellen mit Kern, Atmung und Kohlenstoff-Recycling seit \SI{2.1}{\giga\year},
Sexualität und Vielzelligkeit seit \SIrange{1.5}{0.8}{\giga\year},
Tiere seit \SI{635}{\mega\year} und
Tiere mit Skelett seit \SI{542}{\mega\year}

\textbf{\SI{40}{\mega\year} Kreativität}:
unmittelbar vor der Zeitenwende kam es zu einer raschen Entfaltung vielzelliger
Lebewesen (Ediacara) und zu einer verstärkten Freisetzung von \ce{O2},
der \ce{O2}-Gehalt erreichte \SI{2}{\percent}, dies reichte zur Bildung einer Ozonschicht aus,
damit war das flache Wasser (Schelfgebiete) besiedelbar geworden

\textbf{großes Fressen}:
Zeitenwende deshalb so wichtig, weil durch Skelette Organspezialisierung möglich gemacht wurde,
Bildung selbstoptimierender Lebensgemeinschaften,
Entstehung der Ökosysteme des Meeres

\section{%
    Beschwerlicher Landgang vor 350 Millionen Jahren%
}

\textbf{entscheidende Faktoren}:
Sauerstoff, Ozon und Skelett,
im Unterschied zum Leben im Meer ist ein höherer Energieaufwand für die Fortbewegung notwendig

\textbf{Landgang}:
zuerst wurde das Land durch höhere Pflanzen besiedelt, danach durch Tiere

\textbf{Pflanzen}:
Leitungsgefäße, Wurzeln, Blätter (wg. sinkendem \ce{CO2}-Gehalt)
und flugfähige Samen,
es gab riesige Tief"|landswälder und wegen bis zu \SI{30}{\percent} \ce{O2} in der Atmosphäre
Riesenwuchs bei Farnen usw.

\textbf{Verholzung}:
die neu entwickelte Verholzung führt zu einer langsameren Verrottung,
dadurch bleibt organisches Material länger erhalten (Fixierung von Kohlenstoff),
Senkung des \ce{CO2}-Gehalts,
Entstehung der Wälder

\textbf{Tiere}:
wegen dem hohen \ce{O2}-Gehalt gab es Rieseninsekten und Riesenlurche,
außerdem haben sich Vorfahren der Reptilien entwickelt,
bei Landwirbeltieren kam es zur Entwicklung von tragfähigen Gliedmaßen und von der Lunge

\section{%
    Reptilien%
}

\textbf{neue Entwicklungen}:
Schutz gegen Austrocknung,
verbesserte Zahnformen und Gebisse,
Rippen mit Gelenken (verbesserte Atmung),
Gliedmaßen unter dem Körper (verbesserte Fortbewegung),
innere Befruchtung (wenige Eier, Eischale, Doppelaquarium),
Kinder kommen als fertige Reptilien auf die Welt (sofort lernfähig),
Kommunikation über verbessertes Riechvermögen (verbesserte Brutpflege),
Weitergabe von Informationen während der Brutpflege (Steigerung des kollektiven Gedächtnisses),
exponentiell steigende Datenmenge (stark verbesserte Hirnleistung),
verbesserter Orientierungssinn (Kolonisierung von Pangäa)

\begin{wichtig}
    \item
    \textbf{Errungenschaften gegenüber den Amphibien}:\\
    \emph{Eier mit Schale und Dottervorrat,
    innere Befruchtung,
    Schutz vor Austrocknung durch verhornte Haut,
    leistungsfähiges Gebiss,
    Rippen mit Gelenken}
\end{wichtig}

\pagebreak

\section{%
    Großes Sterben vor 251 Millionen Jahren%
}

\textbf{großes Sterben}:
mehr als \SI{90}{\percent} aller Arten sind ausgestorben,
Rekonvaleszenz-Zeit der Biosphäre
mindestens $\SI{50}{\mega\year}$

\begin{wichtig}
    \item
    \textbf{Ablauf}:
    \emph{Eruption großer Basaltmengen in Sibirien\\
    $\Rightarrow$ hohe \ce{CO2}-Konzentration\\
    $\Rightarrow$ spontane Temperaturerhöhung\\
    $\Rightarrow$ Destabilisierung von Gashydraten und Freisetzung von Methan\\
    $\Rightarrow$ hoher \ce{H2S}- und niedriger \ce{O2}-Gehalt
    im Ozean und in der Atmosphäre
    (Absturz des Sauerstoffgehalts der Atmosphäre auf
    $< \SI[math-rm=\mathit,text-rm=\itshape]{15}{\percent}$)\\
    $\Rightarrow$ Kollaps der Nahrungskette\\
    $\Rightarrow$ kurzfristiges Supertreibhaus}
\end{wichtig}

\section{%
    Jura- und Kreidezeit und Vögel%
}

\textbf{langsam steigender Sauerstoffgehalt}:
Riesenwuchs bei Meeresechsen,
Spezialisierung und Riesenwuchs bei Dinosauriern,
Luftsack-System der Dinosaurier und pneumatische Gelenkstützen (aus Atemnot entstanden),
Flugsaurier

\textbf{Errungenschaften der Vögel}:
Warmblütigkeit,
leichte Knochen (Flugmuskulatur im Dauerbetrieb),
verbesserte Sehschärfe (dreidimensionales Sehen),
verbessertes Erinnerungs- und Navigationsvermögen,
funktionale Frühgeburten bei Singvögeln,
hohe Lernfähigkeit (verbesserte Anpassung)

\textbf{Ursache der Wirbeltiere}:
hoher \ce{O2}-Gehalt

\section{%
    Explosion im Treibhaus vor 65 Millionen Jahren%
}

\textbf{Explosion im Treibhaus}:
vor \SI{65.4}{\mega\year},
\SI{50}{\percent} aller Arten wurden ausgelöscht

\begin{wichtig}
    \item
    \textbf{Ablauf}:
    \emph{Einschlag auf Kalkablagerungen\\
    $\Rightarrow$ Eruption von Basalt\\
    $\Rightarrow$ Erhöhung des \ce{CO2}-Gehalts\\
    $\Rightarrow$ spontane Temperaturerhöhung\\
    $\Rightarrow$ Einschlag eines Himmelskörpers auf Kalkablagerungen,
    diese enthalten Chlor- und Schwefelsalze,
    Verdampfung führt zur Entweichung von Salzsäure und Schwefelsäure\\
    $\Rightarrow$ Mega-Tsunami und saurer Regen\\
    $\Rightarrow$ jahrelanger globaler Winter\\
    $\Rightarrow$ Kollaps der Nahrungskette\\
    $\Rightarrow$ kurzfristiges Supertreibhaus}
\end{wichtig}

\textbf{die Stunde der Säugetiere}:
danach konnten sich die Säugetiere ausbreiten,
sie waren lebendgebärend, warmblütig und behaart,
Säugetiere besaßen einen sechsfach höheren Sauerstoffbedarf
(für Plazenta, denn die Vorsorgung des Kindes im Mutterleib
ist nur bei hohem \ce{O2}-Gehalt möglich),
rückten spontan in die frei gewordenden Nischen

\pagebreak

\section{%
    Erdneuzeit: Zeitalter der Säugetiere%
}

\textbf{Gigantismus im \ce{CO2}-\ce{O2}-Treibhaus vor \SIrange{65}{48}{\mega\year}}:
Anstieg von \ce{O2}, \ce{CO2} und Temperatur,
dadurch Treibhausklima und Gigantismus
(Riesenschlangen, Riesenvögel, Riesennashörner)

\textbf{akzentuiertes und Treibhausklima}:
je nachdem, ob die Polkappen vereist sind oder nicht,
herrscht akzentuiertes (momentanes Klima) oder Treibhausklima vor

\textbf{auf dem Weg in die Eiszeit}:
Kontinent Antarctica driftet zum Südpol,
außerdem entziehen lange Gebirgsketten der Atmosphäre \ce{CO2},
seit \SI{48}{\mega\year} Abkühlung,
seit \SI{34}{\mega\year} Vereisung der Antarktis (Schwelle \SI{750}{\ppm} \ce{CO2}),
seit \SI{2.7}{\mega\year} Vereisung der Nordhalbkugel (Schwelle \SI{280}{\ppm} \ce{CO2}),
dadurch Rückgang von \ce{O2}, \ce{CO2} und Temperatur,
Schließung des Isthmus von Panama und Entstehung des Golfstroms,
dadurch wurde warmes Wasser in den Norden geleitet, was leichter als kaltes Wasser verdunstet,
damit gab es mehr Feuchtigkeit und die Arktis konnte sich auch vereisen

\textbf{Errungenschaften der Säugetiere}:
Warmblütigkeit,
verbesserte Kauwerkzeuge (Kiefer dreidimensional bewegbar),
Plazenta,
Lebendgebären,
Säugen,
intensive Brutpflege

\textbf{Anpassungen der Säugetiere}:
an sinkende Temperaturen,
an Rückgang der Wälder,
an Ausbreitung der Gräser

\textbf{Erfolgsgeschichte der Hominiden}:
aufrechter Gang, Frühgeburten, Turbo-Brutpflege,
Kopfwachstum in der Kindheit, hohe individuelle Lernfähigkeit,
Arbeitsteilung, hohe kollektive Lernfähigkeit, Nutzung von Brennstof"|fen,
sprechen, denken usw.

\begin{wichtig}
    \item
    \textbf{biologische Innovationen, die es nur bei Säugetieren gab}:\\
    \emph{Behaarung, Hirnrinde, aber nicht permanente Warmblütigkeit}

    \item
    \textbf{biologische Innovationen, die es nur beim Menschen gab}:\\
    \emph{aufrechter Gang, Kopfwachstum in der Kindheit, Daumen,
    aber nicht funktionale Frühgeburten, intensive Brutpflege}

    \item
    \textbf{zeitliche Reihenfolge der Entstehung der Tiere des Festlands}:\\
    \emph{Gliedertiere, Amphibien, Reptilien, Säugetiere, Vögel}
\end{wichtig}

\pagebreak

\section{%
    Stammbaum des Lebens%
}

\textbf{Domänen}:
die drei heute akzeptierten Domänen sind Archaea, Bakterien und Eukaryoten (mit Zellkern),
Archaea und Bakterien werden als Prokaryoten bezeichnet (kein Zellkern)

\textbf{Archaea}:
überwiegend anaerob,
chemolithoautotroph (nur anorganische Stof"|fe werden zur Energiegewinnung umgesetzt),
leben von anaeroben Redoxreaktionen,
Zellen mit Membranen,
größer als Viren, aber kleiner als Bakterien,
alle tiefsten und kürzesten Äste im Stammbaum sind von hyperthermophilen (wärmeliebenden)
Archaea besetzt

\textbf{planetare Voraussetzungen für Archaea}:
\ce{CO2} als Kohlenstoffquelle,
Wärme optimal zwischen \SIrange{80}{106}{\celsius},
flüssiges Wasser,
Spurenelemente,
\ce{H2} und \ce{S^0} bzw. \ce{S^{2-}} als Elektronendonor

\textbf{Vorkommen der Archaea}:
\SI{4}{\percent} in kontinentalen Böden bis \SI{8}{\meter} Tiefe,
\SI{39}{\percent} in kontinentalen Böden ab \SI{8}{\meter} Tiefe,
\SI{2}{\percent} in Meerwasser,
\SI{55}{\percent} in Ozeanböden ab \SI{10}{\centi\meter} Tiefe
(Sedimentoberfläche: $10^9$ bis $10^{10}$ Zellen pro \si{\centi\meter\cubed}, hpts. Bakterien,
Ozeanboden in \SI{1}{\kilo\meter} Tiefe: $10^6$ Zellen pro \si{\centi\meter\cubed}, hpts. Archeaa)

\textbf{Archaea im tiefen Untergrund}:
Anpassung an extrem niedrige Energieflüsse und an ein extrem langsames Wachstum,
Beispiel Tiefsee: 1 Elektron pro Zelle pro Sekunde

\textbf{wichtigste Reaktionen der Biosphäre}:
Fotosynthese: \ce{6CO2 + 6H2O -> C6H12O6 + 6O2} (globale Primärproduktion \SI{100}{\tera\watt}
überwiegend durch zwei Gattungen von Cyanobakterien),\\
Atmung: \ce{C6H12O6 + 6O2 -> 6CO2 + 6H2O} (FSE \SI{-2870}{\kilo\joule/\mol}),\\
Gärung: \ce{C6H12O6 -> 3CH4 + 3CO2} (FSE \SI{-390}{\kilo\joule/\mol}),\\
anaerobe Methanoxidation: \ce{CH4 + SO4^{2-} + 2H+ -> CO2 + H2S + 2H2O}
(FSE \SI{-18}{\kilo\joule/\mol}),
zum Vergleich: direkte Methanverbrennung \SI{-2480}{\kilo\joule/\mol}
(\SIrange{+15}{20}{\kilo\joule/\mol} transportiertem \ce{H+} oder \ce{Na+} ist
das geringste Energiequantum, das gerade noch eine ATP-Synthese erlaubt)

\textbf{freie Standardenthalpie}:
Energie, die aufgewendet werden muss, um bspw. Oxide zu spalten, sodass freier Sauerstoff entsteht,
wird als Spaltungsenthalpie bezeichnet und in \si{\kilo\joule/\mol} ausgedrückt

\pagebreak

\section{%
    Plattentektonik und Klima%
}

\textbf{Gebirge sind \ce{CO2}-Verbraucher}:
durch marine Fotosynthese wird die Reaktion
\ce{CO2 + H2O + CaSiO3 -> CaCO3 + SiO2 + H2O} durchgeführt,
wobei sich \ce{CaSiO3} z.\,B. in Gneis befinden kann und \ce{CaCO3} Kalkstein heißt,
besonders, wenn sich ein Gebirge in Äquatornähe befindet, wird der \ce{CO2}-Gehalt deutlich gesenkt

\textbf{Kühlhausfalle}:
durch die Senkung des \ce{CO2}-Gehalts
(auch z.\,B. weil nicht genügend \ce{CO2} aus dem Erdmantel entgast wurde)
kann es zu einer dramatischen Temperaturabsenkung kommen,
die erste solche fand vor \SI{2.5}{\mega\year} statt,
zwischen \SIrange{720}{582}{\mega\year} gab es drei sehr starke Vereisungen,
bei denen nur ein relativer schmaler Streifen am Äquator nicht vereist war,
die komplette Vereisung hätte das Ende der Biosphäre bedeutet,
zusätzliche \ce{CO2}-Entgasung durch Krustenrecycling

\textbf{ehemaliges \ce{CO2} aus der Atmosphäre}:
Teersande,
Schwarzschiefer,
Erdöl,
Erdgas und Gashydrat,
Kohle,
Kalkstein,
Graphitschiefer und graphitische Gneise

\textbf{Recycling von \ce{CO2} durch Plattentektonik}:
kohlenstoffhaltige Gesteine können durch\\
\ce{\text{organische Substanz} + O2 -> H2O + CO2}
herausgehoben werden,
andererseits können kohlenstoffhaltige Gesteine durch
\ce{CaCO3 + SiO2 -> CaSiO3 + CO2}
tief versenkt werden

\textbf{Folgen der Plattentektonik}:
driftende und wachsende Kontinente,
wechselnde Verteilung von Land und Meer,
wechselnde Strömungs- und Verwitterungsmuster

\textbf{Plattentektonik und Klima}:
äquatorparallele Kontinentbarrieren verhindern Nord-Süd-Zir"-kulation und führen zu
globaler Erwärmung,
Superkontinente oder Kontinente an den Polen führen zu globaler Abkühlung,
Evolutionsanreize durch ständig wechselnde Umweltbedingungen

\begin{wichtig}
    \item
    \textbf{mit welcher plattentektonischen Situation ist mit einer Senkung des \ce{CO2}-Gehalts
    deutlich verknüpft}:
    \emph{Gebirge in Äquatornähe}
\end{wichtig}

\pagebreak

\section{%
    Energie%
}

\textbf{Kohlenstoff-Reservoirs}:
die größten Kohlenstoff-Reservoirs stellen Karbonate (Kalke) und Biomineralien dar,
dann folgen Ozeane und kontinentale Gewässer,
Gashydrate,
Kohle,
Böden,
lebende Biomasse,
Atmosphäre,
Erdöl,
Erdgas und
schließlich Torf

\textbf{Energiedichte geordneter Systeme}:
Galaxien \SI{1e-4}{\watt/\kilogram},
Sterne \SI{5e-4}{\watt/\kilogram},
Planeten \SI{8e-3}{\watt/\kilogram},
Bakterien \SI{5e-2}{\watt/\kilogram},
Tiere \SI{2e-1}{\watt/\kilogram},
Säugetiere \SI{1}{\watt/\kilogram},
hochtechnisierte Zivilsation \SI{250}{\watt/\kilogram}

\textbf{Nachteile der Verbrennung von Kohlenwasserstof"|fen}:
Verbrauch von Ressourcen,
Freisetzung von \ce{CO2},
Abwärme,
selbst bei erfolgreicher \ce{CO2}-Verminderung bleibt das Problem der Abwärme
(\SI{69}{\percent} der technologischen Abwärme werden in Atmosphäre, Hydrosphäre und Böden
gespeichert, was eine \ce{CO2}-unabhängige Nettoerwärmung ergibt)

\textbf{\ce{CO2}-neutrale, aber nicht wärmeneutrale Energieerzeugung}:
Kernkraft,
tiefe Geothermie

\textbf{wärmeneutrale Energiequellen}:
Sonnenlicht,
Gezeiten,
Wind und Wellen,
Ozeane und flaches Grundwasser,
Flüsse

\textbf{Sonneneinstrahlung}:
von der Leistung der Sonneneinstrahlung von \SI{178000}{\tera\watt}
gehen knapp \SI{76}{\percent} durch Reflexion und Absorption verloren
und weitere \SI{23}{\percent} durch Verdunstung,
womit \SI{1.6}{\percent} auf der Erdoberfläche verfügbar sind
(\SI{0.5}{\percent} und \SI{1.1}{\percent} auf Kontinenten bzw. Ozeanen)

\textbf{globale Primärproduktion}:
beträgt \SI{100}{\tera\watt} und wird überwiegend durch zwei Gattungen von Cyanobakterien
bewerkstelligt,
der Stoffumsatz im Ozean ist 700 Mal höher als auf Kontienten

\textbf{zum Vergleich}:
der Wärmestrom aus dem Inneren der Erde beträgt \SIrange{29}{34}{\tera\watt},
der zivilisatorische Wärmestrom \SI{10}{\tera\watt}

\begin{wichtig}
    \item
    \textbf{Kohlenstoff-Reservoirs in der Erdkruste}:\\
    \emph{größte sind Karbonate (Kalke), dann Ozeane, kleinste ist Torf}
\end{wichtig}

\pagebreak
