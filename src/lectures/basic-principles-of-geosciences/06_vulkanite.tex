\chapter{%
    Vulkanite%
}

\section{%
    Lava%
}

\textbf{Lava}:
Magma, das an der Oberfläche ausfließt

\textbf{Dif"|ferentationsgrad}:
höhere Dif"|ferentiation bedeutet
höhere Polymerisation,
höhere Dickflüssigkeit und
höheren Gasgehalt,
aber niedrigere Temperatur
(hochdif"|ferenziert: \SI{800}{\celsius} vs. undif"|ferenziert: \SI{1000}{\celsius} bis \SI{1200}{\celsius}),
große Rauchsäulen, Eruptionen und Vulkane in Kegelform entstehen eher bei dif"|ferenzierterem Magma

\textbf{Kissenlava}:
Lava, das unter Wasser ausfließt,
wegen dem hohen Wasserdruck bilden sich kissenförmige Strukturen

\textbf{Stricklava}:
sehr dünnflüssige, basaltische Lava, die an Land ausfließt,
Entstehung von Basaltsäulen senkrecht zur Abkühlungsoberfläche

\textbf{Blocklava}:
höher dif"|ferenzierte Lava,
höherer Gasgehalt führt zur Bildung von Block-Struktu"|ren und Block-Lavaströmen

\textbf{hochdif"|ferenzierte Lava}:
nur bei ausnahmsweise geringem Gasgehalt möglich (sonst Eruption),
sehr zäh

\textbf{Obsidian}:
natürliches, hochdif"|ferenziertes vulkanisches Glas,
auch möglich bei wenig Dif"|ferentiation als basaltisches Glas

\begin{wichtig}
    \item
    \textbf{Kissenlava entsteht}:
    \emph{aus Basaltmagma, das unter Wasser ausfließt}

    \item
    \textbf{Stricklava entsteht}:
    \emph{an Land}

    \item
    \textbf{Blocklava ist}:
    \emph{höher dif"|ferenziert}
\end{wichtig}

\section{%
    Tephra%
}

\textbf{Tephra}:
alles, was explosiv aus Vulkanschloten kommt (Asche, Bomben usw.),
je höher das Magma dif"|ferenziert ist, desto höher ist die Wahrscheinlichkeit für eine
vulkanische Explosion

\textbf{Fallablagerungen}:
fliegen auf ballistischen Flugbahnen und sind gut sortiert,
gleich große Brocken landen gleich weit,
Asche verteilt sich wesentlich weiter

\textbf{Korngrößen-Klassifikation der pyroklastischen Gesteine}:\\
unter \SI{2}{\milli\meter} Asche,
von \SI{2}{\milli\meter} bis \SI{64}{\milli\meter} Lapilli,
über \SI{64}{\milli\meter} Blöcke und Bomben

\textbf{Schichtvulkane}:
typische Schichtvulkane entstehen durch abwechselnden Auswurf von Lava und Tephra,
durch große Korngröße entsteht ein steiler Böschungswinkel

\pagebreak

\section{%
    Pyroklastische Ströme, Surges, Ignimbrite%
}

\textbf{Eruption eines Vulkans}:
spontane adiabatische (ohne Wärmeaustausch verlaufende) Entleerung durch
Bildung von Überdruck im vulkanischen Schlot wegen hochdif"|ferenziertem Magma,
Entstehung einer konvektiven Eruptionssäule wegen Einmischung von kalter Luft,
starke Explosionen heißen auch plinianische Eruptionen

\textbf{Glutwolke und Glutlawine}:
fließen den Berg hinab,
Glutlawine gleitet lautlos auf einem Luftpolster über dem Boden,
vor Glutlawine gleitet ein sog. Ground Surge

\textbf{Ignimbrite}:
Ablagerungen von Ground Surges, Glutlawinen und Glutwolken,
wegen der niedrigen Viskosität können sich Ignimbrit-Ablagerungen sehr weit verbreiten,
Verbreitung auch über Wasser möglich (Luftpolster)

\begin{wichtig}
    \item
    \textbf{Fallablagerungen sind}:
    \emph{gut sortiert}

    \item
    \textbf{Ausgangstemperatur eines rhyolithischen Glutstroms liegt bei}:
    \emph{\SI[math-rm=\mathit,text-rm=\itshape]{800}{\celsius}}

    \item
    \textbf{Aufbau der Ignimbrite von unten nach oben}:\\
    \emph{Base-Surge-Ablagerungen, Glutlawinen-Ablagerungen, Glutwolken-Ablagerungen}
\end{wichtig}

\section{%
    Vulkanformen%
}

\textbf{Schlackenkegel und pyroklastische Kegel}:
wenn zufällig gasreiche, undif"|ferenzierte Magma austritt
(also dünnflüssige, basaltische Magma),
Größe im Kilometer-Bereich

\textbf{Schildvulkane}:
immer noch undif"|ferenzierte Magma,
"`Soße"' läuft aus,
dies führt zur Bildung von großen, extrem flachen Basaltplateaus,
z.\,B. Hawaii, Olympus Mons

\textbf{Schichtvulkane}:
mittlerer Dif"|ferentiationsgrad,
Wechsel von Lava und Tephra (daher steiler Böschungswinkel),
sprengen sich hin und wieder selbst in die Luft (z.\,B. Vesuv),
nach der Explosion führt ein zylinderförmiges Absacken der Kruste zu einer sog. Caldera
(existiert auch beim Olympus Mons)

\textbf{Dome und Maare}:
Lavadome sind hügelförmige Erhebungen, die durch die Eruption (Herauspressen) von
sehr zähflüssiger Lava entstehen,
Maare entstehen überwiegend aus stark untersättigten, extrem dünnflüssigen Magmen,
diese gelangen schnell an die Oberfläche,
bei Kontakt mit Grundwasser kommt es zu Dampfexplosionen,
dadurch entstehen ein Sprengtrichter und ein Tuffwall,
hört auf, wenn Magma oder Wasser zur Neige geht,
Sprengtrichter füllt sich mit einem Maarsee

\begin{wichtig}
    \item
    \textbf{Schlackenkegel/pyroklastische Kegel entstehen bei}:\\
    \emph{Eruption von undif"|ferenziertem Magma}

    \item
    \textbf{Schichtvulkane verdanken ihre Form und Ausdehnung}:\\
    \emph{höher dif"|ferenziertem Magma (zäher und gasreicher)}

    \item
    \textbf{globale Wahrscheinlichkeit, dass Schichtvulkane explodieren}:
    \emph{sehr hoch}

    \item
    \textbf{Maare entstehen, wenn}:
    \emph{extrem niedrig viskoses Magma auf Grundwasser trifft}
\end{wichtig}

\pagebreak

\section{%
    Eruptionstypen%
}

\textbf{wenig dif"|ferenzierte Magma}:
Magmenseen, Spalteneruptionen, Lavafälle
(basaltische, gasarme Magmen)

\textbf{höher dif"|ferenzierte Magma}:
Tephra, Blocklava, Schichtvulkan
(höhere Viskosität, höherer Gasgehalt),
ist Eruptionssäule höher als ca. \SI{10}{\kilo\meter}, dann gelangt Asche in
die weitgehend wetterfreie Stratosphäre und es gibt Probleme für die Zivilisation,
bei Eruption ist starker Gewitterregen möglich wegen Partikel als Kondensationskeime

\section{%
    Spätvulkanische Erscheinungen%
}

\textbf{hydrothermale Gänge}:
wenn ein Vulkan nicht mehr aktiv ist,
kann sich überhitztes (bis \SI{407}{\celsius}) oder überkritisches (bis \SI{650}{\celsius})
Wasser bilden,
überkritisches Wasser ist sehr aggressiv und löst alles,
daher kommt es zur hydrothermalen Ablagerung von Erzen

\textbf{andere spätvulkanische Erscheinungen}:
Schwefelausblühungen,
Schlammvulkane
(Bildung über Mischung von Asche mit Wasser oder
durch Verwitterung von vulkanischem Glas zum Tonmineral Montmorillonit
(wichtigster Bestandteil des Gesteins Bentonit,
"`natürlichstes Waschmittel, das man sich vorstellen kann"')),
hochsaure und -basische Bäche (pH-Werte von 1 oder 12),
Geysire (Ursache: adiabatische Schwingungen der unterirdischen Dampfspeicher),
bunte Kerne in Schichtvulkanen, die Vererzungen enthalten
(gebildet durch die Tätigkeit von überkritischem Wasser)

\begin{wichtig}
    \item
    \textbf{nach dem Ende der ef"|fusiven (durch Ausfließen von Lava bewirkten) Tätigkeit findet
    eine tiefgreifende Umwandlung der Gesteine statt durch}:\\
    \emph{heißes, überkritisches Wasser}

    \item
    \textbf{welcher Prozess verändert vulkanisches Gestein am intensivsten}:\\
    \emph{Umwandlung von vulkanischem Glas in Tonmineral}

    \item
    \textbf{wo sollte man in einem erloschenen Schichtvulkan nach Erzlagerstätten suchen}:
    \emph{im bunten Kern}
\end{wichtig}

\pagebreak

\section{%
    Altersdatierung von Gesteinen%
}

\textbf{relative Methoden zur Altersbestimmung}:
stratigrafische Methoden
(jüngere Schichten liegen normalerweise über älteren),
paläontologische Methoden
(eingeschlossene Fossilien bestimmter Arten),
impaktgeologische Methoden
(auf anderen Planeten und Monden besitzt eine ältere Oberfläche mehr Einschlagskrater
als jüngere Oberflächen)

\textbf{absolute Methoden zur Altersbestimmung}:
radiometrische und isotopenchemische Methoden

\begin{wichtig}
    \item
    \textbf{welche Lebensweise liefert die besten Leitfossilien}:\\
    \emph{im Wasser leben (z.\,B. Plankton und Nekton, d.\,h. die "`Schwimmwelt"')}

    \item
    \textbf{Ammoniten sind}:
    \emph{gute Leitfossilien}

    \item
    \textbf{Körperteile des Menschen mit hohem Fossilisationspotential}:\\
    \emph{Zähne und Titan-Implantate}

    \item
    \textbf{Sauerstoff-Isotopie}:
    \emph{für Paläoklima}

    \item
    \textbf{Reichweite der \ce{^{14}C}-Methode}:
    \emph{ca. \SI[math-rm=\mathit,text-rm=\itshape]{50000}{\year}}

    \item
    \textbf{Oberflächendatierung auf anderen Himmelskörpern}:
    \emph{durch Kraterzählung}
\end{wichtig}

\pagebreak
