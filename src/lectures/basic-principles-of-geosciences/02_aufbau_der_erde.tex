\chapter{%
    Aufbau der Erde%
}

\section{%
    Aufbau nach Elementen%
}

\textbf{Aufbau der Erde nach Elementen}:
nach Atom-Prozent besteht die gesamte Erde nur aus sechs Elementen $\ge \SI{1}{\percent}$,
nämlich
\ce{O} (\SI{50}{\percent}),
\ce{Mg} (\SI{16}{\percent}),
\ce{Si} (\SI{16}{\percent}),
\ce{Fe} (\SI{14}{\percent}),
\ce{Al} (\SI{2}{\percent}),
\ce{Ca} (\SI{1}{\percent})

\textbf{Aufbau der kontinentalen Kruste nach Elementen}:
in der kontinentalen Kruste sieht die Situation anders aus, nämlich
\ce{O} (\SI{63}{\percent}),
\ce{Si} (\SI{23}{\percent}),
\ce{Al} (\SI{6}{\percent}),
\ce{Na} (\SI{2}{\percent}),
\ce{Ca} (\SI{2}{\percent}),
\ce{K} (\SI{2}{\percent}),
\ce{Mg} (\SI{1}{\percent}),
\ce{Fe} (\SI{1}{\percent})

\textbf{Element-Anreicherungen in der kontinentalen Kruste}:
viele für die Biosphäre wichtige Elemente haben sich in der kontinentalen Kruste massiv
angereichert, darunter \ce{N}, \ce{B}, \ce{C}, \ce{Cl}, \ce{H}, \ce{K}, \ce{P},
andere Elemente sind im Vergleich zur Gesamterde ungefähr gleich stark vertreten,
darunter \ce{Ca}, \ce{S}, \ce{O}, \ce{Fe}, \ce{Mn},
nur \ce{Mg} kommt deutlich seltener in der kontinentalen Kruste vor

\section{%
    Dif"|ferentiation%
}

\textbf{treibende Kräfte der Evolution}:
treibende Kraft inorganischer Evolution ist chemische Dif"|ferentiation,
d.\,h. die Auf"|teilung eines ursprünglich homogenen Gemisches nach Elementen,
für organische Evolution ist zusätzlich Plattentektonik wichtig

\textbf{Magmendif"|ferentiation}:
durch fraktionierte Kristallisation im of"|fenen System findet Magmendif"|ferentiation statt

\textbf{trockene partielle Aufschmelzung von Mantelgestein}:
Aufschmelzung von Mantelgestein findet immer nur in Teilschmelzen statt,
da die für vollständige Schmelzen benötigte Temperatur auf der Erde zu niedrig ist
(es gibt höchstens \SI{35}{\percent}-ige Teilschmelzen)

\textbf{Silikate}:
Silikate sind Polymere, die aus Tetraedern der Form \ce{(Si^{4+}O4^{2-})^4-} aufgebaut sind
(Siliziumatom in der Mitte),
durch zunehmende chemische Dif"|ferentiation (Voraussetzung dafür: Gesteinsschmelzen)
entstehen immer längere Ketten und Moleküle,
bei \SI{100}{\percent} vernetzten \ce{SiO4}-Tetraedern erhält man Quarz (\ce{SiO2})

\section{%
    Erdkruste, Erdmantel und Erdkern%
}

\textbf{Steinplaneten entwickeln hoch dif"|ferenzierte Kruste}:
in der Erdkruste gibt es viel \ce{H2O}, \ce{C}, \ce{N}, \ce{P}, \ce{S} und Silikate,
der obere Erdmantel besteht hauptsächlich aus Silikaten (\ce{Si}, \ce{O} und Metalle),
im unteren Erdmantel gibt es Oxide von Eisen und Magnesium (\ce{FeO}, \ce{MgO} usw.),
der Erdkern besteht aus Eisen

\textbf{Eisenkern}:
entstand in den ersten 33 Millionen Jahren,
besteht aus äußerem Kern (dünnflüssiger als Wasser) und innerem Kern (fest),
starkes Magnetfeld durch Turbulenzen

\textbf{Erdmantel}:
eigene radiogene Wärmeproduktion aus \ce{K}, \ce{Th}, \ce{U},
Kühlung von außen erzeugt Konvektion,
Konvektion treibt Plattentektonik an und bildet Hotspots

\textbf{Erdkruste}:
ozeanische Kruste (Dichte \SI{3.0}{\gram/\centi\meter\cubed})
entsteht im Scheitel von Konvektionszellen und wird im Mantel recycelt,
dabei nimmt sie Wasser mit, welches sich im darüberliegenden Mantel löst und Magmen bildet,
so entsteht kontinentale Kruste\\
(Dichte \SIrange{2.7}{2.8}{\gram/\centi\meter\cubed})

\textbf{Atmosphäre}:
enthält hauptsächlich \ce{N2}, \ce{O2}, \ce{H2O}, \ce{CO2}, \ce{Ar}
(Anteil \ce{CO2}: \SI{390}{\ppm} = \SI{0.039}{\percent})

\section{%
    Plattentektonik%
}

\textbf{Frühzeit (\SIrange{4.0}{2.5}{\giga\year})}:
heiße ozeanische Lithosphäre schwimmt oben,
Stapelung von Kruste und Mantelspänen,
wässrige Schmelzen erzeugen granitähnliche Magmen,
Gebirge nicht höher als \SI{2.5}{\kilo\meter},
keine Plattentektonik, da Wärmefluss zu groß (ozeanische Kruste zu heiß),
archaische Grünstein-Gürtel

\textbf{Plattentektonik-Zeit (seit \SI{2.5}{\giga\year})}:
kältere und weniger hydratisierte Kruste,
Basalt wird zu Eklogit, daher tiefreichende Subduktion (da Eklogit dichter ist),
wässrige Schmelzen erzeugen granitähnliche Magmen,
Nettowachstum der Kontinente,
Gebirge bis \SI{10}{\kilo\meter},
heutige Ozeane und Kontinente

\textbf{Plattentektonik-Fenster}:
bestimmtes Zeitfenster, in dem Plattentektonik stattfinden kann,
Mars, Merkur und Mond haben das Fenster vor langer Zeit schnell durchlaufen,
da sie aufgrund ihrer kleinen Größe zu schnell ausgekühlt sind,
die Venus war zwar längere Zeit im Fenster, aber ist seit ca. \SI{1}{\giga\year} außerhalb,
die Erde befindet sich als einziger Planet im Sonnensystem noch innerhalb des Fensters

\textbf{treibende Kräfte}:
treibende Kraft der Plattentektonik ist die Mantelkonvektion,
treibende Kraft von Klimaveränderungen ist die Plattentektonik

\textbf{Plattentektonik und Klima}:
driftende und wachsende Kontinente führen zu einer wechselnden Verteilung von Land und Meer und
wechselnden Strömungs- und Verwitterungsmustern,
die zwei stabilen globalen Klimazustände des Treibhausklimas und des akzentuierten Klimas
(mit vereisten Polkappen) können durch Plattentektonik erreicht werden,
äquatorparallele Kontinent-Barrieren verhinden Nord-Süd-Zirkulation und führen zu globaler
Erwärmung,
Superkontinente oder Kontinente an den Polen führen zu globaler Abkühlung,
die ständig wechselnden Umweltbedingungen liefern Evolutionsanreize

\section{%
    Atmosphäre%
}

\textbf{Uratmosphäre}:
enthielt keinen Sauerstoff, sondern \ce{N}, \ce{CO}, \ce{CO2}, \ce{CH4}, \ce{H2O},
\ce{H2S}, \ce{HCN} (Blausäure, wichtig für Bildung von Leben, aber flüchtig),
also mehr Treibhausgase als heute,
aggressive Verwitterung und rasche Mineralisierung der Ozeane sind Voraussetzungen für die
Entstehung von Leben

\textbf{Energieausstoß der Sonne}:
Sonne steigert Energieausstoß um \SI{10}{\percent/\giga\year},
Oberflächentemperatur der Erde vor \SI{3.8}{\giga\year} war wahrscheinlich ca.
\SI{40}{\celsius} (ohne Treibhausgase wäre sie gleich \SI{-19}{\celsius} gewesen),
wichtig, dass es nicht zu viel und nicht zu wenig Treibhausgase gibt,
eine weitere Temperatursenkung ist nur möglich, wenn \ce{CO2} entnommen und \ce{O2}
hinzugefügt wird

\textbf{Fotosynthese}:
\ce{CO2 + H2O -> \text{organische Substanz} + O2},
Fotosynthese-Leistung steigt mit Energieausstoß der Sonne,
Senkung des Treibhausef"|fekts,
selbstregelndes System, das den steigenden Energieausstoß der Sonne über mindestens
5 Milliarden Jahre hinweg kompensiert

\pagebreak
