\hypersetup{pdftitle={Gesammelte Vorlesungsnotizen}}

\automark[chapter]{part}
\ihead{\leftmark}
\ifoot{\vspace{-1.5mm}\rightmark}

\thispagestyle{empty}
\vspace*{1em}

{%
  \usekomafont{disposition}\huge%
  Gesammelte Vorlesungsnotizen%
}
\vspace*{1em}

\emph{Julian \name{Valentin}}

\vspace*{1em}

Diese Vorlesungsnotizen entstanden als Mitschrieb in Vorlesungen
an der Universität Stuttgart in den Jahren 2009 bis 2014.
Sie dienten hauptsächlich als Lernhilfe für mich;
aus Zeitgründen fehlen viele Skizzen und mathematische Beweise.
Die Notizen werden trotzdem zur Verfügung gestellt, in der Hoffnung,
dass sie jemandem einen Nutzen bringen.
Fehler können unter \url{https://github.com/valentjn/class-notes} gemeldet werden.
Die \LaTeX{}-Umsetzung der Notizen steht unter CC BY-SA 4.0
(\url{https://creativecommons.org/licenses/by-sa/4.0/}).

Die Vorlesungsnotizen behandeln Themen aus
der Analysis (Vorlesungen I bis VI),
der Algebra (VII bis IX),
der Topologie (X),
der angewandten Mathematik (XI bis XIII),
der Numerik (XIV bis XIX),
der Informatik (XX bis XXVIII) und
verschiedenen anderen Gebieten (XXIX bis XXXI).

{%
  % only include parts in list of parts
  \setcounter{tocdepth}{\parttocdepth}

  % use dotted lines for list of parts, make font smaller
  \RedeclareSectionCommand[tocstyle=dottedtocline]{part}

  % list of parts
  \renewcommand*{\contentsname}{Vorlesungen}
  \tableofcontents%
}

\pagebreak

{%
  % use a chapter heading for main table of contents
  \unsettoc{toc}{leveldown}

  % include everything up to subsubsections in main table of contents
  % (and part-wise tables of contents)
  \setcounter{tocdepth}{\subsubsectiontocdepth}

  % main table of contents
  \tableofcontents%
}

\pagebreak

% use a section for table of contents (otherwise there is a page break before it)
\setuptoc{toc}{leveldown}

\input{collectionInclude}
